% A concentrated (point) force in an infinite medium
% (in a limitless medium)

\section{%
\en{Concentrated}\ru{Сосредоточенная}~%
(\en{point}\ru{точечная})
\en{force}\ru{сила}%
}

\label{section:point-force}

%%\newcommand\thefirstlameparameter{\lambda}
%%\newcommand\thesecondlameparameter{\mu}

{\small
\en{A~}\en{concentrated}\ru{Сосредоточенная}
\en{force}\ru{сила}\ru{,}
\en{acting}\ru{действующая}
\en{on a~point}\ru{на точку}\en{ is}\ru{\:---}
\en{a~handy abstraction}\ru{удобная абстракция}
\en{to simplify}\ru{для упрощения}
\en{reality}\ru{реальности}.
\en{However}\ru{Однако},
\en{it does not exist}\ru{её не существует}
\en{in the~real world}\ru{в~реальном мире},
\en{where}\ru{где}
\en{all forces}\ru{все силы}
\en{either}\ru{либо}
\en{act}\ru{действуют}
\en{in a~volume}\ru{в~объёме}\:---
\en{volume}\ru{объёмные}
\en{forces}\ru{силы},
\en{or}\ru{либо}
\en{act}\ru{действуют}
\en{over an~area}\ru{по площади}\:---
\en{surface}\ru{поверхностные}~%
(\en{contact}\ru{контактные})
\en{forces}\ru{силы}.
\par}

\en{Here is}\ru{Вот}
\en{a~rhetorical question}\ru{риторический вопрос}\::
\emph{%
   \en{why}\ru{почему}
   \en{an elastic body}\ru{упругое тело}
   \en{withstands}\ru{сопротивляется}
   \en{an applied load}\ru{приложенной нагрузке},
   \inquotes{\en{bears}\ru{держит}}
   \en{it}\ru{её}?%
}
%
\en{The~book}\ru{Книга}~\cite{gordon-whyyoudontfallthru}
\en{by }James\ru{’а} Gordon\ru{’а}
\en{gives}\ru{даёт}
\en{the~following answer}\ru{следующий ответ}\::
\en{the~body deforms}\ru{тело деформируется},
\en{and thus}\ru{и~потому}
\ru{появляются }\en{the internal forces}\ru{внутренние силы}\en{ appear},
\en{called}\ru{называемые}
\inquotesx{\en{stresses}\ru{напряжениями}}[,]
\en{which can}\ru{которые могут}
\en{compensate}\ru{компенсировать}~(\en{balance, equilibrate}\ru{уравновесить})
\en{an external load}\ru{внешнюю нагрузку}.

\en{But}\ru{Но}
\en{a~}\en{linear elastic}\ru{линейно-упругое}
\en{body}\ru{тело}
\en{cannot}\ru{не~может}
\en{take}\ru{воспринимать}
\en{the~load}\ru{нагрузку}
\en{of a~point force}\ru{точечной силой}.

.... \en{in} \en{the~balance of~forces}\ru{балансе сил}~(\en{of~momentum}\ru{импульса})

\nopagebreak\vspace{-0.2em}\begin{equation*}
\scalebox{.81}{$ \displaystyle\integral\displaylimits_{\mathcal{V}} $} \hspace{-0.4ex}
\scalebox{.95}{$ \bigl( \boldnabla \hspace{-0.1ex} \dotp \hspace{-0.1ex} \linearstress
    + \volumeloadvector \bigr) d\mathcal{V}
    = \hspace{.1ex} \zerovector $}
\end{equation*}

\nopagebreak\vspace{-1em}\noindent
\hfill
${\mathcal{V} \sim r^3}$, ${\volumeloadvector \sim \frac{1}{r^3}}$, ${\linearstress \sim \frac{1}{r^2}}$

\noindent
\hfill
${\linearstress
= 2 \thesecondlameparameter \infinitesimaldeformation
\hspace{-0.1ex} + \hspace{-0.1ex} \thefirstlameparameter \bigl( \trace{\infinitesimaldeformation} \hspace{-0.1ex} \bigr) \hspace{-0.1ex} \UnitDyad }$
$\Rightarrow$
${\infinitesimaldeformation \sim \frac{1}{r^2}}$, ${\bm{u} \sim \frac{1}{r}}$,
thus ${\bm{u} \hspace{-0.33ex}\to\hspace{-0.33ex} \bm{\infty}}$ when ${r \hspace{-0.33ex}\to\hspace{-0.33ex} 0}$

...

\noindent
\emph{\en{the~solution}\ru{решение} \en{by }Kelvin\ru{’а}\hbox{--}Somigliana}
(\href{https://en.wikipedia.org/wiki/Lord_Kelvin}{William Thompson aka Lord Kelvin}\footnote{%
\href{https://en.wikipedia.org/wiki/Lord_Kelvin}{William \inquotes{Lord Kelvin} Thompson}.
Note on~the~Integration of~the~Equations of~Equilibrium of~an~Elastic Solid.
\href{https://ia904507.us.archive.org/25/items/sim_cambridge-and-dublin-mathematical-journal_1848_7/sim_cambridge-and-dublin-mathematical-journal_1848_7.pdf}{The Cambridge and Dublin Mathematical Journal, 1848 volume~iii~(vii)},
pages 87\hbox{--}89
}\hbox{\hspace{-0.55ex},}
\href{https://en.wikipedia.org/wiki/Carlo_Somigliana}{Carlo Somigliana})

... \en{in}\ru{в}~\en{an~infinite medium}\ru{бесконечной среде}

......

\noindent
\emph{\en{the}\ru{принцип} Saint-Venant’\en{s}\ru{а}\en{ principle}}

\en{Real things}\ru{Реальные вещи}
\en{can have}\ru{могут иметь}
\en{non-linearities}\ru{нелинейности}
\en{at the~places}\ru{в~мест\'{а}х}\ru{,}
\en{where}\ru{где}
\ru{приложены }\en{the external loads}\ru{внешние нагрузки}\en{ are applied}.
%%(external loads are forces and\:--- in the moment models\:--- moments)
%%(внешние нагрузки это силы и\:--- в моментных моделях\:--- моменты)
%
\en{But}\ru{Но}
\en{away}\ru{вдал\'{и}}
\en{from such places}\ru{от таких мест}
\en{only the resulants are important}\ru{важн\'{ы} лишь результанты}.
%%(the resultants are the resultant force and\:--- for the moment models\:--- the resultant moment)
%%(результанты это сила-результанта и\:--- для моментных моделей\:--- момент-результант)

\en{As example}\ru{Как пример}\ru{,}
\en{for rods}\ru{для стержней}\en{,}
\en{the~lengths}\ru{дл\'{и}ны}
\en{of the non-linear loading regions}\ru{нелинейных областей нагружения}
\en{are comparable}\ru{сравн\'{и}мы}
\en{with the sizes}\ru{с~размерами}
\en{of the cross sections}\ru{поперечных сечений}.

....
