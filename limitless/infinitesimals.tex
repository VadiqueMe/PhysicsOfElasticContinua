\documentclass[11pt, twoside]{book}

\usepackage[utf8]{inputenc}
\usepackage[T2A,T1]{fontenc}

\usepackage[english, greek, russian]{babel}

\usepackage[usenames]{xcolor}
\definecolor{darkblue}{rgb}{0.0, 0.0, 0.67}

\usepackage[unicode=true, pdfusetitle,%
	bookmarks=true, bookmarksnumbered=false, bookmarksopen=true, linktocpage=true,%
	breaklinks=false, pdfborder={0 0 0}, backref=false,%
	colorlinks=true, linkcolor=darkblue, citecolor=darkmagenta, urlcolor=darkgray]%
{hyperref}

\usepackage{bm} % bold math symbols

\usepackage{amsmath}
\usepackage{amsfonts} % for \mathbb

%%\usepackage{accents}

%%\newcommand{\scircabove}{\hbox{\fontfamily{lmr}\fontsize{6}{0}\selectfont$\,\circ$}}
%%\DeclareRobustCommand{\mathcircleabove}{\accentset{\scircabove}}

\def\spacebetweenparagraphs{2mm minus0.5mm}
\setlength{\parskip}{\spacebetweenparagraphs}

\def\horizontalindent{1.6em}
\setlength{\parindent}{\horizontalindent} % offset of the first line
\newlength\negparindent
\setlength{\negparindent}{-\parindent}

\renewcommand*\footnoterule{} % no line before footnotes

\usepackage[symbol, bottom]{footmisc} % use symbols instead of numbers and place footnotes at the very bottom of page
\setlength{\skip\footins}{1em plus1em} % \footins is the space between the text body and the footnotes

\setlength{\footnotesep}{\baselineskip} % space between footnotes

\makeatletter
\def\@fnsymbol#1{\ensuremath{\ifcase #1 %
	\or {*} %
	\or {*}\hspace{-0.2ex}{*} %
	\or {*}\hspace{-0.2ex}{*}\hspace{-0.2ex}{*} %
	\or {*}\hspace{-0.2ex}{*}\hspace{-0.2ex}{*}\hspace{-0.2ex}{*} %
	\else {\infty} %% \else \@ctrerr
\fi}}
\def\@makefnmark{\raisebox{.8ex}{\hbox{\normalfont\@thefnmark}}}
\makeatother

\usepackage{perpage}
\MakePerPage{footnote} % reset footnote numbering on every page

\newcommand{\lquote}[0]{“} % {``}
\newcommand{\rquote}[0]{”} % {''}

\newcommand{\altlquote}[0]{«} % {<<}
\newcommand{\altrquote}[0]{»} % {>>}

\newcommand{\inquotes}[1]{\lquote{#1}\rquote}
\newcommand{\inaltquotes}[1]{\altlquote{#1}\altrquote}

\usepackage{xargs} % \newcommandx

\newcommandx*{\inquotesx}[2][2=]{\lquote{#1}\hbox{\rquote%
\ifthenelse{\equal{#2}{.}}{\hspace{-0.7ex}.\hspace{.3ex}}{}%
\ifthenelse{\equal{#2}{,}}{\hspace{-0.7ex},\hspace{.3ex}}{}%
\ifthenelse{\equal{#2}{---}}{\hspace{-0.5ex}~---}{}%
}}

\usepackage{graphicx} % \scalebox and others

\newcommand\constant{\scalebox{.92}[.98]{\ensuremath{\mathsf{constant}}}}

\newcommand\sine{\operatorname{sin}} % \operatorname{sine}
\newcommand\cosine{\operatorname{cos}} % \operatorname{cosine}

\RequirePackage{ifthen}
\newif\ifen
\newif\ifru
\newcommand{\en}[1]{\ifen#1\fi}
\newcommand{\ru}[1]{\ifru#1\fi}

%%\entrue
\rutrue

\begin{document}

\en{\selectlanguage{english}}
\ru{\selectlanguage{russian}}

%% \en{}\ru{}
%% \en{}\ru{}

\en{Let}\ru{Пусть}~$dx$
\en{be}\ru{будет}
\en{a~small change}\ru{м\'{а}лым изменением}
\en{in the~variable}\ru{переменной}~$x$
%
\begin{equation}\label{asmallchangeinx}
dx = x' \hspace{-0.4ex} - x
\hspace{.2ex} ,
\end{equation}
%
\en{where}\ru{где}
${x + dx = x'}$
\en{is near}\ru{близко~к}~$x$,
\en{but}\ru{но}
\en{nonetheless}\ru{всё~же}
${x' \hspace{-0.33ex} \neq x}$
\en{and}\ru{и}~%
${dx \neq 0}$.

\en{The~change}\ru{Изменение}~$dx$
\en{is}\ru{есть}
\emph{\en{infinitesimal}\ru{бесконечном\'{а}лое}}\ru{,}
\en{if}\ru{если}
\en{it}\ru{оно}
\en{approaches zero}\ru{подходит к~нулю}
\en{being}\ru{будучи}
\en{not exactly zero}\ru{не в~точности нулём},
\en{while}\ru{в~то время как}
\en{higher powers}\ru{более высокие степени}
\en{of~}$dx$,
\en{such as}\ru{такие как}
${(dx)^2}$\hbox{\hspace{-0.33ex},}
${(dx)^3}$\hbox{\hspace{-0.33ex},}
${(dx)^4}$
\en{and so on}\ru{и~так далее},
\en{are infinitesimally smaller than}\ru{бесконечном\'{а}ло м\'{е}ньше}
%%(\en{are negligible}\ru{пренебрежимы}
%%\en{compared to}\ru{по~сравнению~с})
$dx$.
%
\en{Thus}\ru{Так что}
%
\begin{equation}\label{whatmeansinfinitesimal}
dx \neq 0
\hspace{.1ex} , \hspace{.66em}
\text{\en{but}\ru{но}}
\hspace{.5em}
(dx)^2 \hspace{-0.2ex} = 0
\hspace{.1ex} , \hspace{.33em}
(dx)^3 \hspace{-0.2ex} = 0
\hspace{.1ex} , \hspace{.33em}
\ldots
\end{equation}
%
---
\en{as}\ru{как}
\en{the~definition}\ru{определение}
\en{of~infinitesimality}\ru{бесконечномалости}.
%
\en{An~infinitesimal change}\ru{Бесконечном\'{а}лое изменение}
(\en{infinitesimal difference}\ru{бесконечном\'{а}лая разница})
\en{is called}\ru{называется}
\en{a~}\emph{\ru{дифференциалом \hspace{-0.25ex}(}differential\ru{\hspace{.2ex}\footnote{difference\:--- разница, разность}\hbox{\hspace{-0.4ex})}}}.

\en{The~properties}\ru{Свойства}
\en{of~differential}\ru{дифференциала}\::
\nopagebreak\vspace{-0.75em}
\begin{flalign*}
& \begin{array}{c@{\hspace{.5ex}}l}
\checkmark &
\text{\en{linearity}\ru{линейность}}\hspace{1ex}
d\bigl( \lambda p + \mu q \bigr) \hspace{-0.2ex} = \lambda \hspace{.2ex} dp + \mu \hspace{.2ex} dq
\\
\checkmark &
d(\constant) \hspace{-0.2ex} = 0
\end{array} &
\end{flalign*}
\vspace{-1.5em}

\inquotes{\en{The product rule}\ru{Правило произведения}}\::
\nopagebreak\vspace{-0.75em}
\begin{flalign*}
& \begin{array}{c@{\hspace{.5ex}}l}
\checkmark &
d(uv) = (u \hspace{-0.2ex} + \hspace{-0.2ex} du)(v \hspace{-0.2ex} + \hspace{-0.2ex} dv) \hspace{-0.2ex} - uv
= (du)v + u(dv) \hspace{-0.2ex} + \hspace{-0.2ex} (du)(dv),
\hspace{.5em}
(du)(dv) \hspace{-0.2ex} = 0
\end{array} &
\end{flalign*}
\vspace{-2em}

\en{Differential}\ru{Дифференциал}
\en{of~the~square}\ru{квадрата}~${d\bigl( w^2 \bigr)\hspace{-0.2ex}}$\::
\nopagebreak\vspace{-0.75em}
\begin{flalign*}
& \begin{array}{c@{\hspace{.5ex}}l}
\checkmark &
d\bigl( w^2 \bigr) \hspace{-0.25ex}
= (w + dw)^2 \hspace{-0.1ex} - w^2 \hspace{-0.1ex}
= 2 \hspace{.1ex} w \hspace{.1ex} dw + (dw)^2 \hspace{-0.1ex}
= 2 \hspace{.1ex} w \hspace{.1ex} dw
\end{array} &
\\[-0.25em]
& \hspace{\horizontalindent}
\text{%
\en{or}\ru{или},
\en{applying}\ru{применяя}
\inquotesx{\en{the product rule}\ru{правило произведения}}[,]%
} &
\\[-0.25em]
& \begin{array}{c@{\hspace{.5ex}}l}
\checkmark &
d\bigl( w^2 \bigr) \hspace{-0.25ex}
= d(ww) \hspace{-0.1ex}
= (dw)w + w(dw) \hspace{-0.1ex}
= 2 \hspace{.1ex} w \hspace{.1ex} dw
\end{array} &
\end{flalign*}
\vspace{-1.5em}


....

$\Rightarrow$ ${\forall \lambda,\mu \hspace{-0.33ex} \in \hspace{-0.33ex} \mathbb{R} \hspace{.33ex} ({\neq}\infty)}$
${\sine(\lambda dx + \mu dx) = (\lambda + \mu) \sine(dx)}$

%%✨из аддитивности f(x + y) = x + y получается, что
%%для бесконечнома́лых~$dx$
%%функция синуса $\sine(dx)$
%%ведёт себя линейно,
%%f(x) = kx + e,
%%а из гомогенности f(cx) = cf(x) следует e = 0
%%➡f(x) = kx,
%%➡sin dx = kdx, k=?

%%чтобы функция была равна своему аргументу
%%sin dx = dx,
%%нужно дополнительное условие (extra condition)
%%типа f(1) = 1

%%но как это сделать для бесконечнома́лого аргумента?..🤔

\en{For}\ru{Для}
\en{infinitesimal}\ru{бесконечном\'{а}лых}~$dx$\en{,}
\en{the~sine function}\ru{функция синуса}
${\sine(dx)}$
\en{behaves linearly}\ru{ведёт себя линейно}
\en{and}\ru{и}\ru{,}
%%\en{can therefore be}\ru{может поэтому быть}
%%\en{approximated by}\ru{аппроксимирована}
\textcolor{red}{\en{is therefore equal to}\ru{следовательно, равна}}
\en{its argument}\ru{своему аргументу}\::
%
\begin{equation}\label{sinedxequalsdx}
\sine(dx) = dx
\hspace{.2ex} .
\end{equation}

\end{document}
