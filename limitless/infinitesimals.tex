\documentclass[11pt, twoside]{book}

\usepackage[utf8]{inputenc}
\usepackage[T2A,T1]{fontenc}

\usepackage[english, greek, russian]{babel}

\usepackage{bm} % bold math symbols

\usepackage{amsmath}
\usepackage{amsfonts} % for \mathbb

\usepackage{xcolor} % for \textcolor

%%\usepackage{accents}

%%\newcommand{\scircabove}{\hbox{\fontfamily{lmr}\fontsize{6}{0}\selectfont$\,\circ$}}
%%\DeclareRobustCommand{\mathcircleabove}{\accentset{\scircabove}}

\def\spacebetweenparagraphs{4mm minus0.5mm}
\setlength{\parskip}{\spacebetweenparagraphs}

\newcommand\sine{\operatorname{sin}} % \operatorname{sine}
\newcommand\cosine{\operatorname{cos}} % \operatorname{cosine}

\RequirePackage{ifthen}
\newif\ifen
\newif\ifru
\newcommand{\en}[1]{\ifen#1\fi}
\newcommand{\ru}[1]{\ifru#1\fi}

%%\entrue
\rutrue

\begin{document}

\en{\selectlanguage{english}}
\ru{\selectlanguage{russian}}

%% \en{}\ru{}
%% \en{}\ru{}

\en{Let}\ru{Пусть}~$dx$
\en{be}\ru{будет}
\en{a~small change}\ru{м\'{а}лым изменением}
\en{in the~variable}\ru{переменной}~$x$
%
\begin{equation}\label{asmallchangeinx}
dx = x' \hspace{-0.4ex} - x
\hspace{.2ex} ,
\end{equation}
%
\en{where}\ru{где}
${x' \hspace{-0.33ex} = x + dx}$
\en{is not far}\ru{не~далек\'{о}}
\en{from}\ru{от}~$x$,
%%\en{approaches}\ru{подходит~к}~$x$,
\en{but}\ru{но}
${x' \hspace{-0.33ex} \neq x}$
\en{and hence}\ru{и~отсюда}
${dx \neq 0}$.

\en{The~change}\ru{Изменение}~$dx$
\en{is}\ru{есть}
\emph{\en{infinitesimal}\ru{бесконечном\'{а}лое}}\ru{,}
\en{if}\ru{если}
\en{it}\ru{оно}
\en{approaches zero}\ru{подходит к~нулю}
\en{being}\ru{будучи}
\en{not exactly zero}\ru{не в~точности нулём},
\en{while}\ru{в~то время как}
\en{higher powers}\ru{более высокие степени}
\en{of~}$dx$,
\en{such as}\ru{такие как}
${(dx)^2}$\hbox{\hspace{-0.33ex},}
${(dx)^3}$\hbox{\hspace{-0.33ex},}
${(dx)^4}$
\en{and so on}\ru{и~так далее},
%%\en{are negligible}\ru{пренебрежимы}
%%\en{compared to}\ru{по~сравнению с}~%
\en{are infinitesimally smaller than}\ru{бесконечном\'{а}ло м\'{е}ньше}~%
$dx$.
%
\en{Thus}\ru{Так что}
%
\begin{equation}\label{whatmeansinfinitesimal}
dx \neq 0
\hspace{.1ex} , \hspace{.66em}
\text{\en{but}\ru{но}}
\hspace{.5em}
(dx)^2 \hspace{-0.2ex} = 0
\hspace{.1ex} , \hspace{.33em}
(dx)^3 \hspace{-0.2ex} = 0
\hspace{.1ex} , \hspace{.33em}
\ldots
\end{equation}
%
---
\en{as}\ru{как}
\en{the~definition}\ru{определение}
\en{of~infinitesimality}\ru{бесконечномалости}.
%
\en{An~infinitesimal change}\ru{Бесконечном\'{а}лое изменение}
\en{is called}\ru{называется}
\en{a~}\emph{\en{differential}\ru{дифференциалом}}.

....

$\Rightarrow$ ${\forall \lambda,\mu \in \mathbb{R} \hspace{.5ex} ({\neq}\infty)}$
${\sine(\lambda dx + \mu dx) = (\lambda + \mu) \sine(dx)}$

%%✨из аддитивности f(x + y) = x + y получается, что
%%для бесконечнома́лых~$dx$
%%функция синуса $\sine(dx)$
%%ведёт себя линейно,
%%f(x) = kx + e,
%%а из гомогенности f(cx) = cf(x) следует e = 0
%%➡f(x) = kx,
%%➡sin dx = kdx, k=?

%%чтобы функция была равна своему аргументу
%%sin dx = dx,
%%нужно дополнительное условие (extra condition)
%%типа f(1) = 1

%%но как это сделать для бесконечнома́лого аргумента?..🤔

\en{For}\ru{Для}
\en{infinitesimal}\ru{бесконечном\'{а}лых}~$dx$\en{,}
\en{the~sine function}\ru{функция синуса}
${\sine(dx)}$
\en{behaves linearly}\ru{ведёт себя линейно}
\en{and}\ru{и}\ru{,}
%%\en{can therefore be}\ru{может поэтому быть}
%%\en{approximated by}\ru{аппроксимирована}
\textcolor{red}{\en{is therefore equal to}\ru{следовательно, равна}}
\en{its argument}\ru{своему аргументу}\::
%
\begin{equation}\label{sinedxequalsdx}
\sine(dx) = dx
\hspace{.2ex} .
\end{equation}

\end{document}
