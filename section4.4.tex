\section{\en{Hooke’s law}\ru{Закон Hooke’а}
\en{for an~isotropic material}\ru{для изотропного материала}}

\label{section:hooke.forisotropic}

\en{For an~isotropic material}\ru{Для изотропного материала}\en{,}
\eqref{Hooke.for-symmetric-crystals}
\en{is satisfied}\ru{удовлетворяется}
\en{a~priori}\ru{априори}
\en{for any}\ru{для любого}
\en{orthogonal tensor}\ru{ортогонального тензора}~$\bm{Q}$.
%
\en{Here}\ru{Здесь}
\en{it’s easier to assume that}\ru{проще предположить, что}
$\potentialenergydensity(\infinitesimaldeformation)$
\en{becomes an~isotropic function}\ru{становится изотропной функцией},
\en{which depends}\ru{которая зависит}
\en{only}\ru{лишь}
\en{on the invariants from the solution}\ru{от инвариантов из решения}
\en{of the characteristic equation}\ru{характеристического уравнения}~\chapterdotsectionref{chapter:mathapparatus}{section:eigenvectorseigenvalues}

\begin{equation}
2 \potentialenergydensity = A \bigl( \firstcharacteristicinvariant \bigr)^{\hspace{-0.2ex}2} + B \bigl( \secondcharacteristicinvariant \bigr)
.
\end{equation}

\begin{otherlanguage}{russian}

$\potentialenergydensity(\infinitesimaldeformation)$ это квадратичная функция,
функция с~членами степени не~выше второй,
(или \inquotes{квадратичная форма})

\end{otherlanguage}

\begin{equation}
2 \potentialenergydensity( \infinitesimaldeformation )
= A \infinitesimaldeformation\tracedot \infinitesimaldeformation\tracedot
+ B \infinitesimaldeformation \dotp \infinitesimaldeformation
\end{equation}

...........

\noindent
\begin{equation}\label{derivativeoftracebythetensoritself}
......
=
\UnitDyad
\end{equation}

\en{In the components}\ru{В~компонентах}
\en{for an~isotropic medium}\ru{для изотропной среды}
\en{we have}\ru{имеем}

\begin{equation}\label{componentsofthestiffnesstensor.foranisotropicbody}
A_{ijpq} \hspace{-0.2ex} =
\lambda
\KroneckerDelta{ij}
\KroneckerDelta{pq} \hspace{-0.2ex}
+
\mu
\bigl(
\KroneckerDelta{ip}
\KroneckerDelta{jq} \hspace{-0.2ex}
+
\KroneckerDelta{iq}
\KroneckerDelta{jp}
\bigr)
\end{equation}\:---
\en{these are}\ru{это}
\en{components}\ru{компоненты}
\en{of an~isotropic tensor}\ru{изотропного тензора}
\en{of the fourth complexity}\ru{четвёртой сложности}.
\en{These components}\ru{Эти компоненты}
\en{don’t change}\ru{не меняются}
\en{when}\ru{когда}
\en{a~basis rotates}\ru{базис вращается}.

..................

\chapterdotsectionref{chapter:mathapparatus}{section:tensors.symmetric+skewsymmetric}

\eqref{eigenvalues:eq}
