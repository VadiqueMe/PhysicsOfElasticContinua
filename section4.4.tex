\section{\en{Hooke’s law}\ru{Закон Hooke’а}
\en{for isotropic medium}\ru{для изотропной среды}}

\label{section:hooke.forisotropic}

\newcommand\thefirstlameparameter{\lambda}
\newcommand\thesecondlameparameter{\mu}

\newcommand\shearmodulus\thesecondlameparameter
\newcommand\shearmodulusG{G}
\newcommand\youngsmodulus{E}
\newcommand\poissonratio{\nu}
\newcommand\bulkmodulus{K}

\en{If a~material is isotropic}\ru{Если материал изотропный},
\eqref{Hooke.for-symmetric-crystals}~%
\en{is satisfied}\ru{удовлетворяется}
\en{for any}\ru{для любого}
\en{orthogonal tensor}\ru{ортогонального тензора}~$\orthogonaltensor$.
%
\en{So}\ru{Так что}
\en{here}\ru{здесь},
\en{for}\ru{для}
\en{a~linear elastic}\ru{линейно упругой}
\en{isotropic}\ru{изотропной}
\en{medium}\ru{среды}/\en{body}\ru{т\'{е}ла},
\en{it’s easier}\ru{проще}
\en{to~assume that}\ru{предположить, что}
\en{the~}\en{potential energy density}\ru{плотность потенциальной энергии}~%
$\potentialenergydensity(\infinitesimaldeformation)$
\en{becomes an~isotropic function}\ru{становится изотропной функцией}\footnote{%
\en{Such a~function}\ru{Такая функция}
\en{depends}\ru{зависит}
\en{only}\ru{лишь}
\en{on the~invariants}\ru{от инвариантов},
\en{such as}\ru{таких как}
\en{the~coefficients}\ru{коэффициенты}
\en{from the~solution}\ru{из решения}
\en{of~the~characteristic equation}\ru{характеристического уравнения}~%
\eqrefwithchapterdotsection{theCharacteristicEquation}{chapter:mathapparatus}{section:eigenvectorseigenvalues}.
\en{Any function}\ru{Любая функция}\ru{,}
\en{whose arguments are}\ru{аргументы которой\:---}
\en{only invariants}\ru{только инварианты}\ru{,}
\en{is isotropic}\ru{изотропна}.}

\begin{equation}
2 \hspace{.2ex} \potentialenergydensity
= A {\bigl( \firstcharacteristicinvariant \bigr)}^{\hspace{-0.2ex}2} + B \bigl( \secondcharacteristicinvariant \bigr)
.
\end{equation}

$\potentialenergydensity(\infinitesimaldeformation)$
\en{is a~quadratic function}\ru{это квадратичная функция}
(\en{or}\ru{или} \en{a~}\inquotes{\en{quadratic form}\ru{квадратичная форма}})
\en{with terms}\ru{с~членами}
\en{of the second degree}\ru{второй степени}.

\noindent
\inquotes{\en{quadratic form}\ru{квадратичная форма}}
$=$
\inquotes{\en{homogeneous polynomial}\ru{однородный многочлен}
\en{of~the~second degree}\ru{второй степени}}
$=$
\inquotes{\en{function}\ru{функция}
\en{with quadratic terms only}\ru{с~только квадратичными членами}}

\en{The isotropic}\ru{Изотропная}
\en{function}\ru{функция}

\nopagebreak\vspace{-0.3em}
\begin{equation*}
\anyfirstinvariant(\infinitesimaldeformation),
\anysecondinvariant(\infinitesimaldeformation)
\mapsto
\potentialenergydensity(\infinitesimaldeformation)
\hspace{.2ex} , \hspace{.7em}
%
\potentialenergydensity
\narroweq
\potentialenergydensity(\hspace{.2ex} \anyfirstinvariant, \anysecondinvariant \hspace{.2ex})
\end{equation*}

\vspace{-0.4em}\noindent
\en{has the~terms}\ru{имеет члены}
\en{of only}\ru{только}
\en{the~}\hbox{2$^{\textrm{\en{nd}\ru{ой}}}$\hspace{-0.2ex}} \en{degree}\ru{степени},
\en{not}\ru{не} \en{higher}\ru{выше}.
\en{Therefore}\ru{Поэтому}
\en{there’s no }\en{third invariant}\ru{третьего инварианта}~$\anythirdinvariant(\infinitesimaldeformation)$
\en{among the~arguments of}\ru{среди аргументов}~$\potentialenergydensity$\ru{ нет}.

\nopagebreak\vspace{-0.2em}
\begin{gather*}
\anyfirstinvariant(\infinitesimaldeformation) \hspace{-0.2ex}
= \trace{\infinitesimaldeformation}
= \infinitesimaldeformation\tracedot
= \hspace{-0.2ex} \boldnabla \hspace{-0.2ex} \dotp \hspace{-0.1ex} \fieldofdisplacements
\\
%
\anysecondinvariant(\infinitesimaldeformation) \hspace{-0.2ex}
= \infinitesimaldeformation \hspace{-0.2ex} \dotdotp \hspace{-0.1ex} \infinitesimaldeformation
\end{gather*}

.....

\nopagebreak\vspace{-0.2em}
\begin{equation}
2 \hspace{.2ex} \potentialenergydensity( \infinitesimaldeformation )
=
A \hspace{.2ex} \infinitesimaldeformation\tracedot \hspace{.2ex} \infinitesimaldeformation\tracedot
\hspace{.1ex} + \hspace{-0.1ex}
B \hspace{.1ex} \infinitesimaldeformation \hspace{-0.2ex} \dotdotp \hspace{-0.1ex} \infinitesimaldeformation
\end{equation}

......

\en{The derivative}\ru{Производная}
\en{of}\ru{следа}~\ru{(}trace\ru{)}
\en{by the tensor itself}\ru{по самом\'{у} тензору}
\en{is the~unit dyad}\ru{есть единичная диада}

\nopagebreak\vspace{-0.3em}
\begin{equation}\label{thederivativeoftracebythetensoritself}
\scalebox{.92}{$
   \displaystyle
   \frac{ \raisemath{-0.133em}{
      \partial \infinitesimaldeformation\tracedot
   }%close \raisemath
   }%close \frac
   { \raisemath{-0.07em}{\partial \infinitesimaldeformation} }
$}
=
\UnitDyad
\end{equation}

\en{An~isotropic medium}\ru{Изотропная среда}
\en{is characterized}\ru{характеризуется}
\en{by two}\ru{двумя}
\en{non-zero}\ru{ненулевыми}
\en{elastic constants}\ru{упругими константами}
(\inquotes{\en{elastic moduli}\ru{упругими модулями}})

\nopagebreak\vspace{-0.2em}
\begin{equation}\label{theelasticpotentialforanisotropicmedium}
\potentialenergydensity(\infinitesimaldeformation) =
\alpha \hspace{.2ex} \anyfirstinvariant^2 (\infinitesimaldeformation) \hspace{-0.2ex}
+ \beta \hspace{.2ex} \anysecondinvariant (\infinitesimaldeformation)
\end{equation}

\begin{equation*}
\scalebox{.92}{$
   \displaystyle
   \frac{ \raisemath{-0.133em}{ \partial \hspace{.1ex} \potentialenergydensity } }
   { \raisemath{-0.07em}{ \partial \infinitesimaldeformation } }
$}
\end{equation*}

\begin{gather*}
   \scalebox{.92}{$
   \displaystyle
   \frac{ \raisemath{-0.133em}{
      \partial \infinitesimaldeformation\tracedot
   }%close \raisemath
   }%close \frac
   { \raisemath{-0.07em}{\partial \infinitesimaldeformation} }
   $}
   \dotdotp
   \infinitesimaldeformation\tracedot
   +
   \infinitesimaldeformation\tracedot
   \dotdotp
   \scalebox{.92}{$
   \displaystyle
   \frac{ \raisemath{-0.133em}{
      \partial \infinitesimaldeformation\tracedot
   }%close \raisemath
   }%close \frac
   { \raisemath{-0.07em}{\partial \infinitesimaldeformation} }
   $}
\\
%
=
2 \infinitesimaldeformation\tracedot \dotdotp \UnitDyad
\end{gather*}

.........

\en{In components}\ru{В~компонентах}
\en{for an~isotropic medium}\ru{для изотропной среды}

\begin{equation}\label{componentsofthestiffnesstensor.forisotropic}
A_{ijpq} \hspace{-0.2ex} =
\thefirstlameparameter \hspace{.2ex}
\KroneckerDelta{ij}
\KroneckerDelta{pq} \hspace{-0.2ex}
+
\thesecondlameparameter
\bigl(
\KroneckerDelta{ip}
\KroneckerDelta{jq} \hspace{-0.2ex}
+
\KroneckerDelta{iq}
\KroneckerDelta{jp}
\bigr)
\end{equation}\:---
\en{these are}\ru{это}
\en{components}\ru{компоненты}
\en{of an~isotropic tensor}\ru{изотропного тензора}
\en{of the fourth complexity}\ru{четвёртой сложности},
\en{which}\ru{которые}
\en{don’t change}\ru{не меняются}
\en{when}\ru{когда}
\en{the~basis rotates}\ru{базис вращается}.

.......

\subsection*{\en{Pairs of elastic moduli}\ru{Пары модулей упругости}}

\en{There are}\ru{Вот}
\en{versions}\ru{версии}
\en{of the }\ru{закона }Hooke’\ru{а}\en{s}\en{ law}
\en{for}\ru{для}
\en{the various pairs}\ru{разных пар}
\en{of elastic constants}\ru{упругих констант}
(\en{elastic moduli}\ru{модулей упругости}).
%
$\thefirstlameparameter$ \en{and}\ru{и}~$\thesecondlameparameter$\en{ are}\ru{\:---}
\href{https://en.wikipedia.org/wiki/Lam%C3%A9_parameters}{\en{the~}\ru{параметры }Lam\'{e}\en{ parameters}},
$\shearmodulus$~(\en{sometimes}\ru{иногда}~$\shearmodulusG$)\en{ is}\ru{\:---}
\href{https://en.wikipedia.org/wiki/Shear_modulus}{\en{the shear modulus}\ru{модуль сдвига}},
$\youngsmodulus$\:---
\href{https://en.wikipedia.org/wiki/Young%27s_modulus}{\en{the~}\ru{модуль }Young’\ru{а}\en{s}\en{ modulus}}~%
(\en{the~modulus}\ru{модуль}
\en{of~tension or~compression}\ru{растяжения или сжатия}),
$\poissonratio$\en{ is}\ru{\:---}
\href{https://en.wikipedia.org/wiki/Poisson%27s_ratio}{\en{the~}\ru{коэффициент }Poisson’\ru{а}\en{s}\en{ ratio}},
$\bulkmodulus$\:---
\href{https://en.wikipedia.org/wiki/Bulk_modulus}{\en{the~}\en{bulk}\ru{объёмный}
\en{modulus}\ru{модуль}}.

......

\en{A\:priori}\ru{Априорные}
\en{conditions}\ru{условия}
\en{for values}\ru{для значений}
\en{of the elastic moduli}\ru{модулей упругости}
\en{are}\ru{таковы}

\nopagebreak\vspace{-0.4em}
\begin{equation}\label{inequalitiesforelasticmoduli}
\begin{array}{r@{\hspace{.5em}\text{---}\hspace{.8ex}}l}
\youngsmodulus > 0 &
\begin{minipage}[t]{.7\columnwidth}
\en{if something is stretched}\ru{если что-то растягивается},
\en{it elongates}\ru{оно удлиняется},
\end{minipage}
\vspace{.3em}
\\
%
\shearmodulus > 0 &
\begin{minipage}[t]{.7\columnwidth}
\en{the~shear}\ru{сдвиг}
\en{goes}\ru{идёт}
\en{in~the~same way as}\ru{туда~же, куда и}
\en{the }\en{tangential}\ru{касательная}~(\en{shear}\ru{сдвиговая})
\en{stress component}\ru{компонента напряжения},
\end{minipage}
\vspace{.3em}
\\
%
\bulkmodulus > 0 &
\begin{minipage}[t]{.7\columnwidth}
\en{due to external pressure}\ru{от внешнего давления}
\en{the volume}\ru{объём}
\en{decreases}\ru{уменьшается}.
\end{minipage}
\end{array}
\end{equation}

\vspace{-0.4em}\noindent
\en{Inequalities}\ru{Неравенства}
\en{for the~elastic moduli}\ru{для упругих модулей}~\eqref{inequalitiesforelasticmoduli}
\en{are sufficient}\ru{достаточны}
\en{for}\ru{для}
\en{the~positivity of}\ru{позитивности}~$\potentialenergydensity$.

\en{When}\ru{Когда}
${ \poissonratio \hspace{-0.4ex} \to \hspace{-0.2ex} \smallerdisplaystyleonehalf }$,
\en{the~material}\ru{материал}
\en{becomes incompressible}\ru{становится несжимаемым}
\en{with}\ru{с}~%
\en{an~infinitely large}\ru{бесконечно больш\'{и}м}
\en{bulk modulus}\ru{объёмным модулем}~%
${ \bulkmodulus \hspace{-0.5ex} \to \hspace{-0.25ex} \infty }$.
%
\en{Negative values of}\ru{Отрицательные значения}~$\poissonratio$
\en{are possible}\ru{возможны}\footnote{%
\en{Such a~material}\ru{Такой материал},
\en{called}\ru{называемый}
\href{https://en.wikipedia.org/wiki/Auxetics}{\en{an~auxetic}\ru{ауксетиком}},
\en{becomes}\ru{становится}
\en{thicker}\ru{толще}
\en{when it stretches}\ru{когда растягивается}.}
\en{too}\ru{тоже}.

.........

\subsection*{Inverse relations, the complementary energy}

\begin{equation}\label{invertedHooke.forisotropic}
2 \potentialenergydensity = \linearstress \dotdotp \infinitesimaldeformation,
\hspace{1em}
\infinitesimaldeformation
= \scalebox{.92}{$ \displaystyle
   \frac{ \raisemath{-0.15em}{
      \partial \hspace{.1ex} \complementaryenergydensity
   } }{ \raisemath{.033em}{
      \partial \linearstress
   } }
$}
= \hspace{-0.1ex} \pliabilitytensor \dotdotp \linearstress
\hspace{.1ex} =
\linearstress \dotdotp \hspace{-0.1ex} \pliabilitytensor
\hspace{.1ex} .
\end{equation}

the Legendre transform

The complementary energy~$\complementaryenergydensity$

\begin{equation}\label{someotherlabel.for.thecomplementaryenergy}
\complementaryenergydensity( \hspace{-0.1ex} \linearstress \hspace{.12ex} ) \hspace{-0.1ex}
= \linearstress \dotdotp \infinitesimaldeformation
- \hspace{.1ex} \potentialenergydensity(\infinitesimaldeformation)
\hspace{.1ex} .
\end{equation}

\en{In the~linear theory}\ru{В~линейной теории}\en{,}
\en{the~}\inquotes{\en{complementary energy}\ru{дополнительная энергия}}
\en{is numerically equal to}\ru{численно равна}
\en{the }\en{elastic potential energy}\ru{упругой потенциальной энергии}

\begin{equation*}
\tikzmark{beginTwoElasticEnergies}
2 \potentialenergydensity
\tikzmark{endTwoElasticEnergies}
-
\tikzmark{beginOneElasticPotentialEnergy}
\potentialenergydensity(\infinitesimaldeformation)
\tikzmark{endOneElasticPotentialEnergy}
= \complementaryenergydensity(\linearstress).
\end{equation*}
\AddUnderBrace[line width=.75pt][.1ex,-0.1ex][xshift=-0.1ex, yshift=-0.25em]%
{beginTwoElasticEnergies}{endTwoElasticEnergies}%
{\scalebox{.8}{$ \linearstress \hspace{-0.2ex} \dotdotp \hspace{-0.2ex} \infinitesimaldeformation $}}
\AddUnderBrace[line width=.75pt][-0.1ex,-0.1ex][xshift=-0.1ex, yshift=-0.25em]%
{beginOneElasticPotentialEnergy}{endOneElasticPotentialEnergy}%
{\scalebox{.8}{$ \smash{\smalldisplaystyleonehalf} \hspace{.2ex} \linearstress \hspace{-0.2ex} \dotdotp \hspace{-0.2ex} \infinitesimaldeformation $}}

\begin{equation*}
\complementaryenergydensity(\linearstress) = \potentialenergydensity(\infinitesimaldeformation)
\end{equation*}

...........

