\section{\en{Hooke’s law}\ru{Закон Hooke’а}
\en{for an~isotropic material}\ru{для изотропного материала}}

...........

\noindent
\begin{equation}\label{derivativeoftracebythetensoritself}
{}
=
\UnitDyad
\end{equation}

\en{In the components}\ru{В~компонентах}
\en{for an~isotropic medium}\ru{для изотропной среды}
\en{we have}\ru{имеем}

\begin{equation}\label{componentsofthestiffnesstensor.foranisotropicbody}
A_{ijpq} \hspace{-0.2ex} =
\lambda
\KroneckerDelta{ij}
\KroneckerDelta{pq} \hspace{-0.2ex}
+
\mu
\bigl(
\KroneckerDelta{ip}
\KroneckerDelta{jq} \hspace{-0.2ex}
+
\KroneckerDelta{iq}
\KroneckerDelta{jp}
\bigr)
\end{equation}\:---
\en{these are}\ru{это}
\en{components}\ru{компоненты}
\en{of an~isotropic tensor}\ru{изотропного тензора}
\en{of the fourth complexity}\ru{четвёртой сложности}.
\en{These components}\ru{Эти компоненты}
\en{don’t change}\ru{не меняются}
\en{when}\ru{когда}
\en{a~basis rotates}\ru{базис вращается}.

..................

