\newcommand\resumename{Mon résumé}
\chapter*{\resumename}
\addcontentsline{toc}{chapter}{\resumename}

\en{

\noindent
In this book,
all models of an~elastic continuum are presented:
the non\-linear and linear,
the micropolar and the classical momentless;
the three-di\-men\-sion\-al,
the two-di\-men\-sion\-al (the shells and~plates),
the one-di\-men\-sion\-al (the rods, including the thin\hbox{-}walled ones).
The fundamentals of dynamics\:---
oscillations, waves and stability\:---
are explained.
For the thermo\-elasticity and the magneto\-elasticity,
the~summary of the classical theories
of the thermo\-dynamics and electro\-dynamics is given.
Theories of~defects and fractures are also described.
Some approaches to~modeling inhomogeneous materials
(known as composites)
are also shown.

\vspace{\baselineskip}

\noindent
The~book is written using the compact and elegant direct indexless tensor notation,
operating with the invariant objects, tensors.
The mathematical apparatus for interpreting direct tensor relations
is contained in the first chapter.

}\ru{

\noindent
В~этой книге
представлены все модели упругого континуума:
нелинейные и~линейные,
микрополярные и~классические безмоментные;
трёх\-мерные, дву\-мерные (оболочки и~пластины),
одно\-мерные
(стержни, в~том~числе тонко\-стен\-ные).
Изложены основы динамики\:--- колебания, волны и~устойчивость.
Для термо\-упругости и~магнито\-упругости
даётся сводка классических теорий
термо\-динамики и~электро\-динамики.
Описаны теории дефектов и~трещин.
Также показаны
подходы
к~моделированию
неоднородных материалов — композитов.

\vspace{\baselineskip}

\noindent
Книга написана
с~использованием компактной и~элегантной прямой безиндексной тензорной записи,
оперирующей с~инвариантными объектами — тензорами.
Математический аппарат для интерпретации прямых тензорных соотношений содержится почти во всей первой главе.

\vspace{\baselineskip}

\noindent
\ru{Я пишу эту книгу одновременно на двух языках}\en{I write this book simultaneously in two languages},
\ru{английском и~русском}\en{English and Russian}.
\en{The reader}\ru{Читатель}
\en{is free}\ru{свободен}
\en{to choose for yourself}\ru{выбрать для себя}
\en{any option}\ru{любой вариант}
\en{out of two}\ru{из двух}
\en{or to read both}\ru{или читать оба}.

}

\vspace{3\baselineskip}

\noindent
\scalebox{.9}{\href{https://github.com/VadiqueMe/PhysicsOfElasticContinua}{\textit{github.com/VadiqueMe/PhysicsOfElasticContinua}}}

\newpage

