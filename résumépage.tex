\newcommand\resumename{Mon résumé}
\chapter*{\resumename}
\addcontentsline{toc}{chapter}{\resumename}

\noindent
\en{In this book}\ru{В~этой книге}\en{,}
\en{all models of an~elastic continuum are presented}\ru{представлены все модели упругого континуума}:
\en{the non\-linear and linear}\ru{нелинейные и~линейные},
\en{the micropolar and the classical momentless}\ru{микрополярные и~классические безмоментные};
\en{the three-di\-men\-sion\-al}\ru{трёх\-мерные},
\en{the two-di\-men\-sion\-al}\ru{дву\-мерные}
(\en{shells and~plates}\ru{оболочки и~пластины}),
\en{the one-di\-men\-sion\-al}\ru{одно\-мерные}
(\en{rods}\ru{стержни},
\en{including the thin\hbox{-}walled ones}\ru{включая тонко\-стен\-ные}).
\en{The fundamentals of dynamics\:--- oscillations, waves and stability\:--- are explained}\ru{Объяснены основы динамики\:--- колебания, волны и~устойчивость}.
\en{For the thermo\-elasticity and the magneto\-elasticity}\ru{Для термо\-упругости и~магнито\-упругости}\en{,}
\en{the~summary of the classical theories}\ru{даётся сводка классических теорий}
\en{of the thermo\-dynamics and electro\-dynamics}
\ru{термо\-динамики и~электро\-динамики}\en{ is given}.
\en{The dynamics of destruction}\ru{Динамика разрушения}
\en{is described}\ru{описана}
\en{via the theories of~defects and fractures}\ru{теориями дефектов и~трещин}.
\en{Approaches to~modeling}\ru{Подходы к~моделированию}
\en{the inhomogeneous materials}\ru{неоднородных материалов}
(\en{the composites}\ru{композитов})
\en{are also shown}\ru{также показаны}.

\vspace{\baselineskip}

\noindent
\en{The~book}\ru{Книга}
\en{is written}\ru{написана}
\en{using the compact and elegant}\ru{с~использованием компактной и~элегантной}
\en{direct indexless}\ru{прямой безиндексной}
\en{tensor notation}\ru{тензорной записи}.
\en{The mathematical apparatus}\ru{Математический аппарат}
\en{for interpretation}\ru{для интерпретации}
\en{of the direct tensor relations}\ru{прямых тензорных соотношений}
\en{is located in the first chapter}\ru{находится в~первой главе}.

\vspace{\baselineskip}

\noindent
\en{I am writing this book}\ru{Я пишу эту книгу}
\en{simultaneously in the two languages}\ru{одновременно на двух языках},
\ru{английском и~русском}\en{English and Russian}.
\en{The reader}\ru{Читатель}
\en{is free}\ru{свободен}
\en{to choose for yourself}\ru{выбрать для себя}
\en{any option}\ru{любой вариант}
\en{of the two}\ru{из двух}
\en{or to read both}\ru{или читать оба}.

\vspace{3\baselineskip}

\noindent
\scalebox{.9}{ \href{https://github.com/VadiqueMe/PhysicsOfElasticContinua}{\textit{github.com/VadiqueMe/PhysicsOfElasticContinua}} }

\newpage

