\en{\section{Matrices, permutations and determinants}}

\ru{\section{Матрицы, перестановки и определители}}

\label{para:matrices+permutations+determinants}

\en{The matrices}\ru{Матрицы}
\en{are the convenient tool}\en{это удобный инструмент}
\en{for solving}\ru{для решения}
\en{the systems of linear equations}\ru{систем линейных уравнений}.
\en{And for arranging of elements}\ru{И~для упорядочивания элементов}.
\en{Any matrix has}\ru{У~любой матрицы}
\en{the same number of elements}\ru{одно и то~же число элементов}
\en{in each row}\ru{в~каждой строке}
\en{and}\ru{и}
\en{the same number of elements}\ru{одно и то~же число элементов}
\en{in each column}\ru{в~каждом столбце}.

\en{Do you need}\ru{Тебе нужны}
\en{the two-dimensional arrays}\ru{двумерные массивы}?
\en{The matrices}\ru{Матрицы}
\en{can be presented}\ru{могут быть представлены}
\en{as tables}\ru{как таблицы}\ru{,}
\en{full of rows and columns}\ru{полные строк и~столбцов}.

\en{Do you want}\ru{Ты хочешь}
\en{a~rectangular arrangement}\ru{прямоугольное упорядочивание}
\en{of your elements}\ru{твоих элементов}?
\en{Matrices}\ru{Матрицы}
\en{are full}\ru{полн\'{ы}}
\en{of numbers and expressions}\ru{чисел и~выражений}
\en{in the rows and columns}\ru{в~строках и~столбцах}.

\ru{Ты знаешь}\en{Do you know}\ru{,}
\en{that}\ru{что}
\en{matrices}\ru{матрицы}
\en{are sometimes called}\ru{иногда называют}
\en{arrays}\ru{массивами}?

%--------------------

\subsection*{\en{Matrix dimensions}\ru{Размерности матрицы}}

\en{Matrices}\ru{Матрицы}
\en{come in all sizes}\ru{бывают всех размеров}\ru{,}
\en{that are dimensions}
\ru{которые и~есть размерности}.

\en{The dimension of a~matrix}\ru{Размерность матрицы}
\en{consists of}\ru{состоит из}
\en{the number of rows}\ru{числ\'{а} строк},
\en{then}\ru{затем}
\en{a~multiplication sign}\ru{знака умножения}
(\inquotes{×} \en{is used}\ru{используется}
\en{the most often}\ru{чаще всего})\en{,}
\en{and then}\ru{а~затем}
\en{the number of columns}\ru{числ\'{а} столбцов}.

{\small
\noindent
\en{Examples}\ru{Примеры}.

\nopagebreak\vspace{.1em}
\begin{gather*}
\displaystyle
\underset{ \raisemath{.1em}{ \scriptscriptstyle 3×3 } }{\bigl[ \hspace{.2ex} \mathcal{A} \hspace{.2ex} ]}
\hspace{.1ex} = \hspace{-0.2ex} ......
\end{gather*}

Matrix ${[ \hspace{.2ex}\mathrm{A}\hspace{.2ex} ]}$ is a~${3{×}3}$ matrix, because it has 3~rows and 3~columns.
Matrix ${[ \hspace{.2ex}\mathrm{B}\hspace{.2ex} ]}$ has 2~rows and 4~columns, so its dimension is~${2{×}4}$.
Matrix ${[ \hspace{.2ex}\mathrm{C}\hspace{.2ex} ]}$ is a~column matrix (that is a~matrix with just one column), and its dimension is~${3{×}1}$.
And ${[ \hspace{.2ex}\mathrm{D}\hspace{.2ex} ]}$ is a~row matrix with dimension~${1{×}6}$.
\par}

%--------------------

\subsection*{\en{The matrix algebra}\ru{Матричная алгебра}}

\en{The matrix algebra}\ru{Матричная алгебра}
\en{includes}\ru{включает}
\en{the linear operations}\ru{линейные операции}\:---
\en{the addition of matrices}\ru{сложение матриц}
\en{and}\ru{и}
\en{the multiplication by scalar}\ru{умножение на скаляр}.

\en{The dimension of a~matrix}\ru{Размерность матрицы}
\en{is essential}\ru{существенна}
\en{for the binary operations}\ru{для бинарных операций},
\en{that is}\ru{то есть}
\en{for operations}\ru{для операций}
\en{involving}\ru{с~участием}
\en{the two matrices}\ru{двух матриц}.

\en{An~addition}\ru{Сложение}
\en{or}\ru{или}
\en{subtraction}\ru{вычитание}
\en{of the two matrices}\ru{двух матриц}
\en{is possible}\ru{возможно}
\en{only}\ru{только}
\en{if}\ru{если}
\en{they have}\ru{они имеют}
\en{the same sizes}\ru{те же размеры}.

%----

\subsection{\en{The multiplication of matrices}}\ru{Умножение матриц}}


.................


\nopagebreak\vspace{.1em}\begin{gather*}
\displaystyle\underset{\raisemath{.1em}{\scriptscriptstyle m×n}}{[ \hspace{.2ex}\mathcal{A}\hspace{.2ex} ]} \hspace{.1ex}
= \hspace{-0.2ex} \ldots
\end{gather*}

\en{The matrix of the result}\ru{Матрица результата},
\en{known as}\ru{известная как}
\inquotes{\en{the matrix product}\ru{матричное произведение}},
\en{has}\ru{имеет}
\en{the number of~rows}\ru{число строк}
\en{of the first multiplier matrix}\ru{первой матрицы-сомножителя}
\en{and the number of~columns}\ru{и~число столбцов}
\en{of the second matrix}\ru{второй матрицы}.


....................


\subsection*{\en{Square matrices}\ru{Квадратные матрицы}}

....


\subsection*{\en{Matrices and the one-dimensional arrays}\ru{Матрицы и одномерные массивы}}

\en{The two indices of a~table}\ru{Два индекса таблицы}\en{ is}\ru{\:---} \en{more than}\ru{больше, чем}
\en{the single index}\ru{единственный индекс}
\en{of a~one-dimensional array}\ru{одномерного массива}.
\en{Due to this}\ru{Из-за этого}\en{,}
\en{an~one-dimensional array}\ru{одномерный массив}
\en{could be}\ru{может быть}
\en{presented}\ru{представлен}
\en{as}\ru{как}
\en{a~table of rows}\ru{таблица строк}
\en{or}\ru{или}
\en{as}\ru{как}
\en{a~table of columns}\ru{таблица столбцов}.

\nopagebreak\vspace{-0.2em}\begin{equation*}
\scalebox{.9}{$
\left[ \hspace{-0.1ex}
\begin{array}{c@{\hspace{.5em}}c@{\hspace{.5em}}c}
h_{1\hspace{-0.1ex}1} & h_{12} & h_{13}
\end{array}
\hspace{-0.1ex} \right]
$}
\hspace{-0.1ex} ,
\end{equation*}

\vspace{-0.2em}\noindent
\en{or}\ru{либо}
\en{the vertical}\ru{вертикальные}
\en{tables}\ru{таблицы}

\nopagebreak\vspace{-0.2em}
\begin{equation*}
\scalebox{.9}{$
\left[ \hspace{-0.1ex}
\begin{array}{c}
v_{1\hspace{-0.1ex}1} \\[-0.1em]
v_{21} \\[-0.1em]
v_{31}
\end{array}
\hspace{-0.1ex} \right]
$}
\hspace{-0.2ex} .
\end{equation*}

${
\underset{\raisebox{.15em}{\scalebox{.7}{$i$,$\hspace{.15ex}j$}}}{\operatorname{det}} \hspace{.4ex} \delta_{i\hspace{-0.1ex}j} \hspace{-0.15ex} = 1
}$

...

\begin{otherlanguage}{russian}

\vspace{-0.2em}
А~символ чётности перестановки через детерминант\:--- как

\nopagebreak\vspace{-0.2em}\begin{equation*}
e_{pqr} \hspace{-0.2ex}
= e_{i\hspace{-0.1ex}j\hspace{-0.1ex}k} \hspace{.1ex} \delta_{pi} \delta_{\hspace{-0.1ex}qj} \delta_{rk} \hspace{-0.2ex}
= e_{i\hspace{-0.1ex}j\hspace{-0.1ex}k} \hspace{.1ex} \delta_{ip} \delta_{\hspace{-0.15ex}j\hspace{-0.1ex}q} \delta_{kr}
\hspace{.1ex} ,
\end{equation*}

\nopagebreak\vspace{-0.1em}
\begin{equation}\label{permutationsymbolasdeterminant}
e_{pqr} \hspace{-0.1ex}
= \hspace{.1ex}
\operatorname{det}\hspace{-0.25ex} \scalebox{.92}{$\left[ \begin{array}{ccc}
\delta_{1p} & \delta_{1q} & \delta_{1r} \\
\delta_{2p} & \delta_{2q} & \delta_{2r} \\
\delta_{3p} & \delta_{3q} & \delta_{3r}
\end{array} \hspace{-0.1ex} \right]$} \hspace{-0.2ex}
= \hspace{.1ex}
\operatorname{det}\hspace{-0.25ex} \scalebox{.92}{$\left[ \begin{array}{ccc}
\delta_{p1} & \delta_{p2} & \delta_{p3} \\
\delta_{q1} & \delta_{q2} & \delta_{q3} \\
\delta_{r1} & \delta_{r2} & \delta_{r3}
\end{array} \hspace{-0.1ex} \right]$}
\hspace{-0.2ex} .
\end{equation}

...

\en{A~determinant}\ru{Определитель}
\en{is not sensitive}\ru{не~чувствителен} \en{to transposing}\ru{к~транспонированию}:

\nopagebreak\vspace{-0.25em}\begin{equation*}
\underset{\raisebox{.15em}{\scalebox{.7}{$i$,$\hspace{.15ex}j$}}}{\operatorname{det}} \, A_{i\hspace{-0.1ex}j} \hspace{-0.16ex}
= \hspace{.1ex} \underset{\raisebox{.15em}{\scalebox{.7}{$i$,$\hspace{.15ex}j$}}}{\operatorname{det}} \, A_{j\hspace{-0.06ex}i} \hspace{-0.16ex}
= \hspace{.1ex} \underset{\raisebox{.15em}{\scalebox{.7}{$j\hspace{-0.2ex}$,$\hspace{.1ex}i$}}}{\operatorname{det}} \, A_{i\hspace{-0.1ex}j}
\hspace{.1ex} .
\end{equation*}

...

\inquotes{\en{The determinant}\ru{Определитель} \en{of the matrix product}\ru{матричного произведения} \en{of the two matrices}\ru{двух матриц} \en{is equal}\ru{равен} \en{to the product of determinants}\ru{произведению определителей} \en{of these matrices}\ru{этих матриц}}

\nopagebreak\vspace{-0.2em}\begin{equation}\label{determinantofmatrixproduct}
\underset{\raisebox{.15em}{\scalebox{.7}{$i$,$k$}}}{\operatorname{det}} \, B_{ik} \hspace{.4ex} \underset{\raisebox{.15em}{\scalebox{.7}{$k$,$\hspace{.15ex}j$}}}{\operatorname{det}} \, C_{kj} \hspace{-0.15ex}
= \hspace{.1ex} \underset{\raisebox{.15em}{\scalebox{.7}{$i$,$\hspace{.15ex}j$}}}{\operatorname{det}} \, B_{ik} \hspace{.1ex} C_{kj}
\end{equation}

\[
e_{\hspace{-0.25ex}f\hspace{-0.2ex}gh} \hspace{.33ex} \underset{\raisebox{.15em}{\scalebox{.7}{$m$,$n$}}}{\operatorname{det}} \, B_{m\mathcolor{blue}{s}} \hspace{.1ex} C_{\mathcolor{blue}{s}n} \hspace{-0.2ex}
= e_{pqr} \hspace{.1ex} B_{\hspace{-0.25ex}f\hspace{-0.1ex}\mathcolor{blue}{i}} C_{\mathcolor{blue}{i}p} \hspace{.1ex} B_{\hspace{-0.1ex}g\mathcolor{blue}{j}} C_{\hspace{-0.1ex}\mathcolor{blue}{j}\hspace{-0.1ex}q} \hspace{.1ex} B_{h\mathcolor{blue}{k}} C_{\mathcolor{blue}{k}r}
\hspace{-0.2ex}
\]

\[
e_{\hspace{-0.25ex}f\hspace{-0.2ex}gh} \hspace{.33ex} \underset{\raisebox{.15em}{\scalebox{.7}{$m$,$s$}}}{\operatorname{det}} \, B_{ms} \hspace{-0.2ex}
= e_{i\hspace{-0.1ex}j\hspace{-0.1ex}k} \hspace{.1ex} B_{\hspace{-0.25ex}f\hspace{-0.1ex}i} B_{\hspace{-0.1ex}gj} B_{hk}
\hspace{-0.2ex}
\]

\[
e_{i\hspace{-0.1ex}j\hspace{-0.1ex}k} \hspace{.33ex} \underset{\raisebox{.15em}{\scalebox{.7}{$s$,$n$}}}{\operatorname{det}} \, C_{sn} \hspace{-0.2ex}
= e_{pqr} \hspace{.1ex} C_{ip} C_{\hspace{-0.1ex}j\hspace{-0.1ex}q} C_{kr}
\hspace{-0.2ex}
\]

\[
e_{\hspace{-0.25ex}f\hspace{-0.2ex}gh} \hspace{.2ex} \mathcolor{black!66}{e_{i\hspace{-0.1ex}j\hspace{-0.1ex}k}} \hspace{.33ex} \underset{\raisebox{.15em}{\scalebox{.7}{$m$,$s$}}}{\operatorname{det}} \, B_{ms} \hspace{.33ex} \underset{\raisebox{.15em}{\scalebox{.7}{$s$,$n$}}}{\operatorname{det}} \, C_{sn} \hspace{-0.2ex}
= \mathcolor{black!66}{e_{i\hspace{-0.1ex}j\hspace{-0.1ex}k}} \hspace{.2ex} e_{pqr} \hspace{.1ex} B_{\hspace{-0.25ex}f\hspace{-0.1ex}i} B_{\hspace{-0.1ex}gj} B_{hk} \hspace{.1ex} C_{ip} C_{\hspace{-0.1ex}j\hspace{-0.1ex}q} C_{kr}
\hspace{-0.2ex}
\]

...

Определитель компонент \en{of a~bivalent tensor}\ru{бивалентного тензора} \en{is invariant}\ru{инвариантен}, он не~меняется с~поворотом базиса

\nopagebreak\vspace{-0.2em}\begin{equation*}
A'_{i\hspace{-0.1ex}j} \hspace{-0.16ex} = \cosinematrix{i'\hspace{-0.1ex}m} \hspace{.1ex} \cosinematrix{j'\hspace{-0.1ex}n} \hspace{.16ex} A_{mn}
\end{equation*}

