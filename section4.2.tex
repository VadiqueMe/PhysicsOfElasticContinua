\en{\section{The uniqueness of the solution in dynamics}}

\ru{\section{Уникальность решения в~динамике}}

\label{para:uniquenessfordynamicproblem}

\en{As is typical}\ru{Как это типично}
\en{for}\ru{для}
\en{linear mathematical physics}\ru{линейной математической физики},
\en{the~uniqueness theorem}\ru{теорема единственности}
\en{is proven}\ru{доказывается}
\inquotesx{\en{by~contradiction}\ru{от~противного}}[.]
\en{Assume that}\ru{Предположим, что}
\en{there are}\ru{существуют}
\en{two solutions}\ru{два решения}:
${\bm{u}_1 (\locationvector, t)}$ \en{and}\ru{и}~${\bm{u}_2 (\locationvector, t)}$.
\en{If}\ru{Если}
\en{the difference}\ru{разность}~${\smash{\bm{u}^{\hspace{-0.1ex}*}} \hspace{-0.2ex} \equiv \hspace{.12ex} \bm{u}_1 \hspace{-0.4ex} - \bm{u}_2}$
\en{will be equal to}\ru{будет равной}~$\bm{0}$,
\en{then}\ru{тогда}
\en{these solutions}\ru{эти решения}
\en{coincide}\ru{совпадают},
\en{that is}\ru{то есть}
\en{the~solution is unique}\ru{решение единственно}.

\en{But}\ru{Но}
\en{at first}\ru{сперв\'{а}}
\en{we’ll make sure}\ru{мы убедимся}
\en{of the existence}\ru{в~существовании}
\en{of the~energy integral}\ru{интеграла энергии}\:---
\en{by deriving}\ru{путём вывода}
\en{the balance of~mechanical energy equation}\ru{уравнения баланса механической энергии}
\en{for the linear model}\ru{для линейной модели}
\en{of the small displacements theory}\ru{теории м\'{а}лых смещений}

\nopagebreak
\hspace*{-\parindent}\begin{minipage}{\linewidth}
\begin{equation}\label{dynamicsoflineartheory:integralofenergy}
\displaystyle\integral\displaylimits_{\mathcal{V}}
\hspace{-0.4ex} \Bigl( \hspace{-0.1ex} \kinetic + \potential \Bigr)^{\hspace{-0.33ex}\tikz[baseline=-0.2ex] \draw[black, fill=black] (0,0) circle (.28ex);} d\mathcal{V}
= \hspace{-0.2ex}
\displaystyle\integral\displaylimits_{\mathcal{V}} \hspace{-0.5ex} \bm{f} \hspace{-0.1ex} \dotp \mathdotabove{\bm{u}} \hspace{.3ex} d\mathcal{V}
+ \hspace{-0.2ex}
\displaystyle\integral\displaylimits_{o_2} \hspace{-0.4ex} \bm{p} \dotp \mathdotabove{\bm{u}} \hspace{.3ex} do
\hspace{.15ex} ,
\end{equation}
%
\nopagebreak\vspace{-0.25em}\begin{gather*}
\bm{u} \hspace{.1ex} \bigr|_{o_1} \hspace{-0.64ex} = \hspace{.2ex} \bm{0}
\hspace{.1ex} ,
\hspace{.6em}
%
\unitnormalvector \dotp \linearstress \hspace{.2ex} \bigr|_{o_2} \hspace{-0.64ex} = \hspace{.2ex} \bm{p}
\hspace{.15ex} ,
%
\\[.5em]
%
\bm{u} \hspace{.1ex} \bigr|_{t=0} \hspace{-0.2ex} = \bm{u}^{\hspace{-0.1ex}\circ}
\hspace{-0.4ex} ,
\hspace{.6em}
%
\mathdotabove{\bm{u}} \hspace{.1ex} \bigr|_{t=0} \hspace{-0.2ex} = \mathdotabove{\bm{u}}^{\circ}
\hspace{-0.4ex} .
\end{gather*}
\end{minipage}

\vspace{.3em}
\en{For}\ru{Для}
\en{the \hbox{left\textcolor{gray}{-hand}} side}\ru{левой части}
\en{we have}\ru{мы имеем}

\nopagebreak\begin{equation*}
\begin{array}{c}
\mathdotabove{\kinetic} = \smalldisplaystyleonehalf \bigl( \hspace{.1ex} \rho \hspace{.2ex} \mathdotabove{\bm{u}} \dotp \mathdotabove{\bm{u}} \hspace{.1ex} \bigr)^{\hspace{-0.2ex}\tikz[baseline=-0.2ex] \draw[black, fill=black] (0,0) circle (.28ex);} \hspace{-0.2ex}
= \smalldisplaystyleonehalf \hspace{.2ex} \rho \hspace{.2ex} \bigl( \mathdotabove{\bm{u}} \dotp \mathdotdotabove{\bm{u}} + \mathdotdotabove{\bm{u}} \dotp \mathdotabove{\bm{u}} \hspace{.1ex} \bigr)
\hspace{-0.22ex} =
\rho \hspace{.2ex} \mathdotdotabove{\bm{u}} \dotp \mathdotabove{\bm{u}}
\hspace{.2ex} ,
\\[.5em]
%
\mathdotabove{\potential} \hspace{.1ex} = \hspace{.1ex} \smash{\smalldisplaystyleonehalf} \hspace{-0.25ex} \tikzmark{beginDerivativeOfPotentialEnergyDensity} \left( \hspace{.1ex} \infinitesimaldeformation \hspace{-0.1ex} \dotdotp \hspace{-0.1ex} \stiffnesstensor \dotdotp \hspace{-0.1ex} \infinitesimaldeformation \hspace{.1ex} \right)^{\tikz[baseline=-0.2ex] \draw[black, fill=black] (0,0) circle (.28ex);} \hspace{-1.2ex} \tikzmark{endDerivativeOfPotentialEnergyDensity} \hspace{.7ex}
= \linearstress \dotdotp \mathdotabove{\infinitesimaldeformation}
= \linearstress \hspace{.1ex} \dotdotp \hspace{-0.12ex} \boldnabla \mathdotabove{\bm{u}}^{\hspace{.1ex}\mathsf{S}} \hspace{-0.12ex}
= \boldnabla \hspace{-0.1ex} \dotp \left( \hspace{.1ex} \linearstress \dotp \mathdotabove{\bm{u}} \hspace{.15ex} \right) -
\tikzmark{NablaDotTauDynamicsBegin} \boldnabla \dotp \linearstress \tikzmark{NablaDotTauDynamicsEnd} \dotp \mathdotabove{\bm{u}} =
\\[.4em]
%
\hspace*{4em}
= \boldnabla \hspace{-0.1ex} \dotp \left( \hspace{.1ex} \linearstress \dotp \mathdotabove{\bm{u}} \hspace{.15ex} \right)
+ ( \hspace{.12ex} \bm{f} \hspace{-0.2ex} - \hspace{-0.2ex} \rho \hspace{.1ex} \mathdotdotabove{\bm{u}} \hspace{.15ex} ) \hspace{-0.1ex} \dotp \mathdotabove{\bm{u}}
%%\hspace{.2ex} .
\end{array}
\end{equation*}
\AddUnderBrace[line width=.75pt][.5ex, -0.4ex][yshift = .2ex]%
{beginDerivativeOfPotentialEnergyDensity}{endDerivativeOfPotentialEnergyDensity}%
{${\scalebox{0.75}{$ 2 \hspace{.2ex} \infinitesimaldeformation \hspace{-0.1ex} \dotdotp \hspace{-0.1ex} \stiffnesstensor \dotdotp \hspace{-0.1ex} \mathdotabove{\infinitesimaldeformation} $}}$}
\AddUnderBrace[line width=.75pt][.16ex, .1ex][yshift = .2ex]%
{NablaDotTauDynamicsBegin}{NablaDotTauDynamicsEnd}%
{ \scalebox{.75}{$ - \hspace{.2ex} ( \hspace{.12ex} \bm{f} \hspace{-0.2ex} - \hspace{-0.2ex} \rho \hspace{.1ex} \mathdotdotabove{\bm{u}} \hspace{.15ex} ) $} }

\nopagebreak\vspace{-0.6em}\noindent
(\ru{использован }\en{the balance of~momentum}\ru{баланс импульса}
${\hspace{-0.24ex}\boldnabla \dotp \linearstress \hspace{.15ex} + \bm{f} - \rho \hspace{.1ex} \mathdotdotabove{\bm{u}} \hspace{.1ex} = \bm{0}}$\en{ is used}),

\nopagebreak\vspace{-0.2em}
\begin{equation*}
\mathdotabove{\kinetic} + \mathdotabove{\potential} \hspace{.1ex}
= \boldnabla \hspace{-0.1ex} \dotp \left( \hspace{.1ex} \linearstress \dotp \mathdotabove{\bm{u}} \hspace{.15ex} \right)
+ \bm{f} \hspace{-0.1ex} \dotp \mathdotabove{\bm{u}}
\hspace{.2ex} .
\end{equation*}

\vspace{-0.2em}\noindent
\en{Applying the~divergence theorem}\ru{Применяя теорему о~дивергенции}

\nopagebreak\vspace{-0.2em}\begin{equation*}
\scalebox{.9}{$ \displaystyle \integral\displaylimits_{\mathcal{V}} $} \hspace{-0.2ex} \boldnabla \hspace{-0.1ex} \dotp \left( \hspace{.1ex} \linearstress \dotp \mathdotabove{\bm{u}} \hspace{.15ex} \right) \hspace{-0.15ex} d\mathcal{V}
=
\scalebox{.9}{$ \displaystyle \ointegral\displaylimits_{\mathclap{o\hspace{.1ex}(\boundary \mathcal{V})}} $} \hspace{.1ex} \unitnormalvector \hspace{.1ex} \dotp \linearstress \dotp \mathdotabove{\bm{u}} \hspace{.4ex} do
\end{equation*}

\vspace{-0.33em}\noindent
\en{and}\ru{и}~\en{the boundary condition}\ru{краевое условие}
${\unitnormalvector \dotp \linearstress = \bm{p}}$
\en{on}\ru{на}~$o_2$,
\en{we get}\ru{получаем}~\eqref{dynamicsoflineartheory:integralofenergy}.

\en{From}\ru{Из}~\eqref{dynamicsoflineartheory:integralofenergy} \en{it follows that}\ru{следует, что} \en{without loads}\ru{без нагрузок} (\en{when}\ru{когда} \en{there’re no external forces}\ru{нет внешних сил}, \en{neither volume nor surface}\ru{ни объёмных, ни поверхностных}), \en{and}\ru{и} \en{the~full mechanical energy}\ru{полная механическая энергия} \en{doesn’t change}\ru{не изменяется}:

\nopagebreak\vspace{-0.2em}\begin{equation}\label{dynamicsoflineartheory:fullenergyisconstantwithoutloads}
\bm{f} \hspace{-0.1ex} = \bm{0}
\hspace{1.1ex} \text{\en{and}\ru{и}} \hspace{1.2ex}
\bm{p} = \bm{0}
%
\hspace{.8ex} \Rightarrow \hspace{.55ex}
%
\scalebox{.8}{$\displaystyle \integral\displaylimits_{\mathcal{V}}$} \hspace{-0.2ex} \bigl( \hspace{.1ex} \kinetic + \potential \hspace{.1ex} \bigr) \hspace{.1ex} d\mathcal{V} = \hspace{.1ex} \constant(t)
\hspace{.2ex} .
\vspace{-0.25em}
\end{equation}

\vspace{-0.1em}\noindent
\en{If}\ru{Если} \en{at the~moment}\ru{в~момент}~${t \narroweq 0}$ \en{there was}\ru{был} \en{unstressed}\ru{ненапряжённый}\:(${\potential \narroweq \hspace{.1ex} 0}$) \en{rest}\ru{покой}\:(${\kinetic \narroweq \hspace{.1ex} 0}$), \en{then}\ru{то}

\nopagebreak\vspace{-0.25em}\begin{equation*}
\scalebox{.8}{$\displaystyle \integral\displaylimits_{\mathcal{V}}$} \hspace{-0.2ex} \bigl( \hspace{.1ex} \kinetic + \potential \hspace{.1ex} \bigr) \hspace{.1ex} d\mathcal{V} = \hspace{.1ex} 0
\hspace{.1ex} .
\tag{\theequation \raisebox{.1em}{\textquotesingle}}\label{dynamicsoflineartheory:fullenergyiszero}
\vspace{-0.25em}
\end{equation*}

\en{The kinetic energy}\ru{Кинетическая энергия} \en{is positive}\ru{положительна}:
${\kinetic \hspace{-0.2ex} > \hspace{-0.2ex} 0}$ \en{if}\ru{если}~${\mathdotabove{\bm{u}} \hspace{-0.1ex} \neq \hspace{-0.1ex} \bm{0}}$
\en{and}\ru{и}~\en{vanishes}\ru{исчезает}
(\en{nullifies}\ru{обнуляется})
\en{only when}\ru{лишь когда}~${\mathdotabove{\bm{u}} = \hspace{-0.1ex} \bm{0}}$\:---
\en{this ensues}\ru{это вытекает} \en{from its definition}\ru{из её определения}
${\kinetic \hspace{-0.1ex} \equiv \smallerdisplaystyleonehalf \hspace{.2ex} \rho \hspace{.2ex} \mathdotabove{\bm{u}} \dotp \mathdotabove{\bm{u}}}$.
\en{The potential energy}\ru{Потенциальная энергия},
\en{being}\ru{будучи} \en{a~quadratic form}\ru{квадратичной формой} %%\en{of~}\en{an~infinitesimal linear deformation}\ru{бесконечномалой линейной деформации}
${\potential(\hspace{-0.1ex}\infinitesimaldeformation\hspace{-0.1ex}) \hspace{-0.2ex} = \hspace{.1ex} \smash{\smallerdisplaystyleonehalf} \hspace{.2ex} \infinitesimaldeformation \hspace{-0.1ex} \dotdotp \hspace{-0.1ex} \stiffnesstensor \dotdotp \hspace{-0.1ex} \infinitesimaldeformation}$,
\en{is positive too}\ru{тоже положительна}:
${\potential \hspace{-0.16ex} > \hspace{-0.2ex} 0}$ \en{if}\ru{если}~${\infinitesimaldeformation \hspace{-0.1ex} \neq \hspace{-0.1ex} {^2\bm{0}}}$.
%
\en{Such is}\ru{Таков\'{о}} \en{a~priori requirement}\ru{априорное требование} \en{of~the~positive definiteness}\ru{положительной определённости} \en{for}\ru{для}~\en{stiffness tensor}\ru{тензора жёсткости}~$\stiffnesstensor$.
\en{This is one of}\ru{Это одно из}~\inquotes{\en{additional inequalities in the~theory of~elasticity}\ru{дополнительных неравенств в~теории упругости}}~\cite{lurie-nonlinearelasticity, truesdell-firstcourse}.

\en{Since}\ru{Так как}
$\kinetic$ \en{and}\ru{и}~$\potential$
\en{are positive definite}\ru{положительно определены},
\eqref{dynamicsoflineartheory:fullenergyiszero}
\en{gives}\ru{даёт}

\nopagebreak\vspace{-0.1em}
\hspace*{-\parindent}\begin{minipage}{\linewidth}
\begin{equation*}
\kinetic = 0 \hspace{.1ex}, \; \potential = 0
\hspace{.7ex} \Rightarrow \hspace{.7ex}
%
\mathdotabove{\bm{u}} = \bm{0}
\hspace{.1ex} , \;
\infinitesimaldeformation = \hspace{-0.25ex} \boldnabla {\bm{u}}^{\hspace{.1ex}\mathsf{S}} \hspace{-0.25ex} = \hspace{-0.1ex} {^2\bm{0}}
\hspace{.8ex} \Rightarrow \hspace{.7ex}
%
\bm{u} = \bm{u}^{\hspace{-0.1ex}\circ} \hspace{-0.3ex} + \hspace{.1ex} \bm{\omega}^{\circ} \hspace{-0.5ex} \times \locationvector
\end{equation*}

\nopagebreak\vspace{-0.1em}\noindent
(${\smash{\bm{u}^{\hspace{-0.1ex}\circ}} \hspace{-0.3ex}}$ \en{and}\ru{и}~${\smash{\bm{\omega}^{\circ}} \hspace{-0.4ex}}$\en{ are}\ru{\:---} \en{some constants}\ru{некоторые константы} \en{of~}\en{translation}\ru{трансляции} \en{and}\ru{и}~\en{rotation}\ru{поворота}).
\en{With an~immobile part}\ru{С~неподвижной частью} \en{of~the~surface}\ru{поверхности}

\nopagebreak\vspace{-0.2em}
\begin{equation*}
\bm{u} \hspace{.1ex} \rvert_{\raisemath{-0.1em}{o_1}} \hspace{-0.7ex} = \bm{0}
\hspace{.8ex} \Rightarrow \hspace{.7ex}
%
\smash{\bm{u}^{\hspace{-0.1ex}\circ}} \hspace{-0.4ex} = \bm{0}
\hspace{1.1ex} \text{\en{and}\ru{и}} \hspace{1.1ex}
\smash{\bm{\omega}^{\circ}} \hspace{-0.4ex} = \bm{0}
\hspace{.8ex} \Rightarrow \hspace{.7ex}
%
\bm{u} = \hspace{-0.1ex} \bm{0} \hspace{1ex} \text{\en{everywhere}\ru{всюду}} .
\end{equation*}
\end{minipage}

\en{Now}\ru{Теперь}
\en{remember two solutions}\ru{вспомним о~двух решениях}
${ \bm{u}_1\hspace{-0.25ex} }$
\en{and}\ru{и}~${ \bm{u}_2 }$.
\en{Their}\ru{Их}
\en{difference}\ru{разность}
${\smash{
    \bm{u}^{\hspace{-0.1ex}*}
}
\hspace{-0.25ex} \equiv \hspace{.1ex}
\bm{u}_1
\hspace{-0.36ex} -
\bm{u}_2
}$
\en{is}\ru{есть}
\en{a~solution}\ru{решение}
\en{of~an~entirely}\ru{полностью} \inquotes{\en{homogeneous}\ru{однородной}}
(\en{with no constant terms at~all}\ru{совсем без постоянных членов})
\en{linear}\ru{линейной}
\en{problem}\ru{задачи}:
\en{in a~volume}\ru{в~объёме}
${\bm{f} \hspace{-0.1ex} = \bm{0}}$,
\en{in boundary}\ru{в~краевых}
\en{and }\ru{и~}\en{in initial}\ru{в~начальных} \en{conditions}\ru{условиях}\:---
\en{zeroes}\ru{нули}.
\en{Therefore}\ru{Поэтому}
${ \smash{\bm{u}^{\hspace{-0.1ex}*}}
\hspace{-0.32ex} =
\bm{0} }$,
\en{and}\ru{и}
\en{the uniqueness is proven}\ru{единственность доказана}.

\en{As for}\ru{Что~же для}
\en{the existence}\ru{существования}
\en{of a~solution}\ru{решения}\:---
\en{it cannot be proven}\ru{его не~обосновать}
\en{for the generic case}\ru{для общего случая}
\en{by simple}\ru{простыми}
\en{conclusions}\ru{выводами}.
\en{I could only tell that}\ru{Я могу лишь сказать, что}
\en{a~dynamic problem}\ru{динамическая проблема}
\en{is}\ru{является}
\en{evolutional}\ru{эволюционной},
\en{it describes}\ru{она описывает}
\en{the progress}\ru{прогресс}
\en{of a~process}\ru{процесса}
\en{in time}\ru{во~времени}.

\en{The balance}\ru{Баланс}
(\en{the conservation}\ru{сохранение})
\en{of~momentum}\ru{импульса}
\en{gives}\ru{даёт}
\en{the acceleration}\ru{ускорение}~$\mathdotdotabove{\bm{u}}$.
\en{Then}\ru{Далее},
\en{moving}\ru{переходя}
\ru{на}\en{to the} \inquotes{\en{next time layer}\ru{следующий временн\'{о}й слой}}
${ t \hspace{-0.1ex} + \hspace{-0.15ex} dt }$:

\nopagebreak
\begin{gather*}
\mathdotabove{\bm{u}} ({\bm{r}, \hspace{.1ex} t \hspace{-0.3ex} + \hspace{-0.3ex} dt})
\hspace{-0.2ex} =
\mathdotabove{\bm{u}}(\bm{r}, t) + \mathdotdotabove{\bm{u}} dt \hspace{.1ex} , \\
\bm{u}({ \bm{r}, \hspace{.1ex} t \hspace{-0.3ex} + \hspace{-0.3ex} dt })
\hspace{-0.2ex} =
\bm{u}(\bm{r}, t)
+ \mathdotabove{\bm{u}} dt
\hspace{.1ex} ,
\\[-0.1em]
%
\infinitesimaldeformation({\bm{r}, \hspace{.1ex} t \hspace{-0.3ex} + \hspace{-0.3ex} dt})
\hspace{-0.2ex} = \hspace{-0.2ex}
\bigl( \boldnabla \bm{u}{( \bm{r}, \hspace{.1ex} t \hspace{-0.3ex} + \hspace{-0.3ex} dt )} \bigr)^{\hspace{-0.25ex}\mathsf{S}}
\hspace{.2em} \Rightarrow \hspace{.25em}
\linearstress
\hspace{.1ex} ,
\\[-0.1em]
%
\boldnabla \dotp \linearstress \hspace{.15ex} + \bm{f}
= \rho \hspace{.2ex} \mathdotdotabove{\bm{u}}( \bm{r}, \hspace{.1ex} t \hspace{-0.3ex} + \hspace{-0.3ex} dt )
\end{gather*}

\vspace{-0.1em}\noindent
\en{and so forth}\ru{и так далее}.
\en{Surely}\ru{Разумеется},
\en{these considerations}\ru{эти соображения}
\en{lack}\ru{лишены}
\en{the mathematical scrupulosity}\ru{математической щепетильности}.
\en{The latter}\ru{Последнее}
\en{can be}\ru{может быть}
\en{found}\ru{найдено},
\en{for example}\ru{для примера},
\en{in}\ru{в}~\en{the}\ru{монографии} Philippe\ru{’а} Ciarlet\en{’s}\en{ monograph}~\cite{ciarlet-mathematicalelasticity}.

