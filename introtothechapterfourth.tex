\begin{changemargin}{\parindent}{\parindent}
\vspace{-2.5em}
{\noindent\small
\setlength{\parskip}{\spacebetweenparagraphs}

%%\begin{flushright}
\noindent
\en{This chapter is about}\ru{Эта глава\:--- про}
\en{the geometrically linear model}\ru{геометрически линейную модель}
\en{with}\ru{с}~\en{infinitesimal}\ru{бесконечно м\'{а}лыми}
\en{displacements}\ru{смещениями},
\en{where}\ru{где}
%%\end{flushright}
%
\begin{itemize}
\item
${
   \mathcal{V}
   \hspace{-0.2ex} = \hspace{-0.2ex}
   \smash{\mathcircabove{ \mathcal{V} }}\hspace{-0.1ex}
}$,
${
   \hspace{.2ex} \rho
   \hspace{-0.12ex} = \hspace{-0.22ex}
   \smash{\mathcircabove{ \rho }} \hspace{.2ex}
}$\:---
\inquotes{\en{the~equations}\ru{уравнения}
\en{can~be written in the initial configuration}\ru{можно пис\'{а}ть в~начальной конфигурации}}
(\en{sometimes}\ru{временами}
\en{it is called}\ru{это называют}
\inquotes{\en{the~principle}\ru{принципом}
\en{of initial dimensions}\ru{начальных размеров}}),
%
\vspace{.2em}
\item
\en{operators}\ru{операторы} ${
   \hspace{-0.2ex} \smash{\boldnablacircled} \hspace{.1ex}
}$ \en{and}\ru{и}
${
   \hspace{-0.2ex} \smash{\boldnabla} \hspace{.1ex}
}$
\en{are indistinguishable}\ru{неразличимы},
%
\vspace{.2em}
\item
\en{operators}\ru{операторы}~$\variation$
\en{and}\ru{и}~${\hspace{-0.2ex}\boldnabla}$
\en{commute}\ru{коммутируют},
\en{thus for example}\ru{потому для примера}
${
   \variation{\boldnabla \fieldofdisplacements} \hspace{-0.1ex}
   = \hspace{-0.3ex} \boldnabla \variation{\fieldofdisplacements}
}$.
\end{itemize}
\par}
\vspace{-1.2em}
\end{changemargin}

