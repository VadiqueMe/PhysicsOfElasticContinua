\documentclass[14pt]{extarticle}

\usepackage[utf8]{inputenc}

\usepackage{geometry}
\geometry{papersize={21cm, 29.7cm}} % A4
\geometry{tmargin=2cm, bmargin=2cm, outer=2cm, inner=2cm, head=2cm, foot=2cm}

\usepackage{graphicx}

\renewcommand\thepage{\oldstylenums{\arabic{page}}}

\usepackage{fancyhdr}
\fancyhf{}
\renewcommand{\headrulewidth}{0pt}
\renewcommand{\footrulewidth}{0pt}
\fancyhead[CE,CO]{${(\:}$\raisebox{.1em}{\thepage}${\:)}$}
\pagestyle{fancy}
\geometry{headsep=.5cm}

\usepackage{tikz}
\usepackage{tikz-3dplot}
\usetikzlibrary{calc}
\usetikzlibrary{arrows, arrows.meta}
\usetikzlibrary{decorations.markings}

\usepackage[T2A, T1]{fontenc}
\usepackage[english, russian]{babel}

\def\horizontalindent{4ex}
\setlength{\parindent}{\horizontalindent} % offset of the first line
\usepackage{indentfirst}

\usepackage{setspace}
\setstretch{1.3} % spacing between lines

\usepackage{microtype}

\usepackage{tempora}

\usepackage{xargs}
\usepackage{xifthen}

\begin{document}

\def\beamlinewidth{2.8pt}
\tikzstyle{beam line} = [ line width=\beamlinewidth, line cap=round, color=black ]

\def\auxlinewidth{.6pt}
\tikzstyle{aux line thin} = [ line width=\auxlinewidth, color=black, line cap=round ]
\tikzstyle{aux thin dashed} = [ aux line thin, dash pattern=on 0pt off 1.6\pgflinewidth ]
\tikzstyle{aux line thick} = [ line width=2.5*\auxlinewidth, color=black, line cap=round ]
\tikzstyle{aux thick dashed} = [ aux line thick, dash pattern=on 0pt off 1.6\pgflinewidth ]

\def\externalforcecolor{orange!33!red}
\tikzstyle{external force} =
	[ line width=(2/3)*\beamlinewidth, color=\externalforcecolor, line cap=round, -{Triangle[round, length=12pt, width=8pt]} ]

\def\textscale{1.2}

\def\reflength{5} % reference length

\pgfmathsetmacro\cornerlength{\reflength / 20}

\pgfmathsetmacro\onePl{\reflength / 3}
\pgfmathsetmacro\justP{\onePl / \reflength}
\def\forcearrowscale{4}

\pgfmathsetmacro\xfirst{2*\reflength}
\pgfmathsetmacro\xfirsttosecond{\reflength}
\pgfmathsetmacro\xsecond{\xfirst + abs(\xfirsttosecond)}
\pgfmathsetmacro\zfirst{-.9*\reflength}
\pgfmathsetmacro\zsecond{-.8*\reflength}

\newcommand\drawpartofbeamalongx{
	\draw [ beam line, tdplot_main_coords ]
		( 0, 0, 0 ) -- ++( \xsecond, 0, 0 ) ;
}

\newcommand\drawfirstverticalpartofbeam{
	\draw [ beam line, tdplot_main_coords ]
		($ ( 0, 0, 0 ) + ( \xfirst, 0, 0 ) $) -- ++( 0, 0, \zfirst ) ;

	% corners

	\pgfmathsetmacro\cornerzsign{\zfirst / abs(\zfirst)}

	\draw [ beam line, fill=black, tdplot_main_coords ]
		($ ( 0, 0, 0 ) + ( \xfirst, 0, 0 ) $)
		-- ++( -\cornerlength, 0, 0 )
		-- ++( \cornerlength, 0, {\cornerzsign*\cornerlength} )
		-- cycle ;

	\draw [ beam line, fill=black, tdplot_main_coords ]
		($ ( 0, 0, 0 ) + ( \xfirst, 0, 0 ) $)
		-- ++( \cornerlength, 0, 0 )
		-- ++( -\cornerlength, 0, {\cornerzsign*\cornerlength} )
		-- cycle ;
}

\newcommand\drawsecondverticalpartofbeam{
	\draw [ beam line, tdplot_main_coords ]
		($ ( 0, 0, 0 ) + ( \xsecond, 0, 0 ) $) -- ++( 0, 0, \zsecond ) ;

	% corners

	\pgfmathsetmacro\cornerzsign{\zsecond / abs(\zsecond)}

	\draw [ beam line, fill=black, tdplot_main_coords ]
		($ ( 0, 0, 0 ) + ( \xsecond, 0, 0 ) $)
		-- ++( -\cornerlength, 0, 0 )
		-- ++( \cornerlength, 0, {\cornerzsign*\cornerlength} )
		-- cycle ;
}

\newcommand\drawbeam{
	\drawpartofbeamalongx
	\drawfirstverticalpartofbeam
	\drawsecondverticalpartofbeam
}

\newcommand\drawtextforpartofbeamalongx{
	\pgfmathsetmacro\firstpartlengthmultiplier{\xfirst / \reflength}

	\node [ above, shape=circle, inner sep=0pt, outer sep=8pt, tdplot_main_coords ]
		at ($ ( 0, 0, 0 ) + ( .53*\xfirst, 0, 0 ) $)
	{\scalebox{\textscale}{$
\pgfmathparse{(\firstpartlengthmultiplier == 1) ? 0 : 1}\ifdim\pgfmathresult pt>0pt%
	\pgfmathprintnumber[ precision=3, fixed, zerofill=false ]\firstpartlengthmultiplier\hspace{.1ex}
\fi
\ell
	$}} ;

	\pgfmathsetmacro\secondpartlengthmultiplier{\xfirsttosecond / \reflength}

	\node [ above, shape=circle, inner sep=0pt, outer sep=8pt, tdplot_main_coords ]
		at ($ ( \xfirst, 0, 0 ) + ( .51*\xfirsttosecond, 0, 0 ) $)
	{\scalebox{\textscale}{$
\pgfmathparse{(\secondpartlengthmultiplier == 1) ? 0 : 1}\ifdim\pgfmathresult pt>0pt%
	\pgfmathprintnumber[ precision=3, fixed, zerofill=false ]\secondpartlengthmultiplier\hspace{.1ex}
\fi
\ell
	$}} ;
}

\newcommand\drawtextforfirstverticalpartofbeam{
	\pgfmathsetmacro\firstverticalpartlengthmultiplier{abs(\zfirst / \reflength)}

	\node [ right, inner sep=0pt, outer sep=7pt, tdplot_main_coords ]
		at ($ ( 0, 0, 0 ) + ( \xfirst, 0, .5*\zfirst ) $)
	{\scalebox{\textscale}{$
\pgfmathparse{(\firstverticalpartlengthmultiplier == 1) ? 0 : 1}\ifdim\pgfmathresult pt>0pt%
	\pgfmathprintnumber[ precision=3, fixed, zerofill=false ]\firstverticalpartlengthmultiplier\hspace{.1ex}
\fi
\ell
	$}} ;
}

\newcommand\drawtextforsecondverticalpartofbeam{
	\pgfmathsetmacro\secondverticalpartlengthmultiplier{abs(\zsecond / \reflength)}

	\node [ right, inner sep=0pt, outer sep=7pt, tdplot_main_coords ]
		at ($ ( \xfirst, 0, 0 ) + ( \xfirsttosecond, 0, .5*\zsecond ) $)
	{\scalebox{\textscale}{$
\pgfmathparse{(\secondverticalpartlengthmultiplier == 1) ? 0 : 1}\ifdim\pgfmathresult pt>0pt%
	\pgfmathprintnumber[ precision=3, fixed, zerofill=false ]\secondverticalpartlengthmultiplier\hspace{.1ex}
\fi
\ell
	$}} ;
}

\newcommand\drawbeamtext{
	\drawtextforpartofbeamalongx
	\drawtextforfirstverticalpartofbeam
	\drawtextforsecondverticalpartofbeam
}

\def\clampsize{.8}
\def\clampcolor{gray}

\tikzstyle{clamp line} = [ line width=(2/3)*\beamlinewidth, line cap=round, color=\clampcolor ]
\tikzstyle{clamp line thin} = [ line width=2*\auxlinewidth, line cap=round, color=\clampcolor ]

\newcommand\drawclampyz[1]{
	\draw [ clamp line, tdplot_main_coords ] ($ #1 + ( 0, .6*\clampsize, .5*\clampsize ) $) -- ++( - .6*\clampsize, 0, 0 ) ;
	\draw [ clamp line, tdplot_main_coords ] ($ #1 + ( 0, -.6*\clampsize, .5*\clampsize ) $) -- ++( - .7*\clampsize, 0, 0 ) ;
	\draw [ clamp line, tdplot_main_coords ] ($ #1 + ( 0, .6*\clampsize, -.5*\clampsize ) $) -- ++( - .7*\clampsize, 0, 0 ) ;
	\draw [ clamp line, tdplot_main_coords ] ($ #1 + ( 0, -.6*\clampsize, -.5*\clampsize ) $) -- ++( - .7*\clampsize, 0, 0 ) ;

	\draw [ clamp line, rounded corners=.5\pgflinewidth, fill=white, tdplot_main_coords ]
		($ #1 + ( 0, .6*\clampsize, .5*\clampsize ) $)
		-- ($ #1 + ( 0, -.6*\clampsize, .5*\clampsize ) $)
		-- ($ #1 + ( 0, -.6*\clampsize, -.5*\clampsize ) $)
		-- ($ #1 + ( 0, .6*\clampsize, -.5*\clampsize ) $)
		-- cycle ;

	\draw [ clamp line thin, color=\clampcolor, tdplot_main_coords ]
		plot [ smooth, tension=.8 ] coordinates {
		($ #1 + ( - .6*\clampsize, .6*\clampsize, .5*\clampsize ) $)
		($ #1 + ( -.5*\clampsize, .3*\clampsize, .5*\clampsize ) $)
		($ #1 + ( - .75*\clampsize, -.3*\clampsize, .5*\clampsize ) $)
		($ #1 + ( - .7*\clampsize, -.6*\clampsize, .5*\clampsize ) $)
	} ;
	\draw [ clamp line thin, color=\clampcolor, tdplot_main_coords ]
		plot [ smooth, tension=.5 ] coordinates {
		($ #1 + ( - .6*\clampsize, .6*\clampsize, .5*\clampsize ) $)
		($ #1 + ( -.66*\clampsize, .6*\clampsize, .2*\clampsize ) $)
		($ #1 + ( - .55*\clampsize, .6*\clampsize, -.2*\clampsize ) $)
		($ #1 + ( - .7*\clampsize, .6*\clampsize, -.5*\clampsize ) $)
	} ;
}

% #1: point
% #2: force vector
% #3: options for node
% #4: node text
\newcommand\drawload[4]{
	\draw [ external force, tdplot_main_coords ]
		($ #1 - #2 $) -- ++#2
		node [ #3 ] { #4 } ;
}

% takes two points as cartesian {x}{y}{z} and calculates cross product of their location vectors
% placing the result into last three arguments
\newcommand{\tdcrossproductcartesian}[9]{
\def\crossz{ (#1) * (#5) - (#2) * (#4) }
\def\crossx{ (#2) * (#6) - (#3) * (#5) }
\def\crossy{ (#3) * (#4) - (#1) * (#6) }
\pgfmathsetmacro{#7}{\crossx}
\pgfmathsetmacro{#8}{\crossy}
\pgfmathsetmacro{#9}{\crossz}
}

% {rx}{ry}{rz} of a point in question
% {rfx}{rfy}{rfz} of force point
% {fx}{fy}{fz} of force vector
% places result in \momentx, \momenty, \momentz
\newcommand\calculateinternalmomentatpointfromforce[9]{
	% get vector d from current point to the force
	\pgfmathsetmacro\distancex{(#4) - (#1)}
	\pgfmathsetmacro\distancey{(#5) - (#2)}
	\pgfmathsetmacro\distancez{(#6) - (#3)}
	% balance of moments: M + d ✕ F = 0
	% internal moment M = - d ✕ F = F ✕ d
	\tdcrossproductcartesian%
		{#7}{#8}{#9}%
		{\distancex}{\distancey}{\distancez}%
		{\momentx}{\momenty}{\momentz}
}

\def\twistingmomentepureoffset{\reflength / 3}
\def\twistingmomentmultiplierx{0}
\def\twistingmomentmultipliery{0}
\def\twistingmomentmultiplierz{1}

\newcommand\drawtwistingoffsetlineat[3]{
	\draw [ aux thick dashed, color=\externalforcecolor, tdplot_main_coords ]
		( #1, #2, #3 ) -- ++( {\twistingmomentmultiplierx * \twistingmomentepureoffset}, {\twistingmomentmultipliery * \twistingmomentepureoffset}, {\twistingmomentmultiplierz * \twistingmomentepureoffset} ) ;
}

%%\newcommand\drawtwistingoffsetbasebetween[6]{
%%	\draw [ aux line thick, color=\externalforcecolor, tdplot_main_coords ]
%%		($ ( #1, #2, #3 ) + ( {\twistingmomentmultiplierx * \twistingmomentepureoffset}, {\twistingmomentmultipliery * \twistingmomentepureoffset}, {\twistingmomentmultiplierz * \twistingmomentepureoffset} ) $)
%%		-- ($ ( #4, #5, #6 ) + ( {\twistingmomentmultiplierx * \twistingmomentepureoffset}, {\twistingmomentmultipliery * \twistingmomentepureoffset}, {\twistingmomentmultiplierz * \twistingmomentepureoffset} ) $) ;
%%}

\tikzset {
	set arrow inside/.code={\pgfqkeys{/tikz/arrow inside}{#1}},
	set arrow inside={end/.initial=>, opt/.initial=},
	/pgf/decoration/Mark/.style={
		mark/.expanded=at position #1 with {
			\noexpand\arrow[\pgfkeysvalueof{/tikz/arrow inside/opt}]{\pgfkeysvalueof{/tikz/arrow inside/end}}
		}
	},
	arrow inside/.style 2 args={
		set arrow inside={#1},
		postaction={
			decorate,decoration={
				markings,Mark/.list={#2}
			}
		}
	},
}

\def\spiralinitialangle{-10}
\def\spiralaxialstep{1}
\def\wherefirstarrowonspiral{0.075}

% #1 = begin z, #2 = end z
% #3, #4 = x, y
% needs pre-calculated \momentz
\newcommand\drawtwistingspiralalongzbetween[4]{
	\pgfmathparse{(abs(\momentz) > 0) ? 1 : 0}\ifdim\pgfmathresult pt>0pt%
		\pgfmathsetmacro\momentzsign{\momentz / abs(\momentz)}
	\else
		\pgfmathsetmacro\momentzsign{0}
	\fi

	\pgfmathsetmacro\momentzmidx{#3 + \twistingmomentmultiplierx * ( \twistingmomentepureoffset + .5*abs(\momentz) )}
	\pgfmathsetmacro\momentzmidy{#4 + \twistingmomentmultipliery * ( \twistingmomentepureoffset + .5*abs(\momentz) )}
	\pgfmathsetmacro\momentzmidz{#1 + \twistingmomentmultiplierz * ( \twistingmomentepureoffset + .5*abs(\momentz) )}

	\pgfmathsetmacro\spiralradius{\momentz / 2}
	\pgfmathsetmacro\axialadvance{abs(\spiralaxialstep) * (#2 - #1) / abs(#2 - #1)}
	\pgfmathsetmacro\axialshift{-\spiralinitialangle*\axialadvance/360}
	\pgfmathsetmacro\lengthincircles{abs(#2 - #1) / abs(\spiralaxialstep)}
	\pgfmathsetmacro\endangle{\lengthincircles * 360 + \spiralinitialangle}

	\draw[ aux line thick, color=\externalforcecolor, tdplot_main_coords  ]
		plot [ domain=\spiralinitialangle:\endangle, variable=\t, samples=360 ]
		( {\momentzmidx + \momentzsign*\spiralradius*sin(\t)},
		{\momentzmidy + abs(\momentzsign)*\spiralradius*cos(\t)},
			{\momentzmidz + \axialadvance*\t/360 + \axialshift} )
		[ arrow inside = {}{\wherefirstarrowonspiral, 1} ] ;
}

% #1 = begin x, #2 = end x
% #3, #4 = y, z
% needs pre-calculated \momentx
\newcommand\drawtwistingspiralalongxbetween[4]{
	\pgfmathparse{(abs(\momentx) > 0) ? 1 : 0}\ifdim\pgfmathresult pt>0pt%
		\pgfmathsetmacro\momentxsign{\momentx / abs(\momentx)}
	\else
		\pgfmathsetmacro\momentxsign{0}
	\fi

	\pgfmathsetmacro\momentxmidx{#1 + \twistingmomentmultiplierx * ( \twistingmomentepureoffset + .5*abs(\momentx) )}
	\pgfmathsetmacro\momentxmidy{#3 + \twistingmomentmultipliery * ( \twistingmomentepureoffset + .5*abs(\momentx) )}
	\pgfmathsetmacro\momentxmidz{#4 + \twistingmomentmultiplierz * ( \twistingmomentepureoffset + .5*abs(\momentx) )}

	\pgfmathsetmacro\spiralradius{\momentx / 2}
	\pgfmathsetmacro\axialadvance{abs(\spiralaxialstep) * (#2 - #1) / abs(#2 - #1)}
	\pgfmathsetmacro\axialshift{-\spiralinitialangle*\axialadvance/360}
	\pgfmathsetmacro\lengthincircles{abs(#2 - #1) / abs(\spiralaxialstep)}
	\pgfmathsetmacro\endangle{\lengthincircles * 360 + \spiralinitialangle}

	\draw[ aux line thick, color=\externalforcecolor, tdplot_main_coords  ]
		plot [ domain=\spiralinitialangle:\endangle, variable=\t, samples=360 ]
		( {\momentxmidx + \axialadvance*\t/360 + \axialshift},
			{\momentxmidy + \momentxsign*\spiralradius*sin(\t)},
			{\momentxmidz + abs(\momentxsign)*\spiralradius*cos(\t)} )
		[ arrow inside = {}{\wherefirstarrowonspiral, 1} ] ;
}

% #1,#2,#3 are x,y,z of point
% anything in #4 for thick line
% needs pre-calculated \momentx, \momenty, \momentz
\newcommandx*\drawepurelinesatpointalongx[4][4=]{
	\ifthenelse{\isempty{#4}}%
		{\def\linestyle{aux line thin}}%
		{\def\linestyle{aux line thick}}

	\pgfmathsetmacro\momentzendx{#1 + 0}
	\pgfmathsetmacro\momentzendy{#2 - \momentz}
	\pgfmathsetmacro\momentzendz{#3 + 0}

	\pgfmathsetmacro\momentyendx{#1 + 0}
	\pgfmathsetmacro\momentyendy{#2 + 0}
	\pgfmathsetmacro\momentyendz{#3 - \momenty}

	\pgfmathsetmacro\momentxendx{#1 + \twistingmomentmultiplierx * ( \twistingmomentepureoffset + abs(\momentx) )}
	\pgfmathsetmacro\momentxendy{#2 + \twistingmomentmultipliery * ( \twistingmomentepureoffset + abs(\momentx) )}
	\pgfmathsetmacro\momentxendz{#3 + \twistingmomentmultiplierz * ( \twistingmomentepureoffset + abs(\momentx) )}

	\pgfmathsetmacro\momentxmidx{#1 + \twistingmomentmultiplierx * ( \twistingmomentepureoffset + .5*abs(\momentx) )}
	\pgfmathsetmacro\momentxmidy{#2 + \twistingmomentmultipliery * ( \twistingmomentepureoffset + .5*abs(\momentx) )}
	\pgfmathsetmacro\momentxmidz{#3 + \twistingmomentmultiplierz * ( \twistingmomentepureoffset + .5*abs(\momentx) )}

	\pgfmathparse{(abs(\momentx) > 0) ? 1 : 0}\ifdim\pgfmathresult pt>0pt%
		\pgfmathsetmacro\momentxsign{\momentx / abs(\momentx)}
	\else
		\pgfmathsetmacro\momentxsign{0}
	\fi

	% bending
	\draw [ \linestyle, color=\externalforcecolor, tdplot_main_coords ]
		( #1, #2, #3 ) -- ( \momentzendx, \momentzendy, \momentzendz ) ;
	\draw [ \linestyle, color=\externalforcecolor, tdplot_main_coords ]
		( #1, #2, #3 ) -- ( \momentyendx, \momentyendy, \momentyendz ) ;

	% twisting
	\pgfmathparse{(abs(\momentx) > 0) ? 1 : 0}\ifdim\pgfmathresult pt>0pt%
		\ifthenelse{\isempty{#4}}%
		{}{%
		\draw[ aux thick dashed, color=\externalforcecolor, tdplot_main_coords  ]
			plot [ domain=0:360, variable=\t, samples=60 ]
			( {\momentxmidx + 0}, {\momentxmidy + .5*\momentx*sin(\t)}, {\momentxmidz + .5*\momentx*cos(\t)} ) ;
		}
	\fi
}

% #1,#2,#3 are x,y,z of point
% anything in #4 for thick line
% needs pre-calculated \momentx, \momenty, \momentz
\newcommandx*\drawepurelinesatpointalongz[4][4=]{
	\ifthenelse{\isempty{#4}}%
		{\def\linestyle{aux line thin}}%
		{\def\linestyle{aux line thick}}

	\pgfmathsetmacro\momentxendx{#1 + 0}
	\pgfmathsetmacro\momentxendy{#2 - \momentx}
	\pgfmathsetmacro\momentxendz{#3 + 0}

	\pgfmathsetmacro\momentyendx{#1 - \momenty}
	\pgfmathsetmacro\momentyendy{#2 + 0}
	\pgfmathsetmacro\momentyendz{#3 + 0}

	\pgfmathsetmacro\momentzendx{#1 + \twistingmomentmultiplierx * ( \twistingmomentepureoffset + abs(\momentz) )}
	\pgfmathsetmacro\momentzendy{#2 + \twistingmomentmultipliery * ( \twistingmomentepureoffset + abs(\momentz) )}
	\pgfmathsetmacro\momentzendz{#3 + \twistingmomentmultiplierz * ( \twistingmomentepureoffset + abs(\momentz) )}

	\pgfmathsetmacro\momentzmidx{#1 + \twistingmomentmultiplierx * ( \twistingmomentepureoffset + .5*abs(\momentz) )}
	\pgfmathsetmacro\momentzmidy{#2 + \twistingmomentmultipliery * ( \twistingmomentepureoffset + .5*abs(\momentz) )}
	\pgfmathsetmacro\momentzmidz{#3 + \twistingmomentmultiplierz * ( \twistingmomentepureoffset + .5*abs(\momentz) )}

	% bending
	\draw [ \linestyle, color=\externalforcecolor, tdplot_main_coords ]
		( #1, #2, #3 ) -- ( \momentxendx, \momentxendy, \momentxendz ) ;
	\draw [ \linestyle, color=\externalforcecolor, tdplot_main_coords ]
		( #1, #2, #3 ) -- ( \momentyendx, \momentyendy, \momentyendz ) ;

	% twisting
	\pgfmathparse{(abs(\momentz) > 0) ? 1 : 0}\ifdim\pgfmathresult pt>0pt%
		\ifthenelse{\isempty{#4}}%
		{}{%
		\draw [ aux thick dashed, color=\externalforcecolor, tdplot_main_coords ]
			plot [ domain=0:360, variable=\t, samples=60 ]
			( {\momentzmidx + .5*\momentz*cos(\t)}, {\momentzmidy + .5*\momentz*sin(\t)}, {\momentzmidz + 0} ) ;
		}
	\fi
}

\newcommand\saveepureendpoints{
	\pgfmathsetmacro\savedmomentxendx{\momentxendx}
	\pgfmathsetmacro\savedmomentxendy{\momentxendy}
	\pgfmathsetmacro\savedmomentxendz{\momentxendz}

	\pgfmathsetmacro\savedmomentyendx{\momentyendx}
	\pgfmathsetmacro\savedmomentyendy{\momentyendy}
	\pgfmathsetmacro\savedmomentyendz{\momentyendz}

	\pgfmathsetmacro\savedmomentzendx{\momentzendx}
	\pgfmathsetmacro\savedmomentzendy{\momentzendy}
	\pgfmathsetmacro\savedmomentzendz{\momentzendz}
}

\newcommand\drawlinebetweensavedandcurrentz{
	\draw [ aux line thick, color=\externalforcecolor, tdplot_main_coords ]
		( \savedmomentzendx, \savedmomentzendy, \savedmomentzendz )
		-- ( \momentzendx, \momentzendy, \momentzendz ) ;
}

\newcommand\drawlinebetweensavedandcurrenty{
	\draw [ aux line thick, color=\externalforcecolor, tdplot_main_coords ]
		( \savedmomentyendx, \savedmomentyendy, \savedmomentyendz )
		-- ( \momentyendx, \momentyendy, \momentyendz ) ;
}

\newcommand\drawlinebetweensavedandcurrentx{
	\draw [ aux line thick, color=\externalforcecolor, tdplot_main_coords ]
		( \savedmomentxendx, \savedmomentxendy, \savedmomentxendz )
		-- ( \momentxendx, \momentxendy, \momentxendz ) ;
}

\def\epurehatchsteps{12} % steps per reference length
\pgfmathsetmacro\hatchstep{\reflength / (\epurehatchsteps - 1)}

\newcommand\drawepureofinternalmomentfromfirstforce{
	% along x

	\pgfmathsetmacro\xmax{\xsecond}
	\foreach \xhatch in { 0, \hatchstep, ..., \xmax } {
		\calculateinternalmomentatpointfromforce{\xhatch}{0}{0}%
			{\firstforcepointx}{\firstforcepointy}{\firstforcepointz}%
			{\firstforcevectorx}{\firstforcevectory}{\firstforcevectorz}

		\drawepurelinesatpointalongx{\xhatch}{0}{0}
	}

	\drawtwistingoffsetlineat{0}{0}{0}
	\drawtwistingoffsetlineat{\xfirst}{0}{0}
	\drawtwistingoffsetlineat{\xsecond}{0}{0}

	\calculateinternalmomentatpointfromforce{0}{0}{0}%
		{\firstforcepointx}{\firstforcepointy}{\firstforcepointz}%
		{\firstforcevectorx}{\firstforcevectory}{\firstforcevectorz}
	\drawepurelinesatpointalongx{0}{0}{0}[thick]

	\saveepureendpoints

	\calculateinternalmomentatpointfromforce{\xfirst}{0}{0}%
		{\firstforcepointx}{\firstforcepointy}{\firstforcepointz}%
		{\firstforcevectorx}{\firstforcevectory}{\firstforcevectorz}
	\drawepurelinesatpointalongx{\xfirst}{0}{0}[thick]

	\drawlinebetweensavedandcurrentz

	\saveepureendpoints

	\calculateinternalmomentatpointfromforce{\xsecond}{0}{0}%
		{\firstforcepointx}{\firstforcepointy}{\firstforcepointz}%
		{\firstforcevectorx}{\firstforcevectory}{\firstforcevectorz}
	\drawepurelinesatpointalongx{\xsecond}{0}{0}[thick]

	\drawlinebetweensavedandcurrentz

	\def\spiralaxialstep{1.1}
	\def\spiralinitialangle{-100}
	\def\wherefirstarrowonspiral{0.05}
	\drawtwistingspiralalongxbetween{\xsecond}{0}{0}{0}

	% along z

	\pgfmathsetmacro\zfrom{0}
	\pgfmathsetmacro\zto{\zsecond}
	\pgfmathsetmacro\zstep{\hatchstep * ( \zto - \zfrom ) / abs( \zto - \zfrom )}
	\pgfmathsetmacro\znext{\zfrom + \zstep}
	\foreach \zhatch in { \zfrom, \znext, ..., \zto } {
		\calculateinternalmomentatpointfromforce{\xsecond}{0}{\zhatch}%
			{\firstforcepointx}{\firstforcepointy}{\firstforcepointz}%
			{\firstforcevectorx}{\firstforcevectory}{\firstforcevectorz}

		\drawepurelinesatpointalongz{\xsecond}{0}{\zhatch}
	}

	\calculateinternalmomentatpointfromforce{\xsecond}{0}{0}%
		{\firstforcepointx}{\firstforcepointy}{\firstforcepointz}%
		{\firstforcevectorx}{\firstforcevectory}{\firstforcevectorz}
	\drawepurelinesatpointalongz{\xsecond}{0}{0}[thick]

	\saveepureendpoints

	\calculateinternalmomentatpointfromforce{\xsecond}{0}{\zsecond}%
		{\firstforcepointx}{\firstforcepointy}{\firstforcepointz}%
		{\firstforcevectorx}{\firstforcevectory}{\firstforcevectorz}
	\drawepurelinesatpointalongz{\xsecond}{0}{\zsecond}[thick]

	\drawlinebetweensavedandcurrentx
}

\newcommand\drawepureofinternalmomentfromsecondforce{
	% along x

	\pgfmathsetmacro\xmax{\xfirst}
	\foreach \xhatch in { 0, \hatchstep, ..., \xmax } {
		\calculateinternalmomentatpointfromforce{\xhatch}{0}{0}%
			{\secondforcepointx}{\secondforcepointy}{\secondforcepointz}%
			{\secondforcevectorx}{\secondforcevectory}{\secondforcevectorz}

		\drawepurelinesatpointalongx{\xhatch}{0}{0}
	}

	\drawtwistingoffsetlineat{0}{0}{0}
	\drawtwistingoffsetlineat{\xfirst}{0}{0}

	\calculateinternalmomentatpointfromforce{0}{0}{0}%
		{\secondforcepointx}{\secondforcepointy}{\secondforcepointz}%
		{\secondforcevectorx}{\secondforcevectory}{\secondforcevectorz}
	\drawepurelinesatpointalongx{0}{0}{0}[thick]

	\saveepureendpoints

	\calculateinternalmomentatpointfromforce{\xfirst}{0}{0}%
		{\secondforcepointx}{\secondforcepointy}{\secondforcepointz}%
		{\secondforcevectorx}{\secondforcevectory}{\secondforcevectorz}
	\drawepurelinesatpointalongx{\xfirst}{0}{0}[thick]

	\drawlinebetweensavedandcurrentz

	\def\spiralaxialstep{1.1}
	\def\spiralinitialangle{-140}
	\drawtwistingspiralalongxbetween{\xfirst}{0}{0}{0}

	% along z

	\pgfmathsetmacro\zfrom{0}
	\pgfmathsetmacro\zto{\zfirst}
	\pgfmathsetmacro\zstep{\hatchstep * ( \zto - \zfrom ) / abs( \zto - \zfrom )}
	\pgfmathsetmacro\znext{\zfrom + \zstep}
	\foreach \zhatch in { \zfrom, \znext, ..., \zto } {
		\calculateinternalmomentatpointfromforce{\xfirst}{0}{\zhatch}%
			{\secondforcepointx}{\secondforcepointy}{\secondforcepointz}%
			{\secondforcevectorx}{\secondforcevectory}{\secondforcevectorz}

		\drawepurelinesatpointalongz{\xfirst}{0}{\zhatch}
	}

	\calculateinternalmomentatpointfromforce{\xfirst}{0}{0}%
		{\secondforcepointx}{\secondforcepointy}{\secondforcepointz}%
		{\secondforcevectorx}{\secondforcevectory}{\secondforcevectorz}
	\drawepurelinesatpointalongz{\xfirst}{0}{0}[thick]

	\saveepureendpoints

	\calculateinternalmomentatpointfromforce{\xfirst}{0}{\zfirst}%
		{\secondforcepointx}{\secondforcepointy}{\secondforcepointz}%
		{\secondforcevectorx}{\secondforcevectory}{\secondforcevectorz}
	\drawepurelinesatpointalongz{\xfirst}{0}{\zfirst}[thick]

	\drawlinebetweensavedandcurrentx
}

% #1: color of first force
% #2: color of second force
% #3: color of sum
\newcommand\drawepureofinternalmomentfrombothforces[3]{
	% along x

	\def\externalforcecolor{#3}

	\pgfmathsetmacro\xmin{0}
	\pgfmathsetmacro\xnext{\xmin + \hatchstep}
	\pgfmathsetmacro\xmax{\xfirst}
	\foreach \xhatch in { \xmin, \xnext, ..., \xmax } {
		\calculateinternalmomentatpointfromforce{\xhatch}{0}{0}%
			{\firstforcepointx}{\firstforcepointy}{\firstforcepointz}%
			{\firstforcevectorx}{\firstforcevectory}{\firstforcevectorz}

		\pgfmathsetmacro\firstmomentx{\momentx}
		\pgfmathsetmacro\firstmomenty{\momenty}
		\pgfmathsetmacro\firstmomentz{\momentz}
		
		\calculateinternalmomentatpointfromforce{\xhatch}{0}{0}%
			{\secondforcepointx}{\secondforcepointy}{\secondforcepointz}%
			{\secondforcevectorx}{\secondforcevectory}{\secondforcevectorz}

		\pgfmathsetmacro\secondmomentx{\momentx}
		\pgfmathsetmacro\secondmomenty{\momenty}
		\pgfmathsetmacro\secondmomentz{\momentz}

		\pgfmathsetmacro\momentx{\firstmomentx + \secondmomentx}
		\pgfmathsetmacro\momenty{\firstmomenty + \secondmomenty}
		\pgfmathsetmacro\momentz{\firstmomentz + \secondmomentz}

		\drawepurelinesatpointalongx{\xhatch}{0}{0}
	}

	\calculateinternalmomentatpointfromforce{0}{0}{0}%
		{\firstforcepointx}{\firstforcepointy}{\firstforcepointz}%
		{\firstforcevectorx}{\firstforcevectory}{\firstforcevectorz}
	\pgfmathsetmacro\firstmomentx{\momentx}
	\pgfmathsetmacro\firstmomenty{\momenty}
	\pgfmathsetmacro\firstmomentz{\momentz}
	\calculateinternalmomentatpointfromforce{0}{0}{0}%
			{\secondforcepointx}{\secondforcepointy}{\secondforcepointz}%
			{\secondforcevectorx}{\secondforcevectory}{\secondforcevectorz}
	\pgfmathsetmacro\momentx{\momentx + \firstmomentx}
	\pgfmathsetmacro\momenty{\momenty + \firstmomenty}
	\pgfmathsetmacro\momentz{\momentz + \firstmomentz}

	\drawepurelinesatpointalongx{0}{0}{0}[thick]

	\pgfmathsetmacro\momentxatzero{\momentx}
	\pgfmathparse{(abs(\momentxatzero) > 0) ? 1 : 0}\ifdim\pgfmathresult pt>0pt%
		\drawtwistingoffsetlineat{0}{0}{0}
	\fi

	\saveepureendpoints

	\calculateinternalmomentatpointfromforce{\xfirst}{0}{0}%
		{\firstforcepointx}{\firstforcepointy}{\firstforcepointz}%
		{\firstforcevectorx}{\firstforcevectory}{\firstforcevectorz}
	\pgfmathsetmacro\firstmomentx{\momentx}
	\pgfmathsetmacro\firstmomenty{\momenty}
	\pgfmathsetmacro\firstmomentz{\momentz}
	\calculateinternalmomentatpointfromforce{\xfirst}{0}{0}%
			{\secondforcepointx}{\secondforcepointy}{\secondforcepointz}%
			{\secondforcevectorx}{\secondforcevectory}{\secondforcevectorz}
	\pgfmathsetmacro\momentx{\momentx + \firstmomentx}
	\pgfmathsetmacro\momenty{\momenty + \firstmomenty}
	\pgfmathsetmacro\momentz{\momentz + \firstmomentz}

	\drawepurelinesatpointalongx{\xfirst}{0}{0}[thick]

	\drawlinebetweensavedandcurrentz

	\pgfmathparse{(abs(\momentxatzero) > 0) ? 1 : 0}\ifdim\pgfmathresult pt>0pt%
		\pgfmathsetmacro\spiralaxialstep{.5}
		\def\spiralinitialangle{-20}
		\def\wherefirstarrowonspiral{.11}
		\drawtwistingspiralalongxbetween{\xfirst}{0}{0}{0}
	\fi

	\def\externalforcecolor{#1}

	\pgfmathsetmacro\xmin{\xfirst}
	\pgfmathsetmacro\xnext{\xmin + \hatchstep}
	\pgfmathsetmacro\xmax{\xsecond}
	\foreach \xhatch in { \xmin, \xnext, ..., \xmax } {
		\calculateinternalmomentatpointfromforce{\xhatch}{0}{0}%
			{\firstforcepointx}{\firstforcepointy}{\firstforcepointz}%
			{\firstforcevectorx}{\firstforcevectory}{\firstforcevectorz}

		\pgfmathsetmacro\firstmomentx{\momentx}
		\pgfmathsetmacro\firstmomenty{\momenty}
		\pgfmathsetmacro\firstmomentz{\momentz}

		\pgfmathsetmacro\secondmomentx{0}
		\pgfmathsetmacro\secondmomenty{0}
		\pgfmathsetmacro\secondmomentz{0}

		\pgfmathsetmacro\momentx{\firstmomentx + \secondmomentx}
		\pgfmathsetmacro\momenty{\firstmomenty + \secondmomenty}
		\pgfmathsetmacro\momentz{\firstmomentz + \secondmomentz}

		\drawepurelinesatpointalongx{\xhatch}{0}{0}
	}

	\drawtwistingoffsetlineat{\xsecond}{0}{0}
	\drawtwistingoffsetlineat{\xfirst}{0}{0}

	\calculateinternalmomentatpointfromforce{\xfirst}{0}{0}%
		{\firstforcepointx}{\firstforcepointy}{\firstforcepointz}%
		{\firstforcevectorx}{\firstforcevectory}{\firstforcevectorz}

	\drawepurelinesatpointalongx{\xfirst}{0}{0}[thick]

	\saveepureendpoints

	\calculateinternalmomentatpointfromforce{\xsecond}{0}{0}%
		{\firstforcepointx}{\firstforcepointy}{\firstforcepointz}%
		{\firstforcevectorx}{\firstforcevectory}{\firstforcevectorz}

	\drawepurelinesatpointalongx{\xsecond}{0}{0}[thick]

	\drawlinebetweensavedandcurrentz

	\pgfmathsetmacro\spiralaxialstep{\momentx / (\justP * .8 * \reflength)}
	\def\spiralinitialangle{-40}
	\def\wherefirstarrowonspiral{0.1}
	\drawtwistingspiralalongxbetween{\xsecond}{\xfirst}{0}{0}
	%%\drawtwistingspiralalongxbetween{\xfirst}{\xsecond}{0}{0}

	% along z

	\def\externalforcecolor{#2}

	\pgfmathsetmacro\zfrom{0}
	\pgfmathsetmacro\zto{\zfirst}
	\pgfmathsetmacro\zstep{\hatchstep * ( \zto - \zfrom ) / abs( \zto - \zfrom )}
	\pgfmathsetmacro\znext{\zfrom + \zstep}
	\foreach \zhatch in { \zfrom, \znext, ..., \zto } {
		\pgfmathsetmacro\firstmomentx{0}
		\pgfmathsetmacro\firstmomenty{0}
		\pgfmathsetmacro\firstmomentz{0}

		\calculateinternalmomentatpointfromforce{\xfirst}{0}{\zhatch}%
			{\secondforcepointx}{\secondforcepointy}{\secondforcepointz}%
			{\secondforcevectorx}{\secondforcevectory}{\secondforcevectorz}

		\pgfmathsetmacro\secondmomentx{\momentx}
		\pgfmathsetmacro\secondmomenty{\momenty}
		\pgfmathsetmacro\secondmomentz{\momentz}

		\pgfmathsetmacro\momentx{\firstmomentx + \secondmomentx}
		\pgfmathsetmacro\momenty{\firstmomenty + \secondmomenty}
		\pgfmathsetmacro\momentz{\firstmomentz + \secondmomentz}

		\drawepurelinesatpointalongz{\xfirst}{0}{\zhatch}
	}

	\calculateinternalmomentatpointfromforce{\xfirst}{0}{0}%
		{\secondforcepointx}{\secondforcepointy}{\secondforcepointz}%
		{\secondforcevectorx}{\secondforcevectory}{\secondforcevectorz}

	\drawepurelinesatpointalongz{\xfirst}{0}{0}[thick]

	\saveepureendpoints

	\calculateinternalmomentatpointfromforce{\xfirst}{0}{\zfirst}%
		{\secondforcepointx}{\secondforcepointy}{\secondforcepointz}%
		{\secondforcevectorx}{\secondforcevectory}{\secondforcevectorz}

	\drawepurelinesatpointalongz{\xfirst}{0}{\zfirst}[thick]

	\drawlinebetweensavedandcurrentx

	\def\externalforcecolor{#1}

	\pgfmathsetmacro\zfrom{0}
	\pgfmathsetmacro\zto{\zsecond}
	\pgfmathsetmacro\zstep{\hatchstep * ( \zto - \zfrom ) / abs( \zto - \zfrom )}
	\pgfmathsetmacro\znext{\zfrom + \zstep}
	\foreach \zhatch in { \zfrom, \znext, ..., \zto } {
		\calculateinternalmomentatpointfromforce{\xsecond}{0}{\zhatch}%
			{\firstforcepointx}{\firstforcepointy}{\firstforcepointz}%
			{\firstforcevectorx}{\firstforcevectory}{\firstforcevectorz}

		\pgfmathsetmacro\firstmomentx{\momentx}
		\pgfmathsetmacro\firstmomenty{\momenty}
		\pgfmathsetmacro\firstmomentz{\momentz}

		\pgfmathsetmacro\secondmomentx{0}
		\pgfmathsetmacro\secondmomenty{0}
		\pgfmathsetmacro\secondmomentz{0}

		\pgfmathsetmacro\momentx{\firstmomentx + \secondmomentx}
		\pgfmathsetmacro\momenty{\firstmomenty + \secondmomenty}
		\pgfmathsetmacro\momentz{\firstmomentz + \secondmomentz}

		\drawepurelinesatpointalongz{\xsecond}{0}{\zhatch}
	}

	\calculateinternalmomentatpointfromforce{\xsecond}{0}{0}%
		{\firstforcepointx}{\firstforcepointy}{\firstforcepointz}%
		{\firstforcevectorx}{\firstforcevectory}{\firstforcevectorz}

	\drawepurelinesatpointalongz{\xsecond}{0}{0}[thick]

	\saveepureendpoints

	\calculateinternalmomentatpointfromforce{\xsecond}{0}{\zsecond}%
		{\firstforcepointx}{\firstforcepointy}{\firstforcepointz}%
		{\firstforcevectorx}{\firstforcevectory}{\firstforcevectorz}

	\drawepurelinesatpointalongz{\xsecond}{0}{\zsecond}[thick]

	\drawlinebetweensavedandcurrentx
}

% ~ ~ ~ ~ ~ ~ ~ ~

\def\firstforcepointx{ \xsecond }
\def\firstforcepointy{ 0 }
\def\firstforcepointz{ \zsecond }
\def\firstforcevectorx{ 0 }
\def\firstforcevectory{ \justP }
\def\firstforcevectorz{ 0 }

\def\secondforcepointx{ \xfirst }
\def\secondforcepointy{ 0 }
\def\secondforcepointz{ \zfirst }
\def\secondforcevectorx{ 0 }
\def\secondforcevectory{ -\justP }
\def\secondforcevectorz{ 0 }

\def\thirdforcepointx{ \xsecond }
\def\thirdforcepointy{ 0 }
\def\thirdforcepointz{ 0 }
\def\thirdforcevectorx{ 0 }
\def\thirdforcevectory{ 0 }
\def\thirdforcevectorz{ -\justP }

\def\camerafirstangle{60} % 60
\def\camerasecondangle{135} % 135

\tdplotsetmaincoords{\camerafirstangle}{\camerasecondangle}

\thispagestyle{empty} % no numbering for 1st page

\begin{center}
\vspace*{\fill}

{\large \textbf{Домашнее задание №\hspace{.33ex}3}}

\vspace{1cm}

{\large \MakeUppercase{Сложное напряжённое состояние}}

\vspace{7cm}
\vspace*{\fill}
\end{center}

\newpage

\begin{center}

\textbf{Задача 1\raisebox{.7ex}{\small я}}
\vspace{.4cm}

Определить размер сечения для~бруса кругового профиля с~диаметром~$D$
и~для~бруса прямоугольного сечения с~размерами ${H \!\times\! B}$\hbox{,\hspace{.2ex}} $B$~--- ширина\hbox{,\hspace{.2ex}} ${H \hspace{-.5ex}=\hspace{-.3ex} 2B}$~--- высота
\vspace{.4cm}

${F \hspace{-.5ex}=\hspace{-.4ex} 1\:}$кН\hbox{,\hspace{.5ex}}
${\ell \!=\! 200\:}$мм\hbox{,\hspace{.5ex}}
${\sigma_{\mathrm{\scalebox{.6}{T}p}} \hspace{-.5ex}=\hspace{-.3ex} \sigma_{\mathrm{\scalebox{.6}{T}c}} \hspace{-.5ex}=\hspace{-.3ex} 300\:}$МПа\hbox{,\hspace{.5ex}}
${n_{\mathrm{\scalebox{.6}{T}}} \hspace{-.44ex}=\hspace{-.33ex} 2}$
\vspace{1cm}

\emph{Вариант №\hspace{.33ex}8}
\vspace{.8cm}

\end{center}

% ~ ~ ~ ~
\newpage
% ~ ~ ~ ~

\begin{center}
\textbf{Задача 2\raisebox{.7ex}{\small я}}
\vspace{.4cm}

Построить эпюры изгибающих и крутящих моментов
\vspace{1cm}

\emph{Вариант №\hspace{.33ex}8}
\vspace{.8cm}

\scalebox{1.1}{
\begin{tikzpicture}[ scale=1 ]

\pgfmathsetmacro\cornerlength{\reflength / 20}

\drawclampyz{( 0, 0, 0 )}
\drawbeam
\drawbeamtext

\drawload{( {\firstforcepointx}, {\firstforcepointy}, {\firstforcepointz} )}{( {\firstforcevectorx * \forcearrowscale}, {\firstforcevectory * \forcearrowscale}, {\firstforcevectorz * \forcearrowscale} )}{pos=.1, below, inner sep=0pt, outer sep=11pt}{\scalebox{\textscale}{$ P $}}

\drawload{( {\secondforcepointx}, {\secondforcepointy}, {\secondforcepointz} )}{( {\secondforcevectorx * \forcearrowscale}, {\secondforcevectory * \forcearrowscale}, {\secondforcevectorz * \forcearrowscale} )}{pos=.6, below, inner sep=0pt, outer sep=11pt}{\scalebox{\textscale}{$ P $}}

\end{tikzpicture}
}
\end{center}

\vspace{2cm}

Для линейно-упругих систем примен\'{и}м принцип независимости действия сил, согласно которому
результат от действия \emph{всех} нагрузок аналогичен сумме действий от \emph{каждой} из нагрузок.
Поэтому, для решения задачи найдём внутренний момент от каждой внешней силы по отдельности.

\begin{center}
\vspace*{1cm}
\scalebox{1}{
\begin{tikzpicture}[ scale=1.1 ]

\begin{scope}[ yshift=4.4cm ]

\pgfmathsetmacro\cornerlength{\reflength / 32}

%%\drawclampyz{( 0, 0, 0 )}
\drawbeam

\def\externalforcecolor{red}

\def\twistingmomentmultiplierx{0}
\def\twistingmomentmultipliery{0}
\def\twistingmomentmultiplierz{1}

\drawload{( {\firstforcepointx}, {\firstforcepointy}, {\firstforcepointz} )}{( {\firstforcevectorx * \forcearrowscale}, {\firstforcevectory * \forcearrowscale}, {\firstforcevectorz * \forcearrowscale} )}{pos=.1, below, inner sep=0pt, outer sep=11pt}{\scalebox{\textscale}{$ P $}}

\drawepureofinternalmomentfromfirstforce

\end{scope}

\begin{scope}[ yshift=-4.4cm ]

\pgfmathsetmacro\cornerlength{\reflength / 32}

%%\drawclampyz{( 0, 0, 0 )}
\drawbeam

\def\externalforcecolor{blue}

\def\twistingmomentmultiplierx{0}
\def\twistingmomentmultipliery{1}
\def\twistingmomentmultiplierz{0}

\drawload{( {\secondforcepointx}, {\secondforcepointy}, {\secondforcepointz} )}{( {\secondforcevectorx * \forcearrowscale}, {\secondforcevectory * \forcearrowscale}, {\secondforcevectorz * \forcearrowscale} )}{pos=.6, below, inner sep=0pt, outer sep=11pt}{\scalebox{\textscale}{$ P $}}

\drawepureofinternalmomentfromsecondforce

\end{scope}

\end{tikzpicture}
}
\end{center}

\newpage

Сложение эпюр от отдельных внешних сил даёт суммарную эпюру от всех сил, действующих на конструкцию.
Суммарная эпюра:

\begin{center}
\vspace{1cm}

\scalebox{1}{
\begin{tikzpicture}[ scale=1.2 ]

\pgfmathsetmacro\cornerlength{\reflength / 32}

\drawbeam

\def\twistingmomentepureoffset{.4*\reflength}
\def\twistingmomentmultiplierx{0}
\def\twistingmomentmultipliery{0}
\def\twistingmomentmultiplierz{1}

\def\externalforcecolor{red}

\drawload{( {\firstforcepointx}, {\firstforcepointy}, {\firstforcepointz} )}{( {\firstforcevectorx * \forcearrowscale}, {\firstforcevectory * \forcearrowscale}, {\firstforcevectorz * \forcearrowscale} )}{pos=.1, below, inner sep=0pt, outer sep=11pt}{\scalebox{\textscale}{$ P $}}

\def\externalforcecolor{blue}

\drawload{( {\secondforcepointx}, {\secondforcepointy}, {\secondforcepointz} )}{( {\secondforcevectorx * \forcearrowscale}, {\secondforcevectory * \forcearrowscale}, {\secondforcevectorz * \forcearrowscale} )}{pos=.6, below, inner sep=0pt, outer sep=11pt}{\scalebox{\textscale}{$ P $}}

\drawepureofinternalmomentfrombothforces{red}{blue}{magenta}

\end{tikzpicture}
}
\end{center}

% ~ ~ ~ ~
\newpage
% ~ ~ ~ ~

\def\camerafirstangle{55} % 60, 55
\def\camerasecondangle{130} % 135, 130

\tdplotsetmaincoords{\camerafirstangle}{\camerasecondangle}

\begin{center}
\textbf{Задача 3\raisebox{.7ex}{\small я}}
\vspace{.4cm}

Построить эпюры изгибающих и крутящих моментов
\vspace{1cm}

\emph{Вариант №\hspace{.33ex}8}
\vspace{.8cm}

\scalebox{1.1}{
\begin{tikzpicture}[ scale=1 ]

\pgfmathsetmacro\cornerlength{\reflength / 20}

\drawclampyz{( 0, 0, 0 )}
\drawbeam
\drawbeamtext

\drawload{( {\firstforcepointx}, {\firstforcepointy}, {\firstforcepointz} )}{( {\firstforcevectorx * \forcearrowscale}, {\firstforcevectory * \forcearrowscale}, {\firstforcevectorz * \forcearrowscale} )}{pos=.1, below, inner sep=0pt, outer sep=11pt}{\scalebox{\textscale}{$ P $}}

\drawload{( {\secondforcepointx}, {\secondforcepointy}, {\secondforcepointz} )}{( {\secondforcevectorx * \forcearrowscale}, {\secondforcevectory * \forcearrowscale}, {\secondforcevectorz * \forcearrowscale} )}{pos=.6, below, inner sep=0pt, outer sep=11pt}{\scalebox{\textscale}{$ P $}}

\drawload{( {\thirdforcepointx}, {\thirdforcepointy}, {\thirdforcepointz} )}{( {\thirdforcevectorx * \forcearrowscale}, {\thirdforcevectory * \forcearrowscale}, {\thirdforcevectorz * \forcearrowscale} )}{pos=.33, left, inner sep=0pt, outer sep=5pt}{\scalebox{\textscale}{$ P $}}

\end{tikzpicture}
}
\end{center}

\end{document}
