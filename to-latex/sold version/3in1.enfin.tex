\documentclass[14pt]{extarticle}

\usepackage[utf8]{inputenc}

\usepackage{geometry}
\geometry{papersize={21cm, 29.7cm}} % A4
\geometry{tmargin=2cm, bmargin=2cm, outer=2cm, inner=2cm, head=2cm, foot=2cm}

\usepackage[unicode=true, pdfusetitle,%
	bookmarks=true, bookmarksnumbered=false, bookmarksopen=true, linktocpage=true,%
	breaklinks=false, pdfborder={0 0 0}, backref=false,%
	colorlinks=true, linkcolor=black, citecolor=black, urlcolor=black]%
{hyperref}

\hypersetup{pdftitle={Сложное напряжённое состояние (домашнее задание 3)}}

\usepackage{xcolor}
\usepackage{graphicx}

\renewcommand\thepage{\oldstylenums{\arabic{page}}}

\usepackage{fancyhdr}
\fancyhf{}
\renewcommand{\headrulewidth}{0pt}
\renewcommand{\footrulewidth}{0pt}
\fancyhead[CE,CO]{${(\:}$\raisebox{.1em}{\thepage}${\:)}$}
%%\fancyfoot[CE,CO]{$\hspace{-9em}\textcolor{red}{%
%%\textbf{\textsf{ДОКУМЕНТ НЕ ВЫКУПЛЕН}} \hspace{1em}%
%%\textbf{\textsf{ДОКУМЕНТ НЕ ВЫКУПЛЕН}} \hspace{1em}%
%%\textbf{\textsf{ДОКУМЕНТ НЕ ВЫКУПЛЕН}}}$}
\pagestyle{fancy}
\geometry{headsep=.5cm}

\usepackage{tikz}
\usepackage{tikz-3dplot}
\usetikzlibrary{calc}
\usetikzlibrary{arrows, arrows.meta}
\usetikzlibrary{decorations.markings}
\usetikzlibrary{patterns}

\usepackage[T2A, T1]{fontenc}
\usepackage[english, russian]{babel}

\def\horizontalindent{4ex}
\setlength{\parindent}{\horizontalindent} % offset of the first line
\usepackage{indentfirst}

\usepackage{setspace}
\setstretch{1.3} % spacing between lines

\usepackage{microtype}

\usepackage{tempora}
\usepackage{upgreek}

\usepackage{amssymb}
\usepackage{amsmath}
\usepackage{mathtools}
\usepackage{wasysym}

\usepackage{xargs}
\usepackage{xifthen}

\newcommand{\lquote}[0]{«} % {<<}
\newcommand{\rquote}[0]{»} % {>>}
\newcommand{\inquotes}[1]{\lquote{#1}\rquote}

\usepackage[many]{tcolorbox}
\tcbset{
	frame code={}
	center title,
	left = 0pt,
	right = 0pt,
	top = 0pt,
	bottom = 0pt,
	colback = gray!10,
	colframe = white,
	width = \dimexpr\textwidth\relax,
	enlarge left by = 0mm,
	boxsep = 5pt,
	arc = 0pt, outer arc = 0pt}

\usepackage{verbatim}

%%\usepackage{draftwatermark} % [nostamp]
%%\SetWatermarkText{ВЫКУПАЕТСЯ}
%%\SetWatermarkScale{.7}
%%\SetWatermarkAngle{-40}
%%\SetWatermarkColor{red!50}

\begin{document}

\def\beamlinewidth{2.8pt}
\tikzstyle{beam line} = [ line width=\beamlinewidth, line cap=round, color=black ]

\def\auxlinewidth{.6pt}
\tikzstyle{aux line thin} = [ line width=\auxlinewidth, color=black, line cap=round ]
\tikzstyle{aux thin dashed} = [ aux line thin, dash pattern=on 0pt off 1.6\pgflinewidth ]
\tikzstyle{aux line thick} = [ line width=2.5*\auxlinewidth, color=black, line cap=round ]
\tikzstyle{aux thick dashed} = [ aux line thick, dash pattern=on 0pt off 1.6\pgflinewidth ]

\def\colorforforces{orange!33!red}
\def\externalforcecolor{\colorforforces}
\tikzstyle{external force} =
	[ line width=(2/3)*\beamlinewidth, color=\externalforcecolor, line cap=round, -{Triangle[round, length=12pt, width=8pt]} ]

\def\textscale{1.2}

\def\reflength{5} % reference length

\pgfmathsetmacro\cornerlength{\reflength / 20}

\pgfmathsetmacro\onePl{\reflength / 3}
\pgfmathsetmacro\justP{\onePl / \reflength}

\def\epurehatchsteps{12} % steps per reference length
\pgfmathsetmacro\hatchstep{\reflength / (\epurehatchsteps - 1)}

\tikzset{
	hatch distance/.store in=\hatchdistance,
	hatch distance=15pt,
	hatch thickness/.store in=\hatchthickness,
	hatch thickness=.6pt %%\auxlinewidth
}

\makeatletter
\pgfdeclarepatternformonly[\hatchdistance,\hatchthickness]{flexible hatch}
	{\pgfqpoint{0pt}{0pt}}
	{\pgfqpoint{\hatchdistance}{\hatchdistance}}
	{\pgfpoint{\hatchdistance-1pt}{\hatchdistance-1pt}}%
	{
		\pgfsetcolor{\tikz@pattern@color}
		\pgfsetlinewidth{\hatchthickness}
		\pgfpathmoveto{\pgfqpoint{0pt}{0pt}}
		\pgfpathlineto{\pgfqpoint{\hatchdistance}{\hatchdistance}}
		\pgfusepath{stroke}
	}
\makeatother

\def\clampsize{.8}
\def\clampcolor{gray}

\tikzstyle{clamp line} = [ line width=(2/3)*\beamlinewidth, line cap=round, color=\clampcolor ]
\tikzstyle{clamp line thin} = [ line width=2*\auxlinewidth, line cap=round, color=\clampcolor ]

\newcommand\drawclampyz[1]{
	\draw [ clamp line, tdplot_main_coords ] ($ #1 + ( 0, .6*\clampsize, .5*\clampsize ) $) -- ++( - .6*\clampsize, 0, 0 ) ;
	\draw [ clamp line, tdplot_main_coords ] ($ #1 + ( 0, -.6*\clampsize, .5*\clampsize ) $) -- ++( - .7*\clampsize, 0, 0 ) ;
	\draw [ clamp line, tdplot_main_coords ] ($ #1 + ( 0, .6*\clampsize, -.5*\clampsize ) $) -- ++( - .7*\clampsize, 0, 0 ) ;
	\draw [ clamp line, tdplot_main_coords ] ($ #1 + ( 0, -.6*\clampsize, -.5*\clampsize ) $) -- ++( - .7*\clampsize, 0, 0 ) ;

	\draw [ clamp line, rounded corners=.5\pgflinewidth, fill=white, tdplot_main_coords ]
		($ #1 + ( 0, .6*\clampsize, .5*\clampsize ) $)
		-- ($ #1 + ( 0, -.6*\clampsize, .5*\clampsize ) $)
		-- ($ #1 + ( 0, -.6*\clampsize, -.5*\clampsize ) $)
		-- ($ #1 + ( 0, .6*\clampsize, -.5*\clampsize ) $)
		-- cycle ;

	\draw [ clamp line thin, color=\clampcolor, tdplot_main_coords ]
		plot [ smooth, tension=.8 ] coordinates {
		($ #1 + ( - .6*\clampsize, .6*\clampsize, .5*\clampsize ) $)
		($ #1 + ( -.5*\clampsize, .3*\clampsize, .5*\clampsize ) $)
		($ #1 + ( - .75*\clampsize, -.3*\clampsize, .5*\clampsize ) $)
		($ #1 + ( - .7*\clampsize, -.6*\clampsize, .5*\clampsize ) $)
	} ;
	\draw [ clamp line thin, color=\clampcolor, tdplot_main_coords ]
		plot [ smooth, tension=.5 ] coordinates {
		($ #1 + ( - .6*\clampsize, .6*\clampsize, .5*\clampsize ) $)
		($ #1 + ( -.66*\clampsize, .6*\clampsize, .2*\clampsize ) $)
		($ #1 + ( - .55*\clampsize, .6*\clampsize, -.2*\clampsize ) $)
		($ #1 + ( - .7*\clampsize, .6*\clampsize, -.5*\clampsize ) $)
	} ;
}

% #1: point
% #2: force vector
% #3: options for node
% #4: node text
\newcommand\drawload[4]{
	\draw [ external force, tdplot_main_coords ]
		($ #1 - #2 $) -- ++#2
		node [ #3 ] { #4 } ;
}

% takes two points as cartesian {x}{y}{z} and calculates cross product of their location vectors
% placing the result into last three arguments
\newcommand\tdcrossproductcartesian[9]{
\def\crossz{ (#1) * (#5) - (#2) * (#4) }
\def\crossx{ (#2) * (#6) - (#3) * (#5) }
\def\crossy{ (#3) * (#4) - (#1) * (#6) }
\pgfmathsetmacro{#7}{\crossx}
\pgfmathsetmacro{#8}{\crossy}
\pgfmathsetmacro{#9}{\crossz}
}

% {rx}{ry}{rz} of a point in question
% {rfx}{rfy}{rfz} of force point
% {fx}{fy}{fz} of force vector
% places result in \momentx, \momenty, \momentz
\newcommand\calculateinternalmomentatpointfromforce[9]{
	% get vector d from current point to the force
	\pgfmathsetmacro\distancex{(#4) - (#1)}
	\pgfmathsetmacro\distancey{(#5) - (#2)}
	\pgfmathsetmacro\distancez{(#6) - (#3)}
	% balance of moments: M + d ✕ F = 0
	% internal moment M = - d ✕ F = F ✕ d
	\tdcrossproductcartesian%
		{#7}{#8}{#9}%
		{\distancex}{\distancey}{\distancez}%
		{\momentx}{\momenty}{\momentz}
}

\def\twistingmomentepureoffset{\reflength / 3}
\def\twistingmomentmultiplierx{0}
\def\twistingmomentmultipliery{0}
\def\twistingmomentmultiplierz{1}

% #1,#2,#3 = x,y,z
% #4 = color
\newcommandx*\drawtwistingoffsetlineat[4][4=\epurecolor]{
	\draw [ aux thick dashed, color=#4, tdplot_main_coords ]
		( #1, #2, #3 ) -- ++( {\twistingmomentmultiplierx * \twistingmomentepureoffset}, {\twistingmomentmultipliery * \twistingmomentepureoffset}, {\twistingmomentmultiplierz * \twistingmomentepureoffset} ) ;
}

%%\newcommand\drawtwistingoffsetbasebetween[6]{
%%	\draw [ aux line thick, color=\epurecolor, tdplot_main_coords ]
%%		($ ( #1, #2, #3 ) + ( {\twistingmomentmultiplierx * \twistingmomentepureoffset}, {\twistingmomentmultipliery * \twistingmomentepureoffset}, {\twistingmomentmultiplierz * \twistingmomentepureoffset} ) $)
%%		-- ($ ( #4, #5, #6 ) + ( {\twistingmomentmultiplierx * \twistingmomentepureoffset}, {\twistingmomentmultipliery * \twistingmomentepureoffset}, {\twistingmomentmultiplierz * \twistingmomentepureoffset} ) $) ;
%%}

\tikzset {
	set arrow inside/.code={\pgfqkeys{/tikz/arrow inside}{#1}},
	set arrow inside={end/.initial=>, opt/.initial=},
	/pgf/decoration/Mark/.style={
		mark/.expanded=at position #1 with {
			\noexpand\arrow[\pgfkeysvalueof{/tikz/arrow inside/opt}]{\pgfkeysvalueof{/tikz/arrow inside/end}}
		}
	},
	arrow inside/.style 2 args={
		set arrow inside={#1},
		postaction={
			decorate,decoration={
				markings,Mark/.list={#2}
			}
		}
	},
}

\def\samplespercircle{30}

\def\spiralinitialangle{-10}
\def\spiralaxialstep{1}
\def\wherefirstarrowonspiral{0.075}

% #1 = begin z, #2 = end z
% #3, #4 = x, y
% #5 = color
% needs pre-calculated \momentz
\newcommandx*\drawtwistingspiralalongzbetween[5][5=\epurecolor]{
	\pgfmathparse{(abs(\momentz) > 0) ? 1 : 0}\ifdim\pgfmathresult pt>0pt%
		\pgfmathsetmacro\momentzsign{\momentz / abs(\momentz)}
	\else
		\pgfmathsetmacro\momentzsign{0}
	\fi

	\pgfmathsetmacro\momentzmidx{#3 + \twistingmomentmultiplierx * ( \twistingmomentepureoffset + .5*abs(\momentz) )}
	\pgfmathsetmacro\momentzmidy{#4 + \twistingmomentmultipliery * ( \twistingmomentepureoffset + .5*abs(\momentz) )}
	\pgfmathsetmacro\momentzmidz{#1 + \twistingmomentmultiplierz * ( \twistingmomentepureoffset + .5*abs(\momentz) )}

	\pgfmathsetmacro\spiralradius{\momentz / 2}
	\pgfmathsetmacro\axialadvance{abs(\spiralaxialstep) * (#2 - #1) / abs(#2 - #1)}
	\pgfmathsetmacro\axialshift{-\spiralinitialangle*\axialadvance/360}
	\pgfmathsetmacro\lengthincircles{abs(#2 - #1) / abs(\spiralaxialstep)}
	\pgfmathsetmacro\endangle{\lengthincircles * 360 + \spiralinitialangle}
	\pgfmathsetmacro\howmanysamples{\lengthincircles * \samplespercircle}

	\draw[ aux line thick, color=#5, tdplot_main_coords ]
		plot [ domain=\spiralinitialangle:\endangle, variable=\t, samples=\howmanysamples ]
		( {\momentzmidx + \momentzsign*\spiralradius*sin(\t)},
		{\momentzmidy + abs(\momentzsign)*\spiralradius*cos(\t)},
			{\momentzmidz + \axialadvance*\t/360 + \axialshift} )
		[ arrow inside = {}{\wherefirstarrowonspiral, 1} ] ;
}

% #1 = begin y, #2 = end y
% #3, #4 = x, z
% #5 = color
% needs pre-calculated \momenty
\newcommandx*\drawtwistingspiralalongybetween[5][5=\epurecolor]{
	\pgfmathparse{(abs(\momenty) > 0) ? 1 : 0}\ifdim\pgfmathresult pt>0pt%
		\pgfmathsetmacro\momentysign{\momenty / abs(\momenty)}
	\else
		\pgfmathsetmacro\momentysign{0}
	\fi

	\pgfmathsetmacro\momentymidx{#3 + \twistingmomentmultiplierx * ( \twistingmomentepureoffset + .5*abs(\momenty) )}
	\pgfmathsetmacro\momentymidy{#1 + \twistingmomentmultipliery * ( \twistingmomentepureoffset + .5*abs(\momenty) )}
	\pgfmathsetmacro\momentymidz{#4 + \twistingmomentmultiplierz * ( \twistingmomentepureoffset + .5*abs(\momenty) )}

	\pgfmathsetmacro\spiralradius{\momenty / 2}
	\pgfmathsetmacro\axialadvance{abs(\spiralaxialstep) * (#2 - #1) / abs(#2 - #1)}
	\pgfmathsetmacro\axialshift{-\spiralinitialangle*\axialadvance/360}
	\pgfmathsetmacro\lengthincircles{abs(#2 - #1) / abs(\spiralaxialstep)}
	\pgfmathsetmacro\endangle{\lengthincircles * 360 + \spiralinitialangle}
	\pgfmathsetmacro\howmanysamples{\lengthincircles * \samplespercircle}

	\draw[ aux line thick, color=#5, tdplot_main_coords ]
		plot [ domain=\spiralinitialangle:\endangle, variable=\t, samples=\howmanysamples ]
		( {\momentymidx - \momentysign*\spiralradius*sin(\t)},
			{\momentymidy + \axialadvance*\t/360 + \axialshift},
		{\momentymidz + abs(\momentysign)*\spiralradius*cos(\t)} )
		[ arrow inside = {}{\wherefirstarrowonspiral, 1} ] ;
}

% #1 = begin x, #2 = end x
% #3, #4 = y, z
% #5 = color
% needs pre-calculated \momentx
\newcommandx*\drawtwistingspiralalongxbetween[5][5=\epurecolor]{
	\pgfmathparse{(abs(\momentx) > 0) ? 1 : 0}\ifdim\pgfmathresult pt>0pt%
		\pgfmathsetmacro\momentxsign{\momentx / abs(\momentx)}
	\else
		\pgfmathsetmacro\momentxsign{0}
	\fi

	\pgfmathsetmacro\momentxmidx{#1 + \twistingmomentmultiplierx * ( \twistingmomentepureoffset + .5*abs(\momentx) )}
	\pgfmathsetmacro\momentxmidy{#3 + \twistingmomentmultipliery * ( \twistingmomentepureoffset + .5*abs(\momentx) )}
	\pgfmathsetmacro\momentxmidz{#4 + \twistingmomentmultiplierz * ( \twistingmomentepureoffset + .5*abs(\momentx) )}

	\pgfmathsetmacro\spiralradius{\momentx / 2}
	\pgfmathsetmacro\axialadvance{abs(\spiralaxialstep) * (#2 - #1) / abs(#2 - #1)}
	\pgfmathsetmacro\axialshift{-\spiralinitialangle*\axialadvance/360}
	\pgfmathsetmacro\lengthincircles{abs(#2 - #1) / abs(\spiralaxialstep)}
	\pgfmathsetmacro\endangle{\lengthincircles * 360 + \spiralinitialangle}
	\pgfmathsetmacro\howmanysamples{\lengthincircles * \samplespercircle}

	\draw[ aux line thick, color=#5, tdplot_main_coords ]
		plot [ domain=\spiralinitialangle:\endangle, variable=\t, samples=\howmanysamples ]
		( {\momentxmidx + \axialadvance*\t/360 + \axialshift},
			{\momentxmidy + \momentxsign*\spiralradius*sin(\t)},
			{\momentxmidz + abs(\momentxsign)*\spiralradius*cos(\t)} )
		[ arrow inside = {}{\wherefirstarrowonspiral, 1} ] ;
}

% #1,#2,#3 are x,y,z of point
% #4,#5,#6 are colors of lines for moment around x axis, y axis and z axis
% anything in #7 for thick line
% needs pre-calculated \momentx, \momenty, \momentz
\newcommandx*\drawepurelinesatpointalongx[7][7=]{
	\ifthenelse{\isempty{#7}}%
		{\def\linestyle{aux line thin}}%
		{\def\linestyle{aux line thick}}

	\pgfmathsetmacro\momentzendx{#1 + 0}
	\pgfmathsetmacro\momentzendy{#2 - \momentz}
	\pgfmathsetmacro\momentzendz{#3 + 0}

	\pgfmathsetmacro\momentyendx{#1 + 0}
	\pgfmathsetmacro\momentyendy{#2 + 0}
	\pgfmathsetmacro\momentyendz{#3 + \momenty}

	\pgfmathsetmacro\momentxendx{#1 + \twistingmomentmultiplierx * ( \twistingmomentepureoffset + abs(\momentx) )}
	\pgfmathsetmacro\momentxendy{#2 + \twistingmomentmultipliery * ( \twistingmomentepureoffset + abs(\momentx) )}
	\pgfmathsetmacro\momentxendz{#3 + \twistingmomentmultiplierz * ( \twistingmomentepureoffset + abs(\momentx) )}

	\pgfmathsetmacro\momentxmidx{#1 + \twistingmomentmultiplierx * ( \twistingmomentepureoffset + .5*abs(\momentx) )}
	\pgfmathsetmacro\momentxmidy{#2 + \twistingmomentmultipliery * ( \twistingmomentepureoffset + .5*abs(\momentx) )}
	\pgfmathsetmacro\momentxmidz{#3 + \twistingmomentmultiplierz * ( \twistingmomentepureoffset + .5*abs(\momentx) )}

	% bending
	\draw [ \linestyle, color=#5, tdplot_main_coords ]
		( #1, #2, #3 ) -- ( \momentyendx, \momentyendy, \momentyendz ) ;
	\draw [ \linestyle, color=#6, tdplot_main_coords ]
		( #1, #2, #3 ) -- ( \momentzendx, \momentzendy, \momentzendz ) ;

	% twisting
	\pgfmathparse{(abs(\momentx) > 0) ? 1 : 0}\ifdim\pgfmathresult pt>0pt%
		\ifthenelse{\isempty{#7}}%
		{}{%
		\draw[ aux thick dashed, color=#4, tdplot_main_coords  ]
			plot [ domain=0:360, variable=\t, samples=60 ]
			( {\momentxmidx + 0}, {\momentxmidy + .5*\momentx*sin(\t)}, {\momentxmidz + .5*\momentx*cos(\t)} ) ;
		}
	\fi
}

% #1,#2,#3 are x,y,z of point
% #4,#5,#6 are colors of lines for moment around x axis, y axis and z axis
% anything in #7 for thick line
% needs pre-calculated \momentx, \momenty, \momentz
\newcommandx*\drawepurelinesatpointalongy[7][7=]{
	\ifthenelse{\isempty{#7}}%
		{\def\linestyle{aux line thin}}%
		{\def\linestyle{aux line thick}}

	\pgfmathsetmacro\momentzendx{#1 + \momentz}
	\pgfmathsetmacro\momentzendy{#2 + 0}
	\pgfmathsetmacro\momentzendz{#3 + 0}

	\pgfmathsetmacro\momentxendx{#1 + 0}
	\pgfmathsetmacro\momentxendy{#2 + 0}
	\pgfmathsetmacro\momentxendz{#3 - \momentx}

	\pgfmathsetmacro\momentyendx{#1 + \twistingmomentmultiplierx * ( \twistingmomentepureoffset + abs(\momenty) )}
	\pgfmathsetmacro\momentyendy{#2 + \twistingmomentmultipliery * ( \twistingmomentepureoffset + abs(\momenty) )}
	\pgfmathsetmacro\momentyendz{#3 + \twistingmomentmultiplierz * ( \twistingmomentepureoffset + abs(\momenty) )}

	\pgfmathsetmacro\momentymidx{#1 + \twistingmomentmultiplierx * ( \twistingmomentepureoffset + .5*abs(\momenty) )}
	\pgfmathsetmacro\momentymidy{#2 + \twistingmomentmultipliery * ( \twistingmomentepureoffset + .5*abs(\momenty) )}
	\pgfmathsetmacro\momentymidz{#3 + \twistingmomentmultiplierz * ( \twistingmomentepureoffset + .5*abs(\momenty) )}

	% bending
	\draw [ \linestyle, color=#4, tdplot_main_coords ]
		( #1, #2, #3 ) -- ( \momentxendx, \momentxendy, \momentxendz ) ;
	\draw [ \linestyle, color=#6, tdplot_main_coords ]
		( #1, #2, #3 ) -- ( \momentzendx, \momentzendy, \momentzendz ) ;

	% twisting
	\pgfmathparse{(abs(\momenty) > 0) ? 1 : 0}\ifdim\pgfmathresult pt>0pt%
		\ifthenelse{\isempty{#7}}%
		{}{%
		\draw[ aux thick dashed, color=#5, tdplot_main_coords  ]
			plot [ domain=0:360, variable=\t, samples=60 ]
			( {\momentymidx + .5*\momenty*sin(\t)}, {\momentymidy + 0}, {\momentymidz + .5*\momenty*cos(\t)} ) ;
		}
	\fi
}

% #1,#2,#3 are x,y,z of point
% #4,#5,#6 are colors of lines for moment around x axis, y axis and z axis
% anything in #7 for thick line
% needs pre-calculated \momentx, \momenty, \momentz
\newcommandx*\drawepurelinesatpointalongz[7][7=]{
	\ifthenelse{\isempty{#7}}%
		{\def\linestyle{aux line thin}}%
		{\def\linestyle{aux line thick}}

	\pgfmathsetmacro\momentxendx{#1 + 0}
	\pgfmathsetmacro\momentxendy{#2 - \momentx}
	\pgfmathsetmacro\momentxendz{#3 + 0}

	\pgfmathsetmacro\momentyendx{#1 + \momenty}
	\pgfmathsetmacro\momentyendy{#2 + 0}
	\pgfmathsetmacro\momentyendz{#3 + 0}

	\pgfmathsetmacro\momentzendx{#1 + \twistingmomentmultiplierx * ( \twistingmomentepureoffset + abs(\momentz) )}
	\pgfmathsetmacro\momentzendy{#2 + \twistingmomentmultipliery * ( \twistingmomentepureoffset + abs(\momentz) )}
	\pgfmathsetmacro\momentzendz{#3 + \twistingmomentmultiplierz * ( \twistingmomentepureoffset + abs(\momentz) )}

	\pgfmathsetmacro\momentzmidx{#1 + \twistingmomentmultiplierx * ( \twistingmomentepureoffset + .5*abs(\momentz) )}
	\pgfmathsetmacro\momentzmidy{#2 + \twistingmomentmultipliery * ( \twistingmomentepureoffset + .5*abs(\momentz) )}
	\pgfmathsetmacro\momentzmidz{#3 + \twistingmomentmultiplierz * ( \twistingmomentepureoffset + .5*abs(\momentz) )}

	% bending
	\draw [ \linestyle, color=#4, tdplot_main_coords ]
		( #1, #2, #3 ) -- ( \momentxendx, \momentxendy, \momentxendz ) ;
	\draw [ \linestyle, color=#5, tdplot_main_coords ]
		( #1, #2, #3 ) -- ( \momentyendx, \momentyendy, \momentyendz ) ;

	% twisting
	\pgfmathparse{(abs(\momentz) > 0) ? 1 : 0}\ifdim\pgfmathresult pt>0pt%
		\ifthenelse{\isempty{#7}}%
		{}{%
		\draw [ aux thick dashed, color=#6, tdplot_main_coords ]
			plot [ domain=0:360, variable=\t, samples=60 ]
			( {\momentzmidx + .5*\momentz*cos(\t)}, {\momentzmidy + .5*\momentz*sin(\t)}, {\momentzmidz + 0} ) ;
		}
	\fi
}

% #1,#2,#3 = variables for maximum moments
\newcommand\resetmaxmoments[3]{
	\pgfmathsetmacro{#1}{0}
	\pgfmathsetmacro{#2}{0}
	\pgfmathsetmacro{#3}{0}
}

% #1,#2,#3 = variables for maximum moments
% uses pre-calculated \momentx, \momenty, \momentz
\newcommand\updatemaxmoments[3]{
	\pgfmathparse{( abs(\momentx) > abs(#1) ) ? 1 : 0}%
	\ifdim\pgfmathresult pt>0pt%
		\pgfmathsetmacro{#1}{\momentx}
	\fi

	\pgfmathparse{( abs(\momenty) > abs(#2) ) ? 1 : 0}%
	\ifdim\pgfmathresult pt>0pt%
		\pgfmathsetmacro{#2}{\momenty}
	\fi

	\pgfmathparse{( abs(\momentz) > abs(#3) ) ? 1 : 0}%
	\ifdim\pgfmathresult pt>0pt%
		\pgfmathsetmacro{#3}{\momentz}
	\fi
}

\def\saveepureendpoints{
	\pgfmathsetmacro\savedmomentxendx{\momentxendx}
	\pgfmathsetmacro\savedmomentxendy{\momentxendy}
	\pgfmathsetmacro\savedmomentxendz{\momentxendz}

	\pgfmathsetmacro\savedmomentyendx{\momentyendx}
	\pgfmathsetmacro\savedmomentyendy{\momentyendy}
	\pgfmathsetmacro\savedmomentyendz{\momentyendz}

	\pgfmathsetmacro\savedmomentzendx{\momentzendx}
	\pgfmathsetmacro\savedmomentzendy{\momentzendy}
	\pgfmathsetmacro\savedmomentzendz{\momentzendz}
}

\newcommandx*\drawlinebetweensavedandcurrentz[1][1=\epurecolor]{
	\draw [ aux line thick, color=#1, tdplot_main_coords ]
		( \savedmomentzendx, \savedmomentzendy, \savedmomentzendz )
		-- ( \momentzendx, \momentzendy, \momentzendz ) ;
}

\newcommandx*\drawlinebetweensavedandcurrenty[1][1=\epurecolor]{
	\draw [ aux line thick, color=#1, tdplot_main_coords ]
		( \savedmomentyendx, \savedmomentyendy, \savedmomentyendz )
		-- ( \momentyendx, \momentyendy, \momentyendz ) ;
}

\newcommandx*\drawlinebetweensavedandcurrentx[1][1=\epurecolor]{
	\draw [ aux line thick, color=#1, tdplot_main_coords ]
		( \savedmomentxendx, \savedmomentxendy, \savedmomentxendz )
		-- ( \momentxendx, \momentxendy, \momentxendz ) ;
}

% #1 = begin x, #2 = end x
% #3, #4 = y, z
% #5 = force vector as {fx}{fy}{fz}
% #6 = force point as {x}{y}{z}
% #7, #8, #9 = colorx, colory, colorz
\newcommand\drawepuresofmomentsalongxbetween[9]{
	\pgfmathsetmacro\xfrom{#1}
	\pgfmathsetmacro\xto{#2}
	\pgfmathsetmacro\xstep{\hatchstep * ( \xto - \xfrom ) / abs( \xto - \xfrom )}
	\pgfmathsetmacro\xnext{\xfrom + \xstep}
	\pgfmathsetmacro\xnextnext{\xnext + \xstep}
}

% #1 = begin y, #2 = end y
% #3, #4 = x, z
% #5 = force vector as {fx}{fy}{fz}
% #6 = force point as {x}{y}{z}
% #7, #8, #9 = colorx, colory, colorz
\newcommand\drawepuresofmomentsalongybetween[9]{
	\pgfmathsetmacro\yfrom{#1}
	\pgfmathsetmacro\yto{#2}
	\pgfmathsetmacro\ystep{\hatchstep * ( \yto - \yfrom ) / abs( \yto - \yfrom )}
	\pgfmathsetmacro\ynext{\yfrom + \ystep}
	\pgfmathsetmacro\ynextnext{\ynext + \ystep}

	\foreach \yline in { \ynext, \ynextnext, ..., \yto } {
		\resetmaxmoments{\maxmomentx}{\maxmomenty}{\maxmomentz}

		\pgfmathsetmacro\ybefore{\yline - \ystep}
		\calculateinternalmomentatpointfromforce{#3}{\ybefore}{#4}#6#5
		\drawepurelinesatpointalongy{#3}{\ybefore}{#4}{#7}{#8}{#9}
		\updatemaxmoments{\maxmomentx}{\maxmomenty}{\maxmomentz}
		\saveepureendpoints

		\calculateinternalmomentatpointfromforce{#3}{\yline}{#4}#6#5
		\drawepurelinesatpointalongy{#3}{\yline}{#4}{#7}{#8}{#9}
		\updatemaxmoments{\maxmomentx}{\maxmomenty}{\maxmomentz}

		\pgfmathparse{(abs(\maxmomentx) > 0) ? 1 : 0}%
		\ifdim\pgfmathresult pt>0pt%
			\drawlinebetweensavedandcurrentx[#7]
		\fi
		\pgfmathparse{(abs(\maxmomentz) > 0) ? 1 : 0}%
		\ifdim\pgfmathresult pt>0pt%
			\drawlinebetweensavedandcurrentz[#9]
		\fi
	}

	\pgfmathsetmacro\spiralishere{0} % 0: false, 1: true

	\calculateinternalmomentatpointfromforce{#3}{\yfrom}{#4}#6#5
	\drawepurelinesatpointalongy{#3}{\yfrom}{#4}{#7}{#8}{#9}[thick]

	\pgfmathparse{(abs(\momenty) > 0) ? 1 : 0}%
	\ifdim\pgfmathresult pt>0pt%		
		\drawtwistingoffsetlineat{#3}{\yfrom}{#4}[#8]
		\pgfmathsetmacro\spiralishere{1}
	\fi

	\calculateinternalmomentatpointfromforce{#3}{\yto}{#4}#6#5
	\drawepurelinesatpointalongy{#3}{\yto}{#4}{#7}{#8}{#9}[thick]

	\pgfmathparse{(abs(\momenty) > 0) ? 1 : 0}%
	\ifdim\pgfmathresult pt>0pt%
		\drawtwistingoffsetlineat{#3}{\yto}{#4}[#8]
		\pgfmathsetmacro\spiralishere{1}
	\fi

	\ifdim\spiralishere pt>0pt%
		\drawtwistingspiralalongybetween{#1}{#2}{#3}{#4}[#8]
	\fi
}

% #1 = begin z, #2 = end z
% #3, #4 = x, y
% #5 = force vector as {fx}{fy}{fz}
% #6 = force point as {x}{y}{z}
% #7, #8, #9 = colorx, colory, colorz
\newcommand\drawepuresofmomentsalongzbetween[9]{
	\pgfmathsetmacro\zfrom{#1}
	\pgfmathsetmacro\zto{#2}
	\pgfmathsetmacro\zstep{\hatchstep * ( \zto - \zfrom ) / abs( \zto - \zfrom )}
	\pgfmathsetmacro\znext{\zfrom + \zstep}
	\pgfmathsetmacro\znextnext{\znext + \zstep}
}

\newcommand\drawepureofinternalmomentfromfirstforce{
	% override it
}

\newcommand\drawepureofinternalmomentfromsecondforce{
	% override it
}

% #1: color of first force
% #2: color of second force
% #3: color of sum
\newcommand\drawepureofinternalmomentfrombothforces[3]{
	% override it
}


\newlength{\savedparindent}
\setlength{\savedparindent}{\parindent}

\def\forcearrowscale{4}

\thispagestyle{empty} % no numbering for 1st page

\begin{comment}
\begin{center}
\vspace*{\fill}

{\large \textbf{Домашнее задание №\hspace{.33ex}3}}

\vspace{1cm}

{\large \MakeUppercase{Сложное напряжённое состояние}}

\vspace{7cm}
\vspace*{\fill}
\end{center}
\end{comment}

\def\tabletextscale{.88}
\newcommand\tabletext[1]{\scalebox{\tabletextscale}{#1}}

\begin{tikzpicture}[ remember picture, overlay, shift=(current page.south east), shift={( -.5cm, .5cm )}, rotate=180 ]
\begin{scope}[ x={(current page.north east)}, y={(current page.south west)} ]

\draw [ aux line thick ] ( 0, 0 ) -- ( 12cm, 0 ) -- ( 12cm, -4.5cm ) -- ( 0, -4.5cm ) -- cycle ;

\draw [ aux line thick, shift={( 12cm, -4.5cm )} ] ( 0, 0 ) rectangle ( -2.5cm, 1cm ) node [ midway ] { \tabletext{СМ} } ;
\draw [ aux line thick, shift={( 12cm, -4.5cm )}, shift={( -2.5cm, 0 )} ] ( 0, 0 ) rectangle ( -2.5cm, 1cm ) node [ midway ] { \tabletext{СМ7--31Б} } ;
\draw [ aux line thick, shift={( 12cm, -4.5cm )}, shift={( -2.5cm, 0 )}, shift={( -2.5cm, 0 )} ] ( 0, 0 ) rectangle ( -1.5cm, 1cm ) node [ midway ] { \tabletext{№\hspace{.25ex}3} } ;
\draw [ aux line thick, shift={( 12cm, -4.5cm )}, shift={( -2.5cm, 0 )}, shift={( -2.5cm, 0 )}, shift={( -1.5cm, 0 )} ] ( 0, 0 ) rectangle ( -1.5cm, 1cm ) node [ midway ] { \tabletext{1} } ;

\draw [ aux line thick, shift={( 0, -4.5cm )} ] ( 0, 0 ) rectangle ( 4cm, 1cm ) node [ midway ] { \tabletext{\href{https://repetitors.info/repetitor/?p=NizametdinovFR}{Низаметдинов Ф.\:Р.}} } ; % Низаметдинов Фярит Ринатович

\draw [ aux line thick, shift={( 12cm, -3.5cm )} ] ( 0, 0 ) rectangle ( -2.5cm, 1cm ) node [ midway ] { \tabletext{МГТУ} } ;

\draw [ aux line thick, shift={( 0, -3.5cm )} ] ( 0, 0 ) rectangle ( 9.5cm, 1cm ) node [ midway ] { \tabletext{\href{http://rk5.bmstu.ru/}{Сопротивление материалов}} } ;

\draw [ aux line thick, shift={( 0, -2.5cm )} ] ( 0, 0 ) rectangle ( 12cm, 1.5cm ) node [ midway ] { \tabletext{\MakeUppercase{Сложное напряжённое состояние}} } ;

\draw [ aux line thick, shift={( 12cm, 0 )} ] ( 0, 0 ) rectangle ( -6cm, -1cm ) node [ midway ] { \tabletext{\href{https://vk.com/maximdenisov.chvrches}{Денисов М.\:А.}} } ;

\draw [ aux line thick ] ( 0, 0 ) rectangle ( 6cm, -1cm ) node [ midway ] { \tabletext{Вариант 8} } ;

\end{scope}
\end{tikzpicture}

\newpage


\pgfmathsetmacro\firstlength{3*\reflength}
\pgfmathsetmacro\secondlength{\reflength}
\pgfmathsetmacro\thirdlength{2*\reflength}
\pgfmathsetmacro\fourthlength{\reflength}

\pgfmathsetmacro\xfirst{\firstlength}
\pgfmathsetmacro\ysecond{\secondlength}
\pgfmathsetmacro\deltaxthird{-\thirdlength}
\pgfmathsetmacro\xthird{\xfirst + \deltaxthird}
\pgfmathsetmacro\zfourth{-\fourthlength}

\newcommand\drawfirstpartofbeam{
	\draw [ beam line, tdplot_main_coords ]
		( 0, 0, 0 ) -- ++( \xfirst, 0, 0 ) ;
}

\newcommand\drawsecondpartofbeam{
	\draw [ beam line, tdplot_main_coords ]
		($ ( 0, 0, 0 ) + ( \xfirst, 0, 0 ) $) -- ++( 0, \ysecond, 0 ) ;

	% corners

	\pgfmathsetmacro\cornerysign{\ysecond / abs(\ysecond)}

	\draw [ beam line, fill=black, tdplot_main_coords ]
		($ ( 0, 0, 0 ) + ( \xfirst, 0, 0 ) $)
		-- ++( -\cornerlength, 0, 0 )
		-- ++( \cornerlength, {\cornerysign*\cornerlength}, 0 )
		-- cycle ;
}

\newcommand\drawthirdpartofbeam{
	\draw [ beam line, tdplot_main_coords ]
		($ ( 0, 0, 0 ) + ( \xfirst, \ysecond, 0 ) $) -- ( \xthird, \ysecond, 0 ) ;

	% corners

	\pgfmathsetmacro\cornerysign{\ysecond / abs(\ysecond)}
	\pgfmathsetmacro\cornerxsign{\deltaxthird / abs(\deltaxthird)}

	\draw [ beam line, fill=black, tdplot_main_coords ]
		($ ( 0, 0, 0 ) + ( \xfirst, \ysecond, 0 ) $)
		-- ++( 0, {-1*\cornerysign*\cornerlength}, 0 )
		-- ++( {\cornerxsign*\cornerlength}, {\cornerysign*\cornerlength}, 0 )
		-- cycle ;
}

\newcommand\drawfourthpartofbeam{
	\draw [ beam line, tdplot_main_coords ]
		($ ( 0, 0, 0 ) + ( \xthird, \ysecond, 0 ) $) -- ++( 0, 0, \zfourth ) ;

	% corners

	\pgfmathsetmacro\cornerzsign{\zfourth / abs(\zfourth)}

	\draw [ beam line, fill=black, tdplot_main_coords ]
		($ ( 0, 0, 0 ) + ( \xthird, \ysecond, 0 ) $)
		-- ++( \cornerlength, 0, 0 )
		-- ++( -\cornerlength, 0, {\cornerzsign*\cornerlength} )
		-- cycle ;
}

\newcommand\drawfirstbeam{
	\drawfirstpartofbeam
	\drawsecondpartofbeam
	\drawthirdpartofbeam
	\drawfourthpartofbeam
}

\newcommand\drawtextforfirstpartofbeam{
	\pgfmathsetmacro\firstpartlengthmultiplier{\firstlength / \reflength}

	\node [ above, shape=circle, inner sep=0pt, outer sep=8pt, tdplot_main_coords ]
		at ($ ( 0, 0, 0 ) + ( .53*\xfirst, 0, 0 ) $)
	{\scalebox{\textscale}{$
\pgfmathparse{(\firstpartlengthmultiplier == 1) ? 0 : 1}\ifdim\pgfmathresult pt>0pt%
	\pgfmathprintnumber[ precision=3, fixed, zerofill=false ]\firstpartlengthmultiplier\hspace{.1ex}
\fi
\ell
	$}} ;
}

\newcommand\drawtextforsecondpartofbeam{
	\pgfmathsetmacro\secondpartlengthmultiplier{\secondlength / \reflength}

	\node [ below, shape=circle, inner sep=0pt, outer sep=11pt, tdplot_main_coords ]
		at ($ ( 0, 0, 0 ) + ( \xfirst, .45*\ysecond, 0 ) $)
	{\scalebox{\textscale}{$
\pgfmathparse{(\secondpartlengthmultiplier == 1) ? 0 : 1}\ifdim\pgfmathresult pt>0pt%
	\pgfmathprintnumber[ precision=3, fixed, zerofill=false ]\secondpartlengthmultiplier\hspace{.1ex}
\fi
\ell
	$}} ;
}

\newcommand\drawtextforthirdpartofbeam{
	\pgfmathsetmacro\thirdpartlengthmultiplier{\thirdlength / \reflength}

	\node [ below, shape=circle, inner sep=0pt, outer sep=9pt, tdplot_main_coords ]
		at ($ ( 0, 0, 0 ) + ( \xfirst, \ysecond, 0 ) + ( .5*\deltaxthird, 0, 0 ) $)
	{\scalebox{\textscale}{$
\pgfmathparse{(\thirdpartlengthmultiplier == 1) ? 0 : 1}\ifdim\pgfmathresult pt>0pt%
	\pgfmathprintnumber[ precision=3, fixed, zerofill=false ]\thirdpartlengthmultiplier\hspace{.1ex}
\fi
\ell
	$}} ;
}

\newcommand\drawtextforfourthpartofbeam{
	\pgfmathsetmacro\fourthpartlengthmultiplier{\fourthlength / \reflength}

	\node [ right, shape=circle, inner sep=0pt, outer sep=4pt, tdplot_main_coords ]
		at ($ ( 0, 0, 0 ) + ( \xthird, \ysecond, .5*\zfourth ) $)
	{\scalebox{\textscale}{$
\pgfmathparse{(\fourthpartlengthmultiplier == 1) ? 0 : 1}\ifdim\pgfmathresult pt>0pt%
	\pgfmathprintnumber[ precision=3, fixed, zerofill=false ]\fourthpartlengthmultiplier\hspace{.1ex}
\fi
\ell
	$}} ;
}

\newcommand\drawfirstbeamtext{
	\drawtextforfirstpartofbeam
	\drawtextforsecondpartofbeam
	\drawtextforthirdpartofbeam
	\drawtextforfourthpartofbeam
}

\def\firstforcepointx{ \xthird }
\def\firstforcepointy{ \ysecond }
\def\firstforcepointz{ \zfourth }
\def\firstforcevectorx{ \justP }
\def\firstforcevectory{ 0 }
\def\firstforcevectorz{ 0 }

\def\secondforcepointx{ \xfirst }
\def\secondforcepointy{ \ysecond }
\def\secondforcepointz{ 0 }
\def\secondforcevectorx{ 0 }
\def\secondforcevectory{ 0 }
\def\secondforcevectorz{ -\justP }

\renewcommand\drawepureofinternalmomentfromfirstforce{
	\def\epurecolor{\externalforcecolor}

	% along x for first part

	\pgfmathsetmacro\xmax{\xfirst}
	\foreach \xhatch in { 0, \hatchstep, ..., \xmax } {
		\calculateinternalmomentatpointfromforcename{\xhatch}{0}{0}%
			{firstforcepoint}{firstforcevector}

		\drawepurelinesatpointalongx{\xhatch}{0}{0}%
			{\epurecolor}{\epurecolor}{\epurecolor}
	}

	\calculateinternalmomentatpointfromforcename{\xfirst}{0}{0}%
		{firstforcepoint}{firstforcevector}
	\drawepurelinesatpointalongx{\xfirst}{0}{0}%
		{\epurecolor}{\epurecolor}{\epurecolor}%
		[thick]

	\pgfmathsetmacro\xzeromomentzFlmultiplier{\momentz / \justP / \reflength}	
	\pgfmathsetmacro\absxzeromomentzFlmultiplier{abs(\xzeromomentzFlmultiplier)}
	\pgfmathprintnumberto[ precision=3, fixed, zerofill=false ]{\absxzeromomentzFlmultiplier}{\absxzeromomentzFlroundedmultiplier}

	\node [ left, color=\epurecolor, shape=circle, inner sep=0pt, outer sep=16pt, tdplot_main_coords ]
		at ($ ( \xfirst, 0, 0 ) + ( 0, .1*\momentzendy, 0 ) $)
	{\scalebox{\textscale}{$
\pgfmathparse{(\absxzeromomentzFlroundedmultiplier == 1) ? 0 : 1}\ifdim\pgfmathresult pt>0pt%
	\absxzeromomentzFlroundedmultiplier\hspace{.1ex}
\fi
F \hspace{-0.1ex} \ell
	$}} ;

	\saveepureendpoints

	\calculateinternalmomentatpointfromforcename{0}{0}{0}%
		{firstforcepoint}{firstforcevector}
	\drawepurelinesatpointalongx{0}{0}{0}%
		{\epurecolor}{\epurecolor}{\epurecolor}%
		[thick]

	\drawlinebetweensavedandcurrentz
	\drawlinebetweensavedandcurrenty

	\pgfmathsetmacro\xzeromomentyFlmultiplier{\momenty / \justP / \reflength}	
	\pgfmathsetmacro\absxzeromomentyFlmultiplier{abs(\xzeromomentyFlmultiplier)}
	\pgfmathprintnumberto[ precision=3, fixed, zerofill=false ]{\absxzeromomentyFlmultiplier}{\absxzeromomentyFlroundedmultiplier}

	\node [ right, color=\epurecolor, shape=circle, inner sep=0pt, outer sep=3pt, tdplot_main_coords ]
		at ($ ( 0, 0, 0 ) + ( 0, 0, .53*\momentyendz ) $)
	{\scalebox{\textscale}{$
\pgfmathparse{(\absxzeromomentyFlroundedmultiplier == 1) ? 0 : 1}\ifdim\pgfmathresult pt>0pt%
	\absxzeromomentyFlroundedmultiplier\hspace{.1ex}
\fi
F \hspace{-0.1ex} \ell
	$}} ;

	% along y for second part

	\drawepuresofmomentsalongybetween{\ysecond}{0}{\xfirst}{0}%
		{firstforcevector}{firstforcepoint}%
		{\epurecolor}{\epurecolor}{\epurecolor}

	% along x for third part

	\pgfmathsetmacro\xfrom{\xfirst}
	\pgfmathsetmacro\xto{\xthird}
	\pgfmathsetmacro\xstep{\hatchstep * ( \xto - \xfrom ) / abs( \xto - \xfrom )}
	\pgfmathsetmacro\xnext{\xfrom + \xstep}
	\foreach \xhatch in { \xfrom, \xnext, ..., \xto } {
		\calculateinternalmomentatpointfromforcename{\xhatch}{\ysecond}{0}%
			{firstforcepoint}{firstforcevector}

		\drawepurelinesatpointalongx{\xhatch}{\ysecond}{0}%
			{\epurecolor}{\epurecolor}{\epurecolor}
	}

	\calculateinternalmomentatpointfromforcename{\xfirst}{\ysecond}{0}%
		{firstforcepoint}{firstforcevector}
	\drawepurelinesatpointalongx{\xfirst}{\ysecond}{0}%
		{\epurecolor}{\epurecolor}{\epurecolor}%
		[thick]

	\saveepureendpoints

	\calculateinternalmomentatpointfromforcename{\xthird}{\ysecond}{0}%
		{firstforcepoint}{firstforcevector}
	\drawepurelinesatpointalongx{\xthird}{\ysecond}{0}%
		{\epurecolor}{\epurecolor}{\epurecolor}%
		[thick]

	%%\drawlinebetweensavedandcurrentz
	\drawlinebetweensavedandcurrenty

	% along z for fourth part

	\pgfmathsetmacro\zfrom{0}
	\pgfmathsetmacro\zto{\zfourth}
	\pgfmathsetmacro\zstep{\hatchstep * ( \zto - \zfrom ) / abs( \zto - \zfrom )}
	\pgfmathsetmacro\znext{\zfrom + \zstep}
	\foreach \zhatch in { \zfrom, \znext, ..., \zto } {
		\calculateinternalmomentatpointfromforcename{\xthird}{\ysecond}{\zhatch}%
			{firstforcepoint}{firstforcevector}

		\drawepurelinesatpointalongz{\xthird}{\ysecond}{\zhatch}%
			{\epurecolor}{\epurecolor}{\epurecolor}
	}

	\calculateinternalmomentatpointfromforcename{\xthird}{\ysecond}{0}%
		{firstforcepoint}{firstforcevector}
	\drawepurelinesatpointalongz{\xthird}{\ysecond}{0}%
		{\epurecolor}{\epurecolor}{\epurecolor}%
		[thick]

	\saveepureendpoints

	\calculateinternalmomentatpointfromforcename{\xthird}{\ysecond}{\zfourth}%
		{firstforcepoint}{firstforcevector}
	\drawepurelinesatpointalongz{\xthird}{\ysecond}{\zfourth}%
		{\epurecolor}{\epurecolor}{\epurecolor}%
		[thick]

	\drawlinebetweensavedandcurrenty
}

\renewcommand\drawepureofinternalmomentfromsecondforce{
	\def\epurecolor{\externalforcecolor}

	% along x for first part

	\pgfmathsetmacro\xmax{\xfirst}
	\foreach \xhatch in { 0, \hatchstep, ..., \xmax } {
		\calculateinternalmomentatpointfromforcename{\xhatch}{0}{0}%
			{secondforcepoint}{secondforcevector}

		\drawepurelinesatpointalongx{\xhatch}{0}{0}%
			{\epurecolor}{\epurecolor}{\epurecolor}
	}

	\drawtwistingoffsetlineat{0}{0}{0}
	\drawtwistingoffsetlineat{\xfirst}{0}{0}

	\calculateinternalmomentatpointfromforcename{\xfirst}{0}{0}%
		{secondforcepoint}{secondforcevector}
	\drawepurelinesatpointalongx{\xfirst}{0}{0}%
		{\epurecolor}{\epurecolor}{\epurecolor}%
		[thick]

	\saveepureendpoints

	\calculateinternalmomentatpointfromforcename{0}{0}{0}%
		{secondforcepoint}{secondforcevector}
	\drawepurelinesatpointalongx{0}{0}{0}%
		{\epurecolor}{\epurecolor}{\epurecolor}%
		[thick]

	%%\drawlinebetweensavedandcurrentz
	\drawlinebetweensavedandcurrenty

	\drawtwistingspiralalongxbetween{\xfirst}{0}{0}{0}

	\pgfmathsetmacro\xzeromomentyFlmultiplier{\momenty / \justP / \reflength}	
	\pgfmathsetmacro\absxzeromomentyFlmultiplier{abs(\xzeromomentyFlmultiplier)}
	\pgfmathprintnumberto[ precision=3, fixed, zerofill=false ]{\absxzeromomentyFlmultiplier}{\absxzeromomentyFlroundedmultiplier}

	\node [ right, color=\epurecolor, shape=circle, inner sep=0pt, outer sep=4pt, tdplot_main_coords ]
		at ($ ( 0, 0, 0 ) + ( 0, 0, .54*\momentyendz ) $)
	{\scalebox{\textscale}{$
\pgfmathparse{(\absxzeromomentyFlroundedmultiplier == 1) ? 0 : 1}\ifdim\pgfmathresult pt>0pt%
	\absxzeromomentyFlroundedmultiplier\hspace{.1ex}
\fi
F \hspace{-0.1ex} \ell
	$}} ;

	% along y for second part

	\drawepuresofmomentsalongybetween{\ysecond}{0}{\xfirst}{0}%
		{secondforcevector}{secondforcepoint}%
		{\epurecolor}{\epurecolor}{\epurecolor}
}

% #1: color of first force
% #2: color of second force
% #3: color of sum
\renewcommand\drawepureofinternalmomentfrombothforces[3]{
	% along x for first part

	\def\twistingmomentmultiplierx{0}
	\def\twistingmomentmultipliery{0}
	\def\twistingmomentmultiplierz{1}
	\def\twistingmomentepureoffset{.5*\reflength}

	\def\epurecolor{#3}

	\pgfmathsetmacro\xmax{\xfirst}
	\foreach \xhatch in { 0, \hatchstep, ..., \xmax } {
		\calculateinternalmomentatpointfromforcename{\xhatch}{0}{0}%
			{firstforcepoint}{firstforcevector}

		\pgfmathsetmacro\firstmomentx{\momentx}
		\pgfmathsetmacro\firstmomenty{\momenty}
		\pgfmathsetmacro\firstmomentz{\momentz}

		\calculateinternalmomentatpointfromforcename{\xhatch}{0}{0}%
			{secondforcepoint}{secondforcevector}

		\pgfmathsetmacro\secondmomentx{\momentx}
		\pgfmathsetmacro\secondmomenty{\momenty}
		\pgfmathsetmacro\secondmomentz{\momentz}

		\pgfmathsetmacro\momentx{\firstmomentx + \secondmomentx}
		\pgfmathsetmacro\momenty{\firstmomenty + \secondmomenty}
		\pgfmathsetmacro\momentz{\firstmomentz + \secondmomentz}

		\drawepurelinesatpointalongx{\xhatch}{0}{0}%
			{#2}{#3}{#1}
	}

	\calculateinternalmomentatpointfromforcename{\xfirst}{0}{0}%
		{firstforcepoint}{firstforcevector}

	\pgfmathsetmacro\firstmomentx{\momentx}
	\pgfmathsetmacro\firstmomenty{\momenty}
	\pgfmathsetmacro\firstmomentz{\momentz}

	\calculateinternalmomentatpointfromforcename{\xfirst}{0}{0}%
		{secondforcepoint}{secondforcevector}

	\pgfmathsetmacro\momentx{\momentx + \firstmomentx}
	\pgfmathsetmacro\momenty{\momenty + \firstmomenty}
	\pgfmathsetmacro\momentz{\momentz + \firstmomentz}

	\drawepurelinesatpointalongx{\xfirst}{0}{0}%
		{#2}{#3}{#1}%
		[thick]

	\saveepureendpoints

	\calculateinternalmomentatpointfromforcename{0}{0}{0}%
		{firstforcepoint}{firstforcevector}

	\pgfmathsetmacro\firstmomentx{\momentx}
	\pgfmathsetmacro\firstmomenty{\momenty}
	\pgfmathsetmacro\firstmomentz{\momentz}

	\calculateinternalmomentatpointfromforcename{0}{0}{0}%
		{secondforcepoint}{secondforcevector}

	\pgfmathsetmacro\momentx{\momentx + \firstmomentx}
	\pgfmathsetmacro\momenty{\momenty + \firstmomenty}
	\pgfmathsetmacro\momentz{\momentz + \firstmomentz}

	\drawepurelinesatpointalongx{0}{0}{0}%
		{#2}{#3}{#1}%
		[thick]

	\drawlinebetweensavedandcurrentz[#1]
	\drawlinebetweensavedandcurrenty[#3]

	\pgfmathsetmacro\xzeromomentyFlmultiplier{\momenty / \justP / \reflength}	
	\pgfmathsetmacro\absxzeromomentyFlmultiplier{abs(\xzeromomentyFlmultiplier)}
	\pgfmathprintnumberto[ precision=3, fixed, zerofill=false ]{\absxzeromomentyFlmultiplier}{\absxzeromomentyFlroundedmultiplier}

	\node [ right, color=\epurecolor, shape=circle, inner sep=0pt, outer sep=4pt, tdplot_main_coords ]
		at ($ ( 0, 0, 0 ) + ( 0, 0, .52*\momentyendz ) $)
	{\scalebox{\textscale}{$
\pgfmathparse{(\absxzeromomentyFlroundedmultiplier == 1) ? 0 : 1}\ifdim\pgfmathresult pt>0pt%
	\absxzeromomentyFlroundedmultiplier\hspace{.1ex}
\fi
F \hspace{-0.1ex} \ell
	$}} ;

	\pgfmathsetmacro\xzeromomentzFlmultiplier{\momentz / \justP / \reflength}	
	\pgfmathsetmacro\absxzeromomentzFlmultiplier{abs(\xzeromomentzFlmultiplier)}
	\pgfmathprintnumberto[ precision=3, fixed, zerofill=false ]{\absxzeromomentzFlmultiplier}{\absxzeromomentzFlroundedmultiplier}

	\node [ right, color=#1, shape=circle, inner sep=0pt, outer sep=14pt, tdplot_main_coords ]
		at ($ ( 0, 0, 0 ) + ( 0, 1.4*\momentzendy, 0 ) $)
	{\scalebox{\textscale}{$
\pgfmathparse{(\absxzeromomentzFlroundedmultiplier == 1) ? 0 : 1}\ifdim\pgfmathresult pt>0pt%
	\absxzeromomentzFlroundedmultiplier\hspace{.1ex}
\fi
F \hspace{-0.1ex} \ell
	$}} ;

	\def\epurecolor{#2}

	\def\spiralaxialstep{1.1}
	\def\spiralinitialangle{-160}
	\def\wherefirstarrowonspiral{0.05}

	\drawtwistingoffsetlineat{0}{0}{0}
	\drawtwistingoffsetlineat{\xfirst}{0}{0}
	\drawtwistingspiralalongxbetween{\xfirst}{0}{0}{0}
	
	\pgfmathsetmacro\xzeromomentxFlmultiplier{\momentx / \justP / \reflength}
	\pgfmathsetmacro\absxzeromomentxFlmultiplier{abs(\xzeromomentxFlmultiplier)}
	\pgfmathprintnumberto[ precision=3, fixed, zerofill=false ]{\absxzeromomentxFlmultiplier}{\absxzeromomentxFlroundedmultiplier}

	\node [ above, color=\epurecolor, shape=circle, inner sep=0pt, outer sep=9pt, tdplot_main_coords ]
		at ($ ( 0, 0, 0 ) + ( .5*\xfirst, 0, 0 ) + ( \momentxendx, \momentxendy, \momentxendz ) $)
	{\scalebox{\textscale}{$
\pgfmathparse{(\absxzeromomentxFlroundedmultiplier == 1) ? 0 : 1}\ifdim\pgfmathresult pt>0pt%
	\absxzeromomentxFlroundedmultiplier\hspace{.1ex}
\fi
F \hspace{-0.1ex} \ell
	$}} ;

	% along y for second part

	\def\epurecolor{#2}

	\drawepuresofmomentsalongybetween{\ysecond}{0}{\xfirst}{0}%
		{secondforcevector}{secondforcepoint}%
		{\epurecolor}{\epurecolor}{\epurecolor}

	\def\epurecolor{#1}

	\def\spiralaxialstep{1.25}
	\def\spiralinitialangle{123}
	\def\wherefirstarrowonspiral{0.07}
	\pgfmathsetmacro\twistingmomentmultiplierz{-\twistingmomentmultiplierz}

	\drawepuresofmomentsalongybetween{\ysecond}{0}{\xfirst}{0}%
		{firstforcevector}{firstforcepoint}%
		{\epurecolor}{\epurecolor}{\epurecolor}

	% along x for third part

	\def\epurecolor{#1}

	\pgfmathsetmacro\xfrom{\xfirst}
	\pgfmathsetmacro\xto{\xthird}
	\pgfmathsetmacro\xstep{\hatchstep * ( \xto - \xfrom ) / abs( \xto - \xfrom )}
	\pgfmathsetmacro\xnext{\xfrom + \xstep}
	\foreach \xhatch in { \xfrom, \xnext, ..., \xto } {
		\calculateinternalmomentatpointfromforcename{\xhatch}{\ysecond}{0}%
			{firstforcepoint}{firstforcevector}

		\drawepurelinesatpointalongx{\xhatch}{\ysecond}{0}%
			{\epurecolor}{\epurecolor}{\epurecolor}
	}

	\calculateinternalmomentatpointfromforcename{\xfirst}{\ysecond}{0}%
		{firstforcepoint}{firstforcevector}
	\drawepurelinesatpointalongx{\xfirst}{\ysecond}{0}%
		{\epurecolor}{\epurecolor}{\epurecolor}%
		[thick]

	\saveepureendpoints

	\calculateinternalmomentatpointfromforcename{\xthird}{\ysecond}{0}%
		{firstforcepoint}{firstforcevector}
	\drawepurelinesatpointalongx{\xthird}{\ysecond}{0}%
		{\epurecolor}{\epurecolor}{\epurecolor}%
		[thick]

	%%\drawlinebetweensavedandcurrentz
	\drawlinebetweensavedandcurrenty

	% along z for fourth part

	\def\epurecolor{#1}

	\pgfmathsetmacro\zfrom{0}
	\pgfmathsetmacro\zto{\zfourth}
	\pgfmathsetmacro\zstep{\hatchstep * ( \zto - \zfrom ) / abs( \zto - \zfrom )}
	\pgfmathsetmacro\znext{\zfrom + \zstep}
	\foreach \zhatch in { \zfrom, \znext, ..., \zto } {
		\calculateinternalmomentatpointfromforcename{\xthird}{\ysecond}{\zhatch}%
			{firstforcepoint}{firstforcevector}

		\drawepurelinesatpointalongz{\xthird}{\ysecond}{\zhatch}%
			{\epurecolor}{\epurecolor}{\epurecolor}
	}

	\calculateinternalmomentatpointfromforcename{\xthird}{\ysecond}{0}%
		{firstforcepoint}{firstforcevector}
	\drawepurelinesatpointalongz{\xthird}{\ysecond}{0}%
		{\epurecolor}{\epurecolor}{\epurecolor}%
		[thick]

	\saveepureendpoints

	\calculateinternalmomentatpointfromforcename{\xthird}{\ysecond}{\zfourth}%
		{firstforcepoint}{firstforcevector}
	\drawepurelinesatpointalongz{\xthird}{\ysecond}{\zfourth}%
		{\epurecolor}{\epurecolor}{\epurecolor}%
		[thick]

	\drawlinebetweensavedandcurrenty
}


\tikzstyle{dimension arrow} = [ aux line thin, -{Stealth[ round, length=8pt, width=5.5pt ]} ]
\tikzstyle{dimension two arrows} = [ aux line thin, {Stealth[ round, length=8pt, width=5.5pt ]}-{Stealth[ round, length=8pt, width=5.5pt ]} ]

\tikzstyle{axis with arrow} = [ aux line thin, -{To[round, length=8pt, width=10pt]} ]

% #1, #2, #3 = text for x, y, z axes
\newcommand\drawaxes[3]{
\draw [ axis with arrow, tdplot_main_coords ]
	( 0, 0, 0 ) -- ++( \showcoordinatelength, 0, 0 )
	node [ pos=1, below, shape=circle, inner sep=0pt, outer sep=5.5pt ]
		{\scalebox{\textscale}{$ #1 $}} ;
\draw [ axis with arrow, tdplot_main_coords ]
	( 0, 0, 0 ) -- ++( 0, \showcoordinatelength, 0 )
	node [ pos=1, below, shape=circle, inner sep=0pt, outer sep=6.3pt ]
		{\scalebox{\textscale}{$ #2 $}} ;
\draw [ axis with arrow, tdplot_main_coords ]
	( 0, 0, 0 ) -- ++( 0, 0, .9*\showcoordinatelength )
	node [ pos=1, above, shape=circle, inner sep=0pt, outer sep=3.3pt ]
		{\scalebox{\textscale}{$ #3 $}} ;
}

\newcommand\drawcircularsection{
\draw [ aux line thick, pattern=flexible hatch ]
	( 0, 0 ) circle ( \radiusofcircle ) ;
}

\newcommand\drawcircularsectionthreedimzy{
\draw [ aux line thick, pattern=flexible hatch, tdplot_main_coords ]
	( 0, 0, 0 ) plot [ domain=0:360, variable=\t, samples=60 ]
	( 0, {\radiusofcircle*sin(\t)}, {\radiusofcircle*cos(\t)} ) ;
}

\def\angleofdimension{150}
\def\deltaradiusofdimension{.7}

\newcommand\drawcircularsectiondimensionscustom[1]{
\draw [ aux line thin, opacity=.8 ]
	( {\radiusofcircle * cos(\angleofdimension)}, {\radiusofcircle * sin(\angleofdimension)} )
	-- ( {\radiusofcircle * cos(\angleofdimension + 180)}, {\radiusofcircle * sin(\angleofdimension + 180)} ) ;
\draw [ dimension arrow ]
	( {(\radiusofcircle + \deltaradiusofdimension) * cos(\angleofdimension)}, {(\radiusofcircle + \deltaradiusofdimension) * sin(\angleofdimension)} )
	-- ( {\radiusofcircle * cos(\angleofdimension)}, {\radiusofcircle * sin(\angleofdimension)} ) ;
\draw [ dimension arrow ]
	( {(\radiusofcircle + \deltaradiusofdimension + .22) * cos(\angleofdimension + 180)}, {(\radiusofcircle + \deltaradiusofdimension + .22) * sin(\angleofdimension + 180)} )
	-- ( {\radiusofcircle * cos(\angleofdimension + 180)}, {\radiusofcircle * sin(\angleofdimension + 180)} ) ;
\draw [ aux line thin ]
	( {(\radiusofcircle + \deltaradiusofdimension + .22) * cos(\angleofdimension + 180)}, {(\radiusofcircle + \deltaradiusofdimension + .22) * sin(\angleofdimension + 180)} )
	-- ++( .74, 0 )
	node [ pos=.49, above, inner sep=0pt, outer sep=3pt ]
	{\scalebox{.8}{$ #1 $}} ;
}

\newcommand\drawcircularsectiondimensions{
	\drawcircularsectiondimensionscustom{\scalebox{1.2}{\diameter}\hspace{-.3ex}D}
}

\newcommand\drawrectangularsection{
\draw [ aux line thick, rounded corners=.5\pgflinewidth, pattern=flexible hatch ]
	( -.5*\widthofsquare, .5*\heightofsquare )
	-- ( .5*\widthofsquare, .5*\heightofsquare )
	-- ( .5*\widthofsquare, -.5*\heightofsquare )
	-- ( -.5*\widthofsquare, -.5*\heightofsquare )
	-- cycle ;
}

\newcommand\drawrectangularsectionthreedimzy{
\draw [ aux line thick, rounded corners=.5\pgflinewidth, pattern=flexible hatch, tdplot_main_coords ]
	( 0, -.5*\widthofsquare, .5*\heightofsquare )
	-- ( 0, .5*\widthofsquare, .5*\heightofsquare )
	-- ( 0, .5*\widthofsquare, -.5*\heightofsquare )
	-- ( 0, -.5*\widthofsquare, -.5*\heightofsquare )
	-- cycle ;
}

\def\dimensionlinelength{1}

% #1 : text for width
% #2 : text node options for width
% #3 : text for height
% #4 : text node options for height
% anything in #5 for vertical dimension on left
\newcommandx*\drawrectangularsectiondimensions[5][5=]{
\ifthenelse{\isempty{#5}}%
	{\def\hsign{1}}%
	{\def\hsign{-1}}

\draw [ aux line thin ]
	( .5*\hsign*\widthofsquare, .5*\heightofsquare ) -- ++( \hsign*\dimensionlinelength, 0 ) ;
\draw [ aux line thin ]
	( .5*\hsign*\widthofsquare, -.5*\heightofsquare ) -- ++( \hsign*\dimensionlinelength, 0 ) ;

\draw [ dimension two arrows ]
	( {.5*\hsign*\widthofsquare + .8*\hsign*\dimensionlinelength}, .5*\heightofsquare )
	-- ++( 0, -\heightofsquare )
	node [ #4 ] {\scalebox{.8}{$ #3 $}} ;

\draw [ aux line thin ]
	( .5*\widthofsquare, -.5*\heightofsquare ) -- ++( 0, -\dimensionlinelength ) ;
\draw [ aux line thin ]
	( -.5*\widthofsquare, -.5*\heightofsquare ) -- ++( 0, -\dimensionlinelength ) ;

\draw [ dimension two arrows ]
	( -.5*\widthofsquare, {-.5*\heightofsquare - .8*\dimensionlinelength} )
	-- ++( \widthofsquare, 0 )
	node [ #2 ] {\scalebox{.8}{$ #1 $}} ;
}

\newcommand\drawrectangularsectionwithhorizontaldiagonal{
	\draw [ aux line thick, rounded corners=.5\pgflinewidth, pattern=flexible hatch, rotate around={atan(-2):(0, 0)} ]
	( -.5*\widthofsquare, .5*\heightofsquare )
	-- ( .5*\widthofsquare, .5*\heightofsquare )
	-- ( .5*\widthofsquare, -.5*\heightofsquare )
	-- ( -.5*\widthofsquare, -.5*\heightofsquare )
	-- cycle ;
}

\def\camerafirstangle{57} % 60
\def\camerasecondangle{142} % 140

\tdplotsetmaincoords{\camerafirstangle}{\camerasecondangle}

\begin{center}

\textbf{Задача 1\raisebox{.7ex}{\small я}}
\vspace{.4cm}

Определить размер сечения для~бруса кругового профиля с~диаметром~$D$
и~для~бруса прямоугольного сечения с~размерами ${H \!\times\! B}$\hbox{,\hspace{.2ex}} $B$~--- ширина\hbox{,\hspace{.2ex}} ${H \hspace{-.5ex}=\hspace{-.3ex} 2B}$~--- высота
\vspace{.6cm}

\scalebox{1.25}{
\begin{tikzpicture}[ scale=1 ]

\begin{scope}[xshift=-2.3cm]
\def\radiusofcircle{1}
\drawcircularsection

\def\dimensionlinelength{1}
\def\angleofdimension{158}
\drawcircularsectiondimensions
\end{scope}

\begin{scope}[xshift=2.3cm]
\def\widthofsquare{1.3}
\pgfmathsetmacro\heightofsquare{2*\widthofsquare}

\drawrectangularsection
\drawrectangularsectiondimensions{B}{pos=.5, above, inner sep=0pt, outer sep=4pt}{H}{pos=.51, left, inner sep=0pt, outer sep=3pt}
\end{scope}

\end{tikzpicture}
}

\vspace{.6cm}

${F \hspace{-.5ex}=\hspace{-.4ex} 1\:}$кН\hbox{,\hspace{.5ex}}
${\ell \!=\! 200\:}$мм\hbox{,\hspace{.5ex}}
${\sigma_{\mathrm{\scalebox{.6}{T}p}} \hspace{-.5ex}=\hspace{-.3ex} \sigma_{\mathrm{\scalebox{.6}{T}c}} \hspace{-.5ex}=\hspace{-.3ex} 300\:}$МПа\hbox{,\hspace{.5ex}}
${n_{\mathrm{\scalebox{.6}{T}}} \hspace{-.44ex}=\hspace{-.33ex} 2}$

\vspace{1.2cm}

\emph{Вариант №\hspace{.33ex}8}
\vspace{.8cm}

\scalebox{1}{
\begin{tikzpicture}[ scale=1 ]

\pgfmathsetmacro\cornerlength{\reflength / 20}

\drawclampyzat{( 0, 0, 0 )}
\drawfirstbeam
\drawfirstbeamtext

\def\externalforcecolor{\colorforforces}

\drawload{( {\firstforcepointx}, {\firstforcepointy}, {\firstforcepointz} )}{( {\firstforcevectorx * \forcearrowscale}, {\firstforcevectory * \forcearrowscale}, {\firstforcevectorz * \forcearrowscale} )}{pos=.2, below, inner sep=0pt, outer sep=10pt}{\scalebox{\textscale}{$ F $}}

\drawload{( {\secondforcepointx}, {\secondforcepointy}, {\secondforcepointz} )}{( {\secondforcevectorx * \forcearrowscale}, {\secondforcevectory * \forcearrowscale}, {\secondforcevectorz * \forcearrowscale} )}{pos=.2, right, inner sep=0pt, outer sep=5pt}{\scalebox{\textscale}{$ F $}}
\end{tikzpicture}
}

\end{center}

\newpage

Эпюры внутренних моментов от каждой из внешних сил отдельно:
\vspace{.5cm}

\begin{center}

\scalebox{1}{
\begin{tikzpicture}[ scale=1 ]

\begin{scope}[ yshift=6.5cm ]

\pgfmathsetmacro\cornerlength{\reflength / 32}

\drawfirstbeam

\def\externalforcecolor{blue}

\drawload{( {\firstforcepointx}, {\firstforcepointy}, {\firstforcepointz} )}{( {\firstforcevectorx * \forcearrowscale}, {\firstforcevectory * \forcearrowscale}, {\firstforcevectorz * \forcearrowscale} )}{pos=.2, below, inner sep=0pt, outer sep=10pt}{\scalebox{\textscale}{$ F $}}

\def\twistingmomentmultiplierx{0}
\def\twistingmomentmultipliery{0}
\def\twistingmomentmultiplierz{-1}
\def\twistingmomentepureoffset{.4*\reflength}

\def\spiralaxialstep{1.25}
\def\spiralinitialangle{80}
\def\wherefirstarrowonspiral{0.07}

\drawepureofinternalmomentfromfirstforce

\end{scope}

\begin{scope}[ yshift=-6.5cm ]

\pgfmathsetmacro\cornerlength{\reflength / 32}

\drawfirstbeam

\def\externalforcecolor{red}

\def\twistingmomentmultiplierx{0}
\def\twistingmomentmultipliery{-1}
\def\twistingmomentmultiplierz{0}
\def\twistingmomentepureoffset{\reflength / 3}

\def\spiralaxialstep{1.1}
\def\spiralinitialangle{-150}
\def\wherefirstarrowonspiral{0.055}

\drawepureofinternalmomentfromsecondforce

\drawthirdpartofbeam
\drawfourthpartofbeam

\drawload{( {\secondforcepointx}, {\secondforcepointy}, {\secondforcepointz} )}{( {\secondforcevectorx * \forcearrowscale}, {\secondforcevectory * \forcearrowscale}, {\secondforcevectorz * \forcearrowscale} )}{pos=.2, right, inner sep=0pt, outer sep=5pt}{\scalebox{\textscale}{$ F $}}

\end{scope}

\end{tikzpicture}
}

\end{center}

\newpage

Эпюра внутренних моментов от всех внешних нагрузок:

\begin{center}

\scalebox{1}{
\begin{tikzpicture}[ scale=1 ]

\pgfmathsetmacro\cornerlength{\reflength / 32}

\drawfirstbeam

\def\twistingmomentepureoffset{.4*\reflength}
\def\twistingmomentmultiplierx{0}
\def\twistingmomentmultipliery{0}
\def\twistingmomentmultiplierz{1}

\drawepureofinternalmomentfrombothforces{blue}{red}{magenta}

\def\externalforcecolor{blue}

\drawload{( {\firstforcepointx}, {\firstforcepointy}, {\firstforcepointz} )}{( {\firstforcevectorx * \forcearrowscale}, {\firstforcevectory * \forcearrowscale}, {\firstforcevectorz * \forcearrowscale} )}{pos=.2, below, inner sep=0pt, outer sep=10pt}{\scalebox{\textscale}{$ F $}}

\def\externalforcecolor{red}

\drawload{( {\secondforcepointx}, {\secondforcepointy}, {\secondforcepointz} )}{( {\secondforcevectorx * \forcearrowscale}, {\secondforcevectory * \forcearrowscale}, {\secondforcevectorz * \forcearrowscale} )}{pos=.2, left, inner sep=0pt, outer sep=6pt}{\scalebox{\textscale}{$ F $}}

\end{tikzpicture}
}
\end{center}

\vspace{.5cm}

Наиболее нагруженная точка конструкции~--- у~закрепления (заделки).
В~ней
\nopagebreak\vspace{-.8em}\[
M_{\hspace{-0.1ex}\scalebox{.8}{$z$}} \hspace{-.25ex}
= \hspace{-.2ex} F \hspace{-0.1ex} \ell
\hspace{.1ex} , \hspace{.4em}
M_{\hspace{-0.1ex}\scalebox{.8}{$y$}} \hspace{-.25ex}
= \hspace{-.2ex} - \hspace{.1ex} 2 \hspace{.1ex} F \hspace{-0.1ex} \ell
\hspace{.1ex} , \hspace{.4em}
M_{\hspace{-0.1ex}\scalebox{.8}{$x$}} \hspace{-.25ex}
= \hspace{-.2ex}
M_{\hspace{-0.1ex}\mathrm{\scalebox{.6}{K}}} \hspace{-.25ex}
= \hspace{-.2ex} F \hspace{-0.1ex} \ell
\]

\vspace{-.4em}
\begin{center}
\scalebox{.8}{
\begin{tikzpicture}[ scale=1 ]

\tdplotsetmaincoords{72}{58}

\pgfmathsetmacro\cornerlength{\reflength / 20}

\drawclampyzat{( 0, 0, 0 )}[minusy]
\drawfirstbeam

{
\def\showcoordinatelength{.5*\reflength}
\def\textscale{1.5}
\drawaxes{x}{y}{z}
}

\draw [ aux line thick, color=red, fill=red ] ( 0, 0, 0 ) circle ( 2\pgflinewidth ) ;
\end{tikzpicture}
}
\end{center}

%%\vspace{.5cm}

\newpage

\def\couplecolor{black}
\def\momentcircleradius{5.5pt}
\def\momentarrowlen{.8}
\tikzstyle{moment arrow} =
	[ line width=2.5*\auxlinewidth, color=black, line cap=round, -{Triangle[round, length=9pt, width=6pt]} ]

% #1: length of couple
% #2: angle of couple
% #3: text
% #4: node options
\newcommand\drawmomentaroundx[4]{
	\draw [ aux line thick, color=\couplecolor, opacity=.8 ] ( 0, 0 ) -- ++( {#1*cos(#2)}, {#1*sin(#2)} ) ;
	\draw [ aux line thick, color=\couplecolor, opacity=.8 ] ( 0, 0 ) -- ++( {-1*#1*cos(#2)}, {-1*#1*sin(#2)} ) ;
	\draw [ moment arrow, color=\couplecolor ]
		( {#1*cos(#2)}, {#1*sin(#2)} ) -- ++( {\momentarrowlen*sin(#2)}, {-1*\momentarrowlen*cos(#2)} ) node [ #4 ] { \scalebox{.8}{#3} } ;
	\draw [ moment arrow, color=\couplecolor ]
		( {-1*#1*cos(#2)}, {-1*#1*sin(#2)} ) -- ++( {-1*\momentarrowlen*sin(#2)}, {\momentarrowlen*cos(#2)} ) ;
}

% #1: length of couple
% #2: angle of couple
% #3: text
% #4: node options
\newcommand\drawmomentaroundxthreedim[4]{
	\draw [ aux line thick, color=\couplecolor, opacity=.8, tdplot_main_coords ]
		( 0, 0, 0 ) -- ++( 0, {#1*cos(#2)}, {#1*sin(#2)} ) ;
	\draw [ aux line thick, color=\couplecolor, opacity=.8, tdplot_main_coords ]
		( 0, 0, 0 ) -- ++( 0, {-1*#1*cos(#2)}, {-1*#1*sin(#2)} ) ;
	\draw [ moment arrow, color=\couplecolor, tdplot_main_coords ]
		( 0, {#1*cos(#2)}, {#1*sin(#2)} ) -- ++( 0, {\momentarrowlen*sin(#2)}, {-1*\momentarrowlen*cos(#2)} ) node [ #4 ] { \scalebox{.8}{#3} } ;
	\draw [ moment arrow, color=\couplecolor, tdplot_main_coords ]
		( 0, {-1*#1*cos(#2)}, {-1*#1*sin(#2)} ) -- ++( 0, {-1*\momentarrowlen*sin(#2)}, {\momentarrowlen*cos(#2)} ) ;
}

% #1: length of couple
% #2: text
% #3: node options
\newcommand\drawmomentaroundy[3]{
	\draw [ aux line thick, color=\couplecolor, opacity=.8 ] ( 0, 0 ) -- ++( 0, #1 ) ;
	\draw [ aux line thick, color=\couplecolor, opacity=.8 ] ( 0, 0 ) -- ++( 0, -1*#1 ) ;
	\draw [ aux line thick, color=\couplecolor, fill=white ]
		( 0, #1 ) circle ( \momentcircleradius ) ;
	\draw [ aux line thick, color=\couplecolor, fill=\couplecolor ]
		( 0, #1 ) circle ( .66\pgflinewidth ) node [ #3 ] { \scalebox{.8}{#2} } ;
	\draw [ aux line thick, color=\couplecolor, fill=white ]
		( 0, -1*#1 ) circle ( \momentcircleradius ) ;
	\draw [ aux line thin, color=\couplecolor ]
		($ ( 0, -1*#1 ) - ( {\momentcircleradius*cos(45)}, {\momentcircleradius*sin(45)} ) $) -- ++( {2*\momentcircleradius*cos(45)}, {2*\momentcircleradius*sin(45)} ) ;
	\draw [ aux line thin, color=\couplecolor ]
		($ ( 0, -1*#1 ) + ( {\momentcircleradius*cos(-45)}, {\momentcircleradius*sin(-45)} ) $) -- ++( {-2*\momentcircleradius*cos(-45)}, {-2*\momentcircleradius*sin(-45)} ) ;
}

% #1: length of couple
% #2: text
% #3: node options
\newcommand\drawmomentaroundythreedim[3]{
	\draw [ aux line thick, color=\couplecolor, opacity=.8, tdplot_main_coords ]
		( 0, 0, 0 ) -- ++( 0, 0, #1 ) ;
	\draw [ aux line thick, color=\couplecolor, opacity=.8, tdplot_main_coords ]
		( 0, 0, 0 ) -- ++( 0, 0, -1*#1 ) ;
	\draw [ moment arrow, color=\couplecolor, tdplot_main_coords ]
		( 0, 0, #1 ) -- ++( 1.5*\momentarrowlen, 0, 0 ) node [ #3 ] { \scalebox{.8}{#2} } ;
	\draw [ moment arrow, color=\couplecolor, tdplot_main_coords ]
		( 0, 0, -1*#1 ) -- ++( -1.5*\momentarrowlen, 0, 0 ) ;
}

% #1: length of couple
% #2: text
% #3: node options
\newcommand\drawmomentaroundz[3]{
	\draw [ aux line thick, color=\couplecolor, opacity=.8 ] ( 0, 0 ) -- ++( #1, 0 ) ;
	\draw [ aux line thick, color=\couplecolor, opacity=.8 ] ( 0, 0 ) -- ++( -1*#1, 0 ) ;
	\draw [ aux line thick, color=\couplecolor, fill=white ]
		( #1, 0 ) circle ( \momentcircleradius ) ;
	\draw [ aux line thick, color=\couplecolor, fill=\couplecolor ]
		( #1, 0 ) circle ( .66\pgflinewidth ) node [ #3 ] { \scalebox{.8}{#2} } ;
	\draw [ aux line thick, color=\couplecolor, fill=white ]
		( -1*#1, 0 ) circle ( \momentcircleradius ) ;
	\draw [ aux line thin, color=\couplecolor ]
		($ ( -1*#1, 0 ) - ( {\momentcircleradius*cos(45)}, {\momentcircleradius*sin(45)} ) $) -- ++( {2*\momentcircleradius*cos(45)}, {2*\momentcircleradius*sin(45)} ) ;
	\draw [ aux line thin, color=\couplecolor ]
		($ ( -1*#1, 0 ) + ( {\momentcircleradius*cos(-45)}, {\momentcircleradius*sin(-45)} ) $) -- ++( {-2*\momentcircleradius*cos(-45)}, {-2*\momentcircleradius*sin(-45)} ) ;
}

% #1: length of couple
% #2: text
% #3: node options
\newcommand\drawmomentaroundzthreedim[3]{
	\draw [ aux line thick, color=\couplecolor, opacity=.8, tdplot_main_coords ]
		( 0, 0, 0 ) -- ++( 0, #1, 0 ) ;
	\draw [ aux line thick, color=\couplecolor, opacity=.8, tdplot_main_coords ]
		( 0, 0, 0 ) -- ++( 0, -1*#1, 0 ) ;
	\draw [ moment arrow, color=\couplecolor, tdplot_main_coords ]
		( 0, #1, 0 ) -- ++( 1.5*\momentarrowlen, 0, 0 ) node [ #3 ] { \scalebox{.8}{#2} } ;
	\draw [ moment arrow, color=\couplecolor, tdplot_main_coords ]
		( 0, -1*#1, 0 ) -- ++( -1.5*\momentarrowlen, 0, 0 ) ;
}

% #1: length of couple around z
% #2: length of couple around y
% #3: text
% #4: node options
\newcommand\drawsummarybendingmoment[4]{
	\draw [ aux line thick, color=\couplecolor, opacity=.8 ] ( 0, 0 ) -- ++( #1, #2 ) ;
	\draw [ aux line thick, color=\couplecolor, opacity=.8 ] ( 0, 0 ) -- ++( -1*#1, -1*#2 ) ;
	\draw [ aux line thick, color=\couplecolor, fill=white ]
		( #1, #2 ) circle ( \momentcircleradius ) ;
	\draw [ aux line thick, color=\couplecolor, fill=\couplecolor ]
		( #1, #2 ) circle ( .66\pgflinewidth ) node [ #4 ] { \scalebox{.8}{#3} } ;
	\draw [ aux line thick, color=\couplecolor, fill=white ]
		( -1*#1, -1*#2 ) circle ( \momentcircleradius ) ;
	\draw [ aux line thin, color=\couplecolor ]
		($ ( -1*#1, -1*#2 ) - ( {\momentcircleradius*cos(45)}, {\momentcircleradius*sin(45)} ) $) -- ++( {2*\momentcircleradius*cos(45)}, {2*\momentcircleradius*sin(45)} ) ;
	\draw [ aux line thin, color=\couplecolor ]
		($ ( -1*#1, -1*#2 ) + ( {\momentcircleradius*cos(-45)}, {\momentcircleradius*sin(-45)} ) $) -- ++( {-2*\momentcircleradius*cos(-45)}, {-2*\momentcircleradius*sin(-45)} ) ;
}

\tikzstyle{cube side} = [ aux line thick, rounded corners=.5\pgflinewidth, fill=white, fill opacity=.66 ]

% parameter is the side of cube
\newcommand\drawcube[1]{
	\draw [ cube side, tdplot_main_coords ]
		( 0, 0, 0 ) -- ( #1, 0, 0 ) -- ( #1, #1, 0 ) -- ( 0, #1, 0 )
		-- cycle ;
	\draw [ cube side, tdplot_main_coords ]
		( 0, 0, 0 ) -- ( 0, #1, 0 ) -- ( 0, #1, #1 ) -- ( 0, 0, #1 )
		-- cycle ;
	\draw [ cube side, tdplot_main_coords ]
		( 0, 0, 0 ) -- ( #1, 0, 0 ) -- ( #1, 0, #1 ) -- ( 0, 0, #1 )
		-- cycle ;
	\draw [ cube side, tdplot_main_coords ]
		( 0, #1, 0 ) -- ( #1, #1, 0 ) -- ( #1, #1, #1 ) -- ( 0, #1, #1 )
		-- cycle ;
	\draw [ cube side, tdplot_main_coords ]
		( 0, 0, #1 ) -- ( #1, 0, #1 ) -- ( #1, #1, #1 ) -- ( 0, #1, #1 )
		-- cycle ;
	\draw [ cube side, tdplot_main_coords ]
		( #1, 0, 0 ) -- ( #1, #1, 0 ) -- ( #1, #1, #1 ) -- ( #1, 0, #1 )
		-- cycle ;
}

\hspace*{-\parindent}\begin{minipage}{\linewidth}
\begin{center}

\emph{Круглое сечение}
\vspace{1cm}

\scalebox{1.4}{
\begin{tikzpicture}[ scale=1 ]

\tdplotsetmaincoords{70}{123}

\def\radiusofcircle{.8}

\drawcircularsectionthreedimzy

\pgfmathsetmacro\xaxislinelength{2.2 + \radiusofcircle}
\pgfmathsetmacro\yaxislinelength{3 + \radiusofcircle}
\pgfmathsetmacro\zaxislinelength{5 + \radiusofcircle}

\draw [ axis with arrow, -{To[round, length=6pt, width=7pt]}, tdplot_main_coords ]
	( {-\radiusofcircle - .2*\xaxislinelength}, 0, 0 ) -- ++( {\radiusofcircle + 1.1*\xaxislinelength}, 0, 0 )
	node [ pos=.98, below, shape=circle, inner sep=0pt, outer sep=4pt ]
		{\scalebox{.8}{$ x $}} ;

\draw [ axis with arrow, -{To[round, length=6pt, width=7pt]}, tdplot_main_coords ]
	( 0, {-\radiusofcircle - .2*\yaxislinelength}, 0 ) -- ++( 0, {\radiusofcircle + .9*\yaxislinelength}, 0 )
	node [ pos=.98, below, shape=circle, inner sep=0pt, outer sep=5pt ]
		{\scalebox{.8}{$ y $}} ;

\draw [ axis with arrow, -{To[round, length=6pt, width=7pt]}, tdplot_main_coords ]
	( 0, 0, {-\radiusofcircle - .2*\zaxislinelength} ) -- ++( 0, 0, {\radiusofcircle + .9*\zaxislinelength} )
	node [ pos=.99, right, shape=circle, inner sep=0pt, outer sep=4.5pt ]
		{\scalebox{.8}{$ z $}} ;

\def\couplecolor{red}
\drawmomentaroundxthreedim{1.66}{133}{${F \hspace{-0.1ex} \ell}$}{pos=1.1, left, shape=circle, inner sep=0pt, outer sep=11pt}

\def\couplecolor{blue}
\drawmomentaroundzthreedim{1.66}{${F \hspace{-0.1ex} \ell}$}{pos=1.4, right, shape=circle, inner sep=0pt, outer sep=15pt}

\def\couplecolor{magenta}
\drawmomentaroundythreedim{3.33}{${2 \hspace{.1ex} F \hspace{-0.1ex} \ell}$}{pos=1.1, above, shape=circle, inner sep=0pt, outer sep=5pt}

\begin{scope}[ xshift=5cm ]

\drawcircularsection

%%\def\angleofdimension{123}
%%\drawcircularsectiondimensions

\def\saxislengthmultiplier{2.02}

\draw [ axis with arrow, -{To[round, length=6pt, width=7pt]} ]
	( {-\radiusofcircle - .2*\yaxislinelength}, 0 ) -- ++( {\radiusofcircle + .9*\yaxislinelength}, 0 )
	node [ pos=.98, below, shape=circle, inner sep=0pt, outer sep=6pt ]
		{\scalebox{.8}{$ y $}} ;

\draw [ axis with arrow, -{To[round, length=6pt, width=7pt]} ]
	( 0, {-\radiusofcircle - .2*\zaxislinelength} ) -- ++( 0, {\radiusofcircle + .9*\zaxislinelength} )
	node [ pos=.98, left, shape=circle, inner sep=0pt, outer sep=5pt ]
		{\scalebox{.8}{$ z $}} ;

\draw [ axis with arrow, -{To[round, length=6pt, width=7pt]} ]
	( 0, 0 ) -- ++( {\saxislengthmultiplier*1}, {\saxislengthmultiplier*2} )
	node [ pos=.98, left, shape=circle, inner sep=0pt, outer sep=6pt ]
		{\scalebox{.8}{$ a $}} ;

\def\valueforFl{1.66}

\def\couplecolor{red}
\drawmomentaroundx{\valueforFl}{133}{${F \hspace{-0.1ex} \ell}$}{pos=1.1, left, shape=circle, inner sep=0pt, outer sep=11pt}

\def\couplecolor{blue}
\drawmomentaroundz{\valueforFl}{${F \hspace{-0.1ex} \ell}$}{above, shape=circle, inner sep=0pt, outer sep=6pt}

\def\couplecolor{magenta}
\drawmomentaroundy{2*\valueforFl}{${2 \hspace{.1ex} F \hspace{-0.1ex} \ell}$}{right, shape=circle, inner sep=0pt, outer sep=8pt}

\def\couplecolor{black}
\drawsummarybendingmoment{\valueforFl}{2*\valueforFl}{${\sqrt{5} \hspace{.1ex} F \hspace{-0.1ex} \ell}$}{anchor=north west, shape=circle, inner sep=0pt, outer sep=3pt}

\end{scope}

\end{tikzpicture}
}

\vspace{1.5cm}

\scalebox{1.4}{
\begin{tikzpicture}[ scale=1 ]

\def\radiusofcircle{.8}

\drawcircularsection

\def\saxislengthmultiplier{6}
\def\naxislengthmultiplier{4}

\draw [ axis with arrow, -{To[round, length=6pt, width=7pt]} ]
	( 0, 0 ) -- ++( \saxislengthmultiplier*\radiusofcircle, 0 )
	node [ pos=.98, below, shape=circle, inner sep=0pt, outer sep=6pt ]
		{\scalebox{.8}{$ a $}} ;

\draw [ axis with arrow, -{To[round, length=6pt, width=7pt]} ]
	( 0, -1.5*\radiusofcircle ) -- ++( 0, \naxislengthmultiplier*\radiusofcircle )
	node [ pos=.97, right, shape=circle, inner sep=0pt, outer sep=4pt ]
		{\scalebox{.8}{$ b $}} ;

\def\couplecolor{black}
\def\valueforFl{1.66}
\pgfmathsetmacro\summarybendingmoment{\valueforFl*sqrt(5)}
\drawmomentaroundz{\summarybendingmoment}{${\sqrt{5} \hspace{.1ex} F \hspace{-0.1ex} \ell}$}{below, shape=circle, inner sep=0pt, outer sep=1pt}

\def\couplecolor{red}
\drawmomentaroundx{\valueforFl}{0}{${F \hspace{-0.1ex} \ell}$}{pos=.7, right, shape=circle, inner sep=0pt, outer sep=4pt}

\def\offsetforsigmaepure{-2.2}

\draw [ aux thick dashed ]
	( -\radiusofcircle, 0 ) -- ++( 0, \offsetforsigmaepure ) ;
\draw [ aux thick dashed ]
	( \radiusofcircle, 0 ) -- ++( 0, \offsetforsigmaepure ) ;
\draw [ aux line thick, opacity=.8 ]
	( -\radiusofcircle, \offsetforsigmaepure ) -- ++( 2*\radiusofcircle, 0 ) ;

\def\offsetfortauepure{-2.2}

\draw [ aux thick dashed ]
	( -\radiusofcircle, \offsetforsigmaepure ) -- ++( 0, \offsetfortauepure ) ;
\draw [ aux thick dashed ]
	( \radiusofcircle, \offsetforsigmaepure ) -- ++( 0, \offsetfortauepure ) ;
\draw [ aux line thick, opacity=.8 ]
	( -\radiusofcircle, {\offsetforsigmaepure + \offsetfortauepure} ) -- ++( 2*\radiusofcircle, 0 ) ;

\draw [ aux line thin ] ( 0, -\radiusofcircle ) -- ( 0, {\offsetforsigmaepure + \offsetfortauepure} ) ;

\def\howmanyepurelines{7}
\pgfmathsetmacro\epurestep{\radiusofcircle / (\howmanyepurelines - 1)}

\def\maxsigma{1}
\def\epurecolor{black}

\foreach \x in { 0, \epurestep, ..., \radiusofcircle } {
	\pgfmathsetmacro\y{\maxsigma * \x / \radiusofcircle}
	\draw [ aux line thin, color=\epurecolor ] ( \x, \offsetforsigmaepure ) -- ++( 0, \y ) ;
}
\foreach \x in { 0, -\epurestep, ..., -\radiusofcircle } {
	\pgfmathsetmacro\y{\maxsigma * \x / \radiusofcircle}
	\draw [ aux line thin, color=\epurecolor ] ( \x, \offsetforsigmaepure ) -- ++( 0, \y ) ;
}
\draw [ aux line thick, color=\epurecolor ]
	( \radiusofcircle, \offsetforsigmaepure )
	-- ++( 0, \maxsigma )
	node [ pos=.5, right, shape=circle, inner sep=0pt, outer sep=2.5pt ]
	{\scalebox{.8}{$ \upsigma_{\hspace{-0.2ex}\scalebox{.75}{max}} $}} ;
\draw [ aux line thick, color=\epurecolor ]
	( -\radiusofcircle, \offsetforsigmaepure )
	-- ++( 0, -\maxsigma )
	node [ pos=.5, left, shape=circle, inner sep=0pt, outer sep=3pt ]
	{\scalebox{.8}{$ \upsigma_{\hspace{-0.2ex}\scalebox{.75}{max}} $}} ;
\draw [ aux line thick, color=\epurecolor ]
	( -\radiusofcircle, {\offsetforsigmaepure - \maxsigma} )
	-- ( \radiusofcircle, {\offsetforsigmaepure + \maxsigma} ) ;

\def\maxtau{.7}
\def\epurecolor{red}

\foreach \x in { 0, \epurestep, ..., \radiusofcircle } {
	\pgfmathsetmacro\y{\maxtau * \x / \radiusofcircle}
	\draw [ aux line thin, color=\epurecolor ]
		( \x, {\offsetforsigmaepure + \offsetfortauepure} )
		-- ++( 0, -\y ) ;
}
\foreach \x in { 0, -\epurestep, ..., -\radiusofcircle } {
	\pgfmathsetmacro\y{\maxtau * \x / \radiusofcircle}
	\draw [ aux line thin, color=\epurecolor ]
		( \x, {\offsetforsigmaepure + \offsetfortauepure} )
		-- ++( 0, -\y ) ;
}
\draw [ aux line thick, color=\epurecolor ]
	( \radiusofcircle, {\offsetforsigmaepure + \offsetfortauepure} )
	-- ++( 0, -\maxtau )
	node [ pos=.6, right, shape=circle, inner sep=0pt, outer sep=2.5pt ]
	{\scalebox{.8}{$ \uptau_{\scalebox{.75}{max}} $}} ;
\draw [ aux line thick, color=\epurecolor ]
	( -\radiusofcircle, {\offsetforsigmaepure + \offsetfortauepure} )
	-- ++( 0, \maxtau )
	node [ pos=.5, left, shape=circle, inner sep=0pt, outer sep=3pt ]
	{\scalebox{.8}{$ \uptau_{\scalebox{.75}{max}} $}} ;
\draw [ aux line thick, color=\epurecolor ]
	( -\radiusofcircle, {\offsetforsigmaepure + \offsetfortauepure + \maxtau} )
	-- ( \radiusofcircle, {\offsetforsigmaepure + \offsetfortauepure - \maxtau} ) ;

\draw [ aux line thick, color=black, fill=yellow ]
	( -\radiusofcircle, 0 ) circle ( 1.6\pgflinewidth )
	node [ above left, shape=circle, inner sep=0pt, outer sep=3pt ] {\scalebox{.7}{ (1) }} ;
\draw [ aux line thick, color=black, fill=yellow ]
	( \radiusofcircle, 0 ) circle ( 1.6\pgflinewidth )
	node [ above right, shape=circle, inner sep=0pt, outer sep=3pt ] {\scalebox{.7}{ (2) }} ;

\end{tikzpicture}
}

\end{center}
\end{minipage}
\vspace{.6cm}

\newpage

Осевой момент инерции круглого сечения вокруг любой оси, проходящей через центр круга:
\[
\mathfrak{I}_{\hspace{-0.1ex}\scalebox{.8}{$z$}} \hspace{-0.25ex}=\hspace{-0.1ex}
\mathfrak{I}_{\hspace{-0.1ex}\scalebox{.8}{$y$}} \hspace{-0.25ex}=\hspace{-0.1ex}
\mathfrak{I}_{\hspace{-0.1ex}\scalebox{.8}{$b$}} \hspace{-0.25ex}=\hspace{-0.1ex}
\mathfrak{I}_{\hspace{-0.1ex}\scalebox{.8}{$a$}} \hspace{-0.25ex}=\hspace{-0.1ex}
\displaystyle\frac{\raisebox{-0.18em}{$\pi \hspace{-0.1ex} D^{\hspace{.1ex}4}$}}{64}
\]

Нормальные напряжения от изгиба
\[
\upsigma_{\hspace{-0.2ex}\scalebox{.8}{$x$}} \hspace{-0.2ex}
= \displaystyle\frac{M_{\hspace{-0.1ex}\scalebox{.8}{$z$}}}{\mathfrak{I}_{\hspace{-0.1ex}\scalebox{.8}{$z$}}} \hspace{.25ex} y
- \displaystyle\frac{M_{\hspace{-0.1ex}\scalebox{.8}{$y$}}}{\mathfrak{I}_{\hspace{-0.1ex}\scalebox{.8}{$y$}}} \hspace{.25ex} z
= \displaystyle\frac{M_{\hspace{-0.1ex}\scalebox{.8}{$b$}}}{\mathfrak{I}_{\hspace{-0.1ex}\scalebox{.8}{$b$}}} \hspace{.25ex} a
\hspace{.1ex} ,
\hspace{.4em}
M_{\hspace{-0.1ex}\scalebox{.8}{$b$}}
\hspace{-0.2ex}=\hspace{-0.2ex} \sqrt{\hspace{-0.2ex}M_{\hspace{-0.1ex}\scalebox{.8}{$z$}}^{\hspace{.2ex}\raisebox{.2em}{$\scriptstyle 2$}} \hspace{-0.2ex}+\hspace{-0.2ex} M_{\hspace{-0.1ex}\scalebox{.8}{$y$}}^{\hspace{.2ex}\raisebox{.2em}{$\scriptstyle 2$}}\hspace{.2ex}}
\]
максимальны в~точках контура сечения, где
%%${y^2 \hspace{-0.25ex}+\hspace{-0.2ex} z^2 \hspace{-.15ex}=\hspace{-.3ex} \left( \scalebox{.82}{$ \displaystyle\frac{\raisebox{-.2em}{$D$}}{2} $} \hspace{.1ex} \right)^{\hspace{-0.4ex}2}\hspace{-0.2ex}}$ и
${\hspace{.1ex}a \hspace{-.25ex}=\hspace{-.2ex} \pm \hspace{.2ex} \scalebox{.82}{$ \displaystyle\frac{\raisebox{-.2em}{$D$}}{2} $}\hspace{.2ex}}$:
\[
\upsigma_{\hspace{-0.2ex}\scalebox{.75}{max}} \hspace{-.25ex}
= \hspace{-.2ex} \pm \hspace{.1ex} M_{\hspace{-0.1ex}\scalebox{.8}{$b$}} \hspace{.2ex}
\displaystyle\frac{\raisebox{-.2em}{$64$}}{\pi \hspace{-0.1ex} D^{\hspace{.1ex}4}} \hspace{.25ex} \displaystyle\frac{\raisebox{-.2em}{$D$}}{2}
\hspace{-.1ex}= \hspace{-.1ex}
\pm \hspace{.1ex} \displaystyle\frac{\raisebox{-.2em}{$32$}}{\pi \hspace{-0.1ex} D^{\hspace{.1ex}3}} \hspace{.1ex}
\sqrt{\hspace{-0.2ex}M_{\hspace{-0.1ex}\scalebox{.8}{$z$}}^{\hspace{.2ex}\raisebox{.2em}{$\scriptstyle 2$}} \hspace{-0.2ex}+\hspace{-0.2ex} M_{\hspace{-0.1ex}\scalebox{.8}{$y$}}^{\hspace{.2ex}\raisebox{.2em}{$\scriptstyle 2$}}\hspace{.2ex}}
\]

Полярный момент инерции круглого сечения:
\[
\mathfrak{I}_{\hspace{-0.2ex}\scalebox{.8}{$\rho$}} \hspace{-0.25ex}=\hspace{-0.1ex}
\displaystyle\frac{\raisebox{-0.18em}{$\pi \hspace{-0.1ex} D^{\hspace{.1ex}4}$}}{32}
\]

Касательные напряжения от кручения
\[
\uptau_{\scalebox{.8}{$x \hspace{-.1ex} \rho$}} \hspace{-0.25ex}=\hspace{-0.1ex}
\displaystyle\frac{M_{\hspace{-0.1ex}\mathrm{\scalebox{.6}{K}}}}{\mathfrak{I}_{\hspace{-0.2ex}\scalebox{.8}{$\rho$}}} \hspace{.25ex} \rho
\]
максимальны тоже на~контуре сечения~--- в~точках с~${\hspace{.1ex}\rho \hspace{-.2ex}=\hspace{-.15ex} \scalebox{.82}{$ \displaystyle\frac{\raisebox{-.2em}{$D$}}{2} $}\hspace{.2ex}}$:
\[
\uptau_{\scalebox{.75}{max}} \hspace{-.25ex}
= \hspace{-.2ex} M_{\hspace{-0.1ex}\mathrm{\scalebox{.6}{K}}}\hspace{.2ex}
\displaystyle\frac{\raisebox{-.2em}{$32$}}{\pi \hspace{-0.1ex} D^{\hspace{.1ex}4}} \hspace{.25ex} \displaystyle\frac{\raisebox{-.2em}{$D$}}{2}
\hspace{-.1ex}= \hspace{-.1ex}
\displaystyle\frac{\raisebox{-.2em}{$16$}}{\pi \hspace{-0.1ex} D^{\hspace{.1ex}3}} \hspace{.1ex}
M_{\hspace{-0.1ex}\mathrm{\scalebox{.6}{K}}}
\]

\vspace{.2cm}
В~самых напряжённых точках сечения~--- точках (1) и~(2)~--- имеем
\vspace{.2cm}

\def\sideofcube{1.5}
\def\showcoordinatelength{3}

\begin{center}

\scalebox{1}{
\begin{tikzpicture}[ scale=1.2 ]

\begin{scope}[xshift=-3cm]

\tdplotsetmaincoords{66}{133}

 \node [ outer sep=0pt, inner sep=0pt, tdplot_main_coords ]
	at ( 0, 0, 3.8 )
	{ \scalebox{1}{\emph{точка~(1)}} } ;

\drawaxes{x}{a}{b}

\def\stresscolor{black}

\def\stresslength{1.8}
\draw [ external force, color=\stresscolor, tdplot_main_coords ]
	($ ( 0, {.5*\sideofcube}, {.5*\sideofcube} ) - ( \stresslength, 0, 0 ) $)
	-- ++( \stresslength, 0, 0 ) ;

\def\stresslength{1.1}
\draw [ external force, color=\stresscolor, tdplot_main_coords ]
	( {.5*\sideofcube + .5*\stresslength}, {.5*\sideofcube}, 0 ) -- ++( -\stresslength, 0, 0 ) ;
\draw [ external force, color=\stresscolor, tdplot_main_coords ]
	( 0, {.5*\sideofcube}, {.5*\sideofcube + .5*\stresslength} ) -- ++( 0, 0, -\stresslength ) ;

\drawcube{\sideofcube}

\def\stresslength{1.8}
\draw [ external force, color=\stresscolor, tdplot_main_coords ]
	($ ( {\sideofcube}, {.5*\sideofcube}, {.5*\sideofcube} )+ ( \stresslength, 0, 0 ) $) -- ++( -\stresslength, 0, 0 )
	node [ pos=-0.1, above, inner sep=0pt, outer sep=12pt ]
	{ \scalebox{\textscale}{$ \upsigma_{\hspace{-0.2ex}\scalebox{.75}{max}} $} } ;

\def\stresslength{1.1}
\draw [ external force, color=\stresscolor, tdplot_main_coords ]
	( {\sideofcube}, {.5*\sideofcube}, {.5*\sideofcube - .5*\stresslength} ) -- ++( 0, 0, \stresslength )
	node [ pos=-0.2, below, inner sep=0pt, outer sep=6pt ]
	{ \scalebox{\textscale}{$ \uptau_{\scalebox{.75}{max}} $} } ;
\draw [ external force, color=\stresscolor, tdplot_main_coords ]
	( {.5*\sideofcube - .5*\stresslength}, {.5*\sideofcube}, {\sideofcube} ) -- ++( \stresslength, 0, 0 ) ;

\end{scope}

\begin{scope}[xshift=3cm]

\tdplotsetmaincoords{66}{133}

 \node [ outer sep=0pt, inner sep=0pt, tdplot_main_coords ]
	at ( 0, 0, 3.8 )
	{ \scalebox{1}{\emph{точка~(2)}} } ;

\drawaxes{x}{a}{b}

\def\stresscolor{black}

\def\stresslength{1.8}
\draw [ external force, color=\stresscolor, tdplot_main_coords ]
	( 0, {.5*\sideofcube}, {.5*\sideofcube} ) -- ++( -\stresslength, 0, 0 ) ;

\def\stresslength{1.1}
\draw [ external force, color=\stresscolor, tdplot_main_coords ]
	( {.5*\sideofcube - .5*\stresslength}, {.5*\sideofcube}, 0 ) -- ++( \stresslength, 0, 0 ) ;
\draw [ external force, color=\stresscolor, tdplot_main_coords ]
	( 0, {.5*\sideofcube}, {.5*\sideofcube - .5*\stresslength} ) -- ++( 0, 0, \stresslength ) ;

\drawcube{\sideofcube}

\def\stresslength{1.8}
\draw [ external force, color=\stresscolor, tdplot_main_coords ]
	( {\sideofcube}, {.5*\sideofcube}, {.5*\sideofcube} ) -- ++( \stresslength, 0, 0 )
	node [ pos=1, above, inner sep=0pt, outer sep=13pt ]
	{ \scalebox{\textscale}{$ \upsigma_{\hspace{-0.2ex}\scalebox{.75}{max}} $} } ;

\def\stresslength{1.1}
\draw [ external force, color=\stresscolor, tdplot_main_coords ]
	( {\sideofcube}, {.5*\sideofcube}, {.5*\sideofcube + .5*\stresslength} ) -- ++( 0, 0, -\stresslength )
	node [ pos=1.2, below, inner sep=0pt, outer sep=6pt ]
	{ \scalebox{\textscale}{$ \uptau_{\scalebox{.75}{max}} $} } ;
\draw [ external force, color=\stresscolor, tdplot_main_coords ]
	( {.5*\sideofcube + .5*\stresslength}, {.5*\sideofcube}, {\sideofcube} ) -- ++( -\stresslength, 0, 0 ) ;

\end{scope}

\end{tikzpicture}
}

\end{center}

По энергетическому критерию прочности эквивалентное одноосное напряжение есть
\[
\upsigma_{\hspace{-.15ex}\scalebox{.75}{max}}^{\hspace{.2ex}\scalebox{.75}{э}}
\hspace{-0.15ex}=\hspace{-0.1ex} \sqrt{\hspace{-0.1ex}
\upsigma_{\hspace{-0.2ex}\scalebox{.75}{max}}^{\hspace{.2ex}\raisebox{.2em}{$\scriptstyle 2$}} \hspace{-.3ex}+\hspace{-.1ex} 3 \hspace{.1ex} \uptau_{\scalebox{.75}{max}}^{\hspace{.2ex}\raisebox{.2em}{$\scriptstyle 2$}}
\hspace{.2ex}}
\]

Для круглого сечения
\[
\upsigma_{\hspace{-0.2ex}\scalebox{.75}{max}} \hspace{-.25ex}
= \hspace{-.1ex}
\displaystyle\frac{\raisebox{-.2em}{$32$}}{\pi \hspace{-0.1ex} D^{\hspace{.1ex}3}} \hspace{.1ex}
\sqrt{\hspace{-0.2ex}M_{\hspace{-0.1ex}\scalebox{.8}{$z$}}^{\hspace{.2ex}\raisebox{.2em}{$\scriptstyle 2$}} \hspace{-0.2ex}+\hspace{-0.2ex} M_{\hspace{-0.1ex}\scalebox{.8}{$y$}}^{\hspace{.2ex}\raisebox{.2em}{$\scriptstyle 2$}}\hspace{.2ex}}
, \hspace{.4em}
%
\uptau_{\scalebox{.75}{max}} \hspace{-.25ex}
= \hspace{-.1ex}
\displaystyle\frac{\raisebox{-.2em}{$16$}}{\pi \hspace{-0.1ex} D^{\hspace{.1ex}3}} \hspace{.1ex}
M_{\hspace{-0.1ex}\mathrm{\scalebox{.6}{K}}}
\]
и энергетический критерий даёт
\[
\upsigma_{\hspace{-.15ex}\scalebox{.75}{max}}^{\hspace{.2ex}\scalebox{.75}{э}} \hspace{-0.2ex}
=\hspace{-0.2ex}
\displaystyle\frac{\raisebox{-.2em}{$16$}}{\pi \hspace{-0.1ex} D^{\hspace{.1ex}3}} \hspace{.1ex}
\sqrt{%
4 \hspace{.1ex} \bigl( \hspace{-0.1ex} M_{\hspace{-0.1ex}\scalebox{.8}{$z$}}^{\hspace{.2ex}\raisebox{.2em}{$\scriptstyle 2$}} \hspace{-0.3ex}+\hspace{-0.3ex} M_{\hspace{-0.1ex}\scalebox{.8}{$y$}}^{\hspace{.2ex}\raisebox{.2em}{$\scriptstyle 2$}} \hspace{.2ex} \bigr) \hspace{-.4ex}
+\hspace{-.1ex} 3 M_{\hspace{-0.1ex}\mathrm{\scalebox{.6}{K}}}^{\hspace{.2ex}\raisebox{.2em}{$\scriptstyle 2$}}
\hspace{.2ex}}
\]

Для рассматриваемой задачи в~наиболее нагруженной точке (у~заделки)
\[
M_{\hspace{-0.1ex}\scalebox{.8}{$z$}} \hspace{-.25ex}
= \hspace{-.2ex} F \hspace{-0.1ex} \ell
\hspace{.1ex} , \hspace{.4em}
M_{\hspace{-0.1ex}\scalebox{.8}{$y$}} \hspace{-.25ex}
= \hspace{-.2ex} - \hspace{.1ex} 2 \hspace{.1ex} F \hspace{-0.1ex} \ell
\hspace{.1ex} , \hspace{.4em}
M_{\hspace{-0.1ex}\mathrm{\scalebox{.6}{K}}} \hspace{-.25ex}
= \hspace{-.2ex} F \hspace{-0.1ex} \ell
\]
и эквивалентное (по энергетическому критерию прочности) напряжение в~точках сечения (1) и~(2) равно
\nopagebreak\vspace{-0.5em}\[\begin{aligned}
\upsigma_{\hspace{-.15ex}\scalebox{.75}{max}}^{\hspace{.2ex}\scalebox{.75}{э}} \hspace{-0.2ex}
&=\hspace{-0.1ex}
\displaystyle\frac{\raisebox{-.2em}{$16$}}{\pi \hspace{-0.1ex} D^{\hspace{.1ex}3}} \hspace{.1ex}
\sqrt{%
4 \hspace{.1ex} \bigl( \hspace{-0.2ex} F^{\hspace{.2ex}\raisebox{.2em}{$\scriptstyle 2$}} \hspace{-0.1ex} \ell^{\hspace{.2ex}\raisebox{.2em}{$\scriptstyle 2$}} \hspace{-0.5ex}+\hspace{-0.2ex} 4 \hspace{.1ex} F^{\hspace{.2ex}\raisebox{.2em}{$\scriptstyle 2$}} \hspace{-0.1ex} \ell^{\hspace{.2ex}\raisebox{.2em}{$\scriptstyle 2$}} \hspace{.2ex} \bigr) \hspace{-.4ex}
+\hspace{-.1ex} 3 \hspace{.1ex} F^{\hspace{.2ex}\raisebox{.2em}{$\scriptstyle 2$}} \hspace{-0.1ex} \ell^{\hspace{.2ex}\raisebox{.2em}{$\scriptstyle 2$}}
\hspace{.2ex}}
\\[.2em]
%
&=\hspace{-0.1ex}
\displaystyle\frac{\raisebox{-.2em}{$16$}}{\pi \hspace{-0.1ex} D^{\hspace{.1ex}3}} \hspace{.1ex}
\sqrt{%
20 \hspace{.1ex} F^{\hspace{.2ex}\raisebox{.2em}{$\scriptstyle 2$}} \hspace{-0.1ex} \ell^{\hspace{.2ex}\raisebox{.2em}{$\scriptstyle 2$}} \hspace{-.44ex}
+\hspace{-.1ex} 3 \hspace{.1ex} F^{\hspace{.2ex}\raisebox{.2em}{$\scriptstyle 2$}} \hspace{-0.1ex} \ell^{\hspace{.2ex}\raisebox{.2em}{$\scriptstyle 2$}}
\hspace{.2ex}}
\\[.4em]
%
&=\hspace{-0.1ex}
\displaystyle\frac{\raisebox{-.2em}{$16 \sqrt{\hspace{-.1ex}23} \hspace{.2ex} F \hspace{-.1ex} \ell$}}{\pi \hspace{-0.1ex} D^{\hspace{.1ex}3}}
\end{aligned}\]

По условию прочности
\nopagebreak\vspace{-.1em}\[
\upsigma_{\hspace{-.15ex}\scalebox{.75}{max}}^{\hspace{.2ex}\scalebox{.75}{э}} \hspace{-0.2ex}
= \displaystyle\frac{\raisebox{-.05em}{$ \sigma_{\mathrm{\scalebox{.6}{T}}} $}}{\raisebox{.1em}{$ n_{\mathrm{\scalebox{.6}{T}}} $}}
\vspace{-.4em}\]
с~данными
\[
F \hspace{-.5ex}=\hspace{-.4ex} 1000\:\text{Н}
, \hspace{.4em}
\ell \!=\! 200\:\text{мм}
, \hspace{.4em}
\sigma_{\mathrm{\scalebox{.6}{T}}} \hspace{-.5ex}=\hspace{-.3ex} 300\:\text{МПа}
, \hspace{.4em}
n_{\mathrm{\scalebox{.6}{T}}} \hspace{-.44ex}=\hspace{-.33ex} 2
\]
определяем диаметр сечения:
\[\begin{gathered}
\displaystyle\frac{\raisebox{-.2em}{$16 \sqrt{\hspace{-.1ex}23} \cdot 1000\:\text{Н}\cdot 200\:\text{мм}$}}{\pi \hspace{-0.1ex} D^{\hspace{.1ex}3}}
= \displaystyle\frac{\raisebox{-.1em}{$ 300\:\text{МПа} $}}{2}
\\[.5em]
%
\Rightarrow\hspace{.4em}
D^{\hspace{.1ex}3} \hspace{-.25ex}
= \displaystyle\frac{\raisebox{-.1em}{$ 64\hspace{.2ex}000 \hspace{.1ex} \sqrt{\hspace{-.1ex}23} $}}{3 \hspace{.1ex} \pi} \hspace{.6ex} \text{мм}^3
\hspace{.4em}\Rightarrow\hspace{.4em}
D \hspace{-0.2ex}
=\hspace{-0.2ex} 40 \sqrt[\scalebox{.7}{$6$}]{\hspace{-.2ex}
\displaystyle\frac{\raisebox{-.1em}{$ 23 $}}{9 \pi^2}
\hspace{.1ex}} \hspace{.5ex} \text{мм}
\approx\hspace{-.1ex}
31.9343\:\text{мм}
\hspace{.4em}\Rightarrow\hspace{.4em}
D \!=\hspace{-0.25ex} 32\:\text{мм}
\end{gathered}\]

% ~ ~
\newpage
% ~ ~

\hspace*{-\parindent}\begin{minipage}{\linewidth}
\begin{center}

\emph{Прямоугольное сечение}
\vspace{1cm}

\scalebox{1.4}{
\begin{tikzpicture}[ scale=1 ]

\tdplotsetmaincoords{70}{123}

\def\widthofsquare{1.4}
\pgfmathsetmacro\heightofsquare{2*\widthofsquare}

\drawrectangularsectionthreedimzy

\pgfmathsetmacro\xaxislinelength{2}
\pgfmathsetmacro\yaxislinelength{2.7}
\pgfmathsetmacro\zaxislinelength{4}

\draw [ axis with arrow, -{To[round, length=6pt, width=7pt]}, tdplot_main_coords ]
	( {-.8*\xaxislinelength}, 0, 0 ) -- ++( {2.2*\xaxislinelength}, 0, 0 )
	node [ pos=.98, below, shape=circle, inner sep=0pt, outer sep=4pt ]
		{\scalebox{.8}{$ x $}} ;

\draw [ axis with arrow, -{To[round, length=6pt, width=7pt]}, tdplot_main_coords ]
	( 0, {-.5*\widthofsquare - .3*\yaxislinelength}, 0 ) -- ++( 0, {\widthofsquare + 1*\yaxislinelength}, 0 )
	node [ pos=.98, below, shape=circle, inner sep=0pt, outer sep=5pt ]
		{\scalebox{.8}{$ y $}} ;

\draw [ axis with arrow, -{To[round, length=6pt, width=7pt]}, tdplot_main_coords ]
	( 0, 0, {-.5*\heightofsquare - .3*\zaxislinelength} ) -- ++( 0, 0, {\heightofsquare + 1*\zaxislinelength} )
	node [ pos=.98, right, shape=circle, inner sep=0pt, outer sep=4.5pt ]
		{\scalebox{.8}{$ z $}} ;

\def\couplecolor{red}
\drawmomentaroundxthreedim{1.66}{133}{${F \hspace{-0.1ex} \ell}$}{pos=1.1, left, shape=circle, inner sep=0pt, outer sep=11pt}

\def\couplecolor{blue}
\drawmomentaroundzthreedim{1.66}{${F \hspace{-0.1ex} \ell}$}{pos=1.4, right, shape=circle, inner sep=0pt, outer sep=15pt}

\def\couplecolor{magenta}
\drawmomentaroundythreedim{3.33}{${2 \hspace{.1ex} F \hspace{-0.1ex} \ell}$}{pos=1.1, above, shape=circle, inner sep=0pt, outer sep=5pt}

\begin{scope}[ xshift=5cm ]

\drawrectangularsection

\def\saxislengthmultiplier{2.02}

\draw [ axis with arrow, -{To[round, length=6pt, width=7pt]} ]
	( {-.5*\widthofsquare - .3*\yaxislinelength}, 0 ) -- ++( {\widthofsquare + 1*\yaxislinelength}, 0 )
	node [ pos=1, below, shape=circle, inner sep=0pt, outer sep=5pt ]
		{\scalebox{.8}{$ y $}} ;

\draw [ axis with arrow, -{To[round, length=6pt, width=7pt]} ]
	( 0, {-.5*\heightofsquare - .3*\zaxislinelength} ) -- ++( 0, {\heightofsquare + 1*\zaxislinelength} )
	node [ pos=.98, left, shape=circle, inner sep=0pt, outer sep=5pt ]
		{\scalebox{.8}{$ z $}} ;

\draw [ axis with arrow, -{To[round, length=6pt, width=7pt]} ]
	( 0, 0 ) -- ++( {\saxislengthmultiplier*1}, {\saxislengthmultiplier*2} )
	node [ pos=.98, left, shape=circle, inner sep=0pt, outer sep=6pt ]
		{\scalebox{.8}{$ a $}} ;

\def\valueforFl{1.66}

\def\couplecolor{red}
\drawmomentaroundx{\valueforFl}{133}{${F \hspace{-0.1ex} \ell}$}{pos=1.1, left, shape=circle, inner sep=0pt, outer sep=11pt}

\def\couplecolor{blue}
\drawmomentaroundz{\valueforFl}{${F \hspace{-0.1ex} \ell}$}{above, shape=circle, inner sep=0pt, outer sep=6pt}

\def\couplecolor{magenta}
\drawmomentaroundy{2*\valueforFl}{${2 \hspace{.1ex} F \hspace{-0.1ex} \ell}$}{right, shape=circle, inner sep=0pt, outer sep=8pt}

\def\couplecolor{black}
\drawsummarybendingmoment{\valueforFl}{2*\valueforFl}{${\sqrt{5} \hspace{.1ex} F \hspace{-0.1ex} \ell}$}{anchor=north west, shape=circle, inner sep=0pt, outer sep=3pt}

%%\draw [ aux line thick, color=gray ]
%%	( -.5*\widthofsquare, .5*\heightofsquare ) -- ( 0, 0 ) ;
%%\draw [ aux line thick, color=gray, rotate around={90:(-.5*\widthofsquare, .5*\heightofsquare)}, shift={({-.5*\widthofsquare}, 0)} ]
%%	( -.5*\widthofsquare, .5*\heightofsquare ) -- ( 0, 0 ) ;

%%\draw [ aux line thick, color=cyan ]
%%	( .5*\widthofsquare, .5*\heightofsquare ) -- ++( -\widthofsquare, 0 ) ;
%%\draw [ aux line thick, color=green, rotate around={atan(1/2):(.5*\widthofsquare, .5*\heightofsquare)} ]
%%	( .5*\widthofsquare, .5*\heightofsquare ) -- ++( -\widthofsquare, 0 ) ;

%%\draw [ aux line thick, color=orange, rotate around={atan(1/2):(.5*\widthofsquare, .5*\heightofsquare)} ]
%%	( .5*\widthofsquare, .5*\heightofsquare ) -- ++( {-\widthofsquare*cos(atan(1/2))}, 0 ) ;

\end{scope}

\end{tikzpicture}
}

\end{center}
\end{minipage}
\vspace{.6cm}

\begin{center}

\scalebox{1.4}{
\begin{tikzpicture}[ scale=1 ]

\def\widthofsquare{1.4}
\pgfmathsetmacro\heightofsquare{2*\widthofsquare}

\drawrectangularsectionwithhorizontaldiagonal

\def\axislength{1}
\def\saxislengthmultiplier{5.5}
\def\naxislengthmultiplier{3.6}

\draw [ axis with arrow, -{To[round, length=6pt, width=6pt]} ]
	( 0, 0 ) -- ++( \saxislengthmultiplier*\axislength, 0 )
	node [ pos=.98, below, shape=circle, inner sep=0pt, outer sep=5pt ]
		{\scalebox{.8}{$ a $}} ;

\draw [ axis with arrow, -{To[round, length=6pt, width=6pt]} ]
	( 0, -1.5*\axislength ) -- ++( 0, \naxislengthmultiplier*\axislength )
	node [ pos=.97, right, shape=circle, inner sep=0pt, outer sep=3.3pt ]
		{\scalebox{.8}{$ b $}} ;

\def\offsetforsigmaepure{.9}

\draw [ aux line thin, rotate around={atan(-2):(0, 0)} ]
	( \offsetforsigmaepure*\widthofsquare, -\offsetforsigmaepure*\heightofsquare )
	-- ( -.9*\widthofsquare, .9*\heightofsquare )
	node [ pos=.92, right, shape=circle, inner sep=0pt, outer sep=5pt ]
		{\scalebox{.8}{$ \upsigma_{\hspace{-0.2ex}\scalebox{.8}{$x$}} \hspace{-0.25ex}=\hspace{-.1ex} 0 $}} ; % zero stress line

\draw [ aux line thin, rotate around={atan(-2):(0, 0)} ]
	( 0, -.5*\heightofsquare ) -- ( 0, .5*\heightofsquare ) ;
\draw [ aux line thin, rotate around={atan(-2):(0, 0)} ]
	( -.5*\widthofsquare, 0 ) -- ( .5*\widthofsquare, 0 ) ;

\def\couplecolor{black}
\def\valueforFl{2}
\pgfmathsetmacro\summarybendingmoment{\valueforFl*sqrt(5)}
\drawmomentaroundz{\summarybendingmoment}{${\sqrt{5} \hspace{.1ex} F \hspace{-0.1ex} \ell}$}{below, shape=circle, inner sep=0pt, outer sep=.5pt}

\def\couplecolor{red}
\drawmomentaroundx{\valueforFl}{0}{${F \hspace{-0.1ex} \ell}$}{pos=.7, right, shape=circle, inner sep=0pt, outer sep=4pt}

\pgfmathsetmacro\dmax{\widthofsquare * cos(atan(1/2))}

\draw [ aux line thick, opacity=.8, rotate around={atan(-2):(0, 0)} ]
	( \offsetforsigmaepure*\widthofsquare, -\offsetforsigmaepure*\heightofsquare ) -- ++( {(1/sqrt(5))*\heightofsquare*cos(atan(1/2))}, {(1/sqrt(5))*\widthofsquare*cos(atan(1/2))} ) ;
\draw [ aux line thick, opacity=.8, rotate around={atan(-2):(0, 0)} ]
	( \offsetforsigmaepure*\widthofsquare, -\offsetforsigmaepure*\heightofsquare ) -- ++( {-(1/sqrt(5))*\heightofsquare*cos(atan(1/2))}, {-(1/sqrt(5))*\widthofsquare*cos(atan(1/2))} ) ;

\draw [ aux thick dashed, rotate around={atan(-2):(0, 0)} ]
	( .5*\widthofsquare, .5*\heightofsquare )
	-- ( {\offsetforsigmaepure*\widthofsquare + (1/sqrt(5))*\heightofsquare*cos(atan(1/2))}, {-\offsetforsigmaepure*\heightofsquare + (1/sqrt(5))*\widthofsquare*cos(atan(1/2))} ) ;

\draw [ aux thick dashed, rotate around={atan(-2):(0, 0)} ]
	( -.5*\widthofsquare, -.5*\heightofsquare )
	-- ( {\offsetforsigmaepure*\widthofsquare - (1/sqrt(5))*\heightofsquare*cos(atan(1/2))}, {-\offsetforsigmaepure*\heightofsquare - (1/sqrt(5))*\widthofsquare*cos(atan(1/2))} ) ;

\def\howmanyepurelines{8}
\pgfmathsetmacro\epurestep{\dmax / (\howmanyepurelines - 1)}

\def\maxsigma{1}
\def\epurecolor{black}

%%\draw [ aux line thick, color=black, fill=green, rotate around={atan(-2):(0, 0)} ]
%%	( \offsetforsigmaepure*\widthofsquare, -\offsetforsigmaepure*\heightofsquare ) circle ( 1.6\pgflinewidth ) ; % center of epure

\foreach \x in { 0, \epurestep, ..., \dmax } {
	\pgfmathsetmacro\y{\maxsigma * \x / \dmax}
	\draw [ aux line thin, color=\epurecolor, rotate around={atan(-2):(0, 0)}, shift={( \offsetforsigmaepure*\widthofsquare, -\offsetforsigmaepure*\heightofsquare )}, rotate=atan(1/2) ]
		( \x, 0 ) -- ++( 0, \y ) ;
}
\foreach \x in { 0, -\epurestep, ..., -\dmax } {
	\pgfmathsetmacro\y{\maxsigma * \x / \dmax}
	\draw [ aux line thin, color=\epurecolor, rotate around={atan(-2):(0, 0)}, shift={( \offsetforsigmaepure*\widthofsquare, -\offsetforsigmaepure*\heightofsquare )}, rotate=atan(1/2) ]
		( \x, 0 ) -- ++( 0, \y ) ;
}
\draw [ aux line thick, color=\epurecolor, rotate around={atan(-2):(0, 0)}, shift={( \offsetforsigmaepure*\widthofsquare, -\offsetforsigmaepure*\heightofsquare )}, rotate=atan(1/2) ]
	( \dmax, 0 )
	-- ++( 0, \maxsigma )
	node [ pos=.3, right, shape=circle, inner sep=0pt, outer sep=5pt ]
	{\scalebox{.8}{$ \upsigma_{\hspace{-0.2ex}\scalebox{.75}{max}} $}} ;

\draw [ aux line thick, color=\epurecolor, rotate around={atan(-2):(0, 0)}, shift={( \offsetforsigmaepure*\widthofsquare, -\offsetforsigmaepure*\heightofsquare )}, rotate=atan(1/2) ]
	( -\dmax, 0 )
	-- ++( 0, -\maxsigma )
	node [ pos=.25, left, shape=circle, inner sep=0pt, outer sep=5pt ]
	{\scalebox{.8}{$ \upsigma_{\hspace{-0.2ex}\scalebox{.75}{max}} $}} ;

\draw [ aux line thick, color=\epurecolor, rotate around={atan(-2):(0, 0)}, shift={( \offsetforsigmaepure*\widthofsquare, -\offsetforsigmaepure*\heightofsquare )}, rotate=atan(1/2) ]
	( -\dmax, -\maxsigma )
	-- ( \dmax, \maxsigma ) ;

\draw [ aux line thick, color=\epurecolor, rotate around={atan(-2):(0, 0)}, shift={( \offsetforsigmaepure*\widthofsquare, -\offsetforsigmaepure*\heightofsquare )}, rotate=atan(1/2) ]
	( .5*\dmax, 0 )
	-- ++( 0, .5*\maxsigma )
node [ pos=0, below left, shape=circle, inner sep=0pt, outer sep=-2pt ]
	{\scalebox{.8}{$
\upsigma_{\hspace{-0.15ex}\scalebox{.8}{$x$}\hspace{.1ex}\scalebox{.7}{(3)}}
	$}} ;

\draw [ aux line thick, color=\epurecolor, rotate around={atan(-2):(0, 0)}, shift={( \offsetforsigmaepure*\widthofsquare, -\offsetforsigmaepure*\heightofsquare )}, rotate=atan(1/2) ]
	( .5*\dmax, 0 )
	-- ++( 0, .5*\maxsigma )
node [ pos=-0.6, below left, shape=circle, inner sep=0pt, outer sep=-1pt ]
	{\scalebox{.8}{$
\upsigma_{\hspace{-0.15ex}\scalebox{.8}{$x$}\hspace{.1ex}\scalebox{.7}{(2)}}
	$}} ;

\pgfmathsetmacro\maxlengtha{.5*sqrt(\widthofsquare*\widthofsquare + \heightofsquare*\heightofsquare)}

\draw [ aux line thick, color=black, fill=yellow ]
	( \maxlengtha, 0 ) circle ( 1.6\pgflinewidth )
	node [ above right, shape=circle, inner sep=0pt, outer sep=1pt ] {\scalebox{.7}{ (1) }} ;

\draw [ aux thick dashed, rotate around={atan(-2):(0, 0)} ]
	( 0, .5*\heightofsquare )
	-- ++( {(\offsetforsigmaepure + .5 - (.5/sqrt(5))*sin(atan(1/2)))*\widthofsquare}, {-(\offsetforsigmaepure + .5 - (.5/sqrt(5))*sin(atan(1/2)))*\heightofsquare} ) ; % mid side points

\draw [ rotate around={atan(-2):(0, 0)}, aux line thick, color=black, fill=yellow ]
	( .5*\widthofsquare, 0 ) circle ( 1.6\pgflinewidth )
	node [ below right, yshift=4pt, shape=circle, inner sep=0pt, outer sep=3pt ] {\scalebox{.7}{ (2) }} ;

\draw [ rotate around={atan(-2):(0, 0)}, aux line thick, color=black, fill=yellow ]
	( 0, .5*\heightofsquare ) circle ( 1.6\pgflinewidth )
	node [ above right, shape=circle, inner sep=0pt, outer sep=0.5pt ] {\scalebox{.7}{ (3) }} ;

\end{tikzpicture}
}

\end{center}

\vspace{.4cm}

Осевые моменты инерции прямоугольного сечения:
\[
\mathfrak{I}_{\hspace{-0.1ex}\scalebox{.8}{$y$}} \hspace{-0.25ex}=\hspace{-0.1ex}
\displaystyle\frac{\raisebox{-0.18em}{$B \hspace{-0.1ex} H^{\hspace{.1ex}3}$}}{12}
\hspace{.1ex} , \hspace{.4em}
 \mathfrak{I}_{\hspace{-0.1ex}\scalebox{.8}{$z$}} \hspace{-0.25ex}=\hspace{-0.1ex}
\displaystyle\frac{\raisebox{-0.18em}{$H \hspace{-0.15ex} B^{\hspace{.1ex}3}$}}{12}
\hspace{.1ex} , \hspace{.4em}
 \mathfrak{I}_{\hspace{-0.1ex}\scalebox{.8}{$b$}} \hspace{-0.25ex}=\hspace{-0.1ex}
\displaystyle\frac{\raisebox{-0.18em}{$B \hspace{-0.1ex} H$}}{12} \Bigl(
\hspace{-0.2ex} H^{\hspace{.1ex}2} \hspace{-.2ex} \operatorname{cos}^{2} \hspace{-0.22ex} \operatorname{atan} \scalebox{.88}{$\displaystyle\frac{\raisebox{-0.2em}{$B$}}{H}$}
\hspace{-0.1ex}+\hspace{-0.25ex}
B^{\hspace{.1ex}2} \hspace{-.2ex} \operatorname{sin}^{2} \hspace{-0.22ex} \operatorname{atan} \scalebox{.88}{$\displaystyle\frac{\raisebox{-0.2em}{$B$}}{H}$} \hspace{.2ex}
\Bigr)
\]

\hspace*{-\parindent}%
\begin{minipage}{\linewidth}

\hspace{\savedparindent}
Коэффициент пропорциональности~${\mathfrak{I}_{\hspace{-0.1ex}\mathrm{\scalebox{.6}{K}}}}$ (\inquotes{геометрическая жёсткость}, \inquotes{момент инерции для кручения}) между внутренним крутящим моментом~${M_{\hspace{-0.1ex}\mathrm{\scalebox{.6}{K}}}}$ и~${G \hspace{.1ex} \Theta}$:
\[
M_{\hspace{-0.1ex}\mathrm{\scalebox{.6}{K}}} \hspace{-0.3ex}=\hspace{-0.1ex}
\mathfrak{I}_{\hspace{-0.1ex}\mathrm{\scalebox{.6}{K}}} \hspace{.1ex} G \hspace{.1ex} \Theta
\hspace{.1ex} , \hspace{.33em}
\mathfrak{I}_{\hspace{-0.1ex}\mathrm{\scalebox{.6}{K}}} \hspace{-0.3ex}=\hspace{-0.1ex}
\upbeta \hspace{.1ex} B^{\hspace{.1ex}3} \hspace{-0.2ex} H
, \hspace{.33em}
%
\upbeta \hspace{-0.2ex}=\hspace{-0.1ex}
\scalebox{.88}{$\displaystyle\frac{\raisebox{-0.18em}{$1$}}{3}$}
\hspace{-0.1ex}-\hspace{-0.1ex} \displaystyle\frac{\raisebox{-0.18em}{\scalebox{.88}{$64$}}}{\pi^5} \hspace{.2ex} \displaystyle\frac{\raisebox{-0.2em}{$B$}}{H} \displaystyle\sum_{\mathclap{\hspace{.4em}j=1,3,..}}^{\raisebox{-0.1em}{\scalebox{.75}{$\infty$}}} j^{-5} \operatorname{tanh} \hspace{-0.33ex} \left( \displaystyle\frac{\raisebox{-0.18em}{$\pi \hspace{-0.1ex} j$}}{\scalebox{.88}{$2$}} \hspace{.2ex} \displaystyle\frac{\raisebox{-0.2em}{$H$}}{B} \hspace{.1ex} \right)
\vspace{-.6em}\]
($B$~--- меньшая сторона)

\end{minipage}
\vspace{.5em}

Коэффициент пропорциональности~${W_{\hspace{-0.1ex}\mathrm{\scalebox{.6}{K}}}}$ между внутренним крутящим моментом~${M_{\hspace{-0.1ex}\mathrm{\scalebox{.6}{K}}}}$ и~максимальным касательным напряжением в~сечении (в~точке на~контуре посередине б\'{о}льшей стороны прямоугольника):
\[
M_{\hspace{-0.1ex}\mathrm{\scalebox{.6}{K}}} \hspace{-0.3ex}=\hspace{-0.1ex}
W_{\hspace{-0.1ex}\mathrm{\scalebox{.6}{K}}} \hspace{.1ex} \uptau_{\scalebox{.75}{max}}
\hspace{.1ex} , \hspace{.33em}
W_{\hspace{-0.1ex}\mathrm{\scalebox{.6}{K}}} \hspace{-0.3ex}=\hspace{-0.1ex}
\upalpha \hspace{.2ex} B^{\hspace{.1ex}2} \hspace{-0.2ex} H
, \hspace{.33em}
%
\upalpha \hspace{-0.2ex}=\hspace{-0.1ex}
\displaystyle\frac{\raisebox{-0.1em}{$\upbeta \hspace{.1ex} \pi^2$}}{8 \hspace{.4ex}
\scalebox{1}{$\displaystyle\sum_{\mathclap{\hspace{.4em}\scalebox{1}{$\scriptstyle j=1,3,..$}}}^{\raisebox{-0.1em}{\scalebox{.75}{$\infty$}}}$} \hspace{.3ex} j^{-2} \Biggl( \hspace{-.1ex} 1 - \displaystyle\frac{\raisebox{-0.1em}{\scalebox{.96}{$1$}}}{%
\scalebox{.93}{$
\operatorname{cosh} \hspace{-0.33ex} \left( \hspace{-0.1ex} \displaystyle\frac{\raisebox{-0.18em}{$\pi \hspace{-0.1ex} j$}}{\scalebox{.88}{$2$}} \hspace{.2ex} \displaystyle\frac{\raisebox{-0.2em}{$H$}}{B} \hspace{.1ex} \right)
$}} \Biggr)}
\]

\vspace{-1em}
\hspace*{-\parindent}%
\begin{minipage}{\linewidth}
\hspace{\savedparindent}
Для ${\hspace{.2ex}\scalebox{.84}{$\displaystyle\frac{\raisebox{-0.2em}{$H$}}{B}$} \hspace{-0.3ex}=\hspace{-0.2ex} 2}$:

\begin{tcolorbox}[enhanced, colback = yellow!33, arc=20pt, left=6pt, right=6pt, top=6pt, bottom=6pt, boxsep=2pt, parbox = false]
\[\begin{aligned}
\upbeta \hspace{-0.2ex}&=\hspace{-0.1ex}
0.22868167711958
\hspace{.5em}
\text{(вычислено с~точностью ${10^{-15}}$ до ${j \hspace{-0.33ex}=\hspace{-0.3ex} 1001}$)}
\\[-0.2em]
%
\upalpha \hspace{-0.2ex}&=\hspace{-0.1ex}
0.24587834540483
\hspace{.5em}
\text{(с~точностью ${10^{-15}}$ до ${j \hspace{-0.33ex}=\hspace{-0.3ex} 31622777}$)}
\end{aligned}\]
\end{tcolorbox}
\end{minipage}
\vspace{.6cm}

В~рассматриваемом прямоугольном сечении выделяются три точки~--- (1), (2) и~(3)~--- как кандидаты на самую напряжённую точку сечения.
В~точке~(1)~--- наибольшее нормальное напряжение от изгиба, касательные же напряжения в~угловых точках равны нулю.
В~точках (2) и~(3)~--- максимальные касательные напряжения от кручения.
Напряжённое состояние в~угловой точке~(1):
%%\vspace{.2cm}

\def\sideofcube{1.5}
\def\showcoordinatelength{3}

\begin{center}

\scalebox{1}{
\begin{tikzpicture}[ scale=1.2 ]

\tdplotsetmaincoords{66}{133}

\node [ outer sep=0pt, inner sep=0pt, tdplot_main_coords ]
	at ( 0, 0, 4 )
	{ \scalebox{1}{\emph{точка~(1)}} } ;

\def\stresscolor{black}
\def\stresslength{2}

\drawaxes{x}{a}{b}

\draw [ external force, color=\stresscolor, tdplot_main_coords ]
	( 0, {.5*\sideofcube}, {.5*\sideofcube} ) -- ++( -\stresslength, 0, 0 ) ;

\drawcube{\sideofcube}

\draw [ external force, color=\stresscolor, tdplot_main_coords ]
	( {\sideofcube}, {.5*\sideofcube}, {.5*\sideofcube} ) -- ++( \stresslength, 0, 0 )
	node [ pos=.98, above, inner sep=0pt, outer sep=12pt ]
	{ \scalebox{\textscale}{$ \upsigma_{\hspace{-0.2ex}\scalebox{.75}{max}} $} } ;

\end{tikzpicture}
}

\end{center}

Нормальные напряжения от изгиба
\[
\upsigma_{\hspace{-0.2ex}\scalebox{.8}{$x$}} \hspace{-0.2ex}
= \displaystyle\frac{M_{\hspace{-0.1ex}\scalebox{.8}{$z$}}}{\mathfrak{I}_{\hspace{-0.1ex}\scalebox{.8}{$z$}}} \hspace{.25ex} y
- \displaystyle\frac{M_{\hspace{-0.1ex}\scalebox{.8}{$y$}}}{\mathfrak{I}_{\hspace{-0.1ex}\scalebox{.8}{$y$}}} \hspace{.25ex} z
= \displaystyle\frac{M_{\hspace{-0.1ex}\scalebox{.8}{$b$}}}{\mathfrak{I}_{\hspace{-0.1ex}\scalebox{.8}{$b$}}} \hspace{.25ex} a
\hspace{.1ex} ,
\hspace{.4em}
M_{\hspace{-0.1ex}\scalebox{.8}{$b$}}
\hspace{-0.2ex}=\hspace{-0.2ex} \sqrt{\hspace{-0.2ex}M_{\hspace{-0.1ex}\scalebox{.8}{$z$}}^{\hspace{.2ex}\raisebox{.2em}{$\scriptstyle 2$}} \hspace{-0.2ex}+\hspace{-0.2ex} M_{\hspace{-0.1ex}\scalebox{.8}{$y$}}^{\hspace{.2ex}\raisebox{.2em}{$\scriptstyle 2$}}\hspace{.2ex}}
\]
равны нулю на \inquotes{нулевой линии} (\inquotes{нейтральной оси}), уравнение которой есть
\[
\upsigma_{\hspace{-0.2ex}\scalebox{.8}{$x$}} \hspace{-0.2ex}
=\hspace{-.1ex} 0
\hspace{.4em}\Rightarrow\hspace{.4em}
\displaystyle\frac{M_{\hspace{-0.1ex}\scalebox{.8}{$z$}}}{\mathfrak{I}_{\hspace{-0.1ex}\scalebox{.8}{$z$}}} \hspace{.25ex} y
- \displaystyle\frac{M_{\hspace{-0.1ex}\scalebox{.8}{$y$}}}{\mathfrak{I}_{\hspace{-0.1ex}\scalebox{.8}{$y$}}} \hspace{.25ex} z
= 0
\hspace{.4em}\Rightarrow\hspace{.4em}
z \hspace{-.1ex}=\hspace{-.1ex}
\displaystyle\frac{\raisebox{-0.12em}{$M_{\hspace{-0.1ex}\scalebox{.8}{$z$}}$}}{M_{\hspace{-0.1ex}\scalebox{.8}{$y$}}} \hspace{.2ex}
\displaystyle\frac{\mathfrak{I}_{\hspace{-0.1ex}\scalebox{.8}{$y$}}}{\mathfrak{I}_{\hspace{-0.1ex}\scalebox{.8}{$z$}}}
\hspace{.3ex} y
\]

Отношение моментов в~сечении
\[
M_{\hspace{-0.1ex}\scalebox{.8}{$z$}} \hspace{-.25ex}
= \hspace{-.2ex} F \hspace{-0.1ex} \ell
\hspace{.1ex} , \hspace{.4em}
M_{\hspace{-0.1ex}\scalebox{.8}{$y$}} \hspace{-.25ex}
= \hspace{-.2ex} - \hspace{.1ex} 2 \hspace{.1ex} F \hspace{-0.1ex} \ell
\hspace{.4em} \Rightarrow \hspace{.4em}
\displaystyle\frac{\raisebox{-0.12em}{$M_{\hspace{-0.1ex}\scalebox{.8}{$z$}}$}}{M_{\hspace{-0.1ex}\scalebox{.8}{$y$}}} \hspace{-.1ex}
= \hspace{-.1ex} \displaystyle\frac{\raisebox{-0.18em}{$1$}}{-2}
\hspace{.3ex} ,
\]
отношение осевых моментов инерции для этого сечения
\[
\displaystyle\frac{\mathfrak{I}_{\hspace{-0.1ex}\scalebox{.8}{$y$}}}{\mathfrak{I}_{\hspace{-0.1ex}\scalebox{.8}{$z$}}} \hspace{-.1ex}
= \hspace{-.1ex}
\displaystyle\frac{\raisebox{-0.18em}{$12 \hspace{.1ex} B \hspace{-0.1ex} H^{\hspace{.1ex}3}$}}{12 \hspace{.1ex} H \hspace{-0.15ex} B^{\hspace{.1ex}3}} \hspace{-.1ex}
= \hspace{-.25ex}
\left( \displaystyle\frac{\raisebox{-0.18em}{$H$}}{B} \hspace{.1ex} \right)^{\hspace{-.5ex}2} \hspace{-.5ex}
=\hspace{-.1ex} 4
\]
и уравнение нулевой линии
\[
z \hspace{-.1ex}=\hspace{-.1ex}
- \hspace{.1ex} 2 \hspace{.1ex} y
\]

Максимальное нормальное напряжение
\[
\upsigma_{\hspace{-0.2ex}\scalebox{.75}{max}} \hspace{-.33ex}
= \hspace{-.15ex} \displaystyle\frac{M_{\hspace{-0.1ex}\scalebox{.8}{$b$}}}{\mathfrak{I}_{\hspace{-0.1ex}\scalebox{.8}{$b$}}} \hspace{.25ex} d_{\scalebox{.75}{max}}
\hspace{.2ex} ,
\]
где ${d_{\scalebox{.75}{max}}}$~--- расстояние от нулевой линии до наиболее удалённой точки сечения.

\vspace{.4cm}
Если ${\hspace{.2ex}\scalebox{.84}{$\displaystyle\frac{\raisebox{-0.2em}{$H$}}{B}$} \hspace{-0.3ex}=\hspace{-0.2ex} 2}$, ось~$a$ проходит по~диагонали прямоугольника, а нулевая линия~---по второй диагонали, то
\nopagebreak\vspace{-0.2em}\[
d_{\scalebox{.75}{max}} \hspace{-.3ex}
= \hspace{-.2ex}
B \operatorname{cos} \hspace{.2ex} \operatorname{atan} \scalebox{.88}{$\displaystyle\frac{\raisebox{-0.2em}{$1$}}{2}$} \hspace{-.1ex}
= \hspace{-.2ex}
B \sqrt{0.8} \hspace{-.1ex}
= \hspace{-.2ex}
\scalebox{.88}{$ \displaystyle\frac{\raisebox{-0.18em}{$2$}}{\sqrt{5}} $} \hspace{.2ex} B
\]

\vspace{.2em}
Момент инерции рассматриваемого сечения вокруг оси~$b$
\[
 \mathfrak{I}_{\hspace{-0.1ex}\scalebox{.8}{$b$}} \hspace{-0.25ex}
=\hspace{-0.1ex}
\displaystyle\frac{\raisebox{-0.18em}{$B^{\hspace{.1ex}4}$}}{6} \Bigl(
\hspace{-0.1ex} 4 \operatorname{cos}^{2} \hspace{-0.22ex} \operatorname{atan} \scalebox{.88}{$\displaystyle\frac{\raisebox{-0.2em}{$1$}}{2}$}
\hspace{-0.1ex}+
\operatorname{sin}^{2} \hspace{-0.22ex} \operatorname{atan} \scalebox{.88}{$\displaystyle\frac{\raisebox{-0.2em}{$1$}}{2}$} \hspace{.2ex}
\Bigr) \hspace{-.45ex}
= \hspace{-0.1ex}
\displaystyle\frac{\raisebox{-0.18em}{$B^{\hspace{.1ex}4}$}}{6} \Bigl(
\hspace{-0.1ex} 4 \hspace{-.2ex}\cdot\hspace{-.2ex} \scalebox{.88}{$ \displaystyle\frac{\raisebox{-0.18em}{$4$}}{5} $}
\hspace{-0.1ex}+
\scalebox{.88}{$ \displaystyle\frac{\raisebox{-0.18em}{$1$}}{5} $}
\hspace{.2ex} \Bigr) \hspace{-.45ex}
= \hspace{-0.1ex}
\scalebox{.88}{$ \displaystyle\frac{\raisebox{-0.18em}{$17$}}{30} $} \hspace{.1ex} B^{\hspace{.1ex}4}
\]

Суммарный изгибающий момент в~сечении
\[
M_{\hspace{-0.1ex}\scalebox{.8}{$b$}} \hspace{-0.2ex}
=\hspace{-.2ex} \sqrt{\hspace{-0.2ex}M_{\hspace{-0.1ex}\scalebox{.8}{$z$}}^{\hspace{.2ex}\raisebox{.2em}{$\scriptstyle 2$}} \hspace{-0.2ex}+\hspace{-0.2ex} M_{\hspace{-0.1ex}\scalebox{.8}{$y$}}^{\hspace{.2ex}\raisebox{.2em}{$\scriptstyle 2$}}\hspace{.2ex}} \hspace{-.25ex}
=\hspace{-.2ex} \sqrt{5} \hspace{.1ex} F \hspace{-0.1ex} \ell
\]

Максимальное нормальное напряжение в~сечении
\[
\upsigma_{\hspace{-0.2ex}\scalebox{.75}{max}} \hspace{-.33ex}
= \hspace{-.15ex} \displaystyle\frac{\raisebox{-0.18em}{$30 \sqrt{5} \hspace{.1ex} F \hspace{-0.1ex} \ell$}}{17 B^{\hspace{.1ex}4}} \hspace{.2ex} B \sqrt{0.8}
= \hspace{-.1ex} \displaystyle\frac{\raisebox{-0.18em}{$60$}}{17} \hspace{.2ex} \displaystyle\frac{\raisebox{-0.18em}{$F \hspace{-0.1ex} \ell$}}{B^{\hspace{.1ex}3}} \hspace{-.1ex}
= \hspace{-.25ex} \left( 3 \hspace{-.15ex}+\hspace{-.15ex} \scalebox{.88}{$ \displaystyle\frac{\raisebox{-0.18em}{$9$}}{17} $} \hspace{.1ex} \right) \hspace{-.4ex} \displaystyle\frac{\raisebox{-0.18em}{$F \hspace{-0.1ex} \ell$}}{B^{\hspace{.1ex}3}}
\]

\vspace{.2cm}

Напряжённое состояние в~точках с~наибольшими касательными напряжениями от кручения:

\begin{center}

\scalebox{1}{
\begin{tikzpicture}[ scale=1.2 ]

\begin{scope}[xshift=-3cm]

 \node [ outer sep=0pt, inner sep=0pt, tdplot_main_coords ]
	at ( 0, 0, 4.2 )
	{ \scalebox{1}{\emph{точка~(2)}} } ;

\tdplotsetmaincoords{66}{133}

\def\stresscolor{black}

\drawaxes{x}{y}{z}

\def\stresslength{1.5}
\draw [ external force, color=\stresscolor, tdplot_main_coords ]
	( 0, {.5*\sideofcube}, {.5*\sideofcube} ) -- ++( -\stresslength, 0, 0 ) ;

\def\stresslength{1.2}
\draw [ external force, color=\stresscolor, tdplot_main_coords ]
	( 0, {.5*\sideofcube}, {.5*\sideofcube - .5*\stresslength} ) -- ++( 0, 0, \stresslength ) ;
\draw [ external force, color=\stresscolor, tdplot_main_coords ]
	( {.5*\sideofcube - .5*\stresslength}, {.5*\sideofcube}, 0 ) -- ++( \stresslength, 0, 0 ) ;

\drawcube{\sideofcube}

\def\stresslength{1.5}
\draw [ external force, color=\stresscolor, tdplot_main_coords ]
	( {\sideofcube}, {.5*\sideofcube}, {.5*\sideofcube} ) -- ++( \stresslength, 0, 0 )
	node [ pos=1.1, above, inner sep=0pt, outer sep=12pt ]
	{ \scalebox{\textscale}{$ \upsigma_{\hspace{-0.15ex}\scalebox{.8}{$x$}\hspace{.1ex}\scalebox{.7}{(2)}} $} } ;

\def\stresslength{1.2}
\draw [ external force, color=\stresscolor, tdplot_main_coords ]
	( {\sideofcube}, {.5*\sideofcube}, {.5*\sideofcube + .5*\stresslength} ) -- ++( 0, 0, -\stresslength )
	node [ pos=1.2, below, inner sep=0pt, outer sep=6pt ]
	{ \scalebox{\textscale}{$ \uptau_{\scalebox{.75}{max}} $} } ;
\draw [ external force, color=\stresscolor, tdplot_main_coords ]
	( {.5*\sideofcube + .5*\stresslength}, {.5*\sideofcube}, {\sideofcube} ) -- ++( -\stresslength, 0, 0 ) ;

\end{scope}

\begin{scope}[xshift=3cm]

 \node [ outer sep=0pt, inner sep=0pt, tdplot_main_coords ]
	at ( 0, 0, 4.2 )
	{ \scalebox{1}{\emph{точка~(3)}} } ;

\tdplotsetmaincoords{66}{133}

\drawaxes{x}{y}{z}

\def\stresscolor{black}

\def\stresslength{1.5}
\draw [ external force, color=\stresscolor, tdplot_main_coords ]
	( 0, {.5*\sideofcube}, {.5*\sideofcube} ) -- ++( -\stresslength, 0, 0 ) ;

\def\stresslength{1}
\draw [ external force, color=\stresscolor, tdplot_main_coords ]
	( 0, {.5*\sideofcube + .5*\stresslength}, {.5*\sideofcube} ) -- ++( 0, -\stresslength, 0 ) ;
\draw [ external force, color=\stresscolor, tdplot_main_coords ]
	( {.5*\sideofcube + .5*\stresslength}, 0, {.5*\sideofcube} ) -- ++( -\stresslength, 0, 0 ) ;

\drawcube{\sideofcube}

\def\stresslength{1.5}
\draw [ external force, color=\stresscolor, tdplot_main_coords ]
	( {\sideofcube}, {.5*\sideofcube}, {.5*\sideofcube} ) -- ++( \stresslength, 0, 0 )
	node [ pos=1.1, above, inner sep=0pt, outer sep=11pt ]
	{ \scalebox{\textscale}{$ \upsigma_{\hspace{-0.15ex}\scalebox{.8}{$x$}\hspace{.1ex}\scalebox{.7}{(3)}} $} } ;

\def\stresslength{1}
\draw [ external force, color=\stresscolor, tdplot_main_coords ]
	( {\sideofcube}, {.5*\sideofcube - .5*\stresslength}, {.5*\sideofcube} ) -- ++( 0, \stresslength, 0 )
	node [ pos=.84, below, inner sep=0pt, outer sep=5.5pt ]
	{ \scalebox{\textscale}{$ {\uptau\hspace{.1ex}}' $} } ;
\draw [ external force, color=\stresscolor, tdplot_main_coords ]
	( {.5*\sideofcube - .5*\stresslength}, {\sideofcube}, {.5*\sideofcube} ) -- ++( \stresslength, 0, 0 ) ;

\pgfmathsetmacro\smallertextscale{.88*\textscale}
 \node [ outer sep=0pt, inner sep=0pt ]
	at ( 0, -1.5 )
	{\scalebox{\smallertextscale}{$
%%{\uptau\hspace{.1ex}}' \hspace{-.25ex}\approx\hspace{-.1ex} 0.795 \hspace{.2ex} \uptau_{\scalebox{.75}{max}}
{\uptau\hspace{.1ex}}' \hspace{-.3ex}<\hspace{-.1ex} \uptau_{\scalebox{.75}{max}}
	$}} ;

\end{scope}

\end{tikzpicture}
}

\end{center}

Имеем
\nopagebreak\vspace{-1em}\[\begin{gathered}
\upsigma_{\hspace{-0.15ex}\scalebox{.8}{$x$}\hspace{.1ex}\scalebox{.7}{(2)}} \hspace{-.33ex}
= \hspace{-.2ex}
\upsigma_{\hspace{-0.15ex}\scalebox{.8}{$x$}\hspace{.1ex}\scalebox{.7}{(3)}} \hspace{-.33ex}
= \hspace{-.2ex}
\scalebox{.88}{$ \displaystyle\frac{\raisebox{-0.18em}{$1$}}{2} $} \hspace{.15ex} \upsigma_{\hspace{-0.2ex}\scalebox{.75}{max}} \hspace{-.3ex}
= \hspace{-.1ex} \displaystyle\frac{\raisebox{-0.18em}{$30$}}{17} \hspace{.2ex} \displaystyle\frac{\raisebox{-0.18em}{$F \hspace{-0.1ex} \ell$}}{B^{\hspace{.1ex}3}}
\hspace{.2ex} ,
\\[.4em]
%
\uptau_{\scalebox{.75}{max}} \hspace{-.3ex}
= \hspace{-.1ex} \displaystyle\frac{\raisebox{-0.1em}{$M_{\hspace{-0.1ex}\mathrm{\scalebox{.6}{K}}}$}}{W_{\hspace{-0.1ex}\mathrm{\scalebox{.6}{K}}}}
\hspace{.2ex} , \hspace{.4em}
W_{\hspace{-0.1ex}\mathrm{\scalebox{.6}{K}}} \hspace{-0.3ex}=\hspace{-0.1ex}
\upalpha \hspace{.2ex} B^{\hspace{.1ex}2} \hspace{-0.2ex} H \hspace{-0.3ex}=\hspace{-0.1ex} 2 \upalpha \hspace{.1ex} B^{\hspace{.1ex}3} \hspace{-.3ex}
, \hspace{.4em}
2 \upalpha \approx 0.4917566908 \hspace{.1ex} ,
\\[-.3em]
%
M_{\hspace{-0.1ex}\mathrm{\scalebox{.6}{K}}} \hspace{-.25ex}
= \hspace{-.2ex} F \hspace{-0.1ex} \ell
%%\hspace{.1ex} ,
\\[-.1em]
%
\Rightarrow\hspace{.4em}
\uptau_{\scalebox{.75}{max}} \hspace{-.3ex}
= \hspace{-.1ex} \displaystyle\frac{\raisebox{-0.18em}{$F \hspace{-0.1ex} \ell$}}{2 \upalpha \hspace{.1ex} B^{\hspace{.1ex}3}}
\hspace{.2ex} , \hspace{.6em}
%
\scalebox{.93}{$ \displaystyle\frac{\raisebox{-0.2em}{$1$}}{2 \upalpha} $}
\approx 2.0335259666
\hspace{.1ex} , \hspace{.4em}
\hspace{-.25ex}
\left( \hspace{.1ex} \scalebox{.93}{$ \displaystyle\frac{\raisebox{-0.2em}{$1$}}{2 \upalpha} $} \hspace{.15ex} \right)^{\hspace{-.5ex}2} \hspace{-.4ex}
\approx 4.1352278567
\end{gathered}\]

По энергетическому критерию прочности эквивалентное одноосное напряжение будет равно
\nopagebreak\vspace{.1em}\[\begin{aligned}
\upsigma_{\scalebox{.7}{(2)}}^{\hspace{.2ex}\scalebox{.75}{э}} \hspace{-0.15ex}
&=\hspace{-0.1ex} \sqrt{\hspace{-0.1ex}
\upsigma_{\hspace{-0.15ex}\scalebox{.8}{$x$}\hspace{.1ex}\scalebox{.7}{(2)}}^{\hspace{.2ex}\raisebox{.2em}{$\scriptstyle 2$}} \hspace{-.3ex}+\hspace{-.1ex} 3 \hspace{.1ex} \uptau_{\scalebox{.75}{max}}^{\hspace{.2ex}\raisebox{.2em}{$\scriptstyle 2$}}
\hspace{.2ex}}
\\[.3em]
&=
\displaystyle\frac{\raisebox{-0.18em}{$F \hspace{-0.1ex} \ell$}}{B^{\hspace{.1ex}3}} \hspace{.1ex}
\sqrt{\hspace{.1ex}
\scalebox{.93}{$ \displaystyle\frac{\raisebox{-0.18em}{$30^{2}$}}{17^{2}} $} \hspace{-.15ex}+\hspace{-.2ex} 12.4056835702
\hspace{.2ex}}
\\[.1em]
&\approx
3.9395266748 \hspace{.2ex}
\displaystyle\frac{\raisebox{-0.18em}{$F \hspace{-0.1ex} \ell$}}{B^{\hspace{.1ex}3}}
\end{aligned}\]

Самой напряжённой точкой сечения оказалась точка на середине длинной стороны контура прямоугольника~--- точка~(2)~--- с~эквивалентным напряжением
\[
\upsigma_{\hspace{-.15ex}\scalebox{.75}{max}}^{\hspace{.2ex}\scalebox{.75}{э}} \hspace{-0.2ex}
= \upnu \hspace{.2ex}
\displaystyle\frac{\raisebox{-0.18em}{$F \hspace{-0.1ex} \ell$}}{B^{\hspace{.1ex}3}}
\hspace{.2ex} , \hspace{.5em}
\upnu \hspace{-.25ex}=\hspace{-.2ex} 3.9395266748
\]

Размер $B$ (ширину) сечения найдём из условия прочности
\nopagebreak\vspace{-.1em}\[\begin{gathered}
\upsigma_{\hspace{-.15ex}\scalebox{.75}{max}}^{\hspace{.2ex}\scalebox{.75}{э}} \hspace{-0.2ex}
=\hspace{-.1ex} \displaystyle\frac{\raisebox{-.05em}{$ \sigma_{\mathrm{\scalebox{.6}{T}}} $}}{\raisebox{.1em}{$ n_{\mathrm{\scalebox{.6}{T}}} $}}
\\[.4em]
%
\upnu \hspace{.2ex}
\displaystyle\frac{\raisebox{-0.18em}{$F \hspace{-0.1ex} \ell$}}{B^{\hspace{.1ex}3}} \hspace{-.1ex}
=\hspace{-.1ex} \displaystyle\frac{\raisebox{-.05em}{$ \sigma_{\mathrm{\scalebox{.6}{T}}} $}}{\raisebox{.1em}{$ n_{\mathrm{\scalebox{.6}{T}}} $}}
\hspace{.4em}\Rightarrow\hspace{.33em}
B^{\hspace{.1ex}3} \hspace{-0.33ex}
=\hspace{-.2ex} \upnu F \hspace{-0.1ex} \ell
\hspace{.25ex} \displaystyle\frac{\raisebox{-.05em}{$ n_{\mathrm{\scalebox{.6}{T}}} $}}{\raisebox{.1em}{$ \sigma_{\mathrm{\scalebox{.6}{T}}} $}}
\vspace{-.4em}\end{gathered}\]

Для данных
\nopagebreak\vspace{-.5em}\[
F \hspace{-.5ex}=\hspace{-.4ex} 1000\:\text{Н}
, \hspace{.4em}
\ell \!=\! 200\:\text{мм}
, \hspace{.4em}
\sigma_{\mathrm{\scalebox{.6}{T}}} \hspace{-.5ex}=\hspace{-.3ex} 300\:\text{МПа}
, \hspace{.4em}
n_{\mathrm{\scalebox{.6}{T}}} \hspace{-.44ex}=\hspace{-.33ex} 2
\vspace{-.5em}\]
окончательно находим
\[\begin{multlined}
B \hspace{-0.2ex}
= \hspace{-0.2ex} \sqrt[\scalebox{.7}{$3$}]{%
\upnu F \hspace{-0.1ex} \ell
\hspace{.25ex} \displaystyle\frac{\raisebox{-.05em}{$ n_{\mathrm{\scalebox{.6}{T}}} $}}{\raisebox{.1em}{$ \sigma_{\mathrm{\scalebox{.6}{T}}} $}}
\hspace{.2ex}}
= \hspace{-0.2ex} \sqrt[\scalebox{.7}{$3$}]{ \hspace{.2ex}
\scalebox{.93}{$ \displaystyle\frac{\raisebox{-.15em}{$ 3.9395266748 \cdot 1000\:\text{Н}\cdot 200\:\text{мм} \cdot 2 $}}{300\:\text{МПа}} $}
\hspace{.2ex}} \hspace{-.1ex}
\approx\hspace{-.1ex}
17.3831\:\text{мм}
\\
\hspace{.25em}\Rightarrow\hspace{.25em}
B \!=\hspace{-0.25ex} 17.5\:\text{мм}
\end{multlined}\]

% final figure

\begin{center}
\vspace{\fill}

\scalebox{1.33}{
\begin{tikzpicture}[ scale=1.2 ]

\begin{scope}[xshift=-2.3cm]
\def\radiusofcircle{1}
\drawcircularsection

\def\dimensionlinelength{1}
\def\angleofdimension{123}
\drawcircularsectiondimensionscustom{\scalebox{1.2}{\diameter}32}
\end{scope}

\begin{scope}[xshift=2.3cm]
\def\widthofsquare{1.3}
\pgfmathsetmacro\heightofsquare{2*\widthofsquare}

\drawrectangularsection

\def\dimensionlinelength{1.1}
\drawrectangularsectiondimensions{17.5}{pos=.5, above, inner sep=0pt, outer sep=5pt}{35}{pos=.51, above, inner sep=0pt, outer sep=5pt, rotate=90}
\end{scope}

\end{tikzpicture}
}

\vspace{\fill}
\vspace{1cm}
\end{center}

% ~ ~ ~ ~
\newpage
% ~ ~ ~ ~

%% \begin{comment} %%

\pgfmathsetmacro\xfirst{\reflength}
\pgfmathsetmacro\xfirsttosecond{\reflength}
\pgfmathsetmacro\xsecond{\xfirst + abs(\xfirsttosecond)}
\pgfmathsetmacro\zfirst{-\reflength}
\pgfmathsetmacro\zsecond{-\reflength}

\newcommand\drawpartofbeamalongx{
	\draw [ beam line, tdplot_main_coords ]
		( 0, 0, 0 ) -- ++( \xsecond, 0, 0 ) ;
}

\newcommand\drawfirstverticalpartofbeam{
	\draw [ beam line, tdplot_main_coords ]
		($ ( 0, 0, 0 ) + ( \xfirst, 0, 0 ) $) -- ++( 0, 0, \zfirst ) ;

	% corners

	\pgfmathsetmacro\cornerzsign{\zfirst / abs(\zfirst)}

	\draw [ beam line, fill=black, tdplot_main_coords ]
		($ ( 0, 0, 0 ) + ( \xfirst, 0, 0 ) $)
		-- ++( -\cornerlength, 0, 0 )
		-- ++( \cornerlength, 0, {\cornerzsign*\cornerlength} )
		-- cycle ;

	\draw [ beam line, fill=black, tdplot_main_coords ]
		($ ( 0, 0, 0 ) + ( \xfirst, 0, 0 ) $)
		-- ++( \cornerlength, 0, 0 )
		-- ++( -\cornerlength, 0, {\cornerzsign*\cornerlength} )
		-- cycle ;
}

\newcommand\drawsecondverticalpartofbeam{
	\draw [ beam line, tdplot_main_coords ]
		($ ( 0, 0, 0 ) + ( \xsecond, 0, 0 ) $) -- ++( 0, 0, \zsecond ) ;

	% corners

	\pgfmathsetmacro\cornerzsign{\zsecond / abs(\zsecond)}

	\draw [ beam line, fill=black, tdplot_main_coords ]
		($ ( 0, 0, 0 ) + ( \xsecond, 0, 0 ) $)
		-- ++( -\cornerlength, 0, 0 )
		-- ++( \cornerlength, 0, {\cornerzsign*\cornerlength} )
		-- cycle ;
}

\newcommand\drawbeam{
	\drawpartofbeamalongx
	\drawfirstverticalpartofbeam
	\drawsecondverticalpartofbeam
}

\newcommand\drawbeamtwo{
	\drawpartofbeamalongx
	\drawfirstverticalpartofbeam
}

\newcommand\drawtextforpartofbeamalongx{
	\pgfmathsetmacro\firstpartlengthmultiplier{\xfirst / \reflength}

	\node [ above, shape=circle, inner sep=0pt, outer sep=8pt, tdplot_main_coords ]
		at ($ ( 0, 0, 0 ) + ( .53*\xfirst, 0, 0 ) $)
	{\scalebox{\textscale}{$
\pgfmathparse{(\firstpartlengthmultiplier == 1) ? 0 : 1}\ifdim\pgfmathresult pt>0pt%
	\pgfmathprintnumber[ precision=3, fixed, zerofill=false ]\firstpartlengthmultiplier\hspace{.1ex}
\fi
\ell
	$}} ;

	\pgfmathsetmacro\secondpartlengthmultiplier{\xfirsttosecond / \reflength}

	\node [ above, shape=circle, inner sep=0pt, outer sep=8pt, tdplot_main_coords ]
		at ($ ( \xfirst, 0, 0 ) + ( .51*\xfirsttosecond, 0, 0 ) $)
	{\scalebox{\textscale}{$
\pgfmathparse{(\secondpartlengthmultiplier == 1) ? 0 : 1}\ifdim\pgfmathresult pt>0pt%
	\pgfmathprintnumber[ precision=3, fixed, zerofill=false ]\secondpartlengthmultiplier\hspace{.1ex}
\fi
\ell
	$}} ;
}

\newcommand\drawtextforfirstverticalpartofbeam{
	\pgfmathsetmacro\firstverticalpartlengthmultiplier{abs(\zfirst / \reflength)}

	\node [ right, inner sep=0pt, outer sep=7pt, tdplot_main_coords ]
		at ($ ( 0, 0, 0 ) + ( \xfirst, 0, .5*\zfirst ) $)
	{\scalebox{\textscale}{$
\pgfmathparse{(\firstverticalpartlengthmultiplier == 1) ? 0 : 1}\ifdim\pgfmathresult pt>0pt%
	\pgfmathprintnumber[ precision=3, fixed, zerofill=false ]\firstverticalpartlengthmultiplier\hspace{.1ex}
\fi
\ell
	$}} ;
}

\newcommand\drawtextforsecondverticalpartofbeam{
	\pgfmathsetmacro\secondverticalpartlengthmultiplier{abs(\zsecond / \reflength)}

	\node [ right, inner sep=0pt, outer sep=7pt, tdplot_main_coords ]
		at ($ ( \xfirst, 0, 0 ) + ( \xfirsttosecond, 0, .5*\zsecond ) $)
	{\scalebox{\textscale}{$
\pgfmathparse{(\secondverticalpartlengthmultiplier == 1) ? 0 : 1}\ifdim\pgfmathresult pt>0pt%
	\pgfmathprintnumber[ precision=3, fixed, zerofill=false ]\secondverticalpartlengthmultiplier\hspace{.1ex}
\fi
\ell
	$}} ;
}

\newcommand\drawbeamtext{
	\drawtextforpartofbeamalongx
	\drawtextforfirstverticalpartofbeam
	\drawtextforsecondverticalpartofbeam
}

\newcommand\drawbeamtwotext{
	\drawtextforpartofbeamalongx
	\drawtextforfirstverticalpartofbeam
}

\def\firstforcepointx{ \xsecond }
\def\firstforcepointy{ 0 }
\def\firstforcepointz{ \zsecond }
\def\firstforcevectorx{ 0 }
\def\firstforcevectory{ \justP }
\def\firstforcevectorz{ 0 }

\def\secondforcepointx{ \xfirst }
\def\secondforcepointy{ 0 }
\def\secondforcepointz{ \zfirst }
\def\secondforcevectorx{ 0 }
\def\secondforcevectory{ -\justP }
\def\secondforcevectorz{ 0 }

\def\thirdforcepointx{ \xsecond }
\def\thirdforcepointy{ 0 }
\def\thirdforcepointz{ 0 }
\def\thirdforcevectorx{ 0 }
\def\thirdforcevectory{ 0 }
\def\thirdforcevectorz{ -\justP }

\renewcommand\drawepureofinternalmomentfromfirstforce{
	\def\epurecolor{\externalforcecolor}

	% along x

	\pgfmathsetmacro\xmax{\xsecond}
	\foreach \xhatch in { 0, \hatchstep, ..., \xmax } {
		\calculateinternalmomentatpointfromforce{\xhatch}{0}{0}%
			{\firstforcepointx}{\firstforcepointy}{\firstforcepointz}%
			{\firstforcevectorx}{\firstforcevectory}{\firstforcevectorz}

		\drawepurelinesatpointalongx{\xhatch}{0}{0}%
			{\epurecolor}{\epurecolor}{\epurecolor}
	}

	\drawtwistingoffsetlineat{0}{0}{0}
	\drawtwistingoffsetlineat{\xfirst}{0}{0}
	\drawtwistingoffsetlineat{\xsecond}{0}{0}

	\calculateinternalmomentatpointfromforce{0}{0}{0}%
		{\firstforcepointx}{\firstforcepointy}{\firstforcepointz}%
		{\firstforcevectorx}{\firstforcevectory}{\firstforcevectorz}
	\drawepurelinesatpointalongx{0}{0}{0}%
		{\epurecolor}{\epurecolor}{\epurecolor}%
		[thick]

	\saveepureendpoints

	\calculateinternalmomentatpointfromforce{\xfirst}{0}{0}%
		{\firstforcepointx}{\firstforcepointy}{\firstforcepointz}%
		{\firstforcevectorx}{\firstforcevectory}{\firstforcevectorz}
	\drawepurelinesatpointalongx{\xfirst}{0}{0}%
		{\epurecolor}{\epurecolor}{\epurecolor}%
		[thick]

	\drawlinebetweensavedandcurrentz

	\saveepureendpoints

	\calculateinternalmomentatpointfromforce{\xsecond}{0}{0}%
		{\firstforcepointx}{\firstforcepointy}{\firstforcepointz}%
		{\firstforcevectorx}{\firstforcevectory}{\firstforcevectorz}
	\drawepurelinesatpointalongx{\xsecond}{0}{0}%
		{\epurecolor}{\epurecolor}{\epurecolor}%
		[thick]

	\drawlinebetweensavedandcurrentz

	\def\spiralaxialstep{1.1}
	\def\spiralinitialangle{-100}
	\def\wherefirstarrowonspiral{0.05}
	\drawtwistingspiralalongxbetween{\xsecond}{0}{0}{0}

	% along z

	\pgfmathsetmacro\zfrom{0}
	\pgfmathsetmacro\zto{\zsecond}
	\pgfmathsetmacro\zstep{\hatchstep * ( \zto - \zfrom ) / abs( \zto - \zfrom )}
	\pgfmathsetmacro\znext{\zfrom + \zstep}
	\foreach \zhatch in { \zfrom, \znext, ..., \zto } {
		\calculateinternalmomentatpointfromforce{\xsecond}{0}{\zhatch}%
			{\firstforcepointx}{\firstforcepointy}{\firstforcepointz}%
			{\firstforcevectorx}{\firstforcevectory}{\firstforcevectorz}

		\drawepurelinesatpointalongz{\xsecond}{0}{\zhatch}%
			{\epurecolor}{\epurecolor}{\epurecolor}
	}

	\calculateinternalmomentatpointfromforce{\xsecond}{0}{0}%
		{\firstforcepointx}{\firstforcepointy}{\firstforcepointz}%
		{\firstforcevectorx}{\firstforcevectory}{\firstforcevectorz}
	\drawepurelinesatpointalongz{\xsecond}{0}{0}%
		{\epurecolor}{\epurecolor}{\epurecolor}%
		[thick]

	\saveepureendpoints

	\calculateinternalmomentatpointfromforce{\xsecond}{0}{\zsecond}%
		{\firstforcepointx}{\firstforcepointy}{\firstforcepointz}%
		{\firstforcevectorx}{\firstforcevectory}{\firstforcevectorz}
	\drawepurelinesatpointalongz{\xsecond}{0}{\zsecond}%
		{\epurecolor}{\epurecolor}{\epurecolor}%
		[thick]

	\drawlinebetweensavedandcurrentx
}

\renewcommand\drawepureofinternalmomentfromsecondforce{
	\def\epurecolor{\externalforcecolor}

	% along x

	\pgfmathsetmacro\xmax{\xfirst}
	\foreach \xhatch in { 0, \hatchstep, ..., \xmax } {
		\calculateinternalmomentatpointfromforce{\xhatch}{0}{0}%
			{\secondforcepointx}{\secondforcepointy}{\secondforcepointz}%
			{\secondforcevectorx}{\secondforcevectory}{\secondforcevectorz}

		\drawepurelinesatpointalongx{\xhatch}{0}{0}%
			{\epurecolor}{\epurecolor}{\epurecolor}
	}

	\drawtwistingoffsetlineat{0}{0}{0}
	\drawtwistingoffsetlineat{\xfirst}{0}{0}

	\calculateinternalmomentatpointfromforce{0}{0}{0}%
		{\secondforcepointx}{\secondforcepointy}{\secondforcepointz}%
		{\secondforcevectorx}{\secondforcevectory}{\secondforcevectorz}
	\drawepurelinesatpointalongx{0}{0}{0}%
		{\epurecolor}{\epurecolor}{\epurecolor}%
		[thick]

	\saveepureendpoints

	\calculateinternalmomentatpointfromforce{\xfirst}{0}{0}%
		{\secondforcepointx}{\secondforcepointy}{\secondforcepointz}%
		{\secondforcevectorx}{\secondforcevectory}{\secondforcevectorz}
	\drawepurelinesatpointalongx{\xfirst}{0}{0}%
		{\epurecolor}{\epurecolor}{\epurecolor}%
		[thick]

	\drawlinebetweensavedandcurrentz

	\drawtwistingspiralalongxbetween{\xfirst}{0}{0}{0}

	% along z

	\pgfmathsetmacro\zfrom{0}
	\pgfmathsetmacro\zto{\zfirst}
	\pgfmathsetmacro\zstep{\hatchstep * ( \zto - \zfrom ) / abs( \zto - \zfrom )}
	\pgfmathsetmacro\znext{\zfrom + \zstep}
	\foreach \zhatch in { \zfrom, \znext, ..., \zto } {
		\calculateinternalmomentatpointfromforce{\xfirst}{0}{\zhatch}%
			{\secondforcepointx}{\secondforcepointy}{\secondforcepointz}%
			{\secondforcevectorx}{\secondforcevectory}{\secondforcevectorz}

		\drawepurelinesatpointalongz{\xfirst}{0}{\zhatch}%
			{\epurecolor}{\epurecolor}{\epurecolor}
	}

	\calculateinternalmomentatpointfromforce{\xfirst}{0}{0}%
		{\secondforcepointx}{\secondforcepointy}{\secondforcepointz}%
		{\secondforcevectorx}{\secondforcevectory}{\secondforcevectorz}
	\drawepurelinesatpointalongz{\xfirst}{0}{0}%
		{\epurecolor}{\epurecolor}{\epurecolor}%
		[thick]

	\saveepureendpoints

	\calculateinternalmomentatpointfromforce{\xfirst}{0}{\zfirst}%
		{\secondforcepointx}{\secondforcepointy}{\secondforcepointz}%
		{\secondforcevectorx}{\secondforcevectory}{\secondforcevectorz}
	\drawepurelinesatpointalongz{\xfirst}{0}{\zfirst}%
		{\epurecolor}{\epurecolor}{\epurecolor}%
		[thick]

	\drawlinebetweensavedandcurrentx
}

% #1: color of first force
% #2: color of second force
% #3: color of sum
\renewcommand\drawepureofinternalmomentfrombothforces[3]{
	% along x

	\def\epurecolor{#3}

	\pgfmathsetmacro\xmin{0}
	\pgfmathsetmacro\xnext{\xmin + \hatchstep}
	\pgfmathsetmacro\xmax{\xfirst}
	\foreach \xhatch in { \xmin, \xnext, ..., \xmax } {
		\calculateinternalmomentatpointfromforce{\xhatch}{0}{0}%
			{\firstforcepointx}{\firstforcepointy}{\firstforcepointz}%
			{\firstforcevectorx}{\firstforcevectory}{\firstforcevectorz}

		\pgfmathsetmacro\firstmomentx{\momentx}
		\pgfmathsetmacro\firstmomenty{\momenty}
		\pgfmathsetmacro\firstmomentz{\momentz}

		\calculateinternalmomentatpointfromforce{\xhatch}{0}{0}%
			{\secondforcepointx}{\secondforcepointy}{\secondforcepointz}%
			{\secondforcevectorx}{\secondforcevectory}{\secondforcevectorz}

		\pgfmathsetmacro\secondmomentx{\momentx}
		\pgfmathsetmacro\secondmomenty{\momenty}
		\pgfmathsetmacro\secondmomentz{\momentz}

		\pgfmathsetmacro\momentx{\firstmomentx + \secondmomentx}
		\pgfmathsetmacro\momenty{\firstmomenty + \secondmomenty}
		\pgfmathsetmacro\momentz{\firstmomentz + \secondmomentz}

		\drawepurelinesatpointalongx{\xhatch}{0}{0}%
			{\epurecolor}{\epurecolor}{\epurecolor}
	}

	\calculateinternalmomentatpointfromforce{0}{0}{0}%
		{\firstforcepointx}{\firstforcepointy}{\firstforcepointz}%
		{\firstforcevectorx}{\firstforcevectory}{\firstforcevectorz}

	\pgfmathsetmacro\firstmomentx{\momentx}
	\pgfmathsetmacro\firstmomenty{\momenty}
	\pgfmathsetmacro\firstmomentz{\momentz}

	\calculateinternalmomentatpointfromforce{0}{0}{0}%
			{\secondforcepointx}{\secondforcepointy}{\secondforcepointz}%
			{\secondforcevectorx}{\secondforcevectory}{\secondforcevectorz}

	\pgfmathsetmacro\momentx{\momentx + \firstmomentx}
	\pgfmathsetmacro\momenty{\momenty + \firstmomenty}
	\pgfmathsetmacro\momentz{\momentz + \firstmomentz}

	\drawepurelinesatpointalongx{0}{0}{0}%
		{\epurecolor}{\epurecolor}{\epurecolor}%
		[thick]

	\pgfmathsetmacro\momentxatzero{\momentx}
	\pgfmathparse{(abs(\momentxatzero) > 0) ? 1 : 0}\ifdim\pgfmathresult pt>0pt%
		\drawtwistingoffsetlineat{0}{0}{0}
	\fi

	\saveepureendpoints

	\calculateinternalmomentatpointfromforce{\xfirst}{0}{0}%
		{\firstforcepointx}{\firstforcepointy}{\firstforcepointz}%
		{\firstforcevectorx}{\firstforcevectory}{\firstforcevectorz}

	\pgfmathsetmacro\firstmomentx{\momentx}
	\pgfmathsetmacro\firstmomenty{\momenty}
	\pgfmathsetmacro\firstmomentz{\momentz}

	\calculateinternalmomentatpointfromforce{\xfirst}{0}{0}%
			{\secondforcepointx}{\secondforcepointy}{\secondforcepointz}%
			{\secondforcevectorx}{\secondforcevectory}{\secondforcevectorz}

	\pgfmathsetmacro\momentx{\momentx + \firstmomentx}
	\pgfmathsetmacro\momenty{\momenty + \firstmomenty}
	\pgfmathsetmacro\momentz{\momentz + \firstmomentz}

	\drawepurelinesatpointalongx{\xfirst}{0}{0}%
		{\epurecolor}{\epurecolor}{\epurecolor}%
		[thick]

	\drawlinebetweensavedandcurrentz

	\pgfmathparse{(abs(\momentxatzero) > 0) ? 1 : 0}\ifdim\pgfmathresult pt>0pt%
		\pgfmathsetmacro\spiralaxialstep{.5}
		\def\spiralinitialangle{-20}
		\def\wherefirstarrowonspiral{.11}
		\drawtwistingspiralalongxbetween{\xfirst}{0}{0}{0}
	\fi

	\def\epurecolor{#1}

	\pgfmathsetmacro\xmin{\xfirst}
	\pgfmathsetmacro\xnext{\xmin + \hatchstep}
	\pgfmathsetmacro\xmax{\xsecond}
	\foreach \xhatch in { \xmin, \xnext, ..., \xmax } {
		\calculateinternalmomentatpointfromforce{\xhatch}{0}{0}%
			{\firstforcepointx}{\firstforcepointy}{\firstforcepointz}%
			{\firstforcevectorx}{\firstforcevectory}{\firstforcevectorz}

		\pgfmathsetmacro\firstmomentx{\momentx}
		\pgfmathsetmacro\firstmomenty{\momenty}
		\pgfmathsetmacro\firstmomentz{\momentz}

		\pgfmathsetmacro\secondmomentx{0}
		\pgfmathsetmacro\secondmomenty{0}
		\pgfmathsetmacro\secondmomentz{0}

		\pgfmathsetmacro\momentx{\firstmomentx + \secondmomentx}
		\pgfmathsetmacro\momenty{\firstmomenty + \secondmomenty}
		\pgfmathsetmacro\momentz{\firstmomentz + \secondmomentz}

		\drawepurelinesatpointalongx{\xhatch}{0}{0}%
			{\epurecolor}{\epurecolor}{\epurecolor}
	}

	\drawtwistingoffsetlineat{\xsecond}{0}{0}
	\drawtwistingoffsetlineat{\xfirst}{0}{0}

	\calculateinternalmomentatpointfromforce{\xfirst}{0}{0}%
		{\firstforcepointx}{\firstforcepointy}{\firstforcepointz}%
		{\firstforcevectorx}{\firstforcevectory}{\firstforcevectorz}

	\drawepurelinesatpointalongx{\xfirst}{0}{0}%
		{\epurecolor}{\epurecolor}{\epurecolor}%
		[thick]

	\saveepureendpoints

	\calculateinternalmomentatpointfromforce{\xsecond}{0}{0}%
		{\firstforcepointx}{\firstforcepointy}{\firstforcepointz}%
		{\firstforcevectorx}{\firstforcevectory}{\firstforcevectorz}

	\drawepurelinesatpointalongx{\xsecond}{0}{0}%
		{\epurecolor}{\epurecolor}{\epurecolor}%
		[thick]

	\drawlinebetweensavedandcurrentz

	\pgfmathsetmacro\spiralaxialstep{\momentx / (\justP * .8 * \reflength)}
	\def\spiralinitialangle{40}
	\def\wherefirstarrowonspiral{0.1}
	\drawtwistingspiralalongxbetween{\xsecond}{\xfirst}{0}{0}

	% along z

	\def\epurecolor{#2}

	\pgfmathsetmacro\zfrom{0}
	\pgfmathsetmacro\zto{\zfirst}
	\pgfmathsetmacro\zstep{\hatchstep * ( \zto - \zfrom ) / abs( \zto - \zfrom )}
	\pgfmathsetmacro\znext{\zfrom + \zstep}
	\foreach \zhatch in { \zfrom, \znext, ..., \zto } {
		\pgfmathsetmacro\firstmomentx{0}
		\pgfmathsetmacro\firstmomenty{0}
		\pgfmathsetmacro\firstmomentz{0}

		\calculateinternalmomentatpointfromforce{\xfirst}{0}{\zhatch}%
			{\secondforcepointx}{\secondforcepointy}{\secondforcepointz}%
			{\secondforcevectorx}{\secondforcevectory}{\secondforcevectorz}

		\pgfmathsetmacro\secondmomentx{\momentx}
		\pgfmathsetmacro\secondmomenty{\momenty}
		\pgfmathsetmacro\secondmomentz{\momentz}

		\pgfmathsetmacro\momentx{\firstmomentx + \secondmomentx}
		\pgfmathsetmacro\momenty{\firstmomenty + \secondmomenty}
		\pgfmathsetmacro\momentz{\firstmomentz + \secondmomentz}

		\drawepurelinesatpointalongz{\xfirst}{0}{\zhatch}%
			{\epurecolor}{\epurecolor}{\epurecolor}
	}

	\calculateinternalmomentatpointfromforce{\xfirst}{0}{0}%
		{\secondforcepointx}{\secondforcepointy}{\secondforcepointz}%
		{\secondforcevectorx}{\secondforcevectory}{\secondforcevectorz}

	\drawepurelinesatpointalongz{\xfirst}{0}{0}%
		{\epurecolor}{\epurecolor}{\epurecolor}%
		[thick]

	\saveepureendpoints

	\calculateinternalmomentatpointfromforce{\xfirst}{0}{\zfirst}%
		{\secondforcepointx}{\secondforcepointy}{\secondforcepointz}%
		{\secondforcevectorx}{\secondforcevectory}{\secondforcevectorz}

	\drawepurelinesatpointalongz{\xfirst}{0}{\zfirst}%
		{\epurecolor}{\epurecolor}{\epurecolor}%
		[thick]

	\drawlinebetweensavedandcurrentx

	\def\epurecolor{#1}

	\pgfmathsetmacro\zfrom{0}
	\pgfmathsetmacro\zto{\zsecond}
	\pgfmathsetmacro\zstep{\hatchstep * ( \zto - \zfrom ) / abs( \zto - \zfrom )}
	\pgfmathsetmacro\znext{\zfrom + \zstep}
	\foreach \zhatch in { \zfrom, \znext, ..., \zto } {
		\calculateinternalmomentatpointfromforce{\xsecond}{0}{\zhatch}%
			{\firstforcepointx}{\firstforcepointy}{\firstforcepointz}%
			{\firstforcevectorx}{\firstforcevectory}{\firstforcevectorz}

		\pgfmathsetmacro\firstmomentx{\momentx}
		\pgfmathsetmacro\firstmomenty{\momenty}
		\pgfmathsetmacro\firstmomentz{\momentz}

		\pgfmathsetmacro\secondmomentx{0}
		\pgfmathsetmacro\secondmomenty{0}
		\pgfmathsetmacro\secondmomentz{0}

		\pgfmathsetmacro\momentx{\firstmomentx + \secondmomentx}
		\pgfmathsetmacro\momenty{\firstmomenty + \secondmomenty}
		\pgfmathsetmacro\momentz{\firstmomentz + \secondmomentz}

		\drawepurelinesatpointalongz{\xsecond}{0}{\zhatch}%
			{\epurecolor}{\epurecolor}{\epurecolor}
	}

	\calculateinternalmomentatpointfromforce{\xsecond}{0}{0}%
		{\firstforcepointx}{\firstforcepointy}{\firstforcepointz}%
		{\firstforcevectorx}{\firstforcevectory}{\firstforcevectorz}

	\drawepurelinesatpointalongz{\xsecond}{0}{0}%
		{\epurecolor}{\epurecolor}{\epurecolor}%
		[thick]

	\saveepureendpoints

	\calculateinternalmomentatpointfromforce{\xsecond}{0}{\zsecond}%
		{\firstforcepointx}{\firstforcepointy}{\firstforcepointz}%
		{\firstforcevectorx}{\firstforcevectory}{\firstforcevectorz}

	\drawepurelinesatpointalongz{\xsecond}{0}{\zsecond}%
		{\epurecolor}{\epurecolor}{\epurecolor}%
		[thick]

	\drawlinebetweensavedandcurrentx
}

\newcommand\drawepureofinternalmomentfromthirdforce{
	\def\epurecolor{\externalforcecolor}

	% along x

	\pgfmathsetmacro\xmax{\xsecond}
	\foreach \xhatch in { 0, \hatchstep, ..., \xmax } {
		\calculateinternalmomentatpointfromforce{\xhatch}{0}{0}%
			{\thirdforcepointx}{\thirdforcepointy}{\thirdforcepointz}%
			{\thirdforcevectorx}{\thirdforcevectory}{\thirdforcevectorz}

		\drawepurelinesatpointalongx{\xhatch}{0}{0}%
			{\epurecolor}{\epurecolor}{\epurecolor}
	}

	\calculateinternalmomentatpointfromforce{0}{0}{0}%
		{\thirdforcepointx}{\thirdforcepointy}{\thirdforcepointz}%
		{\thirdforcevectorx}{\thirdforcevectory}{\thirdforcevectorz}
	\drawepurelinesatpointalongx{0}{0}{0}%
		{\epurecolor}{\epurecolor}{\epurecolor}%
		[thick]

	\saveepureendpoints

	\calculateinternalmomentatpointfromforce{\xfirst}{0}{0}%
		{\thirdforcepointx}{\thirdforcepointy}{\thirdforcepointz}%
		{\thirdforcevectorx}{\thirdforcevectory}{\thirdforcevectorz}
	\drawepurelinesatpointalongx{\xfirst}{0}{0}%
		{\epurecolor}{\epurecolor}{\epurecolor}%
		[thick]

	\drawlinebetweensavedandcurrenty
	%%\drawlinebetweensavedandcurrentz

	\saveepureendpoints

	\calculateinternalmomentatpointfromforce{\xsecond}{0}{0}%
		{\thirdforcepointx}{\thirdforcepointy}{\thirdforcepointz}%
		{\thirdforcevectorx}{\thirdforcevectory}{\thirdforcevectorz}
	\drawepurelinesatpointalongx{\xsecond}{0}{0}%
		{\epurecolor}{\epurecolor}{\epurecolor}%
		[thick]

	\drawlinebetweensavedandcurrenty
	%%\drawlinebetweensavedandcurrentz
}

% #1: color of third force
% #2: color of second force
% #3: color of sum
\newcommand\drawepureofinternalmomentfromthirdandsecondforces[3]{
	\def\externalforcecolor{#1}
	\drawepureofinternalmomentfromthirdforce

	\def\externalforcecolor{#2}
	\drawepureofinternalmomentfromsecondforce
}


\def\camerafirstangle{60} % 60
\def\camerasecondangle{140} % 140

\tdplotsetmaincoords{\camerafirstangle}{\camerasecondangle}

\begin{center}
\textbf{Задача 2\raisebox{.7ex}{\small я}}
\vspace{.4cm}

Построить эпюры изгибающих и крутящих моментов
\vspace{1cm}

\emph{Вариант №\hspace{.33ex}8}
\vspace{.8cm}

\scalebox{1.1}{
\begin{tikzpicture}[ scale=1 ]

\pgfmathsetmacro\cornerlength{\reflength / 20}

\drawclampyzat{( 0, 0, 0 )}
\drawbeam
\drawbeamtext

\def\externalforcecolor{\colorforforces}

\drawload{( {\firstforcepointx}, {\firstforcepointy}, {\firstforcepointz} )}{( {\firstforcevectorx * \forcearrowscale}, {\firstforcevectory * \forcearrowscale}, {\firstforcevectorz * \forcearrowscale} )}{pos=.1, below, inner sep=0pt, outer sep=11pt}{\scalebox{\textscale}{$ P $}}

\drawload{( {\secondforcepointx}, {\secondforcepointy}, {\secondforcepointz} )}{( {\secondforcevectorx * \forcearrowscale}, {\secondforcevectory * \forcearrowscale}, {\secondforcevectorz * \forcearrowscale} )}{pos=.6, below, inner sep=0pt, outer sep=11pt}{\scalebox{\textscale}{$ P $}}

\end{tikzpicture}
}
\end{center}

\vspace{2cm}

Для линейно-упругих систем примен\'{и}м принцип независимости действия сил, согласно которому
результат от действия \emph{всех} нагрузок аналогичен сумме действий от \emph{каждой} из нагрузок.
Поэтому, для решения задачи найдём внутренний момент от каждой внешней силы по отдельности.

\begin{center}
\vspace*{1cm}
\scalebox{1}{
\begin{tikzpicture}[ scale=1.1 ]

\begin{scope}[ yshift=4.4cm ]

\pgfmathsetmacro\cornerlength{\reflength / 32}

\drawbeam

\def\externalforcecolor{red}

\drawload{( {\firstforcepointx}, {\firstforcepointy}, {\firstforcepointz} )}{( {\firstforcevectorx * \forcearrowscale}, {\firstforcevectory * \forcearrowscale}, {\firstforcevectorz * \forcearrowscale} )}{pos=.1, below, inner sep=0pt, outer sep=11pt}{\scalebox{\textscale}{$ P $}}

\def\twistingmomentmultiplierx{0}
\def\twistingmomentmultipliery{0}
\def\twistingmomentmultiplierz{1}
\def\twistingmomentepureoffset{\reflength / 3}

\drawepureofinternalmomentfromfirstforce

\end{scope}

\begin{scope}[ yshift=-4.4cm ]

\pgfmathsetmacro\cornerlength{\reflength / 32}

\drawbeam

\def\externalforcecolor{blue}

\drawload{( {\secondforcepointx}, {\secondforcepointy}, {\secondforcepointz} )}{( {\secondforcevectorx * \forcearrowscale}, {\secondforcevectory * \forcearrowscale}, {\secondforcevectorz * \forcearrowscale} )}{pos=.6, below, inner sep=0pt, outer sep=11pt}{\scalebox{\textscale}{$ P $}}

\def\twistingmomentmultiplierx{0}
\def\twistingmomentmultipliery{1}
\def\twistingmomentmultiplierz{0}

\def\spiralaxialstep{1.1}
\def\spiralinitialangle{-140}

\drawepureofinternalmomentfromsecondforce

\end{scope}

\end{tikzpicture}
}
\end{center}

\newpage

Сложение эпюр от отдельных внешних сил даёт суммарную эпюру от всех сил, действующих на конструкцию.
Суммарная эпюра:

\begin{center}
\vspace{1cm}

\scalebox{1}{
\begin{tikzpicture}[ scale=1.2 ]

\pgfmathsetmacro\cornerlength{\reflength / 32}

\drawbeam

\def\externalforcecolor{red}

\drawload{( {\firstforcepointx}, {\firstforcepointy}, {\firstforcepointz} )}{( {\firstforcevectorx * \forcearrowscale}, {\firstforcevectory * \forcearrowscale}, {\firstforcevectorz * \forcearrowscale} )}{pos=.1, below, inner sep=0pt, outer sep=11pt}{\scalebox{\textscale}{$ P $}}

\def\externalforcecolor{blue}

\drawload{( {\secondforcepointx}, {\secondforcepointy}, {\secondforcepointz} )}{( {\secondforcevectorx * \forcearrowscale}, {\secondforcevectory * \forcearrowscale}, {\secondforcevectorz * \forcearrowscale} )}{pos=.6, below, inner sep=0pt, outer sep=11pt}{\scalebox{\textscale}{$ P $}}

\def\twistingmomentepureoffset{.4*\reflength}
\def\twistingmomentmultiplierx{0}
\def\twistingmomentmultipliery{0}
\def\twistingmomentmultiplierz{1}

\drawepureofinternalmomentfrombothforces{red}{blue}{magenta}

\end{tikzpicture}
}
\end{center}

% ~ ~ ~ ~
\newpage
% ~ ~ ~ ~

\def\camerafirstangle{55} % 60, 55
\def\camerasecondangle{130} % 135, 130

\tdplotsetmaincoords{\camerafirstangle}{\camerasecondangle}

\begin{center}
\textbf{Задача 3\raisebox{.7ex}{\small я}}
\vspace{.4cm}

Построить эпюры изгибающих и крутящих моментов
\vspace{1cm}

\emph{Вариант №\hspace{.33ex}8}
\vspace{.8cm}

\scalebox{1.1}{
\begin{tikzpicture}[ scale=1 ]

\pgfmathsetmacro\cornerlength{\reflength / 20}

\drawclampyzat{( 0, 0, 0 )}
\drawbeamtwo
\drawbeamtwotext

\def\externalforcecolor{\colorforforces}

%%\drawload{( {\firstforcepointx}, {\firstforcepointy}, {\firstforcepointz} )}{( {\firstforcevectorx * \forcearrowscale}, {\firstforcevectory * \forcearrowscale}, {\firstforcevectorz * \forcearrowscale} )}{pos=.1, below, inner sep=0pt, outer sep=11pt}{\scalebox{\textscale}{$ P $}}

\drawload{( {\secondforcepointx}, {\secondforcepointy}, {\secondforcepointz} )}{( {\secondforcevectorx * \forcearrowscale}, {\secondforcevectory * \forcearrowscale}, {\secondforcevectorz * \forcearrowscale} )}{pos=.6, below, inner sep=0pt, outer sep=11pt}{\scalebox{\textscale}{$ P $}}

\drawload{( {\thirdforcepointx}, {\thirdforcepointy}, {\thirdforcepointz} )}{( {\thirdforcevectorx * \forcearrowscale}, {\thirdforcevectory * \forcearrowscale}, {\thirdforcevectorz * \forcearrowscale} )}{pos=.33, left, inner sep=0pt, outer sep=6pt}{\scalebox{\textscale}{$ P $}}

\end{tikzpicture}
}
\end{center}

\vspace{2cm}

Для решения, как и в~предыдущей задаче, используется принцип независимости действия сил.

\begin{center}
\vspace*{1cm}
\scalebox{1}{
\begin{tikzpicture}[ scale=1.1 ]

\begin{scope}[ yshift=4.4cm ]

\pgfmathsetmacro\cornerlength{\reflength / 32}

\drawbeamtwo

\def\externalforcecolor{green}

\drawload{( {\thirdforcepointx}, {\thirdforcepointy}, {\thirdforcepointz} )}{( {\thirdforcevectorx * \forcearrowscale}, {\thirdforcevectory * \forcearrowscale}, {\thirdforcevectorz * \forcearrowscale} )}{pos=.33, left, inner sep=0pt, outer sep=6pt}{\scalebox{\textscale}{$ P $}}

\drawepureofinternalmomentfromthirdforce

\end{scope}

\begin{scope}[ yshift=-4.4cm ]

\pgfmathsetmacro\cornerlength{\reflength / 32}

\drawbeamtwo

\def\externalforcecolor{blue}

\drawload{( {\secondforcepointx}, {\secondforcepointy}, {\secondforcepointz} )}{( {\secondforcevectorx * \forcearrowscale}, {\secondforcevectory * \forcearrowscale}, {\secondforcevectorz * \forcearrowscale} )}{pos=.6, below, inner sep=0pt, outer sep=11pt}{\scalebox{\textscale}{$ P $}}

\def\twistingmomentmultiplierx{0}
\def\twistingmomentmultipliery{1}
\def\twistingmomentmultiplierz{0}

\def\spiralaxialstep{1.1}
\def\spiralinitialangle{-140}

\drawepureofinternalmomentfromsecondforce

\end{scope}

\end{tikzpicture}
}
\end{center}

\newpage

Суммарная эпюра:

\begin{center}
\vspace{1cm}

\scalebox{1}{
\begin{tikzpicture}[ scale=1.2 ]

\pgfmathsetmacro\cornerlength{\reflength / 32}

\drawbeamtwo

\def\externalforcecolor{green}

\drawload{( {\thirdforcepointx}, {\thirdforcepointy}, {\thirdforcepointz} )}{( {\thirdforcevectorx * \forcearrowscale}, {\thirdforcevectory * \forcearrowscale}, {\thirdforcevectorz * \forcearrowscale} )}{pos=.33, left, inner sep=0pt, outer sep=6pt}{\scalebox{\textscale}{$ P $}}

\def\externalforcecolor{blue}

\drawload{( {\secondforcepointx}, {\secondforcepointy}, {\secondforcepointz} )}{( {\secondforcevectorx * \forcearrowscale}, {\secondforcevectory * \forcearrowscale}, {\secondforcevectorz * \forcearrowscale} )}{pos=.6, below, inner sep=0pt, outer sep=11pt}{\scalebox{\textscale}{$ P $}}

\def\twistingmomentepureoffset{.55*\reflength}
\def\twistingmomentmultiplierx{0}
\def\twistingmomentmultipliery{-1}
\def\twistingmomentmultiplierz{0}
\def\spiralaxialstep{1.1}
\def\spiralinitialangle{-75}

\drawepureofinternalmomentfromthirdandsecondforces{green}{blue}{cyan}

\end{tikzpicture}
}
\end{center}

%% \end{comment} %%

\end{document}
