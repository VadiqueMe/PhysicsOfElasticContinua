\documentclass[tikz, border=5mm]{standalone}

\usepackage{tikz}
\usepackage{tikz-3dplot}
\usetikzlibrary{calc}
\usetikzlibrary{arrows, arrows.meta}
\usetikzlibrary{decorations.markings}

\usepackage[english]{babel}

\usepackage{xargs}
\usepackage{xifthen}

\begin{document}

\def\beamlinewidth{2.8pt}
\tikzstyle{beam line} = [ line width=\beamlinewidth, line cap=round, color=black ]

\def\auxlinewidth{.6pt}
\tikzstyle{aux line thin} = [ line width=\auxlinewidth, color=black, line cap=round ]
\tikzstyle{aux thin dashed} = [ aux line thin, dash pattern=on 0pt off 1.6\pgflinewidth ]
\tikzstyle{aux line thick} = [ line width=2.5*\auxlinewidth, color=black, line cap=round ]
\tikzstyle{aux thick dashed} = [ aux line thick, dash pattern=on 0pt off 1.6\pgflinewidth ]

\def\colorforforces{orange!33!red}
\def\externalforcecolor{\colorforforces}
\tikzstyle{external force} =
	[ line width=(2/3)*\beamlinewidth, color=\externalforcecolor, line cap=round, -{Triangle[round, length=12pt, width=8pt]} ]

\def\textscale{1.2}

\def\reflength{5} % reference length

\pgfmathsetmacro\cornerlength{\reflength / 20}

\pgfmathsetmacro\onePl{\reflength / 3}
\pgfmathsetmacro\justP{\onePl / \reflength}

\def\epurehatchsteps{12} % steps per reference length
\pgfmathsetmacro\hatchstep{\reflength / (\epurehatchsteps - 1)}

\tikzset{
	hatch distance/.store in=\hatchdistance,
	hatch distance=15pt,
	hatch thickness/.store in=\hatchthickness,
	hatch thickness=.6pt %%\auxlinewidth
}

\makeatletter
\pgfdeclarepatternformonly[\hatchdistance,\hatchthickness]{flexible hatch}
	{\pgfqpoint{0pt}{0pt}}
	{\pgfqpoint{\hatchdistance}{\hatchdistance}}
	{\pgfpoint{\hatchdistance-1pt}{\hatchdistance-1pt}}%
	{
		\pgfsetcolor{\tikz@pattern@color}
		\pgfsetlinewidth{\hatchthickness}
		\pgfpathmoveto{\pgfqpoint{0pt}{0pt}}
		\pgfpathlineto{\pgfqpoint{\hatchdistance}{\hatchdistance}}
		\pgfusepath{stroke}
	}
\makeatother

\def\clampsize{.8}
\def\clampcolor{gray}

\tikzstyle{clamp line} = [ line width=(2/3)*\beamlinewidth, line cap=round, color=\clampcolor ]
\tikzstyle{clamp line thin} = [ line width=2*\auxlinewidth, line cap=round, color=\clampcolor ]

\newcommand\drawclampyz[1]{
	\draw [ clamp line, tdplot_main_coords ] ($ #1 + ( 0, .6*\clampsize, .5*\clampsize ) $) -- ++( - .6*\clampsize, 0, 0 ) ;
	\draw [ clamp line, tdplot_main_coords ] ($ #1 + ( 0, -.6*\clampsize, .5*\clampsize ) $) -- ++( - .7*\clampsize, 0, 0 ) ;
	\draw [ clamp line, tdplot_main_coords ] ($ #1 + ( 0, .6*\clampsize, -.5*\clampsize ) $) -- ++( - .7*\clampsize, 0, 0 ) ;
	\draw [ clamp line, tdplot_main_coords ] ($ #1 + ( 0, -.6*\clampsize, -.5*\clampsize ) $) -- ++( - .7*\clampsize, 0, 0 ) ;

	\draw [ clamp line, rounded corners=.5\pgflinewidth, fill=white, tdplot_main_coords ]
		($ #1 + ( 0, .6*\clampsize, .5*\clampsize ) $)
		-- ($ #1 + ( 0, -.6*\clampsize, .5*\clampsize ) $)
		-- ($ #1 + ( 0, -.6*\clampsize, -.5*\clampsize ) $)
		-- ($ #1 + ( 0, .6*\clampsize, -.5*\clampsize ) $)
		-- cycle ;

	\draw [ clamp line thin, color=\clampcolor, tdplot_main_coords ]
		plot [ smooth, tension=.8 ] coordinates {
		($ #1 + ( - .6*\clampsize, .6*\clampsize, .5*\clampsize ) $)
		($ #1 + ( -.5*\clampsize, .3*\clampsize, .5*\clampsize ) $)
		($ #1 + ( - .75*\clampsize, -.3*\clampsize, .5*\clampsize ) $)
		($ #1 + ( - .7*\clampsize, -.6*\clampsize, .5*\clampsize ) $)
	} ;
	\draw [ clamp line thin, color=\clampcolor, tdplot_main_coords ]
		plot [ smooth, tension=.5 ] coordinates {
		($ #1 + ( - .6*\clampsize, .6*\clampsize, .5*\clampsize ) $)
		($ #1 + ( -.66*\clampsize, .6*\clampsize, .2*\clampsize ) $)
		($ #1 + ( - .55*\clampsize, .6*\clampsize, -.2*\clampsize ) $)
		($ #1 + ( - .7*\clampsize, .6*\clampsize, -.5*\clampsize ) $)
	} ;
}

% #1: point
% #2: force vector
% #3: options for node
% #4: node text
\newcommand\drawload[4]{
	\draw [ external force, tdplot_main_coords ]
		($ #1 - #2 $) -- ++#2
		node [ #3 ] { #4 } ;
}

% takes two points as cartesian {x}{y}{z} and calculates cross product of their location vectors
% placing the result into last three arguments
\newcommand\tdcrossproductcartesian[9]{
\def\crossz{ (#1) * (#5) - (#2) * (#4) }
\def\crossx{ (#2) * (#6) - (#3) * (#5) }
\def\crossy{ (#3) * (#4) - (#1) * (#6) }
\pgfmathsetmacro{#7}{\crossx}
\pgfmathsetmacro{#8}{\crossy}
\pgfmathsetmacro{#9}{\crossz}
}

% {rx}{ry}{rz} of a point in question
% {rfx}{rfy}{rfz} of force point
% {fx}{fy}{fz} of force vector
% places result in \momentx, \momenty, \momentz
\newcommand\calculateinternalmomentatpointfromforce[9]{
	% get vector d from current point to the force
	\pgfmathsetmacro\distancex{(#4) - (#1)}
	\pgfmathsetmacro\distancey{(#5) - (#2)}
	\pgfmathsetmacro\distancez{(#6) - (#3)}
	% balance of moments: M + d ✕ F = 0
	% internal moment M = - d ✕ F = F ✕ d
	\tdcrossproductcartesian%
		{#7}{#8}{#9}%
		{\distancex}{\distancey}{\distancez}%
		{\momentx}{\momenty}{\momentz}
}

\def\twistingmomentepureoffset{\reflength / 3}
\def\twistingmomentmultiplierx{0}
\def\twistingmomentmultipliery{0}
\def\twistingmomentmultiplierz{1}

% #1,#2,#3 = x,y,z
% #4 = color
\newcommandx*\drawtwistingoffsetlineat[4][4=\epurecolor]{
	\draw [ aux thick dashed, color=#4, tdplot_main_coords ]
		( #1, #2, #3 ) -- ++( {\twistingmomentmultiplierx * \twistingmomentepureoffset}, {\twistingmomentmultipliery * \twistingmomentepureoffset}, {\twistingmomentmultiplierz * \twistingmomentepureoffset} ) ;
}

%%\newcommand\drawtwistingoffsetbasebetween[6]{
%%	\draw [ aux line thick, color=\epurecolor, tdplot_main_coords ]
%%		($ ( #1, #2, #3 ) + ( {\twistingmomentmultiplierx * \twistingmomentepureoffset}, {\twistingmomentmultipliery * \twistingmomentepureoffset}, {\twistingmomentmultiplierz * \twistingmomentepureoffset} ) $)
%%		-- ($ ( #4, #5, #6 ) + ( {\twistingmomentmultiplierx * \twistingmomentepureoffset}, {\twistingmomentmultipliery * \twistingmomentepureoffset}, {\twistingmomentmultiplierz * \twistingmomentepureoffset} ) $) ;
%%}

\tikzset {
	set arrow inside/.code={\pgfqkeys{/tikz/arrow inside}{#1}},
	set arrow inside={end/.initial=>, opt/.initial=},
	/pgf/decoration/Mark/.style={
		mark/.expanded=at position #1 with {
			\noexpand\arrow[\pgfkeysvalueof{/tikz/arrow inside/opt}]{\pgfkeysvalueof{/tikz/arrow inside/end}}
		}
	},
	arrow inside/.style 2 args={
		set arrow inside={#1},
		postaction={
			decorate,decoration={
				markings,Mark/.list={#2}
			}
		}
	},
}

\def\samplespercircle{30}

\def\spiralinitialangle{-10}
\def\spiralaxialstep{1}
\def\wherefirstarrowonspiral{0.075}

% #1 = begin z, #2 = end z
% #3, #4 = x, y
% #5 = color
% needs pre-calculated \momentz
\newcommandx*\drawtwistingspiralalongzbetween[5][5=\epurecolor]{
	\pgfmathparse{(abs(\momentz) > 0) ? 1 : 0}\ifdim\pgfmathresult pt>0pt%
		\pgfmathsetmacro\momentzsign{\momentz / abs(\momentz)}
	\else
		\pgfmathsetmacro\momentzsign{0}
	\fi

	\pgfmathsetmacro\momentzmidx{#3 + \twistingmomentmultiplierx * ( \twistingmomentepureoffset + .5*abs(\momentz) )}
	\pgfmathsetmacro\momentzmidy{#4 + \twistingmomentmultipliery * ( \twistingmomentepureoffset + .5*abs(\momentz) )}
	\pgfmathsetmacro\momentzmidz{#1 + \twistingmomentmultiplierz * ( \twistingmomentepureoffset + .5*abs(\momentz) )}

	\pgfmathsetmacro\spiralradius{\momentz / 2}
	\pgfmathsetmacro\axialadvance{abs(\spiralaxialstep) * (#2 - #1) / abs(#2 - #1)}
	\pgfmathsetmacro\axialshift{-\spiralinitialangle*\axialadvance/360}
	\pgfmathsetmacro\lengthincircles{abs(#2 - #1) / abs(\spiralaxialstep)}
	\pgfmathsetmacro\endangle{\lengthincircles * 360 + \spiralinitialangle}
	\pgfmathsetmacro\howmanysamples{\lengthincircles * \samplespercircle}

	\draw[ aux line thick, color=#5, tdplot_main_coords ]
		plot [ domain=\spiralinitialangle:\endangle, variable=\t, samples=\howmanysamples ]
		( {\momentzmidx + \momentzsign*\spiralradius*sin(\t)},
		{\momentzmidy + abs(\momentzsign)*\spiralradius*cos(\t)},
			{\momentzmidz + \axialadvance*\t/360 + \axialshift} )
		[ arrow inside = {}{\wherefirstarrowonspiral, 1} ] ;
}

% #1 = begin y, #2 = end y
% #3, #4 = x, z
% #5 = color
% needs pre-calculated \momenty
\newcommandx*\drawtwistingspiralalongybetween[5][5=\epurecolor]{
	\pgfmathparse{(abs(\momenty) > 0) ? 1 : 0}\ifdim\pgfmathresult pt>0pt%
		\pgfmathsetmacro\momentysign{\momenty / abs(\momenty)}
	\else
		\pgfmathsetmacro\momentysign{0}
	\fi

	\pgfmathsetmacro\momentymidx{#3 + \twistingmomentmultiplierx * ( \twistingmomentepureoffset + .5*abs(\momenty) )}
	\pgfmathsetmacro\momentymidy{#1 + \twistingmomentmultipliery * ( \twistingmomentepureoffset + .5*abs(\momenty) )}
	\pgfmathsetmacro\momentymidz{#4 + \twistingmomentmultiplierz * ( \twistingmomentepureoffset + .5*abs(\momenty) )}

	\pgfmathsetmacro\spiralradius{\momenty / 2}
	\pgfmathsetmacro\axialadvance{abs(\spiralaxialstep) * (#2 - #1) / abs(#2 - #1)}
	\pgfmathsetmacro\axialshift{-\spiralinitialangle*\axialadvance/360}
	\pgfmathsetmacro\lengthincircles{abs(#2 - #1) / abs(\spiralaxialstep)}
	\pgfmathsetmacro\endangle{\lengthincircles * 360 + \spiralinitialangle}
	\pgfmathsetmacro\howmanysamples{\lengthincircles * \samplespercircle}

	\draw[ aux line thick, color=#5, tdplot_main_coords ]
		plot [ domain=\spiralinitialangle:\endangle, variable=\t, samples=\howmanysamples ]
		( {\momentymidx - \momentysign*\spiralradius*sin(\t)},
			{\momentymidy + \axialadvance*\t/360 + \axialshift},
		{\momentymidz + abs(\momentysign)*\spiralradius*cos(\t)} )
		[ arrow inside = {}{\wherefirstarrowonspiral, 1} ] ;
}

% #1 = begin x, #2 = end x
% #3, #4 = y, z
% #5 = color
% needs pre-calculated \momentx
\newcommandx*\drawtwistingspiralalongxbetween[5][5=\epurecolor]{
	\pgfmathparse{(abs(\momentx) > 0) ? 1 : 0}\ifdim\pgfmathresult pt>0pt%
		\pgfmathsetmacro\momentxsign{\momentx / abs(\momentx)}
	\else
		\pgfmathsetmacro\momentxsign{0}
	\fi

	\pgfmathsetmacro\momentxmidx{#1 + \twistingmomentmultiplierx * ( \twistingmomentepureoffset + .5*abs(\momentx) )}
	\pgfmathsetmacro\momentxmidy{#3 + \twistingmomentmultipliery * ( \twistingmomentepureoffset + .5*abs(\momentx) )}
	\pgfmathsetmacro\momentxmidz{#4 + \twistingmomentmultiplierz * ( \twistingmomentepureoffset + .5*abs(\momentx) )}

	\pgfmathsetmacro\spiralradius{\momentx / 2}
	\pgfmathsetmacro\axialadvance{abs(\spiralaxialstep) * (#2 - #1) / abs(#2 - #1)}
	\pgfmathsetmacro\axialshift{-\spiralinitialangle*\axialadvance/360}
	\pgfmathsetmacro\lengthincircles{abs(#2 - #1) / abs(\spiralaxialstep)}
	\pgfmathsetmacro\endangle{\lengthincircles * 360 + \spiralinitialangle}
	\pgfmathsetmacro\howmanysamples{\lengthincircles * \samplespercircle}

	\draw[ aux line thick, color=#5, tdplot_main_coords ]
		plot [ domain=\spiralinitialangle:\endangle, variable=\t, samples=\howmanysamples ]
		( {\momentxmidx + \axialadvance*\t/360 + \axialshift},
			{\momentxmidy + \momentxsign*\spiralradius*sin(\t)},
			{\momentxmidz + abs(\momentxsign)*\spiralradius*cos(\t)} )
		[ arrow inside = {}{\wherefirstarrowonspiral, 1} ] ;
}

% #1,#2,#3 are x,y,z of point
% #4,#5,#6 are colors of lines for moment around x axis, y axis and z axis
% anything in #7 for thick line
% needs pre-calculated \momentx, \momenty, \momentz
\newcommandx*\drawepurelinesatpointalongx[7][7=]{
	\ifthenelse{\isempty{#7}}%
		{\def\linestyle{aux line thin}}%
		{\def\linestyle{aux line thick}}

	\pgfmathsetmacro\momentzendx{#1 + 0}
	\pgfmathsetmacro\momentzendy{#2 - \momentz}
	\pgfmathsetmacro\momentzendz{#3 + 0}

	\pgfmathsetmacro\momentyendx{#1 + 0}
	\pgfmathsetmacro\momentyendy{#2 + 0}
	\pgfmathsetmacro\momentyendz{#3 + \momenty}

	\pgfmathsetmacro\momentxendx{#1 + \twistingmomentmultiplierx * ( \twistingmomentepureoffset + abs(\momentx) )}
	\pgfmathsetmacro\momentxendy{#2 + \twistingmomentmultipliery * ( \twistingmomentepureoffset + abs(\momentx) )}
	\pgfmathsetmacro\momentxendz{#3 + \twistingmomentmultiplierz * ( \twistingmomentepureoffset + abs(\momentx) )}

	\pgfmathsetmacro\momentxmidx{#1 + \twistingmomentmultiplierx * ( \twistingmomentepureoffset + .5*abs(\momentx) )}
	\pgfmathsetmacro\momentxmidy{#2 + \twistingmomentmultipliery * ( \twistingmomentepureoffset + .5*abs(\momentx) )}
	\pgfmathsetmacro\momentxmidz{#3 + \twistingmomentmultiplierz * ( \twistingmomentepureoffset + .5*abs(\momentx) )}

	% bending
	\draw [ \linestyle, color=#5, tdplot_main_coords ]
		( #1, #2, #3 ) -- ( \momentyendx, \momentyendy, \momentyendz ) ;
	\draw [ \linestyle, color=#6, tdplot_main_coords ]
		( #1, #2, #3 ) -- ( \momentzendx, \momentzendy, \momentzendz ) ;

	% twisting
	\pgfmathparse{(abs(\momentx) > 0) ? 1 : 0}\ifdim\pgfmathresult pt>0pt%
		\ifthenelse{\isempty{#7}}%
		{}{%
		\draw[ aux thick dashed, color=#4, tdplot_main_coords  ]
			plot [ domain=0:360, variable=\t, samples=60 ]
			( {\momentxmidx + 0}, {\momentxmidy + .5*\momentx*sin(\t)}, {\momentxmidz + .5*\momentx*cos(\t)} ) ;
		}
	\fi
}

% #1,#2,#3 are x,y,z of point
% #4,#5,#6 are colors of lines for moment around x axis, y axis and z axis
% anything in #7 for thick line
% needs pre-calculated \momentx, \momenty, \momentz
\newcommandx*\drawepurelinesatpointalongy[7][7=]{
	\ifthenelse{\isempty{#7}}%
		{\def\linestyle{aux line thin}}%
		{\def\linestyle{aux line thick}}

	\pgfmathsetmacro\momentzendx{#1 + \momentz}
	\pgfmathsetmacro\momentzendy{#2 + 0}
	\pgfmathsetmacro\momentzendz{#3 + 0}

	\pgfmathsetmacro\momentxendx{#1 + 0}
	\pgfmathsetmacro\momentxendy{#2 + 0}
	\pgfmathsetmacro\momentxendz{#3 - \momentx}

	\pgfmathsetmacro\momentyendx{#1 + \twistingmomentmultiplierx * ( \twistingmomentepureoffset + abs(\momenty) )}
	\pgfmathsetmacro\momentyendy{#2 + \twistingmomentmultipliery * ( \twistingmomentepureoffset + abs(\momenty) )}
	\pgfmathsetmacro\momentyendz{#3 + \twistingmomentmultiplierz * ( \twistingmomentepureoffset + abs(\momenty) )}

	\pgfmathsetmacro\momentymidx{#1 + \twistingmomentmultiplierx * ( \twistingmomentepureoffset + .5*abs(\momenty) )}
	\pgfmathsetmacro\momentymidy{#2 + \twistingmomentmultipliery * ( \twistingmomentepureoffset + .5*abs(\momenty) )}
	\pgfmathsetmacro\momentymidz{#3 + \twistingmomentmultiplierz * ( \twistingmomentepureoffset + .5*abs(\momenty) )}

	% bending
	\draw [ \linestyle, color=#4, tdplot_main_coords ]
		( #1, #2, #3 ) -- ( \momentxendx, \momentxendy, \momentxendz ) ;
	\draw [ \linestyle, color=#6, tdplot_main_coords ]
		( #1, #2, #3 ) -- ( \momentzendx, \momentzendy, \momentzendz ) ;

	% twisting
	\pgfmathparse{(abs(\momenty) > 0) ? 1 : 0}\ifdim\pgfmathresult pt>0pt%
		\ifthenelse{\isempty{#7}}%
		{}{%
		\draw[ aux thick dashed, color=#5, tdplot_main_coords  ]
			plot [ domain=0:360, variable=\t, samples=60 ]
			( {\momentymidx + .5*\momenty*sin(\t)}, {\momentymidy + 0}, {\momentymidz + .5*\momenty*cos(\t)} ) ;
		}
	\fi
}

% #1,#2,#3 are x,y,z of point
% #4,#5,#6 are colors of lines for moment around x axis, y axis and z axis
% anything in #7 for thick line
% needs pre-calculated \momentx, \momenty, \momentz
\newcommandx*\drawepurelinesatpointalongz[7][7=]{
	\ifthenelse{\isempty{#7}}%
		{\def\linestyle{aux line thin}}%
		{\def\linestyle{aux line thick}}

	\pgfmathsetmacro\momentxendx{#1 + 0}
	\pgfmathsetmacro\momentxendy{#2 - \momentx}
	\pgfmathsetmacro\momentxendz{#3 + 0}

	\pgfmathsetmacro\momentyendx{#1 + \momenty}
	\pgfmathsetmacro\momentyendy{#2 + 0}
	\pgfmathsetmacro\momentyendz{#3 + 0}

	\pgfmathsetmacro\momentzendx{#1 + \twistingmomentmultiplierx * ( \twistingmomentepureoffset + abs(\momentz) )}
	\pgfmathsetmacro\momentzendy{#2 + \twistingmomentmultipliery * ( \twistingmomentepureoffset + abs(\momentz) )}
	\pgfmathsetmacro\momentzendz{#3 + \twistingmomentmultiplierz * ( \twistingmomentepureoffset + abs(\momentz) )}

	\pgfmathsetmacro\momentzmidx{#1 + \twistingmomentmultiplierx * ( \twistingmomentepureoffset + .5*abs(\momentz) )}
	\pgfmathsetmacro\momentzmidy{#2 + \twistingmomentmultipliery * ( \twistingmomentepureoffset + .5*abs(\momentz) )}
	\pgfmathsetmacro\momentzmidz{#3 + \twistingmomentmultiplierz * ( \twistingmomentepureoffset + .5*abs(\momentz) )}

	% bending
	\draw [ \linestyle, color=#4, tdplot_main_coords ]
		( #1, #2, #3 ) -- ( \momentxendx, \momentxendy, \momentxendz ) ;
	\draw [ \linestyle, color=#5, tdplot_main_coords ]
		( #1, #2, #3 ) -- ( \momentyendx, \momentyendy, \momentyendz ) ;

	% twisting
	\pgfmathparse{(abs(\momentz) > 0) ? 1 : 0}\ifdim\pgfmathresult pt>0pt%
		\ifthenelse{\isempty{#7}}%
		{}{%
		\draw [ aux thick dashed, color=#6, tdplot_main_coords ]
			plot [ domain=0:360, variable=\t, samples=60 ]
			( {\momentzmidx + .5*\momentz*cos(\t)}, {\momentzmidy + .5*\momentz*sin(\t)}, {\momentzmidz + 0} ) ;
		}
	\fi
}

% #1,#2,#3 = variables for maximum moments
\newcommand\resetmaxmoments[3]{
	\pgfmathsetmacro{#1}{0}
	\pgfmathsetmacro{#2}{0}
	\pgfmathsetmacro{#3}{0}
}

% #1,#2,#3 = variables for maximum moments
% uses pre-calculated \momentx, \momenty, \momentz
\newcommand\updatemaxmoments[3]{
	\pgfmathparse{( abs(\momentx) > abs(#1) ) ? 1 : 0}%
	\ifdim\pgfmathresult pt>0pt%
		\pgfmathsetmacro{#1}{\momentx}
	\fi

	\pgfmathparse{( abs(\momenty) > abs(#2) ) ? 1 : 0}%
	\ifdim\pgfmathresult pt>0pt%
		\pgfmathsetmacro{#2}{\momenty}
	\fi

	\pgfmathparse{( abs(\momentz) > abs(#3) ) ? 1 : 0}%
	\ifdim\pgfmathresult pt>0pt%
		\pgfmathsetmacro{#3}{\momentz}
	\fi
}

\def\saveepureendpoints{
	\pgfmathsetmacro\savedmomentxendx{\momentxendx}
	\pgfmathsetmacro\savedmomentxendy{\momentxendy}
	\pgfmathsetmacro\savedmomentxendz{\momentxendz}

	\pgfmathsetmacro\savedmomentyendx{\momentyendx}
	\pgfmathsetmacro\savedmomentyendy{\momentyendy}
	\pgfmathsetmacro\savedmomentyendz{\momentyendz}

	\pgfmathsetmacro\savedmomentzendx{\momentzendx}
	\pgfmathsetmacro\savedmomentzendy{\momentzendy}
	\pgfmathsetmacro\savedmomentzendz{\momentzendz}
}

\newcommandx*\drawlinebetweensavedandcurrentz[1][1=\epurecolor]{
	\draw [ aux line thick, color=#1, tdplot_main_coords ]
		( \savedmomentzendx, \savedmomentzendy, \savedmomentzendz )
		-- ( \momentzendx, \momentzendy, \momentzendz ) ;
}

\newcommandx*\drawlinebetweensavedandcurrenty[1][1=\epurecolor]{
	\draw [ aux line thick, color=#1, tdplot_main_coords ]
		( \savedmomentyendx, \savedmomentyendy, \savedmomentyendz )
		-- ( \momentyendx, \momentyendy, \momentyendz ) ;
}

\newcommandx*\drawlinebetweensavedandcurrentx[1][1=\epurecolor]{
	\draw [ aux line thick, color=#1, tdplot_main_coords ]
		( \savedmomentxendx, \savedmomentxendy, \savedmomentxendz )
		-- ( \momentxendx, \momentxendy, \momentxendz ) ;
}

% #1 = begin x, #2 = end x
% #3, #4 = y, z
% #5 = force vector as {fx}{fy}{fz}
% #6 = force point as {x}{y}{z}
% #7, #8, #9 = colorx, colory, colorz
\newcommand\drawepuresofmomentsalongxbetween[9]{
	\pgfmathsetmacro\xfrom{#1}
	\pgfmathsetmacro\xto{#2}
	\pgfmathsetmacro\xstep{\hatchstep * ( \xto - \xfrom ) / abs( \xto - \xfrom )}
	\pgfmathsetmacro\xnext{\xfrom + \xstep}
	\pgfmathsetmacro\xnextnext{\xnext + \xstep}
}

% #1 = begin y, #2 = end y
% #3, #4 = x, z
% #5 = force vector as {fx}{fy}{fz}
% #6 = force point as {x}{y}{z}
% #7, #8, #9 = colorx, colory, colorz
\newcommand\drawepuresofmomentsalongybetween[9]{
	\pgfmathsetmacro\yfrom{#1}
	\pgfmathsetmacro\yto{#2}
	\pgfmathsetmacro\ystep{\hatchstep * ( \yto - \yfrom ) / abs( \yto - \yfrom )}
	\pgfmathsetmacro\ynext{\yfrom + \ystep}
	\pgfmathsetmacro\ynextnext{\ynext + \ystep}

	\foreach \yline in { \ynext, \ynextnext, ..., \yto } {
		\resetmaxmoments{\maxmomentx}{\maxmomenty}{\maxmomentz}

		\pgfmathsetmacro\ybefore{\yline - \ystep}
		\calculateinternalmomentatpointfromforce{#3}{\ybefore}{#4}#6#5
		\drawepurelinesatpointalongy{#3}{\ybefore}{#4}{#7}{#8}{#9}
		\updatemaxmoments{\maxmomentx}{\maxmomenty}{\maxmomentz}
		\saveepureendpoints

		\calculateinternalmomentatpointfromforce{#3}{\yline}{#4}#6#5
		\drawepurelinesatpointalongy{#3}{\yline}{#4}{#7}{#8}{#9}
		\updatemaxmoments{\maxmomentx}{\maxmomenty}{\maxmomentz}

		\pgfmathparse{(abs(\maxmomentx) > 0) ? 1 : 0}%
		\ifdim\pgfmathresult pt>0pt%
			\drawlinebetweensavedandcurrentx[#7]
		\fi
		\pgfmathparse{(abs(\maxmomentz) > 0) ? 1 : 0}%
		\ifdim\pgfmathresult pt>0pt%
			\drawlinebetweensavedandcurrentz[#9]
		\fi
	}

	\pgfmathsetmacro\spiralishere{0} % 0: false, 1: true

	\calculateinternalmomentatpointfromforce{#3}{\yfrom}{#4}#6#5
	\drawepurelinesatpointalongy{#3}{\yfrom}{#4}{#7}{#8}{#9}[thick]

	\pgfmathparse{(abs(\momenty) > 0) ? 1 : 0}%
	\ifdim\pgfmathresult pt>0pt%		
		\drawtwistingoffsetlineat{#3}{\yfrom}{#4}[#8]
		\pgfmathsetmacro\spiralishere{1}
	\fi

	\calculateinternalmomentatpointfromforce{#3}{\yto}{#4}#6#5
	\drawepurelinesatpointalongy{#3}{\yto}{#4}{#7}{#8}{#9}[thick]

	\pgfmathparse{(abs(\momenty) > 0) ? 1 : 0}%
	\ifdim\pgfmathresult pt>0pt%
		\drawtwistingoffsetlineat{#3}{\yto}{#4}[#8]
		\pgfmathsetmacro\spiralishere{1}
	\fi

	\ifdim\spiralishere pt>0pt%
		\drawtwistingspiralalongybetween{#1}{#2}{#3}{#4}[#8]
	\fi
}

% #1 = begin z, #2 = end z
% #3, #4 = x, y
% #5 = force vector as {fx}{fy}{fz}
% #6 = force point as {x}{y}{z}
% #7, #8, #9 = colorx, colory, colorz
\newcommand\drawepuresofmomentsalongzbetween[9]{
	\pgfmathsetmacro\zfrom{#1}
	\pgfmathsetmacro\zto{#2}
	\pgfmathsetmacro\zstep{\hatchstep * ( \zto - \zfrom ) / abs( \zto - \zfrom )}
	\pgfmathsetmacro\znext{\zfrom + \zstep}
	\pgfmathsetmacro\znextnext{\znext + \zstep}
}

\newcommand\drawepureofinternalmomentfromfirstforce{
	% override it
}

\newcommand\drawepureofinternalmomentfromsecondforce{
	% override it
}

% #1: color of first force
% #2: color of second force
% #3: color of sum
\newcommand\drawepureofinternalmomentfrombothforces[3]{
	% override it
}


\def\forcearrowscale{4}

\pgfmathsetmacro\xfirst{\reflength}
\pgfmathsetmacro\xfirsttosecond{\reflength}
\pgfmathsetmacro\xsecond{\xfirst + abs(\xfirsttosecond)}
\pgfmathsetmacro\zfirst{-\reflength}
\pgfmathsetmacro\zsecond{-\reflength}

\newcommand\drawpartofbeamalongx{
	\draw [ beam line, tdplot_main_coords ]
		( 0, 0, 0 ) -- ++( \xsecond, 0, 0 ) ;
}

\newcommand\drawfirstverticalpartofbeam{
	\draw [ beam line, tdplot_main_coords ]
		($ ( 0, 0, 0 ) + ( \xfirst, 0, 0 ) $) -- ++( 0, 0, \zfirst ) ;

	% corners

	\pgfmathsetmacro\cornerzsign{\zfirst / abs(\zfirst)}

	\draw [ beam line, fill=black, tdplot_main_coords ]
		($ ( 0, 0, 0 ) + ( \xfirst, 0, 0 ) $)
		-- ++( -\cornerlength, 0, 0 )
		-- ++( \cornerlength, 0, {\cornerzsign*\cornerlength} )
		-- cycle ;

	\draw [ beam line, fill=black, tdplot_main_coords ]
		($ ( 0, 0, 0 ) + ( \xfirst, 0, 0 ) $)
		-- ++( \cornerlength, 0, 0 )
		-- ++( -\cornerlength, 0, {\cornerzsign*\cornerlength} )
		-- cycle ;
}

\newcommand\drawsecondverticalpartofbeam{
	\draw [ beam line, tdplot_main_coords ]
		($ ( 0, 0, 0 ) + ( \xsecond, 0, 0 ) $) -- ++( 0, 0, \zsecond ) ;

	% corners

	\pgfmathsetmacro\cornerzsign{\zsecond / abs(\zsecond)}

	\draw [ beam line, fill=black, tdplot_main_coords ]
		($ ( 0, 0, 0 ) + ( \xsecond, 0, 0 ) $)
		-- ++( -\cornerlength, 0, 0 )
		-- ++( \cornerlength, 0, {\cornerzsign*\cornerlength} )
		-- cycle ;
}

\newcommand\drawbeam{
	\drawpartofbeamalongx
	\drawfirstverticalpartofbeam
	\drawsecondverticalpartofbeam
}

\newcommand\drawbeamtwo{
	\drawpartofbeamalongx
	\drawfirstverticalpartofbeam
}

\newcommand\drawtextforpartofbeamalongx{
	\pgfmathsetmacro\firstpartlengthmultiplier{\xfirst / \reflength}

	\node [ above, shape=circle, inner sep=0pt, outer sep=8pt, tdplot_main_coords ]
		at ($ ( 0, 0, 0 ) + ( .53*\xfirst, 0, 0 ) $)
	{\scalebox{\textscale}{$
\pgfmathparse{(\firstpartlengthmultiplier == 1) ? 0 : 1}\ifdim\pgfmathresult pt>0pt%
	\pgfmathprintnumber[ precision=3, fixed, zerofill=false ]\firstpartlengthmultiplier\hspace{.1ex}
\fi
\ell
	$}} ;

	\pgfmathsetmacro\secondpartlengthmultiplier{\xfirsttosecond / \reflength}

	\node [ above, shape=circle, inner sep=0pt, outer sep=8pt, tdplot_main_coords ]
		at ($ ( \xfirst, 0, 0 ) + ( .51*\xfirsttosecond, 0, 0 ) $)
	{\scalebox{\textscale}{$
\pgfmathparse{(\secondpartlengthmultiplier == 1) ? 0 : 1}\ifdim\pgfmathresult pt>0pt%
	\pgfmathprintnumber[ precision=3, fixed, zerofill=false ]\secondpartlengthmultiplier\hspace{.1ex}
\fi
\ell
	$}} ;
}

\newcommand\drawtextforfirstverticalpartofbeam{
	\pgfmathsetmacro\firstverticalpartlengthmultiplier{abs(\zfirst / \reflength)}

	\node [ right, inner sep=0pt, outer sep=7pt, tdplot_main_coords ]
		at ($ ( 0, 0, 0 ) + ( \xfirst, 0, .5*\zfirst ) $)
	{\scalebox{\textscale}{$
\pgfmathparse{(\firstverticalpartlengthmultiplier == 1) ? 0 : 1}\ifdim\pgfmathresult pt>0pt%
	\pgfmathprintnumber[ precision=3, fixed, zerofill=false ]\firstverticalpartlengthmultiplier\hspace{.1ex}
\fi
\ell
	$}} ;
}

\newcommand\drawtextforsecondverticalpartofbeam{
	\pgfmathsetmacro\secondverticalpartlengthmultiplier{abs(\zsecond / \reflength)}

	\node [ right, inner sep=0pt, outer sep=7pt, tdplot_main_coords ]
		at ($ ( \xfirst, 0, 0 ) + ( \xfirsttosecond, 0, .5*\zsecond ) $)
	{\scalebox{\textscale}{$
\pgfmathparse{(\secondverticalpartlengthmultiplier == 1) ? 0 : 1}\ifdim\pgfmathresult pt>0pt%
	\pgfmathprintnumber[ precision=3, fixed, zerofill=false ]\secondverticalpartlengthmultiplier\hspace{.1ex}
\fi
\ell
	$}} ;
}

\newcommand\drawbeamtext{
	\drawtextforpartofbeamalongx
	\drawtextforfirstverticalpartofbeam
	\drawtextforsecondverticalpartofbeam
}

\newcommand\drawbeamtwotext{
	\drawtextforpartofbeamalongx
	\drawtextforfirstverticalpartofbeam
}

\def\firstforcepointx{ \xsecond }
\def\firstforcepointy{ 0 }
\def\firstforcepointz{ \zsecond }
\def\firstforcevectorx{ 0 }
\def\firstforcevectory{ \justP }
\def\firstforcevectorz{ 0 }

\def\secondforcepointx{ \xfirst }
\def\secondforcepointy{ 0 }
\def\secondforcepointz{ \zfirst }
\def\secondforcevectorx{ 0 }
\def\secondforcevectory{ -\justP }
\def\secondforcevectorz{ 0 }

\def\thirdforcepointx{ \xsecond }
\def\thirdforcepointy{ 0 }
\def\thirdforcepointz{ 0 }
\def\thirdforcevectorx{ 0 }
\def\thirdforcevectory{ 0 }
\def\thirdforcevectorz{ -\justP }

\renewcommand\drawepureofinternalmomentfromfirstforce{
	\def\epurecolor{\externalforcecolor}

	% along x

	\pgfmathsetmacro\xmax{\xsecond}
	\foreach \xhatch in { 0, \hatchstep, ..., \xmax } {
		\calculateinternalmomentatpointfromforce{\xhatch}{0}{0}%
			{\firstforcepointx}{\firstforcepointy}{\firstforcepointz}%
			{\firstforcevectorx}{\firstforcevectory}{\firstforcevectorz}

		\drawepurelinesatpointalongx{\xhatch}{0}{0}%
			{\epurecolor}{\epurecolor}{\epurecolor}
	}

	\drawtwistingoffsetlineat{0}{0}{0}
	\drawtwistingoffsetlineat{\xfirst}{0}{0}
	\drawtwistingoffsetlineat{\xsecond}{0}{0}

	\calculateinternalmomentatpointfromforce{0}{0}{0}%
		{\firstforcepointx}{\firstforcepointy}{\firstforcepointz}%
		{\firstforcevectorx}{\firstforcevectory}{\firstforcevectorz}
	\drawepurelinesatpointalongx{0}{0}{0}%
		{\epurecolor}{\epurecolor}{\epurecolor}%
		[thick]

	\saveepureendpoints

	\calculateinternalmomentatpointfromforce{\xfirst}{0}{0}%
		{\firstforcepointx}{\firstforcepointy}{\firstforcepointz}%
		{\firstforcevectorx}{\firstforcevectory}{\firstforcevectorz}
	\drawepurelinesatpointalongx{\xfirst}{0}{0}%
		{\epurecolor}{\epurecolor}{\epurecolor}%
		[thick]

	\drawlinebetweensavedandcurrentz

	\saveepureendpoints

	\calculateinternalmomentatpointfromforce{\xsecond}{0}{0}%
		{\firstforcepointx}{\firstforcepointy}{\firstforcepointz}%
		{\firstforcevectorx}{\firstforcevectory}{\firstforcevectorz}
	\drawepurelinesatpointalongx{\xsecond}{0}{0}%
		{\epurecolor}{\epurecolor}{\epurecolor}%
		[thick]

	\drawlinebetweensavedandcurrentz

	\def\spiralaxialstep{1.1}
	\def\spiralinitialangle{-100}
	\def\wherefirstarrowonspiral{0.05}
	\drawtwistingspiralalongxbetween{\xsecond}{0}{0}{0}

	% along z

	\pgfmathsetmacro\zfrom{0}
	\pgfmathsetmacro\zto{\zsecond}
	\pgfmathsetmacro\zstep{\hatchstep * ( \zto - \zfrom ) / abs( \zto - \zfrom )}
	\pgfmathsetmacro\znext{\zfrom + \zstep}
	\foreach \zhatch in { \zfrom, \znext, ..., \zto } {
		\calculateinternalmomentatpointfromforce{\xsecond}{0}{\zhatch}%
			{\firstforcepointx}{\firstforcepointy}{\firstforcepointz}%
			{\firstforcevectorx}{\firstforcevectory}{\firstforcevectorz}

		\drawepurelinesatpointalongz{\xsecond}{0}{\zhatch}%
			{\epurecolor}{\epurecolor}{\epurecolor}
	}

	\calculateinternalmomentatpointfromforce{\xsecond}{0}{0}%
		{\firstforcepointx}{\firstforcepointy}{\firstforcepointz}%
		{\firstforcevectorx}{\firstforcevectory}{\firstforcevectorz}
	\drawepurelinesatpointalongz{\xsecond}{0}{0}%
		{\epurecolor}{\epurecolor}{\epurecolor}%
		[thick]

	\saveepureendpoints

	\calculateinternalmomentatpointfromforce{\xsecond}{0}{\zsecond}%
		{\firstforcepointx}{\firstforcepointy}{\firstforcepointz}%
		{\firstforcevectorx}{\firstforcevectory}{\firstforcevectorz}
	\drawepurelinesatpointalongz{\xsecond}{0}{\zsecond}%
		{\epurecolor}{\epurecolor}{\epurecolor}%
		[thick]

	\drawlinebetweensavedandcurrentx
}

\renewcommand\drawepureofinternalmomentfromsecondforce{
	\def\epurecolor{\externalforcecolor}

	% along x

	\pgfmathsetmacro\xmax{\xfirst}
	\foreach \xhatch in { 0, \hatchstep, ..., \xmax } {
		\calculateinternalmomentatpointfromforce{\xhatch}{0}{0}%
			{\secondforcepointx}{\secondforcepointy}{\secondforcepointz}%
			{\secondforcevectorx}{\secondforcevectory}{\secondforcevectorz}

		\drawepurelinesatpointalongx{\xhatch}{0}{0}%
			{\epurecolor}{\epurecolor}{\epurecolor}
	}

	\drawtwistingoffsetlineat{0}{0}{0}
	\drawtwistingoffsetlineat{\xfirst}{0}{0}

	\calculateinternalmomentatpointfromforce{0}{0}{0}%
		{\secondforcepointx}{\secondforcepointy}{\secondforcepointz}%
		{\secondforcevectorx}{\secondforcevectory}{\secondforcevectorz}
	\drawepurelinesatpointalongx{0}{0}{0}%
		{\epurecolor}{\epurecolor}{\epurecolor}%
		[thick]

	\saveepureendpoints

	\calculateinternalmomentatpointfromforce{\xfirst}{0}{0}%
		{\secondforcepointx}{\secondforcepointy}{\secondforcepointz}%
		{\secondforcevectorx}{\secondforcevectory}{\secondforcevectorz}
	\drawepurelinesatpointalongx{\xfirst}{0}{0}%
		{\epurecolor}{\epurecolor}{\epurecolor}%
		[thick]

	\drawlinebetweensavedandcurrentz

	\drawtwistingspiralalongxbetween{\xfirst}{0}{0}{0}

	% along z

	\pgfmathsetmacro\zfrom{0}
	\pgfmathsetmacro\zto{\zfirst}
	\pgfmathsetmacro\zstep{\hatchstep * ( \zto - \zfrom ) / abs( \zto - \zfrom )}
	\pgfmathsetmacro\znext{\zfrom + \zstep}
	\foreach \zhatch in { \zfrom, \znext, ..., \zto } {
		\calculateinternalmomentatpointfromforce{\xfirst}{0}{\zhatch}%
			{\secondforcepointx}{\secondforcepointy}{\secondforcepointz}%
			{\secondforcevectorx}{\secondforcevectory}{\secondforcevectorz}

		\drawepurelinesatpointalongz{\xfirst}{0}{\zhatch}%
			{\epurecolor}{\epurecolor}{\epurecolor}
	}

	\calculateinternalmomentatpointfromforce{\xfirst}{0}{0}%
		{\secondforcepointx}{\secondforcepointy}{\secondforcepointz}%
		{\secondforcevectorx}{\secondforcevectory}{\secondforcevectorz}
	\drawepurelinesatpointalongz{\xfirst}{0}{0}%
		{\epurecolor}{\epurecolor}{\epurecolor}%
		[thick]

	\saveepureendpoints

	\calculateinternalmomentatpointfromforce{\xfirst}{0}{\zfirst}%
		{\secondforcepointx}{\secondforcepointy}{\secondforcepointz}%
		{\secondforcevectorx}{\secondforcevectory}{\secondforcevectorz}
	\drawepurelinesatpointalongz{\xfirst}{0}{\zfirst}%
		{\epurecolor}{\epurecolor}{\epurecolor}%
		[thick]

	\drawlinebetweensavedandcurrentx
}

% #1: color of first force
% #2: color of second force
% #3: color of sum
\renewcommand\drawepureofinternalmomentfrombothforces[3]{
	% along x

	\def\epurecolor{#3}

	\pgfmathsetmacro\xmin{0}
	\pgfmathsetmacro\xnext{\xmin + \hatchstep}
	\pgfmathsetmacro\xmax{\xfirst}
	\foreach \xhatch in { \xmin, \xnext, ..., \xmax } {
		\calculateinternalmomentatpointfromforce{\xhatch}{0}{0}%
			{\firstforcepointx}{\firstforcepointy}{\firstforcepointz}%
			{\firstforcevectorx}{\firstforcevectory}{\firstforcevectorz}

		\pgfmathsetmacro\firstmomentx{\momentx}
		\pgfmathsetmacro\firstmomenty{\momenty}
		\pgfmathsetmacro\firstmomentz{\momentz}

		\calculateinternalmomentatpointfromforce{\xhatch}{0}{0}%
			{\secondforcepointx}{\secondforcepointy}{\secondforcepointz}%
			{\secondforcevectorx}{\secondforcevectory}{\secondforcevectorz}

		\pgfmathsetmacro\secondmomentx{\momentx}
		\pgfmathsetmacro\secondmomenty{\momenty}
		\pgfmathsetmacro\secondmomentz{\momentz}

		\pgfmathsetmacro\momentx{\firstmomentx + \secondmomentx}
		\pgfmathsetmacro\momenty{\firstmomenty + \secondmomenty}
		\pgfmathsetmacro\momentz{\firstmomentz + \secondmomentz}

		\drawepurelinesatpointalongx{\xhatch}{0}{0}%
			{\epurecolor}{\epurecolor}{\epurecolor}
	}

	\calculateinternalmomentatpointfromforce{0}{0}{0}%
		{\firstforcepointx}{\firstforcepointy}{\firstforcepointz}%
		{\firstforcevectorx}{\firstforcevectory}{\firstforcevectorz}

	\pgfmathsetmacro\firstmomentx{\momentx}
	\pgfmathsetmacro\firstmomenty{\momenty}
	\pgfmathsetmacro\firstmomentz{\momentz}

	\calculateinternalmomentatpointfromforce{0}{0}{0}%
			{\secondforcepointx}{\secondforcepointy}{\secondforcepointz}%
			{\secondforcevectorx}{\secondforcevectory}{\secondforcevectorz}

	\pgfmathsetmacro\momentx{\momentx + \firstmomentx}
	\pgfmathsetmacro\momenty{\momenty + \firstmomenty}
	\pgfmathsetmacro\momentz{\momentz + \firstmomentz}

	\drawepurelinesatpointalongx{0}{0}{0}%
		{\epurecolor}{\epurecolor}{\epurecolor}%
		[thick]

	\pgfmathsetmacro\momentxatzero{\momentx}
	\pgfmathparse{(abs(\momentxatzero) > 0) ? 1 : 0}\ifdim\pgfmathresult pt>0pt%
		\drawtwistingoffsetlineat{0}{0}{0}
	\fi

	\saveepureendpoints

	\calculateinternalmomentatpointfromforce{\xfirst}{0}{0}%
		{\firstforcepointx}{\firstforcepointy}{\firstforcepointz}%
		{\firstforcevectorx}{\firstforcevectory}{\firstforcevectorz}

	\pgfmathsetmacro\firstmomentx{\momentx}
	\pgfmathsetmacro\firstmomenty{\momenty}
	\pgfmathsetmacro\firstmomentz{\momentz}

	\calculateinternalmomentatpointfromforce{\xfirst}{0}{0}%
			{\secondforcepointx}{\secondforcepointy}{\secondforcepointz}%
			{\secondforcevectorx}{\secondforcevectory}{\secondforcevectorz}

	\pgfmathsetmacro\momentx{\momentx + \firstmomentx}
	\pgfmathsetmacro\momenty{\momenty + \firstmomenty}
	\pgfmathsetmacro\momentz{\momentz + \firstmomentz}

	\drawepurelinesatpointalongx{\xfirst}{0}{0}%
		{\epurecolor}{\epurecolor}{\epurecolor}%
		[thick]

	\drawlinebetweensavedandcurrentz

	\pgfmathparse{(abs(\momentxatzero) > 0) ? 1 : 0}\ifdim\pgfmathresult pt>0pt%
		\pgfmathsetmacro\spiralaxialstep{.5}
		\def\spiralinitialangle{-20}
		\def\wherefirstarrowonspiral{.11}
		\drawtwistingspiralalongxbetween{\xfirst}{0}{0}{0}
	\fi

	\def\epurecolor{#1}

	\pgfmathsetmacro\xmin{\xfirst}
	\pgfmathsetmacro\xnext{\xmin + \hatchstep}
	\pgfmathsetmacro\xmax{\xsecond}
	\foreach \xhatch in { \xmin, \xnext, ..., \xmax } {
		\calculateinternalmomentatpointfromforce{\xhatch}{0}{0}%
			{\firstforcepointx}{\firstforcepointy}{\firstforcepointz}%
			{\firstforcevectorx}{\firstforcevectory}{\firstforcevectorz}

		\pgfmathsetmacro\firstmomentx{\momentx}
		\pgfmathsetmacro\firstmomenty{\momenty}
		\pgfmathsetmacro\firstmomentz{\momentz}

		\pgfmathsetmacro\secondmomentx{0}
		\pgfmathsetmacro\secondmomenty{0}
		\pgfmathsetmacro\secondmomentz{0}

		\pgfmathsetmacro\momentx{\firstmomentx + \secondmomentx}
		\pgfmathsetmacro\momenty{\firstmomenty + \secondmomenty}
		\pgfmathsetmacro\momentz{\firstmomentz + \secondmomentz}

		\drawepurelinesatpointalongx{\xhatch}{0}{0}%
			{\epurecolor}{\epurecolor}{\epurecolor}
	}

	\drawtwistingoffsetlineat{\xsecond}{0}{0}
	\drawtwistingoffsetlineat{\xfirst}{0}{0}

	\calculateinternalmomentatpointfromforce{\xfirst}{0}{0}%
		{\firstforcepointx}{\firstforcepointy}{\firstforcepointz}%
		{\firstforcevectorx}{\firstforcevectory}{\firstforcevectorz}

	\drawepurelinesatpointalongx{\xfirst}{0}{0}%
		{\epurecolor}{\epurecolor}{\epurecolor}%
		[thick]

	\saveepureendpoints

	\calculateinternalmomentatpointfromforce{\xsecond}{0}{0}%
		{\firstforcepointx}{\firstforcepointy}{\firstforcepointz}%
		{\firstforcevectorx}{\firstforcevectory}{\firstforcevectorz}

	\drawepurelinesatpointalongx{\xsecond}{0}{0}%
		{\epurecolor}{\epurecolor}{\epurecolor}%
		[thick]

	\drawlinebetweensavedandcurrentz

	\pgfmathsetmacro\spiralaxialstep{\momentx / (\justP * .8 * \reflength)}
	\def\spiralinitialangle{40}
	\def\wherefirstarrowonspiral{0.1}
	\drawtwistingspiralalongxbetween{\xsecond}{\xfirst}{0}{0}

	% along z

	\def\epurecolor{#2}

	\pgfmathsetmacro\zfrom{0}
	\pgfmathsetmacro\zto{\zfirst}
	\pgfmathsetmacro\zstep{\hatchstep * ( \zto - \zfrom ) / abs( \zto - \zfrom )}
	\pgfmathsetmacro\znext{\zfrom + \zstep}
	\foreach \zhatch in { \zfrom, \znext, ..., \zto } {
		\pgfmathsetmacro\firstmomentx{0}
		\pgfmathsetmacro\firstmomenty{0}
		\pgfmathsetmacro\firstmomentz{0}

		\calculateinternalmomentatpointfromforce{\xfirst}{0}{\zhatch}%
			{\secondforcepointx}{\secondforcepointy}{\secondforcepointz}%
			{\secondforcevectorx}{\secondforcevectory}{\secondforcevectorz}

		\pgfmathsetmacro\secondmomentx{\momentx}
		\pgfmathsetmacro\secondmomenty{\momenty}
		\pgfmathsetmacro\secondmomentz{\momentz}

		\pgfmathsetmacro\momentx{\firstmomentx + \secondmomentx}
		\pgfmathsetmacro\momenty{\firstmomenty + \secondmomenty}
		\pgfmathsetmacro\momentz{\firstmomentz + \secondmomentz}

		\drawepurelinesatpointalongz{\xfirst}{0}{\zhatch}%
			{\epurecolor}{\epurecolor}{\epurecolor}
	}

	\calculateinternalmomentatpointfromforce{\xfirst}{0}{0}%
		{\secondforcepointx}{\secondforcepointy}{\secondforcepointz}%
		{\secondforcevectorx}{\secondforcevectory}{\secondforcevectorz}

	\drawepurelinesatpointalongz{\xfirst}{0}{0}%
		{\epurecolor}{\epurecolor}{\epurecolor}%
		[thick]

	\saveepureendpoints

	\calculateinternalmomentatpointfromforce{\xfirst}{0}{\zfirst}%
		{\secondforcepointx}{\secondforcepointy}{\secondforcepointz}%
		{\secondforcevectorx}{\secondforcevectory}{\secondforcevectorz}

	\drawepurelinesatpointalongz{\xfirst}{0}{\zfirst}%
		{\epurecolor}{\epurecolor}{\epurecolor}%
		[thick]

	\drawlinebetweensavedandcurrentx

	\def\epurecolor{#1}

	\pgfmathsetmacro\zfrom{0}
	\pgfmathsetmacro\zto{\zsecond}
	\pgfmathsetmacro\zstep{\hatchstep * ( \zto - \zfrom ) / abs( \zto - \zfrom )}
	\pgfmathsetmacro\znext{\zfrom + \zstep}
	\foreach \zhatch in { \zfrom, \znext, ..., \zto } {
		\calculateinternalmomentatpointfromforce{\xsecond}{0}{\zhatch}%
			{\firstforcepointx}{\firstforcepointy}{\firstforcepointz}%
			{\firstforcevectorx}{\firstforcevectory}{\firstforcevectorz}

		\pgfmathsetmacro\firstmomentx{\momentx}
		\pgfmathsetmacro\firstmomenty{\momenty}
		\pgfmathsetmacro\firstmomentz{\momentz}

		\pgfmathsetmacro\secondmomentx{0}
		\pgfmathsetmacro\secondmomenty{0}
		\pgfmathsetmacro\secondmomentz{0}

		\pgfmathsetmacro\momentx{\firstmomentx + \secondmomentx}
		\pgfmathsetmacro\momenty{\firstmomenty + \secondmomenty}
		\pgfmathsetmacro\momentz{\firstmomentz + \secondmomentz}

		\drawepurelinesatpointalongz{\xsecond}{0}{\zhatch}%
			{\epurecolor}{\epurecolor}{\epurecolor}
	}

	\calculateinternalmomentatpointfromforce{\xsecond}{0}{0}%
		{\firstforcepointx}{\firstforcepointy}{\firstforcepointz}%
		{\firstforcevectorx}{\firstforcevectory}{\firstforcevectorz}

	\drawepurelinesatpointalongz{\xsecond}{0}{0}%
		{\epurecolor}{\epurecolor}{\epurecolor}%
		[thick]

	\saveepureendpoints

	\calculateinternalmomentatpointfromforce{\xsecond}{0}{\zsecond}%
		{\firstforcepointx}{\firstforcepointy}{\firstforcepointz}%
		{\firstforcevectorx}{\firstforcevectory}{\firstforcevectorz}

	\drawepurelinesatpointalongz{\xsecond}{0}{\zsecond}%
		{\epurecolor}{\epurecolor}{\epurecolor}%
		[thick]

	\drawlinebetweensavedandcurrentx
}

\newcommand\drawepureofinternalmomentfromthirdforce{
	\def\epurecolor{\externalforcecolor}

	% along x

	\pgfmathsetmacro\xmax{\xsecond}
	\foreach \xhatch in { 0, \hatchstep, ..., \xmax } {
		\calculateinternalmomentatpointfromforce{\xhatch}{0}{0}%
			{\thirdforcepointx}{\thirdforcepointy}{\thirdforcepointz}%
			{\thirdforcevectorx}{\thirdforcevectory}{\thirdforcevectorz}

		\drawepurelinesatpointalongx{\xhatch}{0}{0}%
			{\epurecolor}{\epurecolor}{\epurecolor}
	}

	\calculateinternalmomentatpointfromforce{0}{0}{0}%
		{\thirdforcepointx}{\thirdforcepointy}{\thirdforcepointz}%
		{\thirdforcevectorx}{\thirdforcevectory}{\thirdforcevectorz}
	\drawepurelinesatpointalongx{0}{0}{0}%
		{\epurecolor}{\epurecolor}{\epurecolor}%
		[thick]

	\saveepureendpoints

	\calculateinternalmomentatpointfromforce{\xfirst}{0}{0}%
		{\thirdforcepointx}{\thirdforcepointy}{\thirdforcepointz}%
		{\thirdforcevectorx}{\thirdforcevectory}{\thirdforcevectorz}
	\drawepurelinesatpointalongx{\xfirst}{0}{0}%
		{\epurecolor}{\epurecolor}{\epurecolor}%
		[thick]

	\drawlinebetweensavedandcurrenty
	%%\drawlinebetweensavedandcurrentz

	\saveepureendpoints

	\calculateinternalmomentatpointfromforce{\xsecond}{0}{0}%
		{\thirdforcepointx}{\thirdforcepointy}{\thirdforcepointz}%
		{\thirdforcevectorx}{\thirdforcevectory}{\thirdforcevectorz}
	\drawepurelinesatpointalongx{\xsecond}{0}{0}%
		{\epurecolor}{\epurecolor}{\epurecolor}%
		[thick]

	\drawlinebetweensavedandcurrenty
	%%\drawlinebetweensavedandcurrentz
}

% #1: color of third force
% #2: color of second force
% #3: color of sum
\newcommand\drawepureofinternalmomentfromthirdandsecondforces[3]{
	\def\externalforcecolor{#1}
	\drawepureofinternalmomentfromthirdforce

	\def\externalforcecolor{#2}
	\drawepureofinternalmomentfromsecondforce
}


\newcommand\drawframe{
%
\tikzstyle{empty node} = [ shape=circle, outer sep=10pt ]
\node [ empty node ] at ( 16cm, 12cm ) {} ;
\node [ empty node ] at ( -16cm, -12cm ) {} ;

\def\textscale{2}

\pgfmathsetmacro\cornerlength{\reflength / 32}

\drawbeam

\def\externalforcecolor{green}

\drawepureofinternalmomentfromthirdforce

\def\externalforcecolor{red}

\drawload{( {\firstforcepointx}, {\firstforcepointy}, {\firstforcepointz} )}{( {\firstforcevectorx * \forcearrowscale}, {\firstforcevectory * \forcearrowscale}, {\firstforcevectorz * \forcearrowscale} )}{pos=.1, below, inner sep=0pt, outer sep=11pt}{\scalebox{\textscale}{$ P $}}

\def\externalforcecolor{blue}

\drawload{( {\secondforcepointx}, {\secondforcepointy}, {\secondforcepointz} )}{( {\secondforcevectorx * \forcearrowscale}, {\secondforcevectory * \forcearrowscale}, {\secondforcevectorz * \forcearrowscale} )}{pos=.6, below, inner sep=0pt, outer sep=11pt}{\scalebox{\textscale}{$ P $}}

\def\twistingmomentepureoffset{.5*\reflength}
\def\twistingmomentmultiplierx{0}
\def\twistingmomentmultipliery{0}
\def\twistingmomentmultiplierz{1}

\drawepureofinternalmomentfrombothforces{red}{blue}{magenta}

\def\externalforcecolor{green}

\drawload{( {\thirdforcepointx}, {\thirdforcepointy}, {\thirdforcepointz} )}{( {\thirdforcevectorx * \forcearrowscale}, {\thirdforcevectory * \forcearrowscale}, {\thirdforcevectorz * \forcearrowscale} )}{pos=.33, left, inner sep=0pt, outer sep=6pt}{\scalebox{\textscale}{$ P $}}
%
}

\pgfmathsetmacro\maxzooming{1}

\def\anglestep{2}
\def\anglesecondbegin{120}
\pgfmathsetmacro\nextangle{\anglesecondbegin + \anglestep}
\def\maxincrement{360} % 360
\pgfmathsetmacro\endangle{\anglesecondbegin + \maxincrement - \anglestep}

\foreach \anglesecond  in { \anglesecondbegin, \nextangle, ..., \endangle } {
\begin{tikzpicture}[ scale=1 ]

\pgfmathsetmacro\anglefirst{55 + ( (\anglesecond - 120) / 18 )}
\tdplotsetmaincoords{\anglefirst}{\anglesecond}

\pgfmathsetmacro\zooming{(\anglesecond - 110) / 90}
\pgfmathparse{(\zooming > \maxzooming) ? 1 : 0}\ifdim\pgfmathresult pt>0pt%
\pgfmathsetmacro\zooming{\maxzooming}
\fi

\scalebox{\zooming}{\drawframe}

\end{tikzpicture}
}

\pgfmathsetmacro\endangle{\anglesecondbegin + \maxincrement - \anglestep}

\foreach \anglesecond  in { \anglesecondbegin, \nextangle, ..., \endangle } {
\begin{tikzpicture}[ scale=1 ]

\pgfmathsetmacro\anglefirst{75 - ( (\anglesecond - 120) / 15 )}
\tdplotsetmaincoords{\anglefirst}{\anglesecond}

\pgfmathsetmacro\zooming{\maxzooming}

\scalebox{\zooming}{\drawframe}

\end{tikzpicture}
}

\pgfmathsetmacro\endangle{\anglesecondbegin + \maxincrement}

\foreach \anglesecond  in { \anglesecondbegin, \nextangle, ..., \endangle } {
\begin{tikzpicture}[ scale=1 ]

\pgfmathsetmacro\anglefirst{51 + ( (\anglesecond - 120) / 90 )}
\tdplotsetmaincoords{\anglefirst}{\anglesecond}

\pgfmathsetmacro\zooming{\maxzooming}

\scalebox{\zooming}{\drawframe}

\end{tikzpicture}
}

\end{document}
