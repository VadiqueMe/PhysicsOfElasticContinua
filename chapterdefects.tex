\en{\chapter{Defects}}

\ru{\chapter{Дефекты}}

\thispagestyle{empty}

\label{chapter:defects}

\en{\section{Volterra dislocations}}

\ru{\section{Дислокации Вольтерры}}

\label{para:volterradislocations}

\en{\dropcap{C}{onsider}}\ru{\dropcap{Р}{ассмотрим}} \en{the classical linear three-dimensional elastic medium}\ru{классическую линейную трёхмерную упругую среду}~(\chapref{chapter:linearclassicalelasticity}).
\en{As shown in}\ru{Как показано в}~\chapdotpararef{chapter:linearclassicalelasticity}{para:displacementsfromdeformations}, \en{the equation of compatibility of deformations}\ru{уравнение совместности деформаций}

...

\en{\section{Straight\hbox{-}line dislocations}}

\ru{\section{Прямолинейные дислокации}}

\label{para:rectilineardislocations}

\begin{otherlanguage}{russian}

\en{A~dislocation line}\ru{Линия дислокации} \en{may be}\ru{может быть} \en{any curve}\ru{любой кривой}, \en{closed inside a~body}\ru{замкнутой внутри тела} \en{or}\ru{или} \en{with ends}\ru{с~концами} \en{on the surface}\ru{на~поверхности}.
\en{For}\ru{Для} \en{a~dislocation}\ru{дислокации} \en{of a~random shape}\ru{случайной формы} \en{in a~limitless medium}\ru{в~неограниченной среде}\en{,} \en{it’s not so difficult}\ru{не~так~уж сложно} \en{to obtain}\ru{получить} \en{an~appropriate solution}\ru{подходящее решение}~\cite{eshelby-theoryofdislocations}.
Мы~же ограничимся простейшим случаем прямо\-линей\-ной дислокации.
Ищется решение

...


\end{otherlanguage}

\en{\section{Action of stress field on dislocation}}

\ru{\section{Действие поля напряжений на дислокацию}}

\label{para:actionofstressfieldondislocation}

\begin{otherlanguage}{russian}

Рассмотрим тело, содержащее внутри дислокацию с~замкнутой линией~$C$.
Тело нагружено объёмными~$\bm{f}$ и~поверхностными~$\bm{p}$ силами.
Обозначим

...


\end{otherlanguage}

\en{\section{About movement of dislocations}}

\ru{\section{О движении дислокаций}}

\label{para:dislocationmovement}

\begin{otherlanguage}{russian}

Рассмотрим это явление, следуя~\cite{cottrell-dislocations}.
Ограничимся случаем прямо\-линейной винтовой дислокации, движущейся с~постоянной скоростью

...


\end{otherlanguage}

\en{\section{Point defects}}

\ru{\section{Точечные дефекты}}

\label{para:pointdefects}

\begin{otherlanguage}{russian}

Речь пойдёт о~континуальной модели таких явлений как вакансии, примесные частицы или~междоузельные атомы в~кристаллической решётке.
В~случае дислокации рассматривались

...


\end{otherlanguage}

\en{\section{Force acting on a~point defect}}

\ru{\section{Сила, действующая на точечный дефект}}

\label{para:forceonpointdefect}

\begin{otherlanguage}{russian}

Дефект находится в~теле, нагруженном объёмными~$\bm{f}$ и~поверхностными~$\bm{p}$ силами.
Суперпозиция

...


\end{otherlanguage}

\en{\section{Continuously distributed dislocations}}

\ru{\section{Непрерывно распределённые дислокации}}

\label{para:dislocations.continuouslydistributed}

\begin{otherlanguage}{russian}

Начнём со~сложения векторов Бюргерса.
При обходе сразу двух дислокаций (рис. ?? 40 ??) по~контуру

...


\end{otherlanguage}

\en{\section{Stress during winding of coil}}

\ru{\section{Напряжение при намотке катушки}}

\label{para:coilwindingstress}

\begin{otherlanguage}{russian}

Не~только дислокации и~точечные дефекты, но~и~макроскопические факторы могут быть источниками собственных напряжений.
При намотке катушки (рис. ?? 42 ??) в~ней возникают напряжения от натяжения ленты.
Расчёт этих напряжений очень сложен, если рассматривать детально процесс укладки ленты.

Но~существует чёткий алгоритм Southwell’а~\cite{southwell-introductiontotheoryofelasticity} расчёта напряжений в~катушке: укладка каждого нового витка вызывает внутри катушки приращения напряжений, определяемые соотношениями линейной упругости.
\en{There are two stages}\ru{Здесь два этапа}, \en{and the first one}\ru{и~первый} \en{consists}\ru{состоит} \en{in solving}\ru{в~решении} \en{the }\ru{задачи }Lam\'{e}\en{ problem} \en{about deformation}\ru{о~деформации} \en{of a~hollow cylinder}\ru{полого цилиндра} \en{under external pressure}\ru{под внеш\-ним давлением} (рис. ?? 43 ??)

...



\end{otherlanguage}

\section*{\small \wordforbibliography}

\begin{changemargin}{\parindent}{0pt}
\fontsize{10}{12}\selectfont

\en{Dislocations}\ru{Дислокации} \en{and}\ru{и}~\en{point defects}\ru{точечные дефекты} \en{in}\ru{в}~\en{linear elastic}\ru{линейно\hbox{-}упругих} \en{bodies}\ru{телах} \en{have been considered}\ru{рассматривались} \en{by many authors}\ru{многими авторами}:
John Eshelby~\cite{eshelby-theoryofdislocations},
Roland deWit~\cite{dewit-disclinations},
Cristian Teodosiu~\cite{teodosiu-crystaldefects},
Alan Cottrell~\cite{cottrell-dislocations}.
\en{The theory of self-stresses}\ru{Теория собственных напряжений} (eigenspannungen) \en{is explained}\ru{объясняется} \en{by} Ekkehart Kröner \en{in}\ru{в}~\cite{kroener-kontinuumstheorie}.
\en{Calculation of stresses}\ru{Расчёт напряжений} \en{when winding a~coil}\ru{при намотке катушки} \en{is described}\ru{описан} \en{by }Richard Southwell\ru{’ом} \en{in his book}\ru{в~своей книге}~\cite{southwell-introductiontotheoryofelasticity}.

\end{changemargin}
