\en{\chapter{Defects}}

\ru{\chapter{Дефекты}}

\thispagestyle{empty}

\label{chapter:defects}

\en{\section{Volterra dislocations}}

\ru{\section{Дислокации Вольтерры}}

\label{para:volterradislocations}

\begin{otherlanguage}{russian}

\dropcap{Р}{ассмотрим} классическую линейную трёхмерную среду~(\chapref{chapter:linearclassicalelasticity}).
Как показано в~\chapdotpararef{chapter:linearclassicalelasticity}{para:displacementsfromdeformations}, уравнение совместности деформаций

...


\end{otherlanguage}

\en{\section{Straight\hbox{-}line dislocations}}

\ru{\section{Прямолинейные дислокации}}

\label{para:rectilineardislocations}

\begin{otherlanguage}{russian}

Линия дислокации может быть любой пространственной кривой, замкнутой в~теле или выходящей концами на~поверхность.
Для дислокации произвольной формы в~неограниченной среде не~так~уж сложно получить соответствующее решение~\cite{eshelby-theoryofdislocations}.
Мы~же ограничимся простейшим случаем прямо\-линей\-ной дислокации.
Ищется решение

...


\end{otherlanguage}

\en{\section{Action of stress field on dislocation}}

\ru{\section{Действие поля напряжений на дислокацию}}

\label{para:actionofstressfieldondislocation}

\begin{otherlanguage}{russian}

Рассмотрим тело, содержащее внутри дислокацию с~замкнутой линией~$C$.
Тело нагружено объёмными~$\bm{f}$ и~поверхностными~$\bm{p}$ силами.
Обозначим

...


\end{otherlanguage}

\en{\section{About movement of dislocations}}

\ru{\section{О движении дислокаций}}

\label{para:dislocationmovement}

\begin{otherlanguage}{russian}

Рассмотрим это явление, следуя~\cite{cottrell-dislocations}.
Ограничимся случаем прямо\-линейной винтовой дислокации, движущейся с~постоянной скоростью

...


\end{otherlanguage}

\en{\section{Point defects}}

\ru{\section{Точечные дефекты}}

\label{para:pointdefects}

\begin{otherlanguage}{russian}

Речь пойдёт о~континуальной модели таких явлений как вакансии, примесные частицы или~междоузельные атомы в~кристаллической решётке.
В~случае дислокации рассматривались

...


\end{otherlanguage}

\en{\section{Force acting on point defect}}

\ru{\section{Действующая на точечный дефект сила}}

\label{para:forceonpointdefect}

\begin{otherlanguage}{russian}

Дефект находится в~теле, нагруженном объёмными~$\bm{f}$ и~поверхностными~$\bm{p}$ силами.
Суперпозиция

...


\end{otherlanguage}

\en{\section{Continuously distributed dislocations}}

\ru{\section{Непрерывно распределённые дислокации}}

\label{para:dislocations.continuouslydistributed}

\begin{otherlanguage}{russian}

Начнём со~сложения векторов Бюргерса.
При обходе сразу двух дислокаций (рис. ?? 40 ??) по~контуру

...


\end{otherlanguage}

\en{\section{Stress during winding of coil}}

\ru{\section{Напряжение при намотке катушки}}

\label{para:coilwindingstress}

\begin{otherlanguage}{russian}

Не~только дислокации и~точечные дефекты, но~и~макроскопические факторы могут быть источниками собственных напряжений.
При намотке катушки (рис. ?? 42 ??) в~ней возникают напряжения от натяжения ленты.
Расчёт этих напряжений очень сложен, если рассматривать детально процесс укладки ленты.

Но~существует чёткий алгоритм Southwell’а~\cite{southwell-introductiontotheoryofelasticity} расчёта напряжений в~катушке: укладка каждого нового витка вызывает внутри катушки приращения напряжений, определяемые соотношениями линейной упругости.
Здесь два этапа, и~первый состоит в~решении задачи Lam\'{e} о~деформации полого цилиндра под~внеш\-ним давлением (рис. ?? 43 ??)

Lam\'{e} problem about deformation of a~hollow cylinder under external pressure

...




\end{otherlanguage}

\section*{\small \wordforbibliography}

\begin{changemargin}{\parindent}{0pt}
\fontsize{10}{12}\selectfont

\begin{otherlanguage}{russian}

Дислокации и~точечные дефекты в~линейно\hbox{-}упругих телах рассматривали многие авторы:
John Eshelby~\cite{eshelby-theoryofdislocations},
Roland deWit~\cite{dewit-disclinations},
Cristian Teodosiu~\cite{teodosiu-crystaldefects},
Alan Cottrell~\cite{cottrell-dislocations}.
Теорию собственных напряжений изложил
Ekkehart Kröner в~\cite{kroener-kontinuumstheorie}.
Методику расчёта напряжений при намотке опис\'{а}л
Richard Southwell в~книге~\cite{southwell-introductiontotheoryofelasticity}.

\end{otherlanguage}

\end{changemargin}
