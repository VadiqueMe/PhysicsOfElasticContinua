\section*{\small \wordforbibliography}

\begin{changemargin}{\parindent}{0pt}
\fontsize{10}{12}\selectfont

\en{There are many books worth mentioning}\ru{Есть много ст\'{о}ящих упомнания книг}\ru{,}
\en{that cover}\ru{которые охватывают}
\en{only}\ru{только}
\en{the~apparatus}\ru{аппарат}
\en{of tensor calculus}\ru{тензорного исчисления}~%
\cite{mcconnell-tensoranalysis, schouten-tensoranalysis, sokolnikoff-tensoranalysis, dimitrienko-tensorcalculus, borisenko.tarapov, rashevsky-riemanniangeometry}.

\en{However}\ru{Однако},
\en{the~index notation}\ru{индексная запись}
(\en{it’s}\ru{это}
\en{when}\ru{когда}
\en{tensors are presented}\ru{тензоры представляются}
\en{as the~sets of~components}\ru{наборами компонент})
\en{is still more popular}\ru{всё ещё более популярна}\ru{,}
\en{than the~direct indexless notation}\ru{чем прямая безиндексная запись}.

\en{The~direct notation}\ru{Прямая запись}
\en{is widely used}\ru{широко используется},
\en{for example}\ru{например},
\en{in the~appendices}\ru{в~приложениях}
\en{to the~books}\ru{к~книгам}
\en{by }Anatoliy\ru{’я}~I.\;Lurie
(\russianlanguage{Анатол\en{ий}\ru{ия}~И.\;Лурье})~%
\cite{lurie-nonlinearelasticity, lurie-theoryofelasticity}.

\inquotes{\russianlanguage{Теория упругости}}\en{~(\inquotes{The~theory of~elasticity})}
\en{by }\russianlanguage{Вениамин}\ru{а} \russianlanguage{Блох}\ru{а}\en{~(Veniamin Blokh)}~\cite{veniaminblokh-theoryofelasticity}
\en{is as well written}\ru{также написана}
\en{in the direct indexless notation}\ru{в~прямой безиндексной нотации}.

\en{The}\ru{Лекции} R.\:Feynman’\en{s}\ru{а}\en{ lectures}~\cite{feynman-lecturesonphysics}
\en{contain}\ru{содержат}
\en{the~vivid description}\ru{яркое описание}
\en{of the~vector fields theory}\ru{теории векторных полей}.

\en{Also}\ru{Также},
\en{information}\ru{информация}
\en{about the~tensor calculus}\ru{о~тензорном исчислении}
\en{is}\ru{есть}
\en{the~part}\ru{часть}
\en{of the~unusual}\ru{необычной}
\en{and interesting}\ru{и~интересной}
\en{book}\ru{книги}
\en{by }C.\:Truesdell\ru{’а}~\cite{truesdell-firstcourse}.

\end{changemargin}

