\newcommand\longS{s} %%{\scalebox{1}[2]{s}}
\newcommand\longSS{ss} %%{\hbox{\longS\hspace{-0.1ex}\longS}} %%{\scalebox{1}[2]{ss}}

\begin{tcolorbox}[breakable, enhanced, colback = orange!10, before upper={\parindent3.2ex}, parbox = false]
\small%
\setlength{\abovedisplayskip}{2pt}\setlength{\belowdisplayskip}{2pt}%

\begin{center}
\imfellEnglish
\scalebox{.93}[1]{\scalebox{1.6}{\MakeUppercase{axiomata}}}
\\[.7em]
\scalebox{.93}[1]{\scalebox{1.3}{\MakeUppercase{sive}}}
\\[.7em]
\scalebox{.93}[1]{\scalebox{1.6}{\MakeUppercase{leges motus}}}
\end{center}

\bgroup % to change font to IM Fell
\imfellEnglish

\begin{center}Lex.\;I.\end{center}
\vspace{-1.6em}

\begin{changemargin}{0em}{0em}
\begin{center}\emph{%
Corpus omne per\longS{}everare in \longS{}tatu \longS{}uo quie\hspace{-0.1ex}\longS{}cendi vel movendi unifor\-miter in directum, ni\longS{}i quatenus a~viribus impre\hspace{-0.1ex}\longSS{}is cogitur \longS{}tatum illum mutare.}
\end{center}
\end{changemargin}
\vspace{-0.5em}

Projectilia per\longS{}everant in motibus \longS{}uis ni\longS{}i quatenus a~re\longS{}i\longS{}tentia aeris retardantur \& vi gravitatis impelluntur deor\longS{}um.
Trochus, cujus partes coh\ae{}rendo perpetuo retrahunt \longS{}e\longS{}e a~motibus rectilineis, non ce\longSS{}at rotari ni\longS{}i quatenus ab~aere retardatur.
Majora autem Planetarum \&~Cometarum corpora mo\-tus \longS{}uos \&~progre\longSS{}ivos \&~circulares in \longS{}patiis minus re\longS{}i\longS{}tentibus factos con\longS{}ervant diutius.

\vspace{-1em}
\begin{center}Lex.\;II.\end{center}
\vspace{-1.6em}

\begin{changemargin}{0em}{0em}
\begin{center}\emph{%
Mutationem motus proportionalem e\longSS{}e vi motrici impre\longSS{}\ae{}, \& fieri \longS{}e\-cundum lineam rectam qua vis illa imprimitur.}
\end{center}
\end{changemargin}
\vspace{-0.5em}

Si vis aliqua motum quemvis generet, dupla duplum, tripla triplum generabit, \longS{}ive \longS{}imul \&~\longS{}emel, \longS{}ive gradatim \&~succe\longSS{}ive impre\longSS{}a fuerit.
Et hic motus quoniam in eandem \longS{}emper plagam cum vi generatrice determinatur, \longS{}i corpus antea movebatur, motui ejus vel con\longS{}piranti additur, vel contrario \longS{}ubducitur, vel obliquo oblique adjicitur, \&~cum eo \longS{}ecundum utriu\longS{}q\,; determinationem componitur.

\vspace{-1em}
\begin{center}Lex.\;III.\end{center}
\vspace{-1.6em}

\begin{changemargin}{\parindent}{0em}
\hspace*{\negparindent}\emph{%
Actioni contrariam \longS{}emper \&~\ae{}qualem e\longSS{}e reactionem~:
\longS{}ive corporum duorum actiones in~\longS{}e mutuo \longS{}emper e\longSS{}e \ae{}quales \&~in~partes contrarias dirigi.}
\end{changemargin}
\vspace{-0.5em}

Quicquid premit vel trahit alterum, tantundem ab eo premitur vel trahitur.
Siquis lapidem digito premit, premitur \& hujus digitus a lapide.
Si equus lapidem funi allegatum trahit, retrahetur etiam \& equus aequaliter in lapidem~:
nam funis utrinq\,;
distentus eodem relaxandi se conatu urgebit Equum ver\longS{}us lapidem, ac lapidem ver\longS{}us equum, tantumq\,;
impediet progre\longSS{}um unius quantum promovet progre\longSS{}um alterius.
Si corpus aliquod in corpus aliut impingens, motum ejus vi \longS{}ua quomodocunq\,:
mutaverit, idem quoque vici\longSS{}im in motu proprio eandem mutationem in partem contrariam vi alterius (ob \ae{}qualitatem pre\longSS{}ionis mutu\ae{}) subibit.
His actionibus \ae{}quales fiunt mutationes non velocitatum sed motuum, (scilicet in corporibus non aliunde impeditis\::\,)
Mutationes enim velocitatum, in contrarias itidem partes fact\ae, quia motus \ae{}qualiter mutantur, sunt corporibus reciproce proportionales.

~ % for empty line
\egroup
\end{tcolorbox}
