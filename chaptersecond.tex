\en{\chapter{Classical generic mechanics}}

\ru{\chapter{Классическая общая механика}}

\thispagestyle{empty}

\label{chapter:genericmechanics}

\newcommand\positionofthepole{\bm{p}}

\en{\section{Discrete collection of particles}} %% Initial concepts. Discrete approach

\ru{\section{Дискретная совокупность частиц}} %% Исходные представления. Дискретный подход

\label{para:initialconcepts.discreteapproach}

\en{\dropcap{C}{lassical}}\ru{\dropcap{К}{лассическая}} \en{generic}\ru{общая} \en{mechanics}\ru{механика} \en{models}\ru{моделирует} \en{physical objects}\ru{физические объекты}\en{ by}\ru{,} \en{discretizing them}\ru{дискретизируя их} \en{into }\ru{в~}\en{a~collection of~particles}\ru{совокупность частиц} (\inquotesx{\en{pointlike masses}\ru{точеч\-ных масс}}[,] \inquotes{\en{material points}\ru{материальных точек}}%
\footnote{\en{Point mass}\ru{Точечная масса} (\en{pointlike mass, }\en{material point}\ru{материальная точка})\en{ is}\ru{\:--- это} \en{the~concept}\ru{концепт} \en{of an~object}\ru{объекта}, \en{typically}\ru{типично} \en{matter}\ru{материи}, \en{that}\ru{который} \en{has}\ru{имеет} \en{nonzero mass}\ru{ненулевую массу} \en{and}\ru{и}~\en{is}\ru{является}~(\en{or}\ru{или} \en{is being thought of as}\ru{мыслится}) \en{infinitesimal}\ru{бесконечно-малым} \en{in~its}\ru{по~своем\'{у}} \en{volume}\ru{объёму}~(\en{dimensions}\ru{размерам}).%
}).

\en{In}\ru{В}~\en{a~collection}\ru{совокупности} \en{of }${N\hspace{-0.25ex}}$~\en{particles,}\ru{\hbox{частиц}} \en{each}\ru{каждая} $k$\hbox{-}\en{th}\ru{ая} \en{particle}\ru{частица} \en{has its nonzero mass}\ru{имеет свою ненулевую массу}~${m_k \hspace{-0.25ex} = \hspace{-0.1ex} \constant > \hspace{-0.1ex} 0}$ \en{and}\ru{и}~\en{motion function}\ru{функцию движения}~${\locationvector_{k}(t)}$.
\en{Function}\ru{Функция}~${\locationvector_{k}(t)}$ \en{is measured}\ru{измеряется} \en{relative to the chosen reference system}\ru{относительно выбранной системы отсчёта}.

\vspace{.2em}
%\hspace*{-\parindent}\begin{minipage}{\linewidth}
\setlength{\parindent}{\horizontalindent}
\setlength{\parskip}{\spacebetweenparagraphs}

\begin{wrapfigure}[10]{r}{.4\textwidth}
%%\makebox[.42\textwidth][c]{%%%\begin{minipage}[t]{.43\textwidth}
\vspace{.5em}
\scalebox{1.1}{
\begin{tikzpicture}[scale=0.86]

	\draw[line width=1.2pt, black] (0,0) -- (2.8,0);
	\draw[line width=1.2pt, black] (0,0) -- (-1.8,0);
	\foreach \xground in {-1.6, -1.33, ..., 2.9}
		\draw [line width=0.4pt, black!80] (\xground,0) -- ++(-0.2,-0.2);

	\def\clockat{-0.93}
	\path (\clockat, 0) node [shape=coordinate] (clocktower) {};
	\def\clockradius{0.4}
	\def\clocksquare{\clockradius + 0.1}
	\def\clockbase{\clockradius - 0.05}
	\def\clockheight{0.75}
	\path (\clockat, \clockheight+\clocksquare) node [shape=coordinate] (clock) {};

	\draw [line width=1.2pt, black] ($ (clocktower) + (\clockbase,0) $) -- ++(up:\clockheight);
	\draw [line width=1.2pt, black] ($ (clocktower) - (\clockbase,0) $) -- ++(up:\clockheight);
	\draw [line width=1.2pt, black, rounded corners=1.2pt] ($ (clock) + (\clocksquare,\clocksquare) $) rectangle ($ (clock) - (\clocksquare,\clocksquare) $);
	\draw [line width=1.2pt, black] ($ (clock) + (\clockbase,\clocksquare) $) arc(0:180:\clockbase);

	\draw [line width=1.2pt, blue] (clock) circle(\clockradius);
	\draw [line width=1.2pt, blue, rotate around={33:(clock)}] (clock) -- ++(\clockradius - 0.1, 0);
	\draw [line width=1.2pt, blue, rotate around={148:(clock)}] (clock) -- ++(\clockradius - 0.15, 0);

	\path (0, 0.33) node [shape=coordinate] (O) {};
	\path (O) node [inner sep=0mm, outer sep=1mm] ($mcirc$) {};
	\path (O) -- ++(2.3, 0) node [shape=coordinate] (first) {};
	\path (O) -- ++(-1, -1) node [shape=coordinate] (second) {};
	\path (O) -- ++(0, 1.83) node [shape=coordinate] (third) {};

	\path (O) node [left, inner sep=0mm, outer sep=2mm, yshift=0.5mm] { $o$ } ;

	\tikzstyle{basis vector} =
		[line width=1pt, blue, style=double, double distance=0.5mm, -{Triangle[open, angle=60:3.2mm]}]
	\draw [basis vector] (O) -- (first)
		node [pos=0.88, above, inner sep=1.33ex, outer sep=0] {$\bm{e}_1$};
	\draw [basis vector] (O) -- (second)
		node [pos=0.71, below right, inner sep=0.83ex, outer sep=0] {$\bm{e}_2$};
 	\draw [basis vector] (O) -- (third)
		node [pos=0.86, right, inner sep=1.33ex, outer sep=0] {$\bm{e}_3$};

	\path (2.44,1.7) node [shape=coordinate] (m) {};

	\path (m) node [shape=circle, inner sep=1mm, outer sep=0] (mcirc) {};

	\draw [line width=1.6pt, black, -{Stealth[round, length=5mm, width=3.6mm]}] (O) -- (mcirc)
		node [pos=0.5, above] {$\locationvector$} ;

	\draw [line width=1.6pt, black, fill=black!50] (m) circle (1.6mm) ;

	\path (mcirc) node [xshift=-3.2mm, yshift=3mm] {$m$} ;

	\draw [line width=1pt, blue, fill=white] (O) circle (1.2mm) ;

\end{tikzpicture}
}
\vspace{.5em}\caption{}\label{fig:referencesystem}
%%%}
\end{wrapfigure} %%%\end{minipage}

\en{A~reference system~(also \inquotes{reference frame})}\ru{Система отсчёта} \en{consists of}\ru{состоит из}~(\figureref{fig:referencesystem})

\begin{itemize}
\item \en{some}\ru{н\'{е}которой} \inquotes{\en{null}\ru{нулевой}} \en{reference point}\ru{точки отсчёта} ${o}$,
\item \en{a~set of coordinates}\ru{набора координат}, %%(\en{the coordinates}\ru{координаты} \en{define the units of measurements}\ru{определяют единицы измерения}
\item \en{a~clock}\ru{часы}.
\end{itemize}

%%.... свойства реального физического пространства и~конкретные способы измерения времени ....

%\vspace{-0.1em}\noindent
%\inquotes{\emph{\en{Any}\ru{Любых}}} \en{clock}\ru{часов}\:--- \en{because}\ru{потому что}
%(\en{in}\ru{в}~\en{classical}\ru{классической} \en{generic}\ru{общей} \en{mechanics}\ru{механике})
%\en{time}\ru{время} \en{tick-tocks}\ru{тик-такает}, \en{flows}\ru{течёт} \en{and }\ru{и~}\en{passes}\ru{проходит} \en{identically}\ru{одинаково} \en{in~any clock}\ru{в~любых часах} \en{in~any place}\ru{в~любом месте}, \en{and}\ru{и} \en{all clocks are perfectly synchronized}\ru{все часы идеально синхронизированы}.

\vspace{.33em}
%%%\end{minipage}

\en{Long time ago}\ru{Когда\hbox{-}то давно}, \en{the~reference system}\ru{системой отсчёта} \en{was}\ru{было} \en{some absolute space}\ru{некое абсолютное пространство}: \en{empty at~first}\ru{сначала пустое}, \en{and then}\ru{а~затем} \en{filled with the~continuous elastic medium}\ru{заполненное сплошной упругой \hbox{средой}}\:--- \en{the~æther}\ru{эфиром~(æther)}.
\en{Later it became clear}\ru{Позже стало ясно,} \en{that}\ru{что} \en{for classical mechanics}\ru{для классической механики} \ru{могут быть использованы }\en{any reference systems}\ru{любые системы отсчёта}\en{ can be used}, \en{but}\ru{но} \en{the~preference is given to}\ru{предпочтение отдаётся} \en{the so~called}\ru{так называемым} \inquotes{\en{inertial}\ru{инерциальным}} \en{systems}\ru{системам}, \en{where}\ru{где} \en{a~point}\ru{\hbox{точка}} \en{moves}\ru{движется} \en{with a~constant velocity}\ru{с~постоянной скоростью} \en{without acceleration}\ru{без~ускорения}~(${\mathdotdotabove{\locationvector} = \hspace{-0.1ex} \bm{0}}$, ${\mathdotabove{\locationvector} = \hspace{-0.1ex} \boldconstant}$) \en{in the~absence of external interactions}\ru{в~отсутствии внешних взаимодействий}.

\en{A~motion}\ru{Движение} \en{along a~straight line}\ru{вдоль прямой линии} \en{with the constant~velocity}\ru{с~постоянной скоростью}, \en{also known as}\ru{также известное как} \en{a~free motion}\ru{свободное движение}, \en{supposes}\ru{предполагает} \en{absence of external interactions}\ru{отсутствие внешних взаимодействий} (\en{or}\ru{или} \en{applied forces}\ru{приложенных сил}):

\begin{gather}
\mathdotabove{\locationvector} = \hspace{-0.1ex} \boldconstant = \alpha_i \bm{e}_i
\\
\alpha_i \hspace{-0.2ex} = \hspace{-0.1ex} \constant
\end{gather}

\en{The measure of interaction in mechanics is the~vector of~force}\ru{Мерой взаимодействия в~механике является вектор силы}~${\bm{F}\hspace{-0.2ex}}$.
\en{In}\ru{В}~\en{the~widely known}\ru{широко известном}\footnote{%
\inquotes{Axiomata sive Leges Motus} (\inquotes{Axioms or Laws of Motion})
were written by Isaac Newton in his \href{http://www.gutenberg.org/files/28233/28233-pdf.pdf}{Philosophiæ Naturalis Principia Mathematica}, first published in 1687.
%% http://www.gutenberg.org/ebooks/28233
\href{https://archive.org/details/principia00newtuoft/page/n11/mode/2up}{Reprint (en Latin), 1871.}
\href{https://archive.org/details/newtonspmathema00newtrich/page/n7/mode/2up}{Translated into English by Andrew Motte, 1846.}
% Mathematical Principles of Natural Philosophy
}\hbox{\hspace{-0.2ex}} \ru{уравнении }Newton’\en{s}\ru{а}\en{ equation}

\nopagebreak\vspace{-0.2em}\begin{equation}\label{law:ofnewton}
m \hspace{.2ex} \mathdotdotabove{\locationvector} \hspace{.1ex} = \hspace{-0.1ex} \bm{F} ( \locationvector, \mathdotabove{\locationvector}, t )
\end{equation}

\vspace{-0.25em}\noindent
\en{the~right\hbox{-}hand side}\ru{правая часть} \en{can depend only on}\ru{может зависеть лишь от} \en{position}\ru{положения}, \en{velocity}\ru{скорости} \en{and}\ru{и}~\en{explicitly presented time}\ru{явно входящего времени}, \en{whereas acceleration}\ru{тогда как ускорение}~$\mathdotdotabove{\locationvector}$ \en{is directly proportional to~force}\ru{прямо пропорционально силе}~$\bm{F}$ \en{with coefficient}\ru{с~коэффициентом}~${\raisemath{-0.1em}{\scalebox{1.2}{$\nicefrac{1}{m}$}}\hspace{.1ex}}$.

\vspace{.2em}
\hspace*{-\parindent}\begin{minipage}{\linewidth}
\setlength{\parindent}{\horizontalindent}
\setlength{\parskip}{\spacebetweenparagraphs}

\en{Here’re}\ru{Вот} \en{theses}\ru{тезисы} \en{of the~dynamics of a~collection of particles}\ru{динамики совокупности частиц}.

\en{The~force}\ru{Сила}\ru{,} \en{acting}\ru{действующая} \en{on}\ru{на} \en{the} $k$\hbox{-}\en{th}\ru{ую} \en{particle}\ru{частицу}~(\figureref{fig:particlesandforces})

\nopagebreak\begin{equation}\label{forceactingonparticle}
m_k \hspace{.2ex} \mathdotdotabove{\locationvector}_k \hspace{-0.1ex}
= \bm{F}_k \hspace{.1ex} ,
\hspace{.4em}
\bm{F}_k \hspace{-0.1ex}
= \hspace{-0.1ex} \bm{F}^{\smthexternal}_{\hspace{-0.16ex}k} \hspace{-0.1ex}
+ \scalebox{0.88}{$ \displaystyle \underset{\raisemath{.25ex}{\smash{j}}}{\sum} $} \hspace{.2ex} \bm{F}^{\smthinternal}_{\hspace{-0.16ex}kj}
\hspace{-0.1ex} .
\end{equation}

%\begin{wrapfigure}[15]{r}{.55\textwidth}
\begin{wrapfigure}[\en{15}\ru{16}]{r}{.55\textwidth}
\makebox[.6\textwidth][c]{\begin{minipage}[t]{.58\textwidth}
\vspace{-.6em}

\begin{tikzpicture}[scale=1.05]

% arguments: name, x, y, radius
\newcommand{\setpointmass}[4]{%
\path (#2, #3) node [shape=coordinate] (#1) {} ;
\def\linewidth{1.6pt}
\pgfmathsetmacro\radiusoffset{#4 - \linewidth}
\path (#1) node [line width=1pt, minimum size=#4, circle, inner sep=\radiusoffset, outer sep=0] (#1circle) {} ;
%%\draw [line width=\linewidth, black, fill=black!50] (#1) circle (#4) ;
%%\path (#1circle) node [black, above left, inner sep=6pt, outer sep=0] {#1} ;
}

% arguments: name, radius, color
\newcommand{\drawpointmass}[3]{%
\def\linewidth{1.6pt}
\draw [line width=\linewidth, #3, fill=#3!50] (#1) circle (#2) ;
}

% arguments: name, color, node options, text
\newcommand{\labelpointmass}[4]{%
\path (#1circle) node [#2, #3] {#4} ;
}

% arguments: name of point, name of force, vector length, vector angle in degrees, color
\newcommand{\forceatpoint}[5]{
\tikzstyle{force line} = [line width=1.25pt, line cap=round, -{Triangle[round, length=4.2mm, width=2.7mm]}]

% (from)!length!angle:(to)
\path ($ (#1)!#3!#4:($ (#1) + (0, #3) $) $) node [shape=coordinate] (#1#2force) {} ;

\draw [force line, #5] (#1#2force) -- (#1circle) {} ;
}

% arguments: name of point, name of other point, name of force, force vector length, color
\newcommand{\forcebetweenpoints}[5]{
\tikzstyle{force line} = [line width=1.25pt, line cap=round, -{Triangle[round, length=4.2mm, width=2.7mm]}]

\path ($ (#1)!#4!0:(#2) $) node [shape=coordinate] (#1#3force) {} ;

\draw [force line, #5] (#1#3force) -- (#1circle) {} ;
}

% arguments: name of point, name of force, position, node options, text
\newcommand{\labelforceatpoint}[5]{%
\node at ($ (#1#2force)!#3!(#1) $) [#4] {#5} ;
}

\tikzset{%
radiiline/.style={line cap=round, dash pattern=on 0pt off 1.6\pgflinewidth, -{Stealth[round, length=3.8mm, width=2.7mm]}}%
}

\def\mthirdcolor{green!77!yellow!88!black}

\setpointmass{m1}{35mm}{-28mm}{2mm}
\setpointmass{m2}{27mm}{19mm}{2mm}
\setpointmass{m3}{13mm}{-13mm}{2mm}

\path (63mm, 8mm) node [shape=coordinate] (O) {} ;

% draw position vectors

\draw [radiiline, line width=1.2pt, black!55] (O) -- (m1circle)
	node [red, pos=0.56, below right, inner sep=3pt, outer sep=0] {$\locationvector_1$} ;

\draw [radiiline, line width=1.2pt, black!55] (O) -- (m2circle)
	node [magenta, pos=0.54, above, inner sep=4.7pt, outer sep=0] {$\locationvector_2$} ;

\draw [radiiline, line width=1.2pt, black!55] (m2circle) -- (m1circle)
	node [black, pos=0.55, above right, inner sep=3.3pt, outer sep=0] {$\mathcolor{red}{\locationvector_1} \hspace{-0.25ex} - \mathcolor{magenta}{\locationvector_2}$} ;

\draw [radiiline, line width=1.2pt, black!55] (m3circle) -- (m2circle)
	node [black, pos=0.47, above left, inner sep=2.6pt, outer sep=0] {$\mathcolor{magenta}{\locationvector_2} \hspace{-0.25ex} - \mathcolor{\mthirdcolor}{\locationvector_3}$} ;

\draw [radiiline, line width=1.2pt, black!55] (m3circle) -- (m1circle)
	node [black, pos=0.49, below left, inner sep=2.1pt, outer sep=0] {$\mathcolor{red}{\locationvector_1} \hspace{-0.25ex} - \mathcolor{\mthirdcolor}{\locationvector_3}$} ;

% draw force vectors

\forceatpoint{m1}{ext}{10mm}{263}{black}
\labelforceatpoint{m1}{ext}{0.22}{below, outer sep=6.3pt, inner sep=0}{${\bm{F}^{\smthexternal}_{\hspace{-0.15ex}1}}$}

\forceatpoint{m2}{ext}{13.5mm}{103}{black}
\labelforceatpoint{m2}{ext}{0.17}{above, outer sep=3.3pt, inner sep=0}{${\bm{F}^{\smthexternal}_{\hspace{-0.15ex}2}}$}

\forceatpoint{m3}{ext}{8.8mm}{22}{black}
\labelforceatpoint{m3}{ext}{0.34}{left, outer sep=4.7pt, inner sep=0}{${\bm{F}^{\smthexternal}_{\hspace{-0.15ex}3}}$}

\forcebetweenpoints{m1}{m2}{int12}{16mm}{magenta}{-2mm}
\labelforceatpoint{m1}{int12}{0.27}{magenta, right, outer sep=3pt, inner sep=0}{${\bm{F}^{\smthinternal}_{\raisebox{-0.1em}{$\scriptstyle \hspace{-0.25ex}12$}}}$}

\forcebetweenpoints{m2}{m1}{int21}{16mm}{red}{2mm}
\labelforceatpoint{m2}{int21}{0.3}{red, right, outer sep=3pt, inner sep=0}{${\bm{F}^{\smthinternal}_{\raisebox{-0.1em}{$\scriptstyle \hspace{-0.25ex}21$}}}$}

\forcebetweenpoints{m3}{m2}{int32}{12.5mm}{magenta}
\labelforceatpoint{m3}{int32}{0.38}{magenta, right, outer sep=4pt, inner sep=0}{${\bm{F}^{\smthinternal}_{\raisebox{-0.1em}{$\scriptstyle \hspace{-0.25ex}32$}}}$}

\forcebetweenpoints{m2}{m3}{int23}{12.5mm}{\mthirdcolor}
\labelforceatpoint{m2}{int23}{0.63}{\mthirdcolor, below left, outer sep=7.2pt, inner sep=0}{${\bm{F}^{\smthinternal}_{\raisebox{-0.1em}{$\scriptstyle \hspace{-0.25ex}23$}}}$}

\forcebetweenpoints{m1}{m3}{int13}{10.5mm}{\mthirdcolor}
\labelforceatpoint{m1}{int13}{0.37}{\mthirdcolor, below left, outer sep=2.5pt, inner sep=0}{${\bm{F}^{\smthinternal}_{\raisebox{-0.1em}{$\scriptstyle \hspace{-0.25ex}13$}}}$}

\forcebetweenpoints{m3}{m1}{int31}{10.5mm}{red}
\labelforceatpoint{m3}{int31}{0.28}{red, above right, outer sep=0.8pt, inner sep=0}{${\bm{F}^{\smthinternal}_{\raisebox{-0.1em}{$\scriptstyle \hspace{-0.25ex}31$}}}$}

% draw points

\drawpointmass{m1}{1.8mm}{red}
\labelpointmass{m1}{red}{below left, yshift=-2pt, inner sep=4.7pt, outer sep=0}{$m_1$}

\drawpointmass{m2}{1.8mm}{magenta}
\labelpointmass{m2}{magenta}{above left, xshift=1.4pt, inner sep=6.9pt, outer sep=0}{$m_2$}

\drawpointmass{m3}{1.8mm}{\mthirdcolor}
\labelpointmass{m3}{\mthirdcolor}{left, yshift=-2pt, inner sep=9pt, outer sep=0}{$m_3$}

\draw [line width=1.2pt, blue, fill=white] (O) circle (1mm) ;

\end{tikzpicture}

\vspace{-0.3em}\caption{}\label{fig:particlesandforces}
\end{minipage}}
\end{wrapfigure}

\vspace{-0.66em}\noindent
${\bm{F}^{\smthexternal}_{\hspace{-0.16ex}k}}$ \en{is}\ru{есть} \en{the~external force}\ru{внешняя сила}\:---
\en{such forces}\ru{такие силы} \en{emanate}\ru{исходят} \en{from \hbox{objects}}\ru{от объектов} \en{outside}\ru{вне} \en{the~considered system}\ru{рассматриваемой системы}.
\en{The~second addend}\ru{Второе слагаемое}\en{ is}\ru{\:---} \en{the sum of internal forces}\ru{сумма внутренних сил}
(\en{force}\ru{сила} ${\bm{F}^{\smthinternal}_{\hspace{-0.16ex}kj}}$ \en{is}\ru{есть} \en{the interaction}\ru{взаимодействие}\ru{,} \en{induced}\ru{подаваемое} \en{by the~}$j$\hbox{-}\en{th}\ru{ой} \en{particle}\ru{частицей} \en{on}\ru{на} \en{the~}$k$\hbox{-}\en{th}\ru{ую} \en{particle}\ru{частицу}).
\en{Internal interactions}\ru{Внутренние взаимодействия} \en{happen}\ru{случаются} \en{only}\ru{только} \en{between}\ru{между} \en{elements}\ru{элементами} \en{of the observed system}\ru{наблюдаемой системы} \en{and}\ru{и} \en{don’t affect}\ru{не~влияют} (\en{mechanically}\ru{механически}) \en{anything other}\ru{ни на что другое}.
\en{Neither particle interacts with itself}\ru{Ни~одна частица не~взаимодействует сама с~собой}, ${\bm{F}^{\smthinternal}_{\hspace{-0.16ex}kk} \hspace{-0.25ex} = \hspace{-0.15ex} \bm{0} \hspace{.3em} \forall k}$.

% la force Fkj est l’interaction induite par la j-me particule sur la particule k-ème
% induce (verb) = induire, inciter, amener, produire, provoquer, persuader
% взаимодействие, подаваемое j-ой частицей на k-ую частицу

\en{From}\ru{Из}~\eqref{forceactingonparticle}
\en{together with }\ru{вместе с~}\en{the~action--re\-action principle}\ru{принципом действия--противодействия}

\nopagebreak\begin{equation*}%%\label{actionreactionprinciple.fordiscretepoints}
\bm{F}^{\smthinternal}_{\hspace{-0.16ex}kj} = - \hspace{.12ex} \bm{F}^{\smthinternal}_{\hspace{-0.4ex}j\hspace{-0.05ex}k}
\hspace{.44em} \forall k,j
\hspace{.4em} \Rightarrow \hspace{.4em}
\scalebox{0.88}{$ \displaystyle\sum_{\smash{k}} $}
\scalebox{0.88}{$ \displaystyle\underset{\raisemath{.2ex}{\smash{j}}}{\sum} $} \hspace{.2ex}
\bm{F}^{\smthinternal}_{\hspace{-0.16ex}kj} \hspace{-0.2ex}
= \hspace{-0.1ex} \bm{0}
\hspace{.1ex} ,
\end{equation*}

\end{minipage}

\vspace{-0.6em}\noindent
\en{ensues}\ru{вытекает} \en{the~balance of~momentum}\ru{баланс импульса}

\nopagebreak\ru{\vspace{-0.1em}}\begin{equation}\label{balanceoftranslationalmomentum.discretepoints}
\biggl( \displaystyle\sum_{\smash{k}} m_k \hspace{.2ex} \mathdotabove{\locationvector}_k \hspace{-0.2ex} \biggr)^{\hspace{-0.2em}\tikz[baseline=-0.2ex]\draw[black, fill=black] (0,0) circle (.28ex);} \hspace{-0.15ex}
= \hspace{.1ex} \displaystyle\sum_{\smash{k}} m_k \hspace{.2ex} \mathdotdotabove{\locationvector}_k
= \displaystyle\sum_{\smash{k}} \bm{F}^{\smthexternal}_{\hspace{-0.16ex}k}
\hspace{-0.1ex} .
\end{equation}

\vspace{-0.1em}
\en{The~moment}\ru{Момент}\ru{,} \en{acting}\ru{действующий} \en{on}\ru{на} \en{the} $k$\hbox{-}\en{th}\ru{ую} \en{particle}\ru{частицу}

\nopagebreak\begin{equation}\label{momentactingonparticle}
\locationvector_k \hspace{-0.2ex} \times \hspace{-0.2ex} m_k \hspace{.2ex} \mathdotdotabove{\locationvector}_k \hspace{-0.1ex}
= \locationvector_k \hspace{-0.2ex} \times \hspace{-0.2ex} \bm{F}_k \hspace{-0.1ex}
= \locationvector_k \hspace{-0.2ex} \times \hspace{-0.2ex} \bm{F}^{\smthexternal}_{\hspace{-0.16ex}k} \hspace{-0.2ex}
+ \locationvector_k \hspace{-0.2ex} \times \hspace{-0.2ex} \scalebox{0.88}{$ \displaystyle \underset{\raisemath{.25ex}{\smash{j}}}{\sum} $} \hspace{.2ex} \bm{F}^{\smthinternal}_{\hspace{-0.16ex}kj}
\hspace{-0.1ex} .
\end{equation}

\vspace{-0.6em}\noindent
It is relative to the~reference point.

\en{When in addition}\ru{Когда вдобавок} \en{to the~action--re\-action principle}\ru{к~принципу действия--противодействия},
\en{internal interactions}\ru{внутренние взаимодействия} \en{between particles}\ru{между частицами} \en{are central}\ru{центральны},
\en{that~is}\ru{то~есть}

\nopagebreak\vspace{-0.1em}\begin{equation*}
\bm{F}^{\smthinternal}_{\hspace{-0.16ex}kj} \hspace{.15ex} \parallel \hspace{.1ex} \bigl( \hspace{.1ex} \locationvector_k \hspace{-0.15ex} - \locationvector_{\hspace{-0.16ex}j} \hspace{.1ex} \bigr)
\hspace{.33em} \Leftrightarrow \hspace{.33em}
\bigl( \hspace{.1ex} \locationvector_k \hspace{-0.15ex} - \locationvector_{\hspace{-0.16ex}j} \hspace{.1ex} \bigr) \hspace{-0.35ex} \times \hspace{-0.2ex} \bm{F}^{\smthinternal}_{\hspace{-0.16ex}kj} = \bm{0}
\hspace{.1ex} ,
\end{equation*}

\vspace{-0.1em}\noindent
\en{the~balance of~angular (rotational) momentum}\ru{баланс момента импульса}
\en{comes out}\ru{выходит}\footnote{${%
\Bigl( \hspace{.2ex} \scalebox{.8}{$ \displaystyle\sum_{\smash{k}} $} \hspace{.25ex} \locationvector_k \hspace{-0.25ex} \times \hspace{-0.2ex} m_k \hspace{.2ex} \mathdotabove{\locationvector}_k \Bigr)^{\hspace{-0.15em}\tikz[baseline=-0.2ex]\draw[black, fill=black] (0,0) circle (.28ex);} \hspace{-0.2ex}
= \hspace{.1ex}
\scalebox{.8}{$ \displaystyle\sum_{\smash{k}} $} \hspace{.25ex} \mathdotabove{\locationvector}_k \hspace{-0.25ex} \times \hspace{-0.2ex} m_k \hspace{.2ex} \mathdotabove{\locationvector}_k \hspace{-.15ex}
+ \scalebox{.8}{$ \displaystyle\sum_{\smash{k}} $} \hspace{.25ex} \locationvector_k \hspace{-0.25ex} \times \hspace{-0.2ex} m_k \hspace{.2ex} \mathdotdotabove{\locationvector}_k \hspace{-.1ex}
= \hspace{.1ex}
\scalebox{.8}{$ \displaystyle\sum_{\smash{k}} $} \hspace{.25ex} \locationvector_k \hspace{-0.25ex} \times \hspace{-0.2ex} m_k \hspace{.2ex} \mathdotdotabove{\locationvector}_k
\hspace{.2ex} ,
}$
\\[.33em]
${
\bm{F}^{\smthinternal}_{\hspace{-0.16ex}kj} = - \hspace{.12ex} \bm{F}^{\smthinternal}_{\hspace{-0.4ex}j\hspace{-0.05ex}k}
\hspace{.55em} \text{\en{and}\ru{и}} \hspace{.55em}
\bigl( \hspace{.1ex} \locationvector_k \hspace{-0.15ex} - \locationvector_{\hspace{-0.16ex}j} \hspace{.1ex} \bigr) \hspace{-0.35ex} \times \hspace{-0.2ex} \bm{F}^{\smthinternal}_{\hspace{-0.16ex}kj}
= \bm{0}
\hspace{.4em} \Rightarrow}$
\\[-1.1em]
\begin{flushright}
${\Rightarrow \hspace{.4em}
\scalebox{.8}{$ \displaystyle\sum_{\smash{k=1}}^{\smash{N}} $} \hspace{.1ex} \locationvector_k \hspace{-0.25ex} \times \hspace{-0.2ex} \scalebox{0.8}{$ \displaystyle\sum_{\smash{j=1}}^{\smash{N}} $} \hspace{.2ex} \bm{F}^{\smthinternal}_{\hspace{-0.16ex}kj}
= \scalebox{0.8}{$ \displaystyle\sum_{\smash{k=1}}^{\smash{N}} $} \hspace{.4ex} \scalebox{0.8}{$ \displaystyle\sum_{\smash{j=k+1}}^{\smash{N}} $} \hspace{-.4ex}
\bigl( \hspace{.1ex} \locationvector_k \hspace{-0.15ex} - \locationvector_{\hspace{-0.16ex}j} \hspace{.1ex} \bigr) \hspace{-.25ex} \times \hspace{-0.2ex} \bm{F}^{\smthinternal}_{\hspace{-0.16ex}kj}
= \bm{0}
}$
\end{flushright}%
}

\nopagebreak\vspace{-0.2em}\begin{equation}\label{balanceofrotationalmomentum.discretepoints}
\biggl( \displaystyle\sum_{\smash{k}} \locationvector_k \hspace{-0.1ex} \times \hspace{-0.1ex} m_k \hspace{.2ex} \mathdotabove{\locationvector}_k \hspace{-0.25ex} \biggr)^{\hspace{-0.2em}\tikz[baseline=-0.2ex]\draw[black, fill=black] (0,0) circle (.28ex);} \hspace{-0.15ex}
= \hspace{.16ex}
\displaystyle\sum_{\smash{k}} \locationvector_k \hspace{-0.15ex} \times \hspace{-0.2ex} \bm{F}^{\smthexternal}_{\hspace{-0.16ex}k}
.
\end{equation}

\vspace{-0.2em}
\en{Changes in momentum and angular momentum}\ru{Изменения импульса и момента импульса} \en{are determined}\ru{определяются} \en{only by external forces}\ru{только внешними силами}~{$\bm{F}^{\smthexternal}_{\hspace{-0.16ex}k}$}.

...

\newpage

\en{\section{Absolutely rigid undeformable solid body}}

\ru{\section{Совершенно жёсткое недеформируемое твёрдое тело}}

\label{para:absolutelyrigidundeformablesolidbody}

%% \inquotesx{Абсолютно твёрдое}[,] оно~же \inquotes{абсолютно жёсткое} и~\inquotesx{абсолютно прочное}[---] несбыточная мечта каждого инженера.

\en{An~absolutely rigid undeformable}\ru{Совершенно жёсткое недеформируемое} \en{body}\ru{тело} \en{is a~solid}\ru{это твёрдое}\footnote{%
\inquotes{\en{Rigid}\ru{Жёсткое}} \en{is inelastic}\ru{это неупругое} \en{and not flexible}\ru{и~не~гибкое}, \en{and}\ru{а}~\inquotes{\en{solid}\ru{твёрдое}} \en{is not fluid}\ru{это не~текучее}.
\en{A~solid substance}\ru{Твёрдое вещество} \en{retains}\ru{сохраняет} \en{its size and shape}\ru{свой размер и~форму} \en{without a~container}\ru{без контейнера} (\en{as~opposed to a~fluid substance}\ru{в~отличие от текучего вещества}\:--- \en{liquid or gas}\ru{жидкости или газа}).%
}\hspace{-0.2ex} \en{body}\ru{тело}, \en{in which}\ru{в~котором} \en{deformation}\ru{деформация} \en{is zero}\ru{нулевая} (\en{or}\ru{или} \en{negligibly small}\ru{пренебрежимо мала}\:--- \en{so small}\ru{так мала,} \en{that it can be neglected}\ru{что ею можно пренебречь}).
\en{The~distance}\ru{Расстояние} \en{between}\ru{между} \en{any two points}\ru{любыми двумя точками} \en{of a~non-deformable rigid body}\ru{недеформируемого жёсткого тела} \en{remains constant}\ru{остаётся постоянным} \en{regardless of external forces exerted on it}\ru{независимо от действующих на~него внешних сил}.

\en{A~non-deformable rigid body}\ru{Недеформируемое жёсткое тело}
\en{is modeled}\ru{моделируется}\ru{,}
\en{using}\ru{используя}
\en{the~}\inquotesx{\en{continual approach}\ru{континуальный подход}}
\en{as}\ru{как}
\en{a~continuous distribution of~mass}\ru{непрерывное распределение массы}
(\en{a~material continuum, a~continuous medium}\ru{материальный \rucontinuum, сплошная среда}),
\en{rather than using}\ru{вместо использования}
\en{the~}\inquotesx{\en{discrete approach}\ru{дискретного подхода}}
(\en{that is}\ru{то есть}
\en{modeling}\ru{моделирования}
\en{a~body}\ru{т\'{е}ла}
\en{as}\ru{как}
\en{a~discrete collection}\ru{дискретной коллекции}
\en{of particles}\ru{частиц}).

\en{The~mass}\ru{Масса}
\en{of a~material continuum}\ru{материального \rucontinuum{}а}
\en{is continuously distributed}\ru{непрерывно распределяется}
\en{in its volume}\ru{в~своём объёме}

\nopagebreak\vspace{-1.1em}
\begin{equation}\label{continuousdistributionofmass:materialcontinuum}
dm \equiv \rho \hspace{.2ex} d\mathcal{V}
%%\hspace{.1ex} ,
\end{equation}

\vspace{-0.2em}\noindent
(${\rho\hspace{.1ex}}$\ru{\:---}\en{ is} \en{a~volume(tric) mass density}\ru{объёмная плотность массы} \en{and}\ru{и}~${d \mathcal{V}}$\ru{\:---}\en{ is} \en{an~infinitesimal volume}\ru{бесконечно\-м\'{а}лый объём}).

\en{A~formula}\ru{Формула}
\en{with a~summation}\ru{с~суммированием}
\en{over discrete points}\ru{по~дискретным точкам}
\en{becomes}\ru{становится}
\en{a~formula}\ru{формулой}
\en{for a~continuous body}\ru{для сплошного тела}
\en{by replacing}\ru{заменой}
\en{the masses of particles}\ru{масс частиц}
\en{with the mass}\ru{на~массу}~\eqref{continuousdistributionofmass:materialcontinuum}
\en{of an~infinitesimal volume element}\ru{бесконечно\-м\'{а}лого элемента объёма}~${d\mathcal{V}}$,
\en{and}\ru{и}~\en{with the following}\ru{с~последующим}
\en{integration}\ru{интегрированием}
\en{over the~whole volume of~a~body}\ru{по всему объёму тела}.
\en{In~particular}\ru{В~частности},
\en{here are the formulas}\ru{вот формулы}
\en{for}\ru{для}
\en{the~(linear) momentum}\ru{импульса}

\nopagebreak\vspace{-0.2em}\begin{equation}\label{themomentum.discreteandcontinual}
\scalebox{.93}{$ \displaystyle\sum_{\smash{k}} $} \hspace{.3ex} m_k \hspace{.2ex} \mathdotabove{\locationvector}_k
\hspace{.66em} \text{\en{becomes}\ru{становится}} \hspace{.55em}
\scalebox{.87}{$ \displaystyle\integral_{\mathcal{V}} $} \hspace{-0.1ex} \mathdotabove{\locationvector} \hspace{.15ex} dm
\end{equation}

\nopagebreak\vspace{-0.3em}\noindent
\en{and}\ru{и} \en{the~angular (rotational) momentum}\ru{момента импульса}

\nopagebreak\vspace{-0.2em}\begin{equation}\label{therotationalmomentum.discreteandcontinual}
\scalebox{.93}{$ \displaystyle\sum_{\smash{k}} $} \hspace{.3ex} \locationvector_k \hspace{-0.2ex} \times \hspace{-0.2ex} m_k \hspace{.2ex} \mathdotabove{\locationvector}_k
\hspace{.66em} \text{\en{becomes}\ru{становится}} \hspace{.55em}
\scalebox{.87}{$ \displaystyle\integral_{\mathcal{V}} $} \hspace{-0.1ex} \locationvector \hspace{-0.2ex} \times \hspace{-0.2ex} \mathdotabove{\locationvector} \hspace{.15ex} dm
\hspace{.2ex} .
\end{equation}

\en{To fully describe}\ru{Чтобы полностью опис\'{а}ть} \en{the~location (position, place)}\ru{положение (позицию, место)} \en{of any non-deformable body}\ru{любого недеформируемого тела} \en{with all its points}\ru{со всеми своими точками}, \en{it’s enough}\ru{достаточно} \en{to choose}\ru{выбрать} \en{some unique point}\ru{какую\hbox{-}либо уникальную точку}
\en{as}\ru{за} \en{the~}\inquotesx{\en{pole}\ru{полюс}}[,] \en{to~find or to~set}\ru{найти или задать} \en{the~location}\ru{положение} ${\positionofthepole \hspace{.1ex} \narroweq \hspace{.1ex} \positionofthepole(t)}$ \en{of the~chosen point}\ru{выбранной точки}, \en{as well as the~angular orientation}\ru{а~также угловую ориентацию} \en{of a~body}\ru{тела} \en{relative to the~pole}\ru{относительно полюса}~(\figureref{fig:bodyoffsetandrotation}).
\en{As a~result}\ru{Как результат},
\en{any motion}\ru{любое движение}
\en{of an~undeformable rigid body}\ru{недеформируемого твёрдого тела}
\en{is}\ru{есть}
\en{either}\ru{либо}
\en{a~rotation around the~chosen pole}\ru{поворот вокруг выбранного полюса},
\en{or}\ru{либо}
\en{an~equal displacement of the~pole and all body’s points}\ru{равное смещение полюса и~всех точек тела}\:--- \en{translation~(linear motion)}\ru{трансляция~(линейное движение)}\footnote{%
\en{A~translation}\ru{Трансляция} \en{can also be thought of as}\ru{может также быть мыслима как} \en{a~rotation}\ru{вращение} \en{with the~revolution center}\ru{с~центром переворота} \en{at the~infinity}\ru{на~бесконечности}}\hbox{\hspace{-0.3ex},}
\en{or}\ru{либо}
\en{a~combination of~them both}\ru{комбинация их обоих}.

\begin{comment}
\makeatletter
\newcommand\xofcoordinate[2][center]{{%
	\pgfpointanchor{#2}{#1}%
	\pgfmathparse{\pgf@x/\pgf@xx}%
	\pgfmathprintnumber[precision=2]{\pgfmathresult}%
}}
\newcommand\yofcoordinate[2][center]{{%
	\pgfpointanchor{#2}{#1}%
	\pgfmathparse{\pgf@y/\pgf@yy}%
	\pgfmathprintnumber[precision=2]{\pgfmathresult}%
}}
\makeatother

\begin{tikzpicture}
	\coordinate (point0) at (-4.3, 2.5);
	\coordinate (point1) at (-3.1, 3.2);
	\coordinate (point2) at (-2, 2.4);
	\coordinate (point3) at (-0.4, 1.6);
	\coordinate (point4) at (0.5, 0);
	\coordinate (point5) at (0, -2);
	\coordinate (point6) at (-1.5, -3);
	\coordinate (point7) at (-3, -2.2);
	\coordinate (point8) at (-3.5, -0.5);
	\coordinate (point9) at (-4.5, 1);

	\draw [line width=1.2pt, red] plot [smooth cycle, tension=0.8] coordinates {
		(point0) (point1) (point2) (point3) (point4)
		(point5) (point6) (point7) (point8) (point9)
	};

	\newcommand\xyofcoordinate[1]{\xofcoordinate{#1},\,\yofcoordinate{#1}}

	\draw [black, fill=black] (point0) circle (1mm) node [anchor=south east] {\xyofcoordinate{point0}};
	\draw [black, fill=black] (point1) circle (1mm) node [anchor=south, outer sep=4pt] {\xyofcoordinate{point1}};
	\draw [black, fill=black] (point2) circle (1mm) node [anchor=south west] {\xyofcoordinate{point2}};
	\draw [black, fill=black] (point3) circle (1mm) node [anchor=south west] {\xyofcoordinate{point3}};
	\draw [black, fill=black] (point4) circle (1mm) node [anchor=south west] {\xyofcoordinate{point4}};

	\draw [black,fill=black] (point5) circle (1mm) node [anchor=north west, outer sep=1pt] {\xyofcoordinate{point5}};
	\draw [black,fill=black] (point6) circle (1mm) node [anchor=north, outer sep=4pt] {\xyofcoordinate{point6}};
	\draw [black,fill=black] (point7) circle (1mm) node [anchor=north east, outer sep=2pt] {\xyofcoordinate{point7}};
	\draw [black,fill=black] (point8) circle (1mm) node [anchor=east, outer sep=4pt] {\xyofcoordinate{point8}};
	\draw [black,fill=black] (point9) circle (1mm) node [anchor=east, outer sep=3pt] {\xyofcoordinate{point9}};
\end{tikzpicture}
\end{comment}

%%\begin{wrapfigure}{o}{.5\textwidth}
%%\makebox[.45\textwidth][c]{%
%%\begin{minipage}[t]{.45\textwidth}
\begin{figure}[htb!]
\begin{center}
\vspace{-0.2em}
\scalebox{1.1}{
\begin{tikzpicture}[scale=.63]

\def\angleofrotation{44}

\def\Opointx{-1.65}
\def\Opointy{-1.05}
\def\Oinitialpointx{-6} %-5.8
\def\Oinitialpointy{-2.3} %-2.3

\def\bodypointx{-3.5} %-2
\def\bodypointy{2.5} %1.5

\newcommand\drawnotrotatedbasis{
	\draw [line width=1pt, black!50,
		style=double, double distance=0.5mm,
		rotate around={120:(\Opointx, \Opointy)},
		-{Triangle[open, angle=60:3.2mm]}]
		(\Opointx, \Opointy) -- ++(0, 1.6) ;
	\draw [line width=1pt, black!50,
	style=double, double distance=0.5mm, rotate around={-120:(\Opointx, \Opointy)},
	-{Triangle[open, angle=60:3.2mm]}]
		(\Opointx, \Opointy) -- ++(0, 1.6) ;
	\draw [line width=1pt, black!50,
		style=double, double distance=0.5mm, -{Triangle[open, angle=60:3.2mm]}]
		(\Opointx, \Opointy) -- ++(0, 1.6)
		node [pos=.93, above, inner sep=0pt, outer sep=3.5pt]
		{$ \widetilde{\bm{e}}_i $} ;
}

%%\newcommand\setundeformablebody{
%%	\coordinate (point0) at (-4.3, 2.5);
%%	\coordinate (point1) at (-3.1, 3.2);
%%	\coordinate (point2) at (-2, 2.4);
%%	\coordinate (point3) at (-0.4, 1.6);
%%	\coordinate (point4) at (0.5, 0);
%%	\coordinate (point5) at (0, -2);
%%	\coordinate (point6) at (-1.5, -3);
%%	\coordinate (point7) at (-3, -2.2);
%%	\coordinate (point8) at (-3.5, -0.5);
%%	\coordinate (point9) at (-4.5, 1);
%%}

\newcommand\drawnotrotatedundeformablebody{
	\begin{scope}[rotate around={-\angleofrotation:(\Opointx, \Opointy)}]
	\draw [line width=1pt, black!50, opacity=50]
		plot [smooth cycle, tension=0.8] coordinates {
			(-4.3, 2.5) (-3.1, 3.2) (-2, 2.4) (-0.4, 1.6) (0.5, 0)
			(0, -2) (-1.5, -3) (-3, -2.2) (-3.5, -0.5) (-4.5, 1)
		};
	\end{scope}
}

\newcommand\drawrotatedundeformablebody{
	\draw [line width=1.6pt, black]
		plot [smooth cycle, tension=0.8] coordinates {
			(-4.3, 2.5) (-3.1, 3.2) (-2, 2.4) (-0.4, 1.6) (0.5, 0)
			(0, -2) (-1.5, -3) (-3, -2.2) (-3.5, -0.5) (-4.5, 1)
			%% (point0) (point1) (point2) (point3) (point4)
			%% (point5) (point6) (point7) (point8) (point9)
		};
}

\newcommand\drawnotrotatedtorotated{
	\tkzDefPoint(\bodypointx, \bodypointy){bodypointnotrotated}
	\begin{scope}[rotate around={-\angleofrotation:(\Opointx, \Opointy)}]
	\tkzDefPoint(\Opointx, \Opointy){centerpoint}
	\tkzDefPoint(\bodypointx, \bodypointy){bodypoint}
	\tkzDrawArc[line width=1pt, color=black!50, opacity=50](centerpoint,bodypoint)(bodypointnotrotated) ;

	\path (\bodypointx, \bodypointy) circle (2mm) node [shape=circle, inner sep=.9mm, outer sep=0] (previousbodypoint) {};

	\draw [line width=1pt, black!50, opacity=50, -{Stealth[round, length=4.5mm, width=2.8mm]}]
		(\Opointx, \Opointy) -- (previousbodypoint)
		node [pos=0.57, color=black!50, opacity=99, right, inner sep=0pt, outer sep=3.5pt]
		{$ \widetilde{\bm{x}} $} ;

	\fill [white] (\bodypointx, \bodypointy) circle (2mm) ;
	\draw [line width=1pt, color=black!50, opacity=50] (\bodypointx, \bodypointy) circle (2mm) ;
	\end{scope}
}

\newcommand\drawfirstversionvectors{
	\draw [line width=1.6pt, black, fill=white] (\bodypointx, \bodypointy) circle (2mm)
		node [shape=circle, inner sep=0.9mm, outer sep=0] (pointcirc) {} ;

	\draw [line width=1.6pt, black, -{Stealth[round, length=5mm, width=3.6mm]}] (\Oinitialpointx, \Oinitialpointy) -- (pointcirc)
		node [pos=0.5, above left, inner sep=0pt, outer sep=1.5pt] {$ \locationvector $} ;

	\path (\Opointx, \Opointy) circle (1.6mm) node [shape=circle, inner sep=.64mm, outer sep=0] (Ocirc) {} ;

	\draw [line width=1.6pt, blue, -{Stealth[round, length=5mm, width=3.6mm]}] (\Oinitialpointx, \Oinitialpointy) -- (Ocirc)
		node [pos=0.48, below, inner sep=0pt, outer sep=5pt] {$ \positionofthepole $};

	\draw [line width=1.6pt, black, -{Stealth[round, length=5mm, width=3.6mm]}] (\Opointx, \Opointy) -- (pointcirc)
		node [pos=0.63, left, inner sep=2.5pt, outer sep=3.3pt] {$ \bm{x} $} ;
}

\newcommand\drawsecondversionvectors{
	\draw [line width=1.6pt, black, fill=white] (\bodypointx, \bodypointy) circle (2mm)
		node [shape=circle, inner sep=0.9mm, outer sep=0] (pointcirc) {} ;

	\draw [line width=1.6pt, black, -{Stealth[round, length=5mm, width=3.6mm]}] (\Oinitialpointx, \Oinitialpointy) -- (pointcirc)
		node [pos=0.5, above left, inner sep=0pt, outer sep=1.5pt] {$ \locationvector $} ;

	\path (\Oinitialpointx, \Oinitialpointy) circle (1.6mm) node [shape=circle, inner sep=.64mm, outer sep=0] (Oinitialcirc) {} ;

	\draw [line width=1.6pt, blue, -{Stealth[round, length=5mm, width=3.6mm]}] (\Opointx, \Opointy) -- (Oinitialcirc)
		node [pos=0.6, below, inner sep=0pt, outer sep=4.4pt] {$ - \hspace{.2ex} \positionofthepole $};

	\draw [line width=1.6pt, black, -{Stealth[round, length=5mm, width=3.6mm]}] (\Opointx, \Opointy) -- (pointcirc)
		node [pos=0.63, left, inner sep=2.5pt, outer sep=3.3pt] {$ \bm{x} $} ;
}

\newcommand\drawbodybasis{
	\draw [line width=1pt, blue, rotate around={{\angleofrotation + 120}:(\Opointx, \Opointy)},
		style=double, double distance=0.5mm, -{Triangle[open, angle=60:3.2mm]}]
		(\Opointx, \Opointy) -- ++(0, 1.6);
	\draw [line width=1pt, blue, rotate around={{\angleofrotation - 120}:(\Opointx, \Opointy)},
		style=double, double distance=0.5mm, -{Triangle[open, angle=60:3.2mm]}]
		(\Opointx, \Opointy) -- ++(0, 1.6);
 	\draw [line width=1pt, blue, rotate around={\angleofrotation:(\Opointx, \Opointy)},
		style=double, double distance=0.5mm, -{Triangle[open, angle=60:3.2mm]}]
		(\Opointx, \Opointy) -- ++(0, 1.6);

	\draw [line width=1pt, blue, fill=white] (\Opointx, \Opointy) circle (1.6mm)
		node [below right, inner sep=0pt, outer sep=3.5pt, xshift=-.7mm, yshift=-2.5mm] {$ \bm{e}_i $} ;
}

\newcommand\drawhatchlines{
	\def\hatchlength{.3}
	\def\loopfirst{.55}
	\def\looplast{1.15}
	\pgfmathsetmacro\loopstep{(\looplast - \loopfirst) / 2}
	\pgfmathsetmacro\loopsecond{\loopfirst + \loopstep}
	\foreach \econnection in {\loopfirst, \loopsecond, ..., \looplast} {
		\draw [line width=.5pt, color=blue]
			($ (\Oinitialpointx, \Oinitialpointy) + (0, \econnection) $) -- ++(-\hatchlength, -\hatchlength) ;
		\draw [line width=.5pt, color=blue, rotate around={120:(\Oinitialpointx, \Oinitialpointy)}]
			($ (\Oinitialpointx, \Oinitialpointy) + (0, \econnection) $) -- ++(-\hatchlength, -\hatchlength) ;
		\draw [line width=.5pt, color=blue, rotate around={-120:(\Oinitialpointx, \Oinitialpointy)}]
			($ (\Oinitialpointx, \Oinitialpointy) + (0, \econnection) $) -- ++(\hatchlength, -\hatchlength) ;
	}
}

\newcommand\drawabsolutebasis{
	\draw [line width=1pt, blue,
		style=double, double distance=0.5mm, rotate around={120:(\Oinitialpointx, \Oinitialpointy)}, -{Triangle[open, angle=60:3.2mm]}]
		(\Oinitialpointx, \Oinitialpointy) -- ++(0, 1.6);
	\draw [line width=1pt, blue,
		style=double, double distance=0.5mm, rotate around={-120:(\Oinitialpointx, \Oinitialpointy)}, -{Triangle[open, angle=60:3.2mm]}]
		(\Oinitialpointx, \Oinitialpointy) -- ++(0, 1.6);
 	\draw [line width=1pt, blue,
		style=double, double distance=0.5mm, -{Triangle[open, angle=60:3.2mm]}]
		(\Oinitialpointx, \Oinitialpointy) -- ++(0, 1.6);

	\draw [line width=1pt, blue, fill=white] (\Oinitialpointx, \Oinitialpointy) circle(1.6mm)
		node [anchor=north, inner sep=0pt, outer sep=8pt, yshift=-1.1mm, xshift=.33mm]
			{$ \mathcircabove{\bm{e}}_i $};
}

	%%draw undeformable body

	\drawnotrotatedbasis

	\drawnotrotatedundeformablebody

	\drawnotrotatedtorotated

	\drawrotatedundeformablebody

	\drawfirstversionvectors

	\drawbodybasis

	\drawhatchlines
	\drawabsolutebasis

\pgfmathsetmacro\textpositionx{.5 + \Oinitialpointx}
\pgfmathsetmacro\textpositiony{\Oinitialpointy + 4}

\node [anchor=east] at (\textpositionx, \textpositiony)
	{$ \locationvector = \positionofthepole + \bm{x} $} ;

%%\node [align=center] at (\textpositionx, \textpositiony)
%%	{$ \bm{x} = - \hspace{.2ex} \positionofthepole + \locationvector $} ;

\end{tikzpicture}

\begin{comment}
\makeatletter
\newcommand\xofcoordinate[2][center]{{%
	\pgfpointanchor{#2}{#1}%
	\pgfmathparse{\pgf@x/\pgf@xx}%
	\pgfmathprintnumber[precision=2]{\pgfmathresult}%
}}
\newcommand\yofcoordinate[2][center]{{%
	\pgfpointanchor{#2}{#1}%
	\pgfmathparse{\pgf@y/\pgf@yy}%
	\pgfmathprintnumber[precision=2]{\pgfmathresult}%
}}
\makeatother

\begin{tikzpicture}
	\coordinate (point0) at (-4.3, 2.5);
	\coordinate (point1) at (-3.1, 3.2);
	\coordinate (point2) at (-2, 2.4);
	\coordinate (point3) at (-0.4, 1.6);
	\coordinate (point4) at (0.5, 0);
	\coordinate (point5) at (0, -2);
	\coordinate (point6) at (-1.5, -3);
	\coordinate (point7) at (-3, -2.2);
	\coordinate (point8) at (-3.5, -0.5);
	\coordinate (point9) at (-4.5, 1);

	\draw [line width=1.2pt, red] plot [smooth cycle, tension=0.8] coordinates {
		(point0) (point1) (point2) (point3) (point4)
		(point5) (point6) (point7) (point8) (point9)
	};

	\newcommand\xyofcoordinate[1]{\xofcoordinate{#1},\,\yofcoordinate{#1}}

	\draw [black, fill=black] (point0) circle (1mm) node [anchor=south east] {\xyofcoordinate{point0}};
	\draw [black, fill=black] (point1) circle (1mm) node [anchor=south, outer sep=4pt] {\xyofcoordinate{point1}};
	\draw [black, fill=black] (point2) circle (1mm) node [anchor=south west] {\xyofcoordinate{point2}};
	\draw [black, fill=black] (point3) circle (1mm) node [anchor=south west] {\xyofcoordinate{point3}};
	\draw [black, fill=black] (point4) circle (1mm) node [anchor=south west] {\xyofcoordinate{point4}};

	\draw [black,fill=black] (point5) circle (1mm) node [anchor=north west, outer sep=1pt] {\xyofcoordinate{point5}};
	\draw [black,fill=black] (point6) circle (1mm) node [anchor=north, outer sep=4pt] {\xyofcoordinate{point6}};
	\draw [black,fill=black] (point7) circle (1mm) node [anchor=north east, outer sep=2pt] {\xyofcoordinate{point7}};
	\draw [black,fill=black] (point8) circle (1mm) node [anchor=east, outer sep=4pt] {\xyofcoordinate{point8}};
	\draw [black,fill=black] (point9) circle (1mm) node [anchor=east, outer sep=3pt] {\xyofcoordinate{point9}};
\end{tikzpicture}
\end{comment}


}
\end{center}
\vspace{-1.5em}\caption{}\label{fig:bodyoffsetandrotation}
\vspace{-1.1em}\end{figure}
%%\end{minipage}%%}
%%\end{wrapfigure}

${\mathcircabove{\bm{e}}_i}$\:--- \en{triplet of orthonormal basis vectors}\ru{тройка ортонормальных базисных векторов}, \en{immovable}\ru{неподвижная} \en{relatively to the~absolute (or any inertial) reference system}\ru{относительно абсолютной (или любой инерциальной) системы отсчёта}

\foreignlanguage{russian}{Имея неподвижный базис~${\mathcircabove{\bm{e}}_i}$ \en{and}\ru{и} движущийся вместе с~телом базис~${\bm{e}_i}$, ...}

\foreignlanguage{russian}{Если добавить базис~${\bm{e}_i}$ (этот базис движется вместе с~телом), то угловая ориентация тела может быть определена тензором поворота ${\rotationtensor \equiv \bm{e}_i \widetilde{\bm{e}}_i}$.}

\en{Then}\ru{Тогда} \en{any motion of a~body}\ru{любое движение тела} \en{is completely described}\ru{полностью описывается} \en{by two functions}\ru{двумя функциями}~${\positionofthepole(t)}$ \en{and}\ru{и}~${\rotationtensor(t)}$.

\foreignlanguage{russian}{Вектор положения некоторой точки тела}

\nopagebreak\vspace{-0.2em}\begin{equation}\label{completelyrigidbody.locationvectorofanypointdecomposed}
\locationvector = \positionofthepole + \bm{x}
%%\hspace{.1ex} ,
\end{equation}

${\widetilde{\bm{x}} = x_i \hspace{.1ex} \widetilde{\bm{e}}_i}$,
${\bm{x} = x_i \hspace{.1ex} \bm{e}_i}$

\eqref{rodriguesrotationformula}, \chapterdotsectionref{chapter:mathapparatus}{para:rotationtensor}

${\bm{x} = \rotationtensor \hspace{-0.15ex} \dotp \hspace{.1ex} \widetilde{\bm{x}}}$

\begin{equation*}
\mathdotabove{\locationvector} = \mathdotabove{\positionofthepole} + \mathdotabove{\bm{x}}
\hspace{.1ex} ,
\end{equation*}

\en{For a~non-deformable rigid body}\ru{Для недеформируемого жёсткого тела}, \en{components}\ru{компоненты}~${x_i}$ \en{don’t depend on time}\ru{не~зависят от времени}:
${x_i \hspace{-0.16ex} = \constant(t)}$ \en{and}\ru{и}~${\mathdotabove{\bm{x}} = x_i \hspace{.1ex} \mathdotabove{\bm{e}}_i}$

${\mathdotabove{\bm{x}} = \mathdotabove{\rotationtensor} \hspace{-0.15ex} \dotp \hspace{.1ex} \mathcircabove{\bm{x}}}$

${x_i \mathdotabove{\bm{e}}_i \hspace{-0.12ex} = \mathdotabove{\rotationtensor} \hspace{-0.15ex} \dotp x_i \hspace{.1ex} \mathcircabove{\bm{e}}_i
\:\Leftrightarrow\:
\mathdotabove{\bm{e}}_i \hspace{-0.12ex} = \mathdotabove{\rotationtensor} \hspace{-0.15ex} \dotp \mathcircabove{\bm{e}}_i}$

...

\en{The~linear momentum}\ru{Импульс (количество движения)} \en{and}\ru{и} \en{the~rotational~(angular) momentum}\ru{момент импульса} \en{of a~non-deformable continuous body}\ru{недеформируемого сплошного тела} \en{are described by the following integrals}\ru{описываются следующими интегралами}

...

...

\hspace{-0.4ex}\begin{equation*}

\displaystyle\integral_{\mathcal{V}} \hspace{-0.6ex} \positionofthepole \hspace{.2ex} dm
= \hspace{.1ex} \positionofthepole \hspace{-0.5ex} \displaystyle\integral_{\mathcal{V}} \hspace{-0.6ex} dm
= \hspace{.1ex} \positionofthepole \hspace{.2ex} m
\]

\[
\hspace{-0.4ex} \displaystyle\integral_{\mathcal{V}} \hspace{-0.6ex} \bm{x} \hspace{.1ex} dm = \hspace{.15ex} \bm{\Xi} \hspace{.2ex} m
\hspace{.1ex} , \:\:
\bm{\Xi} \hspace{.1ex} \equiv m^{\hspace{-0.1ex}\expminusone} \hspace{-0.5ex} \displaystyle\integral_{\mathcal{V}} \hspace{-0.6ex} \bm{x} \hspace{.1ex} dm
\]

\en{Three}\ru{Три} \en{inertial characteristics}\ru{инерциальных характеристики} \en{of the~body}\ru{тела}:

\nopagebreak\vspace{.2em}\begin{itemize}
\item \en{integral mass}\ru{интегральная масса} ${m = \hspace{-0.25ex}\scalebox{1.4}{$ \textstyle\integral $}_{\hspace{-0.55ex}\mathcal{V}} \hspace{.3ex} dm = \hspace{-0.25ex}\scalebox{1.4}{$ \textstyle\integral $}_{\hspace{-0.55ex}\mathcal{V}} \hspace{.3ex} \rho \hspace{.2ex} d\mathcal{V}}$\:---
\en{the~mass of the~whole body}\ru{масса всего тела},
\vspace{.2em}
\item \en{eccentricity vector}\ru{вектор экцентриситета}\hbox{~\hspace{.2ex}}$\bm{\Xi}$\:--- \en{measures}\ru{измеряет} \en{the~offset}\ru{смещение} \en{of the chosen pole}\ru{выбранного полюса} \en{from}\ru{от} \en{the body’s }{\inquotes{\en{center of mass}\ru{центра масс}}}\ru{ тела},
%%\en{and}\ru{и}
\vspace{.2em}
\item \en{inertia tensor}\ru{тензор инерции}~${\inertiatensor}$.
\end{itemize}

\en{The eccentricity vector}\ru{Вектор экцентриситета} \en{is equal to}\ru{равняется} \en{the~null vector}\ru{нуль\hbox{-}вектору} \en{only when}\ru{только когда} \en{the chosen pole}\ru{выбранный полюс} \en{coincides with}\ru{совпадает с} \en{the~}\inquotesx{\en{center of~mass}\ru{центром масс}}[---] \en{the~unique point}\ru{уникальной точкой} \en{within a~body}\ru{внутри тела} \en{with }\ru{с~}\en{location vector}\ru{вектором положения}~${\mathboldrcursive\hspace{.2ex}}$, \en{in short}\ru{короче}

\nopagebreak\vspace{-0.2em}\begin{equation*}
\bm{\Xi} = \bm{0}
\hspace{.6ex} \Leftrightarrow \hspace{.5ex}
\positionofthepole = \hspace{.1ex} \mathboldrcursive
\hspace{.3ex} .
\end{equation*}

\begin{gather*}
\bm{x} = \hspace{-0.1ex} \locationvector - \positionofthepole
, \hspace{.6em}
\bm{\Xi} \hspace{.2ex} m = \hspace{-0.4ex} \integral_{\mathcal{V}} \hspace{-0.4ex} \bigl( \locationvector - \hspace{-0.1ex} \mathboldrcursive \hspace{.2ex} \bigr) dm = \hspace{.1ex} \bm{0}
\hspace{.1ex} ,
\\[-0.4em]
%
\integral_{\mathcal{V}} \hspace{-0.7ex} \locationvector \hspace{.25ex} dm \hspace{.1ex}
- \hspace{.2ex} \mathboldrcursive \hspace{-0.3ex} \integral_{\mathcal{V}} \hspace{-0.7ex} dm = \hspace{.1ex} \bm{0}
\hspace{.7ex} \Rightarrow \hspace{.7ex}
\mathboldrcursive = m^{\hspace{-0.1ex}\expminusone} \hspace{-0.5ex} \integral_{\mathcal{V}} \hspace{-0.7ex} \locationvector \hspace{.25ex} dm
\end{gather*}

...

\en{Introducing}\ru{Вводя} \en{(pseudo)vector}\ru{(псевдо)вектор} \en{of~angular velocity}\ru{угловой скорости}~$\bm{\omega}$, ...

\[
\mathdotabove{\bm{e}}_i \hspace{-0.16ex}
= \bm{\omega} \hspace{-0.2ex} \times \hspace{-0.2ex} \bm{e}_i
\]

...

\en{inertia tensor}\ru{тензор инерции}~${\inertiatensor}$

\nopagebreak\begin{equation*}
\inertiatensor
\equiv
- \hspace{-0.4ex} \integral_{\mathcal{V}} \hspace{-0.4ex} \bigl( \bm{x} \hspace{-0.1ex} \times \hspace{-0.22ex} \UnitDyad \hspace{.1ex} \bigr) \hspace{-0.3ex} \dotp \hspace{-0.2ex} \bigl( \bm{x} \hspace{-0.1ex} \times \hspace{-0.22ex} \UnitDyad \hspace{.1ex} \bigr) \hspace{.1ex} dm
=
\hspace{-0.4ex} \integral_{\mathcal{V}} \hspace{-0.4ex} \bigl( \bm{x} \narrowdotp \bm{x} \UnitDyad - \bm{x} \bm{x} \bigr) \hspace{.1ex} dm
\end{equation*}

It is assumed \textcolor{magenta}{(can be proven?)} that the~inertia tensor changes only due to a~rotation

\vspace{-0.1em}\begin{equation*}
\inertiatensor = \rotationtensor \hspace{-0.1ex} \dotp \inertiatensorcircabove \dotp \rotationtensor^{\hspace{-0.1ex}\T}
\end{equation*}

\vspace{-0.1em}\noindent
and its components in basis~${\bm{e}_i}$ (moving together with a~body) don’t change over time

\nopagebreak\vspace{-0.1em}\begin{equation*}
\inertiatensor = \inertiatensorcomponents{ab} \hspace{.1ex} \bm{e}_a \bm{e}_b
\hspace{.1ex} , \hspace{.8ex}
\inertiatensorcomponents{ab} \hspace{-0.2ex} = \constant(t)
\vspace{-0.2em}
\end{equation*}

\vspace{-0.1em}\noindent
thus the~time derivative is

\nopagebreak\vspace{-0.2em}\begin{multline*}
\inertiatensordotabove
= \inertiatensorcomponents{ab} \bigl( \hspace{.1ex} \mathdotabove{\bm{e}}_a \bm{e}_b \hspace{-0.1ex} + \bm{e}_a \mathdotabove{\bm{e}}_b \hspace{.12ex} \bigr) \hspace{-0.33ex}
= \inertiatensorcomponents{ab} \bigl( \hspace{.1ex} \bm{\omega} \hspace{-0.2ex} \times \hspace{-0.2ex} \bm{e}_a \bm{e}_b \hspace{-0.1ex} + \bm{e}_a \hspace{.2ex} \bm{\omega} \hspace{-0.2ex} \times \hspace{-0.2ex} \bm{e}_b \hspace{.12ex} \bigr) \hspace{-0.33ex}
=
\\[-0.1em]
%
= \inertiatensorcomponents{ab} \bigl( \hspace{.1ex} \bm{\omega} \hspace{-0.2ex} \times \hspace{-0.2ex} \bm{e}_a \bm{e}_b \hspace{-0.1ex} - \bm{e}_a \bm{e}_b \hspace{-0.2ex} \times \hspace{-0.2ex} \bm{\omega} \hspace{.12ex} \bigr) \hspace{-0.33ex}
= \bm{\omega} \hspace{-0.1ex} \times \hspace{-0.1ex} \inertiatensor \hspace{.1ex} - \inertiatensor \times \bm{\omega}
\end{multline*}

\textcolor{magenta}{\en{Substitution of}\ru{Подстановка}}
(....)
\en{into}\ru{в}~\eqref{balanceoftranslationalmomentum.discretepoints} \en{and}\ru{и}~\eqref{balanceofrotationalmomentum.discretepoints}
\en{gives}\ru{даёт}
%%\en{fundamental}\ru{фундаментальные}
\en{equations}\ru{уравнения} \en{of~balance}\ru{баланса}\:--- \en{the~balance of linear momentum}\ru{баланс количества движения (импульса)} \en{and}\ru{и} \en{the~balance of rotational momentum}\ru{баланс момента импульса}\:--- \en{for}\ru{для} \en{a~model}\ru{модели} \en{of~continuous non-deformable rigid body}\ru{сплошного недеформируемого жёсткого тела}

...

\noindent
\en{here}\ru{здесь}
$\bm{f}$\en{ is}\ru{\:---} \en{the~external force}\ru{внешняя сила} \en{per mass unit}\ru{на единицу массы},
$\bm{F}$\en{ is}\ru{\:---} \en{the~resultant of external forces}\ru{результанта внешних сил} (\en{also called}\ru{также называемая} \en{the~}\inquotes{\en{equally acting force}\ru{равнодействующей силой}} \en{or}\ru{или} \en{the~}\inquotes{\en{main vector}\ru{главным вектором}}),
$\mathboldM$\en{ is}\ru{\:---} \en{the~resultant of external couples}\ru{результанта внешних пар сил} (\en{the~}\inquotes{\en{main couple}\ru{главная пара}}, \en{the~}\inquotes{\en{main moment}\ru{главный момент}}).

...

--- Are there any scenarios for which the center of mass is not almost exactly equivalent to the center of gravity?

--- \href{http://en.wikipedia.org/wiki/Centers_of_gravity_in_non-uniform_fields}{Non-uniform gravity field.} In a uniform gravitational field, the center of mass is equal to the center of gravity.

...

\subsection*{Work}

\begin{equation*}
W \hspace{-0.12ex} ( \hspace{-0.08ex} \bm{F} \hspace{-0.25ex}, \bm{u} ) \hspace{-0.1ex} = \bm{F} \hspace{-0.2ex} \dotp \bm{u}
\end{equation*}

as the~exact~(full) differential

\nopagebreak\[
d \hspace{.1ex} W \hspace{-0.33ex}  = \hspace{.08ex} \displaystyle \frac{\raisemath{-0.2em}{\partial \hspace{.1ex} W}}{\partial \hspace{-0.1ex} \bm{F}} \hspace{-0.1ex} \dotp d \bm{F} + \frac{\raisemath{-0.2em}{\partial \hspace{.1ex} W}}{\partial \bm{u}} \hspace{-0.1ex} \dotp d \bm{u}
\]

by \inquotes{product rule}

\nopagebreak\[
d \hspace{.1ex} W \hspace{-0.33ex}
= d \hspace{.15ex} \bigl( \hspace{-0.1ex} \bm{F} \hspace{-0.2ex} \dotp \bm{u} \bigr) \hspace{-0.25ex}
= d\bm{F} \hspace{-0.1ex} \dotp \bm{u} \hspace{.12ex} + \bm{F} \hspace{-0.1ex} \dotp d\bm{u}
\]

${\displaystyle \frac{\raisemath{-0.2em}{\partial \hspace{.1ex} W}}{\partial \hspace{-0.1ex} \bm{F}} \hspace{-0.1ex} = \bm{u}}$,
${\displaystyle \frac{\raisemath{-0.2em}{\partial \hspace{.1ex} W}}{\partial \bm{u}} \hspace{-0.1ex} = \bm{F}}$

...

\subsection*{Constraints}

Imposed on the positions and velocities of particles, there are restrictions of a geometrical or kinematical nature, called constraints.

Holonomic constraints are relations between position variables (and possibly time) which can be expressed as equality like
\begin{equation*}
f(q^{1} \hspace{-0.25ex}, q^{2} \hspace{-0.25ex}, q^{3} \hspace{-0.25ex}, \ldots, q^{n} \hspace{-0.25ex}, t) = 0
\hspace{.1ex} ,
\end{equation*}

\noindent
where ${q^{1} \hspace{-0.25ex}, q^{2} \hspace{-0.25ex}, q^{3} \hspace{-0.25ex}, \ldots, q^{n}}$ are $n$ parameters (coordinates) that fully describe the system.

A~constraint that cannot be expressed as such is nonholonomic.

Holonomic constraint depends only on coordinates and time.
It does not depend on velocities or any higher time derivatives.

Velocity-dependent constraints like
\[
f(q^{1} \hspace{-0.25ex}, q^{2} \hspace{-0.25ex}, \ldots, q^{n} \hspace{-0.25ex}, {\mathdotabove{q}}^{\hspace{.2ex}1} \hspace{-0.25ex}, {\mathdotabove{q}}^{\hspace{.2ex}2} \hspace{-0.25ex}, \ldots, {\mathdotabove{q}}^{\hspace{.2ex}n} \hspace{-0.25ex}, t) = 0
\]
are mostly not holonomic.

For example, the motion of a particle constrained to lie on a~sphere’s surface is subject to a~holonomic constraint, but if the particle is able to fall off a~sphere under the influence of gravity, the constraint becomes non-holonomic.
For the first case the holonomic constraint may be given by the equation: ${r^{2} - a^{2} = 0}$, where $r$ is the distance from the centre of a~sphere of radius~$a$.
Whereas the second non-holonomic case may be given by: ${r^{2} - a^{2} \geq 0}$.

Three examples of nonholonomic constraints are: when the constraint equations are nonintegrable, when the constraints have inequalities, or with complicated non-conservative forces like friction.

\[
\bm{r}_{i} \hspace{-0.1ex} = \bm{r}_{i}(q^{1} \hspace{-0.25ex}, q^{2} \hspace{-0.25ex}, \ldots, q^{n} \hspace{-0.25ex}, t)
\]
(assuming $n$ independent parameters/coordinates)

\en{\section{Principle of virtual work}}

\ru{\section{Принцип виртуальной работы}}

\label{para:virtualworkprinciple.genericmechanics}

\emph{Mécanique analytique} (1788--89) is a two volume French treatise on analytical mechanics, written by Joseph Louis Lagrange, and published 101 years following Isaac Newton’s \emph{Philosophiæ Naturalis Principia Mathematica}.

\bookauthor{Joseph Louis Lagrange}.
\href{https://play.google.com/books/reader?id=Q8MKAAAAYAAJ&pg=GBS.PP7}{Mécanique analytique. Nouvelle édition, revue et augmentée par l’auteur. Tome premier. Mme Ve Courcier, Paris, 1811.} \howmanypages{490 pages.}

\bookauthor{Joseph Louis Lagrange}.
\href{https://play.google.com/books/reader?id=TmMSAAAAIAAJ&pg=GBS.PP9}{Mécanique analytique. Troisième édition, revue, corrigée et annotée par M.\:J.\:Bertrand. Tome second. Mallet-Bachelier, Paris, 1855.} \howmanypages{416 pages.}

The historical transition from geometrical methods, as presented in Newton’s Principia, to methods of mathematical analysis.

{\small
\setlength{\abovedisplayskip}{2pt}\setlength{\belowdisplayskip}{2pt}

Consider the~exact differential of any set of location vectors~${\locationvector_{\hspace{-0.1ex}i}}$, that are functions of other variable parameters (coordinates) ${q^{1} \hspace{-0.25ex}, q^{2} \hspace{-0.25ex}, ..., q^{n}}$ and time~$t$.

The actual displacement is the differential
\[
\displaystyle d\locationvector_{\hspace{-0.1ex}i} = \frac{\partial \locationvector_{\hspace{-0.1ex}i}}{\partial t} \hspace{.16ex} dt \hspace{.2ex} + \sum_{j=1}^{n} {\frac{\partial \locationvector_{\hspace{-0.1ex}i}}{\partial q^{\hspace{.1ex}j}}} \hspace{.2ex} dq^{\hspace{.1ex}j}
\]

Now, imagine an arbitrary path through the configuration space/manifold. This means it has to satisfy the constraints of the system but not the actual applied forces
\[
\delta \locationvector_{\hspace{-0.1ex}i} = \displaystyle\sum_{j=1}^{n} {\displaystyle\frac {\partial \locationvector_{\hspace{-0.1ex}i}}{\partial q^{\hspace{.1ex}j}}} \hspace{.2ex} \delta q^{\hspace{.1ex}j}
\]

\par}

A~virtual infinitesimal displacement of a~system of particles refers to a~change in the configuration of a~system as the result of any arbitrary infinitesimal change of location vectors (or coordinates) ${\variation{\hspace{.1ex}\locationvector_k}}$, consistent with the forces and constraints imposed on the system at the current/given instant~$t$.
This displacement is called \inquotes{virtual} to distinguish it from an~actual displacement of the system occurring in a~time interval~${dt}$, during which the forces and constraints may be changing.

Assume the system is in equilibrium, that is the full force on each particle vanishes, ${\bm{F}_i \hspace{-0.2ex} = \hspace{-0.1ex} \bm{0} \hspace{.4em} \forall i}$.
Then clearly the term ${\bm{F}_i \hspace{-0.1ex} \dotp \variation{\hspace{.1ex}\locationvector_i}}$, which is the virtual work of force ${\bm{F}_i}$ in displacement ${\variation{\hspace{.1ex}\locationvector_i}}$, also vanishes for each particle, ${\bm{F}_i \hspace{-0.1ex} \dotp \variation{\hspace{.1ex}\locationvector_i} \hspace{-0.1ex} = 0 \hspace{.4em} \forall i}$.
The~sum of these vanishing products over all particles is likewise equal to zero:
\[
\displaystyle\sum_{\smash{i}} \bm{F}_i \hspace{-0.1ex} \dotp \variation{\hspace{.1ex}\locationvector_i} \hspace{-0.1ex} = 0
\hspace{.15ex} .
\]

Decompose the full force ${\bm{F}_i}$ into the applied (active) force ${\bm{F}_{i}^{\smthactive}}$ and the force of constraint ${\mathboldPhi_{i}}$,
\[
\bm{F}_i \hspace{-0.1ex} = \hspace{-0.1ex} \bm{F}_{i}^{\smthactive} \hspace{-0.2ex} + \hspace{-0.1ex} \mathboldPhi_{i}
\]

We now restrict ourselves to systems for which the net virtual work of the force of every constraint is zero:
\[
\displaystyle\sum_{\smash{i}} \mathboldPhi_i \hspace{-0.1ex} \dotp \variation{\hspace{.1ex}\locationvector_i} \hspace{-0.1ex} = 0
\hspace{.15ex} .
\]

We therefore have as the condition for equilibrium of a system that the virtual work of all applied forces vanishes:
\[
\displaystyle\sum_{\smash{i}} \bm{F}_{i}^{\smthactive} \hspace{-0.25ex} \dotp \variation{\hspace{.1ex}\locationvector_i} \hspace{-0.1ex} = 0
\hspace{.15ex} .
\]
--- the principle of virtual work.

Note that coefficients ${\bm{F}_{i}^{\smthactive}}$ can no longer be thought equal to zero: in common ${\bm{F}_{i}^{\smthactive} \hspace{-0.3ex} \neq \hspace{-0.1ex} 0}$, since ${\variation{\hspace{.1ex}\locationvector_i}}$ are not independent but are bound by constraints.

\en{A~virtual displacement}\ru{Виртуальным смещением}
\en{of~a~particle}\ru{частицы}
\en{with vector radius}\ru{с~вектором\hbox{-}радиусом}~${\locationvector_k}$
\en{is}\ru{это}
\en{variation}\ru{вариация}
${\variation{\hspace{.1ex}\locationvector_k}}$\:---
\en{any}\ru{любое}
\en{infinitesimal}\ru{бесконечно малое}
\en{change}\ru{изменение}
\en{of~vector}\ru{вектора}~${\locationvector_k}$,
\en{which is}\ru{которое}
\en{compatible}\ru{совместимо}
\en{with the }\ru{со связями (}constraints\ru{)}.
\en{If}\ru{Если}
\en{the system is free}\ru{система свободна},
\en{that is}\ru{то есть}
\en{there are no constraints}\ru{связей нет},
\en{then}\ru{тогда}
\en{virtual displacements}\ru{виртуальные смещения}
${\variation{\hspace{.1ex}\locationvector_k}}$
\en{are perfectly random}\ru{совершенно случайны}.

\begin{otherlanguage}{russian}

Связи бывают
голономные
(holonomic, или геометрические),
связывающие только положения~(смещения)\:---
\en{they are functions}\ru{это функции}
\en{of only}\ru{лишь}
\en{the coordinates}\ru{координат}
\en{and}\ru{и}\ru{,}
\en{probably}\ru{возможно}\ru{,}
\en{time}\ru{времени}

\nopagebreak\vspace{-0.1em}
\begin{equation}\label{holonomicconstraint}
c\hspace{.2ex}(\locationvector, t) = 0
\end{equation}

\vspace{-0.1em}\noindent
--- и~неголономные~(или дифференциальные),
содержащие производные координат по~времени:
${c\hspace{.2ex}(\locationvector, \mathdotabove{\locationvector}, t) = 0}$
\en{and not}\ru{и~не} интегрируемые
\en{till}\ru{до}
\en{the geometrical constraints}\ru{геометрических связей}.

\en{When}\ru{Когда}
\en{all}\ru{все}
\en{constraints}\ru{связи}
\en{are holonomic}\ru{голономные},
\en{then}\ru{тогда}
\en{the virtual displacements}\ru{виртуальные смещения}
\en{of~a~particle}\ru{частицы}
\inquotes{${\hspace{.05ex}k\hspace{.25ex}}$}
\en{satisfy}\ru{удовлетворяют}
\en{the equation}\ru{уравнению}

\nopagebreak\vspace{-0.1em}\begin{equation}\label{requirementforvirtualdisplacements}
\displaystyle \sum_{\smash{j=1}}^{m}
\scalebox{.92}{$ \displaystyle
   \frac{\raisemath{-0.12em}{\partial \hspace{.1ex} c_{j}}}{\partial \hspace{.1ex} \locationvector_k} $}
\hspace{-0.1ex} \dotp
\variation{\hspace{.1ex}\locationvector_k}
\hspace{-0.1ex} =
0
\hspace{.1ex} .
\vspace{-0.25em}\end{equation}

В~несвободных системах все силы делятся на две группы: активные и~реакции связей.
Реакция~$\mathboldPhi_k$ действует со~стороны всех материальных ограничителей на~частицу \inquotes{${\hspace{.05ex}k\hspace{.25ex}}$} и~меняется согласно уравнению~\eqref{holonomicconstraint} для каждой связи.
Связи предполагаются идеальными:

\nopagebreak\begin{equation}\label{idealconstraints}
\scalebox{0.9}{$\displaystyle \sum_{\smash{k}}$} \hspace{.2ex} \mathboldPhi_k \hspace{-0.2ex} \dotp \variation{\hspace{.1ex}\locationvector_k} \hspace{-0.16ex} = 0
\quad \textrm{---}
\vspace{-0.25em}\end{equation}
\noindent
работа реакций на~любых виртуальных смещениях равна нулю.

Принцип виртуальной работы is

\nopagebreak\vspace{-0.1em}\begin{equation}\label{discrete:principleofvirtualwork}
\displaystyle \sum_{\smash{k}} \hspace{-0.1ex} \Bigl( \hspace{-0.1ex} \bm{F}_{k}^{\smthactive} \hspace{-0.2ex} - m_k \mathdotdotabove{\locationvector}_k \Bigr) \hspace{-0.3ex} \dotp \variation{\hspace{.1ex}\locationvector_k} \hspace{-0.1ex} = 0
\hspace{.1ex} ,
\vspace{-0.3em}
\end{equation}

\noindent
где~${\bm{F}_{k}^{\smthactive}}$\:--- лишь активные силы, без реакций связей.

Дифференциальное вариационное уравнение~\eqref{discrete:principleofvirtualwork} может показаться тривиальным следствием закона Ньютона~\eqref{law:ofnewton} и~условия идеальности связей~\eqref{idealconstraints}.
Однако содержание~\eqref{discrete:principleofvirtualwork} несравненно обширнее.
Читатель вскоре увидит, что принцип~\eqref{discrete:principleofvirtualwork} может быть положен в~основу механики~\cite{gantmacher-analyticalmechanics}.
\en{The various models}\ru{Разные модели}
\en{of elastic bodies}\ru{упругих тел}\ru{,}
\en{that}\ru{что}
\en{I}\ru{я}
\en{describe}\ru{описываю}
\en{in this book}\ru{в~этой книге}\ru{,}
\en{are based}\ru{основаны}
\en{on this}\ru{на~этом}
\en{principle}\ru{принципе}.

Для~примера рассмотрим совершенно жёсткое~(недеформируемое) твёрдое тело.

.... \eqref{completelyrigidbody.locationvectorofanypointdecomposed} ${\Rightarrow}$
${\variation{\locationvector} = \variation{\positionofthepole} + \variation{\bm{x}}}$

(begin copied from \chapterdotsectionref{chapter:mathapparatus}{para:calculusofvariations})

Варьируя тождество~\eqref{orthogonalityofrotationtensor}, получим ${\variation{\rotationtensor} \hspace{-0.2ex} \dotp \rotationtensor^{\T} \hspace{-0.2ex} = - \hspace{.2ex} \rotationtensor \dotp \variation{\rotationtensor}^{\T}\!}$.
Этот тензор антисимметричен, и~потому выражается через свой сопутствующий вектор~${\varvector{o}}$ как~${\variation{\rotationtensor} \hspace{-0.1ex} \dotp \rotationtensor^{\T} \hspace{-0.3ex} = \varvector{o} \hspace{-0.2ex} \times \hspace{-0.2ex} \UnitDyad}$.
Приходим к~соотношениям

\nopagebreak\vspace{-0.5em}\begin{equation}
\variation{\rotationtensor} \hspace{-0.1ex} = \varvector{o} \hspace{-0.1ex} \times \hspace{-0.1ex} \rotationtensor , \:\:
\varvector{o} = - \hspace{.2ex} \scalebox{.93}{$ \displaystyle\onehalf $} \hspace{-0.1ex} \Bigl( \hspace{-0.1ex} \variation{\rotationtensor} \hspace{-0.1ex} \dotp \rotationtensor^{\T} \Bigr)_{\hspace{-0.25em}\Xcompanion}
\hspace{-0.1ex} ,
\end{equation}

(end of copied from \chapterdotsectionref{chapter:mathapparatus}{para:calculusofvariations})

...


Проявилась замечательная особенность~\eqref{discrete:principleofvirtualwork}: это уравнение эквивалентно системе такого порядка, каково число степеней свободы системы, то~есть сколько независимых вариаций~${\variation{\hspace{.1ex}\locationvector_k}}$ мы имеем.
Если в~системе $N$~точек есть $m$ связей, то число степеней свободы ${n = 3N \hspace{-0.25ex} - m}$.

...


\en{\section{Balance of momentum, rotational momentum, and~energy}}

\ru{\section{Баланс импульса, момента импульса и~энергии}}

Эти уравнения баланса
могут быть связаны
со~свойствами
пространства
\en{and}\ru{и}~времени~\cite{landau.lifshitz-shortcourse}.
Сохранение импульса
(количества движения)
в~з\'{а}мкнутой~(изолированной)%
\footnote{\emph{З\'{а}мкнутая~(изолированная) система}
это такая система частиц,
которые
взаимодействуют
\en{only}\ru{только}
друг с~другом,
\en{but}\ru{но}
ни с~какими другими телами.}\hspace{-0.25ex}
системе выводится из~однородности пространства \emph{(при любом параллельном переносе\:--- трансляции\:--- замкнутой системы как целого свойства этой системы не~меняются)}.
Сохранение момента импульса\:--- следствие изотропии пространства \emph{(свойства замкнутой системы не~меняются при любом повороте этой системы как целого)}.
Энергия~же
изолированной системы
сохраняется,
так~как
время однородно%
%
\footnote{%
Характеристики
\inquotes{однородность}
\en{and}\ru{и}~\inquotes{изотропность}
пространства,
\inquotes{однородность}
времени
не~фигурируют среди аксиом
классической механики.
}\hspace{-0.25ex} % end of footnote
(энергия
${\mathrm{E} \hspace{.1ex} \equiv \kinetic \hspace{.1ex} (q, \mathdotabove{q} \hspace{.2ex}) \hspace{-0.2ex} + \hspace{-0.1ex} \potential(q)}$
такой системы
не~зависит явно от~времени).

Уравнения баланса могут быть выведены из принципа виртуальной работы~\eqref{discrete:principleofvirtualwork}.
Перепишем его в~виде

\nopagebreak\vspace{-0.2em}\begin{equation}\label{discrete:principleofvirtualwork.externalinternal}
\displaystyle \sum_{\smash{k}} \hspace{-0.2ex} \Bigl( \hspace{-0.25ex} \bm{F}^{\smthexternal}_{\hspace{-0.16ex}k} \hspace{-0.1ex} - m_k \mathdotdotabove{\locationvector}_k \Bigr) \hspace{-0.32ex} \dotp \variation{\hspace{.1ex}\locationvector_k} \hspace{-0.1ex}
+ \variation{\internalwork} \hspace{-0.1ex} = 0 \hspace{.1ex},
\vspace{-0.25em}\end{equation}

\vspace{-0.1em}\noindent
где выделены внешние силы~${\bm{F}^{\smthexternal}_{\hspace{-0.16ex}k}}$ и~виртуальная работа внутренних сил
${\variation{\internalwork} \hspace{-0.12ex} = \scalebox{0.8}[0.84]{$\displaystyle \underset{\raisemath{.25ex}{\smash{k}}}{\sum}$} \scalebox{0.8}[0.84]{$\displaystyle \underset{\raisemath{.25ex}{\smash{j}}}{\sum}$} \hspace{.12ex} \bm{F}^{\smthinternal}_{\hspace{-0.16ex}kj} \hspace{-0.1ex} \dotp \variation{\hspace{.1ex}\locationvector_k} \hspace{.1ex}}$.

\vspace{-0.1em}
Предполагается, что внутренние силы не~совершают работы на~виртуальных смещениях тела как жёсткого целого (${\constvarvector{\hspace{-0.1ex}\bm{\rho}}}$ и~${\constvarvector{\bm{o}}}$\:--- произвольные постоянные векторы, определяющие трансляцию и~поворот)

\nopagebreak\vspace{-0.2em}\begin{equation}\label{assumptionforvirtualwork}
\begin{array}{l}
\variation{\hspace{.1ex}\locationvector_k} \hspace{-0.16ex}
= \constvarvector{\hspace{-0.1ex}\bm{\rho}} \hspace{.2ex} + \hspace{.12ex} \constvarvector{\bm{o}} \hspace{-0.2ex} \times \hspace{-0.1ex} \locationvector_k
\hspace{.1ex} ,
\\
\constvarvector{\hspace{-0.1ex}\bm{\rho}} = \boldconstant \hspace{.1ex} , \:
\constvarvector{\bm{o}} = \boldconstant
\end{array}
\hspace{.3ex} \Rightarrow \hspace{.6ex}
\variation{\internalwork} \hspace{-0.1ex} = 0 \hspace{.1ex}.
\end{equation}

\vspace{-0.1em}
Предпосылки-соображения для~этого предположения таковы.

Первое\:--- для случая упругих (потенциальных) внутренних сил.
Тогда ${\variation{\internalwork} = - \hspace{.16ex} \variation{\potential}}$\:--- вариация потенциала с~противоположным знаком.
Достаточно очевидно, что только лишь деформирование меняет~$\potential$.

Второе соображение\:--- в~том, что суммарный вектор и~суммарный момент внутренних сил равен нулю

\begin{equation*}
\sum \ldots
\end{equation*}

...

Принимая~\eqref{assumptionforvirtualwork} и~подставляя в~\eqref{discrete:principleofvirtualwork.externalinternal} сначала ${\variation{\hspace{.1ex}\locationvector_k} \hspace{-0.16ex} = \hspace{-0.08ex} \constvarvector{\hspace{-0.1ex}\bm{\rho}}}$ (трансляция), а~затем ${\variation{\hspace{.1ex}\locationvector_k} \hspace{-0.16ex} = \hspace{-0.08ex} \constvarvector{\bm{o}} \hspace{-0.2ex} \times \hspace{-0.1ex} \locationvector_k}$ (поворот), получаем баланс импульса~(...) и баланс момента импульса~(...).

...



\end{otherlanguage}

\en{\section{Hamilton’s principle and Lagrange’s equations}}

\ru{\section{Принцип Гамильтона и уравнения Лагранжа}}

{\small
\setlength{\abovedisplayskip}{2pt}\setlength{\belowdisplayskip}{2pt}
Two branches of analytical mechanics are Lagrangian mechanics (using generalized coordinates and corresponding generalized velocities in configuration space) and Hamiltonian mechanics (using coordinates and corresponding momenta in phase space). Both formulations are equivalent by a Legendre transformation on the generalized coordinates, velocities and momenta, therefore both contain the same information for describing the dynamics of a system.

\par}

\begin{otherlanguage}{russian}

Вариационное уравнение~\eqref{discrete:principleofvirtualwork} удовлетворяется в~любой момент времени.
Проинтегрируем его\footnote{%
${ \mathcolor{magenta}{\variation{\kinetic} = \scalebox{0.92}[0.95]{$ \displaystyle \sum_{\smash{k}} $} \hspace{.2ex} m_k \mathdotabove{\locationvector}_k \hspace{-0.16ex} \dotp \variation{\hspace{.1ex}\mathdotabove{\locationvector}_k}} \hspace{-0.5ex} }$ , \hspace{-0.1ex}
${%
\left(
\scalebox{0.92}[0.95]{$ \displaystyle \sum_{\smash{k}} $} \hspace{.2ex} m_k \mathdotabove{\locationvector}_k \hspace{-0.16ex} \dotp \variation{\hspace{.1ex}\locationvector_k} \hspace{-0.4ex}
\right)^{\hspace{-0.25em}\tikz[baseline=-0.2ex]\draw[black, fill=black] (0,0) circle (.28ex);} \hspace{-0.1ex}
= \hspace{.2ex}
\scalebox{0.92}[0.95]{$ \displaystyle \sum_{\smash{k}} $} \hspace{.2ex} m_k \mathdotdotabove{\locationvector}_k \hspace{-0.16ex} \dotp \variation{\hspace{.1ex}\locationvector_k}
+ \hspace{-0.2ex} \tikzmark{beginVariationOfKinetic} \hspace{.32ex} \scalebox{0.92}[0.95]{$ \displaystyle \sum_{\smash{k}} $} \hspace{.2ex} m_k \mathdotabove{\locationvector}_k \hspace{-0.16ex} \dotp \variation{\hspace{.1ex}\mathdotabove{\locationvector}_k} \tikzmark{endVariationOfKinetic}
}$
\\[.5em]
${%
- \hspace{.1ex} \scalebox{0.95}[0.96]{$ \displaystyle \integral\displaylimits_{\mathclap{t_1}}^{\raisemath{.12em}{\mathclap{t_2}}} $} \scalebox{0.92}[0.95]{$ \displaystyle \sum_{\smash{k}} $} \hspace{.2ex} m_k \mathdotdotabove{\locationvector}_k \hspace{-0.16ex} \dotp \variation{\hspace{.1ex}\locationvector_k} \hspace{.25ex} dt \hspace{.4ex}
= \hspace{-0.1ex} \scalebox{0.95}[0.96]{$ \displaystyle \integral\displaylimits_{\mathclap{t_1}}^{\raisemath{.12em}{\mathclap{t_2}}} $} \hspace{-0.1ex} \variation{\kinetic} \hspace{.1ex} dt \hspace{.16ex}
- \hspace{-0.1ex} \left[ \hspace{.2ex}
\scalebox{0.92}[0.95]{$ \displaystyle \sum_{\smash{k}} $} \hspace{.2ex} m_k \mathdotabove{\locationvector}_k \hspace{-0.16ex} \dotp \variation{\hspace{.1ex}\locationvector_k} \hspace{.16ex}
\right]_{\hspace{-0.25ex}t_1}^{\hspace{-0.25ex}t_2}
}$}%
\AddUnderBrace[line width=.75pt][-0.1ex,-0.77em]{beginVariationOfKinetic}{endVariationOfKinetic}{${\scriptstyle \variation{\kinetic}}$}
%
по~какому\hbox{-}либо промежутку ${\left[\hspace{.15ex} t_1, t_2 \hspace{.15ex}\right]}$

\nopagebreak\vspace{-0.25em}\begin{equation}
\displaystyle \integral\displaylimits_{t_1}^{\raisemath{.12em}{t_2}}
\hspace{-0.8ex}
\left( \hspace{-0.25ex} \variation{\kinetic}
+ \scalebox{0.95}[0.98]{$ \displaystyle \sum_{\smash{k}} $} \hspace{.16ex} \bm{F}_k \hspace{-0.1ex} \dotp \variation{\hspace{.1ex}\locationvector_k} \hspace{-0.33ex} \right) \hspace{-0.5ex} dt \hspace{.1ex}
- \hspace{-0.2ex} \left[ \hspace{.2ex} \scalebox{0.95}[0.98]{$ \displaystyle \sum_{\smash{k}} $} \hspace{.2ex} m_k \mathdotabove{\locationvector}_k \hspace{-0.1ex} \dotp \variation{\hspace{.1ex}\locationvector_k} \hspace{.16ex} \right]_{\hspace{-0.32ex}t_1}^{\hspace{-0.32ex}t_2}
\hspace{-0.8ex} = 0
\hspace{.1ex} .
\end{equation}

\vspace{-0.16em}\noindent
Без ущерба для общности можно принять ${\variation{\hspace{.1ex}\locationvector_k}\hspace{.1ex}(t_1) \hspace{-0.1ex} = \variation{\hspace{.1ex}\locationvector_k}\hspace{.1ex}(t_2) \hspace{-0.1ex} = \bm{0}}$, тогда внеинтегральный член исчезает.

Вводятся обобщённые координаты~$q^i$~(${i = 1, \ldots, n}$\:--- число степеней свободы).
Векторы-радиусы становятся функциями \hbox{вида} ${\locationvector_{k}(q^i, t)}$, тождественно удовлетворяющими уравнениям связей~\eqref{holonomicconstraint}.
Если связи стационарны, то~есть~\eqref{holonomicconstraint} не~содержат $t$, то остаётся~${\locationvector_{k}(q^i)}$.
Кинетическая энергия превращается в~функцию ${\kinetic \hspace{.1ex} (q^i, \mathdotabove{q}^{\hspace{.2ex}i}, t)}$, где явно входящее~$t$ характерно лишь для нестационарных связей.

Весьма существенно понятие обобщённых сил~${Q_{\hspace{-0.1ex}i}}$.
Они вводятся через выражение виртуальной работы

\nopagebreak\vspace{-0.25em}\begin{equation}
\scalebox{0.92}[0.96]{$ \displaystyle \sum_{\smash{k}} $} \hspace{.16ex}
\bm{F}_k \hspace{-0.1ex} \dotp \variation{\hspace{.1ex}\locationvector_k} \hspace{-0.1ex}
= \hspace{-0.1ex} \scalebox{0.92}[0.96]{$\displaystyle \sum_{i}$} \hspace{.16ex} Q_{\hspace{-0.1ex}i} \hspace{.12ex} \variation{q^i} \hspace{.1ex} ,
\:\:
Q_{\hspace{-0.1ex}i} \equiv
\scalebox{0.92}[0.96]{$ \displaystyle \sum_{\smash{k}} $} \hspace{.16ex}
\bm{F}_k \hspace{-0.1ex} \dotp \scalebox{0.96}{$ \displaystyle \frac{\raisemath{-0.12em}{\partial \hspace{.1ex} \locationvector_k}}{\raisemath{-0.1em}{\partial q^i}} $} \hspace{.16ex} .
\vspace{-0.1em}\end{equation}

\vspace{-0.15em}\noindent
Ст\'{о}ит акцентировать происхождение обобщённых сил через работу.
Установив набор обобщённых координат системы, следует сгруппировать приложенные силы~${\bm{F}_k}$ в~комплексы~${Q_{\hspace{-0.1ex}i}}$.

Если силы потенциальны с~энергией~${\potential \narroweq \potential(q^i \hspace{-0.2ex} , t)}$, то

\nopagebreak\vspace{-0.2em}\begin{equation}
\scalebox{0.92}[0.96]{$\displaystyle \sum_{i}$} \hspace{.16ex} Q_{\hspace{-0.1ex}i} \hspace{.12ex} \variation{q^i} \hspace{-0.1ex}
= - \hspace{.16ex} \variation{\potential} \hspace{.1ex} ,
\:\:
Q_{\hspace{-0.1ex}i} = - \hspace{.16ex} \scalebox{0.96}{$ \displaystyle \frac{\raisemath{-0.12em}{\partial \hspace{.1ex} \potential}}{\raisemath{-0.1em}{\partial q^i}} $} \hspace{.16ex} .
\vspace{-0.1em}\end{equation}

\vspace{-0.15em}\noindent
Явное присутствие $t$ может быть при нестационарности связей или зависимости физических полей от~времени.

...

Известны уравнения Lagrange’а не~только второго, но~и~первого рода.
Рассмотрим их ради методики вывода, много раз применяемой в~этой книге.

При~наличии связей~\eqref{holonomicconstraint} равенство ${\bm{F}_k \hspace{-0.12ex} = m_k \mathdotdotabove{\locationvector}_k}$ не~следует из~вариационного уравнения~\eqref{discrete:principleofvirtualwork}, ведь тогда виртуальные смещения~${\variation{\hspace{.1ex}\locationvector_k}}$ не независимы.
Каждое из $m$ ($m$\:--- число связей) условий для~вариаций~\eqref{requirementforvirtualdisplacements} умножим на~некий скаляр~$\lambda_{\alpha}$ (${\alpha = 1, \ldots, m}$) и~добавим к~\eqref{discrete:principleofvirtualwork}:

\nopagebreak\vspace{-0.2em}\begin{equation}
\displaystyle \sum_{k=1}^{N} \hspace{-0.2ex} \left(^{\mathstrut} \hspace{-0.16ex} \bm{F}_k \right. \hspace{-0.32ex} + \hspace{-0.05ex}
\scalebox{0.88}[0.92]{$ \displaystyle \sum_{\alpha=1}^{m} $} \hspace{.16ex} \lambda_{\alpha} \hspace{.2ex} \scalebox{0.92}{$\displaystyle \frac{\raisemath{-0.12em}{\partial \hspace{.1ex} c_{\alpha}}}{\partial \hspace{.1ex} \locationvector_k}$}
- \left. \hspace{-0.25ex} m_k \mathdotdotabove{\locationvector}_k ^{\mathstrut}\right) \hspace{-0.32ex} \dotp \variation{\hspace{.1ex}\locationvector_k} \hspace{-0.16ex} = 0
\hspace{.1ex} .
\end{equation}

\vspace{-0.1em} \noindent Среди $3N$ компонент вариаций~${\variation{\hspace{.1ex}\locationvector_k}}$ зависимых $m$.
Но столько~же и~множителей Лагранжа: подберём $\lambda_{\alpha}$ так, чтобы коэффициенты\textcolor{red}{(??как\'{и}е?)} при~зависимых вариациях обратились в~нуль.
Но при~остальных вариациях коэффициенты\textcolor{red}{(??)} также должны быть нулями из\hbox{-}за независимости.
Следовательно, все выражения в~скобках~${(\cdots\hspace{-0.2ex})}$ равны нулю\:--- это и~есть уравнения Lagrange’а первого рода.

Поскольку для каждой частицы

...



\end{otherlanguage}

\en{\section{Statics}}

\ru{\section{Статика}}

\label{para:statics}

\begin{otherlanguage}{russian}

Рассмотрим систему со~стационарными (постоянными во~времени) связями при статических (не~меняющихся со~временем) активных силах ${\bm{F}_k}$.
В~равновесии ${\locationvector_k \hspace{-0.12ex} = \boldconstant}$, и формулировка принципа виртуальной работы следующая:

\nopagebreak\vspace{-0.1em}\begin{equation}\label{statics.discrete:principleofvirtualwork}
\scalebox{0.92}[0.96]{$ \displaystyle \sum_{\smash{k}} $} \hspace{.25ex}
\bm{F}_k \hspace{-0.1ex} \dotp \variation{\hspace{.1ex}\locationvector_k} \hspace{-0.1ex} = 0
\:\,\Leftrightarrow\:
\scalebox{0.92}[0.96]{$ \displaystyle \sum_{\smash{k}} $} \hspace{.25ex}
\bm{F}_k \hspace{-0.1ex} \dotp \scalebox{0.96}{$ \displaystyle \frac{\raisemath{-0.12em}{\partial \hspace{.1ex} \locationvector_k}}{\raisemath{-0.1em}{\partial q^i}} $}
= Q_{\hspace{-0.1ex}i} \hspace{-0.1ex} = 0 \hspace{.1ex} .
\vspace{-0.1em}\end{equation}

\vspace{-0.15em}\noindent
Существенны обе стороны этого положения: и вариационное уравнение, и равенство нулю обобщённых сил.

Соотношения~\eqref{statics.discrete:principleofvirtualwork}\:--- это самые общие уравнения статики.
В~литературе распространено узкое представление об~уравнениях равновесия как балансе сил и~моментов.
Но при~этом нужно понимать, что набор уравнений равновесия точно соответствует обобщённым координатам.
\en{The~resultant force}\ru{Результирующая сила}~(также называемая \inquotes{равнодействующей силой} или \inquotes{главным вектором})
\en{and}\ru{и}~\en{the~resultant couple}\ru{результирующая пара сил}~(\inquotes{главный момент})
в~уравнениях равновесия фигурируют\footnote{\en{Since}\ru{Со~времён} \textcolor{magenta}{\en{describing a~composition}\ru{описания приведения}} \en{of~any system of forces}\ru{любой системы сил}, \en{acting on the~same absolutely rigid body}\ru{действующей на~одно и~то~же совершенно жёсткое тело}, \en{\textcolor{magenta}{into} a~single force}\ru{к~одной силе} \en{and}\ru{и} \en{a~single couple~(about a~chosen point)}\ru{одной паре~(вокруг выбранной точки)} \en{in the~book}\ru{в~книге}
\href{https://gallica.bnf.fr/ark:/12148/bpt6k6213152z.texteImage}{\inquotes{Éléments de~statique}}
\en{by }\href{https://en.wikipedia.org/wiki/Louis_Poinsot}{Louis\ru{’а} Poinsot}.}\hbox{\hspace{-0.5ex},}
поскольку у~системы есть степени свободы трансляции и~поворота.
Огромная популярность сил и~моментов связана не~столько с~известностью статики совершенно недеформируемого твёрдого тела, но с~тем, что виртуальная работа внутренних сил на~смещениях системы как жёсткого целого равна нулю в~любой среде.

Пусть в~системе действуют два вида сил: потенциальные с~энергией от~обобщённых координат ${\potential(q^i)}$ и~дополнительные внешние~${\mathcircabove{Q}_{\hspace{-0.1ex}i}}$.
Из~\eqref{statics.discrete:principleofvirtualwork} следуют уравнения равновесия

\nopagebreak\vspace{-0.1em}\begin{equation}\label{staticequilibriumwithpotentialenergy}
\scalebox{0.96}{$ \displaystyle \frac{\raisemath{-0.15em}{\partial \hspace{.1ex} \potential}}{\raisemath{-0.1em}{\partial q^i}} $} = \mathcircabove{Q}_{\hspace{-0.1ex}i}
\hspace{.1ex},
\end{equation}
\nopagebreak\vspace{.1em}\begin{equation*}
d\potential = \scalebox{0.95}[1]{$\displaystyle \sum_{\smash{i}}$} \hspace{.32ex}
\scalebox{0.96}{$ \displaystyle \frac{\raisemath{-0.15em}{\partial \hspace{.1ex} \potential}}{\raisemath{-0.1em}{\partial q^i}} $} \hspace{.2ex} dq^i
= \scalebox{0.95}[1]{$\displaystyle \sum_{\smash{i}}$} \hspace{.2ex} \mathcircabove{Q}_{\hspace{-0.1ex}i} \hspace{.2ex} dq^i
\hspace{.1ex} .
\end{equation*}

\vspace{-0.5em}\noindent
Здесь содержится нелинейная в~общем случае задача статики о~связи положения равновесия~$q^i$ с~нагрузками~${\mathcircabove{Q}_{\hspace{-0.1ex}i}}$.

В~линейной системе с~квадратичным потенциалом вида ${\potential = \smalldisplaystyleonehalf \hspace{.2ex} C_{ik} \hspace{.15ex} q^{k} \hspace{.1ex} q^{i}}$

\nopagebreak\vspace{-1.25em}\begin{equation}\label{staticsoflineardiscretesystem}
\scalebox{0.95}[1]{$\displaystyle \sum_{\smash{k}}$} \hspace{.2ex} C_{ik} \hspace{.12ex} q^k \hspace{-0.1ex}
= \hspace{.1ex} \mathcircabove{Q}_{\hspace{-0.1ex}i} \hspace{.1ex} .
\vspace{-0.1em}\end{equation}

\vspace{-0.2em}\noindent
\en{Here figure}\ru{Тут фигурируют}
\en{elements}\ru{элементы}~$C_{ik}$
\en{of }\inquotes{\en{the stiffness matrix}\ru{матрицы жёсткости}},
\en{coordinates}\ru{координаты}~${q^k}$
\en{and}\ru{и}~\en{loads}\ru{нагрузки}~${\mathcircabove{Q}_{\hspace{-0.1ex}i}}$.

Сказанное возможно обобщить и~на~континуальные линейные упругие среды.

Матрица жёсткости~${C_{ik}}$ обычно бывает положительной (таков\'{о} свойство конструкций и~в~природе, и~в~технике).
Тогда ${\operatorname{det} \hspace{.16ex} C_{ik} > 0}$, линейная алгебраическая система~\eqref{staticsoflineardiscretesystem} однозначно разрешима, а~решение её можно заменить минимизацией квадратичной формы

\nopagebreak\vspace{-0.1em}\begin{equation}\label{discrete:potentialenergyofsystem}
\potentialenergyfunctional \hspace{.2ex} (q^{j}) \hspace{.1ex}
\equiv \hspace{.1ex}
\potential - \scalebox{0.95}[1]{$\displaystyle \sum_{\smash{i}}$} \hspace{.2ex}
\mathcircabove{Q}_{\hspace{-0.1ex}i} \hspace{.12ex} q^{i}
%
= \hspace{.1ex}
\smalldisplaystyleonehalf \hspace{.32ex}
\scalebox{0.95}[1]{$\displaystyle \sum_{\smash{i,k}}$} \hspace{.2ex}
%%\scalebox{0.95}[1]{$\displaystyle \sum_{\smash{k}}$} \hspace{.2ex}
\hspace{.1ex} q^{i} \hspace{.1ex} C_{ik} \hspace{.1ex} q^{k}
- \scalebox{0.95}[1]{$\displaystyle \sum_{\smash{i}}$} \hspace{.2ex}
\mathcircabove{Q}_{\hspace{-0.1ex}i} \hspace{.12ex} q^{i}
%
\hspace{.1ex}\to\hspace{.25ex} \mathrm{min}
\hspace{.16ex} .
\vspace{-0.1em}\end{equation}

\vspace{-0.1em}
Бывает однако, что конструкция неудачно спроектирована, тогда матрица жёсткости сингулярна~(необратима) %%(noninvertible)
и~${\operatorname{det} \hspace{.16ex} C_{ik} = \hspace{.1ex} 0}$ (или~же весьма близок к~нулю\:--- nearly singular матрица с~${\operatorname{det} \hspace{.16ex} C_{ik} \approx \hspace{.1ex} 0}$).
Тогда решение линейной проблемы статики~\eqref{staticsoflineardiscretesystem} существует лишь при ортогональности столбца нагрузок~${\mathcircabove{Q}_{\hspace{-0.1ex}i}}$ всем линейно независимым решениям однородной сопряжённой системы

...

Известные теоремы статики линейно \textcolor{magenta}{упругих} систем легко доказываются в~случае конечного числа степеней свободы. Теорема \href{https://en.wikipedia.org/wiki/Beno%C3%AEt_Paul_%C3%89mile_Clapeyron}{Clapeyron’а} выражается равенством

...

\en{Reciprocal work theorem}\ru{Теорема о~взаимности работ} (\inquotes{работа~${W_{\hspace{-0.1ex}12}}$ сил первого варианта на~смещениях от сил второго равна работе~${W_{\hspace{-0.15ex}21}}$ сил второго варианта на~смещениях от сил первого}) мгновенно выводится из~\eqref{staticsoflineardiscretesystem}:

(...)

\noindent Тут существенна симметрия матрицы жёсткости~$C_{i\hspace{-0.1ex}j}$, то~есть консервативность системы.

...

Но вернёмся к~проблеме~\eqref{staticequilibriumwithpotentialenergy}, иногда называемой теоремой Lagrange’а.
Её можно обратить преобразованием Лежандра Legendre (involution) transform(ation):

\nopagebreak\vspace{-0.2em}\begin{equation*}
\begin{array}{c}\small
d \hspace{-0.1ex} \left( \hspace{-0.1ex} \scalebox{0.95}[1]{$\displaystyle \sum_{\smash{i}}$} \hspace{.2ex} \mathcircabove{Q}_{\hspace{-0.1ex}i} \hspace{.2ex} q^i \hspace{-0.12ex} \right) \hspace{-0.5ex}
= \scalebox{0.95}[1]{$\displaystyle \sum_{\smash{i}}$} \hspace{.25ex} d \hspace{-0.25ex} \left( \hspace{-0.1ex} \mathcircabove{Q}_{\hspace{-0.1ex}i} \hspace{.2ex} q^i \right) \hspace{-0.3ex}
= \scalebox{0.95}[1]{$\displaystyle \sum_{\smash{i}}$} \hspace{-0.16ex} \left(
q^i \hspace{.2ex} d \mathcircabove{Q}_{\hspace{-0.1ex}i}
+ \mathcircabove{Q}_{\hspace{-0.1ex}i} \hspace{.2ex} dq^i \right)
\hspace{-0.64ex} ,
\\[1.25em]
%
\small
d \hspace{-0.1ex} \left( \hspace{-0.1ex} \scalebox{0.95}[1]{$\displaystyle \sum_{\smash{i}}$} \hspace{.2ex} \mathcircabove{Q}_{\hspace{-0.1ex}i} \hspace{.2ex} q^i \hspace{-0.12ex} \right) \hspace{-0.4ex}
- \hspace{.1ex} \tikzmark{beginLegendreOfPotential} \scalebox{0.95}[1]{$\displaystyle \sum_{\smash{i}}$} \hspace{.2ex} \mathcircabove{Q}_{\hspace{-0.1ex}i} \hspace{.2ex} dq^i \tikzmark{endLegendreOfPotential}
= \scalebox{0.95}[1]{$\displaystyle \sum_{\smash{i}}$} \hspace{.32ex} q^i \hspace{.2ex} d\mathcircabove{Q}_{\hspace{-0.1ex}i}
\hspace{.2ex} ,
\\[1em]
%
\small
d \hspace{-0.1ex} \left( \scalebox{0.95}[1]{$\displaystyle \sum_{\smash{i}}$} \hspace{.2ex} \mathcircabove{Q}_{\hspace{-0.1ex}i} \hspace{.2ex} q^i - \potential \hspace{-0.1ex} \right) \hspace{-0.5ex}
= \scalebox{0.95}[1]{$\displaystyle \sum_{\smash{i}}$} \hspace{.32ex} q^i \hspace{.2ex} d\mathcircabove{Q}_{\hspace{-0.1ex}i}
= \scalebox{0.95}[1]{$\displaystyle \sum_{\smash{i}}$} \hspace{.4ex}
\scalebox{0.96}{$ \displaystyle \frac{\raisemath{-0.12em}{\partial \hspace{.1ex} \widehat{\potential}}}{\raisemath{-0.4em}{\partial \mathcircabove{Q}_{\hspace{-0.1ex}i}}} $}
\hspace{.25ex} d\mathcircabove{Q}_{\hspace{-0.1ex}i}
%
\hspace{.33ex} ;
\end{array}\end{equation*}%
\AddOverBrace[line width=.75pt][-0.3ex,0.3em]{beginLegendreOfPotential}{endLegendreOfPotential}{${\scriptstyle d\potential}$}

\nopagebreak\vspace{-0.33em}\begin{equation}\label{Castigliano:theorem}
q^i = \scalebox{0.96}{$ \displaystyle \frac{\raisemath{-0.12em}{\partial \hspace{.1ex} \widehat{\potential}}}{\raisemath{-0.4em}{\partial \mathcircabove{Q}_{\hspace{-0.1ex}i}}} $} \hspace{.2ex} ,
\:\;
\widehat{\potential}(\mathcircabove{Q}_{\hspace{-0.1ex}i})
= \scalebox{0.95}[1]{$\displaystyle \sum_{\smash{i}}$} \hspace{.2ex} \mathcircabove{Q}_{\hspace{-0.1ex}i} \hspace{.12ex} q^i - \potential
\hspace{.1ex} .
\end{equation}

\vspace{-0.1em}\noindent
Это теорема \href{https://en.wikipedia.org/wiki/Carlo_Alberto_Castigliano}{Castigliano}, $\widehat{\potential}$ называется дополнительной энергией.
В~линейной системе \eqref{staticsoflineardiscretesystem} ${\Rightarrow}$ ${\widehat{\potential} = \potential}$.
Теорема~\eqref{Castigliano:theorem} бывает очень полезна\:--- когда легко находится~${\widehat{\potential}(\mathcircabove{Q}_{\hspace{-0.1ex}i})}$.
Встречаются так называемые статически определимые системы, в~которых все внутренние силы удаётся найти лишь из баланса сил и~моментов.
Для них \eqref{Castigliano:theorem} эффективна.

В~отличие от линейной задачи~\eqref{staticsoflineardiscretesystem}, нелинейная задача~\eqref{staticequilibriumwithpotentialenergy} может не~иметь решений вовсе или~же иметь их н\'{е}сколько.

....

Рассказ о~статике в~общей механике закончим \ru{принципом}\en{the} \hbox{d’\hspace{-0.2ex}Alembert’\en{s}\ru{а}}\en{ principle}:
уравнения динамики отличаются от~статических лишь наличием дополнительных \inquotes{сил инерции}~${\hspace{-0.2ex}m_k \hspace{.1ex} \mathdotdotabove{\locationvector}_k}$.
Принцип \hbox{d’\hspace{-0.2ex}Alembert’а} достаточно очевиден, но бездумное применение может привести к~ошибкам.
Например, уравнения вязкой жидкости в~статике и~в~динамике отличаются не~только лишь инерционными добавками.
Но для твёрдых упругих тел принцип \hbox{d’\hspace{-0.2ex}Alembert’а} полностью справедлив.

\end{otherlanguage}

\en{\section{Mechanics of relative motion}}

\ru{\section{Механика относительного движения}}

\label{para:mechanicsofrelativemotion}

\begin{otherlanguage}{russian}

До~этого не~ставился вопрос о~системе отсчёта, всё рассматривалось в~некой \inquotes{абсолютной} системе или одной из инерциальных систем~(\sectionref{para:initialconcepts.discreteapproach}).
Теперь представим себе две системы: \inquotes{абсолютную} и~\inquotes{подвижную}

...

\begin{equation*}
\begin{array}{c}
\initiallocationvector = \locationvector + \bm{x}
\\
\locationvector = \hspace{-0.1ex} \rho_i \hspace{.1ex} \mathcircabove{\bm{e}}_i
\hspace{.1ex} , \:\:
\bm{x} = x_i \bm{e}_i
\\
\mathdotabove{\initiallocationvector} = \mathdotabove{\locationvector} + \mathdotabove{\bm{x}}
\\
\mathdotabove{\locationvector} = \hspace{-0.1ex}
\mathdotabove{\rho}_i \hspace{.1ex} \mathcircabove{\bm{e}}_i
\hspace{.1ex} , \:\:
\mathdotabove{\bm{x}} = \hspace{-0.15ex} \bigl( x_i \bm{e}_i \bigr)^{\hspace{-0.15ex}\tikz[baseline=-0.2ex]\draw[black, fill=black] (0,0) circle (.28ex);} \hspace{-0.15ex}
= \mathdotabove{x}_i \bm{e}_i \hspace{-0.1ex} + x_i \mathdotabove{\bm{e}}_i
\end{array}
\end{equation*}

${x_i \hspace{-0.1ex} \neq \constant}$ $\Rightarrow$ ${\mathdotabove{x}_i \hspace{-0.1ex} \neq 0}$

\en{By}\ru{По}~\eqrefwithchapterdotpara{angularvelocityandbasisvectors}{chapter:mathapparatus}{para:rotationtensor}

\nopagebreak\vspace{-0.2em}\begin{equation*}
\mathdotabove{\bm{e}}_i \hspace{-0.1ex} = \bm{\omega} \hspace{-0.1ex} \times \hspace{-0.1ex} \bm{e}_i
\hspace{.33ex} \Rightarrow \hspace{.4ex}
x_i \mathdotabove{\bm{e}}_i \hspace{-0.1ex} = \bm{\omega} \hspace{-0.1ex} \times \hspace{-0.1ex} x_i \bm{e}_i \hspace{-0.1ex}
= \bm{\omega} \hspace{-0.1ex} \times \hspace{-0.1ex} \bm{x}
\end{equation*}

${
\mathdotabove{\bm{x}} = \mathdotabove{x}_i \bm{e}_i + \hspace{.1ex} \bm{\omega} \hspace{-0.16ex} \times \hspace{-0.16ex} \bm{x}
}$

\begin{equation*}
\bm{v} \equiv \hspace{-0.1ex} \mathdotabove{\initiallocationvector} = \mathdotabove{\locationvector} + \mathdotabove{\bm{x}}
= \hspace{-0.2ex} \tikzmark{beginEVelocity} \hspace{.25ex} \mathdotabove{\locationvector} \hspace{.1ex} + \hspace{.1ex} \bm{\omega} \hspace{-0.2ex} \times \hspace{-0.2ex} \bm{x} \tikzmark{endEVelocity}
\hspace{.4ex} \tikzmark{beginRelativeVelocity} \hspace{-0.4ex} - \bm{\omega} \hspace{-0.2ex} \times \hspace{-0.2ex} \bm{x} \hspace{.1ex} + \hspace{.1ex} \mathdotabove{\bm{x}} \hspace{-0.33ex} \tikzmark{endRelativeVelocity}
\end{equation*}%
\AddUnderBrace[line width=.75pt][0,0]{beginEVelocity}{endEVelocity}{${ \scriptstyle \bm{v}_{\hspace{-0.1ex}e} }$}
\AddUnderBrace[line width=.75pt][.4ex,0][xshift=.2ex]{beginRelativeVelocity}{endRelativeVelocity}{${ \scriptstyle \bm{v}_{r\kern-0.1exel} }$}

${
\mathdotabove{\bm{x}} \hspace{.1ex} - \hspace{.1ex} \bm{\omega} \hspace{-0.2ex} \times \hspace{-0.2ex} \bm{x} = \mathdotabove{x}_i \bm{e}_i
\equiv \hspace{.1ex} \bm{v}_{r\kern-0.1exel}
}$\:--- relative velocity,
${
\mathdotabove{\locationvector} \hspace{.1ex} + \hspace{.1ex} \bm{\omega} \hspace{-0.2ex} \times \hspace{-0.2ex} \bm{x}
\equiv \hspace{.1ex} \bm{v}_{\hspace{-0.1ex}e}
}$

\begin{equation}
\bm{v} = \bm{v}_{\hspace{-0.1ex}e} \hspace{-0.1ex} + \bm{v}_{r\kern-0.1exel}
\end{equation}

...

\begin{equation*}
\begin{array}{c}
\mathdotabove{\initiallocationvector} = \mathdotabove{\locationvector} + \mathdotabove{\bm{x}}
\\
\mathdotdotabove{\initiallocationvector} = \mathdotdotabove{\locationvector} + \mathdotdotabove{\bm{x}}
\\
\bm{w} \equiv \hspace{.1ex} \mathdotabove{\bm{v}} = \hspace{-0.15ex} \mathdotdotabove{\initiallocationvector} = \mathdotdotabove{\locationvector} + \mathdotdotabove{\bm{x}}
\\
\mathdotdotabove{\locationvector} = \hspace{-0.1ex}
\mathdotdotabove{\rho}_i \hspace{.1ex} \mathcircabove{\bm{e}}_i
\hspace{.1ex} , \:\:
\mathdotdotabove{\bm{x}} = \hspace{-0.15ex} \bigl( x_i \bm{e}_i \bigr)^{ \hspace{-0.15ex} \tikz[baseline=-0.2ex] \draw[black, fill=black] (0,0) circle (.28ex); \hspace{.2ex} \tikz[baseline=-0.2ex] \draw[black, fill=black] (0,0) circle (.28ex); } \hspace{-0.2ex}
= \hspace{-0.15ex} \bigl( \mathdotabove{x}_i \bm{e}_i \hspace{-0.1ex} + x_i \mathdotabove{\bm{e}}_i \bigr)^{ \hspace{-0.15ex} \tikz[baseline=-0.2ex] \draw[black, fill=black] (0,0) circle (.28ex); } \hspace{-0.15ex}
= \mathdotdotabove{x}_i \bm{e}_i \hspace{-0.1ex} + \mathdotabove{x}_i \mathdotabove{\bm{e}}_i \hspace{-0.1ex}
+ \mathdotabove{x}_i \mathdotabove{\bm{e}}_i \hspace{-0.1ex} + x_i \mathdotdotabove{\bm{e}}_i \hspace{-0.1ex}
\end{array}
\end{equation*}

\nopagebreak\begin{equation*}
\mathdotabove{\bm{e}}_i \hspace{-0.1ex} = \bm{\omega} \hspace{-0.1ex} \times \hspace{-0.12ex} \bm{e}_i
\hspace{.33ex} \Rightarrow \hspace{.4ex}
\mathdotdotabove{\bm{e}}_i \hspace{-0.1ex}
= \hspace{-0.15ex} \bigl( \hspace{.1ex} \bm{\omega} \hspace{-0.2ex} \times \hspace{-0.2ex} \bm{e}_i \hspace{.1ex} \bigr)^{ \hspace{-0.15ex} \tikz[baseline=-0.2ex] \draw[black, fill=black] (0,0) circle (.28ex); } \hspace{-0.15ex}
= \mathdotabove{\bm{\omega}} \hspace{-0.2ex} \times \hspace{-0.2ex} \bm{e}_i \hspace{-0.1ex} + \bm{\omega} \hspace{-0.2ex} \times \hspace{-0.2ex} \mathdotabove{\bm{e}}_i \hspace{-0.1ex}
= \mathdotabove{\bm{\omega}} \hspace{-0.2ex} \times \hspace{-0.2ex} \bm{e}_i \hspace{-0.1ex}
+ \bm{\omega} \hspace{-0.2ex} \times \hspace{-0.33ex} \bigl( \hspace{.1ex} \bm{\omega} \hspace{-0.2ex} \times \hspace{-0.2ex} \bm{e}_i \hspace{.1ex} \bigr)
\end{equation*}

\nopagebreak\begin{equation*}
x_i \mathdotdotabove{\bm{e}}_i \hspace{-0.1ex}
= x_i \bigl( \hspace{.1ex} \bm{\omega} \hspace{-0.2ex} \times \hspace{-0.2ex} \bm{e}_i \hspace{.1ex} \bigr)^{ \hspace{-0.15ex} \tikz[baseline=-0.2ex] \draw[black, fill=black] (0,0) circle (.28ex); } \hspace{-0.15ex}
= \mathdotabove{\bm{\omega}} \hspace{-0.12ex} \times \hspace{-0.12ex} x_i \bm{e}_i
+ \bm{\omega} \hspace{-0.2ex} \times \hspace{-0.33ex} \bigl( \hspace{.1ex} \bm{\omega} \hspace{-0.2ex} \times \hspace{-0.2ex} x_i \bm{e}_i \hspace{.1ex} \bigr) \hspace{-0.15ex}
= \mathdotabove{\bm{\omega}} \hspace{-0.12ex} \times \hspace{-0.12ex} \bm{x}
+ \hspace{.1ex} \bm{\omega} \hspace{-0.2ex} \times \hspace{-0.33ex} \bigl( \hspace{.1ex} \bm{\omega} \hspace{-0.2ex} \times \hspace{-0.2ex} \bm{x} \hspace{.1ex} \bigr)
\end{equation*}

\nopagebreak\begin{equation*}
\mathdotabove{\bm{e}}_i \hspace{-0.1ex} = \bm{\omega} \hspace{-0.1ex} \times \hspace{-0.12ex} \bm{e}_i
\hspace{.33ex} \Rightarrow \hspace{.4ex}
\mathdotabove{x}_i \mathdotabove{\bm{e}}_i \hspace{-0.1ex}
= \bm{\omega} \hspace{-0.1ex} \times \hspace{-0.1ex} \mathdotabove{x}_i \bm{e}_i \hspace{-0.1ex}
= \bm{\omega} \hspace{-0.1ex} \times \hspace{-0.1ex} \bm{v}_{r\kern-0.1exel}
\end{equation*}

${
\mathdotdotabove{x}_i \bm{e}_i
\equiv \hspace{.1ex} \bm{w}_{r\kern-0.1exel}
}$\:--- relative acceleration

${
2 \hspace{.2ex} \mathdotabove{x}_i \mathdotabove{\bm{e}}_i \hspace{-0.1ex}
= 2 \hspace{.33ex} \bm{\omega} \hspace{-0.2ex} \times \hspace{-0.2ex} \bm{v}_{r\kern-0.1exel} \hspace{-0.1ex}
\equiv \hspace{.1ex} \bm{w}_{\hspace{-0.1ex}\raisemath{-0.16ex}{C}\hspace{-0.1ex}or}
}$\:--- Coriolis acceleration

\begin{equation*}
\mathdotdotabove{\bm{x}}
= \bm{w}_{r\kern-0.1exel} \hspace{-0.1ex}
+ \bm{w}_{\hspace{-0.1ex}\raisemath{-0.16ex}{C}\hspace{-0.1ex}or} \hspace{-0.1ex}
+ x_i \mathdotdotabove{\bm{e}}_i
\end{equation*}

\begin{equation*}
\begin{array}{c}
\bigl( x_i \mathdotabove{\bm{e}}_i \bigr)^{ \hspace{-0.15ex} \tikz[baseline=-0.2ex] \draw[black, fill=black] (0,0) circle (.28ex); } \hspace{-0.15ex}
= \mathdotabove{x}_i \mathdotabove{\bm{e}}_i \hspace{-0.1ex} + x_i \mathdotdotabove{\bm{e}}_i \hspace{-0.12ex}
= \hspace{.12ex} \smalldisplaystyleonehalf \hspace{.1ex} \bm{w}_{\hspace{-0.1ex}\raisemath{-0.16ex}{C}\hspace{-0.1ex}or} \hspace{-0.15ex} + x_i \mathdotdotabove{\bm{e}}_i \hspace{-0.1ex}
\\[.2em]
%
\bigl( x_i \mathdotabove{\bm{e}}_i \bigr)^{ \hspace{-0.15ex} \tikz[baseline=-0.2ex] \draw[black, fill=black] (0,0) circle (.28ex); } \hspace{-0.15ex}
= \hspace{-0.15ex} \bigl( \hspace{.1ex} \bm{\omega} \hspace{-0.2ex} \times \hspace{-0.2ex} \bm{x} \hspace{.1ex} \bigr)^{ \hspace{-0.15ex} \tikz[baseline=-0.2ex] \draw[black, fill=black] (0,0) circle (.28ex); } \hspace{-0.15ex}
= \mathdotabove{\bm{\omega}} \hspace{-0.2ex} \times \hspace{-0.2ex} \bm{x} + \bm{\omega} \hspace{-0.2ex} \times \hspace{-0.2ex} \mathdotabove{\bm{x}}
\end{array}
\end{equation*}

\begin{equation*}
\bm{\omega} \hspace{-0.12ex} \times \hspace{-0.12ex} \mathdotabove{\bm{x}}
= \bm{\omega} \hspace{-0.2ex} \times \hspace{-0.33ex} \bigl( \hspace{.1ex} \mathdotabove{x}_i \bm{e}_i + \hspace{.1ex} \bm{\omega} \hspace{-0.2ex} \times \hspace{-0.2ex} \bm{x} \hspace{.1ex} \bigr) \hspace{-0.15ex}
= \hspace{-0.2ex} \tikzmark{beginCoriolisHalf} \hspace{.2ex} \bm{\omega} \hspace{-0.1ex} \times \hspace{-0.1ex} \mathdotabove{x}_i \bm{e}_i \hspace{.2ex} \tikzmark{endCoriolisHalf} \hspace{-0.25ex}
+ \hspace{.1ex} \bm{\omega} \hspace{-0.2ex} \times \hspace{-0.33ex} \bigl( \hspace{.1ex} \bm{\omega} \hspace{-0.2ex} \times \hspace{-0.2ex} \bm{x} \hspace{.1ex} \bigr)
\end{equation*}%
\AddUnderBrace[line width=.75pt][0,-0.1em]{beginCoriolisHalf}{endCoriolisHalf}{${\scalebox{0.8}{$ \mathdotabove{x}_i \mathdotabove{\bm{e}}_i \hspace{-0.15ex} = \smalldisplaystyleonehalf \hspace{.1ex} \bm{w}_{\hspace{-0.1ex}\raisemath{-0.16ex}{C}\hspace{-0.1ex}or} $}}$}

...

\end{otherlanguage}

\en{\section{Small oscillations (vibrations)}}

\ru{\section{Малые колебания (вибрации)}}

\label{para:smalloscillations}

% periodic motion

\en{If}\ru{Если} \en{the statics}\ru{статика} \en{of a~linear system}\ru{линейной системы} \en{is described}\ru{описывается} \en{by equation}\ru{уравнением}~\eqref{staticsoflineardiscretesystem}, \en{then}\ru{то} \en{in the dynamics}\ru{в~динамике} \en{we have}\ru{мы имеем}

\nopagebreak\vspace{-0.4em}\begin{equation}\label{dynamicsoflineardiscretesystem}
\scalebox{0.95}[1]{$\displaystyle \sum_{\smash{k}}$} \hspace{-0.2ex} \left(^{\mathstrut} \hspace{-0.2ex} A_{ik} \hspace{.12ex} \mathdotdotabove{q}_k + C_{ik} \hspace{.12ex} q^k \right) \hspace{-0.4ex}
= \hspace{.1ex} \mathcircabove{Q}_{\hspace{-0.1ex}i}(t) \hspace{.1ex} ,
\vspace{-0.1em}\end{equation}

\vspace{-0.25em}\noindent
\en{where}\ru{где} ${A_{ik}}$\en{ is}\ru{\:---} \en{the }\en{symmetric and positive}\ru{симметричная и~положительная} \inquotesx{\en{matrix}\ru{матрица} \en{of kinetic energy}\ru{кинетической энергии}}[.]

\en{Any description}\ru{Любое описание} \en{of oscillations}\ru{колебаний} \en{almost always}\ru{почти всегда} \en{includes}\ru{включает} \en{the term}\ru{термин} \inquotesx{\en{mode}\ru{мода}}[.]
\en{A~mode of vibration}\ru{Мода вибрации} \en{can be defined}\ru{может быть определена} \en{as}\ru{как} \en{a~way of vibrating}\ru{способ вибрирования} \en{or}\ru{или} \en{a~pattern}\ru{паттерн} \en{of vibration}\ru{вибрации}.
\en{A~normal mode}\ru{Нормальная мода} \en{is}\ru{есть} \en{a~pattern}\ru{паттерн} \en{of periodic motion}\ru{периодического движения}, \en{when}\ru{когда} \en{all parts}\ru{все части} \en{of an~oscillating system}\ru{колеблющейся системы} \en{move sinusoidally}\ru{движутся синусоидально} \en{with the same frequency}\ru{одинаковой частотой} \en{and}\ru{и} \en{with a~fixed phase relation}\ru{с~фиксированным соотношением фаз}.
\en{The free motion}\ru{Свободное движение}\ru{,} \en{described by the normal modes}\ru{описываемое нормальными модами}\ru{,} \en{takes place}\ru{происходит} \en{at fixed frequencies}\ru{на фиксированных частотах}\:--- \en{the natural resonant frequencies}\ru{натуральных резонансных частотах} \en{of an~oscillating system}\ru{колеблющейся системы}.

\en{The most generic motion}\ru{Самое общее движение} \en{of an~oscillating system}\ru{колеблющейся системы} \en{is}\ru{есть} \ru{некоторая суперпозиция}\en{some superposition} \en{of normal modes}\ru{нормальных мод} \en{of this system}\ru{этой системы}.%
\footnote{\en{The modes}\ru{Моды} \en{are }\inquotes{\en{normal}\ru{нормальны}} \en{in the sense that}\ru{в~смысле, что} \en{they move independently}\ru{они движутся независимо}, \en{and}\ru{и} \en{an~excitation of one mode}\ru{возбуждение одной моды} \en{will never cause}\ru{никогда не вызовет} \en{a~motion of another mode}\ru{движение другой моды}.
\en{In mathematical terms}\ru{В~математических терминах}\en{,} \en{normal modes}\ru{нормальные моды} \en{are orthogonal to each other}\ru{ортогональны друг другу}.
\en{In music}\ru{В~музыке}\en{,} \en{normal modes}\ru{нормальные моды} \en{of vibrating instruments}\ru{вибрирующих инструментов}~(\en{strings}\ru{струн}, \en{air pipes}\ru{воздушных трубок}, \en{percussion}\ru{перкуссии} \en{and others}\ru{и~других}) \en{are called}\ru{называются} \inquotes{harmonics} \en{or}\ru{или} \inquotesx{overtones}[.]}

\en{A~research}\ru{Изучение} \en{of~an~oscillating system}\ru{колеблющейся системы} \en{most often begins}\ru{чаще всего начинается} \en{with }\ru{с~}\en{ortho\-gonal}\ru{орто\-гональ\-ных}~(\en{normal}\ru{нормальных}) \inquotesx{\en{modes}\ru{мод}}[---] \en{harmonics}\ru{гармоник}, \en{free}\ru{свободных} (\en{without any driving or damping force}\ru{без какой-либо движущей или демпфирующей силы}) \en{sinusoidal oscillations}\ru{синусоидальных колебаний}

\nopagebreak\vspace{-0.25em}\begin{equation*}
q^k \hspace{-0.1ex} (t) \hspace{-0.2ex} = \mathasteriskabove{q}_{\hspace{-0.1ex}k} \operatorname{sin} \omega_k \hspace{.1ex} t
\hspace{.1ex} .
\end{equation*}

\vspace{-0.2em} \noindent
\en{Multipliers}\ru{Множители}~${\mathasteriskabove{q}_{\hspace{-0.1ex}k} \hspace{-0.15ex} = \constant}$\en{ are}\ru{\:---} \en{ortho\-gonal}\ru{орто\-гональ\-ные}~(\en{normal}\ru{нормальные}) \inquotes{\en{modes}\ru{моды}} \en{of~oscillation}\ru{колебания}, ${\omega_k\hspace{-0.1ex}}$\en{ are}\ru{\:---} \en{natural}\ru{натуральные}~(\en{resonant}\ru{резонансные}, \en{eigen-}\ru{собственные}) \en{frequencies}\ru{частоты}.
\en{This set}\ru{Этот набор}, \en{dependent on}\ru{зависящий от} \en{the structure}\ru{структуры} \en{of~an~oscillating object}\ru{колеблющегося объекта}, \en{the materials}\ru{материалов} \en{and}\ru{и}~\en{the boundary conditions}\ru{краевых условий}, \en{is found}\ru{находится} \en{from}\ru{из} \en{the eigenvalue problem}\ru{задачи на~собственные значения}

\nopagebreak\vspace{-0.1em}\begin{equation}
\begin{array}{c}
\mathcircabove{Q}_{\hspace{-0.1ex}i} \hspace{-0.15ex} = 0
\hspace{.1ex} ,
\:\;
\mathdotdotabove{q}_k \hspace{-0.12ex} = - \hspace{.2ex} \omega_k^2 \hspace{.25ex} \mathasteriskabove{q}_{\hspace{-0.1ex}k} \hspace{-0.1ex} \operatorname{sin} \omega_k \hspace{.1ex} t
\hspace{.1ex} ,
\:\;
\eqref{dynamicsoflineardiscretesystem}
\:\: \Rightarrow
\\[.3em]
%
\Rightarrow \:\,
\scalebox{0.95}[1]{$\displaystyle \sum_{\smash{k}}$} \Bigl( \hspace{-0.2ex} C_{ik} \hspace{-0.1ex} - \hspace{-0.2ex} A_{ik} \hspace{.25ex} \omega_k^2 \hspace{.1ex} \Bigr) \hspace{.12ex}
\mathasteriskabove{q}_{\hspace{-0.1ex}k} \operatorname{sin} \omega_k \hspace{.1ex} t
= 0
\end{array}
\end{equation}

...

%%\begin{otherlanguage}{russian}
%%\end{otherlanguage}

\ru{Интеграл}\en{The} Duhamel\en{’s}\ru{’я}\en{ integral}
is a~way of calculating the response
\en{of linear systems}\ru{линейных систем}
to an~arbitrary
time-varying
external
perturbation.

...

\section*{\small \wordforbibliography}

\begin{changemargin}{\parindent}{0pt}
\fontsize{10}{12}\selectfont

\en{In a~long list}\ru{В~длинном списке}
\en{of the books}\ru{книг}
\en{about the classical mechanics}\ru{про классическую механику}\en{,}
\en{the~reader}\ru{читатель}
\en{can find}\ru{может найти}
\en{the works}\ru{работы}
\en{of both}\ru{и}
\en{the specialists in mechanics}\ru{специалистов по механике}~\cite{goldstein-classicalmechanics, treatiseonanalyticaldynamics-by-l.a.pars, loitsjanskiy.lurie, lurie-analyticalmechanics, olkhovskiy-theoreticalmechanicsforphysicists}\ru{,}
\en{and}\ru{и}
\en{the broadly oriented}\ru{широко ориентированных}
\en{theoretical physicists}\ru{физиков\hbox{-}теоретиков}~\cite{landau.lifshitz-shortcourse, terhaar-hamiltonianmechanics}.
%
\ru{Весьма интересна }\en{The book}\ru{книга}
\en{by Felix~R.\;Gantmacher (\foreignlanguage{russian}{Феликс~Р.\;Гантмахер})}\ru{Феликса~Р.\;Гантмахер’а}~\cite{gantmacher-analyticalmechanics}
\en{with the~compact but complete}\ru{с~компактным, но~полным}
\en{narration}\ru{изложением}
\en{of the fundamentals}\ru{основ}\en{ is pretty interesting}.

\end{changemargin}

