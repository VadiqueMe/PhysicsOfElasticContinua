\en{\chapter{Fundamentals of generic mechanics}}

\ru{\chapter{Основы общей механики}}

\thispagestyle{empty}

\label{chapter:genericmechanics}

\en{\section{Initial concepts}}

\ru{\section{Исходные представления}}

\en{\emph{Discrete approach}}

\ru{\emph{Дискретный подход}}

\en{\lettrine[lines=2, findent=2pt, nindent=0pt]{I}{n} generic mechanics, systems of~particles~(\inquotes{pointlike masses}, \inquotes{material points}) with masses~${m_k \hspace{-0.2ex}=\hspace{-0.1ex} \const}$ and motion functions~${\bm{r}_k(t)}$ are considered. Vector function of~time~${\bm{r}_k(t)}$ is measured relative to some reference system~--- a~solid body with a~clock~(\figref{fig:referencesystem}).}

\ru{\lettrine[lines=2, findent=2pt, nindent=0pt]{В}{\hspace{-0.25ex}} общей механике рассматриваются системы частиц~(\inquotes{точечных масс}, \inquotes{материальных точек}) с~массами~${m_k \hspace{-0.2ex}=\hspace{-0.1ex} \const}$ и~функциями движения~${\bm{r}_k(t)}$. Векторная функция времени~${\bm{r}_k(t)}$ определяется относительно какой\hbox{-}либо системы отсчёта~--- твёрдого тела с~часами~(\figref{fig:referencesystem}).}

\begin{wrapfigure}[10]{o}{.48\textwidth}
\makebox[.48\textwidth][c]{\begin{minipage}[t]{.44\textwidth}
\vspace{-0.2em}
\scalebox{1.1}[1.11]{
\begin{tikzpicture}[scale=0.9]

	\draw[line width=1.2pt, black] (0,0) -- (3.2,0);
	\draw[line width=1.2pt, black] (0,0) -- (-1.8,0);
	\foreach \xground in {-1.65, -1.4, ..., 3.3}
		\draw [line width=0.4pt, black!80] (\xground,0) -- (\xground-0.2,-0.2);

	\path (-1.1,0) node [shape=coordinate] (clocktower) {};
	\def\clockradius{0.4}
	\def\clocksquare{\clockradius + 0.1}
	\def\clockbase{\clockradius - 0.05}
	\def\clockheight{0.8}
	\path (-1.1,\clockheight+\clocksquare) node [shape=coordinate] (clock) {};

	\draw [line width=1.2pt, black] ($ (clocktower) + (\clockbase,0) $) -- ++(up:\clockheight);
	\draw [line width=1.2pt, black] ($ (clocktower) - (\clockbase,0) $) -- ++(up:\clockheight);
	\draw [line width=1.2pt, black, rounded corners=1.2pt] ($ (clock) + (\clocksquare,\clocksquare) $) rectangle ($ (clock) - (\clocksquare,\clocksquare) $);
	\draw [line width=1.2pt, black] ($ (clock) + (\clockbase,\clocksquare) $) arc(0:180:\clockbase);

	\draw [line width=1.2pt, blue] (clock) circle(\clockradius);
	\draw [line width=1.2pt, blue] (clock) -- ++(\clockradius - 0.2, 0);
	\draw [line width=1pt, blue, rotate around={-30:(clock)}] (clock) -- ++(0, \clockradius - 0.15);

	\path (0.16,0.2) node [shape=coordinate] (O) {};
	\path (O) -- ++(-1,-1) node [shape=coordinate] (first) {};
	\path (O) -- ++(2.4,0) node [shape=coordinate] (second) {};
	\path (O) -- ++(0,2) node [shape=coordinate] (third) {};

	\draw [line width=1pt, blue, style=double, double distance=0.5mm, -{Triangle[open, angle=60:3.2mm]}] (O) -- (first)
		node [pos=0.77, below, xshift=4.5mm, yshift=1mm] {$\bm{e}_1$};
	\draw [line width=1pt, blue, style=double, double distance=0.5mm, -{Triangle[open, angle=60:3.2mm]}] (O) -- (second)
		node [pos=0.8 ,above, xshift=2mm, yshift=1.2mm] {$\bm{e}_2$};
 	\draw [line width=1pt, blue, style=double, double distance=0.5mm, -{Triangle[open, angle=60:3.2mm]}] (O) -- (third)
		node [pos=0.8, right, xshift=1mm, yshift=1mm] {$\bm{e}_3$};

	\path (2.5,1.6) node [shape=coordinate] (m) {};

	\path (m) node [shape=circle, inner sep=1mm, outer sep=0] (mcirc) {};

	\draw [line width=1.6pt, black, -{Stealth[round, length=5mm, width=3.6mm]}] (O) -- (mcirc)
		node [pos=0.52, above] {$\bm{r}$} ;

	\draw [line width=1.6pt, black, fill=black!50] (m) circle (1.6mm) ;

	\path (mcirc) node [xshift=-3.2mm, yshift=3mm] {$m$} ;

	\draw [line width=1pt, blue, fill=white] (O) circle (1.2mm) ;

\end{tikzpicture}}
\vspace{-1.72em}\caption{}\label{fig:referencesystem}
\end{minipage}}
\end{wrapfigure}

\en{Long time ago, some absolute space was accepted as a~reference system: empty at~first, and then filled with continuous elastic medium~--- the~ether. Later it became clear that within the~scope of~classical mechanics any reference systems are usable, but preference is given to inertial systems, where a~point moves without acceleration~(${\mathdotdotabove{\bm{r}} = \bm{0}}$) in the absence of external interactions.}

\ru{Когда\hbox{-}то давно, за~систему отсчёта принималось некое абсолютное пространство: сначала пустое, а~затем заполненное сплошной упругой \hbox{средой}~--- эфиром. Позже стало ясно, что в~пределах классической механики можно пользоваться \hbox{любыми} системами \hbox{отсчёта}, но предпочтение отдаётся инерциальным системам, где \hbox{точка} движется без~ускорения~(${\mathdotdotabove{\bm{r}} = \bm{0}}$) при~отсутствии внешних взаимодействий.}

\en{The measure of interaction in mechanics is a~vector of~force~$\bm{F}$. In~famous Newton’s equation}

\ru{Мерой взаимодействия в~механике является вектор силы~$\bm{F}$. В~известном уравнении Newton’а}

\nopagebreak\vspace{-0.32em}\begin{equation}\label{law:ofnewton}
m \mathdotdotabove{\bm{r}} = \bm{F} ( \bm{r}, \mathdotabove{\bm{r}}, t )
\end{equation}

\en{\vspace{-0.25em} \noindent the right\hbox{-}hand side depends only on position, speed and explicitly presented time, as acceleration is directly proportional to force.}

\ru{\vspace{-0.25em} \noindent правая часть зависит лишь от положения, скорости и~явно входящего времени, поскольку ускорение прямо пропорционально силе.}

\en{Remind theses of dynamics of a~system of particles. Force acting on~$k$\hbox{-}th particle}

\ru{Напомним положения динамики системы частиц. Сила, действующая на~$k$\hbox{-}ую частицу}

\nopagebreak\vspace{-0.64em}\begin{equation}\label{forceonparticle}
\bm{F}_k = \bm{F}^{\expexternal}_{\hspace{-0.16ex}k} \hspace{-0.1ex} + \scalebox{0.84}[0.84]{$\displaystyle \underset{\raisemath{.25ex}{\smash{j}}}{\sum}$} \bm{F}^{\expinternal}_{\hspace{-0.16ex}kj} \hspace{-0.12ex},
\end{equation}

\en{\vspace{-0.2em} \noindent where the first addend~${\bm{F}^{\expexternal}_{\hspace{-0.16ex}k}}$ is external force, and the second is sum of~internal ones (${\bm{F}^{\expinternal}_{\hspace{-0.16ex}kj}}$ is force from particle with number~\inquotes{$\hspace{-0.1ex}j\hspace{0.25ex}$}).}

\ru{\vspace{-0.2em} \noindent где первое слагаемое~${\bm{F}^{\expexternal}_{\hspace{-0.16ex}k}}$~--- это внешняя сила, а~второе~--- сумма внутренних (${\bm{F}^{\expinternal}_{\hspace{-0.16ex}kj}}$~--- сила от~частицы с~номером~\inquotes{$\hspace{-0.1ex}j\hspace{0.25ex}$}).}

\foreignlanguage{russian}{Принимая принцип действия и~противодействия ${\bm{F}^{\expinternal}_{\hspace{-0.16ex}kj} = - \bm{F}^{\expinternal}_{\hspace{-0.4ex}j\hspace{-0.05ex}k}}$, получим из~\eqref{law:ofnewton} и~\eqref{forceonparticle} закон баланса импульса}

...




\ru{\section{Совершенно жёсткое недеформируемое твёрдое тело}}

\en{\section{Absolutely rigid undeformable solid body}}

\en{\emph{Discrete and continual approaches}}

\ru{\emph{Дискретный и континуальный подходы}}

%% \inquotesx{Абсолютно твёрдое}[,] оно~же \inquotes{абсолютно жёсткое} и~\inquotesx{абсолютно прочное}[---] это несбыточная мечта любого инженера.

\en{\noindent To define position (location) of absolutely rigid~(undeformable) body it’s enough to choose some one of its points, to set location of this point, as well as angular orientation of body~(\figref{fig:bodyoffsetandrotation}).}

\ru{\noindent Для определения положения совершенно жёсткого~(недеформируемого) тела достаточно выбрать какую\hbox{-}либо одну его точку\hbox{-}полюс, задать положение~${\bm{r}(t)}$ этой точки, а~также угловую ориентацию тела~(\figref{fig:bodyoffsetandrotation}).}

\begin{comment}
\makeatletter
\newcommand\xofcoordinate[2][center]{{%
	\pgfpointanchor{#2}{#1}%
	\pgfmathparse{\pgf@x/\pgf@xx}%
	\pgfmathprintnumber[precision=2]{\pgfmathresult}%
}}
\newcommand\yofcoordinate[2][center]{{%
	\pgfpointanchor{#2}{#1}%
	\pgfmathparse{\pgf@y/\pgf@yy}%
	\pgfmathprintnumber[precision=2]{\pgfmathresult}%
}}
\makeatother

\begin{tikzpicture}
	\coordinate (point0) at (-4.3, 2.5);
	\coordinate (point1) at (-3.1, 3.2);
	\coordinate (point2) at (-2, 2.4);
	\coordinate (point3) at (-0.4, 1.6);
	\coordinate (point4) at (0.5, 0);
	\coordinate (point5) at (0, -2);
	\coordinate (point6) at (-1.5, -3);
	\coordinate (point7) at (-3, -2.2);
	\coordinate (point8) at (-3.5, -0.5);
	\coordinate (point9) at (-4.5, 1);

	\draw [line width=1.2pt, red] plot [smooth cycle, tension=0.8] coordinates {
		(point0) (point1) (point2) (point3) (point4)
		(point5) (point6) (point7) (point8) (point9)
	};

	\newcommand\xyofcoordinate[1]{\xofcoordinate{#1},\,\yofcoordinate{#1}}

	\draw [black, fill=black] (point0) circle (1mm) node [anchor=south east] {\xyofcoordinate{point0}};
	\draw [black, fill=black] (point1) circle (1mm) node [anchor=south, outer sep=4pt] {\xyofcoordinate{point1}};
	\draw [black, fill=black] (point2) circle (1mm) node [anchor=south west] {\xyofcoordinate{point2}};
	\draw [black, fill=black] (point3) circle (1mm) node [anchor=south west] {\xyofcoordinate{point3}};
	\draw [black, fill=black] (point4) circle (1mm) node [anchor=south west] {\xyofcoordinate{point4}};

	\draw [black,fill=black] (point5) circle (1mm) node [anchor=north west, outer sep=1pt] {\xyofcoordinate{point5}};
	\draw [black,fill=black] (point6) circle (1mm) node [anchor=north, outer sep=4pt] {\xyofcoordinate{point6}};
	\draw [black,fill=black] (point7) circle (1mm) node [anchor=north east, outer sep=2pt] {\xyofcoordinate{point7}};
	\draw [black,fill=black] (point8) circle (1mm) node [anchor=east, outer sep=4pt] {\xyofcoordinate{point8}};
	\draw [black,fill=black] (point9) circle (1mm) node [anchor=east, outer sep=3pt] {\xyofcoordinate{point9}};
\end{tikzpicture}
\end{comment}

\begin{wrapfigure}[16]{o}{.5\textwidth}
\makebox[.45\textwidth][c]{\begin{minipage}[t]{.45\textwidth}
\vspace{.1em}
\scalebox{1.1}[1.11]{
\begin{tikzpicture}[scale=0.6]

	\def\angleofrotation{36}

	\coordinate (O) at (-1.65, -1.05);
	\path (O) circle (1.6mm) node [shape=circle, inner sep=.64mm, outer sep=0] (Ocirc) {};

	\coordinate (Oinitial) at (-6, -2.5);

	\coordinate (bodypoint) at (-2, 1.5);

	\coordinate (point0) at (-4.3, 2.5);
	\coordinate (point1) at (-3.1, 3.2);
	\coordinate (point2) at (-2, 2.4);
	\coordinate (point3) at (-0.4, 1.6);
	\coordinate (point4) at (0.5, 0);
	\coordinate (point5) at (0, -2);
	\coordinate (point6) at (-1.5, -3);
	\coordinate (point7) at (-3, -2.2);
	\coordinate (point8) at (-3.5, -0.5);
	\coordinate (point9) at (-4.5, 1);

	%\draw [line width=1pt, blue!25,
		%style=double, double distance=0.5mm, -{Triangle[open, angle=60:3.2mm]}] (O) -- ++(-1.386,-0.8);
	%\draw [line width=1pt, blue!25,
		%style=double, double distance=0.5mm, -{Triangle[open, angle=60:3.2mm]}] (O) -- ++(1.386,-0.8);
 	%\draw [line width=1pt, blue!25,
		%style=double, double distance=0.5mm, -{Triangle[open, angle=60:3.2mm]}] (O) -- ++(0,1.6);

	\begin{scope}[rotate around={-\angleofrotation:(O)}]
	\draw [line width=1pt, black!50, opacity=50]
		plot [smooth cycle, tension=0.8] coordinates {
			(-4.3, 2.5) (-3.1, 3.2) (-2, 2.4) (-0.4, 1.6) (0.5, 0)
			(0, -2) (-1.5, -3) (-3, -2.2) (-3.5, -0.5) (-4.5, 1)
		};

	\path (-2, 1.5) circle (2mm) node [shape=circle, inner sep=.9mm, outer sep=0] (previousbodypoint) {};

	\tkzDrawArc[line width=.8pt, color=black!50, opacity=50](O,previousbodypoint)(bodypoint);

	\draw [line width=1pt, black!50, opacity=50, -{Stealth[round, length=4.5mm, width=2.8mm]}]
		(O) -- (previousbodypoint)
		node [pos=0.75, color=black!50, opacity=99, right, inner sep=2pt, outer sep=4pt] {$\mathcircabove{\bm{x}}$} ;

	\fill [white] (-2, 1.5) circle (2mm) ;
	\draw [line width=1pt, black!50, opacity=50] (-2, 1.5) circle (2mm) ;
	\end{scope}

	\draw [line width=1.6pt, black]
		plot [smooth cycle, tension=0.8] coordinates {
			(point0) (point1) (point2) (point3) (point4)
			(point5) (point6) (point7) (point8) (point9)
		};

	\draw [line width=1.6pt, black, fill=white] (bodypoint) circle (2mm)
		node [shape=circle, inner sep=0.9mm, outer sep=0] (pointcirc) {};

	\draw [line width=1.6pt, black, -{Stealth[round, length=5mm, width=3.6mm]}] (Oinitial) -- (pointcirc)
		node [pos=0.42, above, xshift=-2mm] {$\bm{R}$};

	\draw [line width=1.6pt, blue, -{Stealth[round, length=5mm, width=3.6mm]}] (Oinitial) -- (Ocirc)
		node [pos=0.5, above] {$\bm{r}$};

	\draw [line width=1.6pt, black, -{Stealth[round, length=5mm, width=3.6mm]}] (O) -- (pointcirc)
		node [pos=0.8, right, inner sep=2.5pt, outer sep=4pt] {$\bm{x}$};

	\draw [line width=1pt, blue, rotate around={\angleofrotation:(O)},
		style=double, double distance=0.5mm, -{Triangle[open, angle=60:3.2mm]}] (O) -- ++(-1.386,-0.8);
	\draw [line width=1pt, blue, rotate around={\angleofrotation:(O)},
		style=double, double distance=0.5mm, -{Triangle[open, angle=60:3.2mm]}] (O) -- ++(1.386,-0.8);
 	\draw [line width=1pt, blue, rotate around={\angleofrotation:(O)},
		style=double, double distance=0.5mm, -{Triangle[open, angle=60:3.2mm]}] (O) -- ++(0,1.6);

	\draw [line width=1pt, blue, fill=white] (O) circle (1.6mm)
		node [below, outer sep=1.6mm, xshift=2.4mm] {$\bm{e}_i$};

	\draw [line width=1pt, blue,
		style=double, double distance=0.5mm, -{Triangle[open, angle=60:3.2mm]}] (Oinitial) -- ++(-1.386,-0.8);
	\draw [line width=1pt, blue,
		style=double, double distance=0.5mm, -{Triangle[open, angle=60:3.2mm]}] (Oinitial) -- ++(1.386,-0.8);
 	\draw [line width=1pt, blue,
		style=double, double distance=0.5mm, -{Triangle[open, angle=60:3.2mm]}] (Oinitial) -- ++(0,1.6);

	\draw [line width=1pt, blue, fill=white] (Oinitial) circle(1.6mm)
		node [anchor=north, yshift=-1mm] {$\mathcircabove{\bm{e}}_i$};

\end{tikzpicture}}
\vspace{-1.6em}\caption{}\label{fig:bodyoffsetandrotation}
\end{minipage}}
\end{wrapfigure}

%% \begin{comment} %%
{\small
\noindent\leavevmode{\indent} As adjectives the difference between \inquotes{rigid} and \inquotes{solid} is that rigid is stiff (rather than flexible) while solid is in the state of a solid (not fluid). As a~noun solid is a~substance in the fundamental state of matter that retains its size and shape without need of container (as opposed to a~liquid or gas).

\noindent\leavevmode{\indent} A~rigid body is a~solid body in which deformation is zero (or negligibly small, so small it can be neglected). The distance between any two points of a~rigid body remains constant, regardless of external forces exerted on it. A~rigid body is usually considered as a~continuous distribution of mass (modeled as a continuous mass rather than as discrete particles).
\par}
%% \end{comment} %%

\foreignlanguage{russian}{В~модели совершенно~(абсолютно) жёсткого твёрдого тела расстояние между любыми двумя точками такого тела остаётся постоянным, не~завися от~внешних сил: деформации нет.}

\foreignlanguage{russian}{Совершенно жёсткое тело рассматривается либо как дискретная совокупность частиц, либо как непрерывное распределение массы (материальный континуум, сплошная среда).}

\foreignlanguage{russian}{Для радиуса\hbox{-}вектора некоторой точки тела имеем}

\begin{equation*}
\begin{array}{c}
\bm{R} = \bm{r} + \bm{x} \hspace{.1ex}, \:\:
\mathdotabove{\bm{R}} = \mathdotabove{\bm{r}} + \mathdotabove{\bm{x}}
\end{array}
\end{equation*}



${\mathcircabove{\bm{x}} = x_i \mathcircabove{\bm{e}}_i}$,
${\bm{x} = x_i \bm{e}_i}$,
${\mathdotabove{\bm{x}} = x_i \mathdotabove{\bm{e}}_i}$

\foreignlanguage{russian}{Компоненты~${x_i}$ в~совершенно жёстком теле не~зависят от~времени: ${x_i \hspace{-0.16ex} = \const}$.}

${\bm{x} = \bm{P} \hspace{-0.16ex} \dotp \mathcircabove{\bm{x}}}$,
${\mathdotabove{\bm{x}} = \mathdotabove{\bm{P}} \hspace{-0.16ex} \dotp \mathcircabove{\bm{x}}}$

${x_i \mathdotabove{\bm{e}}_i \hspace{-0.12ex} = \mathdotabove{\bm{P}} \hspace{-0.16ex} \dotp x_i \mathcircabove{\bm{e}}_i
\:\Rightarrow\:
\mathdotabove{\bm{e}}_i \hspace{-0.12ex} = \mathdotabove{\bm{P}} \hspace{-0.16ex} \dotp \mathcircabove{\bm{e}}_i}$



\foreignlanguage{russian}{Вводя псевдовектор угловой скорости~$\bm{\omega}$, ...}




\foreignlanguage{russian}{Пусть ${\mathcircabove{\bm{e}}_i}$~--- ортогональная тройка базисных ортов, неподвижная относительно системы отсчёта.}

\foreignlanguage{russian}{Имея неподвижный базис~${\mathcircabove{\bm{e}}_i}$ и движущийся вместе с~телом базис~${\bm{e}_i}$, ...}

\foreignlanguage{russian}{Если связать с~телом тройку декартовых осей с~ортами~${\bm{e}_i}$ (этот базис движется вместе с~телом), тогда угловая ориентация тела может быть задана тензором поворота ${\bm{P} \equiv \bm{e}_i \mathcircabove{\bm{e}}_i}$.}

...

\foreignlanguage{russian}{Движение тела полностью определяется функциями~${\bm{r}(t)}$ и~${\bm{P}(t)}$.}

...

\foreignlanguage{russian}{Переход от~формул, содержащих суммирование по~дискретным точкам, к~формулам для сплошного тела осуществляется заменой масс частиц на~массу ${\rho \hspace{0.2ex} dV}$ элемента объёма~$dV$ ($\rho$~--- плотность массы), и интегрированием по~всему объёму тела.}

...


{\small\setlength{\abovedisplayskip}{2pt}\setlength{\belowdisplayskip}{2pt}

Holonomic constraints are relations between position variables (and possibly time) which can be expressed as
\[ f(q_{1}, q_{2}, q_{3}, \ldots, q_{n}, t) = 0 \]

\noindent where ${\{q_{1},q_{2},q_{3},\ldots,q_{n}\}}$ are $n$ coordinates which describe the system. For example, the motion of a particle constrained to lie on sphere’s surface is subject to a holonomic constraint, but if the particle is able to fall off the sphere under the influence of gravity, the constraint becomes non-holonomic. For the first case the holonomic constraint may be given by the equation: ${r^{2} - a^{2} = 0}$, where $r$ is the distance from the centre of a sphere of radius $a$. Whereas the second non-holonomic case may be given by: ${r^{2} - a^{2} \geq 0}$.

Holonomic constraint depends only on coordinates and time. It does not depend on velocities or any higher time derivative. A~constraint that cannot be expressed as shown above is nonholonomic.

Velocity-dependent constraints like
\[ \displaystyle f(q_{1}, q_{2}, \ldots, q_{n}, {\dot {q}}_{1}, {\dot {q}}_{2}, \ldots, {\dot {q}}_{n}, t) = 0 \]
are mostly not holonomic.
\par}



\en{\section{Principle of virtual work}}

\ru{\section{Принцип виртуальной работы}}

\label{para:virtualworkprinciple.genericmechanics}

\begin{otherlanguage}{russian}

Виртуальным перемещением частицы с~радиусом\hbox{-}вектором~${\bm{r}_k}$ называется вариация~${\variation{\hspace{.1ex}\bm{r}_k}}$~--- %% н\'{е}которое
любое (неопределённое) бесконечно малое приращение~${\bm{r}_k}$, происходящее мгновенно вне~времени~(${\variation{t} \hspace{-0.2ex}\equiv\hspace{-0.2ex} 0}$) и~совместимое с~ограничениями\hbox{-}связями. %%~(constraints).
Если связей нет, то~есть система свободна, тогда виртуальные перемещения ${\variation{\hspace{.1ex}\bm{r}_k}}$ совершенно любые.

Связи бывают голономные~(holonomic, или геометрические), связывающие только положения~(перемещения)~--- это функции лишь координат и, возможно, времени
\begin{equation}\label{holonomicconstraint}
c\hspace{.2ex}(\bm{r}, t) = 0
\end{equation}
--- и~неголономные~(или дифференциальные), содержащие производные координат по~времени: ${c\hspace{.2ex}(\bm{r}, \mathdotabove{\bm{r}}, t) = 0}$ и~не~интегрируемые до~геометрических связей.

Далее рассматриваем системы, все связи в~которых~--- голономные. В~системе с~голономной связью виртуальные перемещения должны удовлетворять уравнению
\begin{equation}\label{requirementforvirtualdisplacements}
\displaystyle \sum_k \scalebox{0.92}{$\displaystyle \frac{\raisemath{-0.12em}{\partial \hspace{.1ex} c}}{\partial \hspace{.1ex} \bm{r}_k}$} \hspace{-0.1ex} \dotp \variation{\hspace{.1ex}\bm{r}_k} \hspace{-0.2ex} = 0 \hspace{.16ex}.
\vspace{-0.25em}\end{equation}

В~несвободных системах все силы делятся на две группы: активные и~реакции связей. Реакция~$\mathboldN_k$ действует со~стороны всех материальных ограничителей на~частицу \inquotes{${\hspace{.05ex}k\hspace{.25ex}}$} и~меняется в~соответствии с~уравнением~\eqref{holonomicconstraint} каждой связи. Принимается предложение об~идеальности связей:
\begin{equation}\label{idealconstraints}
\scalebox{0.9}{$\displaystyle \sum_k$} \mathboldN_k \hspace{-0.2ex} \dotp \variation{\hspace{.1ex}\bm{r}_k} \hspace{-0.16ex} = 0
\quad \textrm{---}
\vspace{-0.25em}\end{equation}
\noindent работа реакций на~любых виртуальных перемещениях равна нулю.

Принцип виртуальной работы выражается уравнением
\begin{equation}\label{principleofvirtualwork}
\displaystyle \sum_{\smash{k}} \hspace{-0.2ex} \left(^{\mathstrut} \hspace{-0.25ex} \bm{F}_k \hspace{-0.12ex} - m_k \mathdotdotabove{\bm{r}}_k \right) \hspace{-0.32ex} \dotp \variation{\hspace{.1ex}\bm{r}_k} \hspace{-0.2ex} = 0 \hspace{0.1ex},
\vspace{-0.32em}\end{equation}

\noindent где~$\bm{F}_k$~--- лишь активные силы.

Дифференциальное вариационное уравнение~\eqref{principleofvirtualwork} может показаться тривиальным следствием закона Ньютона~\eqref{law:ofnewton} и~условия идеальности связей~\eqref{idealconstraints}. Однако содержание~\eqref{principleofvirtualwork} несравненно обширнее. Известно~--- и~читатель вскоре это увидит,~--- что принцип~\eqref{principleofvirtualwork} может быть положен в~основу механики~\cite{gantmacher}. Различные модели упругих тел, описываемые в~этой книге, построены с~опорой на~этот принцип.

Для~примера рассмотрим совершенно жёсткое~(недеформируемое) твёрдое тело.

...


Проявилась замечательная особенность~\eqref{principleofvirtualwork}: это уравнение эквивалентно системе такого порядка, каково число степеней свободы системы, то~есть сколько независимых вариаций~${\variation{\hspace{.1ex}\bm{r}_k}}$ мы имеем. Если в~системе $N$~точек есть $m$ связей, то число степеней свободы ${n = 3N \hspace{-0.25ex} - m}$.

...


\en{\section{Balance of momentum, rotational momentum, and~energy}}

\ru{\section{Баланс импульса, момента импульса и~энергии}}

Эти фундаментальные законы баланса можно связать со~свойствами пространства и~времени~\cite{landau.lifshitz-shortcourse}. Сохранение импульса (количества движения) в~изолированной\inquotes{Изолированная (замкнутая) система~--- это система частиц, взаимодействующих друг с~другом, но ни с какими другими телами.} системе выводится из~однородности пространства \emph{(при любом параллельном переносе замкнутой системы как целого в~пространстве механические свойства этой системы не меняются)}. Сохранение момента импульса~--- следствие изотропии пространства \emph{(при любом повороте замкнутой системы как целого в~пространстве механические свойства этой системы не меняются)}. Энергия~же изолированной системы сохраняется, так~как время однородно (энергия ${\mathrm{E} \hspace{.1ex} \equiv \mathrm{T}(q, \mathdotabove{q} \hspace{.2ex}) \hspace{-0.2ex} + \Pi(q)}$ такой системы не~зависит явно от~времени).

Фундаментальные законы баланса можно вывести из принципа виртуальной работы~\eqref{principleofvirtualwork}. Перепишем его в~виде

\nopagebreak\vspace{-0.2em}\begin{equation}\label{principleofvirtualwork}
\displaystyle \sum_k \hspace{-0.2ex} \left(^{\mathstrut} \hspace{-0.25ex} \bm{F}^{\expexternal}_{\hspace{-0.16ex}k} \hspace{-0.12ex} - m_k \mathdotdotabove{\bm{r}}_k \right) \hspace{-0.32ex} \dotp \variation{\hspace{.1ex}\bm{r}_k} \hspace{-0.1ex}
+ \variation{\internalwork} \hspace{-0.2ex} = 0 \hspace{.1ex},
\vspace{-0.25em}\end{equation}

\vspace{-0.1em} \noindent где выделены внешние силы~${\bm{F}^{\expexternal}_{\hspace{-0.16ex}k}}$ и~виртуальная работа внутренних сил~${\variation{\internalwork}}$.

Предполагается, что внутренние силы не~совершают работы на~виртуальных перемещениях тела как жёсткого целого (${\varvector{\hspace{-0.1ex}\bm{\rho}}}$ и~${\varvector{\bm{o}}}$~--- произвольные константы, определяющие трансляцию и~поворот)

\nopagebreak\vspace{-0.8em}\begin{equation}\label{assumptionforvirtualwork}
\variation{\bm{r}_k} \hspace{-0.16ex}
= \varvector{\hspace{-0.1ex}\bm{\rho}} \hspace{.2ex} + \hspace{.12ex} \varvector{\bm{o}} \hspace{-0.2ex} \times \hspace{-0.1ex} \bm{r}_k
\hspace{.32ex} \Rightarrow \hspace{.4ex}
\variation{\internalwork} \hspace{-0.2ex} = 0 \hspace{.1ex}.
\end{equation}

\vspace{-0.1em} Предпосылки-соображения для~этого предположения таковы.

Первое~--- для случая упругих (потенциальных) внутренних сил. При~этом ${\variation{\internalwork} = -\variation{\Pi}}$~--- вариация потенциала с~противоположным знаком. Достаточно очевидно, что~$\Pi$ меняется лишь при~деформации.

Второе соображение~--- в~том, что суммарный вектор и~суммарный момент внутренних сил равен нулю

\begin{equation*}
\sum \ldots
\end{equation*}

...

Принимая~\eqref{assumptionforvirtualwork} и~подставляя в~\eqref{principleofvirtualwork} сначала ${\variation{\bm{r}_k} \hspace{-0.16ex} = \hspace{-0.08ex} \varvector{\hspace{-0.1ex}\bm{\rho}}}$ (трансляция), а~затем ${\variation{\bm{r}_k} \hspace{-0.16ex} = \hspace{-0.08ex} \varvector{\bm{o}} \hspace{-0.2ex} \times \hspace{-0.1ex} \bm{r}_k}$ (поворот), получаем баланс импульса~(...) и баланс момента импульса~(...).

...



\end{otherlanguage}

\en{\section{Hamilton’s principle and Lagrange’s equations}}
\ru{\section{Принцип Гамильтона и уравнения Лагранжа}}

\begin{otherlanguage}{russian}

Вариационное уравнение~\eqref{principleofvirtualwork} удовлетворяется в~любой момент времени. Интегрируя его по~какому\hbox{-}либо промежутку

...

Известны уравнения Lagrange’а не~только второго, но~и~первого рода. Рассмотрим их ради методики вывода, много раз применяемой в~этой книге.

При~наличии связей~\eqref{holonomicconstraint} равенство ${\bm{F}_k \hspace{-0.12ex} = m_k \mathdotdotabove{\bm{r}}_k}$ не~следует из~вариационного уравнения~\eqref{principleofvirtualwork}, поскольку виртуальные перемещения~${\variation{\hspace{.1ex}\bm{r}_k}}$ не~являются независимыми. Каждое из $m$ ($m$~--- число связей) условий для~вариаций~\eqref{requirementforvirtualdisplacements} умножим на~некий скаляр~$\lambda_{\alpha}$ (${\alpha = 1, \ldots, m}$) и~добавим к~\eqref{principleofvirtualwork}:
\begin{equation}
\displaystyle \sum_{k=1}^{N} \hspace{-0.25ex} \left(^{\mathstrut} \hspace{-0.16ex} \bm{F}_k \right. \hspace{-0.32ex} + \hspace{-0.05ex}
\scalebox{0.88}[0.92]{$\displaystyle \sum_{\alpha}$} \hspace{.12ex} \lambda_{\alpha} \hspace{.2ex} \scalebox{0.92}{$\displaystyle \frac{\raisemath{-0.12em}{\partial \hspace{.1ex} c_{\alpha}}}{\partial \hspace{.1ex} \bm{r}_k}$}
- \left. \hspace{-0.25ex} m_k \mathdotdotabove{\bm{r}}_k ^{\mathstrut}\right) \hspace{-0.32ex} \dotp \variation{\hspace{.1ex}\bm{r}_k} \hspace{-0.16ex} = 0 \hspace{0.1ex},
\end{equation}

\vspace{-0.2em} \noindent Среди $3N$ вариаций компонент~${\variation{\hspace{.1ex}\bm{r}_k}}$ имеем $m$ зависимых. Но столько~же и~множителей Лагранжа: подберём $\lambda_{\alpha}$ так, чтобы коэффициенты(??как\'{и}е?) при~зависимых вариациях обратились в~нуль. Но при~остальных вариациях коэффициенты(??) также должны быть нулями из\hbox{-}за независимости. Следовательно, все выражения в~скобках~${(\cdots\hspace{-0.2ex})}$ равны нулю~--- это и~есть уравнения Лагранжа первого рода.

Поскольку для каждой частицы

...



\end{otherlanguage}

\en{\section{Statics}}
\ru{\section{Статика}}

\begin{otherlanguage}{russian}

Рассмотрим систему со~стационарными (постоянными во~времени) связями

...



\end{otherlanguage}

\en{\section{Mechanics of relative motion}}
\ru{\section{Механика относительного движения}}

\begin{otherlanguage}{russian}

До~сих~пор не~ставился вопрос о~системе отсчёта

...



\end{otherlanguage}

\en{\section{Small oscillations}}
\ru{\section{Малые колебания}}
\label{para:smalloscillations}

\begin{otherlanguage}{russian}

Если статика линейно\hbox{-}упругой системы описывается

...


\end{otherlanguage}

\section*{\small \wordforbibliography}

\begin{changemargin}{\parindent}{0pt}
\fontsize{10}{12}\selectfont

\begin{otherlanguage}{russian}

В~длинном списке книг по~общей механике можно найти труды не~только механиков\hbox{-}профессионалов~\cite{goldstein-classicalmechanics, loitsjanskiy.lurie, lurie-analyticalmechanics, olkhovskiy-theoreticalmechanicsforphysicists, treatiseonanalyticaldynamics-by-l.a.pars}, но~и физиков\hbox{-}теоретиков широкой ориентации~\cite{landau.lifshitz-shortcourse, terhaar-hamiltonianmechanics}. Рекомендую курс Ф.\,Р.\;Гантмахера~\cite{gantmacher} с~компактным, но~полным изложением осн\'{о}в.

\end{otherlanguage}

\end{changemargin}
