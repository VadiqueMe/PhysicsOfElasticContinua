\en{\chapter{Fundamentals of generic mechanics}}

\ru{\chapter{Основы общей механики}}

\thispagestyle{empty}

\label{chapter:genericmechanics}

\en{\section{Discrete collection of particles}} %% Initial concepts. Discrete approach

\ru{\section{Дискретная совокупность частиц}} %% Исходные представления. Дискретный подход

\label{para:initialconcepts.discreteapproach}

\en{\lettrine[lines=2, findent=2pt, nindent=0pt]{G}{eneric}}\ru{\lettrine[lines=2, findent=2pt, nindent=0pt]{О}{бщая}} \en{mechanics}\ru{механика} \en{models}\ru{моделирует} \en{physical objects}\ru{физические объекты}\en{ by}\ru{,} \en{discretizing them}\ru{дискретизируя их} \en{into }\ru{в~}\en{a~collection of~particles}\ru{совокупность частиц} (\inquotesx{\en{pointlike masses}\ru{точеч\-ных масс}}[,] \inquotes{\en{material points}\ru{материальных точек}}%
\footnote{\en{Point mass}\ru{Точечная масса} (\en{pointlike mass, }\en{material point}\ru{материальная точка})\en{ is}\ru{\:--- это} \en{the~concept}\ru{концепт} \en{of an~object}\ru{объекта}, \en{typically}\ru{типично} \en{matter}\ru{материи}, \en{that}\ru{который} \en{has}\ru{имеет} \en{nonzero mass}\ru{ненулевую массу} \en{and}\ru{и}~\en{is}\ru{является}~(\en{or}\ru{или} \en{is being thought of as}\ru{мыслится}) \en{infinitesimal}\ru{бесконечно-малым} \en{in~its}\ru{по~своем\'{у}} \en{volume}\ru{объёму}~(\en{dimensions}\ru{размерам}).%
}).
\en{In}\ru{В}~\en{a~collection}\ru{совокупности} \en{of }${N\hspace{-0.25ex}}$~\en{particles,}\ru{\hbox{частиц}} \en{each}\ru{каждая} $k$\hbox{-}\en{th}\ru{ая} \en{particle}\ru{частица} \en{has its nonzero mass}\ru{имеет свою ненулевую массу}~${m_k \hspace{-0.25ex} = \hspace{-0.1ex} \constant > \hspace{-0.1ex} 0}$ \en{and}\ru{и}~\en{motion function}\ru{функцию движения}~${\bm{r}_k(t)}$.
\en{Function}\ru{Функция}~${\bm{r}_k(t)}$ \en{is measured}\ru{измеряется} \en{relative to some reference system}\ru{отно\-си\-тель\-но н\'{е}которой системы отсчёта}\:--- \en{a~solid body}\ru{твёрдого тела} \en{with a~clock}\ru{с~часами}~(\figref{fig:referencesystem}).

\begin{wrapfigure}[8]{r}{.4\textwidth}
\makebox[.42\textwidth][c]{\begin{minipage}[t]{.43\textwidth}
\vspace{-1.55em}
\scalebox{1.1}{
\begin{tikzpicture}[scale=0.86]

	\draw[line width=1.2pt, black] (0,0) -- (2.8,0);
	\draw[line width=1.2pt, black] (0,0) -- (-1.8,0);
	\foreach \xground in {-1.6, -1.33, ..., 2.9}
		\draw [line width=0.4pt, black!80] (\xground,0) -- (\xground-0.2,-0.2);

	\def\clockat{-0.93}
	\path (\clockat, 0) node [shape=coordinate] (clocktower) {};
	\def\clockradius{0.4}
	\def\clocksquare{\clockradius + 0.1}
	\def\clockbase{\clockradius - 0.05}
	\def\clockheight{0.8}
	\path (\clockat, \clockheight+\clocksquare) node [shape=coordinate] (clock) {};

	\draw [line width=1.2pt, black] ($ (clocktower) + (\clockbase,0) $) -- ++(up:\clockheight);
	\draw [line width=1.2pt, black] ($ (clocktower) - (\clockbase,0) $) -- ++(up:\clockheight);
	\draw [line width=1.2pt, black, rounded corners=1.2pt] ($ (clock) + (\clocksquare,\clocksquare) $) rectangle ($ (clock) - (\clocksquare,\clocksquare) $);
	\draw [line width=1.2pt, black] ($ (clock) + (\clockbase,\clocksquare) $) arc(0:180:\clockbase);

	\draw [line width=1.2pt, blue] (clock) circle(\clockradius);
	\draw [line width=1.2pt, blue, rotate around={33:(clock)}] (clock) -- ++(\clockradius - 0.1, 0);
	\draw [line width=1.2pt, blue, rotate around={148:(clock)}] (clock) -- ++(\clockradius - 0.15, 0);

	\path (0, 0.33) node [shape=coordinate] (O) {};
	\path (O) -- ++(2.3, 0) node [shape=coordinate] (first) {};
	\path (O) -- ++(-1, -1) node [shape=coordinate] (second) {};
	\path (O) -- ++(0, 1.83) node [shape=coordinate] (third) {};

	\tikzstyle{basis vector} =
		[line width=1pt, blue, style=double, double distance=0.5mm, -{Triangle[open, angle=60:3.2mm]}]
	\draw [basis vector] (O) -- (first)
		node [pos=0.88, above, inner sep=1.33ex, outer sep=0] {$\bm{e}_1$};
	\draw [basis vector] (O) -- (second)
		node [pos=0.71, below right, inner sep=0.83ex, outer sep=0] {$\bm{e}_2$};
 	\draw [basis vector] (O) -- (third)
		node [pos=0.86, right, inner sep=1.33ex, outer sep=0] {$\bm{e}_3$};

	\path (2.44,1.7) node [shape=coordinate] (m) {};

	\path (m) node [shape=circle, inner sep=1mm, outer sep=0] (mcirc) {};

	\draw [line width=1.6pt, black, -{Stealth[round, length=5mm, width=3.6mm]}] (O) -- (mcirc)
		node [pos=0.5, above] {$\bm{r}$} ;

	\draw [line width=1.6pt, black, fill=black!50] (m) circle (1.6mm) ;

	\path (mcirc) node [xshift=-3.2mm, yshift=3mm] {$m$} ;

	\draw [line width=1pt, blue, fill=white] (O) circle (1.2mm) ;

\end{tikzpicture}}
\vspace{-0.9em}\caption{}\label{fig:referencesystem}
\end{minipage}}
\end{wrapfigure}

\en{Long time ago}\ru{Когда\hbox{-}то давно}, \en{the~reference system}\ru{системой отсчёта} \en{was}\ru{было} \en{some absolute space}\ru{некое абсолютное пространство}: \en{empty at~first}\ru{сначала пустое}, \en{and then}\ru{а~затем} \en{filled with the~continuous elastic medium}\ru{заполненное сплошной упругой \hbox{средой}}\:--- \en{the~ether}\ru{эфиром}.
\en{Later it became clear}\ru{Позже стало ясно,} \en{that}\ru{что} \en{in classical mechanics}\ru{в~классической механике} \ru{могут быть использованы }\en{any reference systems}\ru{любые системы отсчёта}\en{ can be used}, \en{but}\ru{но} \en{the~preference is given to}\ru{предпочтение отдаётся} \en{so~called}\ru{так называемым} \inquotes{\en{inertial}\ru{инерциальным}} \en{systems}\ru{системам}, \en{where}\ru{где} \en{a~point}\ru{\hbox{точка}} \en{moves without acceleration}\ru{движется без~ускорения}~(${\mathdotdotabove{\bm{r}} = \bm{0}}$) \en{in the~absence of external interactions}\ru{в~отсутствии внешних взаимодействий}.

\en{The measure of interaction in mechanics is a~vector of~force}\ru{Мерой взаимодействия в~механике является вектор силы}~${\bm{F}\hspace{-0.2ex}}$.
\en{In}\ru{В}~\en{the~widely known}\ru{широко известном} \ru{уравнении }Newton’\en{s}\ru{а}\en{ equation}

\nopagebreak\vspace{-0.25em}\begin{equation}\label{law:ofnewton}
m \hspace{.2ex} \mathdotdotabove{\bm{r}} \hspace{.12ex} = \bm{F} ( \bm{r}, \mathdotabove{\bm{r}}, t )
\end{equation}

\vspace{-0.25em} \noindent \en{the~right\hbox{-}hand side}\ru{правая часть} \en{can depend only on}\ru{может зависеть лишь от} \en{position}\ru{положения}, \en{velocity}\ru{скорости} \en{and}\ru{и}~\en{explicitly presented time}\ru{явно входящего времени}, \en{whereas acceleration}\ru{тогда как ускорение}~$\mathdotdotabove{\bm{r}}$ \en{is directly proportional to~force}\ru{прямо пропорционально силе}~$\bm{F}$ \en{with coefficient}\ru{с~коэффициентом}~${\raisemath{-0.1em}{\scalebox{1.2}{$\nicefrac{1}{m}$}}\hspace{.1ex}}$.

\en{Here’re}\ru{Вот} \en{theses}\ru{тезисы} \en{of dynamics of a~collection of particles}\ru{динамики совокупности частиц}.

\begin{wrapfigure}[16]{r}{.55\textwidth}
\makebox[.58\textwidth][c]{\begin{minipage}[t]{.58\textwidth}
\vspace{-0.7em}\ru{\vspace{-1em}}

\begin{tikzpicture}[scale=1.05]

% arguments: name, x, y, radius
\newcommand{\setpointmass}[4]{%
\path (#2, #3) node [shape=coordinate] (#1) {} ;
\def\linewidth{1.6pt}
\pgfmathsetmacro\radiusoffset{#4 - \linewidth}
\path (#1) node [line width=1pt, minimum size=#4, circle, inner sep=\radiusoffset, outer sep=0] (#1circle) {} ;
%%\draw [line width=\linewidth, black, fill=black!50] (#1) circle (#4) ;
%%\path (#1circle) node [black, above left, inner sep=6pt, outer sep=0] {#1} ;
}

% arguments: name, radius, color
\newcommand{\drawpointmass}[3]{%
\def\linewidth{1.6pt}
\draw [line width=\linewidth, #3, fill=#3!50] (#1) circle (#2) ;
}

% arguments: name, color, node options, text
\newcommand{\labelpointmass}[4]{%
\path (#1circle) node [#2, #3] {#4} ;
}

% arguments: name of point, name of force, vector length, vector angle in degrees, color
\newcommand{\forceatpoint}[5]{%

\tikzstyle{force line} =
	[line width=1.25pt, line cap=round, -{Triangle[round, length=4.2mm, width=2.7mm]}]

% (from)!length!angle:(to)
\path ($ (#1)!#3!#4:($ (#1) + (0, #3) $) $) node [shape=coordinate] (#1#2force) {} ;

\draw [force line, #5] (#1#2force) -- (#1circle) {} ;
}

% arguments: name of point, name of other point, name of force, force vector length, color
\newcommand{\forcebetweenpoints}[5]{%

\tikzstyle{force line} =
	[line width=1.25pt, line cap=round, -{Triangle[round, length=4.2mm, width=2.7mm]}]

\path ($ (#1)!#4!0:(#2) $) node [shape=coordinate] (#1#3force) {} ;

\draw [force line, #5] (#1#3force) -- (#1circle) {} ;
}

% arguments: name of point, name of force, position, node options, text
\newcommand{\labelforceatpoint}[5]{%
\node at ($ (#1#2force)!#3!(#1) $) [#4] {#5} ;
}

\tikzset{%
radiiline/.style={line cap=round, dash pattern=on 0pt off 1.6\pgflinewidth, -{Stealth[round, length=3.8mm, width=2.7mm]}}%
}

\def\mthirdcolor{green!77!yellow!88!black}

\setpointmass{m1}{35mm}{-28mm}{2mm}
\setpointmass{m2}{27mm}{19mm}{2mm}
\setpointmass{m3}{13mm}{-13mm}{2mm}

\path (63mm, 8mm) node [shape=coordinate] (O) {} ;

% draw position vectors

\draw [radiiline, line width=1.2pt, black!55] (O) -- (m1circle)
	node [red, pos=0.56, below right, inner sep=3pt, outer sep=0] {$\bm{r}_1$} ;

\draw [radiiline, line width=1.2pt, black!55] (O) -- (m2circle)
	node [magenta, pos=0.54, above, inner sep=4.7pt, outer sep=0] {$\bm{r}_2$} ;

\draw [radiiline, line width=1.2pt, black!55] (m2circle) -- (m1circle)
	node [black, pos=0.55, above right, inner sep=3.3pt, outer sep=0] {$\mathcolor{red}{\bm{r}_1} \hspace{-0.25ex} - \mathcolor{magenta}{\bm{r}_2}$} ;

\draw [radiiline, line width=1.2pt, black!55] (m3circle) -- (m2circle)
	node [black, pos=0.47, above left, inner sep=2.6pt, outer sep=0] {$\mathcolor{magenta}{\bm{r}_2} \hspace{-0.25ex} - \mathcolor{\mthirdcolor}{\bm{r}_3}$} ;

\draw [radiiline, line width=1.2pt, black!55] (m3circle) -- (m1circle)
	node [black, pos=0.49, below left, inner sep=2.1pt, outer sep=0] {$\mathcolor{red}{\bm{r}_1} \hspace{-0.25ex} - \mathcolor{\mthirdcolor}{\bm{r}_3}$} ;

% draw force vectors

\forceatpoint{m1}{ext}{10mm}{263}{black}
\labelforceatpoint{m1}{ext}{0.22}{below, outer sep=6.3pt, inner sep=0}{${\bm{F}^{\smthexternal}_{\hspace{-0.15ex}1}}$}

\forceatpoint{m2}{ext}{13.5mm}{103}{black}
\labelforceatpoint{m2}{ext}{0.17}{above, outer sep=3.3pt, inner sep=0}{${\bm{F}^{\smthexternal}_{\hspace{-0.15ex}2}}$}

\forceatpoint{m3}{ext}{8.8mm}{22}{black}
\labelforceatpoint{m3}{ext}{0.34}{left, outer sep=4.7pt, inner sep=0}{${\bm{F}^{\smthexternal}_{\hspace{-0.15ex}3}}$}

\forcebetweenpoints{m1}{m2}{int12}{16mm}{magenta}{-2mm}
\labelforceatpoint{m1}{int12}{0.27}{magenta, right, outer sep=3pt, inner sep=0}{${\bm{F}^{\smthinternal}_{\raisebox{-0.1em}{$\scriptstyle \hspace{-0.25ex}12$}}}$}

\forcebetweenpoints{m2}{m1}{int21}{16mm}{red}{2mm}
\labelforceatpoint{m2}{int21}{0.3}{red, right, outer sep=3pt, inner sep=0}{${\bm{F}^{\smthinternal}_{\raisebox{-0.1em}{$\scriptstyle \hspace{-0.25ex}21$}}}$}

\forcebetweenpoints{m3}{m2}{int32}{12.5mm}{magenta}
\labelforceatpoint{m3}{int32}{0.38}{magenta, right, outer sep=4pt, inner sep=0}{${\bm{F}^{\smthinternal}_{\raisebox{-0.1em}{$\scriptstyle \hspace{-0.25ex}32$}}}$}

\forcebetweenpoints{m2}{m3}{int23}{12.5mm}{\mthirdcolor}
\labelforceatpoint{m2}{int23}{0.63}{\mthirdcolor, below left, outer sep=7.2pt, inner sep=0}{${\bm{F}^{\smthinternal}_{\raisebox{-0.1em}{$\scriptstyle \hspace{-0.25ex}23$}}}$}

\forcebetweenpoints{m1}{m3}{int13}{10.5mm}{\mthirdcolor}
\labelforceatpoint{m1}{int13}{0.37}{\mthirdcolor, below left, outer sep=2.5pt, inner sep=0}{${\bm{F}^{\smthinternal}_{\raisebox{-0.1em}{$\scriptstyle \hspace{-0.25ex}13$}}}$}

\forcebetweenpoints{m3}{m1}{int31}{10.5mm}{red}
\labelforceatpoint{m3}{int31}{0.28}{red, above right, outer sep=0.8pt, inner sep=0}{${\bm{F}^{\smthinternal}_{\raisebox{-0.1em}{$\scriptstyle \hspace{-0.25ex}31$}}}$}

% draw points

\drawpointmass{m1}{1.8mm}{red}
\labelpointmass{m1}{red}{below left, yshift=-2pt, inner sep=4.7pt, outer sep=0}{$m_1$}

\drawpointmass{m2}{1.8mm}{magenta}
\labelpointmass{m2}{magenta}{above left, xshift=1.4pt, inner sep=6.9pt, outer sep=0}{$m_2$}

\drawpointmass{m3}{1.8mm}{\mthirdcolor}
\labelpointmass{m3}{\mthirdcolor}{left, yshift=-2pt, inner sep=9pt, outer sep=0}{$m_3$}

\draw [line width=1.2pt, blue, fill=white] (O) circle (1mm) ;

\end{tikzpicture}

\vspace{-0.6em}\caption{}\label{fig:particlesandforces}
\end{minipage}}
\end{wrapfigure}

\en{Force acting on $k$\hbox{-}th particle}\ru{Сила, действующая на $k$\hbox{-}ую частицу}~(\figref{fig:particlesandforces})

\nopagebreak\vspace{-0.1em}\begin{equation}\label{forceonparticle}
\bm{F}_k \hspace{-0.2ex} = \bm{F}^{\smthexternal}_{\hspace{-0.16ex}k} \hspace{-0.15ex}
+ \scalebox{0.88}{$\displaystyle \underset{\raisemath{.25ex}{\smash{j}}}{\sum}$} \hspace{.2ex} \bm{F}^{\smthinternal}_{\hspace{-0.16ex}kj} \hspace{-0.12ex},
\end{equation}

\vspace{-0.25em}\noindent
\en{where}\ru{где} \en{the first addend}\ru{первое слагаемое}~${\bm{F}^{\smthexternal}_{\hspace{-0.16ex}k}}$\en{ is}\ru{\:--- это} \en{external force}\ru{внешняя сила}, \en{and }\ru{а~}\en{the second}\ru{второе}\en{ is}\ru{\:---} \en{sum of~internal ones}\ru{сумма внутренних} (${\bm{F}^{\smthinternal}_{\hspace{-0.16ex}kj}}$ \en{is}\ru{есть} \en{force from particle}\ru{сила от~частицы} \en{with number}\ru{с~номером}~\inquotes{$\hspace{-0.1ex}j\hspace{.25ex}$}).

\en{From}\ru{Из}~\eqref{law:ofnewton} \en{and}\ru{и}~\eqref{forceonparticle}
\en{together with }\ru{вместе с~}\en{the~action--re\-action principle}\ru{принципом действия--про\-ти\-во\-действия}

\nopagebreak\vspace{-0.12em}\begin{equation}\label{actionreactionprinciple}
\bm{F}^{\smthinternal}_{\hspace{-0.16ex}kj} = - \hspace{.12ex} \bm{F}^{\smthinternal}_{\hspace{-0.4ex}j\hspace{-0.05ex}k}
\hspace{-0.16ex} ,
\end{equation}

\vspace{-0.15em}\noindent
\en{ensues}\ru{вытекает} \en{the~balance of~momentum}\ru{баланс импульса}

\nopagebreak\ru{\vspace{-0.1em}}\begin{equation}\label{balanceoftranslationalmomentum.discrete}
\biggl( \displaystyle \sum_{\smash{k}} m_k \hspace{.2ex} \mathdotabove{\bm{r}}_k \hspace{-0.2ex} \biggr)^{\hspace{-0.2em}\tikz[baseline=-0.2ex]\draw[black, fill=black] (0,0) circle (.28ex);} \hspace{-0.15ex}
= \hspace{.12ex} \displaystyle \sum_{\smash{k}} m_k \hspace{.2ex} \mathdotdotabove{\bm{r}}_k
= \displaystyle \sum_{\smash{k}} \bm{F}^{\smthexternal}_{\hspace{-0.16ex}k}
\hspace{-0.15ex} .
\end{equation}

\vspace{-0.1em}
\en{Assuming also that}\ru{Полагая также, что} \en{internal interactions}\ru{внутренние взаимодействия} \en{between particles}\ru{между частицами} \en{are central}\ru{центральны}, \en{that~is}\ru{то~есть}

\nopagebreak\vspace{-0.25em}\begin{equation*}
\bm{F}^{\smthinternal}_{\hspace{-0.16ex}kj} \hspace{.15ex} \parallel \hspace{.1ex} \bigl( \hspace{.1ex} \bm{r}_k \hspace{-0.15ex} - \bm{r}_{\hspace{-0.16ex}j} \bigr)
\hspace{.33em} \Leftrightarrow \hspace{.33em}
\bigl( \hspace{.1ex} \bm{r}_k \hspace{-0.15ex} - \bm{r}_{\hspace{-0.16ex}j} \bigr) \hspace{-0.15ex} \times \bm{F}^{\smthinternal}_{\hspace{-0.16ex}kj} = \bm{0}
\hspace{.1ex} ,
\end{equation*}

\vspace{-0.2em}\noindent
\en{we come to }\ru{приходим к~}\en{the~balance of~rotational momentum}\ru{балансу момента импульса}

\nopagebreak\vspace{-0.1em}\begin{equation}\label{balanceofrotationalmomentum.discrete}
\biggl( \displaystyle \sum_{\smash{k}} \bm{r}_k \hspace{-0.12ex} \times m_k \hspace{.2ex} \mathdotabove{\bm{r}}_k \hspace{-0.2ex} \biggr)^{\hspace{-0.2em}\tikz[baseline=-0.2ex]\draw[black, fill=black] (0,0) circle (.28ex);} \hspace{-0.15ex}
= \hspace{.16ex}
\displaystyle \sum_{\smash{k}} \bm{r}_k \hspace{-0.12ex} \times \bm{F}^{\smthexternal}_{\hspace{-0.16ex}k}
\hspace{-0.15ex} .
\end{equation}

...


\newpage

\ru{\section{Совершенно жёсткое недеформируемое твёрдое тело}}

\en{\section{Absolutely rigid undeformable solid body}}

%% \inquotesx{Абсолютно твёрдое}[,] оно~же \inquotes{абсолютно жёсткое} и~\inquotesx{абсолютно прочное}[---] это несбыточная мечта любого инженера.

\en{The distance}\ru{Расстояние} \en{between}\ru{между} \en{any two points}\ru{любыми двумя точками} \en{of an~absolutely rigid solid body}\ru{совершенно~(абсолютно) жёсткого твёрдого тела} \en{remains constant}\ru{остаётся постоянным}, \en{regardless of external forces exerted on it}\ru{независимо от действующих на~него внешних сил}: \en{there’s no deformation}\ru{деформации тут нет}.

%% \begin{comment} %%
{\small
\noindent\leavevmode{\indent} The difference between \inquotes{rigid} and~\inquotes{solid} is that rigid is stiff (not flexible) while solid is in the state of a solid (not fluid). Solid is a~substance in the fundamental state of matter that retains its size and shape without need of container (as opposed to a~fluid).

\noindent\leavevmode{\indent} An~absolutely rigid body is a~solid body in which deformation is zero (or negligibly small, so small it can be neglected).
\par}
%% \end{comment} %%

\en{To define position (location)}\ru{Для определения положения (места)} \en{of an~absolutely rigid undeformable body}\ru{совершенно жёсткого недеформируемого тела} \en{it’s enough}\ru{достаточно} \en{to choose}\ru{выбрать} \en{some one of its points}\ru{какую\hbox{-}либо одну его точку}\:--- \inquotesx{\en{the~pole}\ru{полюс}}[,] \en{to~set location}\ru{установить положение}~${\bm{r}(t)}$ \en{of~this point}\ru{этой точки}, \en{as well as angular orientation}\ru{а~также угловую ориентацию} \en{of a~body}\ru{тела}~(\figref{fig:bodyoffsetandrotation}).

\begin{comment}
\makeatletter
\newcommand\xofcoordinate[2][center]{{%
	\pgfpointanchor{#2}{#1}%
	\pgfmathparse{\pgf@x/\pgf@xx}%
	\pgfmathprintnumber[precision=2]{\pgfmathresult}%
}}
\newcommand\yofcoordinate[2][center]{{%
	\pgfpointanchor{#2}{#1}%
	\pgfmathparse{\pgf@y/\pgf@yy}%
	\pgfmathprintnumber[precision=2]{\pgfmathresult}%
}}
\makeatother

\begin{tikzpicture}
	\coordinate (point0) at (-4.3, 2.5);
	\coordinate (point1) at (-3.1, 3.2);
	\coordinate (point2) at (-2, 2.4);
	\coordinate (point3) at (-0.4, 1.6);
	\coordinate (point4) at (0.5, 0);
	\coordinate (point5) at (0, -2);
	\coordinate (point6) at (-1.5, -3);
	\coordinate (point7) at (-3, -2.2);
	\coordinate (point8) at (-3.5, -0.5);
	\coordinate (point9) at (-4.5, 1);

	\draw [line width=1.2pt, red] plot [smooth cycle, tension=0.8] coordinates {
		(point0) (point1) (point2) (point3) (point4)
		(point5) (point6) (point7) (point8) (point9)
	};

	\newcommand\xyofcoordinate[1]{\xofcoordinate{#1},\,\yofcoordinate{#1}}

	\draw [black, fill=black] (point0) circle (1mm) node [anchor=south east] {\xyofcoordinate{point0}};
	\draw [black, fill=black] (point1) circle (1mm) node [anchor=south, outer sep=4pt] {\xyofcoordinate{point1}};
	\draw [black, fill=black] (point2) circle (1mm) node [anchor=south west] {\xyofcoordinate{point2}};
	\draw [black, fill=black] (point3) circle (1mm) node [anchor=south west] {\xyofcoordinate{point3}};
	\draw [black, fill=black] (point4) circle (1mm) node [anchor=south west] {\xyofcoordinate{point4}};

	\draw [black,fill=black] (point5) circle (1mm) node [anchor=north west, outer sep=1pt] {\xyofcoordinate{point5}};
	\draw [black,fill=black] (point6) circle (1mm) node [anchor=north, outer sep=4pt] {\xyofcoordinate{point6}};
	\draw [black,fill=black] (point7) circle (1mm) node [anchor=north east, outer sep=2pt] {\xyofcoordinate{point7}};
	\draw [black,fill=black] (point8) circle (1mm) node [anchor=east, outer sep=4pt] {\xyofcoordinate{point8}};
	\draw [black,fill=black] (point9) circle (1mm) node [anchor=east, outer sep=3pt] {\xyofcoordinate{point9}};
\end{tikzpicture}
\end{comment}

\begin{wrapfigure}[13]{o}{.5\textwidth}
\makebox[.45\textwidth][c]{\begin{minipage}[t]{.45\textwidth}
\vspace{-0.8em}
\scalebox{1.1}{
\begin{tikzpicture}[scale=0.6]

	\def\angleofrotation{36}

	\coordinate (O) at (-1.65, -1.05);
	\path (O) circle (1.6mm) node [shape=circle, inner sep=.64mm, outer sep=0] (Ocirc) {};

	\coordinate (Oinitial) at (-6, -2.5);

	\coordinate (bodypoint) at (-2, 1.5);

	\coordinate (point0) at (-4.3, 2.5);
	\coordinate (point1) at (-3.1, 3.2);
	\coordinate (point2) at (-2, 2.4);
	\coordinate (point3) at (-0.4, 1.6);
	\coordinate (point4) at (0.5, 0);
	\coordinate (point5) at (0, -2);
	\coordinate (point6) at (-1.5, -3);
	\coordinate (point7) at (-3, -2.2);
	\coordinate (point8) at (-3.5, -0.5);
	\coordinate (point9) at (-4.5, 1);

	%\draw [line width=1pt, blue!25,
		%style=double, double distance=0.5mm, -{Triangle[open, angle=60:3.2mm]}] (O) -- ++(-1.386,-0.8);
	%\draw [line width=1pt, blue!25,
		%style=double, double distance=0.5mm, -{Triangle[open, angle=60:3.2mm]}] (O) -- ++(1.386,-0.8);
 	%\draw [line width=1pt, blue!25,
		%style=double, double distance=0.5mm, -{Triangle[open, angle=60:3.2mm]}] (O) -- ++(0,1.6);

	\begin{scope}[rotate around={-\angleofrotation:(O)}]
	\draw [line width=1pt, black!50, opacity=50]
		plot [smooth cycle, tension=0.8] coordinates {
			(-4.3, 2.5) (-3.1, 3.2) (-2, 2.4) (-0.4, 1.6) (0.5, 0)
			(0, -2) (-1.5, -3) (-3, -2.2) (-3.5, -0.5) (-4.5, 1)
		};

	\path (-2, 1.5) circle (2mm) node [shape=circle, inner sep=.9mm, outer sep=0] (previousbodypoint) {};

	\tkzDrawArc[line width=.8pt, color=black!50, opacity=50](O,previousbodypoint)(bodypoint);

	\draw [line width=1pt, black!50, opacity=50, -{Stealth[round, length=4.5mm, width=2.8mm]}]
		(O) -- (previousbodypoint)
		node [pos=0.75, color=black!50, opacity=99, right, inner sep=2pt, outer sep=4pt] {$\mathcircabove{\bm{x}}$} ;

	\fill [white] (-2, 1.5) circle (2mm) ;
	\draw [line width=1pt, black!50, opacity=50] (-2, 1.5) circle (2mm) ;
	\end{scope}

	\draw [line width=1.6pt, black]
		plot [smooth cycle, tension=0.8] coordinates {
			(point0) (point1) (point2) (point3) (point4)
			(point5) (point6) (point7) (point8) (point9)
		};

	\draw [line width=1.6pt, black, fill=white] (bodypoint) circle (2mm)
		node [shape=circle, inner sep=0.9mm, outer sep=0] (pointcirc) {};

	\draw [line width=1.6pt, black, -{Stealth[round, length=5mm, width=3.6mm]}] (Oinitial) -- (pointcirc)
		node [pos=0.42, above, xshift=-2mm] {$\bm{R}$};

	\draw [line width=1.6pt, blue, -{Stealth[round, length=5mm, width=3.6mm]}] (Oinitial) -- (Ocirc)
		node [pos=0.5, above] {$\bm{r}$};

	\draw [line width=1.6pt, black, -{Stealth[round, length=5mm, width=3.6mm]}] (O) -- (pointcirc)
		node [pos=0.8, right, inner sep=2.5pt, outer sep=4pt] {$\bm{x}$};

	\draw [line width=1pt, blue, rotate around={\angleofrotation:(O)},
		style=double, double distance=0.5mm, -{Triangle[open, angle=60:3.2mm]}] (O) -- ++(-1.386,-0.8);
	\draw [line width=1pt, blue, rotate around={\angleofrotation:(O)},
		style=double, double distance=0.5mm, -{Triangle[open, angle=60:3.2mm]}] (O) -- ++(1.386,-0.8);
 	\draw [line width=1pt, blue, rotate around={\angleofrotation:(O)},
		style=double, double distance=0.5mm, -{Triangle[open, angle=60:3.2mm]}] (O) -- ++(0,1.6);

	\draw [line width=1pt, blue, fill=white] (O) circle (1.6mm)
		node [below, outer sep=1.6mm, xshift=2.4mm] {$\bm{e}_i$};

	\draw [line width=1pt, blue,
		style=double, double distance=0.5mm, -{Triangle[open, angle=60:3.2mm]}] (Oinitial) -- ++(-1.386,-0.8);
	\draw [line width=1pt, blue,
		style=double, double distance=0.5mm, -{Triangle[open, angle=60:3.2mm]}] (Oinitial) -- ++(1.386,-0.8);
 	\draw [line width=1pt, blue,
		style=double, double distance=0.5mm, -{Triangle[open, angle=60:3.2mm]}] (Oinitial) -- ++(0,1.6);

	\draw [line width=1pt, blue, fill=white] (Oinitial) circle(1.6mm)
		node [anchor=north, yshift=-1mm] {$\mathcircabove{\bm{e}}_i$};

\end{tikzpicture}}
\vspace{-1.6em}\caption{}\label{fig:bodyoffsetandrotation}
\end{minipage}}
\end{wrapfigure}

%%\emph{\en{Discrete and continual approaches}\ru{Дискретный и континуальный подходы}}.
\en{An~absolutely rigid body}\ru{Совершенно жёсткое тело} \en{is mostly modeled}\ru{чаще всего моделируется} \en{using}\ru{с~использованием} \inquotesx{\en{the~continual approach}\ru{континуального подхода}}[---] \en{as}\ru{как} \en{a~continuous distribution of~mass}\ru{непрерывное распределение массы} (\en{material continuum}\ru{материальный контину\kern-0.11exум, сплошная среда}), \en{rather than using}\ru{вместо использования} \inquotesx{\en{the~discrete approach}\ru{дискретного подхода}}[---] \en{modeling}\ru{моделирования} \en{as}\ru{как} \en{a~discrete collection}\ru{дискретной совокупности} \en{of~body’s particles}\ru{частиц тела}.

\en{A~formula}\ru{Формула} \en{with summation over discrete points}\ru{с~суммированием по~дискретным точкам} \en{turns}\ru{превращается} \en{into a~formula for a~continuous body}\ru{в~формулу для сплошного тела} \en{by replacing}\ru{путём замены} \en{masses of~particles}\ru{масс частиц} \en{with mass}\ru{на~массу} ${dm \hspace{-0.1ex} = \hspace{-0.15ex} \rho \hspace{.2ex} d\mathcal{V}}$ \en{of~volume element}\ru{элемента объёма}~${d\mathcal{V}}$~($\rho$\ru{\:---}\en{~is}~\en{mass density}\ru{плотность массы}) \en{and}\ru{и}~\en{integrating over the~whole volume of~a~body}\ru{интегрирования по~всему объёму тела}.

\foreignlanguage{russian}{${\mathcircabove{\bm{e}}_i}$\:--- ортонормальная тройка векторов базиса, неподвижная относительно системы отсчёта}

Any movement of an~absolutely rigid body is a~rotation, a~translation\footnote{a translation can be thought of as a rotation with the revolution center at infinity}, or a~combination of both.

\foreignlanguage{russian}{Имея неподвижный базис~${\mathcircabove{\bm{e}}_i}$ и движущийся вместе с~телом базис~${\bm{e}_i}$, ...}

\foreignlanguage{russian}{Если связать с~телом тройку декартовых осей с~ортами~${\bm{e}_i}$ (этот базис движется вместе с~телом), тогда угловая ориентация тела может быть задана тензором поворота ${\rotationtensor \equiv \bm{e}_i \mathcircabove{\bm{e}}_i}$.}

\en{Motion of a~body}\ru{Движение тела} \en{is completely defined}\ru{полностью определяется} \en{by functions}\ru{функциями}~${\bm{r}(t)}$ \en{and}\ru{и}~${\rotationtensor(t)}$.

\foreignlanguage{russian}{Вектор\hbox{-}радиус некоторой точки тела}

\begin{equation*}
\bm{R} = \bm{r} + \bm{x}
\hspace{.1ex} ,
\end{equation*}

${\mathcircabove{\bm{x}} = x_i \hspace{.1ex} \mathcircabove{\bm{e}}_i}$,
${\bm{x} = x_i \hspace{.1ex} \bm{e}_i}$

\eqref{rodriguesrotationformula}, \chapdotpararef{chapter:elementsoftensorcalculus}{para:rotationtensor}

${\bm{x} = \rotationtensor \hspace{-0.15ex} \dotp \hspace{.1ex} \mathcircabove{\bm{x}}}$

\begin{equation*}
\mathdotabove{\bm{R}} = \mathdotabove{\bm{r}} + \mathdotabove{\bm{x}}
\hspace{.1ex} ,
\end{equation*}

\en{For an~absolutely rigid body}\ru{Для совершенно жёсткого тела}, \en{components}\ru{компоненты}~${x_i}$ \en{don’t depend on~time}\ru{не~зависят от~времени}~(${x_i \hspace{-0.16ex} = \constant}$) \en{and}\ru{и}~${\mathdotabove{\bm{x}} = x_i \hspace{.1ex} \mathdotabove{\bm{e}}_i}$

${\mathdotabove{\bm{x}} = \mathdotabove{\rotationtensor} \hspace{-0.15ex} \dotp \hspace{.1ex} \mathcircabove{\bm{x}}}$

${x_i \mathdotabove{\bm{e}}_i \hspace{-0.12ex} = \mathdotabove{\rotationtensor} \hspace{-0.15ex} \dotp x_i \hspace{.1ex} \mathcircabove{\bm{e}}_i
\:\Rightarrow\:
\mathdotabove{\bm{e}}_i \hspace{-0.12ex} = \mathdotabove{\rotationtensor} \hspace{-0.15ex} \dotp \mathcircabove{\bm{e}}_i}$

...

\[
\hspace{-0.4ex} \integral_{\mathcal{V}} \hspace{-0.6ex} \bm{r} \hspace{.1ex} dm = \hspace{.1ex} \bm{r} \hspace{-0.5ex} \integral_{\mathcal{V}} \hspace{-0.6ex} dm = \hspace{.1ex} \bm{r} m
\]

\[
\hspace{-0.4ex} \integral_{\mathcal{V}} \hspace{-0.6ex} \bm{x} \hspace{.1ex} dm = \hspace{.15ex} \bm{\Xi} \hspace{.2ex} m
\hspace{.1ex} , \:\:
\bm{\Xi} \hspace{.1ex} \equiv m^{\hspace{-0.1ex}\expminusone} \hspace{-0.5ex} \integral_{\mathcal{V}} \hspace{-0.6ex} \bm{x} \hspace{.1ex} dm
\]

\en{mass of~a~whole body}\ru{масса всего тела} ${m = \hspace{-0.25ex}\scalebox{1.4}{$\integral$}_{\hspace{-0.55ex}\mathcal{V}} \hspace{.3ex} dm = \hspace{-0.25ex}\scalebox{1.4}{$\integral$}_{\hspace{-0.55ex}\mathcal{V}} \hspace{.3ex} \rho \hspace{.2ex} d\mathcal{V}}$

\en{eccentricity vector}\ru{вектор экцентриситета}\hbox{~\hspace{.2ex}}$\bm{\Xi}$

$\bm{\Xi}$ becomes the~null vector when the~chosen \inquotes{pole} is \en{the~center of~mass}\ru{центр масс} (\en{the~unique point}\ru{уникальная точка} \en{within a~body}\ru{внутри тела})

\[
\bm{\Xi} = \bm{0}
\hspace{.6ex} \Leftrightarrow \hspace{.5ex}
\bm{r} = \mathboldrcursive
\]

\[
\bm{x} = \hspace{-0.1ex} \bm{R} - \bm{r}
, \hspace{1ex}
\bm{\Xi} \hspace{.2ex} m = \hspace{-0.4ex} \integral_{\mathcal{V}} \hspace{-0.4ex} \bigl( \bm{R} - \hspace{-0.1ex} \mathboldrcursive \hspace{.15ex} \bigr) dm = \bm{0}
\hspace{.5ex} \Rightarrow \hspace{.6ex}
\mathboldrcursive = m^{\hspace{-0.1ex}\expminusone} \hspace{-0.5ex} \integral_{\mathcal{V}} \hspace{-0.7ex} \bm{R} \hspace{.25ex} dm
\]

...

\en{Introducing}\ru{Вводя} \en{(pseudo)vector}\ru{(псевдо)вектор} \en{of~angular velocity}\ru{угловой скорости}~$\bm{\omega}$, ...

...

\en{inertia tensor}\ru{тензор инерции}~${\inertiatensor}$

\nopagebreak\[
\inertiatensor \equiv \hspace{-0.5ex} \integral_{\mathcal{V}} \hspace{-0.4ex} \bigl( \bm{x} \narrowdotp \bm{x} \bm{E} - \bm{x} \bm{x} \bigr) \hspace{.1ex} dm
\]

\[
\inertiatensor = \inertiatensorcomponents{ab} \hspace{.1ex} \bm{e}_a \bm{e}_b
\hspace{.1ex} , \hspace{.8ex}
\inertiatensordotabove
= \inertiatensorcomponents{ab} \bigl( \hspace{.1ex} \mathdotabove{\bm{e}}_a \bm{e}_b \hspace{-0.1ex} + \bm{e}_a \mathdotabove{\bm{e}}_b \hspace{.12ex} \bigr) \hspace{-0.33ex}
= \inertiatensorcomponents{ab} \bigl( \hspace{.1ex} \bm{\omega} \hspace{-0.2ex} \times \hspace{-0.2ex} \bm{e}_a \bm{e}_b \hspace{-0.1ex} + \bm{e}_a \hspace{.2ex} \bm{\omega} \hspace{-0.2ex} \times \hspace{-0.2ex} \bm{e}_b \hspace{.12ex} \bigr)
\]

...

\subsection*{Work}

\[
W(\bm{F}\hspace{-0.25ex}, \bm{u}) \hspace{-0.1ex} = \bm{F} \hspace{-0.2ex} \dotp \bm{u}
\]

by \inquotes{product rule}

\nopagebreak\[
d \hspace{.1ex} W \hspace{-0.33ex} = d\bm{F} \hspace{-0.1ex} \dotp \bm{u} \hspace{.12ex} + \bm{F} \hspace{-0.1ex} \dotp d\bm{u}
\]

by definition of full differential

\nopagebreak\[
d \hspace{.1ex} W \hspace{-0.33ex}  = \hspace{.08ex} \displaystyle \frac{\raisemath{-0.2em}{\partial \hspace{.1ex} W}}{\partial \bm{F}} \hspace{-0.1ex} \dotp d \bm{F} + \frac{\raisemath{-0.2em}{\partial \hspace{.1ex} W}}{\partial \bm{u}} \hspace{-0.1ex} \dotp d \bm{u}
\]

${\displaystyle \frac{\raisemath{-0.2em}{\partial \hspace{.1ex} W}}{\partial \bm{F}} \hspace{-0.1ex} = \bm{u}}$,
${\displaystyle \frac{\raisemath{-0.2em}{\partial \hspace{.1ex} W}}{\partial \bm{u}} \hspace{-0.1ex} = \bm{F}}$

...

{\small\setlength{\abovedisplayskip}{2pt}\setlength{\belowdisplayskip}{2pt}

Holonomic constraints are relations between position variables (and possibly time) which can be expressed as
\[ f(q_{1}, q_{2}, q_{3}, \ldots, q_{n}, t) = 0 \]

\noindent where ${\{q_{1},q_{2},q_{3},\ldots,q_{n}\}}$ are $n$ coordinates which describe the system. For example, the motion of a particle constrained to lie on sphere’s surface is subject to a holonomic constraint, but if the particle is able to fall off the sphere under the influence of gravity, the constraint becomes non-holonomic. For the first case the holonomic constraint may be given by the equation: ${r^{2} - a^{2} = 0}$, where $r$ is the distance from the centre of a sphere of radius $a$. Whereas the second non-holonomic case may be given by: ${r^{2} - a^{2} \geq 0}$.

Holonomic constraint depends only on coordinates and time. I\kern-0.12ext does not depend on velocities or any higher time derivative. A~constraint that cannot be expressed as shown above is nonholonomic.

Velocity-dependent constraints like
\[ \displaystyle f(q_{1}, q_{2}, \ldots, q_{n}, {\dot {q}}_{1}, {\dot {q}}_{2}, \ldots, {\dot {q}}_{n}, t) = 0 \]
are mostly not holonomic.
\par}



\en{\section{Principle of virtual work}}

\ru{\section{Принцип виртуальной работы}}

\label{para:virtualworkprinciple.genericmechanics}

\begin{otherlanguage}{russian}

Виртуальным перемещением частицы с~радиусом\hbox{-}вектором~${\bm{r}_k}$ называется вариация~${\variation{\hspace{.1ex}\bm{r}_k}}$\:--- %% н\'{е}которое
любое (неопределённое) бесконечно малое приращение~${\bm{r}_k}$, происходящее мгновенно %% вне~времени~(${\variation{t} \hspace{-0.2ex}\equiv\hspace{-0.2ex} 0}$)
и~совместимое с~ограничениями\hbox{-}связями. %%~(constraints).
Если связей нет, то~есть система свободна, тогда виртуальные перемещения ${\variation{\hspace{.1ex}\bm{r}_k}}$ совершенно любые.

Связи бывают голономные~(holonomic, или геометрические), связывающие только положения~(перемещения)\:--- это функции лишь координат и, возможно, времени

\nopagebreak\vspace{-0.1em}\begin{equation}\label{holonomicconstraint}
c\hspace{.2ex}(\bm{r}, t) = 0
\end{equation}

\vspace{-0.12em} \noindent --- и~неголономные~(или дифференциальные), содержащие производные координат по~времени: ${c\hspace{.2ex}(\bm{r}, \mathdotabove{\bm{r}}, t) = 0}$ и~не~интегрируемые до~геометрических связей.

Далее рассматриваем системы, все связи в~которых\:--- голономные. В~системе с~голономной связью виртуальные перемещения должны удовлетворять уравнению

\nopagebreak\vspace{-0.1em}\begin{equation}\label{requirementforvirtualdisplacements}
\displaystyle \sum_k \scalebox{0.92}{$\displaystyle \frac{\raisemath{-0.12em}{\partial \hspace{.1ex} c}}{\partial \hspace{.1ex} \bm{r}_k}$} \hspace{-0.1ex} \dotp \variation{\hspace{.1ex}\bm{r}_k} \hspace{-0.1ex} = 0 \hspace{.16ex}.
\vspace{-0.25em}\end{equation}

В~несвободных системах все силы делятся на две группы: активные и~реакции связей. Реакция~$\mathboldN_k$ действует со~стороны всех материальных ограничителей на~частицу \inquotes{${\hspace{.05ex}k\hspace{.25ex}}$} и~меняется в~соответствии с~уравнением~\eqref{holonomicconstraint} каждой связи. Принимается предложение об~идеальности связей:

\nopagebreak\begin{equation}\label{idealconstraints}
\scalebox{0.9}{$\displaystyle \sum_k$} \hspace{.2ex} \mathboldN_k \hspace{-0.2ex} \dotp \variation{\hspace{.1ex}\bm{r}_k} \hspace{-0.16ex} = 0
\quad \textrm{---}
\vspace{-0.25em}\end{equation}
\noindent работа реакций на~любых виртуальных перемещениях равна нулю.

Принцип виртуальной работы выражается уравнением

\nopagebreak\vspace{-0.1em}\begin{equation}\label{discrete:principleofvirtualwork}
\displaystyle \sum_{\smash{k}} \hspace{-0.2ex} \left(^{\mathstrut} \hspace{-0.25ex} \bm{F}_k \hspace{-0.12ex} - m_k \mathdotdotabove{\bm{r}}_k \right) \hspace{-0.32ex} \dotp \variation{\hspace{.1ex}\bm{r}_k} \hspace{-0.1ex} = 0 \hspace{0.1ex},
\vspace{-0.32em}\end{equation}

\noindent где~${\bm{F}_k}$\:--- лишь активные силы, без реакций связей.

Дифференциальное вариационное уравнение~\eqref{discrete:principleofvirtualwork} может показаться тривиальным следствием закона Ньютона~\eqref{law:ofnewton} и~условия идеальности связей~\eqref{idealconstraints}. Однако содержание~\eqref{discrete:principleofvirtualwork} несравненно обширнее. Известно\:--- и~читатель вскоре это увидит,\:--- что принцип~\eqref{discrete:principleofvirtualwork} может быть положен в~основу механики~\cite{gantmacher}. Различные модели упругих тел, описываемые в~этой книге, построены с~опорой на~этот принцип.

Для~примера рассмотрим совершенно жёсткое~(недеформируемое) твёрдое тело.

...


Проявилась замечательная особенность~\eqref{discrete:principleofvirtualwork}: это уравнение эквивалентно системе такого порядка, каково число степеней свободы системы, то~есть сколько независимых вариаций~${\variation{\hspace{.1ex}\bm{r}_k}}$ мы имеем. Если в~системе $N$~точек есть $m$ связей, то число степеней свободы ${n = 3N \hspace{-0.25ex} - m}$.

...


\en{\section{Balance of momentum, rotational momentum, and~energy}}

\ru{\section{Баланс импульса, момента импульса и~энергии}}

Эти уравнения баланса можно связать со~свойствами пространства и~времени~\cite{landau.lifshitz-shortcourse}. Сохранение импульса (количества движения) в~изолированной\footnote{Изолированная (замкнутая) система\:--- это система частиц, взаимодействующих друг с~другом, но ни с~какими другими телами.}\hspace{-0.25ex} системе выводится из~однородности пространства \emph{(при любом параллельном переносе\:--- трансляции\:--- замкнутой системы как целого свойства этой системы не~меняются)}. Сохранение момента импульса\:--- следствие изотропии пространства \emph{(свойства замкнутой системы не~меняются при любом повороте этой системы как целого)}. Энергия~же изолированной системы сохраняется, так~как время однородно (энергия ${\mathrm{E} \hspace{.1ex} \equiv \mathrm{T}(q, \mathdotabove{q} \hspace{.2ex}) \hspace{-0.2ex} + \Pi(q)}$ такой системы не~зависит явно от~времени).

Уравнения баланса можно вывести из принципа виртуальной работы~\eqref{discrete:principleofvirtualwork}. Перепишем его в~виде

\nopagebreak\vspace{-0.2em}\begin{equation}\label{discrete:principleofvirtualwork.externalinternal}
\displaystyle \sum_k \hspace{-0.2ex} \left(^{\mathstrut} \hspace{-0.25ex} \bm{F}^{\smthexternal}_{\hspace{-0.16ex}k} \hspace{-0.1ex} - m_k \mathdotdotabove{\bm{r}}_k \right) \hspace{-0.32ex} \dotp \variation{\hspace{.1ex}\bm{r}_k} \hspace{-0.1ex}
+ \variation{\internalwork} \hspace{-0.1ex} = 0 \hspace{.1ex},
\vspace{-0.25em}\end{equation}

\vspace{-0.1em} \noindent где выделены внешние силы~${\bm{F}^{\smthexternal}_{\hspace{-0.16ex}k}}$ и~виртуальная работа внутренних сил
${\variation{\internalwork} \hspace{-0.12ex} = \scalebox{0.8}[0.84]{$\displaystyle \underset{\raisemath{.25ex}{\smash{k}}}{\sum}$} \scalebox{0.8}[0.84]{$\displaystyle \underset{\raisemath{.25ex}{\smash{j}}}{\sum}$} \hspace{.12ex} \bm{F}^{\smthinternal}_{\hspace{-0.16ex}kj} \hspace{-0.1ex} \dotp \variation{\hspace{.1ex}\bm{r}_k} \hspace{.1ex}}$.

\vspace{-0.1em} Предполагается, что внутренние силы не~совершают работы на~виртуальных перемещениях тела как жёсткого целого (${\constvarvector{\hspace{-0.1ex}\bm{\rho}}}$ и~${\constvarvector{\bm{o}}}$\:--- произвольные постоянные векторы, определяющие трансляцию и~поворот)

\nopagebreak\vspace{-0.2em}\begin{equation}\label{assumptionforvirtualwork}
\begin{array}{l}
\variation{\hspace{.1ex}\bm{r}_k} \hspace{-0.16ex}
= \constvarvector{\hspace{-0.1ex}\bm{\rho}} \hspace{.2ex} + \hspace{.12ex} \constvarvector{\bm{o}} \hspace{-0.2ex} \times \hspace{-0.1ex} \bm{r}_k
\hspace{.1ex} ,
\\
\constvarvector{\hspace{-0.1ex}\bm{\rho}} = \boldconstant \hspace{.1ex} , \:
\constvarvector{\bm{o}} = \boldconstant
\end{array}
\hspace{.3ex} \Rightarrow \hspace{.6ex}
\variation{\internalwork} \hspace{-0.1ex} = 0 \hspace{.1ex}.
\end{equation}

\vspace{-0.1em} Предпосылки-соображения для~этого предположения таковы.

Первое\:--- для случая упругих (потенциальных) внутренних сил. При~этом ${\variation{\internalwork} = - \hspace{.16ex} \variation{\Pi}}$\:--- вариация потенциала с~противоположным знаком. Достаточно очевидно, что~$\Pi$ меняется лишь при~деформации.

Второе соображение\:--- в~том, что суммарный вектор и~суммарный момент внутренних сил равен нулю

\begin{equation*}
\sum \ldots
\end{equation*}

...

Принимая~\eqref{assumptionforvirtualwork} и~подставляя в~\eqref{discrete:principleofvirtualwork.externalinternal} сначала ${\variation{\hspace{.1ex}\bm{r}_k} \hspace{-0.16ex} = \hspace{-0.08ex} \constvarvector{\hspace{-0.1ex}\bm{\rho}}}$ (трансляция), а~затем ${\variation{\hspace{.1ex}\bm{r}_k} \hspace{-0.16ex} = \hspace{-0.08ex} \constvarvector{\bm{o}} \hspace{-0.2ex} \times \hspace{-0.1ex} \bm{r}_k}$ (поворот), получаем баланс импульса~(...) и баланс момента импульса~(...).

...



\end{otherlanguage}

\en{\section{Hamilton’s principle and Lagrange’s equations}}

\ru{\section{Принцип Гамильтона и уравнения Лагранжа}}

\begin{otherlanguage}{russian}

Вариационное уравнение~\eqref{discrete:principleofvirtualwork} удовлетворяется в~любой момент времени. Проинтегрируем его\footnote{%
%%${ \variation{\mathrm{T}} = \scalebox{0.92}[0.95]{$ \displaystyle \sum_{\smash{k}} $} \hspace{.2ex} m_k \mathdotabove{\bm{r}}_k \hspace{-0.16ex} \dotp \variation{\hspace{.1ex}\mathdotabove{\bm{r}}_k} \hspace{-0.5ex} }$ , \hspace{-0.1ex}
${%
\left(
\scalebox{0.92}[0.95]{$ \displaystyle \sum_{\smash{k}} $} \hspace{.2ex} m_k \mathdotabove{\bm{r}}_k \hspace{-0.16ex} \dotp \variation{\hspace{.1ex}\bm{r}_k} \hspace{-0.4ex}
\right)^{\hspace{-0.25em}\tikz[baseline=-0.2ex]\draw[black, fill=black] (0,0) circle (.28ex);} \hspace{-0.1ex}
= \hspace{.2ex}
\scalebox{0.92}[0.95]{$ \displaystyle \sum_{\smash{k}} $} \hspace{.2ex} m_k \mathdotdotabove{\bm{r}}_k \hspace{-0.16ex} \dotp \variation{\hspace{.1ex}\bm{r}_k}
+ \hspace{-0.2ex} \tikzmark{beginVariationOfKinetic} \hspace{.32ex} \scalebox{0.92}[0.95]{$ \displaystyle \sum_{\smash{k}} $} \hspace{.2ex} m_k \mathdotabove{\bm{r}}_k \hspace{-0.16ex} \dotp \variation{\hspace{.1ex}\mathdotabove{\bm{r}}_k} \tikzmark{endVariationOfKinetic}
}$
\\[.5em]
${%
- \hspace{.1ex} \scalebox{0.95}[0.96]{$ \displaystyle \integral\displaylimits_{\mathclap{t_1}}^{\raisemath{.12em}{\mathclap{t_2}}} $} \scalebox{0.92}[0.95]{$ \displaystyle \sum_{\smash{k}} $} \hspace{.2ex} m_k \mathdotdotabove{\bm{r}}_k \hspace{-0.16ex} \dotp \variation{\hspace{.1ex}\bm{r}_k} \hspace{.25ex} dt \hspace{.4ex}
= \hspace{-0.1ex} \scalebox{0.95}[0.96]{$ \displaystyle \integral\displaylimits_{\mathclap{t_1}}^{\raisemath{.12em}{\mathclap{t_2}}} $} \hspace{-0.1ex} \variation{\mathrm{T}} \hspace{.1ex} dt \hspace{.16ex}
- \hspace{-0.1ex} \left[ \hspace{.2ex}
\scalebox{0.92}[0.95]{$ \displaystyle \sum_{\smash{k}} $} \hspace{.2ex} m_k \mathdotabove{\bm{r}}_k \hspace{-0.16ex} \dotp \variation{\hspace{.1ex}\bm{r}_k} \hspace{.16ex}
\right]_{\hspace{-0.25ex}t_1}^{\hspace{-0.25ex}t_2}
}$}%
\AddUnderBrace[line width=.75pt][-0.1ex,-0.77em]{beginVariationOfKinetic}{endVariationOfKinetic}{${\scriptstyle \variation{\mathrm{T}}}$}
%
по~какому\hbox{-}либо промежутку ${\left[\hspace{.15ex} t_1, t_2 \hspace{.15ex}\right]}$

\nopagebreak\vspace{-0.25em}\begin{equation}
\displaystyle \integral\displaylimits_{t_1}^{\raisemath{.12em}{t_2}}
\hspace{-0.4ex}
\left( \hspace{-0.32ex} \variation{\mathrm{T}}
+ \scalebox{0.95}[0.98]{$ \displaystyle \sum_{\smash{k}} $} \hspace{.16ex} \bm{F}_k \hspace{-0.1ex} \dotp \variation{\hspace{.1ex}\bm{r}_k} \hspace{-0.32ex} \right) \hspace{-0.5ex} dt \hspace{.16ex}
- \hspace{-0.1ex} \left[ \hspace{.2ex} \scalebox{0.95}[0.98]{$ \displaystyle \sum_{\smash{k}} $} \hspace{.2ex} m_k \mathdotabove{\bm{r}}_k \hspace{-0.1ex} \dotp \variation{\hspace{.1ex}\bm{r}_k} \hspace{.16ex} \right]_{\hspace{-0.32ex}t_1}^{\hspace{-0.32ex}t_2}
\hspace{-0.8ex} = 0 \hspace{.1ex} .
\end{equation}

\vspace{-0.16em} \noindent Без ущерба для общности можно принять ${\variation{\hspace{.1ex}\bm{r}_k}\hspace{.1ex}(t_1) \hspace{-0.1ex} = \variation{\hspace{.1ex}\bm{r}_k}\hspace{.1ex}(t_2) \hspace{-0.1ex} = \bm{0}}$, тогда внеинтегральный член исчезает.

Вводятся обобщённые координаты~$q_i$~(${i = 1, \ldots, n}$\:--- число степеней свободы). Векторы-радиусы становятся функциями \hbox{вида} ${\bm{r}_k(q_i, t)}$, тождественно удовлетворяющими уравнениям связей~\eqref{holonomicconstraint}. Если связи стационарны, то~есть~\eqref{holonomicconstraint} не~содержат $t$, то остаётся~${\bm{r}_k(q_i)}$. Кинетическая энергия превращается в~функцию ${\mathrm{T}(q_i, \mathdotabove{q}_i, t)}$, где явно входящее $t$ характерно лишь для нестационарных связей.

Весьма существенно понятие обобщённых сил~${Q_{\hspace{-0.1ex}i}}$. Они вводятся через выражение виртуальной работы

\nopagebreak\vspace{-0.25em}\begin{equation}
\scalebox{0.92}[0.96]{$ \displaystyle \sum_{\smash{k}} $} \hspace{.16ex}
\bm{F}_k \hspace{-0.1ex} \dotp \variation{\hspace{.1ex}\bm{r}_k} \hspace{-0.1ex}
= \hspace{-0.1ex} \scalebox{0.92}[0.96]{$\displaystyle \sum_{i}$} \hspace{.16ex} Q_{\hspace{-0.1ex}i} \hspace{.12ex} \variation{q_i} \hspace{.1ex} ,
\:\:
Q_{\hspace{-0.1ex}i} \equiv
\scalebox{0.92}[0.96]{$ \displaystyle \sum_{\smash{k}} $} \hspace{.16ex}
\bm{F}_k \hspace{-0.1ex} \dotp \scalebox{0.96}{$ \displaystyle \frac{\raisemath{-0.12em}{\partial \hspace{.1ex} \bm{r}_k}}{\raisemath{-0.1em}{\partial q_i}} $} \hspace{.16ex} .
\vspace{-0.1em}\end{equation}

\vspace{-0.15em} \noindent Ст\'{о}ит подчеркнуть происхождение обобщённых сил через работу. Установив набор обобщённых координат системы, следует сгруппировать приложенные силы~${\bm{F}_k}$ в~комплексы~${Q_{\hspace{-0.1ex}i}}$.

Если силы потенциальны с~энергией~${\Pi \!=\! \Pi(q_i, t)}$, то

\nopagebreak\vspace{-0.2em}\begin{equation}
\scalebox{0.92}[0.96]{$\displaystyle \sum_{i}$} \hspace{.16ex} Q_{\hspace{-0.1ex}i} \hspace{.12ex} \variation{q_i} \hspace{-0.1ex}
= - \hspace{.16ex} \variation{\Pi} \hspace{.1ex} ,
\:\:
Q_{\hspace{-0.1ex}i} = - \hspace{.16ex} \scalebox{0.96}{$ \displaystyle \frac{\raisemath{-0.12em}{\partial \hspace{.1ex} \Pi}}{\raisemath{-0.1em}{\partial q_i}} $} \hspace{.16ex} .
\vspace{-0.1em}\end{equation}

\vspace{-0.15em} \noindent Явное присутствие $t$ может быть при нестационарности связей или зависимости физических полей от~времени.

...

Известны уравнения Lagrange’а не~только второго, но~и~первого рода. Рассмотрим их ради методики вывода, много раз применяемой в~этой книге.

При~наличии связей~\eqref{holonomicconstraint} равенство ${\bm{F}_k \hspace{-0.12ex} = m_k \mathdotdotabove{\bm{r}}_k}$ не~следует из~вариационного уравнения~\eqref{discrete:principleofvirtualwork}, ведь тогда виртуальные перемещения~${\variation{\hspace{.1ex}\bm{r}_k}}$ не независимы. Каждое из $m$ ($m$\:--- число связей) условий для~вариаций~\eqref{requirementforvirtualdisplacements} умножим на~некий скаляр~$\lambda_{\alpha}$ (${\alpha = 1, \ldots, m}$) и~добавим к~\eqref{discrete:principleofvirtualwork}:

\nopagebreak\vspace{-0.2em}\begin{equation}
\displaystyle \sum_{k=1}^{N} \hspace{-0.2ex} \left(^{\mathstrut} \hspace{-0.16ex} \bm{F}_k \right. \hspace{-0.32ex} + \hspace{-0.05ex}
\scalebox{0.88}[0.92]{$ \displaystyle \sum_{\alpha=1}^{m} $} \hspace{.16ex} \lambda_{\alpha} \hspace{.2ex} \scalebox{0.92}{$\displaystyle \frac{\raisemath{-0.12em}{\partial \hspace{.1ex} c_{\alpha}}}{\partial \hspace{.1ex} \bm{r}_k}$}
- \left. \hspace{-0.25ex} m_k \mathdotdotabove{\bm{r}}_k ^{\mathstrut}\right) \hspace{-0.32ex} \dotp \variation{\hspace{.1ex}\bm{r}_k} \hspace{-0.16ex} = 0 \hspace{0.1ex} .
\end{equation}

\vspace{-0.1em} \noindent Среди $3N$ компонент вариаций~${\variation{\hspace{.1ex}\bm{r}_k}}$ зависимых $m$. Но столько~же и~множителей Лагранжа: подберём $\lambda_{\alpha}$ так, чтобы коэффициенты\textcolor{red}{(??как\'{и}е?)} при~зависимых вариациях обратились в~нуль. Но при~остальных вариациях коэффициенты\textcolor{red}{(??)} также должны быть нулями из\hbox{-}за независимости. Следовательно, все выражения в~скобках~${(\cdots\hspace{-0.2ex})}$ равны нулю\:--- это и~есть уравнения Лагранжа первого рода.

Поскольку для каждой частицы

...



\end{otherlanguage}

\en{\section{Statics}}

\ru{\section{Статика}}

\label{para:statics}

\begin{otherlanguage}{russian}

Рассмотрим систему со~стационарными (постоянными во~времени) связями при статических (не~меняющихся со~временем) активных силах ${\bm{F}_k}$. В~равновесии ${\bm{r}_k \hspace{-0.12ex} = \boldconstant}$, и формулировка принципа виртуальной работы следующая:

\nopagebreak\vspace{-0.1em}\begin{equation}\label{statics.discrete:principleofvirtualwork}
\scalebox{0.92}[0.96]{$ \displaystyle \sum_{\smash{k}} $} \hspace{.25ex}
\bm{F}_k \hspace{-0.1ex} \dotp \variation{\hspace{.1ex}\bm{r}_k} \hspace{-0.1ex} = 0
\:\,\Leftrightarrow\:
\scalebox{0.92}[0.96]{$ \displaystyle \sum_{\smash{k}} $} \hspace{.25ex}
\bm{F}_k \hspace{-0.1ex} \dotp \scalebox{0.96}{$ \displaystyle \frac{\raisemath{-0.12em}{\partial \hspace{.1ex} \bm{r}_k}}{\raisemath{-0.1em}{\partial q_i}} $}
= Q_{\hspace{-0.1ex}i} \hspace{-0.1ex} = 0 \hspace{.1ex} .
\vspace{-0.1em}\end{equation}

\vspace{-0.15em} \noindent Существенны обе стороны этого положения: и вариационное уравнение, и равенство нулю обобщённых сил.

Соотношения~\eqref{statics.discrete:principleofvirtualwork}\:--- это самые общие уравнения статики. В~литературе распространено узкое представление об~уравнениях равновесия как балансе сил и~моментов. Но при~этом нужно понимать, что набор уравнений равновесия точно соответствует обобщённым координатам.
\en{Resultant force}\ru{Результирующая сила~(также называемая \inquotes{равнодействующей силой} или \inquotes{главным вектором})}
\en{and}\ru{и}~\en{resultant couple}\ru{результирующий~(\inquotes{главный}) момент (пара сил)}
в~уравнениях равновесия фигурируют\footnote{\en{Since describing a~composition of~any system of forces}\ru{Со~времён описания приведения любой системы сил}, \en{acting on the~same absolutely rigid body}\ru{действующей на~одно и~то~же совершенно жёсткое тело}, \en{into a~single force}\ru{к~одной силе} \en{and}\ru{и} \en{a~single couple~(about a~chosen point)}\ru{одной паре~(вокруг выбранной точки)} \en{in the~book}\ru{в~книге}
\href{https://gallica.bnf.fr/ark:/12148/bpt6k6213152z.texteImage}{\inquotes{Éléments de~statique}}
\en{by }\href{https://en.wikipedia.org/wiki/Louis_Poinsot}{Louis\ru{’а} Poinsot}.}\hspace{-1ex},
поскольку у~системы есть степени свободы трансляции и~поворота. Огромная популярность сил и~моментов связана не~столько с~известностью статики совершенно жёсткого твёрдого тела, но с~тем, что виртуальная работа внутренних сил на~перемещениях системы как жёсткого целого равна нулю в~любой среде.

Пусть в~системе действуют два вида сил: потенциальные с~энергией от~обобщённых координат ${\Pi(q_i)}$ и~дополнительные внешние~${\mathcircabove{Q}_{\hspace{-0.1ex}i}}$. Из~\eqref{statics.discrete:principleofvirtualwork} следуют уравнения равновесия

\nopagebreak\vspace{-0.1em}\begin{equation}\label{staticequilibriumwithpotentialenergy}
\scalebox{0.96}{$ \displaystyle \frac{\raisemath{-0.15em}{\partial \hspace{.1ex} \Pi}}{\raisemath{-0.1em}{\partial q_i}} $} = \mathcircabove{Q}_{\hspace{-0.1ex}i}
\hspace{.1ex},
\end{equation}
\nopagebreak\vspace{.1em}\begin{equation*}
d\Pi = \scalebox{0.95}[1]{$\displaystyle \sum_{\smash{i}}$} \hspace{.32ex}
\scalebox{0.96}{$ \displaystyle \frac{\raisemath{-0.15em}{\partial \hspace{.1ex} \Pi}}{\raisemath{-0.1em}{\partial q_i}} $} \hspace{.2ex} dq_i
= \scalebox{0.95}[1]{$\displaystyle \sum_{\smash{i}}$} \hspace{.2ex} \mathcircabove{Q}_{\hspace{-0.1ex}i} \hspace{.2ex} dq_i
\hspace{.1ex} .
\end{equation*}

\vspace{-0.5em} \noindent Здесь содержится нелинейная в~общем случае задача статики о~связи положения равновесия~$q_i$ с~нагрузками~${\mathcircabove{Q}_{\hspace{-0.1ex}i}}$.

В~линейной системе с~квадратичным потенциалом вида ${\Pi = \smalldisplaystyleonehalf \hspace{.2ex} C_{ik} \hspace{.12ex} q_{k} \hspace{.1ex} q_{i}}$

\nopagebreak\vspace{-1.25em}\begin{equation}\label{staticsoflineardiscretesystem}
\scalebox{0.95}[1]{$\displaystyle \sum_{\smash{k}}$} \hspace{.2ex} C_{ik} \hspace{.12ex} q_k \hspace{-0.1ex}
= \hspace{.1ex} \mathcircabove{Q}_{\hspace{-0.1ex}i} \hspace{.1ex} .
\vspace{-0.1em}\end{equation}

\vspace{-0.2em} \noindent Тут фигурируют матрица жёсткости~$C_{ik}$ и столбцы координат~${q_k}$ и~нагрузок~${\mathcircabove{Q}_{\hspace{-0.1ex}i}}$.

Сказанное возможно обобщить и~на~континуальные линейные упругие среды.

Матрица жёсткости~${C_{ik}}$ обычно бывает положительной (таков\'{о} свойство конструкций). % в~природе и~технике).
Тогда ${\operatorname{det} \hspace{.16ex} C_{ik} > 0}$, линейная алгебраическая система~\eqref{staticsoflineardiscretesystem} однозначно разрешима, а~решение её можно заменить минимизацией квадратичной формы

\nopagebreak\vspace{-0.1em}\begin{equation}\label{discrete:potentialenergyofsystem}
\textit{Э} \hspace{.16ex} (q_j) \hspace{.1ex}
\equiv \hspace{.1ex}
\Pi - \scalebox{0.95}[1]{$\displaystyle \sum_{\smash{i}}$} \hspace{.2ex}
\mathcircabove{Q}_{\hspace{-0.1ex}i} \hspace{.12ex} q_i
%
= \hspace{.1ex}
\smalldisplaystyleonehalf \hspace{.32ex}
\scalebox{0.95}[1]{$\displaystyle \sum_{\smash{i,k}}$} \hspace{.2ex}
%%\scalebox{0.95}[1]{$\displaystyle \sum_{\smash{k}}$} \hspace{.2ex}
\hspace{.1ex} q_{i} \hspace{.1ex} C_{ik} \hspace{.1ex} q_{k}
- \scalebox{0.95}[1]{$\displaystyle \sum_{\smash{i}}$} \hspace{.2ex}
\mathcircabove{Q}_{\hspace{-0.1ex}i} \hspace{.12ex} q_i
%
\hspace{.1ex}\to\hspace{.25ex} \mathrm{min} \hspace{.16ex} .
\vspace{-0.1em}\end{equation}

\vspace{-0.1em} Бывает однако, что конструкция неудачно спроектирована, тогда матрица жёсткости сингулярна~(необратима) %%(noninvertible)
и~${\operatorname{det} \hspace{.16ex} C_{ik} = \hspace{.1ex} 0}$ (или~же весьма близок к~нулю\:--- nearly singular матрица с~${\operatorname{det} \hspace{.16ex} C_{ik} \approx \hspace{.1ex} 0}$). Тогда решение линейной проблемы статики~\eqref{staticsoflineardiscretesystem} существует лишь при ортогональности столбца нагрузок~${\mathcircabove{Q}_{\hspace{-0.1ex}i}}$ всем линейно независимым решениям однородной сопряжённой системы

...

Известные теоремы статики линейно \textcolor{magenta}{упругих} систем легко доказываются в~случае конечного числа степеней свободы. Теорема \href{https://en.wikipedia.org/wiki/Beno%C3%AEt_Paul_%C3%89mile_Clapeyron}{Clapeyron’а} выражается равенством

...

\en{Reciprocal work theorem}\ru{Теорема о~взаимности работ} (\inquotes{работа~${W_{\hspace{-0.1ex}12}}$ сил первого варианта на~перемещениях от сил второго равна работе~${W_{\hspace{-0.15ex}21}}$ сил второго варианта на~перемещениях от сил первого}) мгновенно выводится из~\eqref{staticsoflineardiscretesystem}:

(...)

\noindent Тут существенна симметрия матрицы жёсткости~$C_{i\hspace{-0.1ex}j}$, то~есть консервативность системы.

...

Но вернёмся к~проблеме~\eqref{staticequilibriumwithpotentialenergy}, иногда называемой теоремой Lagrange’а. Её можно обратить преобразованием Лежандра Legendre (involution) transform(ation):

\nopagebreak\vspace{-0.2em}\begin{equation*}
\begin{array}{c}\small
d \hspace{-0.1ex} \left( \hspace{-0.1ex} \scalebox{0.95}[1]{$\displaystyle \sum_{\smash{i}}$} \hspace{.2ex} \mathcircabove{Q}_{\hspace{-0.1ex}i} \hspace{.2ex} q_i \hspace{-0.12ex} \right) \hspace{-0.5ex}
= \scalebox{0.95}[1]{$\displaystyle \sum_{\smash{i}}$} \hspace{.25ex} d \hspace{-0.25ex} \left( \hspace{-0.1ex} \mathcircabove{Q}_{\hspace{-0.1ex}i} \hspace{.2ex} q_i \right) \hspace{-0.3ex}
= \scalebox{0.95}[1]{$\displaystyle \sum_{\smash{i}}$} \hspace{-0.16ex} \left(
q_i \hspace{.2ex} d \mathcircabove{Q}_{\hspace{-0.1ex}i}
+ \mathcircabove{Q}_{\hspace{-0.1ex}i} \hspace{.2ex} dq_i \right)
\hspace{-0.64ex} ,
\\[1.25em]
%
\small
d \hspace{-0.1ex} \left( \hspace{-0.1ex} \scalebox{0.95}[1]{$\displaystyle \sum_{\smash{i}}$} \hspace{.2ex} \mathcircabove{Q}_{\hspace{-0.1ex}i} \hspace{.2ex} q_i \hspace{-0.12ex} \right) \hspace{-0.4ex}
- \hspace{.1ex} \tikzmark{beginLegendreOfPotential} \scalebox{0.95}[1]{$\displaystyle \sum_{\smash{i}}$} \hspace{.2ex} \mathcircabove{Q}_{\hspace{-0.1ex}i} \hspace{.2ex} dq_i \tikzmark{endLegendreOfPotential}
= \scalebox{0.95}[1]{$\displaystyle \sum_{\smash{i}}$} \hspace{.32ex} q_i \hspace{.2ex} d\mathcircabove{Q}_{\hspace{-0.1ex}i}
\hspace{.2ex} ,
\\[1em]
%
\small
d \hspace{-0.1ex} \left( \scalebox{0.95}[1]{$\displaystyle \sum_{\smash{i}}$} \hspace{.2ex} \mathcircabove{Q}_{\hspace{-0.1ex}i} \hspace{.2ex} q_i - \Pi \hspace{-0.1ex} \right) \hspace{-0.5ex}
= \scalebox{0.95}[1]{$\displaystyle \sum_{\smash{i}}$} \hspace{.32ex} q_i \hspace{.2ex} d\mathcircabove{Q}_{\hspace{-0.1ex}i}
= \scalebox{0.95}[1]{$\displaystyle \sum_{\smash{i}}$} \hspace{.4ex}
\scalebox{0.96}{$ \displaystyle \frac{\raisemath{-0.12em}{\partial \hspace{.1ex} \widehat{\Pi}}}{\raisemath{-0.4em}{\partial \mathcircabove{Q}_{\hspace{-0.1ex}i}}} $}
\hspace{.25ex} d\mathcircabove{Q}_{\hspace{-0.1ex}i}
%
\hspace{.33ex} ;
\end{array}\end{equation*}%
\AddOverBrace[line width=.75pt][-0.3ex,0.3em]{beginLegendreOfPotential}{endLegendreOfPotential}{${\scriptstyle d\Pi}$}

\nopagebreak\vspace{-0.33em}\begin{equation}\label{Castigliano:theorem}
q_i = \scalebox{0.96}{$ \displaystyle \frac{\raisemath{-0.12em}{\partial \hspace{.1ex} \widehat{\Pi}}}{\raisemath{-0.4em}{\partial \mathcircabove{Q}_{\hspace{-0.1ex}i}}} $} \hspace{.2ex} ,
\:\;
\widehat{\Pi}(\mathcircabove{Q}_{\hspace{-0.1ex}i})
= \scalebox{0.95}[1]{$\displaystyle \sum_{\smash{i}}$} \hspace{.2ex} \mathcircabove{Q}_{\hspace{-0.1ex}i} \hspace{.12ex} q_i - \Pi
\hspace{.1ex} .
\end{equation}

\vspace{-0.1em} \noindent Это теорема \href{https://en.wikipedia.org/wiki/Carlo_Alberto_Castigliano}{Castigliano}, $\widehat{\Pi}$ называется дополнительной энергией. В~линейной системе \eqref{staticsoflineardiscretesystem} ${\Rightarrow}$ ${\widehat{\Pi} = \Pi}$. Теорема~\eqref{Castigliano:theorem} бывает очень полезна\:--- когда легко находится~${\widehat{\Pi}(\mathcircabove{Q}_{\hspace{-0.1ex}i})}$. Встречаются так называемые статически определимые системы, в~которых все внутренние силы удаётся найти лишь из баланса сил и~моментов. Для них \eqref{Castigliano:theorem} эффективна.

В~отличие от линейной задачи~\eqref{staticsoflineardiscretesystem}, нелинейная задача~\eqref{staticequilibriumwithpotentialenergy} может не~иметь решений вовсе или~же иметь их н\'{е}сколько.

...

Разговор о~статике в~общей механике закончим принципом \hbox{д’\hspace{-0.2ex}Аламбера}~(\hbox{d’\hspace{-0.2ex}Alembert}’s principle): уравнения динамики отличаются от~статических лишь наличием дополнительных \inquotes{фиктивных сил инерции}~${m_k \mathdotdotabove{\bm{r}}_k}$. Принцип \hbox{д’\hspace{-0.2ex}Аламбера} очевиден, но бездумное его применение может привести к~ошибкам. Например, уравнения вязкой жидкости в~статике и~в~динамике отличаются не~только лишь инерционными членами. Для упругих~же сред принцип \hbox{д’\hspace{-0.2ex}Аламбера} полностью справедлив.

\end{otherlanguage}

\en{\section{Mechanics of relative motion}}

\ru{\section{Механика относительного движения}}

\label{para:mechanicsofrelativemotion}

\begin{otherlanguage}{russian}

До~этого не~ставился вопрос о~системе отсчёта, всё рассматривалось в~некой \inquotes{абсолютной} системе или одной из инерциальных систем~(\pararef{para:initialconcepts.discreteapproach}). Теперь представим себе две системы: \inquotes{абсолютную} и~\inquotes{подвижную}

...

\begin{equation*}
\begin{array}{c}
\bm{R} = \bm{r} + \bm{x}
\\
\bm{r} = \hspace{-0.1ex} \rho_i \hspace{.1ex} \mathcircabove{\bm{e}}_i
\hspace{.1ex} , \:\:
\bm{x} = x_i \bm{e}_i
\\
\mathdotabove{\bm{R}} = \mathdotabove{\bm{r}} + \mathdotabove{\bm{x}}
\\
\mathdotabove{\bm{r}} = \hspace{-0.1ex}
\mathdotabove{\rho}_i \hspace{.1ex} \mathcircabove{\bm{e}}_i
\hspace{.1ex} , \:\:
\mathdotabove{\bm{x}} = \hspace{-0.15ex} \bigl( x_i \bm{e}_i \bigr)^{\hspace{-0.15ex}\tikz[baseline=-0.2ex]\draw[black, fill=black] (0,0) circle (.28ex);} \hspace{-0.15ex}
= \mathdotabove{x}_i \bm{e}_i \hspace{-0.1ex} + x_i \mathdotabove{\bm{e}}_i
\end{array}
\end{equation*}

${x_i \hspace{-0.1ex} \neq \constant}$ $\Rightarrow$ ${\mathdotabove{x}_i \hspace{-0.1ex} \neq 0}$

\en{By}\ru{По}~\eqrefwithchapdotpara{angularvelocityandbasisvectors}{chapter:elementsoftensorcalculus}{para:rotationtensor}

\nopagebreak\vspace{-0.2em}\begin{equation*}
\mathdotabove{\bm{e}}_i \hspace{-0.1ex} = \bm{\omega} \hspace{-0.1ex} \times \hspace{-0.1ex} \bm{e}_i
\hspace{.33ex} \Rightarrow \hspace{.4ex}
x_i \mathdotabove{\bm{e}}_i \hspace{-0.1ex} = \bm{\omega} \hspace{-0.1ex} \times \hspace{-0.1ex} x_i \bm{e}_i \hspace{-0.1ex}
= \bm{\omega} \hspace{-0.1ex} \times \hspace{-0.1ex} \bm{x}
\end{equation*}

${
\mathdotabove{\bm{x}} = \mathdotabove{x}_i \bm{e}_i + \hspace{.1ex} \bm{\omega} \hspace{-0.16ex} \times \hspace{-0.16ex} \bm{x}
}$

\begin{equation*}
\bm{v} \equiv \hspace{-0.1ex} \mathdotabove{\bm{R}} = \mathdotabove{\bm{r}} + \mathdotabove{\bm{x}}
= \hspace{-0.2ex} \tikzmark{beginEVelocity} \hspace{.25ex} \mathdotabove{\bm{r}} \hspace{.1ex} + \hspace{.1ex} \bm{\omega} \hspace{-0.2ex} \times \hspace{-0.2ex} \bm{x} \tikzmark{endEVelocity}
\hspace{.4ex} \tikzmark{beginRelativeVelocity} \hspace{-0.4ex} - \bm{\omega} \hspace{-0.2ex} \times \hspace{-0.2ex} \bm{x} \hspace{.1ex} + \hspace{.1ex} \mathdotabove{\bm{x}} \hspace{-0.33ex} \tikzmark{endRelativeVelocity}
\end{equation*}%
\AddUnderBrace[line width=.75pt][0,0]{beginEVelocity}{endEVelocity}{${ \scriptstyle \bm{v}_{\hspace{-0.1ex}e} }$}
\AddUnderBrace[line width=.75pt][.4ex,0][xshift=.2ex]{beginRelativeVelocity}{endRelativeVelocity}{${ \scriptstyle \bm{v}_{r\kern-0.1exel} }$}

${
\mathdotabove{\bm{x}} \hspace{.1ex} - \hspace{.1ex} \bm{\omega} \hspace{-0.2ex} \times \hspace{-0.2ex} \bm{x} = \mathdotabove{x}_i \bm{e}_i
\equiv \hspace{.1ex} \bm{v}_{r\kern-0.1exel}
}$\:--- relative velocity,
${
\mathdotabove{\bm{r}} \hspace{.1ex} + \hspace{.1ex} \bm{\omega} \hspace{-0.2ex} \times \hspace{-0.2ex} \bm{x}
\equiv \hspace{.1ex} \bm{v}_{\hspace{-0.1ex}e}
}$

\begin{equation}
\bm{v} = \bm{v}_{\hspace{-0.1ex}e} \hspace{-0.1ex} + \bm{v}_{r\kern-0.1exel}
\end{equation}

...

\begin{equation*}
\begin{array}{c}
\mathdotabove{\bm{R}} = \mathdotabove{\bm{r}} + \mathdotabove{\bm{x}}
\\
\mathdotdotabove{\bm{R}} = \mathdotdotabove{\bm{r}} + \mathdotdotabove{\bm{x}}
\\
\bm{w} \equiv \hspace{.1ex} \mathdotabove{\bm{v}} = \hspace{-0.15ex} \mathdotdotabove{\bm{R}} = \mathdotdotabove{\bm{r}} + \mathdotdotabove{\bm{x}}
\\
\mathdotdotabove{\bm{r}} = \hspace{-0.1ex}
\mathdotdotabove{\rho}_i \hspace{.1ex} \mathcircabove{\bm{e}}_i
\hspace{.1ex} , \:\:
\mathdotdotabove{\bm{x}} = \hspace{-0.15ex} \bigl( x_i \bm{e}_i \bigr)^{ \hspace{-0.15ex} \tikz[baseline=-0.2ex] \draw[black, fill=black] (0,0) circle (.28ex); \hspace{.2ex} \tikz[baseline=-0.2ex] \draw[black, fill=black] (0,0) circle (.28ex); } \hspace{-0.2ex}
= \hspace{-0.15ex} \bigl( \mathdotabove{x}_i \bm{e}_i \hspace{-0.1ex} + x_i \mathdotabove{\bm{e}}_i \bigr)^{ \hspace{-0.15ex} \tikz[baseline=-0.2ex] \draw[black, fill=black] (0,0) circle (.28ex); } \hspace{-0.15ex}
= \mathdotdotabove{x}_i \bm{e}_i \hspace{-0.1ex} + \mathdotabove{x}_i \mathdotabove{\bm{e}}_i \hspace{-0.1ex}
+ \mathdotabove{x}_i \mathdotabove{\bm{e}}_i \hspace{-0.1ex} + x_i \mathdotdotabove{\bm{e}}_i \hspace{-0.1ex}
\end{array}
\end{equation*}

\nopagebreak\begin{equation*}
\mathdotabove{\bm{e}}_i \hspace{-0.1ex} = \bm{\omega} \hspace{-0.1ex} \times \hspace{-0.12ex} \bm{e}_i
\hspace{.33ex} \Rightarrow \hspace{.4ex}
\mathdotdotabove{\bm{e}}_i \hspace{-0.1ex}
= \hspace{-0.15ex} \bigl( \hspace{.1ex} \bm{\omega} \hspace{-0.2ex} \times \hspace{-0.2ex} \bm{e}_i \hspace{.1ex} \bigr)^{ \hspace{-0.15ex} \tikz[baseline=-0.2ex] \draw[black, fill=black] (0,0) circle (.28ex); } \hspace{-0.15ex}
= \mathdotabove{\bm{\omega}} \hspace{-0.2ex} \times \hspace{-0.2ex} \bm{e}_i \hspace{-0.1ex} + \bm{\omega} \hspace{-0.2ex} \times \hspace{-0.2ex} \mathdotabove{\bm{e}}_i \hspace{-0.1ex}
= \mathdotabove{\bm{\omega}} \hspace{-0.2ex} \times \hspace{-0.2ex} \bm{e}_i \hspace{-0.1ex}
+ \bm{\omega} \hspace{-0.2ex} \times \hspace{-0.33ex} \bigl( \hspace{.1ex} \bm{\omega} \hspace{-0.2ex} \times \hspace{-0.2ex} \bm{e}_i \hspace{.1ex} \bigr)
\end{equation*}

\nopagebreak\begin{equation*}
x_i \mathdotdotabove{\bm{e}}_i \hspace{-0.1ex}
= x_i \bigl( \hspace{.1ex} \bm{\omega} \hspace{-0.2ex} \times \hspace{-0.2ex} \bm{e}_i \hspace{.1ex} \bigr)^{ \hspace{-0.15ex} \tikz[baseline=-0.2ex] \draw[black, fill=black] (0,0) circle (.28ex); } \hspace{-0.15ex}
= \mathdotabove{\bm{\omega}} \hspace{-0.12ex} \times \hspace{-0.12ex} x_i \bm{e}_i
+ \bm{\omega} \hspace{-0.2ex} \times \hspace{-0.33ex} \bigl( \hspace{.1ex} \bm{\omega} \hspace{-0.2ex} \times \hspace{-0.2ex} x_i \bm{e}_i \hspace{.1ex} \bigr) \hspace{-0.15ex}
= \mathdotabove{\bm{\omega}} \hspace{-0.12ex} \times \hspace{-0.12ex} \bm{x}
+ \hspace{.1ex} \bm{\omega} \hspace{-0.2ex} \times \hspace{-0.33ex} \bigl( \hspace{.1ex} \bm{\omega} \hspace{-0.2ex} \times \hspace{-0.2ex} \bm{x} \hspace{.1ex} \bigr)
\end{equation*}

\nopagebreak\begin{equation*}
\mathdotabove{\bm{e}}_i \hspace{-0.1ex} = \bm{\omega} \hspace{-0.1ex} \times \hspace{-0.12ex} \bm{e}_i
\hspace{.33ex} \Rightarrow \hspace{.4ex}
\mathdotabove{x}_i \mathdotabove{\bm{e}}_i \hspace{-0.1ex}
= \bm{\omega} \hspace{-0.1ex} \times \hspace{-0.1ex} \mathdotabove{x}_i \bm{e}_i \hspace{-0.1ex}
= \bm{\omega} \hspace{-0.1ex} \times \hspace{-0.1ex} \bm{v}_{r\kern-0.1exel}
\end{equation*}

${
\mathdotdotabove{x}_i \bm{e}_i
\equiv \hspace{.1ex} \bm{w}_{r\kern-0.1exel}
}$\:--- relative acceleration

${
2 \hspace{.2ex} \mathdotabove{x}_i \mathdotabove{\bm{e}}_i \hspace{-0.1ex}
= 2 \hspace{.33ex} \bm{\omega} \hspace{-0.2ex} \times \hspace{-0.2ex} \bm{v}_{r\kern-0.1exel} \hspace{-0.1ex}
\equiv \hspace{.1ex} \bm{w}_{\hspace{-0.1ex}\raisemath{-0.16ex}{C}\hspace{-0.1ex}or}
}$\:--- Coriolis acceleration

\begin{equation*}
\mathdotdotabove{\bm{x}}
= \bm{w}_{r\kern-0.1exel} \hspace{-0.1ex}
+ \bm{w}_{\hspace{-0.1ex}\raisemath{-0.16ex}{C}\hspace{-0.1ex}or} \hspace{-0.1ex}
+ x_i \mathdotdotabove{\bm{e}}_i
\end{equation*}

\begin{equation*}
\begin{array}{c}
\bigl( x_i \mathdotabove{\bm{e}}_i \bigr)^{ \hspace{-0.15ex} \tikz[baseline=-0.2ex] \draw[black, fill=black] (0,0) circle (.28ex); } \hspace{-0.15ex}
= \mathdotabove{x}_i \mathdotabove{\bm{e}}_i \hspace{-0.1ex} + x_i \mathdotdotabove{\bm{e}}_i \hspace{-0.12ex}
= \hspace{.12ex} \smalldisplaystyleonehalf \hspace{.1ex} \bm{w}_{\hspace{-0.1ex}\raisemath{-0.16ex}{C}\hspace{-0.1ex}or} \hspace{-0.15ex} + x_i \mathdotdotabove{\bm{e}}_i \hspace{-0.1ex}
\\[.2em]
%
\bigl( x_i \mathdotabove{\bm{e}}_i \bigr)^{ \hspace{-0.15ex} \tikz[baseline=-0.2ex] \draw[black, fill=black] (0,0) circle (.28ex); } \hspace{-0.15ex}
= \hspace{-0.15ex} \bigl( \hspace{.1ex} \bm{\omega} \hspace{-0.2ex} \times \hspace{-0.2ex} \bm{x} \hspace{.1ex} \bigr)^{ \hspace{-0.15ex} \tikz[baseline=-0.2ex] \draw[black, fill=black] (0,0) circle (.28ex); } \hspace{-0.15ex}
= \mathdotabove{\bm{\omega}} \hspace{-0.2ex} \times \hspace{-0.2ex} \bm{x} + \bm{\omega} \hspace{-0.2ex} \times \hspace{-0.2ex} \mathdotabove{\bm{x}}
\end{array}
\end{equation*}

\begin{equation*}
\bm{\omega} \hspace{-0.12ex} \times \hspace{-0.12ex} \mathdotabove{\bm{x}}
= \bm{\omega} \hspace{-0.2ex} \times \hspace{-0.33ex} \bigl( \hspace{.1ex} \mathdotabove{x}_i \bm{e}_i + \hspace{.1ex} \bm{\omega} \hspace{-0.2ex} \times \hspace{-0.2ex} \bm{x} \hspace{.1ex} \bigr) \hspace{-0.15ex}
= \hspace{-0.2ex} \tikzmark{beginCoriolisHalf} \hspace{.2ex} \bm{\omega} \hspace{-0.1ex} \times \hspace{-0.1ex} \mathdotabove{x}_i \bm{e}_i \hspace{.2ex} \tikzmark{endCoriolisHalf} \hspace{-0.25ex}
+ \hspace{.1ex} \bm{\omega} \hspace{-0.2ex} \times \hspace{-0.33ex} \bigl( \hspace{.1ex} \bm{\omega} \hspace{-0.2ex} \times \hspace{-0.2ex} \bm{x} \hspace{.1ex} \bigr)
\end{equation*}%
\AddUnderBrace[line width=.75pt][0,-0.1em]{beginCoriolisHalf}{endCoriolisHalf}{${\scalebox{0.8}{$ \mathdotabove{x}_i \mathdotabove{\bm{e}}_i \hspace{-0.15ex} = \smalldisplaystyleonehalf \hspace{.1ex} \bm{w}_{\hspace{-0.1ex}\raisemath{-0.16ex}{C}\hspace{-0.1ex}or} $}}$}

...


\end{otherlanguage}

\en{\section{Small oscillations (vibrations)}}

\ru{\section{Малые колебания (вибрации)}}

\label{para:smalloscillations}

\begin{otherlanguage}{russian}

Если статика линейной системы описывается уравнением~\eqref{staticsoflineardiscretesystem}, то в~динамике имеем

\nopagebreak\vspace{-0.4em}\begin{equation}\label{dynamicsoflineardiscretesystem}
\scalebox{0.95}[1]{$\displaystyle \sum_{\smash{k}}$} \hspace{-0.2ex} \left(^{\mathstrut} \hspace{-0.2ex} A_{ik} \hspace{.12ex} \mathdotdotabove{q}_k + C_{ik} \hspace{.12ex} q_k \right) \hspace{-0.4ex}
= \hspace{.1ex} \mathcircabove{Q}_{\hspace{-0.1ex}i}(t) \hspace{.1ex} ,
\vspace{-0.1em}\end{equation}

\vspace{-0.25em} \noindent где ${A_{ik}}$\:--- симметричная и~положительная \inquotesx{матрица кинетической энергии}[.]

{\small%
A normal mode of~an~oscillating system is a pattern of motion in which all parts of the system move sinusoidally with the same frequency and with a fixed phase relation. The free motion described by the normal modes takes place at fixed frequencies. These fixed frequencies of the normal modes of a system are known as its natural resonant frequencies.

In music, normal modes of vibrating instruments (strings, air pipes, drums, etc.) are called \inquotes{harmonics} or \inquotesx{overtones}[.]

The most general motion of a system is a superposition of its normal modes. The modes are normal in the sense that they can move independently, that is to say that an excitation of one mode will never cause motion of a different mode. In mathematical terms, normal modes are orthogonal to each other.
\par}

\en{A~research}\ru{Изучение} \en{of~an~oscillating system}\ru{колеблющейся системы} \en{most often begins with }\ru{чаще всего начинается с~}\en{ortho\-gonal}\ru{орто\-гональ\-ных}~(\en{normal}\ru{нормальных}) \inquotesx{\en{modes}\ru{мод}}[---] \en{harmonics}\ru{гармоник}, собственных~(свободных, без воздействий извне) \en{sinusoidal oscillations}\ru{синусоидальных колебаний} \en{like}\ru{вида}

\nopagebreak\vspace{-0.25em}\begin{equation*}
q_k \hspace{-0.1ex} (t) \hspace{-0.2ex} = \widearc{q}_{\hspace{-0.1ex}k} \operatorname{sin} \omega_k \hspace{.1ex} t
\hspace{.1ex} .
\end{equation*}

\vspace{-0.2em} \noindent
\en{Multipliers}\ru{Множители}~${\widearc{q}_{\hspace{-0.1ex}k} \hspace{-0.15ex} = \constant}$\en{ are}\ru{\:---} \en{ortho\-gonal}\ru{орто\-гональ\-ные}~(\en{normal}\ru{нормальные}) \inquotes{\en{modes}\ru{моды}} \en{of~oscillation}\ru{колебания}, ${\omega_k\hspace{-0.1ex}}$\en{ are}\ru{\:---} \en{natural}\ru{натуральные}~(\en{resonant}\ru{резонансные}, \en{eigen-}\ru{собственные}) \en{frequencies}\ru{частоты}.
\en{This set}\ru{Этот набор}, \en{dependent on}\ru{зависящий от} \en{the~structure}\ru{структуры} \en{of~an~oscillating object}\ru{колеблющегося объекта}, \en{materials}\ru{материалов} \en{and}\ru{и}~\en{boundary conditions}\ru{краевых условий}, находится из задачи на~собственные значения

\nopagebreak\vspace{-0.1em}\begin{equation}
\begin{array}{c}
\mathcircabove{Q}_{\hspace{-0.1ex}i} \hspace{-0.15ex} = 0
\hspace{.1ex} ,
\:\;
\mathdotdotabove{q}_k \hspace{-0.12ex} = - \hspace{.2ex} \omega_k^2 \hspace{.25ex} \widearc{q}_{\hspace{-0.1ex}k} \hspace{-0.1ex} \operatorname{sin} \omega_k \hspace{.1ex} t
\hspace{.1ex} ,
\:\;
\eqref{dynamicsoflineardiscretesystem}
\:\: \Rightarrow
\\[.3em]
%
\Rightarrow \:\,
\scalebox{0.95}[1]{$\displaystyle \sum_{\smash{k}}$} \Bigl( \hspace{-0.2ex} C_{ik} \hspace{-0.1ex} - \hspace{-0.2ex} A_{ik} \hspace{.25ex} \omega_k^2 \hspace{.1ex} \Bigr) \hspace{.12ex}
\widearc{q}_{\hspace{-0.1ex}k} \operatorname{sin} \omega_k \hspace{.1ex} t
= 0
\end{array}
\end{equation}

...


\end{otherlanguage}

\section*{\small \wordforbibliography}

\begin{changemargin}{\parindent}{0pt}
\fontsize{10}{12}\selectfont

\begin{otherlanguage}{russian}

В~длинном списке книг по~общей механике можно найти труды не~только механиков\hbox{-}профессионалов~\cite{goldstein-classicalmechanics, treatiseonanalyticaldynamics-by-l.a.pars, loitsjanskiy.lurie, lurie-analyticalmechanics, olkhovskiy-theoreticalmechanicsforphysicists}, но~и физиков\hbox{-}теоретиков широкой ориентации~\cite{landau.lifshitz-shortcourse, terhaar-hamiltonianmechanics}. Интересен курс Ф.\,Р.\;Гантмахера~\cite{gantmacher} с~компактным, но~полным изложением осн\'{о}в.

\end{otherlanguage}

\end{changemargin}
