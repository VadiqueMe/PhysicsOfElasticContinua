\en{\chapter{Classical linear elasticity}}

\ru{\chapter{Классическая линейная упругость}}

\thispagestyle{empty}

\label{chapter:linearclassicalelasticity}

\begin{changemargin}{\parindent}{\parindent}
\vspace{-2em}
{\noindent \small

\en{Geometrically linear model}\ru{Геометрически линейная модель}: \en{displacements are small}\ru{перемещения м\'{а}лые}.
\en{Operators}\ru{Операторы} ${\hspace{-0.2ex}\boldnablacircled\hspace{.1ex}}$ \en{and}\ru{и}~${\hspace{-0.2ex}\boldnabla\hspace{.1ex}}$ \en{are indistinguishable}\ru{неразличимы}, \inquotesx{\en{equations}\ru{уравнения} \en{can~be written in the~reference configuration}\ru{можно пис\'{а}ть в~отсчётной конфигурации}}[.]

\par}
\vspace{-1em}
\end{changemargin}

\en{\section{Complete set of equations}}

\ru{\section{Полный набор уравнений}}

\label{para:wholesetofequations.lineartheory}

\en{\lettrine[lines=2, findent=2pt, nindent=0pt]{E}{quations} of~nonlinear elasticity}\ru{\lettrine[lines=2, findent=2pt, nindent=0pt]{У}{равнения} нелинейной упругости}, \en{even in simplest cases}\ru{даже в~самых простых случаях}, \en{bring to mathematically complicated problems}\ru{приводят к~математически сложным задачам}.
\en{Therefore}\ru{Поэтому} \en{the~linear theory is applied everywhere}\ru{повсеместно применяется линейная теория}.
\en{I\kern-0.12exts equations were derived}\ru{Её~уравнения были выведены} \en{in the~first half}\ru{в~первой половине} \en{of~the~}\hbox{XIX$^{\textrm{\en{th}\ru{го}}}$\hspace{-0.2ex}}~\en{century}\ru{века}\en{ by}
\href{https://en.wikipedia.org/wiki/Augustin-Louis_Cauchy}{Cauchy},
\href{https://en.wikipedia.org/wiki/Claude-Louis_Navier}{Navier},
\href{https://en.wikipedia.org/wiki/Gabriel_Lam\%C3\%A9}{Lam\'{e}},
\href{https://en.wikipedia.org/wiki/Beno\%C3\%AEt_Paul_\%C3\%89mile_Clapeyron}{Clapeyron}\ru{’ом},
\href{https://en.wikipedia.org/wiki/Sim\%C3\%A9on_Denis_Poisson}{Poisson}\ru{’ом},
\href{https://en.wikipedia.org/wiki/Adh\%C3\%A9mar_Jean_Claude_Barr\%C3\%A9_de_Saint-Venant}{Saint\hbox{-\hspace{-0.2ex}}Venant}\ru{’ом},
\href{https://en.wikipedia.org/wiki/George_Green_(mathematician)}{George Green}\ru{’ом}
\en{and other scientists}\ru{и~другими учёными}.

\en{The~complete closed set}\ru{Полный замкнутый набор}~(\en{system}\ru{система}) \en{of~equations}\ru{уравнений} \en{of~the~classical linear theory}\ru{классической линейной теории} \en{in~the~direct invariant tensor notation}\ru{в~прямой инвариантной тензорной з\'{а}писи}\ru{:}\en{ is}

\nopagebreak \vspace{-0.2em}
\begin{equation}\label{lineartheory:wholesetofequations}
\boldnabla \dotp \linearstress \hspace{.15ex} + \bm{f} = \hspace{.1ex} \bm{0} \hspace{.1ex} ,
\hspace{.7em}
%
\linearstress = \scalebox{0.92}[0.92]{$ \displaystyle \frac{\raisemath{-0.125em}{\partial\hspace{.1ex} \Pi^{\mathstrut}}}{\partial \mathboldepsilon}$} = \stiffnesstensor \dotdotp \mathboldepsilon ,
\hspace{.7em}
%
\mathboldepsilon = \hspace{-0.2ex} \boldnabla {\bm{u}}^{\hspace{.1ex}\mathsf{S}} \hspace{-0.2ex}.
\vspace{.2em}
\end{equation}

\nopagebreak \vspace{-0.1em} \en{\vspace{-0.1em}} \noindent
\en{Here}\ru{Здесь}
$\linearstress$\ru{\:---}\en{~is} \en{linear stress tensor}\ru{линейный тензор напряжения},
$\bm{f}$~\en{is vector of~volume loads}\ru{--- вектор объёмных нагрузок},
$\mathboldepsilon$\ru{\:---}\en{~is} \en{infinitesimal linear deformation tensor}\ru{тензор бесконечномалой линейной деформации},
${\Pi(\hspace{-0.1ex}\mathboldepsilon\hspace{-0.1ex})\hspace{-0.1ex}}$\ru{\:---}\en{~is} \en{potential energy of~deformation per volume unit}\ru{потенциальная энергия деформации единицы объёма},
${\stiffnesstensor}$\ru{\:---}\en{~is} \en{stiffness tensor}\ru{тензор жёсткости} (\en{tetravalent}\ru{четырёхвалентный}, \en{with symmetry}\ru{с~симметрией} ${\stiffnesstensor_{\hspace{.12ex} 12 \scalebox{0.6}[0.8]{$\rightleftarrows$} 34} \hspace{-0.4ex} = \hspace{-0.1ex} \stiffnesstensor}$\hbox{\hspace{.1ex},} ${\stiffnesstensor_{\hspace{.12ex} 1 \scalebox{0.6}[0.8]{$\rightleftarrows$} 2} \hspace{-0.33ex} = \hspace{-0.1ex} \stiffnesstensor}$\hbox{\hspace{.1ex},} ${\stiffnesstensor_{\hspace{.12ex} 3 \scalebox{0.6}[0.8]{$\rightleftarrows$} 4} \hspace{-0.33ex} = \hspace{-0.1ex} \stiffnesstensor}$).

\en{Equations}\ru{Уравнения}~\eqref{lineartheory:wholesetofequations} \en{are exact}\ru{точные}, \en{they can be derived via variation}\ru{они могут быть получены варьированием} \en{of equations of the~nonlinear theory}\ru{уравнений нелинейной теории}.
\en{Variation from an~arbitrary configuration}\ru{Варьирование от~произвольной конфигурации} \en{is~described in}\ru{описано в}~\chapdotpararef{chapter:nonlinearcontinuum}{para:variationofconfiguration}.
\en{The~linear theory}\ru{Линейная теория}\ru{\:---}\en{ is} \en{the~result of variation}\ru{результат варьирования} \en{from the~initial unstressed configuration}\ru{от~начальной ненапряжённой конфигурации}, \en{where}\ru{где}

\nopagebreak\begin{equation}
\begin{array}{c}
\bm{F} = \bm{E} \hspace{.1ex} ,
\:\;
\bm{C} \hspace{-0.1ex} = {\hspace{-0.2ex}^2\bm{0}}
\hspace{.1ex} ,
\:\;
\variation{\hspace{.12ex}\bm{C}} \hspace{-0.1ex} = \hspace{-0.2ex} \boldnabla \hspace{.12ex} \variation{\bm{R}}^{\hspace{.2ex}\mathsf{S}} \hspace{-0.15ex}
\equiv \varbivalent{\mathboldepsilon} ,
\\[.2em]
%
\cauchystress = {\hspace{-0.2ex}^2\bm{0}}
\hspace{.1ex} ,
\:\;
\varbivalent{\hspace{-0.2ex}\cauchystress}
= \variation{\hspace{.1ex}\firstpiolakirchhoffstress}
= \scalebox{0.9}[0.9]{$ \displaystyle \frac{\raisemath{-0.125em}{\partial^2 \hspace{.1ex} \Pi}}{\raisemath{-0.1em}{\partial \hspace{.1ex} \bm{C} \hspace{.1ex} \partial \hspace{.1ex} \bm{C}}} $} \hspace{-0.1ex}
\dotdotp \variation{\hspace{.12ex}\bm{C}} ,
\:\:
\boldnabla \dotp \varbivalent{\hspace{-0.2ex}\cauchystress} \hspace{.1ex}
+ \hspace{.1ex} \rho \hspace{.25ex} \variation{\hspace{-0.2ex}\bm{f}}
= \hspace{.1ex} \bm{0}
\hspace{.1ex} .
\end{array}
\end{equation}

\vspace{-0.1em} \noindent
\en{I\kern-0.12ext remains to~change}\ru{Остаётся поменять}
${\variation{\bm{R}}}$ \en{to}\ru{на}~$\bm{u}$,
$\varbivalent{\mathboldepsilon\hspace{-0.16ex}}$ \en{to}\ru{на}~$\mathboldepsilon$,
$\varbivalent{\hspace{-0.16ex}\cauchystress}$ \en{to}\ru{на}~$\linearstress$,
${\scalebox{0.95}[0.95]{$ \raisemath{.16em}{\scalebox{0.92}[0.92]{$\partial^2 \hspace{.1ex} \Pi$}} / \hspace{-0.1ex} \raisemath{-0.32em}{\scalebox{0.92}[0.92]{$\partial \hspace{.1ex} \bm{C} \hspace{.1ex} \partial \hspace{.1ex} \bm{C}$}} $}}$ \en{to}\ru{на}~$\stiffnesstensor$,
\en{and}\ru{а}~${\hspace{-0.1ex} \rho \hspace{.25ex} \variation{\hspace{-0.2ex}\bm{f}}}$ \en{to}\ru{на}~${\hspace{-0.1ex} \bm{f}\hspace{-0.2ex}}$.

\begin{otherlanguage}{russian}

Если такой вывод кажется читателю мало\-нагляд\-ным, возможно исходить из~уравнений

\nopagebreak\vspace{-0.25em}\begin{equation}\label{nonlinear:setofequations}
\begin{array}{c}
\boldnabla \dotp \cauchystress \hspace{.15ex} + \rho \bm{f} = \hspace{.1ex} \bm{0} \hspace{.1ex}, \:\:
\boldnabla = \bm{F}^{-\T} \hspace{-0.2ex} \dotp \boldnablacircled , \:\:
\bm{F} = \bm{E} \hspace{.1ex} + \hspace{-0.1ex} \boldnablacircled {\bm{u}}^{\T} \hspace{-0.3ex},
\\[.2em]
\cauchystress \hspace{.1ex} = J^{-1} \hspace{0.2ex} \bm{F} \dotp \displaystyle \frac{\partial\hspace{.1ex} \Pi}{\partial\hspace{.1ex} \bm{C}} \dotp \bm{F}^{\hspace{.1ex}\T} \hspace{-0.4ex}, \:\:
\bm{C} = \boldnablacircled {\bm{u}}^{\hspace{.1ex}\mathsf{S}} \hspace{-0.2ex} + \hspace{0.2ex} \displaystyle \onehalf \hspace{0.2ex} \boldnablacircled \bm{u} \dotp \hspace{-0.25ex} \boldnablacircled \bm{u}^{\T} \hspace{-0.25ex}.
\end{array}
\end{equation}

\vspace{-0.1em}\noindent
Полагая перемещение~$\bm{u}$ м\'{а}лым, перейдём от~\eqref{nonlinear:setofequations} к~\eqref{lineartheory:wholesetofequations}.

Или вот как.
Вместо $\bm{u}$ взять $\smallparameter \bm{u}$, тут ${\smallparameter \hspace{-0.1ex}\to 0}$\:--- некоторый весьма малый параметр.
А~неизвестные представить рядами по~целым степеням~$\smallparameter$

\nopagebreak\vspace{-0.1em}\begin{equation*}
\begin{array}{c}
\cauchystress \hspace{.1ex} = \cauchystress^{\hspace{0.2ex}\scalebox{0.66}[0.66]{(0)}} \hspace{-0.2ex} + \smallparameter \cauchystress^{\hspace{0.2ex}\scalebox{0.66}[0.66]{(1)}} \hspace{-0.1ex} + \ldots \hspace{.1ex}, \:\:
\bm{C} = \bm{C}^{\hspace{0.2ex}\scalebox{0.66}[0.66]{(0)}} \hspace{-0.2ex} + \smallparameter \bm{C}^{\hspace{0.2ex}\scalebox{0.66}[0.66]{(1)}} \hspace{-0.1ex} + \ldots \hspace{.1ex} ,
\\[.1em]
\boldnabla \hspace{.1ex} = \boldnablacircled \hspace{.1ex} + \smallparameter \boldnabla^{\hspace{0.2ex}\scalebox{0.66}[0.66]{(1)}} \hspace{-0.1ex} + \ldots \hspace{.1ex} , \:\:
\bm{F} = \bm{E} + \hspace{-0.1ex} \smallparameter \boldnablacircled {\bm{u}}^{\T} \hspace{-0.3ex}, \:\:
J = 1 + \hspace{-0.1ex} \smallparameter J^{\hspace{.1ex}\scalebox{0.66}[0.66]{(1)}} \hspace{-0.1ex} + \ldots
\end{array}
\end{equation*}

\vspace{-0.1em}\noindent
Для первых~(\inquotes{нулевых}) членов этих разложений и~получается~\eqref{lineartheory:wholesetofequations}.
В~\hbox{книге~\cite{truesdell-firstcourse}} си\'{е} названо \inquotesx{формальным приближением}[.]

Невозможно сказать в~общем случае, насколько мал должен~быть параметр~$\smallparameter$\:--- ответ зависит от~ситуации и~определяется лишь тем, описывает линейная модель интересующий эффект или~нет.
Когда, например, интересна связь частот\'{ы} свободных колебаний упругого тела с~амплитудой, то нужна уж\'{е} нелинейная модель.

Линейная задача ставится в~начальном объёме~\hbox{$\mathcal{V} \hspace{-0.2ex} = \hspace{-0.2ex} \mathcircabove{\mathcal{V}}$\hspace{-0.12em}}, ограниченном поверхностью~$o$ с~вектором пл\'{о}щади~${\bm{n} do}$ (\inquotes{\en{the~principle of initial dimensions}\ru{принцип начальных размеров}}).

Краевые~(граничные) условия чаще всего такие: на~\hbox{части~${o_1}$} поверхности известны перемещения~(\en{kinematic boundary condition}\ru{кинематическое краевое условие}), а~на~другой \hbox{части~${o_2}$}\:--- силы~(\en{static boundary condition}\ru{статическое краевое условие})

\nopagebreak\vspace{-0.2em}\begin{equation}\label{lineartheory:boundaryconditions}
\bm{u} \hspace{.1ex} \bigr|_{o_1} \hspace{-0.64ex} = \hspace{.2ex} \bm{u}_{\raisemath{-0.1em}{0}}
\hspace{.2ex} ,
\hspace{.8em}
\bm{n} \dotp \linearstress \hspace{.2ex} \bigr|_{o_2} \hspace{-0.64ex} = \hspace{.2ex} \bm{p}
\hspace{.2ex} .
\end{equation}

Но встречаются и~более сложные комбинации, если знаем одно\-времен\-но некоторые компоненты как~$\bm{u}$, так~и~${\tractionvector{n} \hspace{-0.2ex} = \bm{n} \dotp \linearstress}$.
Для примера, на~плоской грани~${x = \constant}$ при~вдавливании штампа с~гладкой поверхностью ${u_x \hspace{-0.25ex} = v (y , \hspace{-0.1ex} z)}$, ${\mathtau_{xy} \hspace{-0.2ex} = \mathtau_{xz} \hspace{-0.2ex} = 0}$ (функция~$v$ определяется формой штампа).

Начальные условия в~динамических задачах, когда вместо~$\bm{f}$ имеем ${\bm{f} \hspace{-0.1ex} - \hspace{-0.1ex} \rho \hspace{.1ex} \mathdotdotabove{\bm{u}}}$, ставятся как обычно в~механике\:--- на~положения и~на~скорости: в~условный момент времени~${t \narroweq 0}$ определены~$\bm{u}$ и~$\mathdotabove{\bm{u}}$.

Как отмечалось в~\chapref{chapter:genericmechanics}, в~основу механики может быть положен принцип виртуальной работы~(\hbox{d’\hspace{-0.2ex}Alembert--Lagrange} principle).
Этот принцип справедлив и~в~линейной теории (внутренние силы в~упругой среде потенциальны: ${\variation{\internalwork} = - \hspace{.2ex} \variation{\Pi}}$)

\nopagebreak\vspace{-0.2em}\begin{equation}\label{lineartheory:principleofvirtualwork}
\integral\displaylimits_{\mathcal{V}} \hspace{-0.4ex} \left[^{\mathstrut} \hspace{-0.2ex} \left( \bm{f} \hspace{-0.1ex} - \hspace{-0.1ex} \rho \hspace{.1ex} \mathdotdotabove{\bm{u}} \hspace{.2ex} \right) \dotp \variation{\bm{u}} - \variation{\Pi} \hspace{.25ex} \right] \hspace{-0.32ex} d\mathcal{V} + \hspace{-0.2ex}
\integral\displaylimits_{o_2} \hspace{-0.25ex} \bm{p} \dotp \variation{\bm{u}} \hspace{.25ex} do \hspace{.2ex}
= \hspace{.1ex} 0
\hspace{.1ex} ,
\hspace{.6em}
%
\bm{u} \hspace{.1ex} \bigr|_{o_1} \hspace{-0.64ex} = \hspace{.1ex} \bm{0}
\hspace{.1ex},
\end{equation}

\noindent поскольку
\nopagebreak\vspace{.2em}\[
\begin{array}{c}
\variation{\Pi} = \hspace{.1ex} \smash{\smalldisplaystyleonehalf} \hspace{0.25ex} \variation{ \hspace{-0.2ex} \left( \mathboldepsilon \hspace{-0.1ex} \dotdotp \hspace{-0.1ex} \stiffnesstensor \dotdotp \mathboldepsilon \right) } \hspace{-0.2ex}
= \linearstress \dotdotp \variation{\mathboldepsilon}
= \linearstress \dotdotp \hspace{-0.12ex} \boldnabla \variation{\bm{u}}^{\hspace{.1ex}\mathsf{S}} \hspace{-0.16ex}
= \boldnabla \hspace{-0.1ex} \dotp \left( \hspace{.1ex} \linearstress \hspace{-0.1ex} \dotp \variation{\bm{u}} \hspace{.1ex} \right) - \boldnabla \dotp \linearstress \dotp \variation{\bm{u}} \hspace{.1ex},
\\[.4em]
%
\displaystyle \integral\displaylimits_{\mathcal{V}} \hspace{-0.5ex} \variation{\Pi} \hspace{.2ex} d\mathcal{V} =
\ointegral\displaylimits_{\mathclap{o\hspace{.15ex}(\boundary \mathcal{V})}} \hspace{-0.2ex} \bm{n} \hspace{.1ex} \dotp \linearstress \dotp \variation{\bm{u}} \hspace{.32ex} do \hspace{.2ex} - \hspace{-0.1ex}
\integral\displaylimits_{\mathcal{V}} \hspace{-0.5ex} \boldnabla \dotp \linearstress \dotp \variation{\bm{u}} \hspace{.32ex} d\mathcal{V}
\end{array}
\]

\vspace{-0.2em} \noindent и~левая часть~\eqref{lineartheory:principleofvirtualwork} приобретает вид
\[
\integral\displaylimits_{\mathcal{V}} \hspace{-0.5ex} \left(^{\mathstrut} \hspace{-0.2ex} \boldnabla \dotp \linearstress \hspace{.12ex} + \bm{f} \hspace{-0.1ex} - \hspace{-0.1ex} \rho \hspace{.1ex} \mathdotdotabove{\bm{u}} \hspace{0.25ex} \right) \hspace{-0.32ex} \dotp \variation{\bm{u}} \hspace{0.25ex} d\mathcal{V} + \hspace{-0.2ex}
\integral\displaylimits_{o_2} \hspace{-0.5ex} \left(^{\mathstrut} \bm{p} - \bm{n} \dotp \linearstress \right) \hspace{-0.32ex} \dotp \variation{\bm{u}} \hspace{.25ex} do
\hspace{.1ex} ,
\]

\vspace{-0.4em} \noindent что, конечно~же, равно нулю.
Отметим краевое условие ${\bm{u} \hspace{.1ex} \bigr|_{o_1} \hspace{-0.64ex} = \hspace{.2ex} \bm{0}}$: виртуальные перемещения согласованы с~этой связью\:--- ${\variation{\bm{u}} \hspace{.1ex} \bigr|_{o_1} \hspace{-0.64ex} = \hspace{.2ex} \bm{0}}$.

\end{otherlanguage}

\en{\section{Uniqueness of the solution of dynamic problem}}

\ru{\section{Уникальность решения динамической проблемы}}

\label{para:uniquenessfordynamicproblem}

\begin{otherlanguage}{russian}

Как~обычно в~линейной математической физике~\cite{lurie-theoryofelasticity, tihonovsamarsky-mathphysicsequations}, теорема единственности доказывается \inquotesx{от~противного}[.]
Допустим, что существуют какие-либо два решения: ${\bm{u}_1 (\bm{r}, t)}$ \en{and}\ru{и}~${\bm{u}_2 (\bm{r}, t)}$.
Если разность~${\bm{u}^{\hspace{-0.1ex}*} \hspace{-0.2ex} \equiv \hspace{.12ex} \bm{u}_1 \hspace{-0.4ex} - \bm{u}_2}$ окажется равной~$\bm{0}$, тогда эти решения совпадают, то есть решение единственно.

Но~сперв\'{а} убедимся в~существовании интеграла энергии\:--- выведем уравнение баланса механической энергии для~линейной модели

\nopagebreak\vspace{-0.8em}\begin{equation}\label{dynamicsoflineartheory:integralofenergy}
\displaystyle\integral\displaylimits_{\mathcal{V}} \hspace{-0.5ex} \left(^{\mathstrut} \hspace{-0.2ex} \mathrm{T} + \Pi \right)^{\hspace{-0.12em}\tikz[baseline=-0.2ex] \draw[black, fill=black] (0,0) circle (.28ex);} \hspace{-0.25ex} d\mathcal{V} \hspace{.1ex} =
\displaystyle\integral\displaylimits_{\mathcal{V}} \hspace{-0.5ex} \bm{f} \hspace{-0.1ex} \dotp \mathdotabove{\bm{u}} \hspace{0.25ex} d\mathcal{V} + \hspace{-0.12ex}
\displaystyle\integral\displaylimits_{o_2} \hspace{-0.32ex} \bm{p} \dotp \mathdotabove{\bm{u}} \hspace{.25ex} do
\hspace{.1ex} ,
\end{equation}
%
\nopagebreak\vspace{-0.2em}\begin{equation*}
\begin{array}{c}
\bm{u} \hspace{.1ex} \bigr|_{o_1} \hspace{-0.64ex} = \hspace{.2ex} \bm{0}
\hspace{.1ex} ,
\hspace{.6em}
%
\bm{n} \dotp \linearstress \hspace{.2ex} \bigr|_{o_2} \hspace{-0.64ex} = \hspace{.2ex} \bm{p}
\hspace{.15ex} ,
\\[.5em]
%
\bm{u} \hspace{.1ex} \bigr|_{t=0} \hspace{-0.2ex} = \bm{u}^{\hspace{-0.1ex}\circ}
\hspace{-0.4ex} ,
\hspace{.6em}
%
\mathdotabove{\bm{u}} \hspace{.1ex} \bigr|_{t=0} \hspace{-0.2ex} = \mathdotabove{\bm{u}}^{\circ}
\hspace{-0.4ex} .
\end{array}
\end{equation*}

\vspace{-0.2em} Слева имеем

\nopagebreak\vspace{-0.4em}\begin{equation*}\begin{array}{c}
\mathdotabove{\mathrm{T}} =^{\displaystyle \mathstrut^{\displaystyle \mathstrut}} \smalldisplaystyleonehalf \hspace{-0.25ex} \left( \hspace{.15ex} \rho \hspace{.2ex} \mathdotabove{\bm{u}} \dotp^{\mathstrut} \hspace{-0.16ex} \mathdotabove{\bm{u}} \hspace{.15ex} \right)^{\tikz[baseline=-0.2ex] \draw[black, fill=black] (0,0) circle (.28ex);} \hspace{-0.25ex}
= \smalldisplaystyleonehalf \hspace{.2ex} \rho \hspace{-0.1ex} \left( \mathdotabove{\bm{u}} \dotp \mathdotdotabove{\bm{u}} + \mathdotdotabove{\bm{u}} \dotp \mathdotabove{\bm{u}} \hspace{.1ex} \right)
\hspace{-0.12ex} =
\rho \hspace{.2ex} \mathdotdotabove{\bm{u}} \dotp \mathdotabove{\bm{u}}
\hspace{.2ex} ,
\\[.5em]
%
\mathdotabove{\Pi} \hspace{.1ex} = \hspace{.1ex} \smash{\smalldisplaystyleonehalf} \hspace{-0.25ex} \tikzmark{beginDerivativeOfElasticPotential} \left( \hspace{.1ex} \mathboldepsilon \hspace{-0.1ex} \dotdotp \hspace{-0.1ex} \stiffnesstensor \dotdotp \mathboldepsilon \hspace{.1ex} \right)^{\tikz[baseline=-0.2ex] \draw[black, fill=black] (0,0) circle (.28ex);} \hspace{-1.2ex} \tikzmark{endDerivativeOfElasticPotential} \hspace{.8ex}
= \linearstress \dotdotp \mathdotabove{\mathboldepsilon}
= \linearstress \hspace{.1ex} \dotdotp \hspace{-0.12ex} \boldnabla \mathdotabove{\bm{u}}^{\hspace{.1ex}\mathsf{S}} \hspace{-0.12ex}
= \boldnabla \hspace{-0.1ex} \dotp \left( \hspace{.1ex} \linearstress \dotp \mathdotabove{\bm{u}} \hspace{.15ex} \right) -
\tikzmark{NablaDotTauDynamicsBegin} \boldnabla \dotp \linearstress \tikzmark{NablaDotTauDynamicsEnd} \dotp \mathdotabove{\bm{u}} =
\\[.4em]
%
\hspace*{4em}
= \boldnabla \hspace{-0.1ex} \dotp \left( \hspace{.1ex} \linearstress \dotp \mathdotabove{\bm{u}} \hspace{.15ex} \right)
+ ( \hspace{.12ex} \bm{f} \hspace{-0.2ex} - \hspace{-0.2ex} \rho \hspace{.1ex} \mathdotdotabove{\bm{u}} \hspace{.15ex} ) \hspace{-0.1ex} \dotp \mathdotabove{\bm{u}}
%%\hspace{.2ex} .
\end{array}\end{equation*}%
\AddUnderBrace[line width=.75pt][.5ex, -0.4ex][yshift = .2ex]%
{beginDerivativeOfElasticPotential}{endDerivativeOfElasticPotential}%
{${\scalebox{0.75}{$ 2 \hspace{.2ex} \mathboldepsilon \hspace{-0.1ex} \dotdotp \hspace{-0.1ex} \stiffnesstensor \dotdotp \mathdotabove{\mathboldepsilon} $}}$}
\AddUnderBrace[line width=.75pt][.16ex, .1ex][yshift = .2ex]%
{NablaDotTauDynamicsBegin}{NablaDotTauDynamicsEnd}%
{${\scalebox{0.75}{$ - \hspace{.2ex} ( \hspace{.12ex} \bm{f} \hspace{-0.2ex} - \hspace{-0.2ex} \rho \hspace{.1ex} \mathdotdotabove{\bm{u}} \hspace{.15ex} ) $}}$}

\nopagebreak \vspace{-0.6em} \noindent
(использован баланс импульса ${\hspace{-0.24ex}\boldnabla \dotp \linearstress \hspace{.15ex} + \bm{f} - \rho \hspace{.1ex} \mathdotdotabove{\bm{u}} \hspace{.1ex} = \bm{0}}$),

\nopagebreak\vspace{-0.2em}\begin{equation*}
\mathdotabove{\mathrm{T}} + \mathdotabove{\Pi} \hspace{.1ex}
= \boldnabla \hspace{-0.1ex} \dotp \left( \hspace{.1ex} \linearstress \dotp \mathdotabove{\bm{u}} \hspace{.15ex} \right)
+ \bm{f} \hspace{-0.1ex} \dotp \mathdotabove{\bm{u}}
\hspace{.2ex} .
\end{equation*}

\vspace{-0.2em} \noindent
Применяя далее теорему о~дивергенции

\nopagebreak\vspace{-0.2em}\begin{equation*}
\scalebox{.9}{$ \displaystyle \integral\displaylimits_{\mathcal{V}} $} \hspace{-0.2ex} \boldnabla \hspace{-0.1ex} \dotp \left( \hspace{.1ex} \linearstress \dotp \mathdotabove{\bm{u}} \hspace{.15ex} \right) \hspace{-0.15ex} d\mathcal{V}
=
\scalebox{.9}{$ \displaystyle \ointegral\displaylimits_{\mathclap{o\hspace{.15ex}(\boundary \mathcal{V})}} $} \hspace{.1ex} \bm{n} \hspace{.1ex} \dotp \linearstress \dotp \mathdotabove{\bm{u}} \hspace{.4ex} do
\end{equation*}

\vspace{-0.33em}\noindent
и~краевое условие ${\bm{n} \dotp \linearstress = \bm{p}}$ на~$o_2$, убеждаемся, что~\eqref{dynamicsoflineartheory:integralofenergy} удовлетворено.

Из~\eqref{dynamicsoflineartheory:integralofenergy} следует, что без нагрузок (когда нет ни объёмных, ни поверхностных сил) полная механическая энергия постоянна:
\nopagebreak\begin{equation}\label{dynamicsoflineartheory:fullenergyisconstantwithoutloads}
\bm{f} \hspace{-0.1ex} = \bm{0}
\hspace{1.1ex} \text{\en{and}\ru{и}} \hspace{1.2ex}
\bm{p} = \bm{0}
%
\hspace{.8ex} \Rightarrow \hspace{.55ex}
%
\scalebox{.8}{$\displaystyle \integral\displaylimits_{\mathcal{V}}$} \hspace{-0.2ex} \bigl( \hspace{.1ex} \mathrm{T} + \Pi \hspace{.1ex} \bigr) \hspace{.1ex} d\mathcal{V} = \hspace{.1ex} \constant
\hspace{.2ex} .
\vspace{-0.25em}
\end{equation}

\vspace{-0.1em}\noindent
\en{If}\ru{Если} \en{at the~initial moment}\ru{в~начальный момент} \en{there was}\ru{был} \en{unstressed rest}\ru{ненапряжённый покой}, \en{then}\ru{то}
\nopagebreak\begin{equation*}
\scalebox{.8}{$\displaystyle \integral\displaylimits_{\mathcal{V}}$} \hspace{-0.2ex} \bigl( \hspace{.1ex} \mathrm{T} + \Pi \hspace{.1ex} \bigr) \hspace{.1ex} d\mathcal{V} = \hspace{.1ex} 0
\hspace{.1ex} .
\tag{\theequation \raisebox{.1em}{\textquotesingle}}\label{dynamicsoflineartheory:fullenergyiszero}
\vspace{-0.25em}
\end{equation*}

\vspace{-0.1em}
Кинетическая энергия положительна: ${\mathrm{T} \hspace{-0.2ex} > \hspace{-0.2ex} 0}$ при~${\mathdotabove{\bm{u}} \hspace{-0.1ex} \neq \hspace{-0.1ex} \bm{0}}$ и~обращается в~нуль лишь когда~${\mathdotabove{\bm{u}} = \bm{0}}$\:--- это следует из~самог\'{о} определения ${\mathrm{T} \equiv \smalldisplaystyleonehalf \hspace{.2ex} \rho \hspace{0.2ex} \mathdotabove{\bm{u}} \dotp \mathdotabove{\bm{u}}}$.
Потенциальная энергия, при~м\'{а}лых деформациях с~тензором~$\mathboldepsilon$ представляемая квадратичной формой
${\Pi(\hspace{-0.1ex}\mathboldepsilon\hspace{-0.1ex}) \hspace{-0.2ex} = \hspace{.1ex} \smash{\smalldisplaystyleonehalf} \hspace{.25ex} \mathboldepsilon \hspace{-0.1ex} \dotdotp \hspace{-0.1ex} \stiffnesstensor \dotdotp \mathboldepsilon}$, тоже положительна: ${\Pi \hspace{-0.16ex} > \hspace{-0.2ex} 0}$ при~${\mathboldepsilon \hspace{-0.1ex} \neq \hspace{-0.1ex} {^2\bm{0}}}$.
\en{Such is}\ru{Таков\'{о}} \en{a~priori requirement}\ru{априорное требование} \en{of~positive definiteness}\ru{положительной определённости} \en{for}\ru{для}~\en{stiffness tensor}\ru{тензора жёсткости}~$\stiffnesstensor$.
Это одно из \inquotes{дополнительных неравенств в~теории упругости}~\cite{lurie-nonlinearelasticity, truesdell-firstcourse}.

\en{And since}\ru{А~поскольку}~$\mathrm{T}$ \en{and}\ru{и}~$\Pi$ \en{are positive}\ru{положительны}, \en{from}\ru{из}~\eqref{dynamicsoflineartheory:fullenergyiszero} \en{ensues}\ru{вытекает}

\nopagebreak\vspace{-0.2em}\begin{equation*}
\mathrm{T} = 0 \hspace{.1ex}, \; \Pi = 0
\hspace{.7ex} \Rightarrow \hspace{.7ex}
%
\mathdotabove{\bm{u}} = \bm{0}
\hspace{.1ex} , \;
\mathboldepsilon = \hspace{-0.25ex} \boldnabla {\bm{u}}^{\hspace{.1ex}\mathsf{S}} \hspace{-0.2ex} = {^2\bm{0}}
\hspace{.8ex} \Rightarrow \hspace{.7ex}
%
\bm{u} = \bm{u}^{\hspace{-0.1ex}\circ} \hspace{-0.25ex} + \hspace{.1ex} \bm{\vartheta}^{\circ} \hspace{-0.5ex} \times \bm{r} ,
\end{equation*}

\vspace{-0.25em} \noindent
\en{where}\ru{где}~${\bm{u}^{\hspace{-0.1ex}\circ} \hspace{-0.32ex}}$ \en{and}\ru{и}~${\bm{\vartheta}^{\circ} \hspace{-0.5ex}}$\en{ are}\ru{\;---} \en{some constants}\ru{некоторые константы}.
\en{With fixing}\ru{С~закреплением} \en{on}\ru{на}~${o_1}$

\nopagebreak\vspace{-0.22em}\begin{equation*}
\bm{u} \hspace{.16ex} \rvert_{\raisemath{-0.1em}{o_1}} \hspace{-0.5ex} = \hspace{.1ex} \bm{0}
\hspace{.8ex} \Rightarrow \hspace{.7ex}
%
\bm{u}^{\hspace{-0.1ex}\circ} \hspace{-0.3ex} = \bm{0}
\hspace{1.1ex} \text{\en{and}\ru{и}} \hspace{1.1ex}
\bm{\vartheta}^{\circ} \hspace{-0.3ex} = \bm{0}
\hspace{.8ex} \Rightarrow \hspace{.7ex}
%
\bm{u} = \bm{0} \hspace{1ex} \text{\en{everywhere}\ru{всюду}} .
\end{equation*}

\vspace{-0.2em}
Теперь вспомним о~двух решениях~${\bm{u}_1\hspace{-0.25ex}}$ \en{and}\ru{и}~${\bm{u}_2}$.
Разность ${\bm{u}^{\hspace{-0.1ex}*} \hspace{-0.25ex} \equiv \hspace{.1ex} \bm{u}_1 \hspace{-0.36ex} - \bm{u}_2}$ есть решение полностью однородной задачи (в~объёме ${\bm{f} \hspace{-0.1ex} = \bm{0}}$, в~краевых и~в~начальных условиях\:--- нули).
\en{Therefore}\ru{Поэтому} ${\bm{u}^{\hspace{-0.1ex}*} \hspace{-0.32ex} = \bm{0}}$\:--- единственность доказана.

Что~же касается существования решения, то простыми выкладками его в~общем случае не~обосновать.
Отметим лишь, что динамическая задача является эволюционной, то~есть описывает развитие процесса во~времени.
Из~баланса импульса находим ускорение~$\mathdotdotabove{\bm{u}}$, далее переходим на~\inquotes{следующий временной слой} ${t + dt}$:

...


Разумеется, эти соображения лишены математической точности, характерной, например, для~монографии Philippe Ciarlet~\cite{ciarlet-mathematicalelasticity}.

\end{otherlanguage}

\en{\section{Hooke’s ceiiinosssttuv law}}

\ru{\section{Закон Гука ceiiinosssttuv}}

\label{para:hookelaw}

\begin{otherlanguage}{russian}

\nopagebreak\vspace{-2.4em} ${\small \linearstress \hspace{.1ex} = \displaystyle \frac{\raisemath{-0.2em}{\partial\hspace{.1ex} \Pi}}{\raisemath{.04em}{\partial \mathboldepsilon}} = \stiffnesstensor \dotdotp \mathboldepsilon = \mathboldepsilon \dotdotp \stiffnesstensor}$ \nopagebreak\vspace{.55em}

\nopagebreak
То~соотношение между напряжением и~деформацией, которое Robert Hooke в~\hbox{XVII$^{\textrm{ом}}$\hspace{-0.12em}}~веке мог высказать лишь в~весьма неопределённой форме\footnote{\hspace*{.2em}\inquotes{ceiiinosssttuv, that is Ut tensio sic vis; ...}}\hspace{-0.32em},\hspace{.24em} в~современных обозначениях записано в~\eqref{lineartheory:wholesetofequations} и~определяется тензором

\nopagebreak\vspace{-0.1em}\begin{equation}
\stiffnesstensor =
\displaystyle \frac{\partial^2 \hspace{.1ex} \Pi}{\raisemath{-0.1em}{\partial \mathboldepsilon \hspace{.1ex} \partial \mathboldepsilon}} = \hspace{-0.1ex}
A^{i\hspace{-0.1ex}jk\hspace{.06ex}l} \hspace{.16ex} \bm{r}_i \bm{r}_{\hspace{-0.12ex}j} \bm{r}_k \bm{r}_l
\hspace{.1ex} ,
\:\:
%
A^{i\hspace{-0.1ex}jk\hspace{.06ex}l} \hspace{-0.16ex} =
\displaystyle \frac{\partial^2 \hspace{.1ex} \Pi}{\raisemath{-0.1em}{\partial \varepsilon_{i\hspace{-0.1ex}j} \hspace{.1ex} \partial \varepsilon_{k\hspace{.06ex}l}}}
\hspace{.16ex} .
\end{equation}

\begin{comment} %%
\[\scalebox{0.8}{$\begin{array}{c}
\scalebox{1.2}{$A^{abcd} = A^{cdab}$}
\\[.2em]
%
A^{1112} = A^{1211} \\
A^{1113} = A^{1311} \\
A^{1121} = A^{2111} \\
A^{1122} = A^{2211} \\
A^{1123} = A^{2311} \\
A^{1131} = A^{3111} \\
A^{1132} = A^{3211} \\
A^{1133} = A^{3311} \\
%
A^{1213} = A^{1312} \\
A^{1221} = A^{2112} \\
A^{1222} = A^{2212} \\
A^{1223} = A^{2312} \\
A^{1231} = A^{3112} \\
A^{1232} = A^{3212} \\
A^{1233} = A^{3312} \\
%
A^{1321} = A^{2113} \\
A^{1322} = A^{2213} \\
A^{1323} = A^{2313} \\
A^{1331} = A^{3113} \\
A^{1332} = A^{3213} \\
A^{1333} = A^{3313} \\
%
A^{2122} = A^{2221} \\
A^{2123} = A^{2321} \\
A^{2131} = A^{3121} \\
A^{2132} = A^{3221} \\
A^{2133} = A^{3321} \\
%
A^{2223} = A^{2322} \\
A^{2231} = A^{3122} \\
A^{2232} = A^{3222} \\
A^{2233} = A^{3322} \\
%
A^{2331} = A^{3123} \\
A^{2332} = A^{3223} \\
A^{2333} = A^{3323} \\
%
A^{3132} = A^{3231} \\
A^{3133} = A^{3331} \\
%
A^{3233} = A^{3332} \\
\end{array}$}\]
\end{comment} %%

\vspace{-0.1em} Тензор жёсткости, как частная производная упругого потенциала~$\Pi$ дважды по~тому~же тензору~$\mathboldepsilon$, симметричен по~первой и~второй паре индексов:
${\stiffnesstensor_{\hspace{.12ex} 12 \scalebox{0.6}[0.8]{$\rightleftarrows$} 34} \hspace{-0.25ex} = \hspace{-0.1ex} \stiffnesstensor \hspace{.5ex} \Leftrightarrow \hspace{.2ex} A^{i\hspace{-0.1ex}jk\hspace{.06ex}l} \hspace{-0.17ex} = A^{k\hspace{.06ex}l\hspace{.06ex}i\hspace{-0.1ex}j}\hspace{-0.25ex}}$,
от~этого у~36 компонент из~${3^4 \hspace{-0.25ex} = \hspace{-0.12ex} 81}$ \inquotesx{есть двойн\'{и}к}[,] а~лишь~45 независимы.
К~тому~же, из\hbox{-}за симметрии тензора линейной деформации~$\mathboldepsilon$, тензор~$\stiffnesstensor$ симметричен ещё~и~внутри каждой пары индексов: ${A^{i\hspace{-0.1ex}j\hspace{-0.06ex}k\hspace{.06ex}l} \hspace{-0.16ex} = A^{j\hspace{-0.06ex}ik\hspace{.06ex}l} \hspace{-0.16ex} = A^{i\hspace{-0.1ex}jlk}}$~(${= A^{j\hspace{-0.06ex}i\hspace{.06ex}l\hspace{-0.06ex}k}}$). Число независимых компонент при~этом снижается до~21:

\nopagebreak\vspace{-0.25em}\begin{equation*}\scalebox{0.8}{$\begin{array}{l}
\scalebox{1.16}{$A^{abcd^{\mathstrut}} = A^{cdab} = A^{bacd} = A^{abdc}$}
\\[.2em]
%
A^{1\hspace{-0.12ex}1\hspace{-0.12ex}1\hspace{-0.12ex}1} \\
A^{1\hspace{-0.12ex}1\hspace{-0.12ex}12} = A^{1\hspace{-0.12ex}121} = A^{121\hspace{-0.12ex}1} = A^{21\hspace{-0.12ex}1\hspace{-0.12ex}1} \\
A^{1\hspace{-0.12ex}1\hspace{-0.12ex}13} = A^{1\hspace{-0.12ex}131} = A^{131\hspace{-0.12ex}1} = A^{31\hspace{-0.12ex}1\hspace{-0.12ex}1} \\
A^{1\hspace{-0.12ex}122} = A^{221\hspace{-0.12ex}1} \\
A^{1\hspace{-0.12ex}123} = A^{1\hspace{-0.12ex}132} = A^{231\hspace{-0.12ex}1} = A^{321\hspace{-0.12ex}1} \\
A^{1\hspace{-0.12ex}133} = A^{331\hspace{-0.12ex}1} \\
A^{1212} = A^{1221} = A^{21\hspace{-0.12ex}12} = A^{2121} \\
A^{1213} = A^{1231} = A^{1312} = A^{1321} = A^{21\hspace{-0.12ex}13} = A^{2131} = A^{31\hspace{-0.12ex}12} = A^{3121} \\
A^{1222} = A^{2122} = A^{2212} = A^{2221} \\
A^{1223} = A^{1232} = A^{2123} = A^{2132} = A^{2312} = A^{2321} = A^{3212} = A^{3221} \\
A^{1233} = A^{2133} = A^{3312} = A^{3321} \\
A^{1313} = A^{1331} = A^{31\hspace{-0.12ex}13} = A^{3131} \\
A^{1322} = A^{2213} = A^{2231} = A^{3122} \\
A^{1323} = A^{1332} = A^{2313} = A^{2331} = A^{3123} = A^{3132} = A^{3213} = A^{3231} \\
A^{1333} = A^{3133} = A^{3313} = A^{3331} \\
A^{2222} \\
A^{2223} = A^{2232} = A^{2322} = A^{3222} \\
A^{2233} = A^{3322} \\
A^{2323} = A^{2332} = A^{3223} = A^{3232} \\
A^{2333} = A^{3233} = A^{3323} = A^{3332} \\
A^{3333} \\
\end{array}$}
\end{equation*}

\vspace{-0.2em} Нередко компоненты тензора жёсткости записывают симметричной матрицей 6×6 вида
\nopagebreak\vspace{.1em}\[ \displaystyle
\underset{\raisemath{.1em}{\scriptscriptstyle 6×6}}{[\hspace{.2ex}\mathcal{A}\hspace{.2ex}]} \hspace{.12ex} = \hspace{-0.16ex}
%
\scalebox{0.82}{$\left[\hspace{0.4em} {\begin{matrix}
a_{1} & a_{12} & a_{13} & a_{14} & a_{15} & a_{16} \\
{\color{gray}a_{12}} & a_{2} & a_{23} & a_{24} & a_{25} & a_{26} \\
{\color{gray}a_{13}} & {\color{gray}a_{23}} & a_{3} & a_{34} & a_{35} & a_{36} \\
{\color{gray}a_{14}} & {\color{gray}a_{24}} & {\color{gray}a_{34}} & a_{4} & a_{45} & a_{46} \\
{\color{gray}a_{15}} & {\color{gray}a_{25}} & {\color{gray}a_{35}} & {\color{gray}a_{45}} & a_{5} & a_{56} \\
{\color{gray}a_{16}} & {\color{gray}a_{26}} & {\color{gray}a_{36}} & {\color{gray}a_{46}} & {\color{gray}a_{56}} & a_{6}
\end{matrix}} \hspace{0.32em}\right] $}  \hspace{-0.2ex} \equiv \hspace{-0.2ex}
%
\scalebox{0.82}{$\left[\hspace{0.32em} {\begin{matrix}
A^{1\hspace{-0.12ex}1\hspace{-0.12ex}1\hspace{-0.12ex}1} & A^{1\hspace{-0.12ex}122} & A^{1\hspace{-0.12ex}133} & A^{1\hspace{-0.12ex}1\hspace{-0.12ex}12} & A^{1\hspace{-0.12ex}1\hspace{-0.12ex}13} & A^{1\hspace{-0.12ex}123} \\
{\color{gray}A^{221\hspace{-0.12ex}1}} & A^{2222} & A^{2233} & A^{1222} & A^{1322} & A^{2223} \\
{\color{gray}A^{331\hspace{-0.12ex}1}} & {\color{gray}A^{3322}} & A^{3333} & A^{1233} & A^{1333} & A^{2333} \\
{\color{gray}A^{121\hspace{-0.12ex}1}} & {\color{gray}A^{2212}} & {\color{gray}A^{3312}} & A^{1212} & A^{1213} & A^{1223} \\
{\color{gray}A^{131\hspace{-0.12ex}1}} & {\color{gray}A^{2213}} & {\color{gray}A^{3313}} & {\color{gray}A^{1312}} & A^{1313} & A^{1323} \\
{\color{gray}A^{231\hspace{-0.12ex}1}} & {\color{gray}A^{2322}} & {\color{gray}A^{3323}} & {\color{gray}A^{2312}} & {\color{gray}A^{2313}} & A^{2323}
\end{matrix}} \hspace{0.4em}\right] $}
\]

\vspace{.1em} Даже в~декартовых координатах~$x$, $y$, $z$ квадратичная форма упругой энергии $\Pi(\hspace{-0.1ex}\mathboldepsilon\hspace{-0.1ex}) \hspace{-0.2ex}=\hspace{.1ex} \smalldisplaystyleonehalf \hspace{.25ex} \mathboldepsilon \hspace{-0.1ex} \dotdotp \hspace{-0.1ex} \stiffnesstensor \dotdotp \mathboldepsilon$ довольно\hbox{-}таки громоздкая:
\begin{equation}\label{elasticenergylooongcartesian}
\begin{array}{c}
2 \hspace{.1ex} \Pi = a_1 \varepsilon^2_{x} + a_2 \varepsilon^2_{y} + a_3 \varepsilon^2_{z} + a_4 \varepsilon^2_{xy} + a_5 \varepsilon^2_{xz} + a_6 \varepsilon^2_{yz} \:+
\\[.2em]
+\: 2 \hspace{0.2ex} \scalebox{1.25}{$[$} \hspace{0.2ex}
\varepsilon_{x} \hspace{0.2ex} ( a_{12} \varepsilon_{y} + a_{13} \varepsilon_{z} + a_{14} \varepsilon_{xy} + a_{15} \varepsilon_{xz} + a_{16} \varepsilon_{yz} ) \:+
\\[.2em]
+\: \varepsilon_{y} \hspace{0.2ex} ( a_{23} \varepsilon_{z} + a_{24} \varepsilon_{xy} + a_{25} \varepsilon_{xz} + a_{26} \varepsilon_{yz} ) \:+
\\[.2em]
+\: \varepsilon_{z} ( a_{34} \varepsilon_{xy} + a_{35} \varepsilon_{xz} + a_{36} \varepsilon_{yz} ) \:+
\\[.2em]
+\: \varepsilon_{xy} \hspace{0.2ex} ( a_{45} \varepsilon_{xz} + a_{46} \varepsilon_{yz} ) +
a_{56} \varepsilon_{xz} \varepsilon_{yz} \hspace{0.2ex} \scalebox{1.25}{$]$}
\hspace{.1ex} .
\end{array}\end{equation}

\vspace{.1em} Когда добавляется материальная симметрия, число независимых компонент тензора~$\stiffnesstensor$ ещё уменьшается.

Пусть материал имеет плоскость симметрии (упругих свойств) ${z = \constant}$. Тогда энергия~$\Pi$ не~меняется при~перемене знаков у~$\varepsilon_{zx}$ и~$\varepsilon_{zy}$. А~это возможно лишь если
\begin{equation}\label{zeroconstants:oneplaneofsymmetry}
\Pi \hspace{.1ex} \raisemath{-0.2em}{\Bigr\vert}_{\substack{%
\varepsilon_{xz} \hspace{0.2ex}=\hspace{0.2ex} -\varepsilon_{xz} \\
\varepsilon_{yz} \hspace{0.2ex}=\hspace{0.2ex} -\varepsilon_{yz}
}} \hspace{-0.2ex} = \Pi
\:\,\Leftrightarrow \hspace{-3em}
\begin{array}{c}
0 =
a_{15} \hspace{-0.2ex} = a_{16} \hspace{-0.2ex} = a_{25} \hspace{-0.2ex} = a_{26} \hspace{-0.2ex} = \\[-0.1em]
\hspace{8em} = a_{35} \hspace{-0.2ex} = a_{36} \hspace{-0.2ex} = a_{45} \hspace{-0.2ex} = a_{46}
\end{array}
\end{equation}

\noindent --- число независимых констант упало до~13.

Пусть далее плоскостей симметрии две: ${z = \constant}$ и~${y = \constant}$.
Поскольку~$\Pi$ в~таком случае не~чувствительна к~знакам~$\varepsilon_{yx}$ и~$\varepsilon_{yz}$, вдобавок к~\eqref{zeroconstants:oneplaneofsymmetry} имеем

\nopagebreak\vspace{-0.22em}\begin{equation}\label{zeroconstants:2orthogonalplanesofsymmetry:orthotropic}
a_{14} \hspace{-0.2ex} = a_{24} \hspace{-0.2ex} = a_{34} \hspace{-0.2ex} = a_{56} \hspace{-0.2ex} = 0
\end{equation}

\vspace{-0.25em}\noindent --- осталось 9~констант.

Ортотропным~(ортогонально анизотропным) называется материал с~тремя ортогональными плоскостями симметрии\:--- пусть это координатные плоскости~$x$, $y$, $z$.
Легко увидеть, что \eqref{zeroconstants:oneplaneofsymmetry} и~\eqref{zeroconstants:2orthogonalplanesofsymmetry:orthotropic}\:--- это весь набор нулевых констант и~в~этом случае.
Итак, ортотропный материал характеризуется девятью константами, и~\inquotes{для~ортотропности} достаточно двух перпендикулярных плоскостей симметрии.
Вид упругой энергии упрощается до

\nopagebreak\vspace{-0.25em}\begin{multline*}
\Pi =^{\mathstrut^{\mathstrut}} \smalldisplaystyleonehalf a_1 \varepsilon^2_{x} + \smalldisplaystyleonehalf a_2 \varepsilon^2_{y} + \smalldisplaystyleonehalf a_3 \varepsilon^2_{z} + \smalldisplaystyleonehalf a_4 \varepsilon^2_{xy} + \smalldisplaystyleonehalf a_5 \varepsilon^2_{xz} + \smalldisplaystyleonehalf a_6 \varepsilon^2_{yz} \hspace{0.25ex} + \\[-0.1em]
%
+ \hspace{0.5ex} a_{12} \varepsilon_{x} \varepsilon_{y} + a_{13} \varepsilon_{x} \varepsilon_{z} + a_{23} \varepsilon_{y} \varepsilon_{z}
\hspace{.1ex} .
\end{multline*}

В~ортотропном материале сдвиговые~(угловые) деформации $\varepsilon_{xy}$, $\varepsilon_{xz}$, $\varepsilon_{yz}$ никак не~влияют на нормальные напряжения ${\sigma_x \hspace{-0.25ex} = \raisemath{.16em}{\scalebox{0.88}{$\partial \hspace{.1ex} \Pi$}} \hspace{-0.1ex} / \hspace{-0.2ex} \raisemath{-0.32em}{\scalebox{0.88}{$\partial \varepsilon_x$}}}$, ${\sigma_y \hspace{-0.25ex} = \raisemath{.16em}{\scalebox{0.88}{$\partial \hspace{.1ex} \Pi$}} \hspace{-0.1ex} / \hspace{-0.2ex} \raisemath{-0.32em}{\scalebox{0.88}{$\partial \varepsilon_y$}}}$, ${\sigma_z \hspace{-0.25ex} = \raisemath{.16em}{\scalebox{0.88}{$\partial \hspace{.1ex} \Pi$}} \hspace{-0.1ex} / \hspace{-0.2ex} \raisemath{-0.32em}{\scalebox{0.88}{$\partial \varepsilon_z$}}}$ (и~наоборот). Популярный ортотропный материал\:--- древесина; её упругие свойства различны по~трём взаимно перпендикулярным направлениям: по~радиусу, вдоль~окружности и~по~высоте ствола.

Ещё~один случай анизотропии\:--- трансверсально изотропный (transversely isotropic) материал.
Он характеризуется

...

æolotropic (anisotropic)

...

\begin{equation}\label{legendretransformforlinearelasticenergy}
\begin{gathered}
2 \hspace{.1ex} \Pi \hspace{.1ex} = \linearstress \dotdotp \mathboldepsilon
,
\\[-0.2em]
%
\mathboldepsilon = \displaystyle \frac{\raisemath{-0.2em}{\partial\hspace{.1ex} \widehat{\Pi}}}{\raisemath{.04em}{\partial \linearstress}} = \hspace{-0.12ex} \pliabilitytensor \dotdotp \linearstress \hspace{.1ex} = \linearstress \dotdotp \hspace{-0.1ex} \pliabilitytensor
\hspace{.1ex} ,
\\
%
\widehat{\Pi}(\hspace{-0.1ex}\linearstress\hspace{.12ex}) \hspace{-0.1ex}
= \linearstress \dotdotp \mathboldepsilon
- \hspace{.1ex} \Pi(\mathboldepsilon)
\end{gathered}
\end{equation}

...


\end{otherlanguage}

\en{\section{Theorems of statics}}

\ru{\section{Теоремы статики}}

\label{para:theoremsofstatics}

\subsection*{\ru{Теорема }Clapeyron’\ru{а}\en{s theorem}}

% Benoît Paul Émile Clapeyron
% Mémoire sur le travail des forces élastiques dans un corps solide élastique déformé par l’action de forces extérieures
% Comptes rendus hebdomadaires des séances de l’Académie des sciences, Tome XLVI, Janvier–Juin 1858

\en{In~equilibrium}\ru{В~равновесии}
\en{with external forces}\ru{с~внешними силами},
\ru{объёмными}\en{volume ones}~$\bm{f}$
\ru{и~поверхностными}\en{and surface ones}~$\bm{p}$,
\en{the work of~these forces}\ru{работа этих сил}~(\inquotesx{\en{statically frozen}\ru{статически замороженных}}[---] \en{constant along time}\ru{постоянных во~времени})
\en{through actual displacements}\ru{на~актуальных перемещениях}
\ru{равна}\en{is equal to}
\en{the doubled}\ru{удвоенной}\footnote{%
%%\emph{(i)}~\bibauthor{Gabriel Lam\'{e}} et \bibauthor{Benoît Paul \'{E}mile Clapeyron}.
%%\href{https://gallica.bnf.fr/ark:/12148/bpt6k33093/f475.image}{M\'{e}moire sur l'\'{e}quilibre int\'{e}rieur des~corps solides homog\'{e}nes~//
%%Memoires present\'{e}s par Divers Savants, IV, 1833. 465\hbox{--}562~pp.}
%%\emph{(ii)}~
%
% \inquotesx{Ce produit repr\'{e}sentait d’ailleurs le~double de la~force vive que le~ressort pouvait absorber par l’effet de sa flexion et qui \'{e}tait la~mesure naturelle de~sa~puissance}[.]
%
\bibauthor{Benoît Paul \'{E}mile Clapeyron}.
\href{https://gallica.bnf.fr/ark:/12148/bpt6k3003h/f208.image}{M\'{e}moire sur le travail des forces \'{e}lastiques dans un corps solide \'{e}lastique d\'{e}form\'{e} par l’action de~forces ext\'{e}rieures~//
Comptes rendus, XLVI, Janvier--Juin 1858. 208\hbox{--}212~pp.}
}\hspace{-0.4ex}
\en{energy of~deformation}\ru{энергии деформации}

\nopagebreak\vspace{-0.1em}\begin{equation}\label{clapeyron:elasticitytheorem}
2 \hspace{-0.2em}
\integral\displaylimits_{\mathcal{V}} \hspace{-0.5ex} \Pi \hspace{0.2ex} d\mathcal{V} \hspace{.12ex} =
\integral\displaylimits_{\mathcal{V}} \hspace{-0.4ex} \bm{f} \hspace{-0.1ex} \dotp \bm{u} \hspace{.25ex} d\mathcal{V} +
\integral\displaylimits_{o_2} \hspace{-0.4ex} \bm{p} \dotp \bm{u} \hspace{.25ex} do \hspace{.2ex} .
\end{equation}

\vspace{-0.55em}\begin{multline*}
\tikz[baseline=-1ex] \draw [line width=.5pt, color=black, fill=white] (0, 0) circle (.8ex);
\hspace{2.25ex}
2 \hspace{.1ex} \Pi \hspace{.1ex} = \linearstress \dotdotp \mathboldepsilon =
\linearstress \hspace{.1ex} \dotdotp \hspace{-0.12ex} \boldnabla \bm{u}^{\mathsf{S}} \hspace{-0.12ex} =
\boldnabla \hspace{-0.1ex} \dotp \left( \hspace{.1ex} \linearstress \dotp \bm{u} \hspace{.1ex} \right) -
\tikzmark{beginMinusLoad} \boldnabla \dotp \linearstress \tikzmark{endMinusLoad} \dotp \bm{u} \;\Rightarrow
\\[.5em]
%
\Rightarrow\;\,
\displaystyle 2 \hspace{-0.16em}
\integral\displaylimits_{\mathcal{V}} \hspace{-0.5ex} \Pi \hspace{0.2ex} d\mathcal{V} \hspace{.12ex} =
\integral\displaylimits_{o_2} \hspace{-0.4ex} \tikzmark{beginSurfaceLoad} \bm{n} \dotp \linearstress \tikzmark{endSurfaceLoad} \hspace{.1ex} \dotp \bm{u} \hspace{.25ex} do \hspace{.4ex} +
\integral\displaylimits_{\mathcal{V}} \hspace{-0.4ex} \bm{f} \hspace{-0.1ex} \dotp \bm{u} \hspace{.25ex} d\mathcal{V}
\hspace{2.25ex}
\tikz[baseline=-0.6ex] \draw [color=black, fill=black] (0, 0) circle (.8ex);
\end{multline*}%
\AddUnderBrace[line width=.75pt][.2ex, 0]%
{beginMinusLoad}{endMinusLoad}{${\scriptstyle -\bm{f}}$}%
\AddUnderBrace[line width=.75pt]%
{beginSurfaceLoad}{endSurfaceLoad}{${\scriptstyle \bm{p}}$}

\begin{otherlanguage}{russian}

\vspace{-0.2em} Из~\eqref{clapeyron:elasticitytheorem} следует также, что без нагрузки
${\hspace{-0.25ex}\scalebox{1.4}{$\integral$}_{\hspace{-0.5ex}\raisemath{.1em}{\mathcal{V}}} \hspace{.25ex} \Pi \hspace{.2ex} d\mathcal{V} = 0}$.
По\-сколь\-ку $\Pi$ положительна, то и напряжение~$\linearstress$, и~деформация~$\mathboldepsilon$ без нагрузки\:--- нулевые.

\textbold{\en{Paradox of elastostatics}\ru{Парадокс эластостатики}}: ${\Pi}$ \en{is equal to}\ru{равна} \en{only}\ru{лишь} \en{the~half}\ru{половине} \en{of~the~work}\ru{работы} \en{of~external forces}\ru{внешних сил}.

\en{The~accumulated potential energy of~deformation}\ru{Накопленная потенциальная энергия деформации}~$\Pi$ is equal to only the~half of~the~work done by external forces, acting through displacements from the~unstressed configuration to the~equilibrium.

\ru{Теорема }Clapeyron’\en{s}\ru{а}\en{ theorem} \en{implies that}\ru{подразумевает, что} \en{the~accumulated elastic energy}\ru{накопленная упругая энергия} \en{accounts for}\ru{составляет} \en{only the~half of~energy spent on deformation}\ru{лишь половину потраченной на деформацию энергии}.
\en{The~remaining half of the~work}\ru{Оставшаяся половина работы,} \en{done by external forces}\ru{совершённой внешними силами,} \en{is lost somewhere}\ru{теряется где\hbox{-}то} \en{before reaching the~equilibrium}\ru{до достижения равновесия}.

{\small
\en{This apparent paradox is reached within the~framework of purely conservative linear elasticity.
Alternatively, however, within elastostatics the~common characterization of the~work done to reach equilibrium is conceptually ambiguous, and a~novel interpretation may be needed.}

\ru{Этот кажущийся парадокс достигается в~рамках чисто консервативной линейной упругости.
Как альтернатива, однако, в~эластостатике обычная характеризация работы, совершённой для достижения равновесия, концептуально сомнительна, и может быть нужна новая интерпретация.}

\bibauthor{Roger Fosdick} and \bibauthor{Lev Truskinovsky}.
\href{http://www.cityu.edu.hk/ma/ws2007/notes/FTJElast.pdf}{About Clapeyron’s Theorem in Linear Elasticity~// Journal of~Elasticity, Volume 72, July 2003. Pages 145\hbox{--}172.}
\par}

There is always heating due to energy dissipation.

Для решения парадокса в~теории распространена концепция \en{of~}\inquotesx{\en{static loading}\ru{статического нагружения}}[---] \en{infinitely slow}\ru{бесконечно медленного} \en{gradual application}\ru{постепенного приложения} \en{of~the~load}\ru{нагрузки}.

{\small
Статика рассматривает \inquotes{замороженное} равновесие, оно вне времени.
Динамика нагружения до~равновесия\:--- предыстория.
В~линейной теории в~равновесии работа внешних сил на актуальных перемещениях, затраченная на~деформацию, равна удвоенной потенциальной энергии деформации.
\inquotes{Запасается} всего половина потраченной энергии.
Вторая половина есть дополнительная энергия, она теряется до~обретения равновесия на~динамику\:--- на~внутреннюю энергию частиц (\inquotes{диссипацию}), на~колебания и~волны.
Так в~теории.
Однако, в~реальности не~бывает ни~моментального \inquotes{мёртвого} нагружения, ни~бесконечно медленного \inquotesx{следящего}[.]
Это две крайности.
Реальная динамика нагружения всегда где\hbox{-}то между ними.
Посему теоретическая дополнительная энергия всегда больше реальной сопровождающей процесс нагружения \inquotes{диссипативной}, и~в~линейной теории она численно равна упругому потенциалу\:--- половине работы внешних сил, той самой второй половине.

В~области~же бесконечномалых вариаций и~виртуальных работ, работа реальных внешних сил на виртуальных перемещениях точно равна вариации упругого потенциала.
А~упругая среда есть такая, в~которой вариация работы сил внутренних (напряжений) на виртуальных деформациях это минус вариация потенциала.

${- \hspace{.1ex} \variation{\internalwork} \hspace{-0.2ex} = \variation{\Pi} = \variation{\externalwork} \hspace{-0.25ex}}$, когда варьируются только перемещения (нагрузки не~варьируются).
Потому в~принципе виртуальной работы и~варьируются лишь перемещения, чтобы виртуальная работа внешних неварьируемых реальных сил на~вариациях перемещений была равна минус вариации внутренней энергии (в~случае упругой среды\:--- вариации упругого потенциала).
\par}

\subsection*{\en{Theorem for the uniqueness of solution}\ru{Теорема об~уникальности решения}}

Как~и в~динамике~(\pararef{para:uniquenessfordynamicproblem}), допускаем существование двух решений и~ищем их~разность

...

\end{otherlanguage}

% ~ ~ ~ ~ ~
%%\begin{center}

\begin{wrapfigure}{o}{.4\textwidth}
\makebox[.45\textwidth][c]{\begin{minipage}[t]{.5\textwidth}
\vspace{-1em}
\scalebox{0.88}{

\def\cameraangle{166}
\tdplotsetmaincoords{44}{\cameraangle} % orientation of camera

\def\rodheight{10}
\def\rodradius{.2}

\pgfmathsetmacro{\beginangle}{\cameraangle}
\pgfmathsetmacro{\endangle}{\cameraangle - 180}

\tikzset{pics/rod/.style={code={

	% draw rod

	\draw [line width=1pt, color=black, fill=yellow!50!white, opacity=.9]
		plot [domain=\beginangle:\endangle]
			( {\rodradius*cos(\x)}, {\rodradius*sin(\x)}, 0 )
		-- plot [domain=\endangle:\beginangle]
			( {\rodradius*cos(\x)}, {\rodradius*sin(\x)}, \rodheight )
		-- cycle ;

	\draw [line width=1pt, color=black, fill=yellow!50!white, opacity=.9, domain=0:360]
		plot ( {-\rodradius*cos(\x)}, {-\rodradius*sin(\x)}, \rodheight ) ;

}}}

\tikzset{pics/dottedrod/.style={code={

	% draw rod dotted

	\draw [line width=1pt, color=black, line cap=round, dash pattern=on 0pt off 1.6\pgflinewidth]
		plot [domain=\beginangle:\endangle]
			( {\rodradius*cos(\x)}, {\rodradius*sin(\x)}, 0 )
		-- plot [domain=\endangle:\beginangle]
			( {\rodradius*cos(\x)}, {\rodradius*sin(\x)}, \rodheight )
		-- cycle ;

	\draw [line width=1pt, color=black, line cap=round, dash pattern=on 0pt off 1.6\pgflinewidth, opacity=.9, domain=0:360]
		plot ( {-\rodradius*cos(\x)}, {-\rodradius*sin(\x)}, \rodheight ) ;

}}}

\tikzset{pics/rodaxis/.style={code={

	% draw axis
	\draw [line width=0.5pt, blue, line cap=round, dash pattern=on 12pt off 2pt on \the\pgflinewidth off 2pt]
		( 0, 0, -0.4pt ) -- ( 0, 0, \rodheight + 0.4pt ) ;

}}}

\tikzset{pics/externalforce/.style={code={

	% draw force
	\def\forcelength{1.2}

	\draw [line width=1.5pt, red, line cap=round, -{Triangle[round, length=3.6mm, width=2.4mm]}]
		( 0, 0, \rodheight + \forcelength ) -- ( 0, 0, \rodheight )
		node [ pos=0.4, left, inner sep=0, outer sep=4.4pt ]
			{\scalebox{1.2}[1.2]{${\bm{F}}$}} ;

}}}

\begin{tikzpicture}[scale=1, tdplot_main_coords] % use 3dplot

	\coordinate (O) at ( 0, 0, 0 ) ;
	\coordinate (rodTopCenter) at ($ (O) + ( 0, 0, \rodheight ) $) ;

	% draw circle
	\def\circleradius{0.8}
	\def\heightofhatch{0.5}

	\pgfmathsetmacro{\stepangleforcircle}{\beginangle - 10}
	\foreach \angle in { \beginangle, \stepangleforcircle, ..., \endangle }
		\draw [line width=0.4pt, color=black]
			( \angle:\circleradius ) -- ($ ( \angle:\circleradius ) - ( 0, 0, \heightofhatch ) $) ;

	\draw [line width=1pt, color=black, fill=white] (O) circle ( \circleradius ) ;

	% draw rod, axis and force
	\pic (initial) {rod} ;
	\pic (initial) {rodaxis} ;
	\pic (initial) {externalforce} ;

	% draw deformed rod
	\scoped {
		\pgfsetcurvilinearbeziercurve
			{\pgfpointxyz{0}{0}{0}}
			{\pgfpointxyz{0}{0}{0.5cm}}
			{\pgfpointxyz{0.25cm}{0}{1cm}}
			{\pgfpointxyz{1.25cm}{0}{1.25cm}}
		\pgftransformnonlinear{\pgfgetlastxy\x\y\pgfpointcurvilinearbezierorthogonal{\y}{\x}}
			\pic (deformed) {dottedrod} ;
			\pic (deformed) {rodaxis} ;
	}

\end{tikzpicture}
}
\vspace{-0.2em}\caption{}\label{fig:deformedrod}
\end{minipage}}
\end{wrapfigure}

%%\end{center}

% ~ ~ ~ ~ ~

\begin{otherlanguage}{russian}

\en{The uniqueness of solution}\ru{Единственность решения}, \en{determined by}\ru{определённая} G.\:Kirchhoff\ru{’ом}\footnote{\bibauthor{Gustav Robert Kirchhoff}. \href{https://opacplus.bsb-muenchen.de/Vta2/bsb10525510/bsb:2960444?page=291}{Über das~Gleich\-gewicht und die~Bewe\-gung eines unendlich dünnen elastischen Stabes~//~Journal für die reine und angewandte Mathematik (Crelle’s journal), 56. Band (1859). Seiten 285\hbox{--}313.}}\hbox{\hspace{-0.5ex},} \en{is contrary to, as it seems, the everyday experience}\ru{противоречит, казалось~бы, повседневному опыту}.
\en{Imagine}\ru{Вообразим} \en{a~straight rod}\ru{прямой стержень}, закреплённый на~одном конце и~сжатый продольной силой на~другом~(\figref{fig:deformedrod}). Когда нагрузка достаточно больш\'{а}я, задача статики имеет два решения\:--- \inquotes{прямое} и \inquotesx{изогнутое}[.]
Но такое противоречие объясняется нелинейностью задачи. При~м\'{а}лой~же нагрузке решение единственно и~может быть описано линейными уравнениями.

\subsection*{\en{Reciprocal work theorem}\ru{Теорема о~взаимности работ}}

Для т\'{е}ла с~закреплением на~части поверхности~${o_1}$ рассматриваются два варианта: первый с~нагрузками $\bm{f}_1$, $\bm{p}_1$ и~второй с~нагрузками $\bm{f}_2$, $\bm{p}_2$.
Словесная формулировка теоремы та~же, что и~в~\chapdotpararef{chapter:genericmechanics}{para:statics}.
Математическая запись

\nopagebreak\vspace{1.5em}\begin{equation}\label{betti:reciprocalworktheorem}
\tikzmark{beginReciprocalFirstVariant} \scalebox{0.98}{$\displaystyle \integral\displaylimits_{\mathcal{V}} \hspace{-0.4ex} \bm{f}_1 \hspace{-0.2ex} \dotp \bm{u}_2 \hspace{.25ex} d\mathcal{V} $}
+ \hspace{-0.2ex}
\scalebox{0.98}{$\displaystyle \integral\displaylimits_{o_2} \hspace{-0.4ex} \bm{p}_1 \hspace{-0.2ex} \dotp \bm{u}_2 \hspace{.25ex} do $} \tikzmark{endReciprocalFirstVariant}
%
\hspace{.2ex} = \hspace{-0.1ex}
%
\tikzmark{beginReciprocalSecondVariant} \scalebox{0.98}{$\displaystyle \integral\displaylimits_{\mathcal{V}} \hspace{-0.4ex} \bm{f}_2 \hspace{-0.2ex} \dotp \bm{u}_1 \hspace{.25ex} d\mathcal{V} $}
+ \hspace{-0.2ex}
 \scalebox{0.98}{$\displaystyle \integral\displaylimits_{o_2} \hspace{-0.4ex} \bm{p}_2 \hspace{-0.2ex} \dotp \bm{u}_1 \hspace{.25ex} do $} \tikzmark{endReciprocalSecondVariant}
\hspace{.1ex} .
\end{equation}%
\AddOverBrace[line width=.75pt][.1ex, .8em][yshift=-0.11ex]%
{beginReciprocalFirstVariant}{endReciprocalFirstVariant}{\scalebox{0.88}{$ W_{\hspace{-0.1ex}12} $}}%
\AddOverBrace[line width=.75pt][.1ex, .8em][yshift=-0.11ex]%
{beginReciprocalSecondVariant}{endReciprocalSecondVariant}{\scalebox{0.88}{$ W_{\hspace{-0.15ex}21} $}}

...

{\small
Reciprocal work theorem, also known as Betti’s theorem, discovered by Enrico Betti in 1872, claims that for a~linear elastic structure subject to two sets of forces $P$ and $Q$, the~work done by set~$P$ through displacements produced by set~$Q$ is equal to the~work done by set~$Q$ through displacements produced by set~$P$. This theorem has applications in structural engineering where it is used to define influence lines and derive the boundary element method.
\par}

...

\begin{wrapfigure}[10]{o}{.45\textwidth}
\makebox[.5\textwidth][c]{\begin{minipage}[t]{.4\textwidth}
\vspace{-1.2em}

% parameter #1 is height of beam
\newcommand\drawrodbeam[1]%
{
	\draw [line width=1.2pt, black] (-0.5, 0) -- (0.5, 0) ;
	\foreach \xground in {-0.36, -0.16, ..., 0.5}
		\draw [line width=0.4pt, black!80] (\xground, 0) -- (\xground - 0.2, -0.2) ;

	\draw [line width=2pt, line cap=round, black] (0, 0) -- (0, #1) ;
}

\tikzstyle{force line} =
	[line width=1.5pt, red, line cap=round, -{Triangle[round, length=3.6mm, width=2.4mm]}]

\tikzstyle{unitforce line} =
	[line width=1.5pt, blue, line cap=round, dash pattern=on 0pt off 1.6\pgflinewidth, -{Triangle[round, length=3.6mm, width=2.4mm]}]

\begin{tikzpicture}[scale=0.8]

	\def\beamheight{5}
	\def\beamxshift{2}
	\def\forcelength{1}

	\pgfmathsetmacro{\firstforceposition}{\beamheight}
	\pgfmathsetmacro{\secondforceposition}{0.6 * \beamheight}

	\drawrodbeam{\beamheight}

	\draw [force line]
		( -\forcelength, \firstforceposition ) -- ( 0, \firstforceposition )
		node [pos=0.25, below, inner sep=0, outer sep=5pt]
			{${P_{\hspace{-0.25ex}\raisemath{-0.4ex}{1}}}$} ;

	\draw [force line]
		( -\forcelength, \secondforceposition ) -- ( 0, \secondforceposition )
		node [pos=0.25, below, inner sep=0, outer sep=5pt]
			{${P_{\hspace{-0.25ex}\raisemath{-0.4ex}{2}}}$} ;

	\pgfmathsetmacro\secondxshift{\beamxshift}

	\begin{scope}[xshift=\secondxshift cm, yshift=0cm]
		\drawrodbeam{\beamheight}
	\end{scope}

	\draw [unitforce line]
		( \secondxshift - \forcelength, \firstforceposition ) -- ( \secondxshift, \firstforceposition )
		node [pos=0.25, below, inner sep=0, outer sep=5pt] {$1$} ;

	\pgfmathsetmacro\thirdxshift{\beamxshift + \beamxshift}

	\begin{scope}[xshift=\thirdxshift cm, yshift=0cm]
		\drawrodbeam{\beamheight}
	\end{scope}

	\draw [unitforce line]
		( \thirdxshift - \forcelength, \secondforceposition ) -- ( \thirdxshift, \secondforceposition )
		node [pos=0.25, below, inner sep=0, outer sep=5pt] {$1$} ;

\end{tikzpicture}
\vspace{-0.1em}\caption{}\label{fig:exampleofapplicationofreciprocalworktheorem}
\end{minipage}}
\end{wrapfigure}

Теорема о~взаимности работ Enrico Betti находит неожиданные и~эффективные применения. Как иллюстрацию рассмотрим защёмленный одним концом~(\inquotes{консольный}) стержень-балку, изгибаемый силами величиной~$P_{\hspace{-0.25ex}\raisemath{-0.4ex}{1}}$ и~$P_{\hspace{-0.25ex}\raisemath{-0.4ex}{2}}$~(\figref{fig:exampleofapplicationofreciprocalworktheorem}). Используя линейность задачи, перемещения возможно найти как

\nopagebreak\vspace{-0.1em}\begin{equation*}\begin{array}{c}
u_1 = \ldots \hspace{.1ex} ,
\\
u_2 = \ldots \hspace{.1ex} .
\end{array}\end{equation*}

...



\end{otherlanguage}

\en{\section{Equations for displacements}}

\ru{\section{Уравнения в перемещениях}}

\label{para:equationsfordisplacement}

\begin{otherlanguage}{russian}

\en{The~complete set of equations}\ru{Полный набор уравнений}~\eqref{lineartheory:wholesetofequations} содержит неизвестные $\linearstress$, $\mathboldepsilon$ и~$\bm{u}$. Исключая $\linearstress$ и~$\mathboldepsilon$, приходим к~постановке в~перемещениях (симметризация~${\hspace{-0.2ex} \boldnabla \bm{u}}$ тут излишняя, ведь ${\stiffnesstensor_{\hspace{.12ex} 3 \scalebox{0.6}[0.8]{$\rightleftarrows$} 4} \hspace{-0.2ex} = \stiffnesstensor}$)

\nopagebreak\vspace{-0.1em}\begin{equation}\label{lineartheory:equationsfordisplacement}
\begin{array}{c}
\boldnabla \dotp \left( \stiffnesstensor \dotdotp \hspace{-0.12ex} \boldnabla \bm{u} \hspace{.1ex} \right) + \hspace{.1ex} \bm{f} = \hspace{.1ex} \bm{0} \hspace{.1ex}; \\[.4em]
\bm{u} \hspace{.1ex} \bigr|_{o_1} \hspace{-0.64ex} = \hspace{0.2ex} \bm{u}_{\raisemath{-0.1em}{0}} \hspace{.16ex} , \:\:
\bm{n} \dotp %%\tikzmark{TauTensorBegin}
\stiffnesstensor \hspace{-0.08ex} \dotdotp \hspace{-0.24ex} \boldnabla \bm{u}
%%\tikzmark{TauTensorEnd}
\hspace{0.25ex} \bigr|_{o_2} \hspace{-0.64ex} = \hspace{0.2ex} \bm{p} \hspace{.16ex} .
\end{array}
\end{equation}%
%%\AddOverBrace[line width=.75pt][-0.1ex,0.1em]%
%%{TauTensorBegin}{TauTensorEnd}{${\scriptstyle \linearstress}$}

%% \footnote{По\hbox{-}прежнему под~$\bm{f}$ подразумеваем сумму обычной силы и~даламберовой силы инерции~${(- \rho \mathdotdotabove{\bm{u}}\hspace{0.25ex})}$.}

В~изотропном теле~\eqref{lineartheory:equationsfordisplacement} принимает вид

...

Общее решение однородного уравнения (...) нашёл \href{https://de.wikipedia.org/wiki/Heinz_Neuber}{Heinz Neuber}

\href{https://ru.wikipedia.org/wiki/%D0%9F%D0%B0%D0%BF%D0%BA%D0%BE%D0%B2%D0%B8%D1%87,_%D0%9F%D1%91%D1%82%D1%80_%D0%A4%D1%91%D0%B4%D0%BE%D1%80%D0%BE%D0%B2%D0%B8%D1%87}{П.\,Ф.\:Папкович}

...



\end{otherlanguage}

\en{\section{Concentrated force in a limitless medium}}

\ru{\section{Сосредоточенная сила в неограниченной среде}}

{\small
Concentrated force is useful mathematical idealization, but cannot be found in the real world, where all forces are either body forces acting over a~volume or surface forces acting over an~area.
\par}

\begin{otherlanguage}{russian}

Начнём с~риторического вопроса: почему упругое тело сопротивляется приложенной нагрузке, выдерживает её? Удачный ответ можно найти ...

...



\end{otherlanguage}

\en{\section{Finding displacements by deformations}}

\ru{\section{Нахождение перемещений по деформациям}}

\label{para:displacementsfromdeformations}

\begin{otherlanguage}{russian}

Разложив градиент перемещения на~симметричную и~антисимметричную части
\nopagebreak\vspace{.1em}\begin{equation}
\boldnabla \bm{u} \hspace{.2ex} = \tikzmark{beginSymmNablaU} \hspace{0.32ex} \mathboldepsilon \hspace{0.4ex} \tikzmark{endSymmNablaU} \hspace{-0.16ex} - \hspace{.25ex} \tikzmark{beginAsymmNablaU} \bm{\omega} \times \hspace{-0.12ex} \bm{E} \tikzmark{endAsymmNablaU} \hspace{.1ex} , \:\;
\bm{\omega} \equiv \displaystyle \onehalf \hspace{.32ex} \boldnabla \hspace{-0.12ex} \times \hspace{-0.12ex} \bm{u} \hspace{.2ex},
\end{equation}%
\AddOverBrace[line width=.75pt][.1ex,0.1ex]%
{beginSymmNablaU}{endSymmNablaU}{${\scriptstyle \boldnabla {\bm{u}}^{\hspace{.1ex}\mathsf{S}}}$}%
\AddOverBrace[line width=.75pt][.2ex,0.1ex]%
{beginAsymmNablaU}{endAsymmNablaU}{${\scriptstyle - \hspace{.1ex} \boldnabla {\bm{u}}^{\hspace{.1ex}\mathsf{A}}}$}

...

{\small
Saint\hbox{-\hspace{-0.2ex}}Venant’s compatibility condition is the integrability conditions for a~symmetric tensor field to be a~strain.

The compatibility conditions in linear elasticity are obtained by observing that there are six strain\hbox{--}displacement relations that are functions of only three unknown displacements. This suggests that the three displacements may be removed from the system of equations without loss of information. The resulting expressions in terms of only the strains provide constraints on the possible forms of a~strain field.

A body that deforms without developing any gaps/overlaps is called a~compatible body. Compatibility conditions are mathematical conditions that determine whether a~particular deformation will leave a~body in a~compatible state.
\par}

...

\begin{equation*}
\operatorname{inc} \mathboldepsilon \hspace{-0.05ex}
\equiv
\hspace{-0.07ex} \boldnabla \hspace{-0.2ex} \times \hspace{-0.36ex} \bigl( \hspace{.06ex} \boldnabla \hspace{-0.2ex} \times \hspace{-0.16ex} \mathboldepsilon \hspace{.1ex} \bigr)^{\hspace{-0.2ex}\T}
\end{equation*}

Контур здесь произволен, так что приходим к~соотношению

\nopagebreak\vspace{-0.25em}\begin{equation}\label{incompatibilityequalszero}
\operatorname{inc} \mathboldepsilon = \hspace{-0.07ex} {^2\bm{0}}
\hspace{.1ex} ,
\end{equation}

\vspace{-0.33em} \noindent называемому уравнением совместности деформаций.

...

Тензор~${\operatorname{inc} \mathboldepsilon}$ симметричен вместе с~${\mathboldepsilon}$

...

Все уравнения линейной теории имеют аналог~(перво\-источ\-ник) в~нелинейной. Чтобы найти его для~\eqref{incompatibilityequalszero}, вспомним тензор деформации Cauchy\hbox{--}Green’а~(\chapdotpararef{chapter:nonlinearcontinuum}{para:deformationtensors}) и~тензоры кривизны~(\chapdotpararef{chapter:elementsoftensorcalculus}{para:curvaturetensors})

...



\end{otherlanguage}

\en{\section{Equations for stresses}}

\ru{\section{Уравнения в напряжениях}}

\en{Balance of~forces~(of~momentum)}\ru{Баланс сил~(импульса)}

\nopagebreak\en{\vspace{-0.125em}}\ru{\vspace{-0.6em}}
\begin{equation}\label{stresseseq:balanceofforces}
\boldnabla \dotp \linearstress \hspace{.15ex} + \bm{f} = \hspace{.1ex} \bm{0}
\end{equation}

\nopagebreak \vspace{-0.2em} \noindent
\en{does not quite yet determine}\ru{ещё не~определяет} \en{stresses}\ru{напряжения}.
\en{I\kern-0.12ext’s necessary as~well}\ru{Необходимо вдобавок}, \en{that deformations/strains}\ru{чтобы соответствующие напряжениям деформации}~${\mathboldepsilon(\hspace{-0.15ex}\linearstress\hspace{.1ex})}$\en{ corresponding to stresses}~\eqref{legendretransformforlinearelasticenergy}

\nopagebreak\vspace{-0.2em}\begin{equation}\label{stresseseq:strainfromstress}
\mathboldepsilon(\hspace{-0.15ex}\linearstress\hspace{.1ex}) = \displaystyle \frac{\raisemath{-0.2em}{\partial\hspace{.1ex} \widehat{\Pi}}}{\raisemath{.04em}{\partial \linearstress}} = \hspace{-0.12ex} \pliabilitytensor \dotdotp \hspace{-0.07ex} \linearstress
\end{equation}

\nopagebreak \vspace{-0.5em} \noindent
\en{are compatible}\ru{были совместны}~(\pararef{para:displacementsfromdeformations})

\nopagebreak\vspace{-0.5em}\begin{equation}\label{stresseseq:compatibility}
\operatorname{inc} \mathboldepsilon(\hspace{-0.15ex}\linearstress\hspace{.1ex}) \hspace{-0.06ex}
\equiv
\hspace{-0.07ex} \boldnabla \hspace{-0.2ex} \times \hspace{-0.44ex} \Bigl( \boldnabla \hspace{-0.2ex} \times \hspace{-0.16ex} \mathboldepsilon(\hspace{-0.15ex}\linearstress\hspace{.1ex}) \Bigr)^{\raisemath{-0.12em}{\hspace{-0.44ex}\T}} \hspace{-0.52ex}
= {^2\bm{0}}
\hspace{.1ex} .
\end{equation}

\vspace{-0.27em} \noindent
\en{Gathered together}\ru{Взятые вместе}, \eqref{stresseseq:balanceofforces}, \eqref{stresseseq:strainfromstress} and~\eqref{stresseseq:compatibility} \en{present}\ru{являют} \en{the~complete closed set}\ru{полный замкнутый набор}~(\en{system}\ru{систему}) \en{of~equations}\ru{уравнений} \en{for stresses}\ru{в~напряжениях}.


...

%%\begin{otherlanguage}{russian}

...

%%\end{otherlanguage}

\en{\section{Principle of minimum potential energy}}

\ru{\section{Принцип минимума потенциальной энергии}}

\label{para:principleofminimumpotentialenergy}

\begin{otherlanguage}{russian}

Начнём с~формулировки принципа:

\nopagebreak\vspace{-0.1em}\begin{equation}\label{principleofminimumpotentialenergy.formulation}
\textit{Э} \hspace{.2ex} (\hspace{-0.1ex}\bm{u}\hspace{-0.1ex}) \equiv \hspace{-0.2ex}
\displaystyle \integral\displaylimits_{\mathcal{V}} \hspace{-0.7ex}
\left(^{\mathstrut} \hspace{-0.1ex}
\Pi(\boldnabla \bm{u}) - \bm{f} \hspace{-0.1ex} \dotp \bm{u} \right) \hspace{-0.4ex} d\mathcal{V} \hspace{.15ex}
- \hspace{-0.2ex}
\integral\displaylimits_{o_2} \hspace{-0.32ex} \bm{p} \dotp \bm{u} \hspace{.33ex} do \hspace{.2ex}
\hspace{.1ex}\to\hspace{.25ex} \mathrm{min} \hspace{.15ex} , \:\:
\bm{u} \hspace{.1ex} \bigr|_{o_1} \hspace{-0.8ex} = \hspace{.1ex} \bm{u}_{\raisemath{-0.1em}{0}}
\hspace{.16ex} .
\end{equation}

\vspace{-0.2em} \noindent Этот функционал, называемый потенциальной энергией \en{of a~linear-elastic body}\ru{линейно-упругого т\'{е}ла}, принимает наименьшее значение на~истинных перемещениях\:--- то~есть на~решении задачи~\eqref{lineartheory:equationsfordisplacement}.
При~этом функции~$\bm{u}$ должны удовлетворять геометрическому условию на~${o_1}$~(чтобы не~нарушать связи) и~быть непрерывными (иначе ${\Pi(\boldnabla \bm{u})}$ %% ${\Pi\scalebox{0.96}{$\left(\boldnabla {\bm{u}}^{\mathsf{S}}\hspace{.1ex}\right)$}}$
не~будет интегрируемой).

Для обоснования принципа возьмём какое\hbox{-}либо ещё приемлемое поле перемещений~${\bm{u}'}$ и~найдём разность
${\textit{Э} \hspace{.16ex} (\hspace{-0.1ex}\bm{u}'\hspace{.1ex}) -
\textit{Э} \hspace{.16ex} (\hspace{-0.1ex}\bm{u}\hspace{-0.1ex})
=}$

\nopagebreak\vspace{-0.16em}\begin{equation*}
= \displaystyle
\integral\displaylimits_{\mathcal{V}} \hspace{-0.6ex}
\left(^{\mathstrut} \hspace{-0.1ex}
\Pi(\boldnabla \bm{u}'\hspace{.1ex}) - \Pi(\boldnabla \bm{u}) - \bm{f} \hspace{-0.1ex} \dotp ( \bm{u}' \hspace{-0.2ex} - \bm{u} ) \right) \hspace{-0.32ex} d\mathcal{V} \hspace{.1ex}
- \hspace{-0.2ex}
\integral\displaylimits_{o_2} \hspace{-0.32ex} \bm{p} \dotp ( \bm{u}' \hspace{-0.2ex} - \bm{u} ) \hspace{.25ex} do \hspace{.1ex} .
\vspace{-0.25em}\end{equation*}

\vspace{-0.1em} \noindent Поскольку ${\bm{n} \dotp \linearstress \hspace{.16ex} \bigr|_{o_2} \hspace{-0.64ex} = \hspace{.2ex} \bm{p}}$ и~${( \bm{u}' \hspace{-0.2ex} - \bm{u} ) \hspace{.1ex} \bigr|_{o_1} \hspace{-0.64ex} = \bm{0}}$, то

\nopagebreak\vspace{-0.5em}\begin{multline*}
\displaystyle
\integral\displaylimits_{o_2} \hspace{-0.32ex} \bm{p} \dotp ( \bm{u}' \hspace{-0.2ex} - \bm{u} ) \hspace{.25ex} do \hspace{.1ex}
=
\ointegral\displaylimits_{\mathclap{o\hspace{.15ex}(\boundary \mathcal{V})}} \hspace{-0.2ex} \bm{n} \dotp \linearstress \hspace{.12ex} \dotp ( \bm{u}' \hspace{-0.2ex} - \bm{u} ) \hspace{.25ex} do \hspace{.1ex}
= \hspace{-0.1ex}
\integral\displaylimits_{\mathcal{V}} \hspace{-0.64ex} \boldnabla \dotp \left(^{\mathstrut} \hspace{-0.2ex} \linearstress \hspace{.12ex} \dotp ( \bm{u}' \hspace{-0.2ex} - \bm{u} ) \right) \hspace{-0.32ex} d\mathcal{V}
= \\[-0.1em]
%
= \hspace{-0.1ex}
\integral\displaylimits_{\mathcal{V}} \hspace{-0.2ex} ( \hspace{.1ex} \boldnabla \dotp \linearstress \hspace{.12ex} ) \dotp ( \bm{u}' \hspace{-0.2ex} - \bm{u} ) \hspace{.25ex} d\mathcal{V} \hspace{.1ex}
+ \hspace{-0.1ex}
\integral\displaylimits_{\mathcal{V}} \hspace{-0.64ex} \linearstress^{\hspace{.12ex}\T} \hspace{-0.4ex} \dotdotp \hspace{-0.1ex} \boldnabla \hspace{.1ex} ( \bm{u}' \hspace{-0.2ex} - \bm{u} ) \hspace{.25ex} d\mathcal{V}
\end{multline*}

\vspace{-0.5em} \noindent и
${\textit{Э} \hspace{.16ex} (\hspace{-0.1ex}\bm{u}'\hspace{.1ex}) -
\textit{Э} \hspace{.16ex} (\hspace{-0.1ex}\bm{u}\hspace{-0.1ex})
=}$

\nopagebreak\vspace{-0.4em}\begin{multline*}
= \hspace{-0.2ex}
\integral\displaylimits_{\mathcal{V}} \hspace{-0.64ex}
\left(^{\mathstrut} \hspace{-0.1ex}
\Pi(\boldnabla \bm{u}'\hspace{.1ex}) - \Pi(\boldnabla \bm{u}) -
( \hspace{.12ex} \hspace{.4ex} \tikzmark{beginBalanceIsZero} \hspace{-0.4ex} \boldnabla \hspace{-0.1ex} \dotp \hspace{-0.1ex} \linearstress \hspace{.15ex} + \bm{f} \hspace{-0.25ex} \tikzmark{endBalanceIsZero} \hspace{.25ex} \hspace{.12ex} ) \hspace{-0.1ex} \dotp \hspace{-0.1ex} ( \bm{u}' \hspace{-0.2ex} - \bm{u} ) \right) \hspace{-0.32ex} d\mathcal{V}
\hspace{.6ex} -
\\[-1.2em]
- \integral\displaylimits_{\mathcal{V}} \hspace{-0.64ex} \linearstress \dotdotp ( \hspace{.1ex} \boldnabla \bm{u}' \hspace{-0.2ex} - \hspace{-0.32ex} \boldnabla \bm{u} ) \hspace{.25ex} d\mathcal{V} .
\end{multline*}
%
\AddUnderBrace[line width=.75pt][0,-0.1em]%
{beginBalanceIsZero}{endBalanceIsZero}%
{${\scriptstyle \bm{0}}$}

\vspace{-0.36em} \noindent Тут
${\Pi(\boldnabla \bm{u}) \hspace{-0.24ex} = \smalldisplaystyleonehalf \hspace{.2ex} \linearstress \dotdotp \hspace{-0.2ex} \boldnabla \bm{u}^{\hspace{.08ex}\mathsf{S}}\hspace{-0.32ex}}$,
${\Pi(\boldnabla \bm{u}'\hspace{.1ex}) \hspace{-0.24ex} = \smalldisplaystyleonehalf \hspace{.2ex} \linearstress \dotdotp \hspace{-0.2ex} \boldnabla \bm{u}'^{\hspace{.25ex}\mathsf{S}}\hspace{-0.32ex}}$,
а~благодаря симметрии %%тензора напряжения
${\linearstress^{\hspace{.12ex}\T} \hspace{-0.32ex} = \linearstress}$ ${\hspace{.4ex}\Rightarrow\hspace{.2ex}}$ ${\linearstress \dotdotp \hspace{-0.2ex} \boldnabla \bm{a} = \linearstress \dotdotp \hspace{-0.2ex} \boldnabla \bm{a}^{\hspace{.08ex}\mathsf{S}} \hspace{.8ex}\forall \bm{a}}$. Поэтому

\nopagebreak ...




\end{otherlanguage}

\en{\section{Principle of minimum complementary energy}}

\ru{\section{Принцип минимума дополнительной энергии}}

\en{The complementary energy}\ru{Дополнительная энергия} \en{of a~linear-elastic body}\ru{линейно-упругого т\'{е}ла} \en{is}\ru{есть} \en{the~following}\ru{следующий} \en{functional}\ru{функционал} \en{over the~field of~stresses}\ru{над~полем напряжений}:

\nopagebreak\begin{equation}\label{complementaryenergyfunctional}
\mathscr{D} (\hspace{-0.1ex}\linearstress\hspace{.1ex}) \equiv \hspace{-0.2ex}
\displaystyle \integral\displaylimits_{\mathcal{V}} \hspace{-0.5ex}
\widehat{\Pi}(\hspace{-0.1ex}\linearstress\hspace{.1ex}) \hspace{.25ex} d\mathcal{V} \hspace{.1ex}
- \hspace{-0.2ex}
\integral\displaylimits_{o_1} \hspace{-0.5ex} \bm{n} \hspace{.1ex} \dotp \linearstress \dotp \bm{u}_{\raisemath{-0.1em}{0}} \hspace{.25ex} do
\hspace{.2ex} ,
\hspace{.5em}
\bm{u}_{\raisemath{-0.1em}{0}} \hspace{-0.2ex} \equiv \bm{u} \hspace{.1ex} \bigr|_{o_1}
\hspace{-0.1ex} ,
\end{equation}
%
\nopagebreak\vspace{-0.4em}\begin{equation*}
\boldnabla \dotp \linearstress \hspace{.15ex} + \bm{f} = \hspace{.1ex} \bm{0} \hspace{.1ex} ,
\hspace{.6em}
\bm{n} \dotp \linearstress \hspace{.2ex} \bigr|_{o_2} \hspace{-0.66ex} = \hspace{.2ex} \bm{p}
\hspace{.2ex} .
\end{equation*}

...

${
\variation{\hspace{.1ex} \bigl( \hspace{.1ex} \boldnabla \dotp \linearstress \hspace{.15ex} + \bm{f} \hspace{.15ex} \bigr)} \hspace{-0.2ex}
= \hspace{-0.1ex} \boldnabla \dotp \variation{\linearstress} \hspace{.15ex}
= \hspace{.1ex} \bm{0}
}$ \en{inside volume}\ru{в~объёме}~$\mathcal{V}$

...

\en{The~principle of minimum complementary energy}\ru{Принцип минимума дополнительной энергии} \en{is very useful}\ru{очень полезен}, \en{as example}\ru{например}, \en{for estimation}\ru{для оценки} \en{of inexact}\ru{неточных}~(\en{approximate}\ru{приближённых}) \en{solutions}\ru{решений}.
\en{But}\ru{Но} \en{for computations}\ru{для вычислений} \en{it isn’t so essential}\ru{он не~столь существенен,} \en{as}\ru{как} \en{the}\ru{принцип}~(Lagrange\ru{’а})\en{ principle} \en{of~minimum potential energy}\ru{минимума потенциальной энергии} \eqref{principleofminimumpotentialenergy.formulation}.

\en{\section{Mixed principles of stationarity}}

\ru{\section{Смешанные принципы стационарности}}

\label{para:mixedvariationalprinciples}

\vspace{.2em}\begin{changemargin}{2\parindent}{\parindent}
\bgroup % to change \parindent locally
\setlength{\parindent}{\negparindent}
\small

\hspace{\parindent}\href{https://en.wikiversity.org/wiki/Introduction_to_Elasticity/Hellinger-Reissner_principle}{\textbold{Prange\hbox{--}Hellinger\hbox{--}Reissner Variational Principle}},\\
named after \emph{Ernst Hellinger}, \emph{Georg Prange} and \emph{Eric Reissner}.
\par

\nopagebreak\vspace{.16em}
{\scriptsize \noindent Working independently of Hellinger and Prange, Eric Reissner published his famous six\hbox{-}page paper \inquotes{On a~variational theorem in~elasticity} in~1950. In this paper he develops\:--- without, however, considering Hamilton\hbox{--}Jacobi theory\:--- a~variational principle same to that of~Prange and~Hellinger.\par}

\nopagebreak\vspace{.32em}
\href{https://en.wikiversity.org/wiki/Introduction_to_Elasticity/Hu-Washizu_principle}{\textbold{Hu\hbox{--}Washizu Variational Principle}},\\
named after \emph{Hu Haichang} and \emph{Kyuichiro\;Washizu}.
\par
\egroup
\nopagebreak\vspace{.1em}
\end{changemargin}

\begin{otherlanguage}{russian}

\noindent
Именами Reissner’а, Prange’а и~Hellinger’а назван такой функционал над~перемещениями и~напряжениями:

\nopagebreak\vspace{-0.25em}\begin{multline}\label{reissnerhellingervariationalprinciple}
\shoveleft{\displaystyle \mathcal{R}(\bm{u}, \hspace{-0.32ex}\linearstress) \hspace{0.2ex} =
\integral\displaylimits_{\mathcal{V}} \hspace{-0.4ex}
\left[^{\mathstrut} \hspace{-0.2ex}
\linearstress \dotdotp \hspace{-0.16ex} \boldnabla {\bm{u}}^{\hspace{0.08ex}\mathsf{S}} \hspace{-0.2ex} - \hspace{.1ex} \widehat{\Pi}(\hspace{-0.1ex}\linearstress\hspace{.1ex})\hspace{-0.1ex} -
\bm{f} \hspace{-0.1ex} \dotp \bm{u}
\hspace{.1ex} \right] \hspace{-0.32ex} d\mathcal{V} \hspace{.5ex} - \hfill} \\[-1.2em]
%
\hspace{8em} \displaystyle - \integral\displaylimits_{o_1} \hspace{-0.25ex} \bm{n} \dotp \hspace{-0.1ex} \linearstress \dotp \left( \bm{u} - \bm{u}_{\raisemath{-0.1em}{0}} \right) \hspace{-0.1ex} do \hspace{.2ex} -
\integral\displaylimits_{o_2} \hspace{-0.2ex} \bm{p} \dotp \bm{u} \hspace{.25ex} do
\hspace{.2ex}.
\end{multline}


...


Преимущество принципа Рейсснера\hbox{--}Хеллингера\:--- в~свободе варьирования. Но есть и~изъян: у~функционала нет экстремума на~истинном решении, а~лишь стационарность.

Принцип можно использовать для~построения приближённых решений методом Ритца~(Ritz method). Задавая аппроксимации


...


Принцип Ху\hbox{--}Васидзу~\cite{washizu} формулируется так:

\nopagebreak\vspace{-0.2em}\begin{multline}\label{huwashizuvariationalprinciple}
\hfil \variation{\mathcal{W}} \hspace{.1ex} (\bm{u}, \hspace{-0.32ex}\mathboldepsilon, \hspace{-0.32ex}\linearstress) = 0
\hspace{.1ex} ,
\\[-0.1em]
%
\shoveleft{\displaystyle \mathcal{W} \hspace{0.2ex} \equiv
\integral\displaylimits_{\mathcal{V}} \hspace{-0.4ex}
\left[^{\mathstrut} \hspace{-0.2ex}
\linearstress \dotdotp \hspace{-0.4ex} \left( \hspace{-0.2ex} \boldnabla {\bm{u}}^{\hspace{0.08ex}\mathsf{S}} \hspace{-0.2ex} - \mathboldepsilon \hspace{-0.1ex} \right) \hspace{-0.25ex} + \hspace{.1ex} \Pi(\hspace{-0.1ex}\mathboldepsilon\hspace{-0.1ex})\hspace{-0.1ex} -
\bm{f} \hspace{-0.1ex} \dotp \bm{u}
\hspace{.1ex} \right] \hspace{-0.32ex} d\mathcal{V} \hspace{.5ex} - \hfill} \\[-1.2em]
%
\hspace{8em} \displaystyle - \integral\displaylimits_{o_1} \hspace{-0.25ex} \bm{n} \dotp \hspace{-0.1ex} \linearstress \dotp \left( \bm{u} - \bm{u}_{\raisemath{-0.1em}{0}} \right) \hspace{-0.1ex} do \hspace{0.2ex} -
\integral\displaylimits_{o_2} \hspace{-0.2ex} \bm{p} \dotp \bm{u} \hspace{.25ex} do
\hspace{.2ex} .
\end{multline}

Как и~в~принципе Рейсснера\hbox{--}Хеллингера, здесь нет ограничений ни~в~объёме, ни~на~поверхности, но добавляется третий независимый аргумент~$\mathboldepsilon$. Поскольку ${\widehat{\Pi} = \linearstress \dotdotp \mathboldepsilon - \Pi}$, то~\eqref{reissnerhellingervariationalprinciple} и~\eqref{huwashizuvariationalprinciple} кажутся почти одним и~тем~же.

Из~принципа Ху\hbox{--}Васидзу вытекает вся полная система уравнений с~граничными условиями, так~как

...


Об~истории открытия вариационных принципов и~соотношении~их написано, например, у~Ю.\,Н.\;Работнова~\cite{rabotnov-mechanicsofdeformable}.

\end{otherlanguage}

\en{\section{Antiplane shear}} % Antiplane deformation % Antiplane strain

\ru{\section{Антиплоский сдвиг}}

\begin{otherlanguage}{russian}

\newcommand\antiplanedisplacement{\mathcolor{red}{\upupsilon}}

Это та проблема линейной теории упругости, где простыми выкладками получаются нетривиальные результаты\footnote{Нетривиальное в~теории упругости это, например, когда \inquotes{деление силы на~площадь} даёт бесконечно больш\'{у}ю погрешность в~нахождении напряжения.}\hbox{\hspace{-1ex}.}

Рассматривается изотропная среда в~декартовых координатах $x_i$ ($x_1$ и~$x_2$ в~плоскости, $\mathcolor{blue}{x_3}$ перпендикулярна плоскости)
с~базисными ортами
${\bm{e}_i \hspace{-0.2ex} = \partial_i \hspace{.1ex} \bm{r}}$\hbox{\hspace{-0.12ex},}
${\bm{r} = x_i \bm{e}_i}$,
${\bm{e}_i \bm{e}_i \hspace{-0.2ex} = \hspace{-0.15ex} \bm{E} \,\Leftrightarrow\hspace{.25ex} \bm{e}_i \hspace{-0.1ex} \dotp \bm{e}_j \hspace{-0.2ex} = \delta_{i\hspace{-0.1ex}j}}$.
\en{In~case}\ru{В~случае} \en{of~anti\-plane deformation/strain}\ru{анти\-плос\-кой деформации} (\en{anti\-plane shear}\ru{анти\-плос\-кого сдвига}) поле перемещений~$\bm{u}$ параллельно координате $\mathcolor{blue}{x_3}$:
${\bm{u} = \antiplanedisplacement \hspace{.12ex} \mathcolor{blue}{\bm{e}_3}}$, и~$\antiplanedisplacement$ не~зависит от~$\mathcolor{blue}{x_3}$: ${\antiplanedisplacement \narroweq \antiplanedisplacement(x_1, x_2)}$, ${\partial_{\raisemath{-0.15ex}{\mathcolor{blue}{3}}} \antiplanedisplacement \hspace{-0.15ex} = 0}$.

Деформация

\nopagebreak\vspace{-0.25em}\begin{equation*}
\mathboldepsilon \equiv
\hspace{-0.1ex} \boldnabla {\bm{u}}^{\hspace{.1ex}\mathsf{S}} \hspace{-0.2ex}
= \hspace{-0.2ex} \boldnabla \hspace{.1ex} \bigl( \antiplanedisplacement \hspace{.1ex} \mathcolor{blue}{\bm{e}_3} \bigr)^{\hspace{-0.2ex}\mathsf{S}} \hspace{-0.25ex}
= \mathcolor{blue}{\bm{e}_3} \hspace{-0.15ex} \boldnabla \hspace{.1ex} \antiplanedisplacement^{\hspace{.1ex}\mathsf{S}} \hspace{-0.25ex}
+ \antiplanedisplacement \hspace{.1ex} \tikzmark{beginZero2} \boldnabla \mathcolor{blue}{\bm{e}_3}\tikzmark{endZero2}^{\mathsf{S}} \hspace{-0.2ex}
= \hspace{.1ex} \smalldisplaystyleonehalf \hspace{.1ex} \bigl( \boldnabla \hspace{.1ex} \antiplanedisplacement \hspace{.12ex} \mathcolor{blue}{\bm{e}_3} \hspace{-0.1ex} + \mathcolor{blue}{\bm{e}_3} \hspace{-0.15ex} \boldnabla \hspace{.1ex} \antiplanedisplacement \bigr)
\end{equation*}%
\AddUnderBrace[line width=.75pt][.2ex, -0.1ex]%
{beginZero2}{endZero2}{${\scriptstyle {^2\bm{0}}}$}%

...

Возможна неоднородность среды в~плоскости ${x_1, x_2}$: ${\mu \narroweq \mu(x_1, x_2)}$, ${\partial_{\raisemath{-0.15ex}{\mathcolor{blue}{3}}} \hspace{.08ex} \mu \hspace{-0.1ex} = 0}$.

...



\end{otherlanguage}

\en{\section{Twisting of rods}} % torsion twisting of rods

\ru{\section{Кручение стержней}}

\label{para:twistingofrods.saintvenant}

\begin{otherlanguage}{russian}

Эта задача, тщательно изученная Saint\hbox{-\hspace{-0.2ex}}Venant’ом, рассматривается едва~ли не~в~каждой книге по классической теории упругости. Речь идёт о~цилиндре какого\hbox{-}либо сечения, нагруженном лишь поверхностными силами на торцах

...



\end{otherlanguage}

\en{\section{Plane deformation}}

\ru{\section{Плоская деформация}}

\label{para:planedeformation.linearclassicalelasticity}

\begin{otherlanguage}{russian}

Тут вектор перемещения~$\bm{u}$ параллелен плоскости ${x_1, x_2}$ и~не~зависит от~третьей координаты~$z$

...

Для примера рассмотрим полуплоскость с~сосредоточенной нормальной силой~$Q$ на~краю (?? рисунок ??)

...

\end{otherlanguage}

\vspace{8mm}
\hfill\begin{minipage}[b]{0.95\linewidth}
\fontsize{10}{12}\selectfont

\section*{\wordforbibliography}

\begin{otherlanguage}{russian}

Можно назвать несколько десятков книг по~классической теории упругости, представляющих несомненный интерес несмотря на~возрастающую отдалённость во~времени. Подробные литературные указания содержатся в~фундаментальной монографии ...

\end{otherlanguage}

\end{minipage}
