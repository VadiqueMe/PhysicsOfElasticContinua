\en{\chapter{Classical linear elasticity}}

\ru{\chapter{Классическая линейная упругость}}

\thispagestyle{empty}

\label{chapter:linearclassicalelasticity}

\begin{changemargin}{\parindent}{\parindent}
\vspace{-2.5em}
{\noindent\small
\setlength{\parskip}{\spacebetweenparagraphs}

\hspace*{\fill}
\en{Geometrically linear model}\ru{Геометрически линейная модель}: \en{displacements are small}\ru{перемещения м\'{а}лые}.

\en{Operators}\ru{Операторы} ${\hspace{-0.2ex}\smash{\boldnablacircled}\hspace{.1ex}}$ \en{and}\ru{и}~${\hspace{-0.2ex}\smash{\boldnabla}\hspace{.1ex}}$ \en{are indistinguishable}\ru{неразличимы}, ${\mathcal{V} \hspace{-0.2ex} = \hspace{-0.2ex} \smash{\mathcircabove{\mathcal{V}}}\hspace{-0.1ex}}$, ${\hspace{.2ex}\rho \hspace{-0.12ex} = \hspace{-0.22ex} \smash{\mathcircabove{\rho}} \hspace{.2ex}}$\:--- \inquotesx{\en{equations}\ru{уравнения} \en{can~be written in the initial configuration}\ru{можно пис\'{а}ть в~начальной конфигурации}}[,] \en{operators}\ru{операторы} $\variation$ \en{and}\ru{и}~${\hspace{-0.2ex}\boldnabla}$ \en{commute}\ru{коммутируют}~(${\variation{\boldnabla \fieldofdisplacements} \hspace{-0.1ex} = \hspace{-0.3ex} \boldnabla \variation{\fieldofdisplacements}}$).

\par}
\vspace{-1.2em}
\end{changemargin}

\en{\section{Complete set of equations}}

\ru{\section{Полный набор уравнений}}

\label{para:wholesetofequations.lineartheory}

\en{\dropcap{E}{quations} of~nonlinear elasticity}\ru{\dropcap{У}{равнения} нелинейной упругости}, \en{even in simplest cases}\ru{даже в~самых простых случаях}, \en{bring to mathematically complicated problems}\ru{приводят к~математически сложным задачам}.
\en{Therefore}\ru{Поэтому} \en{the~linear theory of~small displacements is applied everywhere}\ru{повсеместно применяется линейная теория малых перемещений}.
\en{Equations}\ru{Уравнения} \en{of this theory}\ru{этой теории} \en{were derived}\ru{были выведены} \en{in the~first half}\ru{в~первой половине} \en{of~the~}\hbox{XIX$^{\textrm{\en{th}\ru{го}}}$\hspace{-0.2ex}}~\en{century}\ru{века}\en{ by}
\href{https://en.wikipedia.org/wiki/Augustin-Louis_Cauchy}{Cauchy},
\href{https://en.wikipedia.org/wiki/Claude-Louis_Navier}{Navier},
\href{https://en.wikipedia.org/wiki/Gabriel_Lam\%C3\%A9}{Lam\'{e}},
\href{https://en.wikipedia.org/wiki/Beno\%C3\%AEt_Paul_\%C3\%89mile_Clapeyron}{Clapeyron}\ru{’ом},
\href{https://en.wikipedia.org/wiki/Sim\%C3\%A9on_Denis_Poisson}{Poisson}\ru{’ом},
\href{https://en.wikipedia.org/wiki/Adh\%C3\%A9mar_Jean_Claude_Barr\%C3\%A9_de_Saint-Venant}{Saint\hbox{-\hspace{-0.2ex}}Venant}\ru{’ом},
\href{https://en.wikipedia.org/wiki/George_Green_(mathematician)}{George Green}\ru{’ом}
\en{and other scientists}\ru{и~другими учёными}.

\en{The~complete closed set}\ru{Полный замкнутый набор}~(\en{system}\ru{система}) \en{of~equations}\ru{уравнений} \en{of~the~classical linear theory}\ru{классической линейной теории} \en{in~the~direct invariant tensor notation}\ru{в~прямой инвариантной тензорной з\'{а}писи}\ru{:}\en{ is}

\nopagebreak\vspace{-0.1em}
\hspace*{-\parindent}\begin{minipage}{\linewidth}
\begin{equation}\label{lineartheory:wholesetofequations}
\boldnabla \dotp \linearstress \hspace{.15ex} + \bm{f} = \hspace{.1ex} \bm{0} \hspace{.1ex} ,
\hspace{.7em}
%
\linearstress = \scalebox{0.92}{$ \displaystyle \frac{\raisemath{-0.133em}{\partial\hspace{.1ex} \potential^{\mathstrut}}}{\raisemath{-0.07em}{\partial \infinitesimaldeformation}}$} = \stiffnesstensor \dotdotp \hspace{-0.1ex} \infinitesimaldeformation ,
\hspace{.7em}
%
\infinitesimaldeformation = \hspace{-0.2ex} \boldnabla {\bm{u}}^{\hspace{.1ex}\mathsf{S}}
\hspace{-0.2ex} .
\vspace{.2em}\end{equation}

\nopagebreak\noindent
\en{Here}\ru{Здесь}
$\linearstress$\ru{\:---}\en{~is} \en{linear stress tensor}\ru{линейный тензор напряжения},
$\bm{f}$~\en{is vector of~volume loads}\ru{--- вектор объёмных нагрузок},
$\infinitesimaldeformation$\ru{\:---}\en{~is} \en{infinitesimal linear deformation tensor}\ru{тензор бесконечномалой линейной деформации},
${\potential(\hspace{-0.1ex}\infinitesimaldeformation\hspace{-0.1ex})\hspace{-0.1ex}}$\ru{\:---}\en{~is} \en{potential energy of~deformation per volume unit}\ru{потенциальная энергия деформации единицы объёма},
${\stiffnesstensor}$\ru{\:---}\en{~is} \en{stiffness tensor}\ru{тензор жёсткости} (\en{tetravalent}\ru{четырёхвалентный}, \en{with symmetry}\ru{с~симметрией} ${\stiffnesstensor_{\hspace{.12ex} 12 \scalebox{0.6}[0.8]{$\rightleftarrows$} 34} \hspace{-0.4ex} = \hspace{-0.1ex} \stiffnesstensor}$\hbox{\hspace{.1ex},} ${\stiffnesstensor_{\hspace{.12ex} 1 \scalebox{0.6}[0.8]{$\rightleftarrows$} 2} \hspace{-0.33ex} = \hspace{-0.1ex} \stiffnesstensor}$\hbox{\hspace{.1ex},} ${\stiffnesstensor_{\hspace{.12ex} 3 \scalebox{0.6}[0.8]{$\rightleftarrows$} 4} \hspace{-0.33ex} = \hspace{-0.1ex} \stiffnesstensor}$).
\end{minipage}

\en{Equations}\ru{Уравнения}~\eqref{lineartheory:wholesetofequations} \en{are exact}\ru{точные}, \en{they can be derived via variation}\ru{они могут быть получены варьированием} \en{of equations of the~nonlinear theory}\ru{уравнений нелинейной теории}.
\en{Variation from an~arbitrary configuration}\ru{Варьирование от~произвольной конфигурации} \en{is~described in}\ru{описано в}~\chapdotpararef{chapter:nonlinearcontinuum}{para:variationofconfiguration}.
\en{The~linear theory}\ru{Линейная теория}\ru{\:---}\en{ is} \en{the~result of variation}\ru{результат варьирования} \en{from the~initial unstressed configuration}\ru{от~начальной ненапряжённой конфигурации}, \en{where}\ru{где}

\nopagebreak\begin{equation}\label{variationfrominitialconfiguration}
\begin{array}{c}
\bm{F} = \UnitDyad \hspace{.1ex} ,
\:\;
\bm{C} \hspace{-0.1ex} = {\hspace{-0.2ex}^2\bm{0}}
\hspace{.1ex} ,
\:\;
\variation{\hspace{.12ex}\bm{C}} \hspace{-0.1ex} = \hspace{-0.2ex} \boldnabla \hspace{.12ex} \variation{\currentlocationvector}^{\hspace{.2ex}\mathsf{S}} \hspace{-0.15ex}
\equiv \infinimentpetitdeformationdevariation ,
\\[.2em]
%
\cauchystress = {\hspace{-0.2ex}^2\bm{0}}
\hspace{.1ex} ,
\:\;
\varbivalent{\hspace{-0.2ex}\cauchystress}
= \variation{\hspace{.1ex}\firstpiolakirchhoffstress}
= \scalebox{0.9}{$ \displaystyle \frac{\raisemath{-0.125em}{\partial^2 \hspace{.1ex} \potential}}{\raisemath{-0.1em}{\partial \hspace{.1ex} \bm{C} \hspace{.1ex} \partial \hspace{.1ex} \bm{C}}} $} \hspace{-0.1ex}
\dotdotp \variation{\hspace{.12ex}\bm{C}} ,
\:\:
\boldnabla \dotp \varbivalent{\hspace{-0.2ex}\cauchystress} \hspace{.1ex}
+ \hspace{.1ex} \rho \hspace{.25ex} \variation{\hspace{-0.2ex}\bm{f}}
= \hspace{.1ex} \bm{0}
\hspace{.11ex} .
\end{array}
\end{equation}

\vspace{-0.1em}\noindent
\en{It remains to~change}\ru{Остаётся поменять}
${\variation{\currentlocationvector}}$ \en{to}\ru{на}~$\bm{u}$,
${\infinimentpetitdeformationdevariation\hspace{-0.15ex}}$ \en{to}\ru{на}~$\infinitesimaldeformation$,
$\varbivalent{\hspace{-0.15ex}\cauchystress}$ \en{to}\ru{на}~$\linearstress$,
${\scalebox{0.95}{$ \raisemath{.16em}{\scalebox{0.92}{$\partial^2 \hspace{.1ex} \potential$}} / \hspace{-0.1ex} \raisemath{-0.32em}{\scalebox{0.92}{$\partial \hspace{.1ex} \bm{C} \hspace{.1ex} \partial \hspace{.1ex} \bm{C}$}} $}}$ \en{to}\ru{на}~$\stiffnesstensor$,
\en{and}\ru{а}~${\hspace{-0.1ex} \rho \hspace{.25ex} \variation{\hspace{-0.2ex}\bm{f}}}$ \en{to}\ru{на}~${\hspace{-0.1ex} \bm{f}\hspace{-0.2ex}}$.

\en{If}\ru{Если} \en{derivation}\ru{вывод}~\eqref{variationfrominitialconfiguration} \en{seems abstruse to the~reader}\ru{кажется читателю малопонятным}, \en{it’s possible to proceed from equations}\ru{возможно исходить из~уравнений}

\nopagebreak\vspace{-0.25em}\begin{equation}\label{nonlinear:setofequations}
\begin{array}{c}
\boldnabla \dotp \cauchystress \hspace{.15ex} + \rho \bm{f} = \hspace{.1ex} \bm{0} \hspace{.1ex},
\:\:
\boldnabla = \bm{F}^{-\T} \hspace{-0.2ex} \dotp \boldnablacircled ,
\:\:
\bm{F} = \UnitDyad \hspace{.1ex} + \hspace{-0.2ex} \boldnablacircled {\bm{u}}^{\T} \hspace{-0.3ex},
\\[.2em]
%
\cauchystress \hspace{.1ex} = J^{-1} \hspace{.2ex} \bm{F} \dotp \scalebox{0.9}{$ \displaystyle \frac{\raisemath{-0.125em}{\partial\hspace{.1ex} \potential}}{\raisemath{-0.1em}{\partial\hspace{.1ex} \bm{C}}} $} \dotp \bm{F}^{\hspace{.1ex}\T} \hspace{-0.4ex},
\:\:
\bm{C} = \boldnablacircled {\bm{u}}^{\hspace{.1ex}\mathsf{S}} \hspace{-0.3ex} + \smalldisplaystyleonehalf \boldnablacircled \bm{u} \dotp \hspace{-0.25ex} \boldnablacircled \bm{u}^{\T} \hspace{-0.25ex}.
\end{array}
\end{equation}

\vspace{-0.1em}\noindent
\en{Assuming}\ru{Полагая} \en{displacement}\ru{перемещение}~$\bm{u}$ \en{is small}\ru{м\'{а}лым}, \en{we’ll move}\ru{перейдём} \en{from}\ru{от}~\eqref{nonlinear:setofequations} \en{to}\ru{к}~\eqref{lineartheory:wholesetofequations}.

\begin{otherlanguage}{russian}

Или вот как.
Вместо $\bm{u}$ взять $\smallparameter \bm{u}$, тут ${\smallparameter \hspace{-0.1ex}\to 0}$\:--- некоторый весьма малый параметр.
А~неизвестные представить рядами по~целым степеням~$\smallparameter$

\nopagebreak\vspace{-0.1em}\begin{equation*}
\begin{array}{c}
\cauchystress \hspace{.1ex} = \cauchystress^{\hspace{.2ex}\scalebox{0.66}{(0)}} \hspace{-0.2ex} + \smallparameter \cauchystress^{\hspace{.2ex}\scalebox{0.66}{(1)}} \hspace{-0.1ex} + \ldots \hspace{.1ex},
\:\:
\bm{C} = \bm{C}^{\hspace{.2ex}\scalebox{0.66}{(0)}} \hspace{-0.2ex} + \smallparameter \bm{C}^{\hspace{.2ex}\scalebox{0.66}{(1)}} \hspace{-0.1ex} + \ldots \hspace{.1ex} ,
\\[.1em]
%
\boldnabla \hspace{.1ex} = \boldnablacircled \hspace{.1ex} + \smallparameter \boldnabla^{\hspace{.2ex}\scalebox{0.66}{(1)}} \hspace{-0.1ex} + \ldots \hspace{.1ex} , \:\:
\bm{F} = \UnitDyad + \hspace{-0.1ex} \smallparameter \boldnablacircled {\bm{u}}^{\T} \hspace{-0.3ex},
\:\:
J = 1 + \hspace{-0.1ex} \smallparameter J^{\hspace{.1ex}\scalebox{0.66}{(1)}} \hspace{-0.1ex} + \ldots
\end{array}
\end{equation*}

\vspace{-0.1em}\noindent
Для первых~(\inquotes{нулевых}) членов этих разложений и~получается~\eqref{lineartheory:wholesetofequations}.
В~\hbox{книге~\cite{truesdell-firstcourse}} си\'{е} названо \inquotesx{формальным приближением}[.]

Невозможно сказать в~общем случае, насколько мал должен~быть параметр~$\smallparameter$\:--- ответ зависит от~ситуации и~определяется лишь тем, описывает линейная модель интересующий эффект или~нет.
Когда, например, интересна связь частот\'{ы} свободных колебаний упругого тела с~амплитудой, то нужна уж\'{е} нелинейная модель.

\en{A~linear problem}\ru{Линейная задача} \en{is posed}\ru{ставится} \en{in the~initial volume}\ru{в~начальном объёме}~\hbox{$\mathcal{V} \hspace{-0.2ex} = \hspace{-0.2ex} \mathcircabove{\mathcal{V}}$\hspace{-0.12em}}, \en{bounded by surface}\ru{ограниченном поверхностью}~$o$ \en{with area vector}\ru{с~вектором пл\'{о}щади}~${\unitnormalvector do}$ (\inquotes{\en{the~principle of initial dimensions}\ru{принцип начальных размеров}}).

\en{Boundary}\ru{Краевые~(граничные)} \en{conditions}\ru{условия} \en{most often}\ru{чаще всего} \en{are}\ru{такие}: \en{on}\ru{на} \hbox{\en{part}\ru{части}}~${o_1}$ \en{of~the~surface}\ru{поверхности} \ru{известны }\en{displacements}\ru{перемещения}\en{ are known}~(\en{geometrical/kinematical condition}\ru{геометрическое/кинематическое условие}), \en{and}\ru{а} \en{on}\ru{на} \en{another}\ru{другой} \hbox{\en{part}\ru{части}}~${o_2}$\:--- \en{forces}\ru{силы}~(\en{mechanical condition}\ru{механическое условие})

\nopagebreak\vspace{-0.2em}\begin{equation}\label{lineartheory:boundaryconditions}
\bm{u} \hspace{.1ex} \bigr|_{o_1} \hspace{-0.64ex} = \hspace{.2ex} \bm{u}_{\raisemath{-0.1em}{0}}
\hspace{.2ex} ,
\hspace{.8em}
\unitnormalvector \dotp \linearstress \hspace{.2ex} \bigr|_{o_2} \hspace{-0.64ex} = \hspace{.2ex} \bm{p}
\hspace{.2ex} .
\end{equation}

\en{More complex combinations happen too}\ru{Бывают и~более сложные комбинации}, \en{if we know}\ru{если мы знаем} \en{certain}\ru{некоторые} \en{components}\ru{компоненты} \en{of both}\ru{как}~$\bm{u}$\ru{,} \en{and}\ru{так~и}~${\tractionvector{n} \hspace{-0.2ex} = \unitnormalvector \dotp \linearstress}$ \en{simultaneously}\ru{одновременно}.
\en{For example}\ru{Для примера}, на~плоской грани~${x = \constant}$ при~вдавливании штампа с~гладкой поверхностью ${u_x \hspace{-0.25ex} = \nu \hspace{.1ex} (y , \hspace{-0.1ex} z)}$, ${\mathtau_{xy} \hspace{-0.2ex} = \mathtau_{xz} \hspace{-0.2ex} = 0}$ (функция~$\nu$ определяется формой штампа).

\en{Initial conditions}\ru{Начальные условия} \en{in dynamic problems}\ru{в~динамических задачах}, \en{when}\ru{когда} \en{instead of}\ru{вместо}~$\bm{f}$ \en{we have}\ru{мы имеем} ${\bm{f} \hspace{-0.1ex} - \hspace{-0.1ex} \rho \hspace{.1ex} \mathdotdotabove{\bm{u}}}$, ставятся как обычно в~механике\:--- на~положения и~на~скорости: в~условный момент времени~${t \narroweq \hspace{.1ex} 0}$ определены~$\bm{u}$ и~$\mathdotabove{\bm{u}}$.

Плотность потенциальной энергии деформации

\nopagebreak\vspace{-0.2em}\begin{equation*}
\begin{gathered}
\variation{\hspace{.1ex}\potential}(\hspace{-0.1ex}\infinitesimaldeformation\hspace{-0.1ex}) \hspace{-0.2ex}
= \scalebox{0.9}{$ \displaystyle\frac{\raisemath{-0.16em}{\partial\hspace{.1ex} \potential}}{\partial \infinitesimaldeformation}$} \hspace{-0.1ex} \dotdotp \variation{\infinitesimaldeformation}
= \linearstress \dotdotp \variation{\infinitesimaldeformation}
= \infinitesimaldeformation \hspace{-0.1ex} \dotdotp \hspace{-0.1ex} \stiffnesstensor \dotdotp \variation{\infinitesimaldeformation}
\\[.1em]
%
\secondvariation{\hspace{.1ex}\potential}(\hspace{-0.1ex}\infinitesimaldeformation\hspace{-0.1ex}) \hspace{-0.2ex}
= \variation{\infinitesimaldeformation} \hspace{-0.1ex} \dotdotp \scalebox{0.9}{$ \displaystyle\frac{\raisemath{-0.16em}{\partial^2\hspace{.1ex} \potential}}{\partial \infinitesimaldeformation \hspace{.1ex} \partial \infinitesimaldeformation}$} \hspace{-0.1ex} \dotdotp \variation{\infinitesimaldeformation}
= \variation{\infinitesimaldeformation} \hspace{-0.1ex} \dotdotp \stiffnesstensor \dotdotp \variation{\infinitesimaldeformation}
= 2 \potential(\variation{\infinitesimaldeformation})
\\[.3em]
%
\potential(\hspace{-0.1ex}\infinitesimaldeformation\hspace{-0.1ex}) \hspace{-0.2ex}
= \hspace{.1ex} \smash{\smalldisplaystyleonehalf} \infinitesimaldeformation \hspace{-0.1ex} \dotdotp \hspace{-0.1ex} \stiffnesstensor \dotdotp \hspace{-0.1ex} \infinitesimaldeformation
\\[.2em]
%
\variation{\potential}
= \hspace{.1ex} \smash{\smalldisplaystyleonehalf} \hspace{.25ex} \variation{ \bigl( \infinitesimaldeformation \hspace{-0.1ex} \dotdotp \hspace{-0.1ex} \stiffnesstensor \dotdotp \hspace{-0.1ex} \infinitesimaldeformation \bigr) } \hspace{-0.25ex}
= \hspace{.1ex} \smash{\smalldisplaystyleonehalf} \hspace{-0.1ex} \bigl( \scalebox{0.95}{$ \variation{\infinitesimaldeformation} \hspace{-0.1ex} \dotdotp \hspace{-0.1ex} \stiffnesstensor \dotdotp \hspace{-0.1ex} \infinitesimaldeformation + \infinitesimaldeformation \hspace{-0.1ex} \dotdotp \hspace{-0.1ex} \stiffnesstensor \dotdotp \variation{\infinitesimaldeformation} $} \bigr) \hspace{-0.25ex}
= \tikzmark{beginLinearStressAsDeformationAndStiffness} \infinitesimaldeformation \hspace{-0.1ex} \dotdotp \hspace{-0.1ex} \stiffnesstensor \tikzmark{endLinearStressAsDeformationAndStiffness} \dotdotp \variation{\infinitesimaldeformation}
\end{gathered}
\end{equation*}%
\AddUnderBrace[line width=.75pt][0, .1ex][yshift = .2ex]%
{beginLinearStressAsDeformationAndStiffness}{endLinearStressAsDeformationAndStiffness}%
{${\scalebox{0.8}{$ \linearstress $}}$}
\vspace{.1em}

Как отмечалось в~\chapref{chapter:genericmechanics}, в~основу механики может быть положен принцип виртуальной работы~(\hbox{d’\hspace{-0.2ex}Alembert--Lagrange} principle).
Этот принцип справедлив и~в~линейной теории (\en{internal forces}\ru{внутренние силы} \en{in~an~elastic medium}\ru{в~упругой среде} \en{are potential}\ru{потенциальны}: ${\variation{\internalwork} = - \hspace{.2ex} \variation{\potential}}$)

\nopagebreak\vspace{-0.2em}\begin{equation}\label{lineartheory:principleofvirtualwork}
\displaystyle\integral\displaylimits_{\mathcal{V}} \hspace{-0.2ex} \Bigl[ \left( \bm{f} \hspace{-0.1ex} - \hspace{-0.1ex} \rho \hspace{.1ex} \mathdotdotabove{\bm{u}} \hspace{.2ex} \right) \hspace{-0.1ex} \dotp \variation{\bm{u}} - \variation{\potential} \hspace{.25ex} \Bigr] \hspace{-0.2ex} d\mathcal{V}
+ \hspace{-0.3ex}
\displaystyle\integral\displaylimits_{o_2} \hspace{-0.4ex} \bm{p} \dotp \variation{\bm{u}} \hspace{.25ex} do \hspace{.1ex}
= \hspace{.1ex} 0
\hspace{.1ex} ,
\hspace{.6em}
%
\bm{u} \hspace{.1ex} \bigr|_{o_1} \hspace{-0.64ex} = \hspace{.1ex} \bm{0}
\hspace{.1ex} ,
\vspace{-0.2em}\end{equation}

\noindent\vspace{-0.2em}
\en{because}\ru{потому что}

\nopagebreak\vspace{-0.2em}\begin{equation*}
\begin{gathered}
\variation{\potential}
= \linearstress \dotdotp \variation{\infinitesimaldeformation}
= \linearstress \dotdotp \hspace{-0.12ex} \boldnabla \variation{\bm{u}}^{\hspace{.1ex}\mathsf{S}} \hspace{-0.16ex}
= \boldnabla \hspace{-0.1ex} \dotp \left( \hspace{.1ex} \linearstress \hspace{-0.1ex} \dotp \variation{\bm{u}} \hspace{.1ex} \right) - \boldnabla \dotp \linearstress \dotp \variation{\bm{u}}
\hspace{.1ex} ,
\\[.1em]
%
\displaystyle \integral\displaylimits_{\mathcal{V}} \hspace{-0.5ex} \variation{\potential} \hspace{.2ex} d\mathcal{V} =
\ointegral\displaylimits_{\mathclap{o\hspace{.1ex}(\boundary \mathcal{V})}} \hspace{-0.2ex} \unitnormalvector \hspace{.1ex} \dotp \linearstress \dotp \variation{\bm{u}} \hspace{.32ex} do \hspace{.2ex} - \hspace{-0.1ex}
\integral\displaylimits_{\mathcal{V}} \hspace{-0.5ex} \boldnabla \dotp \linearstress \dotp \variation{\bm{u}} \hspace{.32ex} d\mathcal{V}
\end{gathered}
\end{equation*}

\vspace{-0.2em}\noindent
и~левая часть~\eqref{lineartheory:principleofvirtualwork} приобретает вид
\[
\displaystyle\integral\displaylimits_{\mathcal{V}} \hspace{-0.4ex} \Bigl( \hspace{-0.1ex} \boldnabla \dotp \linearstress \hspace{.1ex} + \bm{f} \hspace{-0.1ex} - \hspace{-0.1ex} \rho \hspace{.1ex} \mathdotdotabove{\bm{u}} \hspace{.1ex} \Bigr) \hspace{-0.3ex} \dotp \variation{\bm{u}} \hspace{.25ex} d\mathcal{V}
+ \hspace{-0.3ex}
\displaystyle\integral\displaylimits_{o_2} \hspace{-0.4ex} \Bigl( \bm{p} - \unitnormalvector \dotp \linearstress \Bigr) \hspace{-0.3ex} \dotp \variation{\bm{u}} \hspace{.25ex} do
\hspace{.1ex} ,
\]

\vspace{-0.4em}\noindent
что, конечно~же, равно нулю.
Отметим краевое условие ${\bm{u} \hspace{.1ex} \bigr|_{o_1} \hspace{-0.64ex} = \hspace{.2ex} \bm{0}}$: виртуальные перемещения согласованы с~этой связью\:--- ${\variation{\bm{u}} \hspace{.1ex} \bigr|_{o_1} \hspace{-0.64ex} = \hspace{.2ex} \bm{0}}$.

\end{otherlanguage}

\en{\section{Uniqueness of the solution in dynamics}}

\ru{\section{Уникальность решения в~динамике}}

\label{para:uniquenessfordynamicproblem}

\en{As is typical}\ru{Как это типично} \en{for}\ru{для} \en{linear mathematical physics}\ru{линейной математической физики},
\en{the~uniqueness theorem}\ru{теорема единственности} \en{is proven}\ru{доказывается} \inquotesx{\en{by~contradiction}\ru{от~противного}}[.]
\en{Assume}\ru{Предположим, что} \en{there are}\ru{существуют} \en{some}\ru{какие-либо} \en{two solutions}\ru{два решения}: ${\bm{u}_1 (\locationvector, t)}$ \en{and}\ru{и}~${\bm{u}_2 (\locationvector, t)}$.
\en{If}\ru{Если} \en{difference}\ru{разность}~${\smash{\bm{u}^{\hspace{-0.1ex}*}} \hspace{-0.2ex} \equiv \hspace{.12ex} \bm{u}_1 \hspace{-0.4ex} - \bm{u}_2}$ \en{will be equal to}\ru{окажется равной}~$\bm{0}$, \en{then}\ru{тогда} \en{these solutions}\ru{эти решения} \en{coincide}\ru{совпадают}, \en{that is}\ru{то есть} \en{the~solution is unique}\ru{решение единственно}.

\en{But first}\ru{Но сперв\'{а}} \en{we’ll make sure}\ru{убедимся} \en{the~existence}\ru{в~существовании} \en{of the~energy integral}\ru{интеграла энергии}\:--- \en{by deriving}\ru{путём вывода} \en{the balance of~energy equation}\ru{уравнения баланса механической энергии} \en{for the~linear model}\ru{для линейной модели} \en{of the~small displacement theory}\ru{теории малых перемещений}

\nopagebreak
\hspace*{-\parindent}\begin{minipage}{\linewidth}
\begin{equation}\label{dynamicsoflineartheory:integralofenergy}
\displaystyle\integral\displaylimits_{\mathcal{V}} \hspace{-0.4ex} \Bigl( \hspace{-0.1ex} \kinetic + \potential \Bigr)^{\hspace{-0.33ex}\tikz[baseline=-0.2ex] \draw[black, fill=black] (0,0) circle (.28ex);} d\mathcal{V}
= \hspace{-0.2ex}
\displaystyle\integral\displaylimits_{\mathcal{V}} \hspace{-0.5ex} \bm{f} \hspace{-0.1ex} \dotp \mathdotabove{\bm{u}} \hspace{.3ex} d\mathcal{V}
+ \hspace{-0.2ex}
\displaystyle\integral\displaylimits_{o_2} \hspace{-0.4ex} \bm{p} \dotp \mathdotabove{\bm{u}} \hspace{.3ex} do
\hspace{.15ex} ,
\end{equation}
%
\nopagebreak\vspace{-0.25em}\begin{equation*}
\begin{array}{c}
\bm{u} \hspace{.1ex} \bigr|_{o_1} \hspace{-0.64ex} = \hspace{.2ex} \bm{0}
\hspace{.11ex} ,
\hspace*{.6em}
%
\unitnormalvector \dotp \linearstress \hspace{.2ex} \bigr|_{o_2} \hspace{-0.64ex} = \hspace{.2ex} \bm{p}
\hspace{.15ex} ,
\\[.5em]
%
\bm{u} \hspace{.1ex} \bigr|_{t=0} \hspace{-0.2ex} = \bm{u}^{\hspace{-0.1ex}\circ}
\hspace{-0.4ex} ,
\hspace{.6em}
%
\mathdotabove{\bm{u}} \hspace{.1ex} \bigr|_{t=0} \hspace{-0.2ex} = \mathdotabove{\bm{u}}^{\circ}
\hspace{-0.4ex} .
\end{array}
\end{equation*}
\end{minipage}

\vspace{.3em}
\en{For}\ru{Для} \en{the~left\hbox{-}hand side}\ru{левой части} \en{we have}\ru{имеем}

\nopagebreak\begin{equation*}
\begin{array}{c}
\mathdotabove{\kinetic} = \smalldisplaystyleonehalf \bigl( \hspace{.1ex} \rho \hspace{.2ex} \mathdotabove{\bm{u}} \dotp \mathdotabove{\bm{u}} \hspace{.1ex} \bigr)^{\hspace{-0.2ex}\tikz[baseline=-0.2ex] \draw[black, fill=black] (0,0) circle (.28ex);} \hspace{-0.2ex}
= \smalldisplaystyleonehalf \hspace{.2ex} \rho \hspace{.2ex} \bigl( \mathdotabove{\bm{u}} \dotp \mathdotdotabove{\bm{u}} + \mathdotdotabove{\bm{u}} \dotp \mathdotabove{\bm{u}} \hspace{.1ex} \bigr)
\hspace{-0.22ex} =
\rho \hspace{.2ex} \mathdotdotabove{\bm{u}} \dotp \mathdotabove{\bm{u}}
\hspace{.2ex} ,
\\[.5em]
%
\mathdotabove{\potential} \hspace{.1ex} = \hspace{.1ex} \smash{\smalldisplaystyleonehalf} \hspace{-0.25ex} \tikzmark{beginDerivativeOfElasticPotential} \left( \hspace{.1ex} \infinitesimaldeformation \hspace{-0.1ex} \dotdotp \hspace{-0.1ex} \stiffnesstensor \dotdotp \hspace{-0.1ex} \infinitesimaldeformation \hspace{.1ex} \right)^{\tikz[baseline=-0.2ex] \draw[black, fill=black] (0,0) circle (.28ex);} \hspace{-1.2ex} \tikzmark{endDerivativeOfElasticPotential} \hspace{.7ex}
= \linearstress \dotdotp \mathdotabove{\infinitesimaldeformation}
= \linearstress \hspace{.1ex} \dotdotp \hspace{-0.12ex} \boldnabla \mathdotabove{\bm{u}}^{\hspace{.1ex}\mathsf{S}} \hspace{-0.12ex}
= \boldnabla \hspace{-0.1ex} \dotp \left( \hspace{.1ex} \linearstress \dotp \mathdotabove{\bm{u}} \hspace{.15ex} \right) -
\tikzmark{NablaDotTauDynamicsBegin} \boldnabla \dotp \linearstress \tikzmark{NablaDotTauDynamicsEnd} \dotp \mathdotabove{\bm{u}} =
\\[.4em]
%
\hspace*{4em}
= \boldnabla \hspace{-0.1ex} \dotp \left( \hspace{.1ex} \linearstress \dotp \mathdotabove{\bm{u}} \hspace{.15ex} \right)
+ ( \hspace{.12ex} \bm{f} \hspace{-0.2ex} - \hspace{-0.2ex} \rho \hspace{.1ex} \mathdotdotabove{\bm{u}} \hspace{.15ex} ) \hspace{-0.1ex} \dotp \mathdotabove{\bm{u}}
%%\hspace{.2ex} .
\end{array}
\end{equation*}%
\AddUnderBrace[line width=.75pt][.5ex, -0.4ex][yshift = .2ex]%
{beginDerivativeOfElasticPotential}{endDerivativeOfElasticPotential}%
{${\scalebox{0.75}{$ 2 \hspace{.2ex} \infinitesimaldeformation \hspace{-0.1ex} \dotdotp \hspace{-0.1ex} \stiffnesstensor \dotdotp \hspace{-0.1ex} \mathdotabove{\infinitesimaldeformation} $}}$}
\AddUnderBrace[line width=.75pt][.16ex, .1ex][yshift = .2ex]%
{NablaDotTauDynamicsBegin}{NablaDotTauDynamicsEnd}%
{${\scalebox{0.75}{$ - \hspace{.2ex} ( \hspace{.12ex} \bm{f} \hspace{-0.2ex} - \hspace{-0.2ex} \rho \hspace{.1ex} \mathdotdotabove{\bm{u}} \hspace{.15ex} ) $}}$}

\nopagebreak\vspace{-0.6em}\noindent
(\ru{использован }\en{balance of~momentum}\ru{баланс импульса} ${\hspace{-0.24ex}\boldnabla \dotp \linearstress \hspace{.15ex} + \bm{f} - \rho \hspace{.1ex} \mathdotdotabove{\bm{u}} \hspace{.1ex} = \bm{0}}$\en{ is used}),

\nopagebreak\vspace{-0.2em}\begin{equation*}
\mathdotabove{\kinetic} + \mathdotabove{\potential} \hspace{.1ex}
= \boldnabla \hspace{-0.1ex} \dotp \left( \hspace{.1ex} \linearstress \dotp \mathdotabove{\bm{u}} \hspace{.15ex} \right)
+ \bm{f} \hspace{-0.1ex} \dotp \mathdotabove{\bm{u}}
\hspace{.2ex} .
\end{equation*}

\vspace{-0.2em}\noindent
\en{Applying the~divergence theorem}\ru{Применяя теорему о~дивергенции}

\nopagebreak\vspace{-0.2em}\begin{equation*}
\scalebox{.9}{$ \displaystyle \integral\displaylimits_{\mathcal{V}} $} \hspace{-0.2ex} \boldnabla \hspace{-0.1ex} \dotp \left( \hspace{.1ex} \linearstress \dotp \mathdotabove{\bm{u}} \hspace{.15ex} \right) \hspace{-0.15ex} d\mathcal{V}
=
\scalebox{.9}{$ \displaystyle \ointegral\displaylimits_{\mathclap{o\hspace{.1ex}(\boundary \mathcal{V})}} $} \hspace{.1ex} \unitnormalvector \hspace{.1ex} \dotp \linearstress \dotp \mathdotabove{\bm{u}} \hspace{.4ex} do
\end{equation*}

\vspace{-0.33em}\noindent
\en{and}\ru{и}~\en{boundary condition}\ru{краевое условие} ${\unitnormalvector \dotp \linearstress = \bm{p}}$ \en{on}\ru{на}~$o_2$, \en{we get}\ru{получаем}~\eqref{dynamicsoflineartheory:integralofenergy}.

\en{From}\ru{Из}~\eqref{dynamicsoflineartheory:integralofenergy} \en{it follows that}\ru{следует, что} \en{without loads}\ru{без нагрузок} (\en{when}\ru{когда} \en{there’re no external volume nor surface forces}\ru{нет внешних ни объёмных, ни поверхностных сил})\en{,} \en{the~full mechanical energy}\ru{полная механическая энергия} \en{doesn’t change}\ru{не изменяется}:

\nopagebreak\vspace{-0.2em}\begin{equation}\label{dynamicsoflineartheory:fullenergyisconstantwithoutloads}
\bm{f} \hspace{-0.1ex} = \bm{0}
\hspace{1.1ex} \text{\en{and}\ru{и}} \hspace{1.2ex}
\bm{p} = \bm{0}
%
\hspace{.8ex} \Rightarrow \hspace{.55ex}
%
\scalebox{.8}{$\displaystyle \integral\displaylimits_{\mathcal{V}}$} \hspace{-0.2ex} \bigl( \hspace{.1ex} \kinetic + \potential \hspace{.1ex} \bigr) \hspace{.1ex} d\mathcal{V} = \hspace{.1ex} \constant(t)
\hspace{.2ex} .
\vspace{-0.25em}
\end{equation}

\vspace{-0.1em}\noindent
\en{If}\ru{Если} \en{at the~moment}\ru{в~момент}~${t \narroweq 0}$ \en{there was}\ru{был} \en{unstressed}\ru{ненапряжённый}\:(${\potential \narroweq \hspace{.1ex} 0}$) \en{rest}\ru{покой}\:(${\kinetic \narroweq \hspace{.1ex} 0}$), \en{then}\ru{то}

\nopagebreak\vspace{-0.25em}\begin{equation*}
\scalebox{.8}{$\displaystyle \integral\displaylimits_{\mathcal{V}}$} \hspace{-0.2ex} \bigl( \hspace{.1ex} \kinetic + \potential \hspace{.1ex} \bigr) \hspace{.1ex} d\mathcal{V} = \hspace{.1ex} 0
\hspace{.1ex} .
\tag{\theequation \raisebox{.1em}{\textquotesingle}}\label{dynamicsoflineartheory:fullenergyiszero}
\vspace{-0.25em}
\end{equation*}

\en{Kinetic energy}\ru{Кинетическая энергия} \en{is positive}\ru{положительна}:
${\kinetic \hspace{-0.2ex} > \hspace{-0.2ex} 0}$ \en{if}\ru{если}~${\mathdotabove{\bm{u}} \hspace{-0.1ex} \neq \hspace{-0.1ex} \bm{0}}$
\en{and}\ru{и}~\en{turns to zero}\ru{обращается в~нуль} \en{only when}\ru{лишь когда}~${\mathdotabove{\bm{u}} = \hspace{-0.1ex} \bm{0}}$\:--- \en{this ensues}\ru{это вытекает} \en{out of the~very definition}\ru{из~самог\'{о} определения}
${\kinetic \hspace{-0.1ex} \equiv \smallerdisplaystyleonehalf \hspace{.2ex} \rho \hspace{.2ex} \mathdotabove{\bm{u}} \dotp \mathdotabove{\bm{u}}}$.
\en{Potential energy}\ru{Потенциальная энергия}, \en{being}\ru{будучи} \en{a~quadratic form}\ru{квадратичной формой} \en{of~}\en{infinitesimal linear deformation}\ru{бесконечномалой линейной деформации}
${\potential(\hspace{-0.1ex}\infinitesimaldeformation\hspace{-0.1ex}) \hspace{-0.2ex} = \hspace{.1ex} \smash{\smallerdisplaystyleonehalf} \hspace{.2ex} \infinitesimaldeformation \hspace{-0.1ex} \dotdotp \hspace{-0.1ex} \stiffnesstensor \dotdotp \hspace{-0.1ex} \infinitesimaldeformation}$,
\en{is positive too}\ru{тоже положительна}: ${\potential \hspace{-0.16ex} > \hspace{-0.2ex} 0}$ \en{if}\ru{если}~${\infinitesimaldeformation \hspace{-0.1ex} \neq \hspace{-0.1ex} {^2\bm{0}}}$.
\en{Such is}\ru{Таков\'{о}} \en{a~priori requirement}\ru{априорное требование} \en{of~the~positive definiteness}\ru{положительной определённости} \en{for}\ru{для}~\en{stiffness tensor}\ru{тензора жёсткости}~$\stiffnesstensor$.
\en{This is one of}\ru{Это одно из}~\inquotes{\en{additional inequalities in the~theory of~elasticity}\ru{дополнительных неравенств в~теории упругости}}~\cite{lurie-nonlinearelasticity, truesdell-firstcourse}.

\en{Since}\ru{Так как}~$\kinetic$ \en{and}\ru{и}~$\potential$ \en{are positive-definite}\ru{положительно определены}, \eqref{dynamicsoflineartheory:fullenergyiszero} \en{gives}\ru{даёт}

\nopagebreak\vspace{-0.1em}
\hspace*{-\parindent}\begin{minipage}{\linewidth}
\begin{equation*}
\kinetic = 0 \hspace{.1ex}, \; \potential = 0
\hspace{.7ex} \Rightarrow \hspace{.7ex}
%
\mathdotabove{\bm{u}} = \bm{0}
\hspace{.1ex} , \;
\infinitesimaldeformation = \hspace{-0.25ex} \boldnabla {\bm{u}}^{\hspace{.1ex}\mathsf{S}} \hspace{-0.25ex} = \hspace{-0.1ex} {^2\bm{0}}
\hspace{.8ex} \Rightarrow \hspace{.7ex}
%
\bm{u} = \bm{u}^{\hspace{-0.1ex}\circ} \hspace{-0.3ex} + \hspace{.1ex} \bm{\omega}^{\circ} \hspace{-0.5ex} \times \locationvector
\end{equation*}

\nopagebreak\vspace{-0.1em}\noindent
(${\smash{\bm{u}^{\hspace{-0.1ex}\circ}} \hspace{-0.3ex}}$ \en{and}\ru{и}~${\smash{\bm{\omega}^{\circ}} \hspace{-0.4ex}}$\en{ are}\ru{\:---} \en{some constants}\ru{некоторые константы} \en{of~}\en{translation}\ru{трансляции} \en{and}\ru{и}~\en{rotation}\ru{поворота}).
\en{With an~immobile part}\ru{С~неподвижной частью} \en{of~the~surface}\ru{поверхности}

\nopagebreak\vspace{-0.2em}\begin{equation*}
\bm{u} \hspace{.1ex} \rvert_{\raisemath{-0.1em}{o_1}} \hspace{-0.7ex} = \bm{0}
\hspace{.8ex} \Rightarrow \hspace{.7ex}
%
\smash{\bm{u}^{\hspace{-0.1ex}\circ}} \hspace{-0.4ex} = \bm{0}
\hspace{1.1ex} \text{\en{and}\ru{и}} \hspace{1.1ex}
\smash{\bm{\omega}^{\circ}} \hspace{-0.4ex} = \bm{0}
\hspace{.8ex} \Rightarrow \hspace{.7ex}
%
\bm{u} = \hspace{-0.1ex} \bm{0} \hspace{1ex} \text{\en{everywhere}\ru{всюду}} .
\end{equation*}
\end{minipage}

\begin{otherlanguage}{russian}

Теперь вспомним о~двух решениях~${\bm{u}_1\hspace{-0.25ex}}$ \en{and}\ru{и}~${\bm{u}_2}$.
Разность ${\smash{\bm{u}^{\hspace{-0.1ex}*}} \hspace{-0.25ex} \equiv \hspace{.1ex} \bm{u}_1 \hspace{-0.36ex} - \bm{u}_2}$ \en{is}\ru{есть} \en{a~solution}\ru{решение} \en{of~an~entirely}\ru{полностью} \inquotes{\en{homogeneous}\ru{однородной}} (\en{with no constant terms at~all}\ru{совсем без постоянных членов}) \en{linear}\ru{линейной} \en{problem}\ru{задачи}: в~объёме ${\bm{f} \hspace{-0.1ex} = \bm{0}}$, в~краевых и~в~начальных условиях\:--- нули.
\en{Therefore}\ru{Поэтому} ${\smash{\bm{u}^{\hspace{-0.1ex}*}} \hspace{-0.32ex} = \bm{0}}$\:--- \en{uniqueness is proven}\ru{единственность доказана}.

Что~же касается существования решения, то простыми выкладками его в~общем случае не~обосновать.
Отметим лишь, что динамическая задача является эволюционной, то~есть описывает развитие процесса во~времени.
Из~баланса импульса находим ускорение~$\mathdotdotabove{\bm{u}}$, далее переходим на~\inquotes{следующий временной слой} ${t \hspace{-0.1ex} + \hspace{-0.15ex} dt}$:

...


Разумеется, эти соображения лишены математической точности, характерной, например, для~монографии Philippe Ciarlet~\cite{ciarlet-mathematicalelasticity}.

\end{otherlanguage}

\en{\section{Hooke’s law}}

\ru{\section{Закон Гука}}

\label{para:hookelaw}

\nopagebreak\vspace{-2.4em}
${\small \linearstress \hspace{.1ex} = \displaystyle \frac{\raisemath{-0.2em}{\partial\hspace{.1ex} \potential}}{\partial \infinitesimaldeformation} = \stiffnesstensor \dotdotp \hspace{-0.1ex} \infinitesimaldeformation = \infinitesimaldeformation \hspace{-0.1ex} \dotdotp \stiffnesstensor}$
\nopagebreak\vspace{.5em}

\nopagebreak
\en{That}\ru{То} \en{relation}\ru{соотношение} \en{between}\ru{между} \en{stress}\ru{напряжением} \en{and}\ru{и}~\en{deformation~(strain)}\ru{деформацией}, \en{which}\ru{которое} \en{in}\ru{в}~\en{the~}\hbox{XVII$^{\hspace{.15ex}\textrm{\en{th}\ru{ом}}}$\hspace{-0.12em}}~\en{century}\ru{веке} Robert Hooke \en{could only phrase}\ru{мог высказать лишь} \en{pretty vaguely}\ru{весьма расплывчато}\footnote{\hspace*{.2em}\inquotesx{\textit{ceiiinosssttuu, id est, Ut tensio sic vis}}[---] \bibauthor{Robert Hooke}. \href{https://play.google.com/books/reader?id=LAtPAAAAcAAJ&pg=GBS.PA1}{Lectures de Potentia Restitutiva, Or of Spring Explaining the Power of Springing Bodies. London, 1678. \howmanypages{56~pages.}}}\hbox{\hspace{-0.32em},}\hspace{.2em} \en{is written}\ru{записано} \en{in the~direct invariant notation}\ru{в~прямой инвариантной нотации} \en{in}\ru{в}~\eqref{lineartheory:wholesetofequations} \en{and is carried out}\ru{и~осуществляется} \en{by the~stiffness tensor}\ru{тензором жёсткости}

\nopagebreak\vspace{-0.1em}\begin{equation}
\stiffnesstensor \hspace{.1ex} =
\displaystyle\frac{\raisemath{-0.1em}{\partial^2 \hspace{.1ex} \potential}}{\raisemath{-0.1em}{\partial \infinitesimaldeformation \hspace{.1ex} \partial \infinitesimaldeformation}} = \hspace{-0.1ex}
A^{i\hspace{-0.1ex}jk\hspace{.06ex}l} \hspace{.16ex} \locationvector_\differentialindex{i} \locationvector_\differentialindex{\hspace{-0.1ex}j} \locationvector_\differentialindex{k} \locationvector_\differentialindex{l}
\hspace{.1ex} ,
\:\:
%
A^{i\hspace{-0.1ex}jk\hspace{.06ex}l} \hspace{-0.16ex} =
\displaystyle\frac{\raisemath{-0.1em}{\partial^2 \hspace{.1ex} \potential}}{\raisemath{-0.1em}{\partial \varepsilon_{i\hspace{-0.1ex}j} \hspace{.1ex} \partial \varepsilon_{\hspace{-0.1ex}k\hspace{.06ex}l}}}
\hspace{.16ex} .
\end{equation}

\vspace{-0.1em}
\en{The~stiffness tensor}\ru{Тензор жёсткости}, \en{as the~partial derivative}\ru{как частная производная} \en{of~the~scalar}\ru{скалярной} \en{elastic potential energy density}\ru{плотности упругой потенциальной энергии}~$\potential$ \en{twice}\ru{дважды} \en{by}\ru{по}~\en{the~same}\ru{тому~же} \en{bivalent tensor}\ru{бивалентному тензору}~$\infinitesimaldeformation$, \en{is symmetric}\ru{симметричен} \en{in~pairs of~indices}\ru{по~парам индексов}:
${\stiffnesstensor_{\hspace{.12ex} 12 \scalebox{0.6}[0.8]{$\rightleftarrows$} 34} \hspace{-0.25ex} = \hspace{-0.1ex} \stiffnesstensor \hspace{.5ex} \Leftrightarrow \hspace{.2ex} A^{i\hspace{-0.1ex}jk\hspace{.06ex}l} \hspace{-0.17ex} = A^{k\hspace{.06ex}l\hspace{.06ex}i\hspace{-0.1ex}j}\hspace{-0.25ex}}$.
\en{Therefrom}\ru{Оттог\'{о}} 36 \en{components}\ru{компонент} \en{out of}\ru{из}~${3^4 \hspace{-0.25ex} = \hspace{-0.12ex} 81}$ \inquotes{\en{have a~twin}\ru{имеют двойник\'{а}}} \en{and}\ru{и}~\en{only}\ru{только}~45 \en{are independent}\ru{независимы}.
\en{Furthermore}\ru{К~тому~же}, \en{due to the~symmetry}\ru{из\hbox{-}за симметрии} \en{of~infinitesimal deformation tensor}\ru{тензора бесконечномалой деформации}~$\infinitesimaldeformation$, \en{tensor}\ru{тензор}~$\stiffnesstensor$ \en{is symmetric}\ru{симметричен} \ru{ещё~и~}\en{inside}\ru{внутри} \en{each pair of~indices}\ru{каждой пары индексов}\en{ too}:
${A^{i\hspace{-0.1ex}j\hspace{-0.06ex}k\hspace{.06ex}l} \hspace{-0.16ex} = A^{j\hspace{-0.06ex}ik\hspace{.06ex}l} \hspace{-0.16ex} = A^{i\hspace{-0.1ex}jlk}}$~(${= A^{j\hspace{-0.06ex}i\hspace{.06ex}l\hspace{-0.06ex}k}}$).
\en{This}\ru{Это} \en{reduces}\ru{снижает} \en{the~number}\ru{число} \en{of~independent components}\ru{независимых компонент} \en{to}\ru{до}~21:

\nopagebreak\vspace{-0.25em}\begin{equation*}\scalebox{0.8}{$\begin{array}{l}
\scalebox{1.16}{$A^{abcd^{\mathstrut}} = A^{cdab} = A^{bacd} = A^{abdc}$}
\\[.2em]
%
A^{1\hspace{-0.1ex}1\hspace{-0.1ex}1\hspace{-0.1ex}1} \\
A^{1\hspace{-0.1ex}1\hspace{-0.1ex}12} = A^{1\hspace{-0.1ex}121} = A^{121\hspace{-0.1ex}1} = A^{21\hspace{-0.1ex}1\hspace{-0.1ex}1} \\
A^{1\hspace{-0.1ex}1\hspace{-0.1ex}13} = A^{1\hspace{-0.1ex}131} = A^{131\hspace{-0.1ex}1} = A^{31\hspace{-0.1ex}1\hspace{-0.1ex}1} \\
A^{1\hspace{-0.1ex}122} = A^{221\hspace{-0.1ex}1} \\
A^{1\hspace{-0.1ex}123} = A^{1\hspace{-0.1ex}132} = A^{231\hspace{-0.1ex}1} = A^{321\hspace{-0.1ex}1} \\
A^{1\hspace{-0.1ex}133} = A^{331\hspace{-0.1ex}1} \\
A^{1212} = A^{1221} = A^{21\hspace{-0.1ex}12} = A^{2121} \\
A^{1213} = A^{1231} = A^{1312} = A^{1321} = A^{21\hspace{-0.1ex}13} = A^{2131} = A^{31\hspace{-0.1ex}12} = A^{3121} \\
A^{1222} = A^{2122} = A^{2212} = A^{2221} \\
A^{1223} = A^{1232} = A^{2123} = A^{2132} = A^{2312} = A^{2321} = A^{3212} = A^{3221} \\
A^{1233} = A^{2133} = A^{3312} = A^{3321} \\
A^{1313} = A^{1331} = A^{31\hspace{-0.1ex}13} = A^{3131} \\
A^{1322} = A^{2213} = A^{2231} = A^{3122} \\
A^{1323} = A^{1332} = A^{2313} = A^{2331} = A^{3123} = A^{3132} = A^{3213} = A^{3231} \\
A^{1333} = A^{3133} = A^{3313} = A^{3331} \\
A^{2222} \\
A^{2223} = A^{2232} = A^{2322} = A^{3222} \\
A^{2233} = A^{3322} \\
A^{2323} = A^{2332} = A^{3223} = A^{3232} \\
A^{2333} = A^{3233} = A^{3323} = A^{3332} \\
A^{3333} \\
\end{array}$}
\end{equation*}

\begin{otherlanguage}{russian}

\vspace{-0.2em}
Нередко компоненты тензора жёсткости записывают симметричной матрицей 6×6 вида
\nopagebreak\vspace{.1em}\[ \displaystyle
\underset{\raisemath{.1em}{\scriptscriptstyle 6×6}}{[\hspace{.2ex}\mathcal{A}\hspace{.2ex}]} \hspace{.12ex} = \hspace{-0.16ex}
%
\scalebox{0.82}{$\left[\hspace{0.4em} {\begin{matrix}
a_{1} & a_{12} & a_{13} & a_{14} & a_{15} & a_{16} \\
{\color{gray}a_{12}} & a_{2} & a_{23} & a_{24} & a_{25} & a_{26} \\
{\color{gray}a_{13}} & {\color{gray}a_{23}} & a_{3} & a_{34} & a_{35} & a_{36} \\
{\color{gray}a_{14}} & {\color{gray}a_{24}} & {\color{gray}a_{34}} & a_{4} & a_{45} & a_{46} \\
{\color{gray}a_{15}} & {\color{gray}a_{25}} & {\color{gray}a_{35}} & {\color{gray}a_{45}} & a_{5} & a_{56} \\
{\color{gray}a_{16}} & {\color{gray}a_{26}} & {\color{gray}a_{36}} & {\color{gray}a_{46}} & {\color{gray}a_{56}} & a_{6}
\end{matrix}} \hspace{0.32em}\right] $}  \hspace{-0.2ex} \equiv \hspace{-0.2ex}
%
\scalebox{0.82}{$\left[\hspace{0.32em} {\begin{matrix}
A^{1\hspace{-0.1ex}1\hspace{-0.1ex}1\hspace{-0.1ex}1} & A^{1\hspace{-0.1ex}122} & A^{1\hspace{-0.1ex}133} & A^{1\hspace{-0.1ex}1\hspace{-0.1ex}12} & A^{1\hspace{-0.1ex}1\hspace{-0.1ex}13} & A^{1\hspace{-0.1ex}123} \\
{\color{gray}A^{221\hspace{-0.1ex}1}} & A^{2222} & A^{2233} & A^{1222} & A^{1322} & A^{2223} \\
{\color{gray}A^{331\hspace{-0.1ex}1}} & {\color{gray}A^{3322}} & A^{3333} & A^{1233} & A^{1333} & A^{2333} \\
{\color{gray}A^{121\hspace{-0.1ex}1}} & {\color{gray}A^{2212}} & {\color{gray}A^{3312}} & A^{1212} & A^{1213} & A^{1223} \\
{\color{gray}A^{131\hspace{-0.1ex}1}} & {\color{gray}A^{2213}} & {\color{gray}A^{3313}} & {\color{gray}A^{1312}} & A^{1313} & A^{1323} \\
{\color{gray}A^{231\hspace{-0.1ex}1}} & {\color{gray}A^{2322}} & {\color{gray}A^{3323}} & {\color{gray}A^{2312}} & {\color{gray}A^{2313}} & A^{2323}
\end{matrix}}
\hspace{.4em}\right] $}
\]

\vspace{.1em}
Даже в~декартовых координатах~$x$,\:$y$,\:$z$ квадратичная форма упругой энергии ${\potential(\hspace{-0.1ex}\infinitesimaldeformation\hspace{-0.1ex}) \hspace{-0.2ex}=\hspace{.1ex} \smallerdisplaystyleonehalf \hspace{.25ex} \infinitesimaldeformation \hspace{-0.1ex} \dotdotp \hspace{-0.1ex} \stiffnesstensor \dotdotp \hspace{-0.1ex} \infinitesimaldeformation}$ довольно\hbox{-}таки громоздкая:

\nopagebreak\vspace{-0.2em}\begin{equation}\label{elasticenergylooongcartesian}
\begin{gathered}
\hspace{-5em}
2 \hspace{.1ex} \potential
= a_1 \varepsilon^2_{x} \hspace{-0.1ex} + a_2 \varepsilon^2_{y} \hspace{-0.1ex} + a_3 \varepsilon^2_{z} \hspace{-0.1ex} + a_4 \varepsilon^2_{xy} \hspace{-0.2ex} + a_5 \varepsilon^2_{xz} \hspace{-0.2ex} + a_6 \varepsilon^2_{yz} \hspace{-0.2ex} +{}
\\[-0.1em]
\hspace{.2em}
{}+ 2 \hspace{.2ex} \bigl[ \hspace{.1ex}
\varepsilon_{x} \hspace{.1ex} ( a_{12} \varepsilon_{y} \hspace{-0.2ex} + a_{13} \varepsilon_{z} \hspace{-0.2ex} + a_{14} \varepsilon_{xy} \hspace{-0.2ex} + a_{15} \varepsilon_{xz} \hspace{-0.2ex} + a_{16} \varepsilon_{yz} ) \hspace{-0.11ex} +{}
\\[-0.1em]
\hspace{3em}
{}+ \varepsilon_{y} \hspace{.1ex} ( a_{23} \varepsilon_{z} \hspace{-0.2ex} + a_{24} \varepsilon_{xy} \hspace{-0.2ex} + a_{25} \varepsilon_{xz} \hspace{-0.2ex} + a_{26} \varepsilon_{yz} ) \hspace{-0.11ex} +{}
\\[-0.1em]
\hspace{5em}
{}+ \varepsilon_{z} ( a_{34} \varepsilon_{xy} \hspace{-0.2ex} + a_{35} \varepsilon_{xz} \hspace{-0.2ex} + a_{36} \varepsilon_{yz} ) \hspace{-0.11ex} +{}
\\[-0.1em]
\hspace{11em}
{}+ \varepsilon_{xy} \hspace{.1ex} ( a_{45} \varepsilon_{xz} \hspace{-0.2ex} + a_{46} \varepsilon_{yz} ) \hspace{-0.11ex}
+ a_{56} \varepsilon_{xz} \varepsilon_{yz} \hspace{.1ex} \bigr]
.
\end{gathered}\end{equation}

Когда добавляется материальная симметрия, число независимых компонент тензора~$\stiffnesstensor$ ещё уменьшается.

Пусть материал имеет плоскость симметрии упругих свойств ${z = \constant}$.
Тогда энергия~$\potential$ не~меняется при~перемене знаков у~$\varepsilon_{zx}$ и~$\varepsilon_{zy}$.
А~это возможно лишь если
\begin{equation}\label{zeroconstants:oneplaneofsymmetry}
\potential \hspace{.1ex} \raisemath{-0.2em}{\Bigr\vert}_{\substack{%
\varepsilon_{xz} \hspace{.2ex}=\hspace{.2ex} -\varepsilon_{xz} \\
\varepsilon_{yz} \hspace{.2ex}=\hspace{.2ex} -\varepsilon_{yz}
}} \hspace{-0.2ex} = \potential
\:\,\Leftrightarrow \hspace{-3em}
\begin{array}{c}
0 =
a_{15} \hspace{-0.2ex} = a_{16} \hspace{-0.2ex} = a_{25} \hspace{-0.2ex} = a_{26} \hspace{-0.2ex} =
\\[-0.1em]
\hspace{8em} = a_{35} \hspace{-0.2ex} = a_{36} \hspace{-0.2ex} = a_{45} \hspace{-0.2ex} = a_{46}
\end{array}
\vspace{.1em}\end{equation}

\noindent
--- число независимых констант падает до~13.

Пусть далее плоскостей симметрии две: ${z = \constant}$ и ${y = \constant}$.
Поскольку~$\potential$ в~таком случае не~чувствительна к~знакам~$\varepsilon_{yx}$ и~$\varepsilon_{yz}$, вдобавок к~\eqref{zeroconstants:oneplaneofsymmetry} имеем

\nopagebreak\vspace{-0.22em}\begin{equation}\label{zeroconstants:2orthogonalplanesofsymmetry:orthotropic}
a_{14} \hspace{-0.2ex} = a_{24} \hspace{-0.2ex} = a_{34} \hspace{-0.2ex} = a_{56} \hspace{-0.2ex} = 0
\end{equation}

\vspace{-0.25em}\noindent
--- осталось 9~констант.

Ортотропным~(ортогонально анизотропным) называется материал с~тремя ортогональными плоскостями симметрии\:--- пусть это координатные плоскости~$x$, $y$, $z$.
Легко увидеть, что \eqref{zeroconstants:oneplaneofsymmetry} и~\eqref{zeroconstants:2orthogonalplanesofsymmetry:orthotropic}\:--- это весь набор нулевых констант и~в~этом случае.
Итак, ортотропный материал характеризуется девятью константами, и~\inquotes{для ортотропности} достаточно двух перпендикулярных плоскостей симметрии.
Вид упругой энергии упрощается до

\nopagebreak\vspace{-0.25em}\begin{multline*}
\potential =^{\mathstrut^{\mathstrut}} \smalldisplaystyleonehalf a_1 \varepsilon^2_{x} \hspace{-0.1ex} + \smalldisplaystyleonehalf a_2 \varepsilon^2_{y} \hspace{-0.1ex} + \smalldisplaystyleonehalf a_3 \varepsilon^2_{z} \hspace{-0.1ex} + \smalldisplaystyleonehalf a_4 \varepsilon^2_{xy} \hspace{-0.2ex} + \smalldisplaystyleonehalf a_5 \varepsilon^2_{xz} \hspace{-0.2ex} + \smalldisplaystyleonehalf a_6 \varepsilon^2_{yz} \hspace{-0.2ex} +{}
\\[-0.1em]
%
{}+ a_{12} \varepsilon_{x} \varepsilon_{y} \hspace{-0.2ex} + a_{13} \varepsilon_{x} \varepsilon_{z} \hspace{-0.2ex} + a_{23} \varepsilon_{y} \varepsilon_{z}
\hspace{.1ex} .
\end{multline*}

В~ортотропном материале сдвиговые~(угловые) деформации $\varepsilon_{xy}$, $\varepsilon_{xz}$, $\varepsilon_{yz}$ никак не~влияют на нормальные напряжения ${\sigma_x \hspace{-0.25ex} = \raisemath{.16em}{\scalebox{0.88}{$\partial \hspace{.1ex} \potential$}} \hspace{-0.1ex} / \hspace{-0.2ex} \raisemath{-0.32em}{\scalebox{0.88}{$\partial \varepsilon_x$}}}$, ${\sigma_y \hspace{-0.25ex} = \raisemath{.16em}{\scalebox{0.88}{$\partial \hspace{.1ex} \potential$}} \hspace{-0.1ex} / \hspace{-0.2ex} \raisemath{-0.32em}{\scalebox{0.88}{$\partial \varepsilon_y$}}}$, ${\sigma_z \hspace{-0.25ex} = \raisemath{.16em}{\scalebox{0.88}{$\partial \hspace{.1ex} \potential$}} \hspace{-0.1ex} / \hspace{-0.2ex} \raisemath{-0.32em}{\scalebox{0.88}{$\partial \varepsilon_z$}}}$ (и~наоборот).
Популярный ортотропный материал\:--- древесина; её упругие свойства различны по~трём взаимно перпендикулярным направлениям: по~радиусу, вдоль~окружности и~по~высоте ствола.

Ещё~один случай анизотропии\:--- трансверсально изотропный (transversely isotropic) материал.
Он характеризуется

...

æolotropic (anisotropic)

...

\begin{equation}\label{potentialenergydensityfromstressandstrain}
\begin{gathered}[b]
2 \hspace{.1ex} \potential \hspace{.1ex}
= \infinitesimaldeformation \hspace{-0.1ex} \dotdotp \hspace{-0.1ex} \stiffnesstensor \dotdotp \hspace{-0.1ex} \infinitesimaldeformation
\hspace{.1ex} ,
\hspace{.4em}
\linearstress \hspace{.1ex} = \scalebox{0.9}{$\displaystyle\frac{\raisemath{-0.2em}{\partial\hspace{.1ex} \potential}}{\raisemath{.04em}{\partial \infinitesimaldeformation}}$} = \stiffnesstensor \dotdotp \hspace{-0.1ex} \infinitesimaldeformation = \infinitesimaldeformation \dotdotp \stiffnesstensor
\hspace{.4em}\Rightarrow
\\[-0.1em]
%
2 \hspace{.1ex} \potential \hspace{.1ex} = \linearstress \dotdotp \infinitesimaldeformation
\end{gathered}
\end{equation}

...

\noindent
преобразование Лежандра\\
Legendre (involution) transform(ation)\\
плотность потенциальной энергии внутренних сил (напряжений)\\
\en{complementary energy}\ru{дополнительная энергия}

\nopagebreak\vspace{-0.1em}\begin{equation}\label{legendretransformforlinearelasticenergy}
\begin{gathered}
\infinitesimaldeformation = \scalebox{0.9}{$\displaystyle\frac{\raisemath{-0.2em}{\partial\hspace{.1ex} \widehat{\potential}}}{\raisemath{.04em}{\partial \linearstress}}$} = \hspace{-0.12ex} \pliabilitytensor \dotdotp \linearstress \hspace{.1ex} = \linearstress \dotdotp \hspace{-0.1ex} \pliabilitytensor
\hspace{.1ex} ,
\\
%
\widehat{\potential}(\hspace{-0.1ex}\linearstress\hspace{.12ex}) \hspace{-0.1ex}
= \linearstress \dotdotp \infinitesimaldeformation
- \hspace{.1ex} \potential(\infinitesimaldeformation)
\end{gathered}
\end{equation}

...

В~линейной теории дополнительная энергия численно равна энергии деформации (\inquotes{упругому потенциалу}).

\end{otherlanguage}

\en{\section{Theorems of statics}}

\ru{\section{Теоремы статики}}

\label{para:theoremsofstatics}

\subsection*{\ru{Теорема }Clapeyron’\ru{а}\en{s theorem}}

% Benoît Paul Émile Clapeyron
% Mémoire sur le travail des forces élastiques dans un corps solide élastique déformé par l’action de forces extérieures
% Comptes rendus hebdomadaires des séances de l’Académie des sciences, Tome XLVI, Janvier–Juin 1858

\en{In~equilibrium}\ru{В~равновесии}
\en{with external forces}\ru{с~внешними силами},
\ru{объёмными}\en{volume ones}~$\bm{f}$
\ru{и~поверхностными}\en{and surface ones}~$\bm{p}$,
\en{the work of~these forces}\ru{работа этих сил}~(\inquotesx{\en{statically frozen}\ru{статически замороженных}}[---] \en{constant along time}\ru{постоянных во~времени})
\en{through actual displacements}\ru{на~актуальных перемещениях}
\ru{равна}\en{is equal to}
\en{the doubled}\ru{удвоенной}\footnote{%
%%\emph{(i)}~\bibauthor{Gabriel Lam\'{e}} et \bibauthor{Benoît Paul \'{E}mile Clapeyron}.
%%\href{https://gallica.bnf.fr/ark:/12148/bpt6k33093/f475.image}{M\'{e}moire sur l'\'{e}quilibre int\'{e}rieur des~corps solides homog\'{e}nes. \emph{Memoires present\'{e}s par Divers Savants}, IV, 1833. Pagine~465\hbox{--}562.}
%%\emph{(ii)}~

\inquotesx{Ce produit repr\'{e}sentait d’ailleurs le~double de la~force~vive que le~ressort pouvait absorber par l’effet de sa flexion et qui \'{e}tait la~mesure naturelle de~sa~puissance.}[---]\\
\bibauthor{Benoît Paul \'{E}mile Clapeyron}.
\href{https://gallica.bnf.fr/ark:/12148/bpt6k3003h/f208.image}{M\'{e}moire sur le travail des forces \'{e}lastiques dans un corps solide \'{e}lastique d\'{e}form\'{e} par l’action de~forces ext\'{e}rieures. \emph{Comptes rendus}, Tome\;XLVI, Janvier--Juin 1858. Pagine~208\hbox{--}212.}
}\hspace{-0.4ex}
\en{energy of~deformation}\ru{энергии деформации}

\nopagebreak\vspace{-0.1em}\begin{equation}\label{clapeyron:elasticitytheorem}
2 \hspace{-0.2em}
\integral\displaylimits_{\mathcal{V}} \hspace{-0.5ex} \potential \hspace{.2ex} d\mathcal{V} \hspace{.12ex}
= \hspace{-0.2ex}
\integral\displaylimits_{\mathcal{V}} \hspace{-0.4ex} \bm{f} \hspace{-0.1ex} \dotp \bm{u} \hspace{.25ex} d\mathcal{V}
+ \hspace{-0.3ex}
\integral\displaylimits_{o_2} \hspace{-0.4ex} \bm{p} \dotp \bm{u} \hspace{.25ex} do
\hspace{.2ex} .
\end{equation}

\vspace{-0.55em}\begin{multline*}
\tikz[baseline=-1ex] \draw [line width=.5pt, color=black, fill=white] (0, 0) circle (.8ex);
\hspace{2.25ex}
2 \hspace{.1ex} \potential \hspace{.1ex} = \linearstress \dotdotp \infinitesimaldeformation =
\linearstress \hspace{.1ex} \dotdotp \hspace{-0.12ex} \boldnabla \bm{u}^{\mathsf{S}} \hspace{-0.12ex} =
\boldnabla \hspace{-0.1ex} \dotp \left( \hspace{.1ex} \linearstress \dotp \bm{u} \hspace{.1ex} \right) -
\tikzmark{beginMinusLoad} \boldnabla \dotp \linearstress \tikzmark{endMinusLoad} \dotp \bm{u} \;\Rightarrow
\\[.5em]
%
\Rightarrow\;\,
\displaystyle 2 \hspace{-0.16em}
\integral\displaylimits_{\mathcal{V}} \hspace{-0.5ex} \potential \hspace{.2ex} d\mathcal{V} \hspace{.12ex} =
\integral\displaylimits_{o_2} \hspace{-0.4ex} \tikzmark{beginSurfaceLoad} \unitnormalvector \dotp \linearstress \tikzmark{endSurfaceLoad} \hspace{.1ex} \dotp \bm{u} \hspace{.25ex} do \hspace{.4ex} +
\integral\displaylimits_{\mathcal{V}} \hspace{-0.4ex} \bm{f} \hspace{-0.1ex} \dotp \bm{u} \hspace{.25ex} d\mathcal{V}
\hspace{2.25ex}
\tikz[baseline=-0.6ex] \draw [color=black, fill=black] (0, 0) circle (.8ex);
\end{multline*}%
\AddUnderBrace[line width=.75pt][.2ex, 0][yshift = .1em]%
{beginMinusLoad}{endMinusLoad}{${\scalebox{0.75}{$ - \bm{f} $}}$}%
\AddUnderBrace[line width=.75pt][-0.1ex, 0][yshift = .1em]%
{beginSurfaceLoad}{endSurfaceLoad}{${\scalebox{0.75}{$ \bm{p} $}}$}

\begin{otherlanguage}{russian}

\vspace{-0.2em}
Из~\eqref{clapeyron:elasticitytheorem} следует также, что без нагрузки
${\hspace{-0.25ex}\scalebox{1.4}{$\integral$}_{\hspace{-0.5ex}\raisemath{.1em}{\mathcal{V}}} \hspace{.25ex} \potential \hspace{.2ex} d\mathcal{V} = 0}$.
По\-сколь\-ку $\potential$ положительна, то и напряжение~$\linearstress$, и~деформация~$\infinitesimaldeformation$ без нагрузки\:--- нулевые.

\begin{align*}
2 \hspace{.1ex} \potential \hspace{.1ex} &= \linearstress \dotdotp \infinitesimaldeformation
\\
\mathdotabove{\potential} \hspace{.1ex} &= \linearstress \dotdotp \mathdotabove{\infinitesimaldeformation}
\\
\variation{\potential} &= \linearstress \dotdotp \variation{\infinitesimaldeformation}
\end{align*}

\textbold{\en{Paradox of elastostatics}\ru{Парадокс эластостатики}}: ${\potential}$ \en{is equal to}\ru{равна} \en{only}\ru{лишь} \en{the~half}\ru{половине} \en{of~the~work}\ru{работы} \en{of~external forces}\ru{внешних сил}.

\en{The~accumulated potential energy of~deformation}\ru{Накопленная потенциальная энергия деформации}~$\potential$ is equal to only the~half of~the~work done by external forces, acting through displacements from the~unstressed configuration to the~equilibrium.

\ru{Теорема }Clapeyron’\en{s}\ru{а}\en{ theorem} \en{implies that}\ru{подразумевает, что} \en{the~accumulated elastic energy}\ru{накопленная упругая энергия} \en{accounts for}\ru{составляет} \en{only the~half of~energy spent on deformation}\ru{лишь половину потраченной на деформацию энергии}.
\en{The~remaining half of the~work}\ru{Оставшаяся половина работы,} \en{done by external forces}\ru{совершённой внешними силами,} \en{is lost somewhere}\ru{теряется где\hbox{-}то} \en{before reaching the~equilibrium}\ru{до достижения равновесия}.

{\small
\en{This apparent paradox is reached within the~framework of purely conservative linear elasticity.
Alternatively, however, within elastostatics the~common characterization of the~work done to reach equilibrium is conceptually ambiguous, and a~novel interpretation may be needed.}

\ru{Этот кажущийся парадокс достигается в~рамках чисто консервативной линейной упругости.
Как альтернатива, однако, в~эластостатике обычная характеризация работы, совершённой для достижения равновесия, концептуально сомнительна, и может быть нужна новая интерпретация.}

\bibauthor{Roger Fosdick} and \bibauthor{Lev Truskinovsky}.
\href{http://www.cityu.edu.hk/ma/ws2007/notes/FTJElast.pdf}{About Clapeyron’s Theorem in Linear Elasticity. \emph{Journal of~Elasticity}, Volume~72, July 2003. Pages 145\hbox{--}172.}
\par}

There is always heating due to energy dissipation.

Для решения парадокса в~теории распространена концепция \en{of~}\inquotesx{\en{static loading}\ru{статического нагружения}}[---] \en{infinitely slow}\ru{бесконечно медленного} \en{gradual application}\ru{постепенного приложения} \en{of~the~load}\ru{нагрузки}.

{\small
Статика рассматривает \inquotes{замороженное} равновесие вне времени.
Динамика нагружения до~равновесия\:--- предыстория.
\en{In the~linear small displacement theory}\ru{В~линейной теории малых перемещений} в~равновесии затраченная на~деформацию работа внешних сил на актуальных перемещениях равна удвоенной потенциальной энергии деформации.
\inquotes{Запасается} всего половина потраченной энергии.
Вторая половина есть дополнительная энергия, она теряется до~обретения равновесия на~динамику\:--- на~внутреннюю энергию частиц (\inquotes{диссипацию}), на~колебания и~волны.
Так в~теории.
Однако, в~реальности не~бывает ни~моментального \inquotes{мёртвого} нагружения, ни~бесконечно медленного \inquotesx{следящего}[.]
Это две крайности.
Реальная динамика нагружения всегда где\hbox{-}то между ними.

В~области~же бесконечномалых вариаций и~виртуальных работ, работа реальных внешних сил на виртуальных перемещениях точно равна вариации упругого потенциала.
А~упругая среда есть такая, в~которой вариация работы сил внутренних (напряжений) на виртуальных деформациях это минус вариация потенциала.

${- \hspace{.1ex} \variation{\internalwork} \hspace{-0.2ex} = \variation{\potential} = \variation{\externalwork} \hspace{-0.25ex}}$, когда варьируются только перемещения (нагрузки не~варьируются).
Потому в~принципе виртуальной работы и~варьируются лишь перемещения, чтобы виртуальная работа внешних неварьируемых реальных сил на~вариациях перемещений была равна минус вариации внутренней энергии (в~случае упругой среды\:--- вариации упругого потенциала).
\par}

\subsection*{\en{Uniqueness of solution theorem}\ru{Теорема единственности решения}}

Как~и в~динамике~(\pararef{para:uniquenessfordynamicproblem}), допускаем существование двух решений и~ищем их~разность

...

\end{otherlanguage}

\hspace*{-\parindent}\begin{minipage}{\linewidth}

% ~ ~ ~ ~ ~
%%\begin{center}

\begin{wrapfigure}{o}{.4\textwidth}
\makebox[.45\textwidth][c]{\begin{minipage}[t]{.5\textwidth}
\vspace{-1em}
\scalebox{0.88}{

\def\cameraangle{166}
\tdplotsetmaincoords{44}{\cameraangle} % orientation of camera

\def\rodheight{10}
\def\rodradius{.2}

\pgfmathsetmacro{\beginangle}{\cameraangle}
\pgfmathsetmacro{\endangle}{\cameraangle - 180}

\tikzset{pics/rod/.style={code={

	% draw rod

	\draw [line width=1pt, color=black, fill=yellow!50!white, opacity=.9]
		plot [domain=\beginangle:\endangle]
			( {\rodradius*cos(\x)}, {\rodradius*sin(\x)}, 0 )
		-- plot [domain=\endangle:\beginangle]
			( {\rodradius*cos(\x)}, {\rodradius*sin(\x)}, \rodheight )
		-- cycle ;

	\draw [line width=1pt, color=black, fill=yellow!50!white, opacity=.9, domain=0:360]
		plot ( {-\rodradius*cos(\x)}, {-\rodradius*sin(\x)}, \rodheight ) ;

}}}

\tikzset{pics/dottedrod/.style={code={

	% draw rod dotted

	\draw [line width=1pt, color=black, line cap=round, dash pattern=on 0pt off 1.6\pgflinewidth]
		plot [domain=\beginangle:\endangle]
			( {\rodradius*cos(\x)}, {\rodradius*sin(\x)}, 0 )
		-- plot [domain=\endangle:\beginangle]
			( {\rodradius*cos(\x)}, {\rodradius*sin(\x)}, \rodheight )
		-- cycle ;

	\draw [line width=1pt, color=black, line cap=round, dash pattern=on 0pt off 1.6\pgflinewidth, opacity=.9, domain=0:360]
		plot ( {-\rodradius*cos(\x)}, {-\rodradius*sin(\x)}, \rodheight ) ;

}}}

\tikzset{pics/rodaxis/.style={code={

	% draw axis
	\draw [line width=0.5pt, blue, line cap=round, dash pattern=on 12pt off 2pt on \the\pgflinewidth off 2pt]
		( 0, 0, -0.4pt ) -- ( 0, 0, \rodheight + 0.4pt ) ;

}}}

\tikzset{pics/externalforce/.style={code={

	% draw force
	\def\forcelength{1.2}

	\draw [line width=1.5pt, red, line cap=round, -{Triangle[round, length=3.6mm, width=2.4mm]}]
		( 0, 0, \rodheight + \forcelength ) -- ( 0, 0, \rodheight )
		node [ pos=0.4, left, inner sep=0, outer sep=4.4pt ]
			{\scalebox{1.2}[1.2]{${\bm{F}}$}} ;

}}}

\begin{tikzpicture}[scale=1, tdplot_main_coords] % use 3dplot

	\coordinate (O) at ( 0, 0, 0 ) ;
	\coordinate (rodTopCenter) at ($ (O) + ( 0, 0, \rodheight ) $) ;

	% draw circle
	\def\circleradius{0.8}
	\def\heightofhatch{0.5}

	\pgfmathsetmacro{\stepangleforcircle}{\beginangle - 10}
	\foreach \angle in { \beginangle, \stepangleforcircle, ..., \endangle }
		\draw [line width=0.4pt, color=black]
			( \angle:\circleradius ) -- ($ ( \angle:\circleradius ) - ( 0, 0, \heightofhatch ) $) ;

	\draw [line width=1pt, color=black, fill=white] (O) circle ( \circleradius ) ;

	% draw rod, axis and force
	\pic (initial) {rod} ;
	\pic (initial) {rodaxis} ;
	\pic (initial) {externalforce} ;

	% draw deformed rod
	\scoped {
		\pgfsetcurvilinearbeziercurve
			{\pgfpointxyz{0}{0}{0}}
			{\pgfpointxyz{0}{0}{0.5cm}}
			{\pgfpointxyz{0.25cm}{0}{1cm}}
			{\pgfpointxyz{1.25cm}{0}{1.25cm}}
		\pgftransformnonlinear{\pgfgetlastxy\x\y\pgfpointcurvilinearbezierorthogonal{\y}{\x}}
			\pic (deformed) {dottedrod} ;
			\pic (deformed) {rodaxis} ;
	}

\end{tikzpicture}
}
\vspace{-0.2em}\caption{}\label{fig:deformedrod}
\end{minipage}}
\end{wrapfigure}

%%\end{center}

% ~ ~ ~ ~ ~

\hspace{\horizontalindent}%
\en{The~uniqueness of~solution}\ru{Единственность решения}, \en{dis\-covered}\ru{открытая} \en{by~}G.\:Kirchhoff\ru{’ом} \en{for bodies}\ru{для тел} \en{with }\ru{с~}\en{simply connected}\ru{одно\-св\'{я}зным} \en{contour}\ru{контуром}\stepcounter{footnote}\setcounter{auxfootnotecounter}{\value{footnote}}\footnotemark[\value{auxfootnotecounter}]\hbox{\hspace{-0.4ex},} \en{is contrary to}\ru{противоречит}, \en{as it seems}\ru{\hbox{казалось~бы}}, \en{the everyday experience}\ru{повседневному опыту}.
\en{Imagine}\ru{Вообразим} \en{a~straight rod}\ru{прямой стержень}, \en{clamped}\ru{зажатый} \en{at one end}\ru{на одном конц\'{е}}~(\inquotes{\en{cantilever}\ru{конс\'{о}льный}}) \en{and compressed}\ru{и~сжимаемый} \en{at second end}\ru{на~втором конц\'{е}} \en{with a~longitudinal force}\ru{продольной силой}~(\figref{fig:deformedrod}).
\en{When a~load is large enough}\ru{Когда нагрузка достаточно больш\'{а}я}, \en{the~problem of statics}\ru{задача статики} \en{has two solutions}\ru{имеет два решения}\:--- \inquotes{\en{straight}\ru{прямое}} \en{and}\ru{и}~\inquotesx{\en{bent}\ru{изогнутое}}[.]
\en{But}\ru{Но} \en{such a~contradiction}\ru{такое противоречие} \en{with the~uniqueness theorem}\ru{с~теоремой единственности} \en{comes from}\ru{происходит от} \en{nonlinearity of this problem}\ru{нелинейности этой задачи}.
\en{If a~load is small}\ru{Если нагрузка мал\'{а}}, \en{then}\ru{то} \en{the~solution}\ru{решение} \en{is described by linear equations}\ru{описывается линейными уравнениями} \en{and}\ru{и} \en{is unique}\ru{единственно}.

...

...

...

...

\end{minipage}

\nicefootnotetext{auxfootnotecounter}{\bibauthor{Gustav Robert Kirchhoff}. \href{https://opacplus.bsb-muenchen.de/Vta2/bsb10525510/bsb:2960444?page=291}{Über das~Gleich\-gewicht und die~Bewe\-gung eines unendlich dünnen elastischen Stabes. \emph{Journal für die reine und angewandte Mathematik (Crelle’s journal)}, 56.\:Band (1859). Seiten 285\hbox{--}313.} (Seite 291)}

\subsection*{\en{Reciprocal work theorem}\ru{Теорема взаимности работ}}

Proposed by Enrico Betti\footnote{\bibauthor{Enrico Betti}. \href{https://play.google.com/books/reader?id=RHDzLUBIlUUC&pg=GBS.PA69}{Teoria della elasticit\`{a}. \emph{Il Nuovo Cimento}~(1869--1876), VII\:e\:VIII\;(1872). Pagina 69.}}{\hspace{-0.5ex}.}

%%Il Nuovo Cimento (1869--1876)
%%Betti, E. Teoria della elasticità. Nuovo Cimento 7\hbox{--}8 (1872)
%%Betti, Nuovo Cimento VII--VIII (1872): 5--21, 69--97, 158--180

\begin{otherlanguage}{russian}

Для т\'{е}ла с~фиксированием части поверхности~${o_1}$ рассматриваются два варианта: первый с~нагрузками $\bm{f}_1$, $\bm{p}_1$ и~второй с~нагрузками $\bm{f}_2$, $\bm{p}_2$.
Словесная формулировка теоремы та~же, что и~в~\chapdotpararef{chapter:genericmechanics}{para:statics}.
Математическая запись

\nopagebreak\vspace{1.3em}\begin{equation}\label{betti:reciprocalworktheorem}
\tikzmark{beginReciprocalFirstVariant} \scalebox{0.98}{$\displaystyle \integral\displaylimits_{\mathcal{V}} \hspace{-0.4ex} \bm{f}_1 \hspace{-0.2ex} \dotp \bm{u}_2 \hspace{.25ex} d\mathcal{V} $}
+ \hspace{-0.3ex}
\scalebox{0.96}{$\displaystyle\integral\displaylimits_{o_2} \hspace{-0.4ex} \bm{p}_1 \hspace{-0.2ex} \dotp \bm{u}_2 \hspace{.25ex} do $} \tikzmark{endReciprocalFirstVariant}
%
\hspace{.2ex} = \hspace{-0.2ex}
%
\tikzmark{beginReciprocalSecondVariant} \scalebox{0.98}{$\displaystyle \integral\displaylimits_{\mathcal{V}} \hspace{-0.4ex} \bm{f}_2 \hspace{-0.2ex} \dotp \bm{u}_1 \hspace{.25ex} d\mathcal{V} $}
+ \hspace{-0.3ex}
 \scalebox{0.96}{$\displaystyle\integral\displaylimits_{o_2} \hspace{-0.4ex} \bm{p}_2 \hspace{-0.2ex} \dotp \bm{u}_1 \hspace{.25ex} do $} \tikzmark{endReciprocalSecondVariant}
\hspace{.1ex} .
\end{equation}%
\AddOverBrace[line width=.75pt][.1ex, .66em][yshift=-0.11ex]%
{beginReciprocalFirstVariant}{endReciprocalFirstVariant}{\scalebox{0.88}{$ W_{\hspace{-0.1ex}12} $}}%
\AddOverBrace[line width=.75pt][.1ex, .66em][yshift=-0.11ex]%
{beginReciprocalSecondVariant}{endReciprocalSecondVariant}{\scalebox{0.88}{$ W_{\hspace{-0.15ex}21} $}}

...

{\small
Reciprocal work theorem, also known as Betti’s theorem, claims that for a~linear elastic structure subject to two sets of forces $P$ and $Q$, the~work done by set~$P$ through displacements produced by set~$Q$ is equal to the~work done by set~$Q$ through displacements produced by set~$P$. This theorem has applications in structural engineering where it is used to define influence lines and derive the boundary element method.
\par}

...

\hspace*{-\parindent}\begin{minipage}{\linewidth}

\begin{wrapfigure}[12]{o}{.42\textwidth}
\makebox[.42\textwidth][c]{\begin{minipage}[t]{.42\textwidth}
\vspace{-0.5em}

% parameter #1 is the height of beam
\newcommand\drawrodbeam[1]%
{
	\draw [line width=1.2pt, black] (-0.5, 0) -- (0.5, 0) ;
	\foreach \xground in {-0.36, -0.16, ..., 0.5}
		\draw [line width=0.4pt, black!80] (\xground, 0) -- (\xground - 0.2, -0.2) ;

	\draw [line width=2pt, line cap=round, black] (0, 0) -- (0, #1) ;
}

\tikzstyle{force line} =
	[line width=1.25pt, red, line cap=round, -{Triangle[round, length=3.2mm, width=2mm]}]

\tikzstyle{unitforce line} =
	[line width=1.25pt, blue, line cap=round, dash pattern=on 0pt off 1.6\pgflinewidth, -{Triangle[round, length=3.2mm, width=2mm]}]

\begin{tikzpicture}[scale=0.8]

	\def\beamheight{5}
	\def\beamxshift{1.9}
	\def\firstforcelength{0.9}
	\def\secondforcelength{1.1}
	\def\firstunitforcelength{0.69}
	\pgfmathsetmacro\secondunitforcelength{\firstunitforcelength * 3 / 4}

	\pgfmathsetmacro{\firstforceposition}{\beamheight}
	\pgfmathsetmacro{\secondforceposition}{0.6 * \beamheight}

	\drawrodbeam{\beamheight}

	\foreach \loadpos in { 0, 0.3, ..., 0.6 }
		\draw [force line]
			( -\firstforcelength, \firstforceposition - \loadpos ) -- ( 0, \firstforceposition - \loadpos ) ;

	\node at ( -\firstforcelength, \firstforceposition - 0.3 )
		[left, inner sep=0, outer sep=3.3pt, red]
			{${P_{\hspace{-0.25ex}\raisemath{-0.4ex}{1}}}$} ;

	\foreach \loadpos in { 0, 0.3, ..., 1 }
		\draw [force line]
			( -\secondforcelength, \secondforceposition - \loadpos ) -- ( 0, \secondforceposition - \loadpos ) ;

	\node at ( -\secondforcelength, \secondforceposition - 0.5 )
		[left, inner sep=0, outer sep=3.3pt, red]
			{${P_{\hspace{-0.25ex}\raisemath{-0.4ex}{2}}}$} ;

	\pgfmathsetmacro\secondxshift{\beamxshift}

	\begin{scope}[xshift=\secondxshift cm, yshift=0cm]
		\drawrodbeam{\beamheight}
	\end{scope}

	\foreach \loadpos in { 0, 0.3, ..., 0.6 }
		\draw [unitforce line]
			( \secondxshift - \firstunitforcelength, \firstforceposition - \loadpos )
			-- ( \secondxshift, \firstforceposition - \loadpos ) ;

	\node at ( \secondxshift - \firstunitforcelength, \firstforceposition - 0.3 )
		[left, inner sep=0, outer sep=3.2pt, blue] {$ 1 $} ;

	\pgfmathsetmacro\thirdxshift{\beamxshift + \beamxshift}

	\begin{scope}[xshift=\thirdxshift cm, yshift=0cm]
		\drawrodbeam{\beamheight}
	\end{scope}

	\foreach \loadpos in { 0, 0.3, ..., 1 }
		\draw [unitforce line]
			( \thirdxshift - \secondunitforcelength, \secondforceposition - \loadpos )
			-- ( \thirdxshift, \secondforceposition - \loadpos ) ;

	\node at ( \thirdxshift - \secondunitforcelength, \secondforceposition - 0.5 )
		[left, inner sep=0, outer sep=3.2pt, blue] {$ 1 $} ;

\end{tikzpicture}
\vspace{-1.5em}\caption{}\label{fig:exampleofapplicationofreciprocalworktheorem}
\end{minipage}}
\end{wrapfigure}

\en{The reciprocal work theorem}\ru{Теорема взаимности работ} находит неожиданные и~эффективные применения.
\en{For example}\ru{Для примера}\en{,} \en{consider}\ru{рассмотрим} \en{a~rod-beam}\ru{стержень-балку}\ru{,} \en{clamped}\ru{защёмленную} \en{at one end}\ru{на одн\'{о}м конц\'{е}}~(\inquotes{\en{cantilever}\ru{консольную}}) \en{and}\ru{и} \en{bent}\ru{изгибаемую} \en{by two forces}\ru{двумя силами} \en{with}\ru{с}~\en{integral values}\ru{интегральными значениями}~$P_{\hspace{-0.25ex}\raisemath{-0.4ex}{1}}$ \en{and}\ru{и}~$P_{\hspace{-0.25ex}\raisemath{-0.4ex}{2}}$~(\figref{fig:exampleofapplicationofreciprocalworktheorem}).
\en{While the~linear theory is applied}\ru{Тогда как применяется линейная теория}, перемещения-пр\'{о}гибы могут быть представлены в~виде

\nopagebreak\vspace{-0.1em}\begin{equation*}\begin{array}{c}
u_1 \hspace{-0.22ex} = \alpha_{1\hspace{-0.1ex}1} P_{\hspace{-0.25ex}\raisemath{-0.4ex}{1}} \hspace{-0.2ex} + \alpha_{12} P_{\hspace{-0.25ex}\raisemath{-0.4ex}{2}}
\hspace{.2ex} ,
\\[.1em]
u_2 \hspace{-0.22ex} = \alpha_{21} P_{\hspace{-0.25ex}\raisemath{-0.4ex}{1}} \hspace{-0.2ex} + \alpha_{22} P_{\hspace{-0.25ex}\raisemath{-0.4ex}{2}}
\hspace{.2ex} .
\end{array}\end{equation*}

...

\end{minipage}


\end{otherlanguage}

\en{\section{Equations for displacements}}

\ru{\section{Уравнения в перемещениях}}

\label{para:equationsfordisplacements.linearelasticity}

\begin{otherlanguage}{russian}

\en{The~complete set of equations}\ru{Полный набор уравнений}~\eqref{lineartheory:wholesetofequations} \en{contains unknowns}\ru{содержит неизвестные} $\linearstress$, $\infinitesimaldeformation$ \en{and}\ru{и}~$\bm{u}$.
\en{Excluding}\ru{Исключая} $\linearstress$ \en{and}\ru{и}~$\infinitesimaldeformation$, приходим к~постановке в~перемещениях
(симметризация~${\hspace{-0.2ex} \boldnabla \bm{u}}$ тут лишняя, ведь ${\stiffnesstensor_{\hspace{.12ex} 3 \scalebox{0.6}[0.8]{$\rightleftarrows$} 4} \hspace{-0.25ex} = \stiffnesstensor}$)

\nopagebreak\vspace{-0.1em}\begin{equation}\label{lineartheory:equationsfordisplacements}
\begin{array}{c}
\boldnabla \dotp \left( \stiffnesstensor \dotdotp \hspace{-0.12ex} \boldnabla \bm{u} \hspace{.1ex} \right) + \hspace{.1ex} \bm{f} = \hspace{.1ex} \bm{0}
\hspace{.16ex} , \\[.4em]
%
\bm{u} \hspace{.1ex} \bigr|_{o_1} \hspace{-0.64ex} = \hspace{.2ex} \bm{u}_{\raisemath{-0.1em}{0}}
\hspace{.16ex} , \:\:
\unitnormalvector \dotp %%\tikzmark{TauTensorBegin}
\stiffnesstensor \hspace{-0.08ex} \dotdotp \hspace{-0.24ex} \boldnabla \bm{u}
%%\tikzmark{TauTensorEnd}
\hspace{.25ex} \bigr|_{o_2} \hspace{-0.64ex} = \hspace{.2ex} \bm{p}
\hspace{.16ex} .
\end{array}
\end{equation}%
%%\AddOverBrace[line width=.75pt][-0.1ex,0.1em]%
%%{TauTensorBegin}{TauTensorEnd}{${\scriptstyle \linearstress}$}

%% \footnote{По\hbox{-}прежнему под~$\bm{f}$ подразумеваем сумму обычной силы и~даламберовой силы инерции~${(- \rho \mathdotdotabove{\bm{u}}\hspace{.25ex})}$.}

В~изотропном теле~\eqref{lineartheory:equationsfordisplacements} принимает вид

...

Общее решение однородного уравнения (...) нашёл \href{https://de.wikipedia.org/wiki/Heinz_Neuber}{Heinz Neuber}

\href{https://ru.wikipedia.org/wiki/%D0%9F%D0%B0%D0%BF%D0%BA%D0%BE%D0%B2%D0%B8%D1%87,_%D0%9F%D1%91%D1%82%D1%80_%D0%A4%D1%91%D0%B4%D0%BE%D1%80%D0%BE%D0%B2%D0%B8%D1%87}{П.\,Ф.\:Папкович}

...



\end{otherlanguage}

\en{\section{Concentrated force in an infinite medium}}

\ru{\section{Сосредоточенная сила в бесконечной среде}}

{\small
Concentrated force is useful mathematical idealization, but cannot be found in the real world, where all forces are either body forces acting over a~volume or surface forces acting over an~area.
\par}

\begin{otherlanguage}{russian}

Начнём с~риторического вопроса: почему упругое тело сопротивляется приложенной нагрузке, выдерживает её? Удачный ответ можно найти ...

...



\end{otherlanguage}

\en{\section{Finding displacements by deformations}}

\ru{\section{Нахождение перемещений по деформациям}}

\label{para:displacementsfromdeformations}

\begin{otherlanguage}{russian}

Разложив градиент перемещения на~симметричную и~антисимметричную части
\nopagebreak\vspace{.1em}\begin{equation}
\boldnabla \bm{u} \hspace{.2ex} = \tikzmark{beginSymmNablaU} \hspace{.32ex} \infinitesimaldeformation \hspace{.4ex} \tikzmark{endSymmNablaU} \hspace{-0.16ex} - \hspace{.25ex} \tikzmark{beginAsymmNablaU} \bm{\omega} \times \hspace{-0.12ex} \UnitDyad \tikzmark{endAsymmNablaU} \hspace{.1ex} , \:\;
\bm{\omega} \equiv \displaystyle \onehalf \hspace{.32ex} \boldnabla \hspace{-0.12ex} \times \hspace{-0.12ex} \bm{u} \hspace{.2ex},
\end{equation}%
\AddOverBrace[line width=.75pt][.1ex,0.1ex]%
{beginSymmNablaU}{endSymmNablaU}{${\scriptstyle \boldnabla {\bm{u}}^{\hspace{.1ex}\mathsf{S}}}$}%
\AddOverBrace[line width=.75pt][.2ex,0.1ex]%
{beginAsymmNablaU}{endAsymmNablaU}{${\scriptstyle - \hspace{.1ex} \boldnabla {\bm{u}}^{\hspace{.1ex}\mathsf{A}}}$}

...

{\small
Saint\hbox{-\hspace{-0.2ex}}Venant’s compatibility condition is the integrability conditions for a~symmetric tensor field to be a~strain.

The compatibility conditions in linear elasticity are obtained by observing that there are six strain\hbox{--}displacement relations that are functions of only three unknown displacements. This suggests that the three displacements may be removed from the system of equations without loss of information.

A body that deforms without developing any gaps/overlaps is called a~compatible body. Compatibility conditions are mathematical conditions that determine whether a~particular deformation will leave a~body in a~compatible state.
\par}

...

\begin{equation*}
\operatorname{inc} \infinitesimaldeformation \hspace{-0.05ex}
\equiv
\hspace{-0.07ex} \boldnabla \hspace{-0.2ex} \times \hspace{-0.36ex} \bigl( \hspace{.06ex} \boldnabla \hspace{-0.2ex} \times \hspace{-0.16ex} \infinitesimaldeformation \hspace{.1ex} \bigr)^{\hspace{-0.2ex}\T}
\end{equation*}

Контур здесь произволен, так что приходим к~соотношению

\nopagebreak\vspace{-0.25em}\begin{equation}\label{incompatibilityequalszero}
\operatorname{inc} \infinitesimaldeformation = \hspace{-0.07ex} {^2\bm{0}}
\hspace{.1ex} ,
\end{equation}

\vspace{-0.33em}\noindent
называемому уравнением совместности деформаций.

{\small
Resulting expression~\eqref{incompatibilityequalszero} in terms of only deformation/strain provide constraints on possible variants of a~deformation/strain field.
\par}

...

Тензор~${\operatorname{inc} \infinitesimaldeformation}$ симметричен вместе с~${\infinitesimaldeformation}$

...

Все уравнения линейной теории имеют аналог~(перво\-источ\-ник) в~нелинейной.
Чтобы найти его для~\eqref{incompatibilityequalszero}, вспомним тензор деформации Cauchy\hbox{--}Green’а~(\chapdotpararef{chapter:nonlinearcontinuum}{para:deformationtensors}) и~тензоры кривизны~(\chapdotpararef{chapter:elementsoftensorcalculus}{para:curvaturetensors})

...



\end{otherlanguage}

\en{\section{Equations for stresses}}

\ru{\section{Уравнения в напряжениях}}

\en{Balance of~forces~(of~momentum)}\ru{Баланс сил~(импульса)}

\nopagebreak\en{\vspace{-0.125em}}\ru{\vspace{-0.6em}}
\begin{equation}\label{stresseseq:balanceofforces}
\boldnabla \dotp \linearstress \hspace{.15ex} + \bm{f} = \hspace{.1ex} \bm{0}
\end{equation}

\nopagebreak \vspace{-0.2em} \noindent
\en{does not quite yet determine}\ru{ещё не~определяет} \en{stresses}\ru{напряжения}.
\en{It’s necessary as~well}\ru{Необходимо вдобавок}, \en{that deformations/strains}\ru{чтобы соответствующие напряжениям деформации}~${\infinitesimaldeformation(\hspace{-0.15ex}\linearstress\hspace{.1ex})}$\en{ corresponding to stresses}~\eqref{legendretransformforlinearelasticenergy}

\nopagebreak\vspace{-0.2em}\begin{equation}\label{stresseseq:strainfromstress}
\infinitesimaldeformation(\hspace{-0.15ex}\linearstress\hspace{.1ex}) = \displaystyle \frac{\raisemath{-0.2em}{\partial\hspace{.1ex} \widehat{\potential}}}{\raisemath{.04em}{\partial \linearstress}} = \hspace{-0.12ex} \pliabilitytensor \dotdotp \hspace{-0.07ex} \linearstress
\end{equation}

\nopagebreak\vspace{-0.5em}\noindent
\en{are compatible}\ru{были совместны}~(\pararef{para:displacementsfromdeformations})

\nopagebreak\vspace{-0.5em}\begin{equation}\label{stresseseq:compatibility}
\operatorname{inc} \infinitesimaldeformation(\hspace{-0.15ex}\linearstress\hspace{.1ex}) \hspace{-0.06ex}
\equiv
\hspace{-0.07ex} \boldnabla \hspace{-0.2ex} \times \hspace{-0.44ex} \Bigl( \boldnabla \hspace{-0.2ex} \times \hspace{-0.16ex} \infinitesimaldeformation(\hspace{-0.15ex}\linearstress\hspace{.1ex}) \Bigr)^{\raisemath{-0.12em}{\hspace{-0.44ex}\T}} \hspace{-0.52ex}
= {^2\bm{0}}
\hspace{.1ex} .
\end{equation}

\vspace{-0.3em}\noindent
\en{Gathered together}\ru{Взятые вместе}, \eqref{stresseseq:balanceofforces}, \eqref{stresseseq:strainfromstress} and~\eqref{stresseseq:compatibility} \en{present}\ru{являют} \en{the~complete closed set}\ru{полный замкнутый набор}~(\en{system}\ru{систему}) \en{of~equations}\ru{уравнений} \en{for stresses}\ru{в~напряжениях}.


...

%%\begin{otherlanguage}{russian}

...

%%\end{otherlanguage}

\en{\section{Principle of minimum potential energy}}

\ru{\section{Принцип минимума потенциальной энергии}}

\label{para:principleofminimumpotentialenergy}

\vspace{.2em}\begin{changemargin}{\parindent}{\parindent}
\small
\en{When}\ru{Когда} \en{the~existence}\ru{существование} \en{of a~deformation energy function}\ru{функции энергии деформации} \en{is assured}\ru{несомненно}, \en{and}\ru{и}~\en{external forces}\ru{внешние силы} \en{are assumed constant}\ru{считаются постоянными} \en{during}\ru{во~вр\'{е}мя} \en{variation of~displacements}\ru{варьирования перемещений}, \en{the~principle of virtual work}\ru{принцип виртуальной работы} \en{leads}\ru{приводит} \en{to}\ru{к}~\en{the~principle of~minimum potential energy}\ru{принципу минимума потенциальной энергии}.
\par
\nopagebreak\vspace{.2em}
\end{changemargin}

\noindent
\en{Formulation}\ru{Формулировка} \en{of~the~principle}\ru{принципа}:

\nopagebreak\vspace{-0.2em}\begin{equation}\label{principleofminimumpotentialenergy.formulation}
\potentialenergyfunctional \hspace{.2ex} (\hspace{-0.1ex}\bm{u}\hspace{-0.1ex}) \equiv \hspace{-0.2ex}
\displaystyle\integral\displaylimits_{\mathcal{V}} \hspace{-0.4ex}
\Bigl(
\potential(\hspace{-0.1ex}\bm{u}\hspace{-0.1ex}) \hspace{-0.1ex} - \bm{f} \hspace{-0.1ex} \dotp \bm{u}
\Bigr) \hspace{-0.1ex} d\mathcal{V} \hspace{.1ex}
- \hspace{-0.3ex}
\displaystyle\integral\displaylimits_{o_2} \hspace{-0.4ex}
\bm{p} \dotp \bm{u} \hspace{.33ex} do \hspace{.2ex}
\hspace{.1ex}\to\hspace{.25ex} \mathrm{min}
\hspace{.15ex} , \hspace{.5em}
\bm{u} \hspace{.1ex} \bigr|_{o_1} \hspace{-0.8ex} = \hspace{.1ex} \bm{u}_{\raisemath{-0.1em}{0}}
\hspace{.16ex} .
\end{equation}

\vspace{-0.2em}\noindent
\en{Functional}\ru{Функционал}~${\potentialenergyfunctional \hspace{.2ex} (\hspace{-0.1ex}\bm{u}\hspace{-0.1ex})}$, \en{called}\ru{называемый} \en{the~(full)}\ru{(полной)} \en{potential energy}\ru{потенциальной энергией} \en{of a~linear-elastic body}\ru{линейно-упругого т\'{е}ла}, \en{is minimal}\ru{минимален} \en{when}\ru{тогда, когда} \en{displacements}\ru{перемещения}~$\bm{u}$ \en{are true}\ru{истинны}\:--- \en{that is}\ru{то~есть} \en{for}\ru{для} \en{the~solution}\ru{решения} \en{of~problem}\ru{задачи}~\eqref{lineartheory:equationsfordisplacements}.
\en{Input functions}\ru{Аргументы\hbox{-}функции}~$\bm{u}$ \en{must satisfy}\ru{должны удовлетворять} \en{the~geometrical condition}\ru{геометрическому условию} \en{on}\ru{на}~${o_1\hspace{-0.2ex}}$~(\en{so they don’t break existing constraints}\ru{так они не~рвут существующие связи}) \en{and}\ru{и}~\en{be}\ru{быть} \en{continuous}\ru{непрерывными} (\en{or else}\ru{иначе} ${\smash{\potential(\hspace{-0.1ex}\bm{u}\hspace{-0.1ex})}\hspace{-0.1ex}}$ \en{will not be}\ru{не~будет} \en{integrable}\ru{интегрируемой}).

\en{For the~true field of~displacements}\ru{Для истинного поля перемещений}~$\bm{u}$, \en{quadratic function}\ru{квадратичная функция}

\nopagebreak\vspace{-0.15em}\begin{equation*}
\potential(\hspace{-0.1ex}\bm{u}\hspace{-0.1ex}) \hspace{-0.22ex} = \hspace{.1ex} \smalldisplaystyleonehalf \boldnabla \bm{u} \hspace{-0.1ex} \dotdotp \hspace{-0.1ex} \stiffnesstensor \dotdotp \hspace{-0.25ex} \boldnabla \bm{u}
\end{equation*}

\nopagebreak\vspace{-0.1em}\noindent
\en{becomes equal to}\ru{становится равной} \en{the~true deformation energy}\ru{истинной энергии деформации}.
\en{Then}\ru{Тогда} $\potentialenergyfunctional \hspace{-0.1ex} = \hspace{-0.1ex} \potentialenergyfunctional_{\hspace{-0.1ex}\text{min}}$, \en{which}\ru{которая} \en{according to}\ru{согласно} \en{the }\ru{теореме }Clapeyron’\en{s}\ru{а}\en{ theorem}~\eqref{clapeyron:elasticitytheorem} \en{is}\ru{есть}

\nopagebreak\vspace{-0.33em}\begin{equation*}
\potentialenergyfunctional_{\hspace{-0.1ex}\text{min}}
= \hspace{-0.2ex}
\scalebox{0.9}{$\displaystyle\integral\displaylimits_{\mathcal{V}}$} \hspace{-0.2ex}
\potential(\hspace{-0.1ex}\bm{u}\hspace{-0.1ex}) \hspace{.2ex} d\mathcal{V} \hspace{.1ex}
- \scalebox{0.88}[1]{$\biggl($} \hspace{.1ex}
\scalebox{0.9}{$\displaystyle\integral\displaylimits_{\mathcal{V}}$} \hspace{-0.2ex}
\bm{f} \hspace{-0.1ex} \dotp \bm{u} \hspace{.3ex} d\mathcal{V}
+ \hspace{-0.3ex}
\scalebox{0.9}{$\displaystyle\integral\displaylimits_{o_2}$} \hspace{-0.1ex}
\bm{p} \dotp \bm{u} \hspace{.33ex} do
\scalebox{0.88}[1]{$\biggr)$} \hspace{-0.3ex}
= \hspace{.1ex} - \hspace{.1ex}
\scalebox{0.9}{$\displaystyle\integral\displaylimits_{\mathcal{V}}$} \hspace{-0.2ex}
\potential(\hspace{-0.1ex}\bm{u}\hspace{-0.1ex}) \hspace{.2ex} d\mathcal{V}
.
\vspace{-0.1em}\end{equation*}

\en{Taking}\ru{Взяв} \en{some other}\ru{какое\hbox{-}то другое} \en{satisfactory}\ru{приемлемое}~(\inquotes{admissible}) \en{field of~displacements}\ru{поле перемещений}~${\bm{u}'\hspace{-0.25ex}}$, \en{have a~look at}\ru{взгл\'{я}нем на} \en{finite difference}\ru{конечную разность}

\nopagebreak\vspace{-0.25em}\begin{equation*}
\potentialenergyfunctional \hspace{.16ex} (\hspace{-0.1ex}\bm{u}'\hspace{.1ex}) \hspace{-0.2ex} -
\potentialenergyfunctional \hspace{.16ex} (\hspace{-0.1ex}\bm{u}\hspace{-0.1ex}) \hspace{-0.2ex}
= \hspace{-0.3ex}
\scalebox{0.9}{$\displaystyle\integral\displaylimits_{\mathcal{V}}$} \hspace{-0.44ex}
\Bigl(
\potential(\hspace{-0.1ex}\bm{u}'\hspace{.1ex}) \hspace{-0.2ex} - \potential(\hspace{-0.1ex}\bm{u}\hspace{-0.1ex}) \hspace{-0.2ex} - \hspace{-0.2ex} \bm{f} \hspace{-0.17ex} \dotp \hspace{-0.1ex} ( \bm{u}' \hspace{-0.4ex} - \hspace{-0.1ex} \bm{u} ) \Bigr) \hspace{-0.1ex} d\mathcal{V}
- \hspace{-0.3ex}
\scalebox{0.9}{$\displaystyle\integral\displaylimits_{o_2}$} \hspace{-0.1ex}
\bm{p} \dotp \hspace{-0.1ex} ( \bm{u}' \hspace{-0.4ex} - \hspace{-0.1ex} \bm{u} ) \hspace{.2ex} do
\hspace{.1ex} ,
\vspace{-0.1em}\end{equation*}

\vspace{-0.2em}\noindent
\en{seeking}\ru{ищ\'{а}}
${\potentialenergyfunctional \hspace{.16ex} (\hspace{-0.1ex}\bm{u}'\hspace{.1ex}) \hspace{-0.2ex} -
\potentialenergyfunctional \hspace{.16ex} (\hspace{-0.1ex}\bm{u}\hspace{-0.1ex}) \hspace{-0.1ex} \geq 0}$
\en{or}\ru{или} (\en{ditto}\ru{то~же самое}) ${\potentialenergyfunctional \hspace{.16ex} (\hspace{-0.1ex}\bm{u}'\hspace{.1ex}) \hspace{-0.2ex} \geq \potentialenergyfunctional \hspace{.16ex} (\hspace{-0.1ex}\bm{u}\hspace{-0.1ex})}$.

${\bm{f} \hspace{-0.2ex} = \hspace{-0.1ex} \boldconstant}$ \en{and}\ru{и}~${\bm{p} \hspace{-0.1ex} = \hspace{-0.1ex} \boldconstant}$

${\potential(\hspace{-0.1ex}\bm{a}\hspace{-0.1ex}) \hspace{-0.2ex} = \hspace{.1ex} \smallerdisplaystyleonehalf \boldnabla \hspace{-0.1ex} \bm{a} \hspace{-0.1ex} \dotdotp \hspace{-0.1ex} \stiffnesstensor \dotdotp \hspace{-0.25ex} \boldnabla \hspace{-0.1ex} \bm{a}}$
(but \emph{not} linear ${\smallerdisplaystyleonehalf \boldnabla \bm{u} \hspace{-0.1ex} \dotdotp \hspace{-0.1ex} \stiffnesstensor \dotdotp \hspace{-0.25ex} \boldnabla \hspace{-0.1ex} \bm{a}}$\:--- this means ${\potential(\hspace{-0.1ex}\bm{a}\hspace{-0.1ex}) \hspace{-0.2ex} \neq \smallerdisplaystyleonehalf \hspace{.1ex} \linearstress \dotdotp \hspace{-0.25ex} \boldnabla \hspace{-0.1ex} \bm{a}}$)

\en{Constraints don’t change}\ru{Связи не~меняются}: ${( \bm{u}' \hspace{-0.4ex} - \hspace{-0.1ex} \bm{u} ) \bigr|_{o_1} \hspace{-0.8ex} = \bm{u}_{\raisemath{-0.1em}{0}} \hspace{-0.3ex} - \bm{u}_{\raisemath{-0.1em}{0}} \hspace{-0.22ex} = \bm{0}}$.
\en{External}\ru{Внешняя} \en{surface force}\ru{поверхностная сила} ${ \hspace{.1ex}\bm{p} \hspace{.15ex} \bigr|_{o_2} \hspace{-0.9ex} = \hspace{-0.1ex} \tractionvector{\currentunitnormal} \hspace{-0.1ex} = \currentunitnormal \dotp \linearstress }$ \en{on}\ru{на}~${o_2}$
\en{and}\ru{и}~${{=} \hspace{.33ex} \bm{0}}$ \en{elsewhere}\ru{где-либо ещё} \en{on}\ru{на}~${o(\boundary \mathcal{V})}$.
${\linearstress \hspace{-0.1ex} = \hspace{-0.3ex} \boldnabla \bm{u} \dotdotp \stiffnesstensor \hspace{-0.1ex} = \hspace{-0.1ex} \constantbivalent}$ along with constant~$\bm{p}$ \en{and}\ru{и}~$\bm{f}$\hbox{\hspace{-0.2ex}.}
\en{There\-fore}\ru{Поэтому}

\nopagebreak\vspace{-0.3em}\begin{multline*}
\hspace{-1em}\begin{gathered}
\scalebox{0.9}{$\displaystyle\integral\displaylimits_{o_2}$} \hspace{-0.1ex}
\bm{p} \dotp \hspace{-0.1ex} ( \bm{u}' \hspace{-0.4ex} - \hspace{-0.1ex} \bm{u} ) \hspace{.2ex} do \hspace{.1ex}
= \hspace{-0.1ex}
\scalebox{0.9}{$\displaystyle\ointegral\displaylimits_{\mathclap{o\hspace{.1ex}(\boundary \mathcal{V})}}$} \hspace{.1ex}
\unitnormalvector \dotp \linearstress \dotp \hspace{-0.1ex} ( \bm{u}' \hspace{-0.4ex} - \hspace{-0.1ex} \bm{u} ) \hspace{.2ex} do \hspace{.1ex}
= \hspace{-0.3ex}
\scalebox{0.9}{$\displaystyle\integral\displaylimits_{\mathcal{V}}$} \hspace{-0.33ex}
\boldnabla \dotp \hspace{-0.1ex} \Bigl( \hspace{-0.2ex} \linearstress \dotp \hspace{-0.1ex} ( \bm{u}' \hspace{-0.4ex} - \hspace{-0.1ex} \bm{u} ) \Bigr) \hspace{-0.1ex} d\mathcal{V}
=
\end{gathered}
\\[-0.6em]
%
\begin{gathered}
= \hspace{-0.3ex}
\scalebox{0.9}{$\displaystyle\integral\displaylimits_{\mathcal{V}}$} \hspace{-0.33ex}
( \hspace{.1ex} \boldnabla \dotp \linearstress \hspace{.12ex} ) \hspace{-0.2ex} \dotp \hspace{-0.2ex} ( \bm{u}' \hspace{-0.4ex} - \hspace{-0.1ex} \bm{u} ) \hspace{.2ex} d\mathcal{V}
+ \hspace{-0.3ex}
\scalebox{0.9}{$\displaystyle\integral\displaylimits_{\mathcal{V}}$} \hspace{-0.3ex}
\linearstress^{\hspace{.12ex}\T} \hspace{-0.4ex} \dotdotp \hspace{-0.1ex} \boldnabla \hspace{.1ex} ( \bm{u}' \hspace{-0.4ex} - \hspace{-0.1ex} \bm{u} ) \hspace{.2ex} d\mathcal{V}
.
\end{gathered}\hspace{-1em}
\end{multline*}

\vspace{-0.3em}\noindent
\en{Due to}\ru{из\hbox{-}за} \en{symmetry}\ru{симметрии} ${\linearstress^{\hspace{.12ex}\T} \hspace{-0.5ex} = \hspace{-0.1ex} \linearstress}$ ${\hspace{.4ex}\Rightarrow\hspace{.2ex}}$ ${\linearstress^{\hspace{.12ex}\T} \hspace{-0.4ex} \dotdotp \hspace{-0.2ex} \boldnabla \bm{a} = \linearstress \dotdotp \hspace{-0.2ex} \boldnabla \bm{a} = \linearstress \dotdotp \hspace{-0.2ex} \boldnabla \bm{a}^{\mathsf{S}} \hspace{.8ex}\forall \bm{a}}$.
\textcolor{magenta}{\foreignlanguage{russian}{Разность преобразуется до}}

\nopagebreak\vspace{-0.1em}\begin{equation*}
\begin{multlined}[\displaywidth]
\potentialenergyfunctional \hspace{.16ex} (\hspace{-0.1ex}\bm{u}'\hspace{.1ex}) \hspace{-0.2ex} -
\potentialenergyfunctional \hspace{.16ex} (\hspace{-0.1ex}\bm{u}\hspace{-0.1ex}) \hspace{-0.2ex}
=
\\[-0.1em]
%
= \hspace{-0.3ex}
\scalebox{0.9}{$\displaystyle\integral\displaylimits_{\mathcal{V}}$} \hspace{-0.44ex}
\Bigl(
\potential(\hspace{-0.1ex}\bm{u}'\hspace{.1ex}) \hspace{-0.2ex}
- \potential(\hspace{-0.1ex}\bm{u}\hspace{-0.1ex}) \hspace{-0.2ex}
- \hspace{-0.2ex} \bigl( \hspace{.1ex} \boldnabla \hspace{-0.1ex} \dotp \hspace{-0.1ex} \linearstress + \hspace{-0.1ex} \bm{f} \hspace{.1ex} \bigr) \hspace{-0.3ex} \dotp \hspace{-0.2ex} ( \bm{u}' \hspace{-0.4ex} - \hspace{-0.1ex} \bm{u} ) \hspace{-0.2ex}
- \linearstress \dotdotp \hspace{-0.1ex} \boldnabla \hspace{.1ex} ( \bm{u}' \hspace{-0.4ex} - \hspace{-0.1ex} \bm{u} ) \hspace{-0.1ex}
\Bigr) d\mathcal{V}
.
\end{multlined}
\end{equation*}

\noindent
\en{And}\ru{И} \en{with the~balance of~momentum}\ru{с~балансом импульса} ${\hspace{-0.24ex}\boldnabla \hspace{-0.1ex} \dotp \hspace{-0.1ex} \linearstress + \hspace{-0.1ex} \bm{f} \hspace{-0.15ex} = \bm{0}}$

\nopagebreak\vspace{-0.2em}\begin{equation*}
\potentialenergyfunctional \hspace{.16ex} (\hspace{-0.1ex}\bm{u}'\hspace{.1ex}) \hspace{-0.2ex} -
\potentialenergyfunctional \hspace{.16ex} (\hspace{-0.1ex}\bm{u}\hspace{-0.1ex}) \hspace{-0.2ex}
= \hspace{-0.3ex}
\scalebox{0.9}{$\displaystyle\integral\displaylimits_{\mathcal{V}}$} \hspace{-0.44ex}
\Bigl(
\potential(\hspace{-0.1ex}\bm{u}'\hspace{.1ex}) \hspace{-0.2ex}
- \potential(\hspace{-0.1ex}\bm{u}\hspace{-0.1ex}) \hspace{-0.2ex}
- \linearstress \dotdotp \hspace{-0.1ex} \boldnabla \hspace{.1ex} ( \bm{u}' \hspace{-0.4ex} - \hspace{-0.1ex} \bm{u} ) \hspace{-0.1ex}
\Bigr) d\mathcal{V}
.
\end{equation*}

\noindent
\en{Here}\ru{Тут}

\begin{equation*}
\potential(\hspace{-0.1ex}\bm{u}'\hspace{.1ex}) \hspace{-0.2ex} = \smalldisplaystyleonehalf \boldnabla \bm{u}' \hspace{-0.4ex} \dotdotp \hspace{-0.1ex} \stiffnesstensor \dotdotp \hspace{-0.25ex} \boldnabla \bm{u}'
\hspace{-0.2ex} ,
\hspace{.6em}
\potential(\hspace{-0.1ex}\bm{u}\hspace{-0.1ex}) \hspace{-0.2ex} = \smalldisplaystyleonehalf \boldnabla \bm{u} \dotdotp \stiffnesstensor \dotdotp \hspace{-0.25ex} \boldnabla \bm{u}
\hspace{.1ex} ,
\end{equation*}

\[
\potential(\hspace{-0.1ex}\bm{u}'\hspace{.1ex}) \hspace{-0.2ex}
- \potential(\hspace{-0.1ex}\bm{u}\hspace{-0.1ex}) \hspace{-0.2ex}
= \smalldisplaystyleonehalf \hspace{-0.1ex} \scalebox{0.95}{$ \Bigl( \hspace{-0.1ex}
\boldnabla \bm{u}' \hspace{-0.4ex} \dotdotp \hspace{-0.1ex} \stiffnesstensor \dotdotp \hspace{-0.25ex} \boldnabla \bm{u}' \hspace{-0.25ex}
- \hspace{-0.2ex} \boldnabla \bm{u} \dotdotp \stiffnesstensor \dotdotp \hspace{-0.25ex} \boldnabla \bm{u}
\Bigr) $}
\]

\[
\stiffnesstensor_{\hspace{.12ex} 12 \scalebox{0.6}[0.8]{$\rightleftarrows$} 34} \hspace{-0.25ex} = \stiffnesstensor
\hspace{.4em}\Rightarrow\hspace{.25em}
\boldnabla \bm{u} \dotdotp \hspace{-0.1ex} \stiffnesstensor \dotdotp \hspace{-0.25ex} \boldnabla \bm{u}' \hspace{-0.25ex}
= \hspace{-0.2ex} \boldnabla \bm{u}' \hspace{-0.4ex} \dotdotp \hspace{-0.1ex} \stiffnesstensor \dotdotp \hspace{-0.25ex} \boldnabla \bm{u}
\]

\[
\smalldisplaystyleonehalf \hspace{-0.1ex} \scalebox{0.95}{$ \Bigl( \hspace{-0.1ex}
\boldnabla \bm{u}' \hspace{-0.4ex} \dotdotp \hspace{-0.1ex} \stiffnesstensor \dotdotp \hspace{-0.25ex} \boldnabla \bm{u}' \hspace{-0.25ex}
- \hspace{-0.2ex} \boldnabla \bm{u} \dotdotp \stiffnesstensor \dotdotp \hspace{-0.25ex} \boldnabla \bm{u}
+ \hspace{-0.2ex} \boldnabla \bm{u} \dotdotp \hspace{-0.1ex} \stiffnesstensor \dotdotp \hspace{-0.25ex} \boldnabla \bm{u}' \hspace{-0.25ex}
- \hspace{-0.2ex} \boldnabla \bm{u}' \hspace{-0.4ex} \dotdotp \hspace{-0.1ex} \stiffnesstensor \dotdotp \hspace{-0.25ex} \boldnabla \bm{u}
\Bigr) $}
\]

\noindent
dot product is distributive ...

\noindent
differentiation is linear

\[
\bigl( \boldnabla \bm{u}' \hspace{-0.3ex} - \hspace{-0.32ex} \boldnabla \bm{u} \hspace{.1ex} \bigr) \hspace{-0.2ex}
= \hspace{-0.1ex} \boldnabla \hspace{.1ex} ( \bm{u}' \hspace{-0.4ex} - \hspace{-0.1ex} \bm{u} )
\]

\noindent
... \foreignlanguage{russian}{для конечной разности потенциалов}

\[
\smalldisplaystyleonehalf \boldnabla \bigl(
\bm{u}' \hspace{-0.4ex} + \hspace{-0.1ex} \bm{u}
\hspace{.1ex} \bigr) \hspace{-0.2ex}
\dotdotp \stiffnesstensor \hspace{-0.1ex} \dotdotp \hspace{-0.15ex} \boldnabla \bigl(
\bm{u}' \hspace{-0.4ex} - \hspace{-0.1ex} \bm{u}
\hspace{.1ex} \bigr) \hspace{-0.25ex}
= \hspace{.1ex} \potential(\hspace{-0.1ex}\bm{u}'\hspace{.1ex}) \hspace{-0.2ex}
- \potential(\hspace{-0.1ex}\bm{u}\hspace{-0.1ex})
\hspace{.1ex} ,
\]

\noindent
\foreignlanguage{russian}{добавляя к кот\'{о}рой}

\[
- \boldnabla \bm{u} \dotdotp \stiffnesstensor \dotdotp \hspace{-0.25ex} \boldnabla \hspace{.1ex} ( \bm{u}' \hspace{-0.4ex} - \hspace{-0.1ex} \bm{u} ) \hspace{-0.2ex}
= - \hspace{.2ex} \linearstress \dotdotp \hspace{-0.1ex} \boldnabla \hspace{.1ex} ( \bm{u}' \hspace{-0.4ex} - \hspace{-0.1ex} \bm{u} )
\]

\noindent
\foreignlanguage{russian}{получаем}

\[
\smalldisplaystyleonehalf \boldnabla \bigl(
\bm{u}' \hspace{-0.4ex} - \hspace{-0.1ex} \bm{u}
\hspace{.1ex} \bigr) \hspace{-0.2ex}
\dotdotp \stiffnesstensor \hspace{-0.1ex} \dotdotp \hspace{-0.15ex} \boldnabla \bigl(
\bm{u}' \hspace{-0.4ex} - \hspace{-0.1ex} \bm{u}
\hspace{.1ex} \bigr) \hspace{-0.25ex}
= \hspace{.1ex} \potential(\hspace{-0.1ex}\bm{u}' \hspace{-0.4ex} - \hspace{-0.1ex} \bm{u}\hspace{.1ex})
%%\hspace{.1ex} ,
\]

\noindent
\foreignlanguage{russian}{и в~конц\'{е}\hbox{-}конц\'{о}в}\footnote{%
${b^{\hspace{.1ex}2} \hspace{-0.2ex} - a^2 \hspace{-0.2ex} - 2 \hspace{.1ex} a \hspace{.1ex} (b \hspace{-0.1ex} - \hspace{-0.2ex} a) \hspace{-0.1ex}
=\hspace{-0.1ex}  (b \hspace{-0.1ex} + \hspace{-0.2ex} a) \hspace{.1ex} (b \hspace{-0.1ex} - \hspace{-0.2ex} a) \hspace{-0.15ex} - 2 \hspace{.1ex} a \hspace{.1ex} (b \hspace{-0.1ex} - \hspace{-0.2ex} a) \hspace{-0.1ex}
= \hspace{-0.1ex} (b \hspace{-0.1ex} - \hspace{-0.2ex} a)^2}$}

\nopagebreak\vspace{-0.2em}\begin{equation*}
\potentialenergyfunctional \hspace{.16ex} (\hspace{-0.1ex}\bm{u}'\hspace{.1ex}) \hspace{-0.2ex} -
\potentialenergyfunctional \hspace{.16ex} (\hspace{-0.1ex}\bm{u}\hspace{-0.1ex}) \hspace{-0.2ex}
= \hspace{-0.3ex}
\scalebox{0.9}{$\displaystyle\integral\displaylimits_{\mathcal{V}}$} \hspace{-0.22ex}
\potential(\hspace{-0.1ex}\bm{u}' \hspace{-0.4ex} - \hspace{-0.1ex} \bm{u}\hspace{.1ex})
\hspace{.1ex} d\mathcal{V}
.
\end{equation*}

\en{Since}\ru{Потому как} $\stiffnesstensor$ \en{is positive definite}\ru{положительно определён} (\pararef{para:uniquenessfordynamicproblem}),
${\potential(\hspace{-0.1ex}\bm{w}\hspace{-0.1ex}) \hspace{-0.25ex} = \smallerdisplaystyleonehalf \scalebox{0.95}{$\boldnabla \bm{w} \hspace{-0.1ex} \dotdotp \hspace{-0.1ex} \stiffnesstensor \dotdotp \hspace{-0.25ex} \boldnabla \bm{w}$} \hspace{-0.1ex} \geq \hspace{-0.1ex} 0 \hspace{.8ex} \forall \bm{w}}$
(\en{and}\ru{и} ${= \hspace{-0.2ex} 0}$ \en{only if}\ru{только если} ${\boldnabla \bm{w} \hspace{-0.2ex} = \hspace{-0.2ex} \bm{0} \hspace{.2ex}\Leftrightarrow\hspace{.2ex} \bm{w} \hspace{-0.2ex} = \hspace{-0.2ex} \boldconstant}$: \en{for a~case of~translation}\ru{для случая трансляции} \en{as a~whole}\ru{как целого} \en{without deformation}\ru{без деформации}).

...

${\variation{\boldnabla \fieldofdisplacements} \hspace{-0.1ex} = \hspace{-0.3ex} \boldnabla \variation{\fieldofdisplacements}}$

...

\ru{метод }Ritz\ru{’а}\en{ method}

\begin{otherlanguage}{russian}

Задача о~минимуме функционала~${\potentialenergyfunctional \hspace{.2ex} (\hspace{-0.1ex}\bm{u}\hspace{-0.1ex})}$ приближённо решается как

...

метод конечных элементов, finite element method

...

\end{otherlanguage}

\en{\section{Principle of minimum complementary energy}}

\ru{\section{Принцип минимума дополнительной энергии}}

\label{para:principleofminimumcomplementaryenergy}

\vspace{.2em}\begin{changemargin}{\parindent}{\parindent}
\small
When the~constitutive stress\hbox{--}strain relations assure the~existence of a~complementary energy function and the~geometrical boundary conditions are assumed constant during variation of~stresses, then the~principle of~minimum complementary energy emerges.
\par
\nopagebreak\vspace{.2em}
\end{changemargin}

\en{The complementary energy}\ru{Дополнительная энергия} \en{of a~linear-elastic body}\ru{линейно-упругого т\'{е}ла} \en{is}\ru{есть} \en{the~following}\ru{следующий} \en{functional}\ru{функционал} \en{over the~field of~stresses}\ru{над~полем напряжений}:

\nopagebreak\begin{equation}\label{complementaryenergyfunctional}
\complementaryenergyfunctional (\hspace{-0.1ex}\linearstress\hspace{.1ex}) \equiv \hspace{-0.2ex}
\displaystyle\integral\displaylimits_{\mathcal{V}} \hspace{-0.5ex}
\widehat{\potential}(\hspace{-0.1ex}\linearstress\hspace{.1ex}) \hspace{.2ex} d\mathcal{V} \hspace{.1ex}
- \hspace{-0.2ex}
\displaystyle\integral\displaylimits_{o_1} \hspace{-0.5ex}
\unitnormalvector \hspace{.1ex} \dotp \linearstress \dotp \bm{u}_{\raisemath{-0.1em}{0}} \hspace{.25ex} do
\hspace{.2ex} ,
\hspace{.5em}
\bm{u}_{\raisemath{-0.1em}{0}} \hspace{-0.2ex} \equiv \bm{u} \hspace{.1ex} \bigr|_{o_1}
\hspace{-0.1ex} ,
\end{equation}
%
\nopagebreak\vspace{-0.4em}\begin{equation*}
\boldnabla \dotp \linearstress \hspace{.15ex} + \bm{f} = \hspace{.1ex} \bm{0} \hspace{.1ex} ,
\hspace{.6em}
\unitnormalvector \dotp \linearstress \hspace{.2ex} \bigr|_{o_2} \hspace{-0.66ex} = \hspace{.2ex} \bm{p}
\hspace{.2ex} .
\end{equation*}

...

${
\variation{\hspace{.1ex} \bigl( \hspace{.1ex} \boldnabla \dotp \linearstress \hspace{.15ex} + \bm{f} \hspace{.15ex} \bigr)} \hspace{-0.2ex}
= \hspace{-0.1ex} \boldnabla \dotp \variation{\linearstress} \hspace{.15ex}
= \hspace{.1ex} \bm{0}
}$ \en{inside volume}\ru{в~объёме}~$\mathcal{V}$

...

\en{The~principle of minimum complementary energy}\ru{Принцип минимума дополнительной энергии} \en{is very useful}\ru{очень полезен} \en{for estimating}\ru{для оценки} \en{inexact}\ru{неточных}~(\en{approximate}\ru{приближённых}) \en{solutions}\ru{решений}.
\en{But}\ru{Но} \en{for computations}\ru{для вычислений} \en{it isn’t so essential}\ru{он не~столь существенен,} \en{as}\ru{как} \en{the}\ru{принцип}~(Lagrange\ru{’а})\en{ principle} \en{of~minimum potential energy}\ru{минимума потенциальной энергии} \eqref{principleofminimumpotentialenergy.formulation}.

\en{\section{Mixed principles of stationarity}}

\ru{\section{Смешанные принципы стационарности}}

\label{para:mixedvariationalprinciples}

\vspace{.2em}\begin{changemargin}{2\parindent}{\parindent}
\bgroup % to change \parindent locally
\setlength{\parindent}{\negparindent}
\small

\hspace{\parindent}\href{https://en.wikiversity.org/wiki/Introduction_to_Elasticity/Hellinger-Reissner_principle}{\textbold{Prange\hbox{--}Hellinger\hbox{--}Reissner Variational Principle}},\\
named after \emph{Ernst Hellinger}, \emph{Georg Prange} and \emph{Eric Reissner}.
\par

\nopagebreak\vspace{.16em}
{\scriptsize \noindent Working independently of Hellinger and Prange, Eric Reissner published his famous six\hbox{-}page paper \inquotes{On a~variational theorem in~elasticity} in~1950. In this paper he develops\:--- without, however, considering Hamilton\hbox{--}Jacobi theory\:--- a~variational principle same to that of~Prange and~Hellinger.\par}

\nopagebreak\vspace{.32em}
\href{https://en.wikiversity.org/wiki/Introduction_to_Elasticity/Hu-Washizu_principle}{\textbold{Hu\hbox{--}Washizu Variational Principle}},\\
named after \emph{Hu Haichang} and \emph{Kyuichiro\;Washizu}.
\par
\egroup
\nopagebreak\vspace{.1em}
\end{changemargin}

\noindent
\en{The~following}\ru{Следующий} \en{functional}\ru{функционал} \en{over}\ru{над} \en{displacements}\ru{перемещениями} \en{and}\ru{и}~\en{stresses}\ru{напряжениями}

\nopagebreak\vspace{-0.3em}\begin{multline}\label{reissnerhellingervariationalprinciple}
\shoveleft{\mathcal{R}(\bm{u}, \hspace{-0.32ex}\linearstress) \hspace{.2ex} =
\displaystyle\integral\displaylimits_{\mathcal{V}} \hspace{-0.3ex}
\Bigl[
\linearstress \dotdotp \hspace{-0.16ex} \boldnabla {\bm{u}}^{\mathsf{S}} \hspace{-0.2ex} - \hspace{.1ex} \widehat{\potential}(\hspace{-0.1ex}\linearstress\hspace{.1ex})\hspace{-0.1ex} -
\bm{f} \hspace{-0.1ex} \dotp \bm{u}
\hspace{.2ex} \Bigr] \hspace{-0.1ex} d\mathcal{V} \hspace{.5ex}
- \hfill}
\\[-1.2em]
%
\hspace{8em}
- \hspace{-0.2ex} \displaystyle\integral\displaylimits_{o_1} \hspace{-0.4ex} \unitnormalvector \dotp \hspace{-0.1ex} \linearstress \dotp \hspace{-0.1ex} \bigl( \bm{u} - \bm{u}_{\raisemath{-0.1em}{0}} \bigr) do \hspace{.2ex}
- \hspace{-0.2ex} \integral\displaylimits_{o_2} \hspace{-0.4ex} \bm{p} \dotp \bm{u} \hspace{.25ex} do
\end{multline}

\nopagebreak\vspace{-0.2em}\noindent
\en{carries}\ru{н\'{о}сит} \en{names}\ru{имена} \en{of~}Reissner\ru{’а}, Prange\ru{’а} \en{and}\ru{и}~Hellinger\ru{’а}.

...

\begin{otherlanguage}{russian}

Преимущество принципа Reissner’а\hbox{--}Hellinger’а\:--- в~свободе варьирования.
Но есть и~изъян: у~функционала нет экстремума на~истинном решении, а~лишь стационарность.

Принцип можно использовать для~построения приближённых решений методом Ритца~(Ritz method).
Задавая аппроксимации

...

Принцип Hu\hbox{--}Washizu\ru{~(Ху\hbox{--}Васидзу)}~\cite{washizubook} формулируется так:

\nopagebreak\vspace{-0.2em}\begin{multline}\label{huwashizuvariationalprinciple}
\hfil \variation{\mathcal{W}} \hspace{.1ex} (\bm{u}, \hspace{-0.32ex}\infinitesimaldeformation, \hspace{-0.32ex}\linearstress) = 0
\hspace{.1ex} ,
\\[-0.1em]
%
\shoveleft{\displaystyle \mathcal{W} \hspace{.2ex} \equiv
\displaystyle\integral\displaylimits_{\mathcal{V}} \hspace{-0.3ex}
\Bigl[
\linearstress \dotdotp \hspace{-0.2ex} \Bigl( \hspace{-0.2ex} \boldnabla {\bm{u}}^{\mathsf{S}} \hspace{-0.2ex} - \infinitesimaldeformation \hspace{-0.1ex} \Bigr) \hspace{-0.2ex} + \hspace{.1ex} \potential(\hspace{-0.1ex}\infinitesimaldeformation\hspace{-0.1ex})\hspace{-0.1ex} -
\bm{f} \hspace{-0.1ex} \dotp \bm{u}
\hspace{.2ex} \Bigr] \hspace{-0.1ex} d\mathcal{V} \hspace{.5ex}
- \hfill}
\\[-1.2em]
%
\hspace{8em}
- \hspace{-0.2ex} \displaystyle\integral\displaylimits_{o_1} \hspace{-0.4ex} \unitnormalvector \dotp \hspace{-0.1ex} \linearstress \dotp \hspace{-0.1ex} \bigl( \bm{u} - \bm{u}_{\raisemath{-0.1em}{0}} \bigr) do \hspace{.2ex}
- \hspace{-0.2ex} \integral\displaylimits_{o_2} \hspace{-0.4ex} \bm{p} \dotp \bm{u} \hspace{.25ex} do
\hspace{.2ex} .
\end{multline}

Как и~в~принципе Рейсснера\hbox{--}Хеллингера, здесь нет ограничений ни~в~объёме, ни~на~поверхности, но добавляется третий независимый аргумент~$\infinitesimaldeformation$.
Поскольку ${\widehat{\potential} = \linearstress \dotdotp \infinitesimaldeformation - \potential}$, то~\eqref{reissnerhellingervariationalprinciple} и~\eqref{huwashizuvariationalprinciple} кажутся почти одним и~тем~же.

Из~принципа Ху\hbox{--}Васидзу вытекает вся полная система уравнений с~граничными условиями, так~как

...


Об~истории открытия вариационных принципов и~соотношении~их написано, например, у~Ю.\,Н.\;Работнова~\cite{rabotnov-mechanicsofdeformable}.

\end{otherlanguage}

\en{\section{Antiplane shear}} % Antiplane deformation % Antiplane strain

\ru{\section{Антиплоский сдвиг}}

\begin{otherlanguage}{russian}

\newcommand\antiplanedisplacement{\mathcolor{red}{\upupsilon}}

Это та проблема линейной теории упругости, где простыми выкладками получаются нетривиальные результаты\footnote{Нетривиальное в~теории упругости это, например, когда \inquotes{деление силы на~площадь} даёт бесконечно больш\'{у}ю погрешность в~нахождении напряжения.}\hbox{\hspace{-1ex}.}

Рассматривается изотропная среда в~декартовых координатах $x_i$ ($x_1$ и~$x_2$ в~плоскости, $\mathcolor{blue}{x_3}$ перпендикулярна плоскости)
с~базисными ортами
${\bm{e}_i \hspace{-0.2ex} = \partial_i \hspace{.1ex} \locationvector}$\hbox{\hspace{-0.12ex},}
${\locationvector = x_i \bm{e}_i}$,
${\bm{e}_i \bm{e}_i \hspace{-0.2ex} = \hspace{-0.15ex} \UnitDyad \,\Leftrightarrow\hspace{.25ex} \bm{e}_i \hspace{-0.1ex} \dotp \bm{e}_j \hspace{-0.2ex} = \delta_{i\hspace{-0.1ex}j}}$.
\en{In~case}\ru{В~случае} \en{of~anti\-plane deformation/strain}\ru{анти\-плос\-кой деформации} (\en{anti\-plane shear}\ru{анти\-плос\-кого сдвига}) поле перемещений~$\bm{u}$ параллельно координате $\mathcolor{blue}{x_3}$:
${\bm{u} = \antiplanedisplacement \hspace{.12ex} \mathcolor{blue}{\bm{e}_3}}$, и~$\antiplanedisplacement$ не~зависит от~$\mathcolor{blue}{x_3}$: ${\antiplanedisplacement \narroweq \antiplanedisplacement(x_1, x_2)}$, ${\partial_{\raisemath{-0.15ex}{\mathcolor{blue}{3}}} \antiplanedisplacement \hspace{-0.15ex} = 0}$.

Деформация

\nopagebreak\vspace{-0.25em}\begin{equation*}
\infinitesimaldeformation \equiv
\hspace{-0.1ex} \boldnabla {\bm{u}}^{\hspace{.1ex}\mathsf{S}} \hspace{-0.2ex}
= \hspace{-0.2ex} \boldnabla \hspace{.1ex} \bigl( \antiplanedisplacement \hspace{.1ex} \mathcolor{blue}{\bm{e}_3} \bigr)^{\hspace{-0.2ex}\mathsf{S}} \hspace{-0.25ex}
= \mathcolor{blue}{\bm{e}_3} \hspace{-0.15ex} \boldnabla \hspace{.1ex} \antiplanedisplacement^{\hspace{.1ex}\mathsf{S}} \hspace{-0.25ex}
+ \antiplanedisplacement \hspace{.1ex} \tikzmark{beginZero2} \boldnabla \mathcolor{blue}{\bm{e}_3}\tikzmark{endZero2}^{\mathsf{S}} \hspace{-0.2ex}
= \hspace{.1ex} \smalldisplaystyleonehalf \hspace{.1ex} \bigl( \boldnabla \hspace{.1ex} \antiplanedisplacement \hspace{.12ex} \mathcolor{blue}{\bm{e}_3} \hspace{-0.1ex} + \mathcolor{blue}{\bm{e}_3} \hspace{-0.15ex} \boldnabla \hspace{.1ex} \antiplanedisplacement \bigr)
\end{equation*}%
\AddUnderBrace[line width=.75pt][.2ex, -0.1ex]%
{beginZero2}{endZero2}{${\scriptstyle {^2\bm{0}}}$}%

...

Возможна неоднородность среды в~плоскости ${x_1, x_2}$: ${\mu \narroweq \mu(x_1, x_2)}$, ${\partial_{\raisemath{-0.15ex}{\mathcolor{blue}{3}}} \hspace{.08ex} \mu \hspace{-0.1ex} = 0}$.

...



\end{otherlanguage}

\en{\section{Twisting of rods}} % torsion twisting of rods

\ru{\section{Кручение стержней}}

\label{para:twistingofrods.saintvenant}

\begin{otherlanguage}{russian}

{\small
\bibauthor{M.\:de\;Saint\hbox{-\hspace{-0.2ex}}Venant}. \href{https://babel.hathitrust.org/cgi/pt?id=hvd.32044091959866&seq=7}{Memoire sur la torsion des prismes (1853)}

Adhémar-Jean-Claude Barré de Saint\hbox{-\hspace{-0.2ex}}Venant.
Mémoire sur la torsion des prismes, avec des considérations sur leur flexion ainsi que sur l'équilibre intérieur des solides élastiques en général, et des formules pratiques pour le calcul de leur résistance à divers efforts s’exerçant simultanément.
1856.
327~pages.

1. Memoire sur la torsion des prismes, avec des considerations sur leur flexion, etc. Memoires presentes par divers savants a l'Academie des scienees, t. 14, 1856.

2. Memoire sur la flexion des prismes, etc. Journal de mathematiques pures et appliquees, publie par J. Liouville, 2me serie, t. 1, 1856.

Перевод на русский язык:
\bibauthor{Сен-Венан Б.} Мемуар о~кручении призм. Мемуар об~изгибе призм. М.:\;Физ\-мат\-гиз, 1961.
\howmanypages{518~страниц.}
стр. 379--494
\par}

Эта задача, тщательно изученная Saint\hbox{-\hspace{-0.2ex}}Venant’ом, рассматривается едва~ли не~в~каждой книге по классической упругости.
Речь идёт о~цилиндре какого\hbox{-}либо сечения, нагруженном лишь поверхностными силами на торцах

...



\end{otherlanguage}

\en{\section{Plane deformation}}

\ru{\section{Плоская деформация}}

\label{para:planedeformation.linearclassicalelasticity}

\begin{otherlanguage}{russian}

Тут вектор перемещения~$\bm{u}$ параллелен плоскости ${x_1, x_2}$ и~не~зависит от~третьей координаты~$z$

...

Для примера рассмотрим полуплоскость с~сосредоточенной нормальной силой~$Q$ на~краю (?? рисунок ??)

...

\end{otherlanguage}

\vspace{8mm}
\hfill\begin{minipage}[b]{0.95\linewidth}
\fontsize{10}{12}\selectfont

\section*{\wordforbibliography}

\begin{otherlanguage}{russian}

Можно назвать несколько десятков книг по~классической теории упругости, представляющих несомненный интерес несмотря на~возрастающую отдалённость во~времени. Подробные литературные указания содержатся в~фундаментальной монографии ...

\end{otherlanguage}

\end{minipage}
