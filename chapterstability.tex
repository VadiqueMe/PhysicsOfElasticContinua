\en{\chapter{Stability}}

\ru{\chapter{Устойчивость}}

\thispagestyle{empty}

\label{chapter:stability}

\en{\section{Various approaches to the problem of stability}}

\ru{\section{Разные подходы к проблеме устойчивости}}

\en{\dropcap{T}{here’s}}\ru{\dropcap{Е}{сть}} \en{the }\en{well developed}\ru{весьма р\'{а}звитая} \en{theory of stability}\ru{теория устойчивости} \en{by }\foreignlanguage{russian}{Ляпунов}~(Lyapunov)\ru{по Ляпунову}.
\en{It says that}\ru{Она говорит, что} \en{if}\ru{если} \en{initial deviations}\ru{начальные отклонения} \en{are }\inquotes{\en{close enough}\ru{достаточно близкие}}, \en{they}\ru{они} \en{don’t increase in the future}\ru{не~возрастают в~будущем} \en{and}\ru{и} \en{stay}\ru{остаются} \inquotes{\en{close enough}\ru{достаточно близкими}} \en{forever}\ru{навсегда}, \en{then}\ru{то} \en{the process is stable}\ru{процесс устойчив}.
\en{This applies to the~condition of~equilibrium too.}\ru{Это относится и~к~состоянию равновесия тоже.}
\en{Considering}\ru{Рассматривая} \en{the dynamics}\ru{динамику} \en{of small deviations}\ru{малых отклонений} \en{from an~equilibrium configuration}\ru{от равновесной конфигурации}, \en{we make sure}\ru{мы убеждаемся}\ru{,} \en{that these deviations}\ru{что эти отклонения} \en{do not increase}\ru{не~возрастают}.
\en{This is}\ru{Так\'{о}в} \en{the~}\textbold{\en{dynamic approach}\ru{динамический подход}} \en{to the~problem of~stability}\ru{к~проблеме устойчивости}, \en{recognized as the~most reasonable}\ru{признаваемый самым обоснованным}.

\en{However}\ru{Однако}, \en{for problems}\ru{для~задач} \en{about}\ru{об} \en{the~stability of~equilibrium}\ru{устойчивости равновесия} \en{of~elastic systems}\ru{упругих систем}\en{,} \ru{обрёл популярность }\en{the~different}\ru{иной} \en{approach}\ru{подход}\en{ has gained popularity}, \en{called}\ru{называемый} \en{the }\textbold{\en{static}\ru{статическим}} \en{approach}\ru{подходом}.
\en{It}\ru{Он} \en{associates}\ru{ассоциируется} \en{with the }\ru{с~именем }Leonhard\ru{’а} Euler’\en{s}\ru{а}\en{ name}.
\en{When}\ru{Когда} \en{the equations of statics}\ru{уравнения статики} \en{give}\ru{дают} \en{a~nontrivial solution}\ru{нетривиальное решение} \en{for small deviations}\ru{для малых отклонений}, \en{then the values of parameters}\ru{тогда значения параметров} \en{are assumed to be}\ru{предполагаются} \inquotesx{\en{critical}\ru{критическими}}[.]
\en{In~other words}\ru{Иными словами}, \en{if}\ru{если} \en{an~equilibrium condition}\ru{равновесное состояние} \en{ceases to be isolated}\ru{перестаёт быть изолированным}, \en{then}\ru{то} \en{it’s considered}\ru{оно считается} \inquotes{\en{critical}\ru{критическим}}, \en{and}\ru{и} \en{many}\ru{много} \en{adjacent forms of the equilibrium}\ru{смежных форм равновесия} \en{appear then}\ru{появляется тогда} (\inquotes{\en{indifferent equilibrium}\ru{безразличное равновесие}}).
\en{Using this approach}\ru{Используя этот подход}, \en{it’s enough}\ru{достаточно} \en{to~solve}\ru{решить} \en{the~eigenvalue problem}\ru{задачу на~собственные числа}.

%%\en{in its neighborhood}\ru{в~его окрестности}

\en{And}\ru{И} \en{moreover}\ru{более того}, \en{there are also other approaches}\ru{есть ещё и~другие подходы}.
\en{For example}\ru{Например}, \en{the~}\textbold{\en{imperfection method}\ru{метод несовершенств}}: \en{if}\ru{если} \en{small random changes}\ru{м\'{а}лые случайные изменения} \en{in the initial shape}\ru{начальной формы}, \en{stiffnesses}\ru{жёсткостей}, \en{loads}\ru{нагрузок} \en{and so on}\ru{и так далее} \en{cause only a~small change}\ru{вызывают лишь м\'{а}лое изменение} \en{of a~deformed configuration in~equilibrium}\ru{деформированной конфигурации в~равновесии}, \en{then}\ru{то} \en{this equilibrium}\ru{это равновесие} \en{is stable}\ru{устойчиво}.
\en{Or}\ru{Или} \en{the~}\textbold{\en{energy approach}\ru{энергетический подход}}: \en{a~loss of stability}\ru{потеря устойчивости} \en{occurs}\ru{происходит}\ru{,} \en{when it becomes}\ru{когда она становится} \en{energetically beneficial}\ru{энергетически выгодной}, \en{that is}\ru{то~есть} \en{it leads}\ru{она ведёт} \en{to a~decrease in energy}\ru{к~уменьшению энергии}.

\en{The mentioned approaches}\ru{Упомянутые подходы} \en{draw}\ru{рисуют} \en{a~motley picture}\ru{пёструю картину}.
\en{Yet}\ru{Но} \en{it is pretty easy to~visualize}\ru{её довольно просто визуализировать} \en{for a~discrete model}\ru{для дискретной модели}\:--- \en{a~model}\ru{модели} \en{with a~finite number of~degrees of~freedom}\ru{с~конечным числом степеней свободы}.

\begin{otherlanguage}{russian}

Больш\'{о}й общностью обладают

...

\end{otherlanguage}

\en{\section{Classical problems with rods}}

\ru{\section{Классические проблемы со стержнями}}

\begin{otherlanguage}{russian}

Состояние перед варьированием описывается уравнениями нелинейной теории стержней Kirchhoff’а

...



\end{otherlanguage}

\en{\section{\inquotes{Tracking} loads}}

\ru{\section{\inquotes{Следящие} нагрузки}}

%% “dead” loads and “tracking” loads

\begin{otherlanguage}{russian}

В~проблемах устойчивости весьма вес\'{о}мо поведение нагрузки в~процессе деформирования.
Ведь в~уравнения входит вариация \textcolor{red}{(of what?)}~${\variation{\bm{q}}}$, она равна нулю лишь для \inquotes{\en{dead}\ru{мёртвых}} \en{loads}\ru{нагрузок}.
\textcolor{magenta}{Распространены} \inquotes{\en{tracking}\ru{следящие}} \en{loads}\ru{нагрузки}, \en{which}\ru{которые} \en{change in a~definite way}\ru{меняются определённым путём} \en{along with displacements of the particles}\ru{вместе со~смещениями частиц} \en{of an~elastic continuum}\ru{упругого контину\kern-0.11exума}.

\ru{Статический подход }Euler’\ru{а}\en{s}\en{ static approach} \en{to the~problem of~stability}\ru{к~проблеме устойчивости}

...



\end{otherlanguage}

\en{\section{The role of additional pliabilities}}

\ru{\section{Роль добавочных податливостей}}

\begin{otherlanguage}{russian}

Для прямого \textcolor{magenta}{консольного} стержня, сжатого постоянной силой~$\bm{F}$ на~свободном конце, критическая нагрузка определяется формулой Euler’а

...



\end{otherlanguage}

\en{\section{Variational formulations}}

\ru{\section{Вариационные формулировки}}

\begin{otherlanguage}{russian}

Во~всех разделах линейной теории упругости больш\'{у}ю роль играют вариационные постановки.
Среди прочего, они составляют основу метода конечных элементов как варианта метода Ritz’а.

Менее развиты вариационные постановки для проблем устойчивости.
Здесь получил популярность метод

...



\end{otherlanguage}

\en{\section{Nonconservative problems}}

\ru{\section{Неконсервативные задачи}}

\begin{otherlanguage}{russian}

В~уравнении динамики (...) матрица позиционных сил

...



\end{otherlanguage}

\en{\section{Case of multiple roots}}

\ru{\section{Случай кратных корней}}

\begin{otherlanguage}{russian}

Вернёмся к~проблеме устойчивости (...) в~случае циркуляционных сил.
Как уже отмечалось\textcolor{red}{(где??)}, критическая ситуация характеризуется

...



\end{otherlanguage}

\section*{\small \wordforbibliography}

\begin{changemargin}{\parindent}{0pt}
\fontsize{10}{12}\selectfont

\begin{otherlanguage}{russian}

Увлекательные вопросы устойчивости упругих систем освещены в~книгах ...

\end{otherlanguage}

\end{changemargin}
