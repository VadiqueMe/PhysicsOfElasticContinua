\en{\chapter{Stability}}

\ru{\chapter{Устойчивость}}

\thispagestyle{empty}

\label{chapter:stability}

\begin{comment} %%%
\begin{changemargin}{\parindent}{\parindent}
\vspace{-2.5em}
{\noindent\small
\setlength{\parskip}{\spacebetweenparagraphs}

\inquotes{\en{indifferent equilibrium}\ru{безразличное равновесие}}
\en{or}\ru{или}
\inquotes{\en{neutral equilibrium}\ru{нейтральное равновесие}}
(\en{neither stability nor instability}\ru{ни устойчивость, ни неустойчивость},
a~disturbance introduced into such an~equilibrium will be neither amplified nor damped)

\par}
\vspace{-1.2em}
\end{changemargin}
\end{comment} %%%

\en{\section{Various approaches to the problem of stability}}

\ru{\section{Разные подходы к проблеме устойчивости}}

\label{section:approachestostability}

\en{\dropcap{T}{here’s}}\ru{\dropcap{Е}{сть}}
\en{the }\en{well developed}\ru{весьма р\'{а}звитая}
\href{https://en.wikipedia.org/wiki/Lyapunov_stability}{\en{theory of stability}\ru{теория устойчивости}}
\en{by}\ru{по}
\href{https://en.wikipedia.org/wiki/Aleksandr_Lyapunov}{\en{\russianlanguage{А.\,Ляпунов}~(Aleksandr Lyapunov)}\ru{Александру Ляпунову}}.
\en{It looks at}\ru{Она смотрит на}
\en{the~dynamics}\ru{динамику}
\en{of small deviations}\ru{малых отклонений}
\en{and says that}\ru{и~говорит, что}
\en{if}\ru{если}
\en{initial deviations}\ru{начальные отклонения}
(\en{for example}\ru{для примера},
\en{from the~equilibrium}\ru{от равновесия})
\en{were}\ru{были}
\inquotesx{\en{quite close~(small)}\ru{довольно близкие~(м\'{а}лые)}}[,]
\en{and }\ru{и~}\en{they}\ru{они}
\en{don’t rise}\ru{не~растут}
\en{any further}\ru{как-либо в~дальнейшем},
\en{remaining}\ru{оставаясь}
\inquotesx{\en{quite close}\ru{довольно близкими}}
\en{forever}\ru{навсегда},
\en{then}\ru{то}
\en{the~process is stable}\ru{процесс устойчив}.
\en{This is}\ru{Так\'{о}в}
\en{the~}\textbold{\en{dynamic approach}\ru{динамический подход}}
\en{to the~problem of~stability}\ru{к~проблеме устойчивости},
\en{and this approach is}\ru{и~этот подход}
\en{the~most reasonable}\ru{самый обоснованный}.

\en{However}\ru{Однако},
\en{for problems}\ru{для~задач}
\en{about}\ru{об}
\en{the~stability of the~equilibrium}\ru{устойчивости равновесия}\en{,}
%\en{of~elastic systems}\ru{упругих систем}\en{,}
\ru{нашёл больш\'{у}ю популярность }\en{the~other}\ru{другой}
\en{approach}\ru{подход}\en{ has found much popularity}.
%
\en{It is called}\ru{Он называется}
\en{the~}\textbold{\en{static}\ru{статическим}
\en{approach}\ru{подходом}},
\en{and }\ru{и~}\en{is }\href{https://en.wikipedia.org/wiki/Euler%27s_critical_load}{\en{widely known}\ru{широк\'{о} известен}}
\en{under}\ru{под}
\ru{именем }\en{the~}\href{https://en.wikipedia.org/wiki/Leonhard_Euler}{Leonhard}\ru{’а}
\href{https://en.wikipedia.org/wiki/List_of_things_named_after_Leonhard_Euler}{Euler}’\en{s}\ru{а}\en{ name}.
%
\en{When}\ru{Когда}
\en{the~equations of~statics}\ru{уравнения статики}
\en{give}\ru{дают}
\en{a~nontrivial solution}\ru{нетривиальное решение}
\en{for small}\ru{для м\'{а}лых}
\en{disturbances-displacements}\ru{возмущений-смещений},
\en{then}\ru{тогда}
\en{the~values of~parameters}\ru{значения параметров}
\en{are assumed to be}\ru{предполагаются}
\inquotesx{\en{critical}\ru{критическими}}[.]
%
\en{In~other words}\ru{Другими словами},
\en{if}\ru{если}
\en{there’s}\ru{есть}
\en{a~non\hbox{-}isolated equilibrium}\ru{не изолированное равновесие}\footnote{% begins
\inquotes{\en{Non\hbox{-}isolated}\ru{не изолированное}}
\en{means that}\ru{значит, что}
\en{at the~same}\ru{в~одно и~то~же}
\en{time}\ru{время}
\en{there appear}\ru{появляется}
\en{many}\ru{много}
\en{possible}\ru{возможных}
\en{adjacent forms}\ru{смежных форм}
\en{of the~same one}\ru{одного и~того~же}
\en{equilibrium}\ru{равновесия}.}\hbox{\hspace{-0.5ex},} % \footnote ends
\en{then}\ru{то}
\en{it’s considered}\ru{оно считается}
\inquotesx{\en{critical}\ru{критическим}}[.]
%
\en{With this approach}\ru{С~этим подходом}
\en{it’s enough}\ru{достаточно}
\en{to~solve}\ru{решить}
\en{the~eigenvalue problem}\ru{проблему собственных чисел}.

\en{More approaches exist as well}\ru{Существуют и~другие подходы}.
%
\en{One of them}\ru{Один из них}\en{ is}\ru{\:---}
\en{the~}\textbold{\en{method of~imperfections}\ru{метод несовершенств}}\::
\en{if}\ru{если}
\en{small random changes}\ru{м\'{а}лые случайные изменения}
\en{of the initial shape}\ru{начального \'{о}блика},
\en{the stiffnesses}\ru{жёсткостей},
\en{the loads}\ru{нагрузок}
\en{and other}\ru{и~других}
\en{variables}\ru{переменных}
\en{cause only a~small change}\ru{вызывают лишь м\'{а}лое изменение}
\en{of a~deformed configuration}\ru{деформированной конфигурации},
\en{then}\ru{то}
\en{this equilibrium}\ru{это равновесие}
\en{is stable}\ru{устойчиво}.
%
\en{Or}\ru{Или}
\en{the~}\textbold{\en{energy approach}\ru{энергетический подход}}\::
\en{when}\ru{когда}
\en{a~loss of stability}\ru{потеря устойчивости}
\en{becomes}\ru{становится}
\en{energetically beneficial}\ru{энергетически выгодной},
\en{that is}\ru{то~есть}
\en{when}\ru{когда}
\en{it leads}\ru{она ведёт}
\en{to a~decrease}\ru{к~уменьшению}
\en{in energy}\ru{энергии},
\en{then}\ru{тогда}
\en{such a~stability loss}\ru{такая потеря устойчивости}
\en{really happens}\ru{действительно случается}.

\en{The mentioned approaches}\ru{Упомянутые подходы}
\en{draw}\ru{рисуют}
\en{a~motley picture}\ru{пёструю картину}.
%
\en{Yet}\ru{Но}
\en{it is pretty easy}\ru{её довольно просто}
\en{to~visualize}\ru{визуализировать}
\en{for}\ru{для}
\en{a~model}\ru{модели}
\en{with a~finite number}\ru{с~конечным числом}
\en{of~degrees o’freedom}\ru{степеней свободы}.

\en{Pretty}\ru{Весьма}
\en{abstract}\ru{абстрактны}
\en{and }\ru{и~}\en{universal}\ru{универсальны}\en{,}
\en{the~}\ru{уравнения }Lagrange’\en{s}\ru{а}\en{ equations}

\nopagebreak
\begin{equation}
A_{\hspace{-0.1ex}jk} \hspace{.2ex} \mathdotdotabove{q}_k \hspace{-0.1ex} = Q_{\hspace{-0.1ex}j} \bigl( q_k, p \bigr)
\hspace{.1ex} .
\end{equation}

\noindent
\en{Here}\ru{Тут}
$q_k$\:--- \en{generalized coordinates}\ru{обобщённые координаты},
$Q_{\hspace{-0.1ex}j}$\:--- \en{generalized forces}\ru{обобщённые силы},
......

...

...........



\en{\section{Classical problems with rods}}

\ru{\section{Классические проблемы со стержнями}}

\label{section:rodsclassicalstability}

\begin{otherlanguage}{russian}

Состояние перед варьированием описывается уравнениями нелинейной теории стержней Kirchhoff’а

...



\end{otherlanguage}

\en{\section{\inquotes{Tracking} loads}}

\ru{\section{\inquotes{Следящие} нагрузки}}

\label{section:trackingloads}

%% “dead” loads and “tracking” loads

\begin{otherlanguage}{russian}

В~проблемах устойчивости весьма вес\'{о}мо поведение нагрузки в~процессе деформирования.
Ведь в~уравнения входит вариация \textcolor{red}{(of what?)}~${\variation{\bm{q}}}$, она равна нулю лишь для \inquotes{\en{dead}\ru{мёртвых}} \en{loads}\ru{нагрузок}.
\textcolor{magenta}{Распространены}
\en{the }\inquotes{\en{tracking}\ru{следящие}}
\en{loads}\ru{нагрузки},
\en{which}\ru{которые}
\en{change}\ru{меняются}
\en{in a~certain way}\ru{определённым путём}
\en{along}\ru{вместе}
\en{with the deviations}\ru{с~отклонениями}
(\en{displacements}\ru{смещениями})
\en{of particles}\ru{частиц}.

\en{The}\ru{Статический подход} Euler’\ru{а}\en{s}\en{ static approach}
\en{to the~problem}\ru{к~проблеме}
\en{of~stability}\ru{устойчивости}


...



\end{otherlanguage}

\en{\section{The role of additional pliabilities}}

\ru{\section{Роль добавочных податливостей}}

\label{section:additionalpliabilities}

\begin{otherlanguage}{russian}

Для прямого \textcolor{magenta}{консольного} стержня, сжатого постоянной силой~$\bm{F}$ на~свободном конце, критическая нагрузка определяется формулой Euler’а

...



\end{otherlanguage}

\en{\section{Variational formulations}}

\ru{\section{Вариационные формулировки}}

\label{section:variationalformulationsforstability}

\begin{otherlanguage}{russian}

Во~всех разделах линейной теории упругости больш\'{у}ю роль играют вариационные постановки.
Среди прочего, они составляют основу метода конечных элементов как варианта метода Ritz’а.

Менее развиты вариационные постановки для проблем устойчивости.
Здесь получил популярность метод

...



\end{otherlanguage}

\en{\section{Nonconservative problems}}

\ru{\section{Неконсервативные задачи}}

\label{section:nonconservativeproblemsofstability}

\begin{otherlanguage}{russian}

В~уравнении динамики (...) матрица позиционных сил

...



\end{otherlanguage}

\en{\section{Case of multiple roots}}

\ru{\section{Случай кратных корней}}

\label{section:caseofmultipleroots}

\begin{otherlanguage}{russian}

Вернёмся к~проблеме устойчивости (...) в~случае циркуляционных сил.
Как уже отмечалось\textcolor{red}{(где??)}, критическая ситуация характеризуется

...



\end{otherlanguage}

\section*{\small \wordforbibliography}

\begin{changemargin}{\parindent}{0pt}
\fontsize{10}{12}\selectfont

\begin{otherlanguage}{russian}

Увлекательные вопросы устойчивости упругих систем освещены в~книгах ...

\end{otherlanguage}

\end{changemargin}
