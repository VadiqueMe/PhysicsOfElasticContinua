\en{\chapter{Stability}}

\ru{\chapter{Устойчивость}}

\thispagestyle{empty}

\label{chapter:stability}

\en{\section{Various approaches to the problem of stability}}

\ru{\section{Различные подходы к проблеме устойчивости}}

\begin{otherlanguage}{russian}

\lettrine[lines=2, findent=2pt, nindent=0pt]{С}{уществует} классическая, хорошо развитая теория устойчивости движения~\cite{merkin-stabilityintro}. По~Ляпунову, процесс устойчив, если м\'{а}лые начальные отклонения остаются м\'{а}лыми и~в~будущем. Это относится и~к~состоянию равновесия. Нужно рассмотреть динамику малых отклонений от равновесной конфигурации и~убедиться, что они не~растут. В~этом состоит \textbf{динамический подход} к~проблеме устойчивости, и он справедливо считается наиболее достоверным.

Однако в~задачах устойчивости равновесия упругих систем нашёл распространение иной подход, называемый \textbf{статическим} и связываемый с~именем Leonhard’а Euler’а. Здесь значения параметров, при которых уравнения статики для малых отклонений приобретают нетривиальное решение, считаются критическими. Иными словами, критическим считается то равновесное состояние, которое перестаёт быт изолированным,\:--- в~его окрестности появляется множество смежных равновесных форм. При этом подходе достаточно решить задачу на собственные числа.

Но есть и другие подходы. Например, \textbf{метод несовершенств}: если м\'{а}лые случайные изменения начальной формы, жёсткостей, нагрузок и другие приводят лишь к~м\'{а}лому изменению равновесной деформированной конфигурации, то имеем устойчивость. Отметим также \textbf{энергетический подход}: потеря устойчивости происходит, когда она становится энергетически выгодной, то~есть ведёт к~уменьшению энергии.

Перечисленные подходы составляют пёструю картину. Но в~ней нетрудно разобраться на~модели с~конечным числом степеней свободы. Больш\'{о}й общностью обладают

...



\end{otherlanguage}

\en{\section{Classical problems with rods}}

\ru{\section{Классические проблемы со стержнями}}

\begin{otherlanguage}{russian}

Состояние перед варьированием описывается уравнениями нелинейной теории стержней Kirchhoff’а

...



\end{otherlanguage}

\en{\section{\inquotes{Tracking} loads}}

\ru{\section{\inquotes{Следящие} нагрузки}}

%% “dead” loads and “tracking” loads

\begin{otherlanguage}{russian}

В~проблемах устойчивости весьма вес\'{о}мо поведение нагрузки при деформации. Ведь в~уравнения входит вариация~${\variation{\bm{q}}}$, она равна нулю лишь для \inquotes{мёртвых} нагрузок. Распространены \inquotes{следящие} нагрузки, то~есть определённым образом меняющиеся при смещениях частиц упругого тела. Статический подход Euler’а

...



\end{otherlanguage}

\en{\section{The role of additional yieldings}}

\ru{\section{Роль добавочных податливостей}}

\begin{otherlanguage}{russian}

Для прямого \textcolor{magenta}{консольного} стержня, сжатого постоянной силой~$\bm{F}$ на~свободном конце, критическая нагрузка определяется формулой Euler’а

...



\end{otherlanguage}

\en{\section{Variational formulations}}

\ru{\section{Вариационные формулировки}}

\begin{otherlanguage}{russian}

Во~всех разделах линейной теории упругости больш\'{у}ю роль играют вариационные постановки. Среди прочего, они составляют основу метода конечных элементов как варианта метода Ritz’а.

Менее развиты вариационные постановки для проблем устойчивости. Здесь получил популярность метод

...



\end{otherlanguage}

\en{\section{Nonconservative problems}}

\ru{\section{Неконсервативные задачи}}

\begin{otherlanguage}{russian}

В~уравнении динамики (...) матрица позиционных сил

...



\end{otherlanguage}

\en{\section{Case of multiple roots}}

\ru{\section{Случай кратных корней}}

\begin{otherlanguage}{russian}

Вернёмся к~проблеме устойчивости (...) в~случае циркуляционных сил. Как уже отмечалось\textcolor{red}{(где??)}, критическая ситуация характеризуется

...



\end{otherlanguage}

\section*{\small \wordforbibliography}

\begin{changemargin}{\parindent}{0pt}
\fontsize{10}{12}\selectfont

\begin{otherlanguage}{russian}

Увлекательные вопросы устойчивости упругих систем освещены в~книгах ...

\end{otherlanguage}

\end{changemargin}
