\en{\chapter{Stability}}

\ru{\chapter{Устойчивость}}

\thispagestyle{empty}

\label{chapter:stability}

\en{\section{Various approaches to the problem of stability}}

\ru{\section{Разные подходы к проблеме устойчивости}}

\en{\dropcap{T}{here’s}}\ru{\dropcap{Е}{сть}} \en{the }\en{well developed}\ru{весьма р\'{а}звитая} \en{theory of \foreignlanguage{russian}{Ляпунов}~(Lyapunov) stability}\ru{теория устойчивости по Ляпунову}.
\en{It says that}\ru{Она говорит, что} \en{if}\ru{если} \en{near}\ru{близкие} \en{initial deviations}\ru{начальные отклонения} \en{don’t increase in the future}\ru{не~увеличиваются в~будущем} \en{and}\ru{и} \en{stay near}\ru{остаются близкими} \en{forever}\ru{навсегда}, \en{then}\ru{то} \en{the process is stable}\ru{процесс устойчив}.
\en{This applies to the~condition of~equilibrium too.}\ru{Это относится и~к~состоянию равновесия тоже.}
\en{Consider}\ru{Рассмотреть} \en{the dynamics}\ru{динамику} \en{of small deviations}\ru{малых отклонений} \en{from an~equilibrium configuration}\ru{от равновесной конфигурации}\ru{,} \en{to make sure}\ru{чтобы убедиться}\ru{,} \en{that these deviations}\ru{что эти отклонения} \en{do not rise}\ru{не~растут}.
\en{This is}\ru{Так\'{о}в} \en{the~}\textbold{\en{dynamic approach}\ru{динамический подход}} \en{to the~problem of~stability}\ru{к~проблеме устойчивости}, \en{recognized as the~most reasonable}\ru{признаваемый самым обоснованным}.

\begin{otherlanguage}{russian}

\en{However}\ru{Однако}, \en{for problems}\ru{для~задач} \en{about}\ru{об} \en{the~stability of~equilibrium}\ru{устойчивости равновесия} \en{of~elastic systems}\ru{упругих систем}\en{,} \ru{обрёл популярность }\en{the~different}\ru{иной} \en{approach}\ru{подход}\en{ has gained popularity}, \en{called}\ru{называемый} \textbold{\en{static}\ru{статическим}} \en{and}\ru{и}~\en{associated}\ru{связываемый} \en{with }\ru{с~}\ru{именем }Leonhard\ru{’а} Euler’\en{s}\ru{а}\en{ name}.
Здесь значения параметров, при которых уравнения статики для малых отклонений получают \en{a~nontrivial solution}\ru{нетривиальное решение}, \en{are assumed to be}\ru{предполагаются} \inquotesx{\en{critical}\ru{критическими}}[.]
\en{In~other words}\ru{Иными словами}, \inquotes{критическим} считается такое \en{equilibrium condition}\ru{равновесное состояние}, которое перестаёт быть изолированным\:--- в~его окрестности появляется множество смежных равновесных форм (\inquotes{\en{indifferent equilibrium}\ru{безразличное равновесие}}).
\en{With this approach}\ru{С~этим подходом} \en{it’s enough}\ru{достаточно} \en{to~solve}\ru{решить} \en{the~eigenvalue problem}\ru{задачу на~собственные числа}.

\en{And}\ru{И} \en{there are more approaches}\ru{есть ещё подходы}.
\en{For example}\ru{Например}, \en{the~}\textbold{\en{imperfection method}\ru{метод несовершенств}}: если м\'{а}лые случайные изменения начальной формы, жёсткостей, нагрузок и \textcolor{magenta}{чего-нибудь ещё} приводят лишь к~м\'{а}лому изменению равновесной деформированной конфигурации, то равновесие устойчиво.
\en{Or}\ru{Или} \en{the~}\textbold{\en{energy approach}\ru{энергетический подход}}: потеря устойчивости происходит, когда она становится энергетически выгодной, то~есть ведёт к~уменьшению энергии.

\en{Mentioned}\ru{Упомянутые} \en{approaches}\ru{подходы} \en{draw}\ru{рисуют} \en{a~motley picture}\ru{пёструю картину}.
\en{Yet}\ru{Но} \en{that is pretty easy to~visualize}\ru{её довольно просто визуализировать} \en{for a~discrete model}\ru{для дискретной модели}\:--- \en{a~model}\ru{модели} \en{with finite number of~degrees of~freedom}\ru{с~конечным числом степеней свободы}.

Больш\'{о}й общностью обладают



...


\end{otherlanguage}

\en{\section{Classical problems with rods}}

\ru{\section{Классические проблемы со стержнями}}

\begin{otherlanguage}{russian}

Состояние перед варьированием описывается уравнениями нелинейной теории стержней Kirchhoff’а

...



\end{otherlanguage}

\en{\section{\inquotes{Tracking} loads}}

\ru{\section{\inquotes{Следящие} нагрузки}}

%% “dead” loads and “tracking” loads

\begin{otherlanguage}{russian}

В~проблемах устойчивости весьма вес\'{о}мо поведение нагрузки в~процессе деформирования.
Ведь в~уравнения входит вариация \textcolor{red}{(of what?)}~${\variation{\bm{q}}}$, она равна нулю лишь для \inquotes{\en{dead}\ru{мёртвых}} \en{loads}\ru{нагрузок}.
\textcolor{magenta}{Распространены} \inquotes{\en{tracking}\ru{следящие}} \en{loads}\ru{нагрузки}, \en{which}\ru{которые} \en{change in a~definite way}\ru{меняются определённым путём} \en{along with displacements of the particles}\ru{вместе со~смещениями частиц} \en{of an~elastic continuum}\ru{упругого контину\kern-0.11exума}.

\ru{Статический подход }Euler’\ru{а}\en{s}\en{ static approach} \en{to the~problem of~stability}\ru{к~проблеме устойчивости}

...



\end{otherlanguage}

\en{\section{The role of additional pliabilities}}

\ru{\section{Роль добавочных податливостей}}

\begin{otherlanguage}{russian}

Для прямого \textcolor{magenta}{консольного} стержня, сжатого постоянной силой~$\bm{F}$ на~свободном конце, критическая нагрузка определяется формулой Euler’а

...



\end{otherlanguage}

\en{\section{Variational formulations}}

\ru{\section{Вариационные формулировки}}

\begin{otherlanguage}{russian}

Во~всех разделах линейной теории упругости больш\'{у}ю роль играют вариационные постановки.
Среди прочего, они составляют основу метода конечных элементов как варианта метода Ritz’а.

Менее развиты вариационные постановки для проблем устойчивости.
Здесь получил популярность метод

...



\end{otherlanguage}

\en{\section{Nonconservative problems}}

\ru{\section{Неконсервативные задачи}}

\begin{otherlanguage}{russian}

В~уравнении динамики (...) матрица позиционных сил

...



\end{otherlanguage}

\en{\section{Case of multiple roots}}

\ru{\section{Случай кратных корней}}

\begin{otherlanguage}{russian}

Вернёмся к~проблеме устойчивости (...) в~случае циркуляционных сил.
Как уже отмечалось\textcolor{red}{(где??)}, критическая ситуация характеризуется

...



\end{otherlanguage}

\section*{\small \wordforbibliography}

\begin{changemargin}{\parindent}{0pt}
\fontsize{10}{12}\selectfont

\begin{otherlanguage}{russian}

Увлекательные вопросы устойчивости упругих систем освещены в~книгах ...

\end{otherlanguage}

\end{changemargin}
