\en{\chapter{Stability}}

\ru{\chapter{Устойчивость}}

\thispagestyle{empty}

\label{chapter:stability}

\en{\section{The various approaches to the problem of stability}}

\ru{\section{Разные подходы к проблеме устойчивости}}

\en{\dropcap{T}{here’s}}\ru{\dropcap{Е}{сть}}
\en{the }\en{well developed}\ru{весьма р\'{а}звитая}
\en{theory of stability}\ru{теория устойчивости}
\en{by }\foreignlanguage{russian}{Ляпунов}~(Lyapunov)\ru{по Ляпунову}.
\en{It says that}\ru{Она говорит, что}
\en{if}\ru{если}
\en{the initial deviations}\ru{начальные отклонения}%
\footnote{\en{From the~equilibrium configuration}\ru{От равновесной конфигурации}.}%
\en{are }\inquotesx{\en{close (small) enough}\ru{достаточно близкие (м\'{а}лые)}}[,]
\en{and afterwards}\ru{и~впоследствии}
\en{they}\ru{они}
\en{don’t increase}\ru{не~растут},
\en{remaining}\ru{оставаясь}
\inquotesx{\en{close enough}\ru{достаточно близкими}}
\en{forever}\ru{навсегда},
\en{then}\ru{то}
\en{the process is stable}\ru{процесс устойчив}.
\en{This applies}\ru{Это применимо}
\en{to the~condition of~equilibrium}\ru{и~к~состоянию равновесия}
\en{too}.
%\en{Considering}\ru{Рассматривая}
%\en{the dynamics}\ru{динамику}
%\en{of the small deviations}\ru{малых отклонений}
%....
\en{This is}\ru{Так\'{о}в}
\en{the~}\textbold{\en{dynamic approach}\ru{динамический подход}}
\en{to the~problem of~stability}\ru{к~проблеме устойчивости},
\en{and this approach is}\ru{и~этот подход}
\en{the~most reasonable}\ru{самый обоснованный}.

\en{However}\ru{Однако},
\en{for problems}\ru{для~задач}
\en{about}\ru{об}
\en{the~stability of the~equilibrium}\ru{устойчивости равновесия},
%\en{of~elastic systems}\ru{упругих систем}\en{,}
\ru{обрёл популярность }\en{another}\ru{другой} \en{approach}\ru{подход}\en{ has gained popularity}.
%
\en{It is called}\ru{Он называется}
\en{the }\textbold{\en{static}\ru{статическим}}
\en{approach}\ru{подходом},
\en{the}\ru{имя} Leonhard\ru{’а} Euler’\en{s}\ru{а}\en{ name}
\en{is associated}\ru{ассоциируется}
\en{with it}\ru{с~ним}.
%
\en{When}\ru{Когда}
\en{the equations of statics}\ru{уравнения статики}
\en{give}\ru{дают}
\en{a~nontrivial solution}\ru{нетривиальное решение}
\en{for the small}\ru{для м\'{а}лых}
\en{deviations-displacements}\ru{отклонений-смещений},
\en{then}\ru{тогда}
\en{the values of parameters}\ru{тогда значения параметров}
\en{are assumed to be}\ru{предполагаются}
\inquotesx{\en{critical}\ru{критическими}}[.]
%
\en{In~other words}\ru{Иными словами},
\en{if}\ru{если}
\en{the~equilibrium}\ru{равновесие}
\en{isn’t isolated}\ru{не~изолированное},
\en{then}\ru{то}
\en{it’s considered}\ru{оно считается}
\en{the }\inquotesx{\en{critical}\ru{критическим}}[.]
%
\en{At the~same}\ru{В~то~же}
\en{time}\ru{время},
\en{many}\ru{много}
\en{adjacent forms of the equilibrium}\ru{смежных форм равновесия}
\en{appear}\ru{появляется}%
\footnote{%
   \en{This is known}\ru{Это известно}
   \en{as}\ru{как}
   \en{the }\inquotesx{\en{indifferent equilibrium}\ru{безразличное равновесие}}[.]%
} % end of footnote
\en{Using this approach}\ru{Используя этот подход},
\en{it’s enough}\ru{достаточно}
\en{to~solve}\ru{решить}
\en{the~problem}\ru{задачу}
\en{about the~eigenvalues}\ru{о~собственных числах}.

%%\en{in its neighborhood}\ru{в~его окрестности}

\en{There are also}\ru{Есть ещё}
\en{more}\ru{больше}
\en{approaches}\ru{подходов}.
%
\en{For example}\ru{Для примера},
\en{the~}\textbold{\en{method of~imperfections}\ru{метод несовершенств}}:
\en{if}\ru{если}
\en{small random changes}\ru{м\'{а}лые случайные изменения}
\en{of the initial shape}\ru{начального \'{о}блика},
\en{the stiffnesses}\ru{жёсткостей},
\en{the loads}\ru{нагрузок}
\en{and other}\ru{и~других}
\en{variables}\ru{переменных}
\en{cause only a~small change}\ru{вызывают лишь м\'{а}лое изменение}
\en{of a~deformed configuration}\ru{деформированной конфигурации},
\en{then}\ru{то}
\en{this equilibrium}\ru{это равновесие}
\en{is stable}\ru{устойчиво}.
%
\en{Or}\ru{Или}
\en{the~}\textbold{\en{energy approach}\ru{энергетический подход}}:
\en{when}\ru{когда}
\en{a~loss of stability}\ru{потеря устойчивости}
\en{becomes}\ru{становится}
\en{energetically beneficial}\ru{энергетически выгодной},
\en{that is}\ru{то~есть}
\en{when}\ru{когда}
\en{it leads}\ru{она ведёт}
\en{to a~decrease}\ru{к~уменьшению}
\en{of energy}\ru{энергии},
\en{then}\ru{тогда}
\en{this loss of stability}\ru{эта потеря устойчивости}
\en{happens}\ru{случается}.

\en{The mentioned approaches}\ru{Упомянутые подходы}
\en{draw}\ru{рисуют}
\en{a~motley picture}\ru{пёструю картину}.
%
\en{Yet}\ru{Но}
\en{it is pretty easy}\ru{её довольно просто}
\en{to~visualize}\ru{визуализировать}
\en{for}\ru{для}
\en{a~model}\ru{модели}
\en{with a~finite number}\ru{с~конечным числом}
\en{of~degrees of~freedom}\ru{степеней свободы}.

\begin{otherlanguage}{russian}

Больш\'{о}й общностью обладают

...

\end{otherlanguage}

\en{\section{Classical problems with rods}}

\ru{\section{Классические проблемы со стержнями}}

\begin{otherlanguage}{russian}

Состояние перед варьированием описывается уравнениями нелинейной теории стержней Kirchhoff’а

...



\end{otherlanguage}

\en{\section{\inquotes{Tracking} loads}}

\ru{\section{\inquotes{Следящие} нагрузки}}

%% “dead” loads and “tracking” loads

\begin{otherlanguage}{russian}

В~проблемах устойчивости весьма вес\'{о}мо поведение нагрузки в~процессе деформирования.
Ведь в~уравнения входит вариация \textcolor{red}{(of what?)}~${\variation{\bm{q}}}$, она равна нулю лишь для \inquotes{\en{dead}\ru{мёртвых}} \en{loads}\ru{нагрузок}.
\textcolor{magenta}{Распространены}
\en{the }\inquotes{\en{tracking}\ru{следящие}}
\en{loads}\ru{нагрузки},
\en{which}\ru{которые}
\en{change}\ru{меняются}
\en{in a~certain way}\ru{определённым путём}
\en{along}\ru{вместе}
\en{with the deviations}\ru{с~отклонениями}
(\en{displacements}\ru{смещениями})
\en{of particles}\ru{частиц}.

\en{The}\ru{Статический подход} Euler’\ru{а}\en{s}\en{ static approach}
\en{to the~problem}\ru{к~проблеме}
\en{of~stability}\ru{устойчивости}


...



\end{otherlanguage}

\en{\section{The role of additional pliabilities}}

\ru{\section{Роль добавочных податливостей}}

\begin{otherlanguage}{russian}

Для прямого \textcolor{magenta}{консольного} стержня, сжатого постоянной силой~$\bm{F}$ на~свободном конце, критическая нагрузка определяется формулой Euler’а

...



\end{otherlanguage}

\en{\section{Variational formulations}}

\ru{\section{Вариационные формулировки}}

\begin{otherlanguage}{russian}

Во~всех разделах линейной теории упругости больш\'{у}ю роль играют вариационные постановки.
Среди прочего, они составляют основу метода конечных элементов как варианта метода Ritz’а.

Менее развиты вариационные постановки для проблем устойчивости.
Здесь получил популярность метод

...



\end{otherlanguage}

\en{\section{Nonconservative problems}}

\ru{\section{Неконсервативные задачи}}

\begin{otherlanguage}{russian}

В~уравнении динамики (...) матрица позиционных сил

...



\end{otherlanguage}

\en{\section{Case of multiple roots}}

\ru{\section{Случай кратных корней}}

\begin{otherlanguage}{russian}

Вернёмся к~проблеме устойчивости (...) в~случае циркуляционных сил.
Как уже отмечалось\textcolor{red}{(где??)}, критическая ситуация характеризуется

...



\end{otherlanguage}

\section*{\small \wordforbibliography}

\begin{changemargin}{\parindent}{0pt}
\fontsize{10}{12}\selectfont

\begin{otherlanguage}{russian}

Увлекательные вопросы устойчивости упругих систем освещены в~книгах ...

\end{otherlanguage}

\end{changemargin}
