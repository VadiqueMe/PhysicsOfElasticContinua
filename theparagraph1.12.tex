\en{\section{The variations}}

\ru{\section{Вариации}}

\label{para:calculusofvariations}






\begin{otherlanguage}{russian}

\en{Further}\ru{Далее}\en{,}
\en{we will often use}\ru{мы будем часто использовать}
\en{the operation of varying}\ru{операцию варьирования}.
\en{It is similar to differentiation}\ru{Она похожа на дифференцирование}.

Не~отсылая читателя
к~книгам о~вариационном исчислении
(calculus of~variations),
ограничимся представлениями
\en{about a~variation}\ru{о~вариации}~$\variation{x}$
величины~$x$
как о~задаваемом нами
бесконечно малом приращении,
совместимом с~ограничениями
(связями, constraints).
\en{If}\ru{Если}
ограничений для $x$ нет,
\en{}\ru{то}
${\variation{x}}$
произвольна
(случайна).
\en{But when}\ru{Но когда}
${x \!=\! x(y\hspace{-0.1ex})}$\:---
функция независимого аргумента~$y$,
\en{then}\ru{тогда}
${\variation{x} = x'\hspace{-0.25ex}(y\hspace{-0.1ex}) \hspace{.1ex} \variation{y}}$.

\en{Writings with variations}\ru{З\'{а}писи с~вариациями}
\en{have}\ru{имеют}
\en{the~same features}\ru{те~же особенности}\ru{,} \en{as}\ru{как и} \en{writings}\ru{з\'{а}писи} \en{with differentials}\ru{с~дифференциалами}.
Если, например, ${\variation{x}}$ \en{and}\ru{и}~${\variation{y}}$\en{ are}\ru{\:---} \en{variations of}\ru{вариации}~$x$ \en{and}\ru{и}~$y$, $u$ \en{and}\ru{и}~$v$\en{ are}\ru{\:---} кон\'{е}чные величины, то пишем ${u \variation{x} + v \variation{y} = \variation{w}}$, а~не~$w$\:--- даже когда ${\variation{w}}$ не~является вариацией величины~$w$; в~этом случае ${\variation{w}}$ это единое обозначение.
\en{Surely}\ru{Разумеется}, \en{if}\ru{если} ${u \narroweq u(x,y)}$, ${v \narroweq v(x,y)}$ \en{and}\ru{и}~${\partial_x v = \partial_y u}$ (${\hspace{.16ex}\frac{\partial}{\partial x} v = \frac{\partial}{\partial y} u\hspace{.16ex}}$), \en{then}\ru{то} \en{the~sum}\ru{сумма}~${\variation{w}}$ \en{will be a~variation}\ru{будет вариацией} \en{of some}\ru{н\'{е}кой}~$w$.

Варьируя тождество~\eqref{orthogonalityofrotationtensor}, получим ${\variation{\rotationtensor} \hspace{-0.2ex} \dotp \rotationtensor^{\T} \hspace{-0.2ex} = - \hspace{.2ex} \rotationtensor \dotp \variation{\rotationtensor}^{\T}\!}$.
Этот тензор антисимметричен, и~потому выражается через свой сопутствующий вектор~${\varvector{o}}$ как~${\variation{\rotationtensor} \hspace{-0.1ex} \dotp \rotationtensor^{\T} \hspace{-0.3ex} = \varvector{o} \hspace{-0.2ex} \times \hspace{-0.2ex} \UnitDyad}$.
Приходим к~соотношениям

\nopagebreak\vspace{-0.5em}\begin{equation}
\variation{\rotationtensor} \hspace{-0.1ex} = \varvector{o} \hspace{-0.1ex} \times \hspace{-0.1ex} \rotationtensor , \:\:
\varvector{o} = - \hspace{.2ex} \scalebox{.93}{$ \displaystyle\onehalf $} \hspace{-0.1ex} \Bigl( \hspace{-0.1ex} \variation{\rotationtensor} \hspace{-0.1ex} \dotp \rotationtensor^{\T} \Bigr)_{\hspace{-0.25em}\Xcompanion}
\hspace{-0.1ex} ,
\end{equation}

\vspace{-0.5em}\noindent
аналогичным~\eqref{angularvelocityvector}.
Вектор бесконечно малого поворота~${\varvector{o}}$ это не~\inquotesx{вариация $\bm{\mathrm{o}}$}[,] но единый символ (в~отличие от~${\variation{\rotationtensor}}$).

Малый поворот
определяется
вектором~${\varvector{o}}$,
но конечный поворот
тоже возможно представить как вектор
...
can also be represented as a vector

.............




\end{otherlanguage}

