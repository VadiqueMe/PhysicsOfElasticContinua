\en{\section{Variations}}

\ru{\section{Вариации}}

\label{para:calculusofvariations}

\en{Further}\ru{Дальше}\ru{,}
\en{we will}\ru{мы будем}
\en{pretty often}\ru{довольно часто}
\en{use}\ru{использовать}
\en{the operation of varying}\ru{операцию варьирования}.
\en{It is similar}\ru{Она похожа}
\en{to the differentiation}\ru{на дифференцирование}.

\en{The variations}\ru{Вариации}
\en{are seen as}\ru{видятся как}
\en{the infinitesimal displacements}\ru{бесконечно малые смещения},
\en{compatible with the constraints}\ru{совместимые с~ограничениями}\ru{( \inquotes{связями} )}.
\en{If}\ru{Если}
\en{there are no restrictions}\ru{ограничений}
\en{for}\ru{для}
\en{the variable}\ru{переменной} $x$\ru{ нет},
\en{then}\ru{то}
\en{the variations}\ru{вариации}
${\variation{x}}$
\en{are completely random}\ru{совершенно случайны}.
\en{But when}\ru{Но когда}
\begin{equation*}
x \!=\! x(y\hspace{-0.1ex})
\end{equation*}
\en{is}\ru{это}
\en{the function}\ru{функция}
\en{of the independent argument}\ru{независимого аргумента}~$y$,
\en{then}\ru{тогда}
\begin{equation*}
\variation{x} = x'\hspace{-0.25ex}(y\hspace{-0.1ex}) \hspace{.1ex} \variation{y} .
\end{equation*}

\en{Variations}\ru{Вариации}\en{ are}
\en{similar}\ru{похожи}
\en{to differentials}\ru{на~дифференциалы}.
\en{As example}\ru{Как пример},
\en{if}\ru{если}
${\variation{x}}$
\en{and}\ru{и}~${\variation{y}}$
\en{are}\ru{это}
\en{variations of}\ru{вариации}~$x$ \en{and}\ru{и}~$y$,
$u$ \en{and}\ru{и}~$v$\en{ are}\ru{\:---}
\en{the finite}\ru{кон\'{е}чные}
\en{values}\ru{значения},
\en{then we write}\ru{то мы пишем}
${u \variation{x} + v \variation{y} = \variation{w}}$
\en{even if}\ru{даже если}
${\variation{w}}$
\en{is not a~variation of}\ru{это не вариация}~$w$.

\en{In this case}\ru{В~этом случае}~${\variation{w}}$
\en{is a~single symbol}\ru{это одиночный символ}.
\en{Surely}\ru{Разумеется}\ru{,}
\en{if}\ru{если}
${u \narroweq u(x,y)}$,
${v \narroweq v(x,y)}$
\en{and}\ru{и}~${\partial_x v = \partial_y u}$
(${ \hspace{.16ex}
   \frac{\partial}{\partial x} v = \frac{\partial}{\partial y} u
\hspace{.16ex} }$),
\en{then}\ru{то}
\en{the~sum}\ru{сумма}~${\variation{w} = u \variation{x} + v \variation{y}}$
\en{will be a~variation}\ru{будет вариацией}
\en{of some}\ru{н\'{е}кой}~$w$.

\en{Varying}\ru{Варьируя}
\en{the identity}\ru{тождество}~\eqref{orthogonalityofrotationtensor},
\en{we get}\ru{мы получаем}

\begin{equation*}
\variation{\rotationtensor} \hspace{-0.2ex} \dotp \rotationtensor^{\T} \hspace{-0.2ex} = - \hspace{.2ex} \rotationtensor \dotp \variation{\rotationtensor}^{\T}\!
.
\end{equation*}

\en{This tensor}\ru{Этот тензор}
\en{is antisymmetric}\ru{антисимметричен},
\en{and thus}\ru{и~потому}
\en{is representable}\ru{представляется}
\en{via}\ru{через}
\en{its}\ru{свой}
\en{companion}\ru{сопутствующий}
\en{pseudovector}\ru{псевдовектор}~${\varvector{o}}$
\en{as}\ru{как}

\begin{equation*}
\variation{\rotationtensor} \hspace{-0.1ex} \dotp \rotationtensor^{\T} \hspace{-0.3ex} = \varvector{o} \hspace{-0.2ex} \times \hspace{-0.2ex} \UnitDyad
.
\end{equation*}

\en{We have}\ru{Мы имеем}
\en{the following}\ru{следующие}
\en{relations}\ru{отношения}

\nopagebreak\vspace{-0.5em}
\begin{equation}
\variation{\rotationtensor}
\hspace{-0.1ex} =
\varvector{o}
\hspace{-0.1ex} \times \hspace{-0.1ex}
\rotationtensor
,
\:\:
\varvector{o}
=
- \hspace{.2ex} \scalebox{.93}{$ \displaystyle\onehalf $}
\hspace{-0.1ex} \Bigl( \hspace{-0.1ex}
\variation{\rotationtensor} \hspace{-0.1ex} \dotp \rotationtensor^{\T}
\Bigr)_{\hspace{-0.25em}\Xcompanion}
\hspace{-0.1ex} ,
\end{equation}

\vspace{-0.5em}\noindent
\en{similar}\ru{похожие}
\en{to}\ru{на}~\eqref{angularvelocityvector}.
\en{Vector}\ru{Вектор}~${\varvector{o}}$
\en{of an~infinitesimal rotation}\ru{бесконечно малого поворота}
\en{is not}\ru{это не}
\inquotesx{\en{a~variation of}\ru{вариация}~$\bm{\mathrm{o}}$},
\en{but}\ru{но}
\en{a~single}\ru{один}
\en{symbol}\ru{символ}.

\en{An~infinitesimal rotation}\ru{Бесконечно малый поворот}
\en{is defined}\ru{определяется}
\en{by vector}\ru{вектором}~${\varvector{o}}$,
\en{but}\ru{но}
\en{a~finite rotation}\ru{конечный поворот}
\en{is also}\ru{тоже}
\en{possible to represent}\ru{возможно представить}
\en{as a~vector}\ru{как вектор}

...

