\documentclass[tikz,margin=5]{standalone}

\usepackage{tikz}
\usepackage{tikz-3dplot} % needs tikz-3dplot.sty in same folder
\usetikzlibrary{calc}
\usetikzlibrary{arrows, arrows.meta}
\usetikzlibrary{angles, quotes}
\usetikzlibrary{decorations.markings}
\usetikzlibrary{decorations.pathreplacing}
\usetikzlibrary{decorations.text}

\usepackage{amsmath}
\usepackage{bm}

\usepackage{verbatim}

\makeatletter
\def\mathcolor#1#{\@mathcolor{#1}}
\def\@mathcolor#1#2#3{%
	\protect\leavevmode
	\begingroup\color#1{#2}#3\endgroup
}
\makeatother

\newcommand\constant{\scalebox{0.92}[0.98]{\ensuremath{\mathsf{constant}}}}
\newcommand\boldconstant{\mathcolor{green!50!black}{\boldmath \constant}}

\newcommand\expminusone{\hspace{.16ex}\scalebox{0.8}[1]{\hbox{--}}\raisemath{.1ex}{1}}

\makeatletter
\newcommand{\raisemath}[1]{\mathpalette{\raisem@th{#1}}}
\newcommand{\raisem@th}[3]{\raisebox{#1}{$#2#3$}}
\makeatother

\newcommand\locationvector{\bm{r}}

\newcommand\differentialindex[1]{{\hspace{-0.1ex}\scalebox{0.63}{$\mathcolor{black!70}{\partial}$} #1}}

\begin{document}

%%\begin{center}

\def\cameraanglesecond{66} % 66 % 90 for plane of meridian, 0 for plane of parallèle
\def\cameraanglefirst{42} % 42 % 90 for plane of parallèle or meridian

\tdplotsetmaincoords{\cameraanglefirst}{\cameraanglesecond} % orientation of camera

\pgfmathsetmacro{\revolutionbeginangle}{-82} % 0
\pgfmathsetmacro{\revolutionendangle}{199} % 360

\def\enlargetext{1.5}
\def\enlargeless{0.8}
\pgfmathsetmacro{\enlargelesstext}{\enlargeless * \enlargetext}

\tikzset{%
	show curve controls/.style={
		postaction={
			decoration={
				show path construction,
				curveto code={
					\fill [black, opacity=.5]
						(\tikzinputsegmentfirst) circle (.4ex)
						(\tikzinputsegmentlast) circle (.4ex) ;
					\draw [black, opacity=.5, line cap=round, dash pattern=on 0pt off 1.6\pgflinewidth]
						(\tikzinputsegmentfirst) -- (\tikzinputsegmentsupporta)
						(\tikzinputsegmentlast) -- (\tikzinputsegmentsupportb) ;
					\fill [magenta, opacity=.5, line cap=round, dash pattern=on 0pt off 1.6\pgflinewidth]
						(\tikzinputsegmentsupporta) circle [radius=.4ex]
						(\tikzinputsegmentsupportb) circle [radius=.4ex] ;
				}
			},
			decorate
}	}	}

\tikzset{pics/boldmeridianfirst/.style={code={

	% meine ebene Kurve, first part
	\draw [ line width=\ebeneKurveThickness, color=\ebeneKurveColor, opacity=\ebeneKurveOpacity
		, line cap=round
		, tdplot_rotated_coords
		%%, show curve controls
		]
		( 0, -2, 2.25 )
		.. controls ( 0, -1.5, 1.75 ) and ( 0, -0.5, 1.7 ) ..
		( 0, 0, 1.7 )
		.. controls ( 0, 0.5, 1.7 ) and ( 0, 1.5, 1.5 ) ..
		( 0, 3, 2.7 ) ;
		%%{ [turn] ( 0, -1, 0 ) coordinate(tangentm1) ( 0, 0, 0 ) coordinate(tangent0) ( 0, 1, 0 ) coordinate(tangentp1) } ;

}}}

\tikzset{pics/boldmeridiansecond/.style={code={

	% meine ebene Kurve, 2nd part
	\draw [ line width=\ebeneKurveThickness, color=\ebeneKurveColor, opacity=\ebeneKurveOpacity
		, line cap=round
		, tdplot_rotated_coords
		, postaction={decorate, decoration = {text along path,
			text align = {left indent = {0.8\dimexpr\pgfdecoratedpathlength\relax}},
			text format delimiters = {<}{>},
			text = {<\normalsize>{\scalebox{\enlargelesstext}{$\varphi = \constant$}}},
			raise = 1.2ex}}
		]
		( 0, 3, 2.7 )
		.. controls ( 0, 6, 5.1 ) and ( 0, 8, 7 ) ..
		( 0, 15, 6 ) ;

}}}

\tikzset{pics/dottedmeridian/.style={code={

	% meine ebene Kurve, dotted
	\draw [ line width=\ebeneKurveThickness, color=\ebeneKurveColor, opacity=\ebeneKurveOpacity
		, line cap=round, dash pattern=on 0pt off 1.5\pgflinewidth
		, tdplot_rotated_coords
		%%, show curve controls
		]
		( 0, -2, 2.25 )
		.. controls ( 0, -1.5, 1.75 ) and ( 0, -0.5, 1.7 ) ..
		( 0, 0, 1.7 )
		.. controls ( 0, 0.5, 1.7 ) and ( 0, 1.5, 1.5 ) ..
		( 0, 3, 2.7 )
		.. controls ( 0, 6, 5.1 ) and ( 0, 8, 7 ) ..
		( 0, 15, 6 );

}}}

\tikzset{pics/planeofmeridian/.style={code={

	\def\meridianplaneheight{7.7}
	\def\meridianxibegin{-2}
	\def\meridianxiend{15}

	\fill [color=yellow!40!white, opacity=.5]
		( 0, \meridianxibegin, 0 )
		-- ( 0, \meridianxiend, 0 )
		-- ( 0, \meridianxiend, \meridianplaneheight )
		-- ( 0, \meridianxibegin, \meridianplaneheight ) ;

}}}

\pgfmathsetmacro{\rotationanglefrom}{\revolutionbeginangle - 0.5}
\pgfmathsetmacro{\rotationangleto}{\revolutionendangle + 0.5}
\pgfmathsetmacro{\stepsofrotation}{ abs( \revolutionendangle - \revolutionbeginangle ) + 2 }
\pgfmathsetmacro{\stepsofrotationfirst}{ abs( \revolutionbeginangle ) + 1 }
\pgfmathsetmacro{\stepsofrotationsecond}{ abs( \revolutionendangle ) + 1 }

\tikzset{pics/paralleleBefore/.style={code={

	\draw [ line width=\paralleleThickness, color=\paralleleColor
		, line cap=round
		, domain=\rotationanglefrom:0 ]
		plot [ samples=\stepsofrotationfirst ] ( { \paralleleRadius*sin(\x) }, \paralleleXi, { \paralleleRadius*cos(\x) } ) ;

}}}

%%\tikzset{pics/paralleleAfter/.style={code={

	%%\draw [ line width=\paralleleThickness, color=\paralleleColor
		%%, line cap=round
		%%, domain=0:\rotationangleto ]
		%%plot [ samples=\stepsofrotationsecond ] ( { \paralleleRadius*sin(\x) }, \paralleleXi, { \paralleleRadius*cos(\x) } ) ;

%%}}}

\tikzset{pics/paralleleConstCoodrinate/.style={code={

	\draw [ line width=\paralleleThickness, color=\paralleleColor
		, line cap=round
		, domain=0:\rotationangleto
		, postaction={decorate, decoration = {text along path,
			text align = {left indent = {0.38\dimexpr\pgfdecoratedpathlength\relax}},
			text format delimiters = {<}{>},
			text = {<\normalsize>{\scalebox{\enlargelesstext}{$s = \constant$}}},
			%%reverse path,
			raise = 1.1ex}}
		]
		plot [ samples=\stepsofrotationsecond ] ( { \paralleleRadius*sin(\x) }, \paralleleXi, { \paralleleRadius*cos(\x) } ) ;

}}}

\tikzset{pics/dottedparalleleBefore/.style={code={

	\draw [ line width=\paralleleThickness, color=\paralleleColor
		, line cap=round, dash pattern=on 0pt off 1.5\pgflinewidth
		, domain=\rotationanglefrom:0 ]
		plot [ samples=\stepsofrotationfirst ] ( { \paralleleRadius*sin(\x) }, \paralleleXi, { \paralleleRadius*cos(\x) } ) ;

}}}

\tikzset{pics/dottedparalleleAfter/.style={code={

	\draw [ line width=\paralleleThickness, color=\paralleleColor
		, line cap=round, dash pattern=on 0pt off 1.5\pgflinewidth
		, domain=0:\rotationangleto ]
		plot [ samples=\stepsofrotationsecond ] ( { \paralleleRadius*sin(\x) }, \paralleleXi, { \paralleleRadius*cos(\x) } ) ;

}}}

\tikzset{pics/parallelesBefore/.style={code={

	% parallèles before 0°
	\def\paralleleColor{lime!88!black}
	\def\paralleleThickness{1.2pt}
	\def\paralleleRadius{2.25}
	\def\paralleleXi{-2}
	\pic (i) {dottedparalleleBefore} ;
	\def\paralleleRadius{6}
	\def\paralleleXi{15}
	\pic (i) {dottedparalleleBefore} ;
	\def\paralleleColor{black}
	\def\paralleleRadius{2.7}
	\def\paralleleXi{3}
	\pic (i) {paralleleBefore} ;

}}}

\tikzset{pics/parallelesAfter/.style={code={

	% parallèles after 0°
	\def\paralleleColor{lime!88!black}
	\def\paralleleThickness{1.2pt}
	\def\paralleleRadius{2.25}
	\def\paralleleXi{-2}
	\pic (i) {dottedparalleleAfter} ;
	\def\paralleleRadius{6}
	\def\paralleleXi{15}
	\pic (i) {dottedparalleleAfter} ;
	\def\paralleleColor{black}
	\def\paralleleRadius{2.7}
	\def\paralleleXi{3}
	\pic (i) {paralleleConstCoodrinate} ;

}}}

\tikzset{pics/axisofrotation/.style={code={

	\draw [ line width=0.5pt, -{Stealth[round, length=2.5mm, width=2mm]}, blue
		, line cap=round, dash pattern=on 2cm off 2pt on \the\pgflinewidth off 2pt ]
		( 0, -3.5, 0 ) -- ( 0, 18.5, 0 )
	node [ pos=0.98, above, inner sep=0pt, outer sep=6pt ]
		{\scalebox{\enlargetext}{$ \xi $}}
	;

}}}

\tikzset{pics/axisofradius/.style={code={

	\ifthenelse{\NOT \cameraanglefirst = 0}
	{%
	\draw [ line width=0.5pt, -{Stealth[round, length=2.5mm, width=2mm]}, blue, line cap=round ]
		( 0, \originalxicoordinate, -5 ) -- ( 0, \originalxicoordinate, 11 )
	node [ pos=0.93, above left, inner sep=0pt, outer sep=5.5pt ]
		{\scalebox{\enlargetext}{$ \rho $}}
	;
	} {}

}}}

\tikzset{pics/axisofangleandziaxis/.style={code={

	\tdplotsetrotatedcoords{0}{\phiAngle}{0} ;
	\draw [ line width=0.5pt, -{Stealth[round, length=2.5mm, width=2mm]}, blue, line cap=round
		, tdplot_rotated_coords ]
		( 0, \originalxicoordinate, 0 ) -- ( 0, \originalxicoordinate, 6 )
	node [ pos=0.85, above left, shape=circle, inner sep=0pt, outer sep=16pt ]
		{\scalebox{\enlargetext}{$ y $}} ;

	\pgfmathsetmacro{\ziaxisAngle}{\phiAngle + 90}
	\tdplotsetrotatedcoords{0}{\ziaxisAngle}{0} ;
	\draw [ line width=0.5pt, -{Stealth[round, length=2.5mm, width=2mm]}, blue, line cap=round
		, tdplot_rotated_coords ]
		( 0, \originalxicoordinate, 0 ) -- ( 0, \originalxicoordinate, 4 )
	node [ pos=0.8, below, shape=circle, inner sep=0pt, outer sep=7pt ]
		{\scalebox{\enlargetext}{$ z $}} ;

}}}

\begin{tikzpicture}[scale=1
	, tdplot_main_coords % use 3dplot
	]

	%
	\def\generateSurface{1}
	%
	\def\surfaceAngleStep{2}

	\ifthenelse{\generateSurface = 1}
	{%
	\def\ebeneKurveThickness{2.5pt}
	\def\ebeneKurveColor{lime!66!yellow}
	\def\ebeneKurveOpacity{0.5}

	\pgfmathsetmacro{\loopbeginangle}{\revolutionbeginangle}
	\pgfmathsetmacro{\loopendangle}{\revolutionendangle}

	\pgfmathsetmacro{\loopstepangle}{\loopbeginangle + \surfaceAngleStep}
	\foreach \angle in {\loopbeginangle, \loopstepangle, ..., 0} {
		\tdplotsetrotatedcoords{0}{\angle}{0} ;
		\pic (rotated) {dottedmeridian} ;
	}
	} {}

	\def\ebeneKurveColor{lime!88!black}
	\def\ebeneKurveThickness{1.2pt}
	\def\ebeneKurveOpacity{1}
	\tdplotsetrotatedcoords{0}{\revolutionbeginangle}{0} ;
	\pic (begin) {dottedmeridian} ;

	\pic (b) {parallelesBefore} ;

	\def\bezierpointbeginy{3} \def\bezierpointendy{6}
	\def\bezierpointbeginz{2.7} \def\bezierpointendz{5.1}
	\pgfmathsetmacro{\bezierpointlengthy}{\bezierpointendy - \bezierpointbeginy}
	\pgfmathsetmacro{\bezierpointlengthz}{\bezierpointendz - \bezierpointbeginz}

	\coordinate (pointOfInterest) at ( 0, \bezierpointbeginy, \bezierpointbeginz ) ;

	\pgfmathsetmacro{\tangentlengthy}{0.8 * \bezierpointlengthy}
	\pgfmathsetmacro{\tangentlengthz}{\bezierpointlengthz * \tangentlengthy / \bezierpointlengthy}
	\pgfmathsetmacro{\tangentToMeridianY}{\bezierpointbeginy + \tangentlengthy}
	\pgfmathsetmacro{\tangentToMeridianZ}{\bezierpointbeginz + \tangentlengthz}

	\coordinate (tangentToMeridian) at ( 0, \tangentToMeridianY, \tangentToMeridianZ ) ;

	\pgfmathsetmacro{\firstlinelengthy}{1.6 * \bezierpointlengthy}
	\pgfmathsetmacro{\firstlinelengthz}{\bezierpointlengthz * \firstlinelengthy / \bezierpointlengthy}

	% tangent plane (1st part)
	\ifthenelse{\NOT \cameraanglefirst = 90 \OR \NOT \cameraanglesecond = 90}
	{%
	\pgfmathsetmacro{\secondlinelengthminus}{-1.25}

	\fill [color=cyan!33!white, opacity=.25]
		(pointOfInterest) { [rounded corners=0.5cm] --
		( \secondlinelengthminus, \bezierpointbeginy, \bezierpointbeginz ) } --
		( 0, \bezierpointbeginy + \firstlinelengthy, \bezierpointbeginz + \firstlinelengthz )
		-- cycle ;
	} {}

	% plane of meridian (meridional plane)
	\pic (m) {planeofmeridian} ;

	\ifthenelse{\generateSurface = 1}
	{%
	\def\ebeneKurveThickness{2.5pt}
	\def\ebeneKurveColor{lime!66!yellow}
	\def\ebeneKurveOpacity{0.5}

	\pgfmathsetmacro{\loopstepangle}{\surfaceAngleStep}
	\foreach \angle in {0, \loopstepangle, ..., \loopendangle} {
		\tdplotsetrotatedcoords{0}{\angle}{0} ;
		\pic (rotated) {dottedmeridian} ;
	}
	} {}

	\def\ebeneKurveColor{lime!88!black}
	\def\ebeneKurveThickness{1.2pt}
	\def\ebeneKurveOpacity{1}
	\tdplotsetrotatedcoords{0}{\revolutionendangle}{0} ;
	\pic (end) {dottedmeridian} ;

	\pic (a) {parallelesAfter} ;

	\def\originalxicoordinate{0.66}
	\def\phiAngle{-42}

	\pic (i) {axisofrotation} ;
	\pic (i) {axisofradius} ;
	\pic (i) {axisofangleandziaxis} ;

	\tdplotsetrotatedcoords{0}{0}{0} ;
	\def\ebeneKurveColor{black}
	\def\ebeneKurveThickness{1.2pt}
	\def\ebeneKurveOpacity{1}

	% meridian
	\pic (i) {boldmeridianfirst} ;
	\pic (i) {boldmeridiansecond} ;

	% parametric coordinate
	\def\ebeneKurveColor{blue}
	\def\ebeneKurveThickness{0.5pt}
	\def\ebeneKurveOpacity{1}
	\def\shiftOfParamatericCoordinate{0.4}
	\def\lengthOfParamatericCoordinate{2.2}
	\begin{scope}
		\clip ( 0, \originalxicoordinate, 0 )
			-- ( 0, \lengthOfParamatericCoordinate, 0 )
			-- ( 0, \lengthOfParamatericCoordinate, 10 )
			-- ( 0, \originalxicoordinate, 10 )
			-- cycle ;

		\path ( 0, 0, \shiftOfParamatericCoordinate ) pic (s) {boldmeridianfirst} ;
	\end{scope}
	\draw [ line width=0.5pt, color=blue, fill=white ]
		( 0, \originalxicoordinate, 1.69 + \shiftOfParamatericCoordinate ) circle ( 2pt ) ;
	\draw [ line width=0.5pt, color=blue, -{Stealth[round, length=2.5mm, width=2mm]} ]
		( 0, \lengthOfParamatericCoordinate, 2.16 + \shiftOfParamatericCoordinate )
		-- ( 0, \lengthOfParamatericCoordinate + 0.2, 2.16 + \shiftOfParamatericCoordinate + 0.16 )
	node [ pos=0, above left, inner sep=0pt, outer sep=3pt ]
		{\scalebox{\enlargetext}{$ s $}} ;

	% angular coordinate, phi is the angle to the plane of meridian
	\tdplotdefinepoints%
		( 0, \originalxicoordinate, 0 )%
		( 4 * sin( \phiAngle ), \originalxicoordinate, 4 * cos( \phiAngle ) )%
		( 0, \originalxicoordinate, 1 )
	\tdplotdrawpolytopearc[color=blue, -{To[length=1.7mm, width=2mm]}]{4}%
		{anchor=south, outer sep=4pt, blue}%
		{\scalebox{\enlargetext}{$ \varphi $}}
	\draw [ line width=0.5pt, color=blue, fill=white ]
		( { 4 * sin( \phiAngle ) }, \originalxicoordinate, { 4 * cos( \phiAngle ) } ) circle ( 2pt ) ;

	% psi angle to tangent
	\ifthenelse{\NOT \cameraanglesecond = 0 \AND \NOT \cameraanglefirst = 0}
	{%
	\coordinate (pointXi) at ( 0, \bezierpointbeginy + 3, \bezierpointbeginz ) ;

	\tdplotdefinepoints%
		( 0, \bezierpointbeginy, \bezierpointbeginz )%
		( 0, \bezierpointbeginy + 3, \bezierpointbeginz )%
		( 0, \tangentToMeridianY, \tangentToMeridianZ )
	\tdplotdrawpolytopearc[-{To[length=1.7mm, width=2mm]}]{2}{anchor=west}{\scalebox{\enlargetext}{$ \psi $}}

	\draw [line width=0.5pt, blue, line cap=round]
		(pointOfInterest) -- (pointXi) ;
	} {}

	% tangent plane (2nd part)
	\ifthenelse{\NOT \cameraanglefirst = 90 \OR \NOT \cameraanglesecond = 90}
	{%
	\pgfmathsetmacro{\secondlinelengthplus}{4.4}

	\fill [color=cyan!33!white, opacity=.25]
		(pointOfInterest) { [rounded corners=1cm] --
		( \secondlinelengthplus, \bezierpointbeginy, \bezierpointbeginz ) } --
		( 0, \bezierpointbeginy + \firstlinelengthy, \bezierpointbeginz + \firstlinelengthz )
		-- cycle ;
	} {}

	% tangent vectors

	\draw [line width=1.25pt, color=orange, -{Latex[round, length=4mm, width=2.5mm]}]
		(pointOfInterest) -- (tangentToMeridian)
	node [above left, pos=0.77, shape=circle, inner sep=0pt, outer sep=2pt]
		{\scalebox{\enlargetext}{$ \bm{t} \hspace{-0.1ex} \equiv \hspace{-0.2ex}
		\scalebox{\enlargeless}{$ \displaystyle\frac{\raisemath{-0.2em}%
			{\partial {\locationvector}}}{\partial s} $} $}} ;

	\ifthenelse{\NOT \cameraanglefirst = 90 \OR \NOT \cameraanglesecond = 90}
	{%
	\pgfmathsetmacro{\tangenttoparallelelength}{2.8}
	\draw [line width=1.25pt, color=magenta, -{Latex[round, length=4mm, width=2.5mm]}]
		(pointOfInterest) -- ( \tangenttoparallelelength, \bezierpointbeginy, \bezierpointbeginz )
	node [above right, pos=0.6, shape=circle, inner sep=0pt, outer sep=-1.5pt]
		{\scalebox{\enlargetext}{$ \bm{e}_{\varphi} \hspace{-0.4ex}
		\equiv \hspace{-0.2ex}
		\scalebox{\enlargeless}{$ \displaystyle\frac{\raisemath{-0.1em}%
			{\partial \smash{\bm{e}_{\hspace{-0.15ex}\rho}}}}{\partial \varphi} $} $}} ;
	} {}

	% normal vector

	\pgfmathsetmacro{\normallengthz}{2.4}
	\pgfmathsetmacro{\normallengthy}{- \normallengthz * \bezierpointlengthz / \bezierpointlengthy}
	\pgfmathsetmacro{\normalToMeridianY}{\bezierpointbeginy + \normallengthy}
	\pgfmathsetmacro{\normalToMeridianZ}{\bezierpointbeginz + \normallengthz}

	\coordinate (normalToMeridian) at ( 0, \normalToMeridianY, \normalToMeridianZ ) ;

	\draw [line width=1.25pt, color=red, -{Latex[round, length=4mm, width=2.5mm]}]
		(pointOfInterest) -- (normalToMeridian)
	node [above right, pos=0.97
		, inner sep=1pt, outer sep=3.3pt
		, shape=circle %%, fill=white
		] {\scalebox{\enlargetext}{$ \bm{n} $}} ;

	% xi and rho unit vectors

	\draw [line width=1.25pt, color=blue!75!black, -{Latex[round, length=4mm, width=2.5mm]}]
		( 0, 6.6, 0 ) -- ++( 0, 2.7, 0 )
	node [above, pos=0.51
		, inner sep=1pt, outer sep=1pt
		, shape=circle %%, fill=white
		] {\scalebox{\enlargetext}{$ \bm{e}_{\hspace{-0.1ex}\xi} $}} ;

	\draw [line width=1.25pt, color=blue!75!black, -{Latex[round, length=4mm, width=2.5mm]}]
		( 0, \originalxicoordinate, 5.5 ) -- ++( 0, 0, 3.3 )
	node [right, pos=0.5
		, inner sep=1pt, outer sep=2pt
		, shape=circle %%, fill=white
		] {\scalebox{\enlargetext}{$ \bm{e}_{\hspace{-0.2ex}\rho}\scalebox{\enlargeless}{$(\varphi)$} $}} ;

	% point of interest
	\fill [blue] (pointOfInterest) circle ( 2.4pt ) ;

	% psi angle to normal
	\ifthenelse{\NOT \cameraanglesecond = 0 \AND \NOT \cameraanglefirst = 0}
	{%
	\coordinate (pointRho) at ( 0, \bezierpointbeginy, \bezierpointbeginz + 3 ) ;

	\tdplotdefinepoints%
		( 0, \bezierpointbeginy, \bezierpointbeginz )%
		( 0, \bezierpointbeginy, \bezierpointbeginz + 3 )%
		( 0, \normalToMeridianY, \normalToMeridianZ )
	\tdplotdrawpolytopearc[-{To[length=1.7mm, width=2mm]}]{2.2}{anchor=south}{\scalebox{\enlargetext}{$ \psi $}}

	\draw [line width=0.5pt, blue, line cap=round]
		(pointOfInterest) -- (pointRho) ;

	} {}

	% location vector of point
	\draw [line width=1.25pt, color=blue, -{Latex[round, length=4mm, width=2.5mm]}]
		( 0, \originalxicoordinate, 0 ) -- (pointOfInterest)
	node [below right, pos=0.65
		, inner sep=1pt, outer sep=1pt
		, shape=circle %%, fill=white
		] {\scalebox{\enlargetext}{$ \locationvector\scalebox{\enlargeless}{$(\varphi, s)$} $}} ;

\end{tikzpicture}

\vspace*{.5cm}

\begin{equation*}\begin{array}{c}
q^1 \hspace{-0.2ex} = \varphi
\hspace{.1ex} , \:\:
q^{\hspace{.05ex}2} \hspace{-0.2ex} = s
\\[.4em]
\xi \hspace{-0.4ex}=\hspace{-0.32ex} \xi(s)
\hspace{.1ex} , \:\:
\rho \hspace{-0.32ex}=\hspace{-0.4ex} \rho\hspace{.1ex}(s)
\hspace{.1ex} , \:
\scalebox{.85}{$ \displaystyle\frac{\raisemath{-0.12em}{\partial \hspace{-0.1ex} \rho}}{\partial \varphi} $} = 0
\\[.6em]
\bm{e}_{\xi} \hspace{-0.25ex} = \hspace{-0.1ex} \boldconstant
\hspace{.1ex} , \:\:
%%\bm{e}_{\hspace{-0.1ex}\rho} \hspace{-0.45ex}=\hspace{-0.32ex}
\bm{e}_{\hspace{-0.1ex}\rho}(\varphi) \hspace{-0.2ex}
= \bm{e}_y \hspace{-0.2ex} \operatorname{cos} \varphi + \bm{e}_{\hspace{-0.1ex}z} \hspace{-0.2ex} \operatorname{sin} \varphi
\end{array}\end{equation*}

\begin{equation*}\begin{array}{c}
\locationvector(q^1 \hspace{-0.44ex}, q^{\hspace{.05ex}2}) \hspace{-0.2ex} = \locationvector(\varphi, s) \hspace{-0.2ex}
= \xi(s) \hspace{.1ex} \bm{e}_{\xi} + \rho\hspace{.1ex}(s) \hspace{.2ex} \bm{e}_{\hspace{-0.1ex}\rho}(\varphi)
\\[.3em]
\locationvector = \xi \bm{e}_{\xi} + \rho \hspace{.15ex} \bm{e}_{\hspace{-0.1ex}\rho}
\end{array}\end{equation*}

\begin{equation*}\begin{array}{c}
\locationvector_\differentialindex{1} \hspace{-0.2ex} = \partial_1 \hspace{-0.1ex} \locationvector
= \partial_{\hspace{-0.1ex}\varphi} \locationvector
= \scalebox{.85}{$ \displaystyle\frac{\raisemath{-0.15em}{\partial \hspace{.1ex} \locationvector}}{\partial \varphi} $}
= \hspace{-0.1ex} \rho \hspace{.2ex} \scalebox{.85}{$ \displaystyle\frac{\raisemath{-0.15em}{\partial \bm{e}_{\hspace{-0.1ex}\rho}}}{\partial \varphi} $}
= \hspace{-0.1ex} \rho \hspace{.15ex} \bm{e}_{\varphi}
\hspace{.1ex} , \:\,
\bm{e}_{\varphi} \hspace{-0.2ex} \equiv
\scalebox{.85}{$ \displaystyle\frac{\raisemath{-0.15em}{\partial \bm{e}_{\hspace{-0.1ex}\rho}}}{\partial \varphi} $}
= - \hspace{-0.2ex} \operatorname{sin} \varphi \hspace{.2ex} \bm{e}_y \hspace{-0.1ex} + \hspace{.1ex} \operatorname{cos} \varphi \hspace{.2ex} \bm{e}_{\hspace{-0.1ex}z}
\\[.8em]
%
\locationvector_\differentialindex{2} \hspace{-0.2ex} = \partial_2 \locationvector
= \partial_{\hspace{-0.2ex}s} \locationvector
= \scalebox{.85}{$ \displaystyle\frac{\raisemath{-0.15em}{\partial \hspace{.1ex} \locationvector}}{\partial s} $}
= \scalebox{.85}{$ \displaystyle\frac{\raisemath{-0.15em}{\partial \hspace{.1ex} \xi}}{\partial s} $} \hspace{.2ex} \bm{e}_{\xi} \hspace{-0.1ex}
+ \scalebox{.85}{$ \displaystyle\frac{\raisemath{-0.15em}{\partial \hspace{-0.1ex} \rho}}{\partial s} $} \hspace{.2ex} \bm{e}_{\hspace{-0.1ex}\rho} \hspace{-0.15ex}
= \hspace{.1ex} \bm{t}
\end{array}\end{equation*}

\begin{equation*}\begin{array}{c}
\locationvector^1 \hspace{-0.25ex} = \rho^{\hspace{-0.2ex}\expminusone} \hspace{-0.1ex} \bm{e}_{\varphi}
\\[.25em]
\locationvector^2 \hspace{-0.2ex} = \hspace{.1ex} \bm{t}
\end{array}\end{equation*}

\vspace{-0.5em}
\begin{equation*}\begin{array}{c}
\locationvector_\differentialindex{3} \hspace{-0.2ex} = \locationvector^3 \hspace{-0.24ex}
= \bm{n}
= \bm{e}_{\varphi} \hspace{-0.25ex} \times \bm{t}
\end{array}\end{equation*}

\tikzset{
	mark position/.style args={#1(#2)}{
		postaction={ decorate,
			decoration={
				markings,
				mark=at position #1 with \coordinate (#2);
}	}	}	}

\def\tikzunit{cm}
\pgfmathsetmacro\scaleofunits{2.5}

\begin{tikzpicture}[scale=\scaleofunits]

	\def\sinelength{1.2}
	\def\dxilength{0.33}
	\def\drhomultiplier{0.77}

	\draw [line width=0.5pt, color=blue, line cap=round, dash pattern=on 0pt off 1.6\pgflinewidth]
		( 0, 0 ) -- ( 1.77, 0 ) ;

	\draw [line width=0.5pt, color=orange, line cap=round, dash pattern=on 0pt off 1.6\pgflinewidth]
		( 0, 0 ) -- ( 1.42, \drhomultiplier * 1.42 ) ;

	\draw [line width=0.6pt, color=blue, line cap=round]
		( 0, 0 ) -- ( \dxilength, 0 )
	node [pos=0.55, below, inner sep=0pt, outer sep=3.3pt] {$ d \xi $} ;

	\draw [line width=0.6pt, color=blue, line cap=round]
		( \dxilength, 0 ) -- ( \dxilength, \drhomultiplier*\dxilength )
	node [pos=0.5, right, inner sep=0pt, outer sep=2.2pt] {$ d \hspace{-0.16ex} \rho $} ;

	\draw [line width=1.2pt, color=black, domain=0:\sinelength, line cap=round, mark position=0.69(fornode)]
		plot [samples=100] (\x, {\drhomultiplier*sin(deg(\x))}) ;
	\node at (fornode) [black, below right, inner sep=0pt, outer sep=2.5pt]
		{$ s $} ; %%\scalebox{.66}{$\Delta$}

	\draw [line width=0.5pt, color=blue, line cap=round, dash pattern=on 0pt off 1.6\pgflinewidth]
		( \sinelength, 0 ) -- ( \sinelength, \drhomultiplier*\sinelength ) ;

	%%\pgfmathsetmacro{\argumentatmid}{.5 * \dxilength}
	%%\pgfmathsetmacro{\resultatmid}{\drhomultiplier*sin(deg(\argumentatmid))}
	%%\pgfmathsetmacro{\dybydxatmid}{\drhomultiplier*cos(deg(\argumentatmid))}
	%%\pgfmathsetmacro{\neighborhoodradius}{\resultatmid / \dybydxatmid}
	%%\draw [line width=1.2pt, color=yellow, line cap=round]
		%%( -\neighborhoodradius + \argumentatmid, - \neighborhoodradius*\dybydxatmid + \resultatmid )
		%%-- ++( 2*\neighborhoodradius, 2*\neighborhoodradius*\dybydxatmid )
	%%node [pos=0.5, above left, inner sep=0pt, outer sep=1.6pt]
		%%{$ ds $} ;

	\draw [line width=1.2pt, color=orange, domain=0:\dxilength, line cap=round, mark position=0.6(fornode)]
		plot [samples=10] (\x, \drhomultiplier*\x) ;
	\node at (fornode) [orange, above left, inner sep=0pt, outer sep=1.6pt]
		{$ ds $} ;

	\coordinate (O) at ( 0, 0 );
	\coordinate (xi) at ( \dxilength, 0 );
	\coordinate (ds) at ( \dxilength, \drhomultiplier*\dxilength );

	\pic [draw, line width=0.5pt, color=black, -{To},
		angle radius=\scaleofunits * 1.72 \tikzunit, angle eccentricity=1.05, "$\psi$"]
	{ angle = xi--O--ds };

	\node [black, inner sep=0pt, outer sep=0pt] at ( .8, -0.66 )
		{$ \begin{array}{c}
		\scalebox{.85}{$ \displaystyle\frac{\raisemath{-0.15em}{d \hspace{.1ex} \xi}}{d s} $}
		\hspace{-0.1ex} = \operatorname{cos} \psi
		\\[.66em]
		\scalebox{.85}{$ \displaystyle\frac{\raisemath{-0.15em}{d \hspace{-0.06ex} \rho}}{d s} $}
		\hspace{-0.1ex} = \operatorname{sin} \psi
		\end{array} $} ;

\end{tikzpicture}

%%\end{center}

\end{document}
