\begin{figure}[!htbp]

\vspace*{-0.5em}\[
\mathcircabove{\bm{e}}_i = \mathcircabove{\bm{e}}_i \dotp \bm{e}_j \bm{e}_j
\]

\vspace{-1.5em}\[
\left[ \begin{array}{c} \mathcircabove{\bm{e}}_1 \\ \mathcircabove{\bm{e}}_2 \\ \mathcircabove{\bm{e}}_3 \end{array} \right] =
\left[ \begin{array}{ccc}
\mathcircabove{\bm{e}}_1 \dotp \bm{e}_1 & \mathcircabove{\bm{e}}_1 \dotp \bm{e}_2 & \mathcircabove{\bm{e}}_1 \dotp \bm{e}_3 \\
\mathcircabove{\bm{e}}_2 \dotp \bm{e}_1 & \mathcircabove{\bm{e}}_2 \dotp \bm{e}_2 & \mathcircabove{\bm{e}}_2 \dotp \bm{e}_3 \\
\mathcircabove{\bm{e}}_3 \dotp \bm{e}_1 & \mathcircabove{\bm{e}}_3 \dotp \bm{e}_2 & \mathcircabove{\bm{e}}_3 \dotp \bm{e}_3
\end{array} \right] \hspace{-0.5ex}
\left[ \hspace{-0.12ex} \begin{array}{c} {\bm{e}}_1 \\ {\bm{e}}_2 \\ {\bm{e}_3} \end{array} \right]
\]

\vspace{-1.25em}

\begin{center}
\tdplotsetmaincoords{60}{120} % set orientation of axes
\pgfmathsetmacro{\angletheta}{42}
% three parameters for vector
\pgfmathsetmacro{\lengthofvector}{0.55}
\pgfmathsetmacro{\anglefromz}{40}
\pgfmathsetmacro{\anglefromx}{240}

\begin{tikzpicture}[scale=4, tdplot_main_coords] % tdplot_main_coords style to use 3dplot

	\coordinate (O) at (0,0,0);

	% draw initial axes
	\draw [line width=1.2pt, black, -{Stealth[round, length=4mm, width=2.4mm]}]
		(O) -- (1,0,0)
		node[pos=0.9, above, xshift=-0.8em] {$\mathcircabove{\bm{e}}_1$};

	\draw [line width=1.2pt, black, -{Stealth[round, length=4mm, width=2.4mm]}]
		(O) -- (0,1,0)
		node[pos=0.9, above, xshift=1em, yshift=-0.2em] {$\mathcircabove{\bm{e}}_2$};

	\draw [line width=1.2pt, red, -{Stealth[round,length=4mm,width=2.4mm]}]
		(O) -- (0,0,0.9)
		node[anchor=south] {$\mathcircabove{\bm{e}}_3 = \bm{e}_3 = \bm{k}$};

	% draw initial vector
	\tdplotsetcoord{point}{\lengthofvector}{\anglefromz}{\anglefromx} % {length}{angle from z}{angle from x}
		% it also defines (pointxy), (pointxz), and (pointyz) projections of point
	\draw [line width=1.2pt, black, -{Stealth[round, length=4mm, width=2.4mm]}]
		(O) -- (point)
		node[anchor=south] {$\initiallocationvector$};
	% draw its projection on xy plane
	\draw [line width=0.4pt, dotted, color=black] (O) -- (pointxy);
	\draw [line width=0.4pt, dotted, color=black] (pointxy) -- (point);

	% draw the angle, and label it
	% syntax: \tdplotdrawarc[coordinate frame, draw options]{center point}{r}{angle}{end angle}{label options}{label}
	\tdplotdrawarc [line width=0.5pt, red, ->]
		{(O)}{0.4}{0}{\angletheta}{anchor=north}{$\vartheta$}
	\tdplotdrawarc [line width=0.5pt, red, ->]
		{(O)}{0.4}{90}{90+\angletheta}{anchor=west}{$\vartheta$}

	% rotate coordinates using Euler angles "z(\alpha)y(\beta)z(\gamma)"
	\tdplotsetrotatedcoords{\angletheta}{0}{0}

	% draw rotated axes
	\draw [line width=1.2pt, blue, tdplot_rotated_coords, -{Stealth[round, length=4mm, width=2.4mm]}]
		(O) -- (1,0,0)
		node[pos=0.9, left, xshift=-0.1em] {$\bm{e}_1$};

	\draw [line width=1.2pt, blue, tdplot_rotated_coords, -{Stealth[round, length=4mm, width=2.4mm]}]
		(O) -- (0,1,0)
		node[pos=0.9, above, xshift=0.2em, yshift=0.2em] {$\bm{e}_2$};

	%%\draw [line width=1.2pt, blue, tdplot_rotated_coords, -{Stealth[round, length=4mm, width=2.4mm]}]
		%%(O) -- (0,0,0.8) ;

	% draw rotated vector
	\tdplotsetcoord{rotatedpoint}%
		{\lengthofvector}{\anglefromz}{\anglefromx+\angletheta}
	\draw [line width=1.2pt, blue, tdplot_rotated_coords, -{Stealth[round, length=4mm, width=2.4mm]}]
		(O) -- (rotatedpoint)
		node[anchor=south] {$\currentlocationvector$};
	% draw its projection on xy plane
	\draw [line width=0.4pt, dotted, color=blue, tdplot_rotated_coords] (O) -- (rotatedpointxy);
	\draw [line width=0.4pt, dotted, color=blue, tdplot_rotated_coords] (rotatedpointxy) -- (rotatedpoint);

	\tdplotdrawarc [line width=0.5pt, red, ->]
		{(O)}{0.28}{\anglefromx}{\anglefromx+\angletheta}{anchor=south east, xshift=0.3em, yshift=-0.1em}{$\vartheta$}

\end{tikzpicture}
\end{center}

\vspace{-1em}\[
\scalebox{0.8}[0.85]{$\left[ \begin{array}{ccc}
\mathcircabove{\bm{e}}_1 \dotp \bm{e}_1 & \mathcircabove{\bm{e}}_1 \dotp \bm{e}_2 & \mathcircabove{\bm{e}}_1 \dotp \bm{e}_3 \\
\mathcircabove{\bm{e}}_2 \dotp \bm{e}_1 & \mathcircabove{\bm{e}}_2 \dotp \bm{e}_2 & \mathcircabove{\bm{e}}_2 \dotp \bm{e}_3 \\
\mathcircabove{\bm{e}}_3 \dotp \bm{e}_1 & \mathcircabove{\bm{e}}_3 \dotp \bm{e}_2 & \mathcircabove{\bm{e}}_3 \dotp \bm{e}_3
\end{array} \right]$} \hspace{-0.32ex} = \hspace{-0.2ex}
%
\scalebox{0.8}[0.85]{$\left[ \hspace{-0.2ex} \begin{array}{ccc}
\operatorname{cos} \vartheta & \hspace{-1ex} \operatorname{cos} \left( 90\degree \!+ \vartheta \right) & \operatorname{cos} 90\degree \\
\operatorname{cos} \left( 90\degree \!- \vartheta \right) & \operatorname{cos} \vartheta & \operatorname{cos} 90\degree \\
\operatorname{cos} 90\degree & \operatorname{cos} 90\degree & \operatorname{cos} 0\degree
\end{array} \right]$} \hspace{-0.32ex} = \hspace{-0.2ex}
%
\scalebox{0.8}[0.85]{$\left[ \hspace{-0.1ex} \begin{array}{ccc}
\operatorname{cos} \vartheta & - \operatorname{sin} \vartheta & 0 \\
\operatorname{sin} \vartheta & \operatorname{cos} \vartheta & 0 \\
0 & 0 & 1
\end{array} \right]$}
\]

\vspace{-0.8em}
\[\begin{array}{c}
\mathcircabove{\bm{e}}_1 \hspace{-0.16ex} = \bm{e}_1 \operatorname{cos} \vartheta \hspace{.1ex} - \hspace{.1ex} \bm{e}_2 \operatorname{sin} \vartheta \\[0.1em]
\mathcircabove{\bm{e}}_2 \hspace{-0.16ex} = \bm{e}_1 \operatorname{sin} \vartheta \hspace{.1ex} + \hspace{.1ex} \bm{e}_2 \operatorname{cos} \vartheta \\[0.1em]
\mathcircabove{\bm{e}}_3 \hspace{-0.16ex} = \bm{e}_3 = \bm{k}
\end{array}\]

\vspace{-1em}
\begin{multline*}
\shoveleft{ \rotationtensor = \bm{e}_1 \hspace{-0.1ex} \mathcircabove{\bm{e}}_1 + \bm{e}_2 \mathcircabove{\bm{e}}_2 + \bm{e}_3 \mathcircabove{\bm{e}}_3 = \hfill }\\[1.5em]
%
= \hspace{.2ex} \tikzmark{StartBraceE1E1} {\bm{e}_1 \bm{e}_1 \operatorname{cos} \vartheta - \bm{e}_1 \bm{e}_2 \operatorname{sin} \vartheta \hspace{.2em}} \tikzmark{EndBraceE1E1} \hspace{-0.1ex} + \hspace{.1ex} \tikzmark{StartBraceE2E2} {\bm{e}_2 \bm{e}_1 \operatorname{sin} \vartheta + \bm{e}_2 \bm{e}_2 \operatorname{cos} \vartheta \hspace{.2em}} \tikzmark{EndBraceE2E2} \hspace{-0.1ex} + \tikzmark{StartBraceE3E3} {\hspace{.25ex} \bm{k} \bm{k} \hspace{.1ex}} \tikzmark{EndBraceE3E3} \hspace{.1ex}
=\\[0.32em]
%
= \hspace{.1ex} \UnitDyad \operatorname{cos} \vartheta - \hspace{-0.1ex} \tikzmark{StartBraceKk} {\hspace{.1ex}\bm{e}_3 \bm{e}_3\hspace{.1ex}} \tikzmark{EndBraceKk} \hspace{-0.25ex} \operatorname{cos} \vartheta \hspace{.1ex} + \tikzmark{StartBraceLeviCivita} {\left( \bm{e}_2 \bm{e}_1 - \bm{e}_1 \bm{e}_2 \right)} \tikzmark{EndBraceLeviCivita} \operatorname{sin} \vartheta + \bm{k} \bm{k} \hspace{.1ex}
=\\[1.5em]
%
\shoveright{ \hfill = \UnitDyad \operatorname{cos} \vartheta + \bm{k} \times \hspace{-0.2ex} \UnitDyad \operatorname{sin} \vartheta + \bm{k} \bm{k} \left({1 - \operatorname{cos} \vartheta}\right) }
\end{multline*}

\AddOverBrace[line width=0.75pt]{StartBraceE1E1}{EndBraceE1E1}{${\scriptstyle \bm{e}_1 \mathcircabove{\bm{e}}_1}$}
\AddOverBrace[line width=0.75pt]{StartBraceE2E2}{EndBraceE2E2}{${\scriptstyle \bm{e}_2 \mathcircabove{\bm{e}}_2}$}
\AddOverBrace[line width=0.75pt]{StartBraceE3E3}{EndBraceE3E3}{${\scriptstyle \bm{e}_3 \mathcircabove{\bm{e}}_3}$}
\AddUnderBrace[line width=0.75pt][-0.1ex,-0.2ex]{StartBraceKk}{EndBraceKk}{${\scriptstyle \bm{k}\bm{k}}$}
\AddUnderBrace[line width=0.75pt][-0.1ex,-0.2ex][xshift=0.4ex]{StartBraceLeviCivita}{EndBraceLeviCivita}{${\scriptstyle \bm{e}_3 \times \bm{e}_i \bm{e}_i \:=\: \levicivita_{3i\hspace{-0.1ex}j} \bm{e}_j \bm{e}_i}$}

\vspace{-0.5em}
\caption{\inquotes{\en{Finite rotation}\ru{Конечный поворот}}}\label{fig:eulerfiniterotation}
\end{figure}
