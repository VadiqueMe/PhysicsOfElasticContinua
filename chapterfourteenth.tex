\en{\chapter{Defects}}

\ru{\chapter{Дефекты}}

\thispagestyle{empty}

\label{chapter:defects}

\en{\section{Volterra dislocations}}

\ru{\section{Дислокации Вольтерры}}

\label{para:volterradislocations}

\begin{otherlanguage}{russian}

\lettrine[lines=2, findent=2pt, nindent=0pt]{Р}{ассмотрим} классическую линейную трёхмерную среду~(\chapref{chapter:linearclassicalelasticity}). Как показано в~\chapdotpararef{chapter:linearclassicalelasticity}{para:displacementsfromdeformations}, уравнение совместности деформаций

...



\section{Прямолинейные дислокации}

Линия дислокации может быть любой пространственной кривой, замкнутой в~теле или выходящей концами на~поверхность. Для дислокации произвольной формы в~неограниченной среде не~так~уж сложно получить соответствующее решение~\cite{eshelby-theoryofdislocations}. Мы~же ограничимся простейшим случаем прямо\-линей\-ной дислокации. Разыскивается решение

...



\section{Действие поля напряжений на дислокацию}

...



\section{О движении дислокаций}

...



\section{Точечные дефекты}

...



\section{Сила, действующая на точечный дефект}

...



\section{Непрерывно распределённые дислокации}

...



\section{Напряжения при намотке катушки}

Не~только дислокации и~точечные дефекты, но~и~макроскопические факторы могут быть источниками собственных напряжений. При намотке катушки (рис. ?? 42 ??) в~ней возникают напряжения от натяжения ленты. Расчёт этих напряжений очень сложен, если рассматривать детально процесс укладки ленты.

Но~существует чёткий алгоритм Southwell’а~\cite{southwell-introductiontotheoryofelasticity} расчёта напряжений в~катушке: укладка каждого нового витка вызывает внутри катушки приращения напряжений, определяемые соотношениями линейной упругости. Здесь два этапа, и~первый состоит в~решении задачи Ляме(??) для цилиндра при нагрузке на~внешнем радиусе

...



\vspace{8mm}
\hfill\begin{minipage}[b]{0.95\linewidth}
\fontsize{10}{12}\selectfont

\section*{\wordforbibliography}

Библиоloremipsum...

\end{minipage}

\end{otherlanguage}
