\en{\section{Tensor algebra, or operations with tensors}}

\ru{\section{Тензорная алгебра, или операции с~тензорами}}

\label{section:operationswithtensors}

\en{The whole tensor algebra}\ru{Целая тензорная алгебра}
\en{can be built}\ru{может быть построена}
\en{on the~only}\ru{на только лишь}
\en{five}\ru{пятёрке}\footnote{%
\en{The~four}\ru{Четвёрке}
\en{without the~equality}\ru{без равенства}.}
\en{operations}\ru{операций}~(\en{or}\ru{или}
\en{actions}\ru{действий}).
\en{This section}\ru{Этот раздел}\en{ is}\ru{\:---}
\en{just}\ru{как раз}
\en{about them}\ru{про них}.

% equality
\subsection*{\en{Equality}\ru{Равенство}}

\en{The~first}\ru{Первое}
(\en{or}\ru{или}
\en{the~zeroth}\ru{нулевое})
\en{is}\ru{это}
\textbold{\en{the~equality}\ru{равенство}}~\inquotes{$=$}.
\en{This operation}\ru{Эта операция}
\en{shows}\ru{показывает}\ru{,}
\ru{равен~ли}\en{whether}
\en{one tensor}\ru{один тензор}
\inquotes{\en{on the~left}\ru{слева}}
\en{is equal to another tensor}\ru{другому тензору}
\inquotesx{\en{on the~right}\ru{справа}}[.]
\en{Tensors}\ru{Тензоры}
\en{can be equal}\ru{могут быть равны}
\en{only when}\ru{лишь тогда, когда}
\en{their}\ru{их}
\en{complexities}\ru{сложности}
(\en{valencies}\ru{валентности})
\en{are the~same}\ru{одинаковы}.
\en{Tensors of different valencies}\ru{Тензоры разных валентностей}
\en{cannot be}\ru{не~могут быть}
\en{equal}\ru{равн\'{ы}}
\en{or}\ru{или}
\en{not equal}\ru{не~равн\'{ы}}.

\begin{equation}\label{tensoralgebra:equality}
.....
\end{equation}


....



% linear combination
\subsection*{\en{Linear combination}\ru{Линейная комбинация}}

\en{The~next operation}\ru{Следующая операция}
\en{is}\ru{это}
\textbold{\en{the~linear combination}\ru{линейная комбинация}}.
\en{It aggregates}\ru{Оно объединяет}
\en{the~addition}\ru{сложение}
\en{and}\ru{и}
\en{the~multiplication}\ru{умножение}
\en{by a~number}\ru{на~число}~(\en{by a~scalar}\ru{на~скаляр},
\en{or}\ru{или},
\en{in another word}\ru{другим словом},
\ru{шкалирование, }scaling).
\en{The~arguments}\ru{Аргументы}
\en{of this operation}\ru{этой операции}
\en{and the~result}\ru{и~результат}\en{ are}\ru{\:---}
\en{of the~same complexity}\ru{одинаковой сложности}.
\en{For}\ru{Для}
\en{a~pair}\ru{пары}
\en{of~tensors}\ru{тензоров}

\nopagebreak\vspace{-0.2em}
\begin{equation}\label{tensoralgebra:linearcombination}
\lambda \hspace{.1ex} a_{i\hspace{-0.1ex}j\ldots}
\hspace{-0.2ex} + \hspace{.2ex}
\mu \hspace{.1ex} b_{i\hspace{-0.1ex}j\ldots}
\hspace{-0.2ex} = \hspace{.2ex}
c_{i\hspace{-0.1ex}j\ldots} \hspace{-0.2ex}
\;\Leftrightarrow\;
\lambda \hspace{.1ex} \bm{a} + \mu \hspace{.1ex} \bm{b} = \bm{c}
\hspace{.2ex} .
\end{equation}

\vspace{-0.1em}\noindent
\en{Here}\ru{Здесь}
$\lambda$ \en{and}\ru{и}~$\mu$~\en{are}\ru{---} \en{scalar coefficients}\ru{коэффициенты\hbox{-}скаляры};
$\bm{a}$, $\bm{b}$ \en{and}\ru{и}~$\bm{c}$~\en{are}\ru{---} \en{tensors}\ru{тензоры} \en{of the~same complexity}\ru{одной и~той~же сложности}.
\en{It’s easy to show}\ru{Легко показать}\ru{,}
\en{that}\ru{что}
\en{the components}\ru{компоненты}
\en{of the result}\ru{результата}~$\bm{c}$
\en{satisfy}\ru{удовлетворяют}
\en{an~orthogonal transformation}\ru{ортогональному преобразованию}
\en{like}\ru{типа}~\eqref{orthotransform:2}.

\en{The~decomposition}\ru{Разложение}
\en{of~a~vector}\ru{вектора}
\en{by some basis}\ru{по какому-либо базису},
\en{that is}\ru{то есть}
\en{the~representation}\ru{представление}
\en{of~a~vector}\ru{вектора}
\en{as the~sum}\ru{суммой}~${\bm{v} = v_i \bm{e}_i}$,
\en{is nothing else but}\ru{есть не~что~иное как}
\en{the~linear combination}\ru{линейная комбинация}
\en{of~the~basis vectors}\ru{векторов базиса}~${\bm{e}_i}$
\en{with the~coefficients}\ru{с~коэффициентами}~${v_i}$.

\en{This operation}\ru{Эта операция}
\en{is }\emph{\en{linear}\ru{линейная}}\ru{,}
\en{because}\ru{потому что}
\en{the~only}\ru{только}
\en{two atomary kinds}\ru{два атомарных вида}
\en{of~motion}\ru{движения}
\en{are possible}\ru{возможны}
\en{on a~line}\ru{на~линии}:
\en{the~translation}\ru{трансляция}
(\en{the~movement}\ru{движение}
\en{along a~straight line}\ru{вдоль прямой линии})
\en{and}\ru{и}~\en{the~reflexion~(mirroring)}\ru{зеркальное отображение}
(\en{the~backward movement}\ru{движение в~обратную сторону}).

% multiplication
\subsection*{\en{Multiplication of tensors}\ru{Умножение тензоров}}

\en{One more}\ru{Ещё одна}
\en{operation}\ru{операция}\:---
\textbold{\en{the~multiplication}\ru{умножение}
(\en{the~tensor product}\ru{тензорное произведение},
\en{the~direct product}\ru{прямое произведение})}.
\en{It~takes arguments}\ru{Оно принимает аргументы}
\en{of~any complexities}\ru{любых сложностей},
\en{returning}\ru{возвращая}
\en{the~result}\ru{результат}
\en{of the~cumulative complexity}\ru{суммарной сложности}.
\en{Examples}\ru{Примеры}:

\nopagebreak\vspace{-0.1em}\begin{equation}\label{tensoralgebra:multiplication}
\begin{array}{rcl}
v_i a_{j\hspace{-0.1ex}k} \hspace{-0.16ex} = C_{i\hspace{-0.1ex}j\hspace{-0.1ex}k} & \Leftrightarrow & \bm{v} \, {^2\hspace{-0.2ex}\bm{a}} = {^3\hspace{-0.15ex}\bm{C}},
\\[.1em]
a_{i\hspace{-0.1ex}j} B_{abc} \hspace{-0.16ex} = D_{i\hspace{-0.1ex}jabc} & \Leftrightarrow & {^2\hspace{-0.2ex}\bm{a}} \hspace{.25ex} {^3\hspace{-0.33ex}\bm{B}} = {^5\hspace{-0.33ex}\bm{D}}.
\end{array}\hspace{1.5em}
\end{equation}

\vspace{-0.1em}
\en{Transformation of a~collection of result’s components}\ru{Преобразование совокупности компонент результата},
\en{such as}\ru{такой как}
${C_{i\hspace{-0.1ex}j\hspace{-0.1ex}k} \hspace{-0.2ex} = v_i a_{j\hspace{-0.1ex}k}}$,
\en{during a~rotation of~basis}\ru{при повороте базиса}\en{ is}\ru{\:---} \en{orthogonal}\ru{ортогональное},
\en{similar to}\ru{подобное}~\eqref{orthotransform:3},
\en{thus}\ru{поэтому}
\en{here’s no~doubt that}\ru{тут нет сомнений, что}
\en{such a~collection}\ru{такая совокупность}
\en{is a~set of tensor components}\ru{это набор компонент тензора}.

\en{The~primary}\ru{Первичный}
\en{and}\ru{и}
\en{already known}\ru{уж\'{е} знакомый}
(\en{from}\ru{по}~\sectionref{section:tensoranditscomponents})
\en{subtype of~multiplication}\ru{подвид умножения}\en{ is}\ru{\:---}
\en{the~dyadic product}\ru{диадное произведение}
\en{of~two vectors}\ru{двух векторов}
${^2\hspace{-0.25em}\bm{A} \hspace{-0.15ex} = \bm{b}\bm{c}}$.

% contraction
\subsection*{\en{Contraction}\ru{Свёртка}}

\en{The~fourth}\ru{Четвёртая}
(\en{or}\ru{или}
\en{the~third}\ru{третья})
\en{operation}\ru{операция}
\en{is called}\ru{называется}
\textbold{\ru{свёрткой~(}\en{the~}contraction\ru{)}}.
\en{It~applies}\ru{Оно применяется}
\en{to bivalent}\ru{к~бивалентным}
\en{and more complex tensors}\ru{и~более сложным тензорам}.
\en{This operation acts}\ru{Это действие}
\en{upon a~single tensor}\ru{над одним тензором},
\en{without}\ru{без}
\en{other}\ru{других}
\inquotesx{\en{participants}\ru{участников}}[.]
\en{Roughly speaking}\ru{Грубо говоря},
\en{contracting a~tensor}\ru{свёртывание тензора}
\en{is}\ru{есть}
\en{summing of its components}\ru{суммирование его компонент}
\en{over}\ru{по}
\en{some}\ru{какой\hbox{-}либо}
\en{pair of~indices}\ru{паре индексов}.
\en{As a~result}\ru{В~результате}\en{,}
\en{the~tensor’s complexity}\ru{сложность тензора}
\en{decreases by two}\ru{уменьшается на~два}.

\en{For}\ru{Для}
\en{a~trivalent tensor}\ru{трёхвалентного тензора}~${\hspace{-0.1ex} ^3\hspace{-0.33ex}\bm{D}}$
\en{there are the three possible contractions}\ru{возможны три свёртки}.
\en{They}\ru{Они}
\en{give vectors}\ru{дают векторы}
${\bm{a}}$, ${\bm{b}}$ \en{and}\ru{и}~${\bm{c}}$
\en{with components}\ru{с~компонентами}

\nopagebreak\vspace{-0.4em}
\begin{equation}\label{tensoralgebra:contraction}
a_{i} = D_{kki} \hspace{.1ex} ,
\;\;
b_{i} = D_{kik} \hspace{.1ex} ,
\;\;
c_{i} = D_{ikk} \hspace{.16ex} .
\end{equation}

\vspace{-0.33em}\noindent
\en{A~rotation of~basis}\ru{Поворот базиса}

\nopagebreak\vspace{-0.3em}
\begin{equation*}
a'_{i} = D\hspace{.16ex}'_{\hspace{-0.32ex}kki} \hspace{-0.16ex}
= \tikzmark{BeginDeltaPQBrace} {\cosinematrix{k'\hspace{-0.1ex}p} \hspace{.1ex} \cosinematrix{k'\hspace{-0.1ex}q}} \tikzmark{EndDeltaPQBrace} \hspace{.1ex} \cosinematrix{i'\hspace{-0.1ex}r} \hspace{.16ex} D_{pqr} \hspace{-0.16ex}
= \cosinematrix{i'\hspace{-0.1ex}r} \hspace{.16ex} D_{ppr} \hspace{-0.16ex}
= \cosinematrix{i'\hspace{-0.1ex}r} \hspace{.16ex} a_{r}
\end{equation*}
\AddUnderBrace[line width = .75pt][0, -0.22ex]{BeginDeltaPQBrace}{EndDeltaPQBrace}%
{${\scriptstyle \delta_{pq}}$}

\nopagebreak\vspace{-0.4em}\noindent
\en{shows}\ru{показывает}
\inquotes{\en{the~tensorial nature}\ru{тензорную природу}}
\en{of~the~result}\ru{результата}
\en{of~contraction}\ru{свёртки}.

\en{For a~tensor}\ru{Для~тензора}
\en{of~second complexity}\ru{второй сложности}\en{,}
\ru{возможен }\en{the~only one}\ru{лишь один} \en{kind}\ru{вид} \en{of~contraction}\ru{свёртки}\en{ is possible}.
\en{It gives a~scalar}\ru{дающий скаляр},
\en{known}\ru{известный}
\en{as}\ru{как}
\emph{\inquotes{\hspace{.3ex}\en{trace}\ru{след~(trace)}\hspace{.2ex}}}

\nopagebreak\en{\vspace{-0.2em}}\ru{\vspace{-0.8em}}
\begin{equation*}
\bm{B}\tracedot \hspace{.25ex} \equiv \hspace{.3ex}
\trace{\bm{B}} \hspace{.15ex} \equiv \hspace{.4ex}
\mathrm{I}\hspace{.16ex}(\bm{B}) \hspace{-0.15ex}
= \somebivalenttensorcomponents{kk}
\hspace{.1ex} .
\end{equation*}

\vspace{-0.2em}
\en{The~trace}\ru{След}
\en{of~the~unit tensor}\ru{единичного тензора}
(\inquotes{\en{contraction of~the~Kronecker delta}\ru{свёртка дельты Kronecker’а}})
\en{is equal to}\ru{равен}
\en{the~dimension of~space}\ru{размерности пространства}

\nopagebreak\vspace{-0.2em}
\begin{equation*}
\trace{\UnitDyad} = \hspace{-0.1ex} \UnitDyad\tracedot = \delta_{kk} \hspace{-0.2ex} = \hspace{.1ex} \delta_{1\hspace{-0.1ex}1} \hspace{-0.2ex} + \delta_{22} \hspace{-0.2ex} + \delta_{33} \hspace{-0.1ex} = \hspace{.1ex} 3
\hspace{.1ex} .
\end{equation*}

% index juggling
\subsection*{\en{Index juggling, transposing}\ru{Жонглирование индексами, транспонирование}}

\en{The~last}\ru{Последняя}
\en{operation}\ru{операция}
\en{is applicable}\ru{применима}
\en{to a~single tensor}\ru{к~одному тензору}
\en{of~the~second}\ru{второй}\footnote{%
\en{Transposing}\ru{Транспонирование}
\en{a~vector}\ru{вектора}
\en{makes no sense}\ru{не~имеет смысла}.}\hspace{-0.4ex}
\en{and}\ru{и}~\en{bigger complexities}\ru{б\'{о}льших сложностей}.
\en{It}\ru{Оно}
\en{is named}\ru{именуется}
\en{as}\ru{как}
\textbold{\ru{перестановка индексов~(}\en{the~}index swap\ru{)},
\ru{жонглирование индексами~(}index juggling\ru{)},
\ru{транспонирование~(}transposing\ru{)}}.
\en{From}\ru{Из}
\en{components}\ru{компонент}
\en{of~a~tensor}\ru{тензора}\en{,}
\ru{возникает }\en{the~new collection}\ru{новая совокупность}\en{ emerges}
\en{with another}\ru{с~другой}
\en{sequence of~indices}\ru{последовательностью индексов},
\en{and }\ru{а~}\en{the result’s complexity}\ru{сложность результата}
\en{stays}\ru{остаётся}
\en{the~same}\ru{той~же}.
\en{For example}\ru{Для примера},
\en{a~trivalent tensor}\ru{трёхвалентный тензор}~${^3\hspace{-0.16em}\bm{D}}$
\en{can give}\ru{может дать}
\en{tensors}\ru{тензоры}
${^3\hspace{-0.28em}\bm{A}}$,
${^3\hspace{-0.15em}\bm{B}}$,
${^3\hspace{-0.05em}\bm{C}}$
\en{with components}\ru{с~компонентами}

\nopagebreak\vspace{-0.2em}
\begin{equation}\label{tensoralgebra:transposing}
\begin{array}{rcl}
{^3\hspace{-0.3em}\bm{A}} = {^3\hspace{-0.16em}\bm{D}}_{\indexjuggling{1}{2}}
& \!\Leftrightarrow\!\! &
A_{i\hspace{-0.1ex}j\hspace{-0.1ex}k} = D_{j\hspace{-0.06ex}ik}
\hspace{.1ex} ,
\\
{^3\hspace{-0.15em}\bm{B}} = {^3\hspace{-0.16em}\bm{D}}_{\indexjuggling{1}{3}}
& \!\Leftrightarrow\!\! &
B_{i\hspace{-0.1ex}j\hspace{-0.1ex}k} = D_{kj\hspace{-0.06ex}i}
\hspace{.1ex} ,
\\
{^3\hspace{-0.05em}\bm{C}} = {^3\hspace{-0.16em}\bm{D}}_{\indexjuggling{2}{3}}
& \!\Leftrightarrow\!\! &
C_{i\hspace{-0.1ex}j\hspace{-0.1ex}k} = D_{ikj}
\hspace{.1ex} .
\end{array}
\end{equation}

\en{For a~bivalent tensor}\ru{Для бивалентного тензора}\en{,}
\ru{возможно }\en{the~only one transposition}\ru{лишь одно транспонирование}\en{ is possible}\::
${\bm{A}^{\hspace{-0.05em}\T} \hspace{-0.15ex} \equiv \bm{A}_{\indexjuggling{1}{2}} = \hspace{-0.1ex} \bm{B}
\hspace{.4ex}\Leftrightarrow\hspace{.25ex}
B_{i\hspace{-0.1ex}j} \hspace{-0.1ex} = A_{j\hspace{-0.06ex}i}}$.
\en{Obviously}\ru{Очевидно},
${\bigl( \hspace{-0.1ex} \bm{A}^{\hspace{-0.05em}\T} \hspace{.15ex} \bigr)^{\hspace{-0.25ex}\T} \hspace{-0.2ex} = \bm{A}}$.

\en{For}\ru{Для}
\en{the~dyadic product}\ru{диадного произведения}
\en{of~two vectors}\ru{двух векторов},
${\bm{a} \bm{b} = \bm{b} \bm{a} ^{\hspace{-0.05em}\T}\hspace{-0.4ex}}$.

% combining operations
\subsection*{\en{Combining operations}\ru{Комбинирование операций}}

\en{The~four}\ru{Четыре}
\en{presented}\ru{представленные}
\en{algebraic operations}\ru{алгебраические операции}
(\en{actions})\ru{действия}
\en{can be}\ru{могут быть}
\en{combined}\ru{скомбинированы}
\en{in various sequences}\ru{в~разных последовательностях}.

\en{The~combination}\ru{Комбинация}
\en{of~}\en{multiplication}\ru{умножения}~\eqref{tensoralgebra:multiplication}
\en{and}\ru{и}~\en{contraction}\ru{свёртки}~\eqref{tensoralgebra:contraction}\:---
\en{the~}\dotproductinquotes\hbox{-}\en{product}\ru{произведение}~(dot product)\:---
\en{is the~most frequently used}\ru{самая часто используемая}.
\en{In the direct indexless notation}\ru{В~прямой безиндексной записи}
\en{this is denoted}\ru{это обозначается}
\en{by the~large dot}\ru{крупной точкой}~\hbox{\dotproductinquotes\hspace{-0.5ex},}
\en{which}\ru{которая}
\en{shows}\ru{показывает}
\en{the~contraction}\ru{свёртку}
\en{by adjacent indices}\ru{по соседним~(смежным) индексам}\::

\nopagebreak\vspace{-0.2em}
\begin{equation}\label{tensoralgebra:dotproductexamples}
\bm{a} = \bm{B} \dotp \bm{c}
\,\Leftrightarrow\,
a_i \hspace{-0.15ex} = B_{i\hspace{-0.1ex}j} c_j
\hspace{.1ex} , \;\:
\bm{A} = \bm{B} \dotp \bm{C}
\,\Leftrightarrow\,
A_{i\hspace{-0.1ex}j} \hspace{-0.2ex} = B_{ik} C_{kj}
\hspace{.1ex} .
\end{equation}

\en{The defining property}\ru{Определяющее свойство}
\en{of the unit tensor}\ru{единичного тензора}\:---
\en{the~neutrality}\ru{нейтральность}
(\en{it is}\ru{это}
\en{the~}\inquotes{\href{https://en.wikipedia.org/wiki/Identity_element}{identity element}})
\en{for}\ru{для}
\en{the~}\dotproductinquotes\hbox{-}\en{product}\ru{произведения}
(\en{the~tensor product}\ru{тензорного произведения}
\en{with the subsequent contraction}\ru{с~последующей свёрткой}
\en{by adjacent indices}\ru{по соседним индексам})

\nopagebreak\vspace{-0.2em}
\begin{equation}
\label{definingpropertyoftheidentitytensor}
{^\mathrm{n}\hspace{-0.2ex}\bm{a}} \dotp \UnitDyad
= \UnitDyad \dotp \hspace{-0.15ex} {^\mathrm{n}\hspace{-0.2ex}\bm{a}}
= {^\mathrm{n}\hspace{-0.2ex}\bm{a}} \;\:\:
\forall \, {^\mathrm{n}\hspace{-0.2ex}\bm{a}} \;\; \forall \hspace{.1ex} \mathrm{n \!>\! 0}
\hspace{.1ex} .
\end{equation}

\en{In the}\ru{В}~({\en{commutative}\ru{коммутативном}})
\en{scalar product}\ru{скалярном произведении}
\en{of two vectors}\ru{двух векторов}\en{,}
\en{the~dot}\ru{точка}
\en{represents}\ru{представляет}
\en{the~same}\ru{то~же самое}\::
\en{the~dyadic product}\ru{диадное произведение}
\en{and}\ru{и}
\en{the~subsequent contraction}\ru{последующую свёртку}

\nopagebreak\vspace{-0.2em}
\begin{equation}
\label{transposeofdotproductforvectors}
\bm{a} \dotp \bm{b}
= \hspace{-0.1ex} ( \bm{a} \bm{b} )\hspace{.1ex}\tracedot
= a_i b_i \hspace{-0.1ex}
= b_i a_i \hspace{-0.1ex}
= \hspace{-0.1ex}( \bm{b} \bm{a} )\hspace{.1ex}\tracedot
= \bm{b} \dotp \bm{a}
\hspace{.1ex} .
\end{equation}

\en{And here’s}\ru{А~вот}
\en{how}\ru{как}
\en{the~multipliers}\ru{множители}
\en{of the~}\dotproductinquotes\hbox{-}\en{product}\ru{произведения}~%
(dot product\ru{’а})
\en{of~two}\ru{двух}
\en{second complexity tensors}\ru{тензоров второй сложности}
\en{are swapped}\ru{меняются местами}

\nopagebreak\vspace{-0.2em}
\begin{equation}\label{transposeofdotproductforbivalenttensors}
\begin{array}{r@{\hspace{.8ex}}c@{\hspace{.8ex}}l}
\bm{B} \hspace{-0.1ex} \dotp \bm{Q} & = & \hspace{-0.2ex} \bigl( \bm{Q}^{\T} \hspace{-0.2ex}\dotp \bm{B}^{\T} \bigr)^{\hspace{-0.25ex}\T}
\\[.1em]
\bigl( \bm{B} \hspace{-0.1ex} \dotp \bm{Q} \bigr)^{\hspace{-0.25ex}\T} \hspace{-0.4ex} & \hspace{-0.4ex} = & \hspace{-0.3ex} \bm{Q}^{\T} \hspace{-0.2ex}\dotp \bm{B}^{\T}
\hspace{-0.2ex} .
\end{array}
\end{equation}

\noindent
\en{For}\ru{Для}
\en{two dyads}\ru{двух диад}
${\bm{B} \hspace{-0.1ex} = \bm{b} \hspace{.1ex} \bm{d}}$
\en{and}\ru{и}~${\bm{Q} \hspace{-0.1ex} = \bm{p} \hspace{.1ex} \bm{q}}$

\nopagebreak\vspace{-0.2em}
\begin{equation*}
\begin{array}{r@{\hspace{.8ex}}c@{\hspace{.8ex}}l}
\bigl(\hspace{.1ex} \bm{b} \hspace{.1ex} \bm{d} \dotp \bm{p} \bm{q} \hspace{.1ex}\bigr)^{\hspace{-0.25ex}\T} \hspace{-0.4ex} & \hspace{-0.4ex} = & \hspace{-0.1ex} \bm{p} \hspace{.1ex} \bm{q}^{\T} \hspace{-0.4ex} \dotp \bm{b} \hspace{.1ex} \bm{d}^{\hspace{.1ex}\T}
\\[.2em]
d_i  \hspace{.1ex} p_i \hspace{.25ex} \bm{b} \hspace{.1ex} \bm{q}^{\T} & \hspace{-0.5ex} = & \hspace{-0.1ex} \bm{q} \hspace{.1ex} \bm{p} \dotp \bm{d}  \hspace{.1ex} \bm{b}
\\[.1em]
d_i  \hspace{.1ex} p_i \hspace{.25ex} \bm{q} \bm{b} & = & p_i  \hspace{.1ex} d_i \hspace{.25ex} \bm{q} \bm{b}
\hspace{.1ex} .
\end{array}
\end{equation*}

\noindent
\en{For}\ru{Для}
\en{a~vector}\ru{вектора}
\en{and a~bivalent tensor}\ru{и~бивалентного тензора}

\nopagebreak\vspace{-0.2em}
\begin{equation}%%\label{vector.dot.bivalenttensor}
\bm{c} \hspace{.2ex} \dotp \bm{B}
= \bm{B}^{\T} \hspace{-0.4ex} \dotp \bm{c}
\hspace{.2ex} , \hspace{.7em}
%
\bm{B} \dotp \bm{c}
= \bm{c} \hspace{.2ex} \dotp \bm{B}^{\T}
\hspace{-0.3ex} .
\end{equation}

%%\en{Tensor of~second valence \inquotes{squared} is}\ru{Тензор второй валентности \inquotes{в~квадрате} это}

%%\nopagebreak\vspace{-0.2em}\begin{equation}\label{exponentiation:two}
%%\bm{B}^2 \equiv\hspace{.2ex} \bm{B} \hspace{-0.1ex} \dotp \hspace{-0.1ex} \bm{B} .
%%\end{equation}

\vspace{-0.2em}
\en{Contraction}\ru{Свёртка}
\en{can be}\ru{может}
\en{repeated}\ru{повторяться}
\en{for}\ru{для}
\en{two or more}\ru{двух или более}
\en{adjacent indices}\ru{смежных индексов},

\nopagebreak\vspace{-0.3em}
\begin{equation}\label{tensoralgebra.contractiondotrepeated}
\bigl( \hspace{-0.1ex} \bm{A} \dotp \hspace{-0.1ex} \bm{B} \hspace{.1ex} \bigr)\tracedot
= \hspace{-0.1ex} \bm{A} \dotdotp \hspace{-0.1ex} \bm{B}
= \hspace{-0.1ex} A_{\hspace{.1ex}i\hspace{-0.1ex}j} B_{\hspace{-0.1ex}j\hspace{-0.06ex}i}
\hspace{.2ex} .
\end{equation}

\noindent
\en{The~double contraction}\ru{Двойная свёртка}
\en{of~a~bivalent tensor}\ru{бивалентного тензора}
\en{with}\ru{с}~\en{the~unit dyad}\ru{единичной диадой}
\en{gives}\ru{даёт}
\en{the~trace}\ru{след}

\nopagebreak\vspace{-0.4em}
\begin{equation}\label{doubledotwiththeunitdyad.istrace}
\bm{A} \hspace{-0.1ex} \dotdotp \hspace{-0.1ex} \UnitDyad
= \UnitDyad \dotdotp \hspace{-0.1ex} \bm{A}
= \bm{A}\hspace{.15ex}\tracedot
= \trace{\bm{A}}
= A_{j\hspace{-0.15ex}j}
\hspace{.2ex} .
\end{equation}

\noindent
\en{The~commutativity}\ru{Коммутативность}
\en{is guaranteed}\ru{гарантируется}

\nopagebreak\vspace{-0.4em}
\begin{equation}\label{doubledotcommutativity.tensoralgebra}
\bm{A} \dotdotp \hspace{-0.1ex} \bm{B}
= \hspace{-0.1ex} A_{\hspace{.1ex}i\hspace{-0.1ex}j} B_{\hspace{-0.1ex}j\hspace{-0.06ex}i} \hspace{-0.1ex}
= B_{\hspace{-0.1ex}j\hspace{-0.06ex}i} A_{\hspace{.1ex}i\hspace{-0.1ex}j} \hspace{-0.1ex}
= \bm{B} \dotdotp \hspace{-0.1ex} \bm{A}
\end{equation}

\vspace{-0.4em}\noindent
\en{for any two}\ru{для любых двух}
\en{bivalent tensors}\ru{бивалентных тензоров}
${\bm{A}}$ \en{and}\ru{и}~${\bm{B}}$,
\en{contracted twice}\ru{свёртываемых дважды}.

\newcommand\colonproductinquotes{\hbox{\hspace{-0.2ex}\inquotes{${\smash{\colonp}\hspace{.15ex}}$}\hspace{-0.2ex}}}

\en{In other texts}\ru{В~других текстах}\en{,}
\en{a~double contraction}\ru{двойная свёртка}
\en{may be written}\ru{может быть написана}
\inquotes{\en{vertically}\ru{вертикально}}
\en{as}\ru{как}
${\bm{A} \hspace{.2ex} \smash{\colonp} \hspace{.2ex} \bm{B}\hspace{.1ex}}$.
%
${\bm{A} \hspace{.2ex} \smash{\colonp} \hspace{.3ex} \bm{1} = \bm{1} \hspace{.3ex} \smash{\colonp} \hspace{.2ex} \bm{A}}$
\en{is nothing but}\ru{есть не что иное, как}
${\trace{\bm{A}}}$
\en{or}\ru{или}
${A_{\hspace{.1ex}ii}}$~\eqref{doubledotwiththeunitdyad.istrace}
\en{in}\ru{в}~\colonproductinquotes-\en{notation}\ru{записи}
\en{and}\ru{и}~$\bm{1}$
\en{for}\ru{для}
\en{the~unit tensor}\ru{единичного тензора}.
%
\en{To further confuse}\ru{Чтобы ещё больше запутать}
\en{the~reader}\ru{читателя},
\en{such a~colon}\ru{такое двоеточие}
\en{can denote}\ru{может обозначать}
\en{either}\ru{или}
${\smash{\bm{A} \hspace{.2ex} \colonp \hspace{.2ex} \bm{B} \overset{\scalebox{.5}{$(1)$}}{=} \hspace{-0.2ex} A_{\hspace{.1ex}i\hspace{-0.1ex}j} B_{\hspace{-0.1ex}j\hspace{-0.06ex}i}}}$\ru{,}
\en{or}\ru{или}
${\smash{\bm{A} \hspace{.2ex} \colonp \hspace{.2ex} \bm{B} \overset{\scalebox{.5}{$(2)$}}{=} \hspace{-0.2ex} A_{\hspace{.1ex}i\hspace{-0.1ex}j} B_{\hspace{.1ex}i\hspace{-0.1ex}j}}}$
\en{with the~additional}\ru{с~дополнительным}
\en{transposition}\ru{транспонированием}
\en{of~one of~the~tensors}\ru{одного из тензоров}.
%
\en{But don’t worry}\ru{Но не~волнуйтесь},
\en{in this book}\ru{в~этой книге}
\en{you can meet}\ru{вы можете встретить}
\en{the~}\colonproductinquotes-\en{product}\ru{произведение}
\en{only in this paragraph}\ru{только в~этом абзаце}, %% параграфе
\en{and when}\ru{и~когда}
${\bm{B}}$
\en{is transposed}\ru{транспонируется},
\en{then it is}\ru{тогда это}
${\bm{A} \dotdotp \hspace{-0.1ex} \smash{\bm{B}^{\T}}\hspace{-0.5ex}}$.
%
\en{Or}\ru{Или}
${\smash{\bm{A}^{\hspace{-0.16ex}\T}} \hspace{-0.4ex} \dotdotp \bm{B}}$,
\en{because}\ru{потому что}
\en{these are equal}\ru{они равны}

\nopagebreak\vspace{-0.3em}
\begin{equation}\label{transposingoneofthetwo.doubledotproduct}
\begin{gathered}
\bm{A} \dotdotp \bm{B}^{\T} \hspace{-0.3ex}
=
\bm{A}^{\hspace{-0.16ex}\T} \hspace{-0.4ex} \dotdotp \bm{B}
=
A_{\hspace{.1ex}i\hspace{-0.1ex}j} B_{i\hspace{-0.1ex}j}
\hspace{.1ex} ,
\\
\bm{A} \dotdotp \bm{B}
=
\bm{A}^{\hspace{-0.16ex}\T} \hspace{-0.4ex} \dotdotp \bm{B}^{\T} \hspace{-0.3ex}
=
A_{\hspace{.1ex}i\hspace{-0.1ex}j} B_{\hspace{-0.1ex}j\hspace{-0.06ex}i}
\hspace{.2ex} .
\end{gathered}
\end{equation}

\vspace{-0.2em}
\en{And as a~bonus}\ru{И~как бонус},
\en{here are}\ru{вот}
\en{more}\ru{ещё}
\en{useful equalities}\ru{полезные равенства}
\en{for}\ru{для}
\en{bivalent}\ru{бивалентных}
\en{tensors}\ru{тензоров}

\nopagebreak\vspace{-0.1em}
\begin{equation}\label{moreusefulequalities.forbivalenttensors}
\begin{gathered}
\bm{d} \hspace{.1ex} \dotp \hspace{-0.15ex} \bm{A} \dotp \bm{b}
= d_{\hspace{.1ex}i} A_{\hspace{.1ex}i\hspace{-0.1ex}j} b_{\hspace{-0.1ex}j} \hspace{-0.2ex}
= \hspace{-0.1ex} \bm{A} \dotdotp \hspace{.1ex} \bm{b} \hspace{.1ex} \bm{d}
= \bm{b} \hspace{.1ex} \bm{d} \hspace{.1ex} \dotdotp \hspace{-0.1ex} \bm{A}
= b_{\hspace{-0.1ex}j} d_{\hspace{.1ex}i} A_{\hspace{.1ex}i\hspace{-0.1ex}j}
\hspace{.1ex} ,
\\[.2em]
%
\bm{A} \narrowdotp \bm{B} \narrowdotdotp \UnitDyad = \hspace{-0.16ex} A_{\hspace{.1ex}i\hspace{-0.1ex}j} B_{\hspace{-0.1ex}j\hspace{-0.1ex}k} \hspace{.2ex} \delta_{ki} \hspace{-0.1ex} = \bm{A} \narrowdotdotp \bm{B}
\hspace{.1ex} ,
\:\:
\bm{A} \narrowdotp \bm{A} \narrowdotdotp \UnitDyad = \bm{A} \narrowdotdotp \bm{A}
\hspace{.1ex} ,
\\[.2em]
%
\bm{A} \narrowdotdotp \bm{B} \narrowdotp \bm{C} = \bm{A} \narrowdotp \bm{B} \narrowdotdotp \hspace{.1ex} \bm{C} = \hspace{.1ex} \bm{C} \narrowdotdotp \hspace{-0.1ex} \bm{A} \narrowdotp \bm{B} = A_{\hspace{.1ex}i\hspace{-0.1ex}j} B_{\hspace{-0.1ex}j\hspace{-0.1ex}k} C_{ki}
\hspace{.1ex} ,
\\[.15em]
%
\begin{multlined}
\bm{A} \hspace{-0.1ex}\narrowdotdotp \bm{B} \narrowdotp \bm{C} \narrowdotp \bm{D}
= \bm{A} \narrowdotp \bm{B} \narrowdotdotp \bm{C} \narrowdotp \bm{D}
= \bm{A} \narrowdotp \bm{B} \narrowdotp \bm{C} \narrowdotdotp \bm{D} \hspace*{4em} \\
= \bm{D} \narrowdotdotp \bm{A} \narrowdotp \bm{B} \narrowdotp \bm{C}
= A_{\hspace{.1ex}i\hspace{-0.1ex}j} B_{\hspace{-0.1ex}j\hspace{-0.1ex}k} C_{kh} D_{hi}
\hspace{.2ex} .
\end{multlined}
\end{gathered}
\end{equation}
