\begin{tcolorbox}[breakable, enhanced, colback = orange!8, before upper={\parindent3.2ex}, parbox = false]
\small%
\setlength{\abovedisplayskip}{2pt}\setlength{\belowdisplayskip}{2pt}%

\noindent
\textit{from \textboldoblique{Nowacki~W.} The Linear Theory of~Micropolar Elasticity.}
In: \textit{Micropolar Elasticity. International Centre for~Mechanical Sciences (Courses and~Lectures),
vol.\:151, 1974, pp.\:1\hbox{--}43}
\\[.55em]
%
\indent
Woldemar Voigt
\en{tried}\ru{пробовал}
to remove the shortcomings
\en{of the classical theory of elasticity}\ru{классической теории упругости}
[\textit{\textboldoblique{W.\:Voigt}. Theoretische Studien über die~Elasticitätsverhältnisse der~Krystalle. Abhandlungen der~Königlichen Gesellschaft der~Wissenschaften in~Göttingen, 34:\:3\hbox{--}51, 1887}]
by the assumption that
\en{the interaction}\ru{взаимодействие}
of two parts of the body
is transmitted through
an area element~$do$
by means
not only of the force vector~$\bm{p}do$\ru{,}
\en{but}\ru{но}
\en{also}\ru{также}
by the moment vector~$\bm{m}do$.
\en{Thus}\ru{Поэтому},
besides
the force stresses~$\sigma_{ji}$
also the moment stresses have been defined.

\en{However}\ru{Однако},
the complete theory
of asymmetric elasticity
was developed
by the brothers
\textboldoblique{François et~Eugène Cosserat}
who published it
in 1909
in the work
\textit{\inquotes{Théorie des corps déformables}}.

\en{They assumed that}\ru{Они предположили, что}
\en{the bodies}\ru{тел\'{а}}
\en{are composed}\ru{составлены}
\en{from the connected particles}\ru{из соединённых частиц}.
\en{These particles}\ru{Эти частицы}
\en{are like a~small}\ru{похожи на маленькие}
\en{rigid bodies}\ru{жёсткие тела}.

\en{During}\ru{В~течение}
\en{the deforming}\ru{деформирования}
\en{each particle}\ru{каждая частица}
\en{is displaced}\ru{перемещается}
\en{by vector}\ru{вектором}~${\bm{u}(\bm{r}, t)}$
\en{and rotated}\ru{и~поворачивается}
\en{by vector}\ru{вектором}~$\bm{\varphi}(\bm{r}, t)$,
\en{the functions}\ru{функциями}
\en{of the~location}\ru{положения}~$\bm{r}$
\en{and time}\ru{и~времени}~$t$.
\en{Thus}\ru{Поэтому}
\en{the points}\ru{точки}
(\en{particles}\ru{частицы})
\en{of the}\ru{\rucontinuum{}а} Cosserat\en{’s}\en{ continuum}
\en{possess}\ru{обладают}
\en{the orientation}\ru{ориентацией}
(\en{it is}\ru{это}
\en{a~}\inquotes{\en{polar media}\ru{полярная среда}}).
\en{So}\ru{Так что}
\en{we can}\ru{мы можем}
\en{speak}\ru{говорить}
\en{of the~rotation}\ru{о~повороте}
\en{of a~point}\ru{точки}.
\en{The mutually independent}\ru{взаимно независимые}
\en{vectors}\ru{в\'{е}кторы}
$\bm{u}$
\en{and}\ru{и}~$\bm{\varphi}$
\en{define}\ru{определяют}
\en{deformations}\ru{деформации}
\en{of a~body}\ru{т\'{е}ла}.

\en{The introduction}\ru{Введение}
\en{of~}$\bm{u}$
\en{and}\ru{и}~$\bm{\varphi}$
\en{with the assumption that}\ru{с~предположением, что}
\en{the transmission}\ru{передача}
\en{of forces}\ru{сил}
\en{through}\ru{через}
\en{an~area element}\ru{элемент площади}~$do$
\en{is happened}\ru{происходит}
\en{by the~force vector}\ru{вектором силы}~$\bm{p}$
\en{and}\ru{и}
\en{the~moment vector}\ru{вектором момента}~$\bm{m}$
\en{leads in the consequence}\ru{ведёт впоследствии}
\en{to the asymmetric}\ru{к~несимметричным}
\en{stress tensors}\ru{тензорам напряжений}
$\sigma_{\hspace{-0.25ex}qp}$
\en{and}\ru{и}~$\mu_{qp}$.

\en{The~theory}\ru{Теория}
\en{of~the~brothers}\ru{братьев}
E.~\en{and}\ru{и}~F.\;Cosserat
\en{remained unnoticed}\ru{оставалась незамеченной}
%and was not duly appreciated
\en{during}\ru{в~течение}
\en{their lifetime}\ru{времени их жизни}.
\en{It was so}\ru{Это было так}\ru{,}
\en{because}\ru{потому что}
\en{their theory was non-linear}\ru{их теория была нелинейной}
\en{and thus}\ru{и потому}
\en{included}\ru{включала}
\en{large deformations}\ru{конечные деформации},
\en{because}\ru{потому что}
\en{the frames of their theory}\ru{рамки их теории}
\en{were out}\ru{были вне}
\en{of the frames}\ru{рамок}
\en{of the classical linear elasticity}\ru{классической линейной упругости}.
\en{They tried}\ru{Они пробовали}
\en{to construct}\ru{сконструировать}
\en{the unified field theory}\ru{единую теорию поля},
\en{containing}\ru{содержащую}
\en{mechanics}\ru{механику},
\en{optics}\ru{оптику}
\en{and}\ru{и}~\en{electrodynamics}\ru{электродинамику},
\en{combined}\ru{объединённых}
\en{by a~principle}\ru{принципом}
\en{of the least action}\ru{наименьшего действия}.

The research in the field of the general theories of continuous media conducted in the last fifteen years,
drew the attention of the scientists to the Cosserats’ work.
Looking for the new models,
describing the behaviour of the real elastic media more precisely,
the models similar to, or identical with that of Cosserats’
have been encountered.
Here I want to mention, first of all,
the papers
by C.\:Truesdell and R.\,A.\;Toupin
[\textit{ \textboldoblique{C.\:Truesdell}
and \textboldoblique{R.\,A.\;Toupin}.
The classical field theories.
Encyclopædia of Physics, Chapter 1, Springer\hbox{-}Verlag, Berlin, 1960 }],
%
G.\:Grioli [\textit{\textboldoblique{Grioli~G.} Elasticité asymmetrique. Ann. di Mat. Pura et Appl. Ser. IV, 50 (1960)}],
R.\,D.\;Mindlin and H.\,F.\;Tiersten
[\textit{\textboldoblique{Mindlin, R.\,D.};
\textboldoblique{Tiersten, H.\,F.}
Effects of couple-stresses in linear elasticity. Arch. Rational Mech. Anal. 11. 1962. 415\hbox{--}448}].
\par
\end{tcolorbox}
