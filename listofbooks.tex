
\vspace*{10mm}

\addcontentsline{toc}{chapter}{\bibliographyname}

\newcommand\mirpublisher{<<Мир>>} % М.:\;Мир
\newcommand\naukapublisher{<<Наука>>} % М.:\;Наука
\newcommand\maschinenbaumoskauerverlag{М.:\;Машино\-строение}
\newcommand\maschinenbauleningraderverlag{Л.:\;Машино\-строение}
\newcommand\fizmatgiz{М.:\;Физ\-мат\-гиз} % Государственное издательство физико-математической литературы (Физматгиз)

\begin{thebibliography}{123}
\interlinepenalty=10000 % inhibit page breaks inside a paragraph
\small

\thispagestyle{empty}

%% “Mechanics of Solids. Volume II. Linear Theories of Elasticity and Thermoelasticity. Linear and Nonlinear Theories of Rods, Plates, and Shells (Editor: C. Truesdell)” originally appeared in hardcover as Volume VIa/2 of Encyclopedia of Physics © Springer-Verlag (Berlin, Heidelberg) 1972

\begin{otherlanguage}{russian}

\bibitem{stuartantman-theoryofrods} % Stuart S. Antman
\bibauthor{Antman,}[Stuart~S.] The~theory of~rods. In: Truesdell~C.\:(editor) Mechanics of~solids. Volume II. Linear theories of~elasticity and~thermoelasticity. Linear and~nonlinear theories of~rods, plates, and~shells. Springer\hbox{-}Verlag, 1973. Pages~641\hbox{--}703.

\bibitem{alfutov-stabilitycalc}
\bibauthor{Алфутов}[Н.][А.]
Основы расчета на~устойчивость упругих систем. Издание~2\hbox{-}е.
\maschinenbaumoskauerverlag, 1991.
\howmanypages{336~с.}

\bibitem{artobolevskiyetother}
\bibauthor{Артоболевский}[И.][И.],
\bibauthor{Бобровницкий}[Ю.][И.],
\bibauthor{Генкин}[М.][Д.]
Введение в~акустическую динамику машин.
\naukapublisher, 1979.
\howmanypages{296~с.}

\bibitem{the contact problem with the variable contact area by Ахтырец Г. П. and Короткин В. И.}
\bibauthor{Ахтырец}[Г.][П.],
\bibauthor{Короткин}[В.][И.]
Использование МКЭ
при решении
контактной задачи
теории упругости
с~переменной
зоной контакта~//~Известия
северо-кавказского научного центра высшей школы
(СКНЦ ВШ).
Серия естественные науки.
Ростов-на-Дону:
Издательство РГУ, 1984. №\:1.
С.\:38\hbox{--}42.

\bibitem{the solution of the contact problem using the finite element method by Ахтырец Г. П. and Короткин В. И.}
\bibauthor{Ахтырец}[Г.][П.],
\bibauthor{Короткин}[В.][И.]
К~решению
контактной задачи
с~помощью
метода конечных элементов~//~Механика сплошной среды.
Ростов-на-Дону:
Издательство РГУ, 1988.
С.\:43\hbox{--}48.
%To the solution of the contact problem using the finite element method

\bibitem{biderman-thinwalled}
\bibauthor{Бидерман}[В.][Л.]
Механика тонкостенных конструкций.
\maschinenbaumoskauerverlag,
1977.
\howmanypages{488~с.}

\bibitem{vlasov-thinwalledrods}
\bibauthor{Власов}[В.][З.]
Тонкостенные упругие стержни.
\fizmatgiz, 1959.
\howmanypages{568~с.}

\bibitem{goldenveizer-thinshells}
\bibauthor{Гольденвейзер}[А.][Л.] Теория упругих тонких оболочек. \naukapublisher, 1976. \howmanypages{512~с.}

\bibitem{goldenveizer.lidskiy.tovstik}
\bibauthor{Гольденвейзер}[А.][Л.], \bibauthor{Лидский}[В.][Б.], \bibauthor{Товстик}[П.][Е.] Свободные ко\-леба\-ния тонких упругих оболочек. \naukapublisher, 1979. \howmanypages{383~с.}

\bibitem{gordon-constructions}
\bibauthor{Gordon,}[James~E.] Structures, or Why things don’t fall down. Penguin Books, 1978. \howmanypages{395~pages.}
\emph{Перевод:}
\bibauthor{Гордон}[Дж.]
Конструкции, или почему не~ломаются вещи.
\mirpublisher,
1980.
\howmanypages{390~с.}

\bibitem{gordon-whyyoudontfallthru}
\bibauthor{Gordon,}[James~E.] The new science of strong materials, or Why you don’t fall through the floor. Penguin Books, 1968. \howmanypages{269~pages.}
\emph{Перевод:}
\bibauthor{Гордон}[Дж.]
Почему мы не~проваливаемся сквозь пол.
\mirpublisher,
1971.
\howmanypages{272~с.}

% by deWit, Roland
% Theory of disclinations:
% II. Continuous and discrete disclinations in anisotropic elasticity
% III. Continuous and discrete disclinations in isotropic elasticity
% IV. Straight disclinations https://pdfs.semanticscholar.org/8834/3a6514cd502a5fa0f40f5bae4a87101deef8.pdf

\bibitem{dewit-disclinations}
\emph{Перевод:}
\bibauthor{Де~Вит}[Р.]
Континуальная теория дисклинаций.
\mirpublisher,
1977.
\howmanypages{208~с.}

\bibitem{janelidzepanovko-thinwalledrods}
\bibauthor{Джанелидзе}[Г.][Ю.],
\bibauthor{Пановко}[Я.][Г.]
Статика упругих тонкостенных стержней.
Л.,\:М.:\;Гостехиздат, 1948.
\howmanypages{208~с.}

\bibitem{dimitrienko-tensorcalculus}
\bibauthor{Димитриенко}[Ю.][И.]
Тензорное исчисление: Учебное пособие для вузов.
М.:\;\inquotes{Высшая школа}, 2001.
\howmanypages{575~с.}

\bibitem{eliseev-models}
\bibauthor{Владимир~В.\;Елисеев}
\href{https://www.researchgate.net/publication/320895320_Odnomernye_i_trehmernye_modeli_v_mehanike_uprugih_sterznej}%
{Одномерные и~трёхмерные модели в~механике упругих стержней. Диссертация
на соискание учёной степени
доктора
физико-математических наук.
ЛГТУ, 1991.}
\howmanypages{300~с.}

\bibitem{eshelby-theoryofdislocations} % John Douglas Eshelby
\bibauthor{Eshelby,}[John~D.] The continuum theory of lattice defects~//~Solid State Physics, Academic Press, vol.\:3, 1956, pp.\:79\hbox{--}144.
\emph{Перевод:} \bibauthor{Эшелби}[Дж.] Континуальная теория дислокаций. М.:\;ИИЛ, 1963. \howmanypages{247~с.}

\bibitem{zhuravlyov.osnovyteoreticheskoymehaniki}
\bibauthor{Журавлёв}[В.][Ф.]
Основы теоретической механики.
3-е издание, переработанное.
М.: ФИЗМАТЛИТ,
2008.
\howmanypages{304~с.}

\bibitem{zubov}
\bibauthor{Зубов}[Л.][М.]
Методы нелинейной теории упругости в~теории оболочек.
Изд\hbox{-}во Ростовского ун\hbox{-}та, 1982.
\howmanypages{144~с.}

\bibitem{kac-theoryofelasticity}
\bibauthor{Кац, Арнольд М. }
Теория упругости.
2-е издание, стереотипное.
{Санкт-Петербург}:
Издательство <<Лань>>, 2002.
\howmanypages{208~с.}

\bibitem{kachanov-fracturemechanics}
\bibauthor{Качанов}[Л.][М.]
Основы механики разрушения.
\naukapublisher,
1974.
\howmanypages{312~с.}

\bibitem{kerstein.klyushnikov.lomakin.shesterikov-experimentalfracturemechanics}
\bibauthor{Керштейн}[И.][М.], \bibauthor{Клюшников}[В.][Д.], \bibauthor{Ломакин}[Е.][В.], \bibauthor{Шестериков}[С.][А.]
Основы экспериментальной механики разрушения. Изд\hbox{-}во~МГУ, 1989. \howmanypages{140~с.}

\bibitem{cosserat}
\bibauthor{Cosserat}[E.] et \bibauthor{Cosserat}[F.]
\href{
   https://jscholarship.library.jhu.edu/bitstream/handle/1774.2/34209/31151000327233.pdf
}{
   Théorie des corps déformables.
   Paris:
   A.\:Hermann et~Fils,
   1909.
}
\howmanypages{226~p.}

\bibitem{cottrell-dislocations} % Theory of crystal dislocations by Alan Cottrell
\bibauthor{Cottrell,}[Alan]. Theory of crystal dislocations. (Documents on Modern Physics) Gordon and Breach, 1964. \howmanypages{94~p.}
\emph{Перевод:} \bibauthor{Коттрел}[А.] Теория дислокаций. \mirpublisher, 1969. \howmanypages{96~с.}

\bibitem{kroener-kontinuumstheorie}
\bibauthor{Kröner,}[Ekkehart] \emph{(i)}~Kontinuumstheorie der~Versetzungen und Eigen\-spannung\-en. Springer\hbox{-}Verlag, 1958. \howmanypages{180~pages.}
\emph{(ii)}~Allgemeine Kontinuumstheorie der~Versetzungen und Eigen\-span\-nung\-en~//~Archive for Rational Mechanics and Analysis. Volume~4, Issue~1 (January 1959), pp.\:273\hbox{--}334.
\emph{Перевод:} \bibauthor{Крёнер}[Э.] Общая континуальная теория дислокаций и~собственных напряжений. \mirpublisher, 1965. \howmanypages{104~с.}

\bibitem{lurie-nonlinearelasticity}
\bibauthor{Лурье}[А.][И.] Нелинейная теория упругости. \naukapublisher, 1980. \howmanypages{512~с.}
\emph{Translation:}
\bibauthor{Lurie,}[A.][I.] Nonlinear Theory of~Elasticity: translated from the~Russian by~K.\,A.\;Lurie. Elsevier Science Publishers B.\hspace{.1ex}V\hspace{-0.2ex}., 1990. \howmanypages{617~p.}

\bibitem{lurie-theoryofelasticity}
\bibauthor{Лурье}[А.][И.]
Теория упругости.
\naukapublisher, 1970.
\howmanypages{940~с.}
\emph{Translation:}
\bibauthor{Lurie,}[A.][I.]
Theory of~Elasticity (translated by~A.\:Belyaev).
Springer-Verlag, 2005.
\howmanypages{1050~p.}

\bibitem{lurie-spatialproblems}
\bibauthor{Лурье}[А.][И.]
Пространственные задачи теории упругости.
М.:\;Гос\-тех\-издат, 1955.
\howmanypages{492~с.}

\bibitem{lurie-thinwalledshells}
\bibauthor{Лурье}[А.][И.]
Статика тонкостенных упругих оболочек.
М.,\:Л.:\;Гос\-тех\-издат, 1947.
\howmanypages{252~с.}

\bibitem{love-mathematicaltheoryofelasticity}
\bibauthor{Augustus Edward Hough Love}.
A~treatise on the mathematical theory of elasticity.
\href{https://hal.archives-ouvertes.fr/hal-01307751/document}{
Volume~I.
Cambridge, 1892.
\howmanypages{354~p.}
}
\href{https://archive.org/details/in.ernet.dli.2015.503659}{
Volume~II.
Cambridge, 1893.
\howmanypages{327~p.}
}
\href{https://archive.org/details/in.ernet.dli.2015.462644/page/n1}{
4th~edition.
Cambridge, 1927.
Dover, 1944.
\howmanypages{643~p.}
}
\emph{Перевод:}
\bibauthor{Аугустус Ляв}
Математическая теория упругости.
М.:\;ОНТИ,
1935.
\howmanypages{674~с.}

\bibitem{mase-continuummechanics}
\bibauthor{George E. Mase}.
Schaum’s outline of theory and problems of continuum mechanics
(Schaum’s outline series).
McGraw\hbox{--}Hill,
1970.
\howmanypages{221~p.}
\emph{Перевод:}
\bibauthor{Джордж Мейз}.
Теория и~задачи механики сплошных сред.
Издание~3\hbox{-}е.
URSS, 2010.
\howmanypages{320~с.}

\bibitem{parkus.melan-waermespannungen} % Wärmespannungen infolge stationärer Temperaturfelder, von Ernst Melan und Heinz Parkus
\bibauthor{Ernst Melan}, \bibauthor{Heinz Parkus}. Wärmespannungen infolge stationärer Temperaturfelder. Wein, Springer-Verlag, 1953. \howmanypages{114~Seiten.}
\emph{Перевод:} \bibauthor{Мелан}[Э.], \bibauthor{Паркус}[Г.] Термоупругие напряжения, вызываемые стационарными температурными полями. \fizmatgiz, 1958. \howmanypages{167~с.}

\bibitem{merkin-threadmechanics}
\bibauthor{Меркин}[Д.][Р.] Введение в~механику гибкой нити. \naukapublisher, 1980. \howmanypages{240~с.}

\bibitem{merkin-stabilityintro}
\bibauthor{Меркин}[Д.][Р.] Введение в~теорию устойчивости движения. 3\hbox{-}е~издание. \naukapublisher, 1987. \howmanypages{304~с.}

\bibitem{mindlin.tiersten}
\bibauthor{Mindlin,}[Raymond~David] % Raymond David Mindlin
and~\bibauthor{Tiersten,}[Harry~F.] % Harry F. Tiersten
Effects of couple-stresses in linear elasticity~//~Archive for Rational Mechanics and Analysis. Volume~11, Issue~1 (January 1962), pp.\:415\hbox{--}448.
\emph{Перевод:}
\bibauthor{Миндлин}[Р.][Д.], \bibauthor{Тирстен}[Г.][Ф.]
Эффекты моментных напряжений в~линейной теории упругости~//~Механика: Сборник переводов и~обзоров иностранной периодической литературы. \mirpublisher, 1964. №\,4\:(86). С.\:80\hbox{--}114.

\bibitem{morozov-fractures}
\bibauthor{Морозов}[Н.][Ф.] Математические вопросы теории трещин. \naukapublisher, 1984. \howmanypages{256~с.}

\bibitem{naghdi-theoryofshellsandplates}
\bibauthor{Naghdi}[P.][M.] The theory of shells and plates. In: Truesdell~C.\:(editor) Mechanics of~solids. Volume II. Linear theories of~elasticity and~thermoelasticity. Linear and~nonlinear theories of~rods, plates, and~shells. Springer\hbox{-}Verlag, 1973. Pages~425\hbox{--}640.

\bibitem{nowacki-problemsofthermoelasticity}
\bibauthor{Witold Nowacki}.
Dynamiczne zagadnienia termosprężystości.
Warsza\-wa: Państwowe wydawnictwo naukowe, 1966. \howmanypages{366~stron.}
\emph{Translation:}
\bibauthor{Nowacki,\;Witold}.
Dynamic problems of~thermoelasticity.
Leyden: Noordhoff international publishing, 1975.
\howmanypages{436~pages.}
\emph{Перевод:}
\bibauthor{Витольд Новацкий}.
Динамические задачи термоупругости.
\mirpublisher, 1970.
\howmanypages{256~с.}

\bibitem{nowacki-elasticity}
\bibauthor{Witold Nowacki}.
Teoria sprężystości.
Warszawa: Państwowe wy\-daw\-nic\-two naukowe, 1970. \howmanypages{769~stron.}
\emph{Перевод:}
\bibauthor{Новацкий Витольд}.
Теория упругости.
\mirpublisher, 1975.
\howmanypages{872~с.}

\bibitem{nowacki-electromagneticeffects}
\bibauthor{Witold Nowacki}. Efekty elektromagnetyczne w stałych ciałach od\-kształ\-cal\-nych. Państwowe wydawnictwo naukowe, 1983. \howmanypages{147~stron.}
\emph{Перевод:}
\bibauthor{Новацкий}[В.] Электромагнитные эффекты в~твёрдых телах. \mirpublisher, 1986. \howmanypages{160~с.}

\bibitem{novozhilov-theoryofthinshells}
\bibauthor{Новожилов}[В.][В.] Теория тонких оболочек. 2\hbox{-}е~издание. Л.:\;Судпромгиз, 1962. \howmanypages{431~с.}

\bibitem{panovko.beylin-thinwalledrods}
\bibauthor{Пановко}[Я.][Г.], \bibauthor{Бейлин}[Е.][А.] Тонкостенные стержни и~системы, составленные из тонкостенных стержней. В~сборнике: Рабинович~И.\:М.\:(редактор) Строительная механика в~СССР 1917\hbox{--}1967. М.:\;Строй\-издат, 1969. С.\:75\hbox{--}98.

\bibitem{panovko.gubanova-stabilityandvibrations}
\bibauthor{Пановко}[Я.][Г.], \bibauthor{Губанова}[И.][И.] Устойчивость и~колебания упругих систем. Современные концепции, парадоксы и~ошибки. 4\hbox{-}е~издание. \naukapublisher, 1987. \howmanypages{352~с.}

\bibitem{parkus-waermespannungen} % Instationäre Wärmespannungen von Heinz Parkus
\bibauthor{Heinz Parkus}. Instation\"{a}re W\"{a}rmespannungen. Springer\hbox{-}Verlag, 1959. \howmanypages{176~Seiten.}
\emph{Перевод:} \bibauthor{Паркус}[Г.] Неустановившиеся температурные напряжения. \fizmatgiz, 1963. \howmanypages{252~с.}

\bibitem{parton-fracturemechanics}
\bibauthor{Партон}[В.][З.] Механика разрушения: от~теории к~практике. \naukapublisher, 1990. \howmanypages{240~с.}

\bibitem{parton-electromagneticelasticity}
\bibauthor{Партон}[В.][З.], \bibauthor{Кудрявцев}[Б.][А.] Электромагнитоупругость пьезоэлектрических и~электропроводных тел. \naukapublisher, 1988. \howmanypages{472~с.}

\bibitem{parton.morozov-destructionofelastoplastic}
\bibauthor{Партон}[В.][З.], \bibauthor{Морозов}[Е.][М.] Механика упругопластического разрушения. 2\hbox{-}е~издание. \naukapublisher, 1985. \howmanypages{504~с.}

\bibitem{podstrigach.burak.kondrat-magnetothermoelasticity}
\bibauthor{Подстригач}[Я.][С.], \bibauthor{Бурак}[Я.][И.], \bibauthor{Кондрат}[В.][Ф.]
Магнито\-термо\-упру\-гость электропроводных тел.
Киев:\;Наукова~думка, 1982.
\howmanypages{296~с.}

\bibitem{poruchikov-dynamicelasticity}
\bibauthor{Поручиков}[В.][Б.] Методы динамической теории упругости. \naukapublisher, 1986.
\howmanypages{328~с.}

\bibitem{southwell-introductiontotheoryofelasticity} % An Introduction to the Theory of Elasticity for Engineers and Physicists by Richard Vynne Southwell, originally published 1936
\bibauthor{Southwell,}[Richard~V.]
An~introduction to the~theory of~elasticity for engineers and physicists.
Dover Publications, 1970.
\howmanypages{509~pages.}
\emph{Перевод:}
\bibauthor{Саусвелл}[Р.][В.]
Введение в~теорию упругости для~инженеров и~физиков.
М.:\;ИИЛ, 1948.
\howmanypages{675~с.}

\bibitem{sedov-continuummechanics}
\bibauthor{Седов}[Л.][И.] Механика сплошной среды. Том~2. 6\hbox{-}е~издание. <<Лань>>, 2004.
\howmanypages{560~с.}

\bibitem{ciarlet-mathematicalelasticity}
\bibauthor{Ciarlet,}[Philippe~G.]
Mathematical elasticity.
Volume~1:~Three\hbox{-}dimensional elasticity.
Elsevier Science Publishers B.\hspace{.1ex}V\hspace{-0.2ex}.,
1988.
\howmanypages{xlii\:+\:452~pp.} % (Studies in Mathematics and its Applications, vol.\:20.)
\emph{Перевод:}
\bibauthor{Филипп Сьярле}
Математическая теория упругости.
\mirpublisher,
1992.
\howmanypages{472~с.}

\bibitem{Mémoire sur la torsion des prismes}
\bibauthor{Adhémar-Jean-Claude Barré de Saint\hbox{-\hspace{-0.2ex}}Venant}.
%
Mémoire sur la torsion des prismes, avec des considérations sur leur flexion ainsi que sur l'équilibre intérieur des solides élastiques en général, et des formules pratiques pour le calcul de leur résistance à divers efforts s’exerçant simultanément.
%
Memoires presentes par divers savants a l'Academie des scienees, t. 14, année~1856.
\howmanypages{327~pages.}
\emph{Перевод на русский язык:}
\bibauthor{Сен-Венан Б.}
Мемуар о~кручении призм.
Мемуар об~изгибе призм.
М.:\;Физ\-мат\-гиз, 1961.
\howmanypages{518~страниц.}
%
\bibitem{Memoire sur la flexion des prismes}
\bibauthor{Adhémar-Jean-Claude Barré de Saint\hbox{-\hspace{-0.2ex}}Venant}.
%.....................
Journal de mathematiques pures et appliquees, publie par J. Liouville.
2me serie, t. 1, année~1856.
%
\emph{Перевод на русский язык:}
\bibauthor{Сен-Венан Б.}
Мемуар о~кручении призм.
Мемуар об~изгибе призм.
М.:\;Физ\-мат\-гиз, 1961.
\howmanypages{518~страниц.}

\bibitem{teodosiu-crystaldefects} % Elastic Models of Crystal Defects by Cristian Teodosiu
\bibauthor{Teodosiu,}[Cristian]. Elastic models of crystal defects. Springer\hbox{-}Verlag, 1982. \howmanypages{336~pages.}
\emph{Перевод:} \bibauthor{Теодосиу}[К.] Упругие модели дефектов в~кристаллах. \mirpublisher, 1985. \howmanypages{352~с.}

\bibitem{timoshenko-stability}
\bibauthor{Тимошенко Степан П.}
Устойчивость стержней, пластин и~оболочек.
\naukapublisher, 1971.
\howmanypages{808~с.}

\bibitem{timoshenko-platesnshells}
\bibauthor{Тимошенко Степан П.},
\bibauthor{Войновский\hbox{-}Кригер}[С.]
Пластинки и~оболочки.
\naukapublisher, 1966.
\howmanypages{635~с.}

\bibitem{timoshenkogoodier}
\bibauthor{Stephen P.}[Timoshenko] and \bibauthor{James N.}[Goodier].
Theory of~Elasticity.
2nd~edition. McGraw\hbox{--}Hill, 1951. \howmanypages{506~pages.}
3rd~edition. McGraw\hbox{--}Hill, 1970. \howmanypages{567~pages.}
\emph{Перевод:}
\bibauthor{Тимошенко Степан П.}, \bibauthor{Джеймс Гудьер}.
Теория упругости.
2\hbox{-}е~издание.
\naukapublisher, 1979.
\howmanypages{560~с.}

\bibitem{truesdell-firstcourse} % A First Course in Rational Continuum Mechanics by Clifford Ambrose Truesdell
% Truesdell, C. A First Course in Rational Continuum Mechanics. The Johns Hopkins University, Baltimore, Maryland, 1972.
\bibauthor{Truesdell,}[Clifford~A.] A~first course in~rational continuum mechanics. Volume~1:~General concepts. 2nd~edition. Academic Press, 1991. \howmanypages{391~pages.} % (Pure and Applied Mathematics, vol.\:71.)
\emph{Перевод:}
\bibauthor{Трусделл}[К.] Первоначальный курс рациональной механики сплошных сред. \mirpublisher, 1975. \howmanypages{592~с.}

\bibitem{feodosiev-talks}
\bibauthor{Феодосьев}[В.][И.]
Десять лекций\hbox{-}бесед по~сопротивлению материалов.
2\hbox{-}е~издание.
\naukapublisher, 1975.
\howmanypages{173~с.}

\bibitem{hellan-fracturemechintro} % Introduction to Fracture Mechanics by Kåre Hellan, originally published 1984
\emph{Перевод:}
\bibauthor{Хеллан}[К.]
Введение в~механику разрушения.
\mirpublisher,
1988.
\howmanypages{364~с.}

\bibitem{ziegler-structuralstability} % Principles of structural stability by Hans Ziegler, originally published 1968
\emph{Перевод:}
\bibauthor{Циглер}[Г.]
Основы теории устойчивости конструкций.
\mirpublisher, 1971.
\howmanypages{192~с.}

\bibitem{cherepanov-fragilefracture}
\bibauthor{Черепанов}[Г.][П.]
Механика хрупкого разрушения.
\naukapublisher, 1974.
\howmanypages{640~с.}

\bibitem{chernyh-anisotropicelasticity}
\bibauthor{Черн\'{ы}х}[К.][Ф.]
Введение в~анизотропную упругость.
\naukapublisher,
1988.
\howmanypages{192~с.}

\bibitem{chernyh-nonlinearelasticity}
\bibauthor{Черн\'{ы}х}[К.][Ф.]
Нелинейная теория упругости в~машиностроительных расчетах.
\maschinenbauleningraderverlag,
1986.
\howmanypages{336~с.}

\bibitem{shermergor}
\bibauthor{Шермергор}[Т.][Д.] Теория упругости микронеоднородных сред. \naukapublisher, 1977. \howmanypages{400~с.}

%
% oscillations and waves
%

\en{\subsection*{Oscillations and waves}}

\ru{\subsection*{Колебания и волны}}

\bibitem{timoshenko.young.weaver}
\bibauthor{Timoshenko,}[Stephen~P.];
\bibauthor{Young,}[Donovan~H.];
\bibauthor{William Weaver,}[jr.]
Vibration problems in engineering.
5th~edition.
John~Wiley~\&~Sons, 1990.
\howmanypages{624~pages.}
\emph{Перевод:} \bibauthor{Тимошенко Степан П.},
\bibauthor{Янг Донован Х.},
\bibauthor{Уильям Уивер}.
Колебания в~инженерном деле.
\maschinenbaumoskauerverlag, 1985.
\howmanypages{472~с.}

\bibitem{babakov-theoryofoscillations}
\bibauthor{Бабаков}[И.][М.] Теория колебаний. 4\hbox{-}е~издание. <<Дрофа>>, 2004. \howmanypages{592~с.}

\bibitem{biderman-oscillations}
\bibauthor{Бидерман}[В.][Л.] Теория механических колебаний. М.:\;Высшая школа, 1980. \howmanypages{408~с.}

\bibitem{bolotin-randomoscillations}
\bibauthor{Болотин}[В.][В.] Случайные колебания упругих систем. \naukapublisher, 1979. \howmanypages{336~с.}

\bibitem{grinchenko.meleshko}
\bibauthor{Гринченко}[В.][Т.], \bibauthor{Мелешко}[В.][В.] Гармонические колебания и~волны в~упругих телах. Киев:\;Наукова думка, 1981. \howmanypages{284~с.}

\bibitem{whitham-waves}
\bibauthor{Whitham,}[Gerald~B.] Linear and~nonlinear waves. John~Wiley~\&~Sons, 1974. \howmanypages{636~pages.}
\emph{Перевод:} \bibauthor{Уизем}[Дж.] Линейные и~нелинейные волны. \mirpublisher, 1977. \howmanypages{624~с.}

\bibitem{kolsky-stresswavesinsolids}
\bibauthor{Kolsky,}[Herbert]. Stress waves in solids. Oxford, Clarendon Press, 1953. \howmanypages{211~p.} 2nd~edition. Dover Publications, 2012. \howmanypages{224~p.}
\emph{Перевод:} \bibauthor{Кольский}[Г.] Волны напряжения в~твёрдых телах. М.:\;ИИЛ, 1955. \howmanypages{192~с.}

\bibitem{engelbrecht.nigul-nonlineardeformationwaves} % Uno Karlovich Nigul, Jüri Engelbrecht
\bibauthor{Энгельбрехт}[Ю.][К.], \bibauthor{Нигул}[У.][К.] Нелинейные волны деформации. \naukapublisher, 1981. \howmanypages{256~с.}

\bibitem{slepyan-nonstationeryelasticwaves}
\bibauthor{Слепян}[Л.][И.] Нестационарные упругие волны. Л.:\;Судостроение, 1972. \howmanypages{376~с.}

\bibitem{grigolyuk.selezov}
\bibauthor{Григолюк}[Э.][И.], \bibauthor{Селезов}[И.][Т.] Неклассические теории колебаний стержней, пластин и~оболочек. (Итоги науки и~техники. Механика твёрдых деформируемых тел. Том~5.) М.:\;ВИНИТИ, 1973. \howmanypages{272~с.}

%
% composites
%

\en{\subsection*{Composites}}

\ru{\subsection*{Композиты}}

\bibitem{christensen-compositematerials}
\bibauthor{Christensen,}[Richard~M.] Mechanics of~composite materials. New~York: Wiley, 1979. \howmanypages{348~p.}
\emph{Перевод:} \bibauthor{Кристенсен}[Р.] Введение в~механику композитов. \mirpublisher, 1982. \howmanypages{336~с.}

\bibitem{kravchuk.mayboroda.urzhumtsev-polymericandcompositematerials}
\bibauthor{Кравчук}[А.][С.], \bibauthor{Майборода}[В.][П.], \bibauthor{Уржумцев}[Ю.][С.] Механика полимерных и~композиционных материалов. Экспериментальные и численные методы. \naukapublisher, 1985. \howmanypages{304~с.}

\bibitem{pobedrya-composites}
\bibauthor{Победря}[Б.][Е.] Механика композиционных материалов. Изд\hbox{-}во~Моск.\:ун\hbox{-}та, 1984. \howmanypages{336~с.}

\bibitem{cherepanov-compositematerialfracture}
\bibauthor{Черепанов}[Г.][П.] Механика разрушения композиционных материалов. \naukapublisher, 1983. \howmanypages{296~с.}

\bibitem{bakhvalov.panasenko}
\bibauthor{Бахвалов}[Н.][С.], \bibauthor{Панасенко}[Г.][П.] Осреднение процессов в~периодических средах. Математические задачи механики композиционных материалов. \naukapublisher, 1984. \howmanypages{352~с.}

\bibitem{asymptoticanalysisforperiodicstructures}
\bibauthor{Bensoussan}[A.], % Alain Bensoussan
\bibauthor{Lions}[J.-L.], % Jacques-Louis Lions
\bibauthor{Papanicolaou}[G.]
Asymptotic analysis for periodic structures. Amsterdam: North\hbox{-}Holland, 1978. \howmanypages{700~p.}

%
% finite element method
%

\en{\subsection*{The finite element method}}

\ru{\subsection*{Метод конечных элементов}}

\bibitem{zienkiewicz.morgan-finiteelementsandapproximation}
\bibauthor{Зенкевич}[О.], \bibauthor{Морган}[К.] Конечные элементы и~аппроксимация. \mirpublisher, 1986. \howmanypages{318~с.}

\bibitem{shabrov-finiteelementmethod}
\bibauthor{Шабров}[Н.][Н.] Метод конечных элементов в~расчётах деталей тепловых двигателей. \maschinenbauleningraderverlag, 1983. \howmanypages{212~с.}

%
% mechanics, thermodynamics, electromagnetism
%

\en{\subsection*{Mechanics, thermodynamics, electromagnetism}}

\ru{\subsection*{Механика, термодинамика, электромагнетизм}}

% The Feynman Lectures on Physics by Richard Phillips Feynman, Robert B. Leighton and Matthew Sands
% Volume I: Mainly mechanics, radiation, and heat
% Volume II: Mainly electromagnetism and matter
% Volume III: Quantum mechanics

\bibitem{feynman-lecturesonphysics}
\bibauthor{Feynman,}[Richard Ph.] ${\hspace{-0.2ex}\bigdot}$ \bibauthor{Leighton,}[Robert B.] ${\hspace{-0.2ex}\bigdot}$ \bibauthor{Sands,}[Matthew]. The Feynman Lectures on Physics. New millennium edition. Volume II: Mainly electromagnetism and matter. Basic Books, 2011. \howmanypages{566~pages.}
\emph{Online:}
\href{http://www.feynmanlectures.caltech.edu/}{The Feynman Lectures on Physics. Online edition.}
%%\emph{Перевод:}
%%\bibauthor{Фейнман}[Р.], \bibauthor{Лейтон}[Р.], \bibauthor{Сэндс}[М.]
%%Фейнмановские лекции по физике.
%%Том~5: Электричество и магнетизм. Издание~8\hbox{-}е. URSS, 2014. \howmanypages{310~с.}
%%Том~6: Электродинамика. Издание~9\hbox{-}е. URSS, 2016. \howmanypages{352~с.}

% Classical Mechanics by Herbert Goldstein
%
% In 2001, a new (third) edition of the book was released, with the collaboration of Charles P. Poole and John L. Safko

\bibitem{goldstein-classicalmechanics}
\bibauthor{Goldstein,}[Herbert]; \bibauthor{Poole,}[Charles~P.]; \bibauthor{Safko,}[John L.] Classical Mechanics. 3rd~edition. Addison\hbox{--}Wesley, 2001. \howmanypages{638~pages.}
\emph{Перевод:} \bibauthor{Голдстейн}[Г.], \bibauthor{Пул}[Ч.], \bibauthor{Сафко}[Дж.] Классическая механика. URSS, 2012. \howmanypages{828~с.}

% Leopold Alexander “Alan” Pars is most remembered for his textbooks Introduction to dynamics (1953), Calculus of variations (1962), and his monumental 650 page Treatise on analytical dynamics (1965)

\bibitem{treatiseonanalyticaldynamics-by-l.a.pars} % A Treatise on Analytical Dynamics by Leopold A. Pars, originally published London: Heinemann, 1965
\bibauthor{Pars,}[Leopold~A.] A~treatise on analytical dynamics.
London: Heinemann, 1965. \howmanypages{641~pages.}
\emph{Перевод:} \bibauthor{Парс}[Л.][А.] Аналитическая динамика. \naukapublisher, 1971. \howmanypages{636~с.}

\bibitem{terhaar-hamiltonianmechanics} % Elements of Hamiltonian Mechanics by Dirk Ter Haar
\bibauthor{Ter Haar,}[Dirk]. Elements of hamiltonian mechanics. 2nd~edition. Pergamon Press, 1971. \howmanypages{201~pages.}
\emph{Перевод:} \bibauthor{Тер~Хаар}[Д.] Основы гамильтоновой механики. \naukapublisher, 1974. \howmanypages{223~с.}

\bibitem{belyaev.ryadno}
\bibauthor{Беляев}[Н.][М.], \bibauthor{Рядно}[А.][А.] Методы теории теплопроводности. М.:\;Высшая школа, 1982. В~2\hbox{-}х томах.
\howmanypages{Том~1, 328~с.}
\howmanypages{Том~2, 304~с.}

\bibitem{classicalelectrodynamics}
\bibauthor{Бредов}[М.][М.], \bibauthor{Румянцев}[В.][В.], \bibauthor{Топтыгин}[И.][Н.] Классическая электродинамика. \naukapublisher, 1985. \howmanypages{400~с.}

\bibitem{gantmacher}
\bibauthor{Гантмахер}[Ф.][Р.] Лекции по~аналитической механике. Издание~2\hbox{-}е. \naukapublisher, 1966. \howmanypages{300~с.}

\bibitem{landau.lifshitz-shortcourse}
\bibauthor{Ландау}[Л.][Д.], \bibauthor{Лифшиц}[Е.][М.] Краткий курс теоретической физики. Книга~1. Механика. Электродинамика. \naukapublisher, 1969. \howmanypages{271~с.}
% Лев Давидович Ландау, Евгений Михайлович Лифшиц

\bibitem{loitsjanskiy.lurie}
\bibauthor{Лойцянский}[Л.][Г.], \bibauthor{Лурье}[А.][И.] Курс теоретической механики: В~2\hbox{-}х томах. <<Дрофа>>, 2006.
Том~1:~Статика и~кинематика. 9\hbox{-}е~издание. \howmanypages{447~с.}
Том~2:~Динамика. 7\hbox{-}е~издание. \howmanypages{719~с.}

\bibitem{lurie-analyticalmechanics}
\bibauthor{Лурье}[А.][И.] Аналитическая механика. \fizmatgiz, 1961. \howmanypages{824~с.}

\bibitem{olkhovskiy-theoreticalmechanicsforphysicists}
\bibauthor{Ольховский}[И.][И.] Курс теоретической механики для~физиков. 3\hbox{-}е~издание. Изд\hbox{-}во~МГУ, 1978. \howmanypages{575~с.}

\bibitem{tamm-electricity}
\bibauthor{Тамм}[И.][Е.] % Тамм Игорь Евгеньевич
Основы теории электричества. 11\hbox{-}е~издание. М.:\;Физматлит, 2003.
\howmanypages{616~с.}
% http://www.samomudr.ru/d/Tamm%20I.E.%20_Osnovy%20teorii%20Elektrichestva_616str_2003g.pdf

%
% tensors
%

\en{\subsection*{Tensors and tensor calculus}}

\ru{\subsection*{Тензоры и тензорное исчисление}}

\bibitem{mcconnell-tensoranalysis} % Applications of Tensor Analysis by Albert Joseph McConnell
\bibauthor{McConnell,}[Albert Joseph]. Applications of tensor analysis. New~York: Dover Publications, 1957. \howmanypages{318 pages.}
\emph{Перевод:} \bibauthor{Мак\hbox{-}Коннел}[А.][Дж.] Введение в~тензорный анализ с~приложениями к~геометрии, механике и~физике. \fizmatgiz, 1963. \howmanypages{412~с.}

\bibitem{schouten-tensoranalysis} % Tensor analysis for physicists by Jan Arnoldus Schouten
\bibauthor{Schouten,}[Jan~A.] Tensor analysis for physicists. 2nd~edition. Dover Publications, 2011. \howmanypages{320~pages.}
\emph{Перевод:} \bibauthor{Схоутен}[Я.][А.] Тензорный анализ для~физиков. \naukapublisher, 1965. \howmanypages{456~с.}

\bibitem{sokolnikoff-tensoranalysis}
\bibauthor{Sokolnikoff,}[I.][S.] Tensor analysis: Theory and applications to geometry and mechanics of~continua. 2nd~edition. John~Wiley~\&~Sons, 1965. \howmanypages{361~pages.}
\emph{Перевод:} \bibauthor{Сокольников}[И.][С.] Тензорный анализ (с~приложениями к~геометрии и~механике сплошных сред). \naukapublisher, 1971. \howmanypages{376~с.}

\bibitem{rashevsky-riemanniangeometry}
\bibauthor{Рашевский}[П.][К.] Риманова геометрия и~тензорный анализ. Издание~3\hbox{-}е. \naukapublisher, 1967. \howmanypages{664~с.}

%
% variational methods
%

\en{\subsection*{Variational methods}}

\ru{\subsection*{Вариационные методы}}

\bibitem{rektorys-variationalmethods}
\bibauthor{Karel~Rektorys}. Varia\v{c}ní metody v in\v{z}en\'{y}rsk\'{y}ch probl\'{e}mech a~v~pro\-bl\'{e}\-mech matematick\'{e} fyziky. SNTL (St\'{a}tní nakladatelství technick\'{e} literatury), 1974. \howmanypages{593~s.}
\emph{Translation:}
\bibauthor{Rektorys,}[Karel]. Variational Methods in Mathematics, Science and Engineering. Second edition. D.\,Reidel Publishing Company, 1980. \howmanypages{571~p.}
\emph{Перевод:}
\bibauthor{Ректорис}[К.] Вариационные методы в~математической физике. \mirpublisher, 1985. \howmanypages{590~с.}

\bibitem{washizubook} % Variational Methods in Elasticity and Plasticity by Kyuichiro Washizu, originally published: 1968
\bibauthor{Washizu,}[Kyuichiro]. Variational methods in elasticity and plasticity. 3rd~edition. Pergamon Press, Oxford, 1982. \howmanypages{630~pages.}
\emph{Перевод:} \bibauthor{Васидзу}[К.] Вариационные методы в~теории упругости и~пластичности. \mirpublisher, 1987. \howmanypages{542~с.}

\bibitem{berdichevsky}
\bibauthor{Бердичевский}[В.][Л.] Вариационные принципы механики сплошной среды. \naukapublisher, 1983. \howmanypages{448~с.}

\bibitem{mihlin-variationalmethods}
\bibauthor{Михлин}[С.][Г.] Вариационные методы в~математической физике. Издание~2\hbox{-}е. \naukapublisher, 1970. \howmanypages{512~с.}

%
% perturbation (asymptotic) methods
%

\en{\subsection*{Perturbation methods (asymptotic methods)}}

\ru{\subsection*{Методы возмущений (асимптотические методы)}}

\bibitem{cole-perturbationmethods}
\bibauthor{Cole,}[Julian~D.] Perturbation methods in applied mathematics. Blaisdell Publishing Co., 1968. \howmanypages{260~pages.}
\emph{Перевод:} \bibauthor{Коул}[Дж.] Методы возмущений в~прикладной математике. \mirpublisher, 1972. \howmanypages{274~с.}

\bibitem{nayfeh-introtoperturbation} % Introduction to perturbation techniques by Ali Hasan Nayfeh
\bibauthor{Nayfeh,}[Ali~H.] Introduction to perturbation techniques. Wiley, 1981. \howmanypages{536~pages.}
\emph{Перевод:} \bibauthor{Найфэ}[Али~Х.] Введение в~методы возмущений. \mirpublisher, 1984. \howmanypages{535~с.}

\bibitem{nayfeh-perturbation} % Perturbation methods by Ali Hasan Nayfeh
\bibauthor{Nayfeh,}[Ali~H.] Perturbation methods. Wiley-VCH, 2004. \howmanypages{425~pages.}
%%\emph{Перевод:} \bibauthor{Найфэ}[Али~Х.] Методы возмущений. \mirpublisher, 1976. \howmanypages{456~с.}

\bibitem{bogolyubovmitropolsky}
\bibauthor{Боголюбов}[Н.][Н.], \bibauthor{Митропольский}[Ю.][А.] Асимптотические методы в~теории нелинейных колебаний. \naukapublisher, 1974. \howmanypages{504~с.}

\bibitem{vasiljevabutuzov}
\bibauthor{Васильева}[А.][Б.], \bibauthor{Бутузов}[В.][Ф.] Асимптотические методы в~теории сингулярных возмущений. М.:\;Высшая школа, 1990. \howmanypages{208~с.}

\bibitem{zino.tropp}
\bibauthor{Зино}[И.][Е.], % Зино Игорь Евгеньевич
\bibauthor{Тропп}[Э.][А.] % Тропп Эдуард Абрамович
Асимптотические методы в~задачах теории теплопроводности и~термоупругости. Изд\hbox{-}во~ЛГУ, 1978. \howmanypages{224~с.}

\bibitem{moiseev-asymptoticalmethods}
\bibauthor{Моисеев}[Н.][Н.] Асимптотические методы нелинейной механики. 2\hbox{-}е~издание. \naukapublisher, 1981. \howmanypages{400~с.}

\bibitem{tovstik-thinwalledshellsstability}
\bibauthor{Товстик}[П.][Е.] Устойчивость тонких оболочек: асимптотические методы. \naukapublisher, 1995. \howmanypages{319~с.}

%
% other math
%

\en{\subsection*{Other topics of mathematics}}

\ru{\subsection*{Другие темы математики}}

\bibitem{collatz-eigenwertaufgaben}
\bibauthor{Collatz,}[Lothar]. Eigenwertaufgaben mit technischen Anwendungen. 2.\:Auflage. Akademische Verlagsgesellschaft Geest\;\&\;Portig, Leipzig, 1963. \howmanypages{500~Seiten.}
\emph{Перевод:} \bibauthor{Коллатц}[Л.] Задачи на~собственные значения~(с~техническими приложениями). \naukapublisher, 1968. \howmanypages{504~с.}

\bibitem{dwight-tables}
\bibauthor{Dwight,}[Herbert Bristol]. Tables of integrals and other mathematical data. 4th~edition. The~Macmillan~Co., 1961. \howmanypages{336~pages.}
\emph{Перевод:} \bibauthor{Двайт}[Г.][Б.] Таблицы интегралов и~другие математические формулы. Издание~4\hbox{-}е. \naukapublisher, 1973. \howmanypages{228~с.}

\bibitem{kamke-ordinarydifferentialequations}
\bibauthor{Kamke,}[Erich]. Differentialgleichungen, Lösungsmethoden und~Lö\-sun\-gen. Bd.\:I. Gewöhnliche Differentialgleichungen. 10.\:Auflage. Teubner Verlag, 1977. \howmanypages{670~Seiten.}
\emph{Перевод:} \bibauthor{Камке}[Э.] Справочник по~обыкновенным дифференциальным уравнениям. 6\hbox{-}е~издание. <<Лань>>, 2003. \howmanypages{576~с.}
%% 4\hbox{-}е~издание. \naukapublisher, 1971. \howmanypages{576~с.}

\bibitem{graninokorn.theresakorn-mathematicalhandbook} % Mathematical Handbook for Scientists and Engineers by Granino Arthur Korn and Theresa M. Korn
\bibauthor{Korn,}[Granino~A.] and \bibauthor{Korn,}[Theresa M.]
Mathematical handbook for scientists and engineers: definitions, theorems, and formulas for reference and review.
Revised edition. Dover Publications, 2013. \howmanypages{1152~pages.}
\emph{Перевод:}
\bibauthor{Корн}[Г.],
\bibauthor{Корн}[Т.]
Справочник по~математике для научных работников и~инженеров.
\naukapublisher,
1974.
\howmanypages{832~с.}

\bibitem{lavrentiev.shabat}
\bibauthor{Лаврентьев}[М.][А.],
\bibauthor{Шабат}[Б.][В.]
Методы теории функций комплексного переменного.
4\hbox{-}е~издание.
\naukapublisher,
1973.
\howmanypages{736~с.}

\bibitem{pogorelov-differentialgeometry}
\bibauthor{Погорелов}[А.][В.]
Дифференциальная геометрия.
Издание~6\hbox{-}е.
\naukapublisher,
1974.
\howmanypages{176~с.}

\end{otherlanguage}

\normalsize
\end{thebibliography}
