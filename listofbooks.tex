\thispagestyle{empty}

\newcommand\mirpublisher{<<Мир>>} % М.:\;Мир
\newcommand\naukapublisher{<<Наука>>} % М.:\;Наука
\newcommand\maschinenbaumoskauerverlag{М.:\;Машино\-строение}
\newcommand\maschinenbauleningraderverlag{Л.:\;Машино\-строение}
\newcommand\fizmatgiz{М.:\;Физ\-мат\-гиз} % Государственное издательство физико-математической литературы (<<Физматгиз>>)

\vspace*{10mm}

\begin{thebibliography}{123}
\interlinepenalty=10000 % inhibit page breaks inside a paragraph
\small

\addcontentsline{toc}{chapter}{\bibliographyname}

%% “Mechanics of Solids. Volume II. Linear Theories of Elasticity and Thermoelasticity. Linear and Nonlinear Theories of Rods, Plates, and Shells (Editor: C. Truesdell)” originally appeared in hardcover as Volume VIa/2 of Encyclopedia of Physics © Springer-Verlag (Berlin, Heidelberg) 1972

\begin{otherlanguage}{russian}

\publication{stuartantman-theoryofrods} % Stuart S. Antman
\bookauthor{Antman,}[Stuart~S.] The~theory of~rods. In: Truesdell~C.\:(editor) Mechanics of~solids. Volume II. Linear theories of~elasticity and~thermoelasticity. Linear and~nonlinear theories of~rods, plates, and~shells. Springer\hbox{-}Verlag, 1973. Pages~641\hbox{--}703.

\publication{alfutov-stabilitycalc}
\bookauthor{Алфутов}[Н.][А.]
Основы расчета на~устойчивость упругих систем. Издание~2\hbox{-}е.
\maschinenbaumoskauerverlag, 1991.
\howmanypages{336~с.}

\publication{artobolevskiyetother}
\bookauthor{Артоболевский}[И.][И.],
\bookauthor{Бобровницкий}[Ю.][И.],
\bookauthor{Генкин}[М.][Д.]
Введение в~акустическую динамику машин.
\naukapublisher, 1979.
\howmanypages{296~с.}

\publication{contact-problem-with-variable-contact-area.by-Ахтырец-and-Короткин}
\bookauthor{Ахтырец}[Г.][П.],
\bookauthor{Короткин}[В.][И.]
Использование МКЭ
при решении
контактной задачи
теории упругости
с~переменной
зоной контакта~//~Известия
северо-кавказского научного центра высшей школы
(СКНЦ ВШ).
Серия естественные науки.
Ростов-на-Дону:
Издательство РГУ,
1984.
№\:1.
С.\:38\hbox{--}42.

\publication{contact-problem-solution-via-finite-element-method.by-Ахтырец-and-Короткин}
\bookauthor{Ахтырец}[Г.][П.],
\bookauthor{Короткин}[В.][И.]
К~решению
контактной задачи
с~помощью
метода конечных элементов~//~Механика сплошной среды.
Ростов-на-Дону:
Издательство РГУ,
1988.
С.\:43\hbox{--}48.
%To the solution of the contact problem using the finite element method

\publication{biderman-thinwalled}
\bookauthor{Бидерман}[В.][Л.]
Механика тонкостенных конструкций.
\maschinenbaumoskauerverlag,
1977.
\howmanypages{488~с.}

\publication{veniaminblokh-theoryofelasticity}
\bookauthor{Вениамин И. Блох}.
Теория упругости.
Харьков:
Издательство Харьковского Государственного Университета,
1964.
\howmanypages{484~с.}

\publication{vlasov-thinwalledrods}
\bookauthor{Власов}[В.][З.]
Тонкостенные упругие стержни.
\fizmatgiz, 1959.
\howmanypages{568~с.}

\publication{goldenveizer-thinshells}
\bookauthor{Алексей Л. Гольденвейзер}.
Теория упругих тонких оболочек.
2-е издание. \naukapublisher, 1976.
\howmanypages{512~с.}
\emph{Translation:}
\bookauthor{Alexey L. Goldenveizer}.
Theory of elastic thin shells.
Pergamon Press, 1961.
\howmanypages{658~pages.}

\publication{goldenveizer.lidskiy.tovstik}
\bookauthor{Алексей Л. Гольденвейзер},
\bookauthor{Виктор Б. Лидский},
\bookauthor{Пётр Е. Товстик}.
Свободные ко\-леба\-ния тонких упругих оболочек.
\naukapublisher, 1979.
\howmanypages{384~с.}

\publication{gordon-constructions}
\bookauthor{Gordon,}[James~E.]
Structures, or Why things don’t fall down.
Penguin Books,
1978.
\howmanypages{395~pages.}
\emph{Перевод:}
\bookauthor{Гордон}[Дж.]
Конструкции, или почему не~ломаются вещи.
\mirpublisher,
1980.
\howmanypages{390~с.}

\publication{gordon-whyyoudontfallthru}
\bookauthor{Gordon,}[James~E.]
The new science of strong materials, or Why you don’t fall through the floor.
Penguin Books,
1968.
\howmanypages{269~pages.}
\emph{Перевод:}
\bookauthor{Гордон}[Дж.]
Почему мы не~проваливаемся сквозь пол.
\mirpublisher,
1971.
\howmanypages{272~с.}

\publication{alexandergouze-stability}
\bookauthor{Александр}[Н.][Гузь].
Устойчивость упругих тел при конечных деформациях.
Киев:\;\inquotes{Наукова думка}, 1973.
\howmanypages{271~с.}

% by Roland de Wit
% Theory of disclinations:
% II. Continuous and discrete disclinations in anisotropic elasticity
% III. Continuous and discrete disclinations in isotropic elasticity
% IV. Straight disclinations https://pdfs.semanticscholar.org/8834/3a6514cd502a5fa0f40f5bae4a87101deef8.pdf

\publication{dewit-disclinations}
\emph{Перевод:}
\bookauthor{Де~Вит}[Р.]
Континуальная теория дисклинаций.
\mirpublisher,
1977.
\howmanypages{208~с.}

\publication{janelidzepanovko-thinwalledrods}
\bookauthor{Джанелидзе}[Г.][Ю.],
\bookauthor{Пановко}[Я.][Г.]
Статика упругих тонкостенных стержней.
Л.,\:М.:\;Гостехиздат, 1948.
\howmanypages{208~с.}

\publication{dorin:models-of-elastic-rods}
\bookauthor{Dorin Ieşan}.
Classical and generalized models of elastic rods.
2nd~edition.
CRC~Press, Taylor\:\&\:Francis~Group,
2009.
\howmanypages{369~pages}

\publication{eliseev-models}
\bookauthor{Владимир В. Елисеев}.
\href{https://www.researchgate.net/publication/320895320_Odnomernye_i_trehmernye_modeli_v_mehanike_uprugih_sterznej}%
{Одномерные и~трёхмерные модели в~механике упругих стержней.
Диссертация
на соискание
учёной степени
доктора
физико-математических наук.
ЛГТУ, 1991.}
\howmanypages{300~с.}

\publication{eshelby-theoryofdislocations} % John Douglas Eshelby
\bookauthor{Eshelby,}[John~D.]
The continuum theory of lattice defects~//~Solid State Physics,
Academic Press,
vol.\:3, 1956,
pp.\:79\hbox{--}144.
\emph{Перевод:}
\bookauthor{Эшелби}[Дж.]
Континуальная теория дислокаций.
М.:\;ИИЛ,
1963.
\howmanypages{247~с.}

\publication{zhuravlyov.osnovyteoreticheskoymehaniki}
\bookauthor{Журавлёв}[В.][Ф.]
Основы теоретической механики.
3-е издание, переработанное.
М.: ФИЗМАТЛИТ,
2008.
\howmanypages{304~с.}

\publication{zubov}
\bookauthor{Зубов}[Л.][М.]
Методы нелинейной теории упругости в~теории оболочек.
Изд\hbox{-}во Ростовского ун\hbox{-}та, 1982.
\howmanypages{144~с.}

\publication{kac-theoryofelasticity}
\bookauthor{Кац, Арнольд М. }
Теория упругости.
2-е издание, стереотипное.
{Санкт-Петербург}:
Издательство <<Лань>>, 2002.
\howmanypages{208~с.}

\publication{ciarlet-mathematicalelasticity}
\bookauthor{Ciarlet,}[Philippe~G.]
Mathematical elasticity.
Volume~1\:: Three-dimensional elasticity.
Elsevier Science Publishers \hbox{B.\hspace{.1ex}V\hspace{-0.2ex}.},
1988.
\howmanypages{xlii\:+\:452~pp.} % (Studies in Mathematics and its Applications, vol.\:20.)
\emph{Перевод:}
\bookauthor{Филипп Сьярле}.
Математическая теория упругости.
\mirpublisher,
1992.
\howmanypages{472~с.}

\publication{cosserat}
\bookauthor{Cosserat}[E.] et \bookauthor{Cosserat}[F.]
\href{
   https://jscholarship.library.jhu.edu/bitstream/handle/1774.2/34209/31151000327233.pdf
}{
   Théorie des corps déformables.
   Paris:
   A.\:Hermann et~Fils,
   1909.
}
\howmanypages{226~p.}

% Theory of crystal dislocations by Alan Cottrell
\publication{cottrell-dislocations}
\bookauthor{Cottrell,}[Alan].
Theory of crystal dislocations.
Gordon and Breach
(Documents on Modern Physics),
1964.
\howmanypages{94~p.}
\emph{Перевод:}
\bookauthor{Коттрел}[А.]
Теория дислокаций.
\mirpublisher, 1969.
\howmanypages{96~с.}

\publication{kroener-kontinuumstheorie}
\bookauthor{Kröner,}[Ekkehart] \emph{(i)}~Kontinuumstheorie der~Versetzungen und Eigen\-spannung\-en. Springer\hbox{-}Verlag, 1958. \howmanypages{180~pages.}
\emph{(ii)}~Allgemeine Kontinuumstheorie der~Versetzungen und Eigen\-span\-nung\-en~//~Archive for Rational Mechanics and Analysis. Volume~4, Issue~1 (January 1959), pp.\:273\hbox{--}334.
\emph{Перевод:} \bookauthor{Крёнер}[Э.] Общая континуальная теория дислокаций и~собственных напряжений. \mirpublisher, 1965. \howmanypages{104~с.}

\publication{love-mathematicaltheoryofelasticity}
\bookauthor{Augustus Edward Hough Love}.
A~treatise on the mathematical theory of elasticity.
\href{https://hal.archives-ouvertes.fr/hal-01307751/document}{
Volume~I.
Cambridge, 1892.
\howmanypages{354~p.}
}
\href{https://archive.org/details/in.ernet.dli.2015.503659}{
Volume~II.
Cambridge, 1893.
\howmanypages{327~p.}
}
\href{https://archive.org/details/in.ernet.dli.2015.462644/page/n1}{
4th~edition.
Cambridge, 1927.
Dover, 1944.
\howmanypages{643~p.}
}
\emph{Перевод:}
\bookauthor{Аугустус Ляв}.
Математическая теория упругости.
М.:\;ОНТИ,
1935.
\howmanypages{674~с.}

\publication{lurie-nonlinearelasticity}
\bookauthor{Анатолий И. Лурье}.
Нелинейная теория упругости.
\naukapublisher, 1980.
\howmanypages{512~с.}
\emph{Translation:}
\bookauthor{Lurie,}[A.][I.]
Nonlinear Theory of~Elasticity:
translated from the~Russian by~K.\,A.\;Lurie.
Elsevier Science Publishers \hbox{B.\hspace{.1ex}V\hspace{-0.2ex}.}, 1990.
\howmanypages{617~p.}

\publication{lurie-theoryofelasticity}
\bookauthor{Анатолий И. Лурье}.
Теория упругости.
\naukapublisher, 1970.
\howmanypages{940~с.}
\emph{Translation:}
\bookauthor{Lurie,}[A.][I.]
Theory of~Elasticity (translated by~A.\:Belyaev).
Springer-Verlag, 2005.
\howmanypages{1050~p.}

\publication{lurie-spatialproblems}
\bookauthor{Анатолий И. Лурье}.
Пространственные задачи теории упругости.
М.:\;Гос\-тех\-издат, 1955.
\howmanypages{492~с.}

\publication{lurie-thinwalledshells}
\bookauthor{Анатолий И. Лурье}.
Статика тонкостенных упругих оболочек.
М.,\:Л.:\;Гос\-тех\-издат, 1947.
\howmanypages{252~с.}

\publication{mase-continuummechanics}
\bookauthor{George E. Mase}.
Schaum’s outline of theory and problems of continuum mechanics
(Schaum’s outline series).
McGraw\hbox{--}Hill,
1970.
\howmanypages{221~p.}
\emph{Перевод:}
\bookauthor{Джордж Мейз}.
Теория и~задачи механики сплошных сред.
Издание~3\hbox{-}е.
URSS, 2010.
\howmanypages{320~с.}

%%\publication{history of continuum mechanics by Maugin}
%%Gérard A.\;Maugin.
%%\inquotesx{Continuum mechanics through the~eighteenth and nineteenth centuries. Historical perspectives from John Bernoulli~(1727) to Ernst Hellinger~(1914)}[.]
%%[\inquotesx{Solid mechanics and its applications}[,] volume 214.]
%%Springer, 2014.
%%\howmanypages{269~pages.}

\publication{parkus.melan-waermespannungen} % Wärmespannungen infolge stationärer Temperaturfelder, von Ernst Melan und Heinz Parkus
\bookauthor{Ernst Melan}, \bookauthor{Heinz Parkus}. Wärmespannungen infolge stationärer Temperaturfelder. Wein, Springer-Verlag, 1953. \howmanypages{114~Seiten.}
\emph{Перевод:} \bookauthor{Мелан}[Э.], \bookauthor{Паркус}[Г.] Термоупругие напряжения, вызываемые стационарными температурными полями. \fizmatgiz, 1958. \howmanypages{167~с.}

\publication{merkin-threadmechanics}
\bookauthor{Меркин}[Д.][Р.] Введение в~механику гибкой нити. \naukapublisher, 1980. \howmanypages{240~с.}

\publication{merkin-stabilityintro}
\bookauthor{Меркин}[Д.][Р.] Введение в~теорию устойчивости движения. 3\hbox{-}е~издание. \naukapublisher, 1987. \howmanypages{304~с.}

\publication{mindlin.tiersten}
\bookauthor{Mindlin,}[Raymond~David] % Raymond David Mindlin
and~\bookauthor{Tiersten,}[Harry~F.] % Harry F. Tiersten
Effects of couple-stresses in linear elasticity~//~Archive for Rational Mechanics and Analysis. Volume~11, Issue~1 (January 1962), pp.\:415\hbox{--}448.
\emph{Перевод:}
\bookauthor{Миндлин}[Р.][Д.], \bookauthor{Тирстен}[Г.][Ф.]
Эффекты моментных напряжений в~линейной теории упругости~//~Механика: Сборник переводов и~обзоров иностранной периодической литературы. \mirpublisher, 1964. №\,4\:(86). С.\:80\hbox{--}114.

\publication{naghdi-theoryofshellsandplates}
\bookauthor{Naghdi}[P.][M.] The theory of shells and plates. In: Truesdell~C.\:(editor) Mechanics of~solids. Volume II. Linear theories of~elasticity and~thermoelasticity. Linear and~nonlinear theories of~rods, plates, and~shells. Springer\hbox{-}Verlag, 1973. Pages~425\hbox{--}640.

\publication{nowacki-problemsofthermoelasticity}
\bookauthor{Witold Nowacki}.
Dynamiczne zagadnienia termosprężystości.
Warsza\-wa: Państwowe wydawnictwo naukowe, 1966. \howmanypages{366~stron.}
\emph{Translation:}
\bookauthor{Nowacki,\;Witold}.
Dynamic problems of~thermoelasticity.
Leyden: Noordhoff international publishing, 1975.
\howmanypages{436~pages.}
\emph{Перевод:}
\bookauthor{Витольд Новацкий}.
Динамические задачи термоупругости.
\mirpublisher, 1970.
\howmanypages{256~с.}

\publication{nowacki-elasticity}
\bookauthor{Witold Nowacki}.
Teoria sprężystości.
Warszawa: Państwowe wy\-daw\-nic\-two naukowe, 1970.
\howmanypages{769~stron.}
\emph{Перевод:}
\bookauthor{Новацкий Витольд}.
Теория упругости.
\mirpublisher, 1975.
\howmanypages{872~с.}

\publication{nowacki-electromagneticeffects}
\bookauthor{Witold Nowacki}. Efekty elektromagnetyczne w stałych ciałach od\-kształ\-cal\-nych. Państwowe wydawnictwo naukowe, 1983. \howmanypages{147~stron.}
\emph{Перевод:}
\bookauthor{Новацкий}[В.] Электромагнитные эффекты в~твёрдых телах. \mirpublisher, 1986. \howmanypages{160~с.}

\publication{novozhilov-theoryofthinshells}
\bookauthor{Новожилов}[В.][В.] Теория тонких оболочек. 2\hbox{-}е~издание. Л.:\;Судпромгиз, 1962. \howmanypages{431~с.}

\publication{panovko.beylin-thinwalledrods}
\bookauthor{Пановко}[Я.][Г.], \bookauthor{Бейлин}[Е.][А.] Тонкостенные стержни и~системы, составленные из тонкостенных стержней. В~сборнике: Рабинович~И.\:М.\:(редактор) Строительная механика в~СССР 1917\hbox{--}1967. М.:\;Строй\-издат, 1969. С.\:75\hbox{--}98.

\publication{panovko.gubanova-stabilityandvibrations}
\bookauthor{Пановко}[Я.][Г.], \bookauthor{Губанова}[И.][И.] Устойчивость и~колебания упругих систем. Современные концепции, парадоксы и~ошибки. 4\hbox{-}е~издание. \naukapublisher, 1987. \howmanypages{352~с.}

\publication{parkus-waermespannungen} % Instationäre Wärmespannungen von Heinz Parkus
\bookauthor{Heinz Parkus}. Instation\"{a}re W\"{a}rmespannungen. Springer\hbox{-}Verlag, 1959. \howmanypages{176~Seiten.}
\emph{Перевод:} \bookauthor{Паркус}[Г.] Неустановившиеся температурные напряжения. \fizmatgiz, 1963. \howmanypages{252~с.}

\publication{parton-electromagneticelasticity}
\bookauthor{Партон}[Владимир З.], \bookauthor{Кудрявцев}[Борис А.] Электромагнитоупругость пьезоэлектрических и~электропроводных тел. \naukapublisher, 1988. \howmanypages{472~с.}

\publication{podstrigach.burak.kondrat-magnetothermoelasticity}
\bookauthor{Подстригач}[Я.][С.], \bookauthor{Бурак}[Я.][И.], \bookauthor{Кондрат}[В.][Ф.]
Магнито\-термо\-упру\-гость электропроводных тел.
Киев:\;Наукова~думка, 1982.
\howmanypages{296~с.}

\publication{poruchikov-dynamicelasticity}
\bookauthor{Поручиков}[В.][Б.] Методы динамической теории упругости. \naukapublisher, 1986.
\howmanypages{328~с.}

\publication{rabotnov-mechanicsofdeformable}
\bookauthor{Юрий Н.}[Работнов].
Механика деформируемого твёрдого тела.
2\hbox{-}е~издание. \naukapublisher, 1988.
Издание~3\hbox{-}е. URSS, 2019.
\howmanypages{712~с.}

\publication{De la torsion des prismes}
\bookauthor{\scalebox{.95}[1]{Adhémar Jean Claude} Barré de\:Saint\hbox{-\hspace{-0.2ex}}Venant}.
%
\href{https://gallica.bnf.fr/ark:/12148/bpt6k99739z/}{De la~torsion des prismes, avec des considérations sur leur flexion ainsi que sur l’équilibre des solides élastiques en général, et des formules pratiques pour le calcul de leur résistance à divers efforts s’exerçant simultanément.}
%
\href{https://gallica.bnf.fr/ark:/12148/bpt6k99739z/}{Extrait du tome~\textsc{xiv} des mémoires présentés par divers savants a l’académie des sciences.
Imprimerie Impériale, Paris, M\:DCCC\:LV\:(1855).}
\howmanypages{332~pages.}
\emph{Перевод на русский язык:}
\bookauthor{Сен-Венан Б.}
Мемуар о~кручении призм.
Мемуар об~изгибе призм.
М.:\;Физ\-мат\-гиз, 1961.
\howmanypages{518~страниц.}

\publication{Mémoire sur la flexion des prismes}
\bookauthor{\scalebox{.96}[1]{Adhémar Jean Claude} Barré de\:Saint\hbox{-\hspace{-0.2ex}}Venant}.
%
\href{http://www.numdam.org/item/JMPA_1856_2_1__89_0.pdf}{Mémoire sur la~flexion des prismes, sur les glissements transversaux et longitudinaux qui l’accompagnent lorsqu’elle ne s’opère pas uniformément ou en arc de cercle, et sur la forme courbe affectée alors par leurs sections transversales primitivement planes.}
%
\href{http://www.numdam.org/item/JMPA_1856_2_1__89_0.pdf}{Journal de~mathématiques pures et~appliquées, publié par Joseph Liouville.
2me~serie, tome~1, année~1856.
Pages 89~à~189.}
%
\emph{Перевод на русский язык:}
\bookauthor{Сен-Венан Б.}
Мемуар о~кручении призм.
Мемуар об~изгибе призм.
М.:\;Физ\-мат\-гиз, 1961.
\howmanypages{518~страниц.}

\publication{southwell-introductiontotheoryofelasticity} % An Introduction to the Theory of Elasticity for Engineers and Physicists by Richard Vynne Southwell, originally published 1936
\bookauthor{Southwell,}[Richard~V.]
An~introduction to the~theory of~elasticity for engineers and physicists.
Dover Publications, 1970.
\howmanypages{509~pages.}
\emph{Перевод:}
\bookauthor{Саусвелл}[Р.][В.]
Введение в~теорию упругости для~инженеров и~физиков.
М.:\;ИИЛ, 1948.
\howmanypages{675~с.}

% Elastic Models of Crystal Defects by Cristian Teodosiu
\publication{teodosiu-crystaldefects}
\bookauthor{Cristian Teodosiu}.
Elastic models of crystal defects.
Springer\hbox{-}Verlag,
1982.
\howmanypages{336~pages.}
\emph{Перевод:}
\bookauthor{Теодосиу}[К.]
Упругие модели дефектов в~кристаллах.
\mirpublisher,
1985.
\howmanypages{352~с.}

\publication{timoshenko-stability}
\bookauthor{Тимошенко Степан П.}
Устойчивость стержней, пластин и~оболочек.
\naukapublisher, 1971.
\howmanypages{808~с.}

\publication{timoshenko-platesnshells}
\bookauthor{Тимошенко Степан П.},
\bookauthor{Войновский\hbox{-}Кригер}[С.]
Пластинки и~оболочки.
\naukapublisher, 1966.
\howmanypages{635~с.}

\publication{timoshenkogoodier}
\bookauthor{Stephen P.}[Timoshenko] and \bookauthor{James N.}[Goodier].
Theory of~Elasticity.
2nd~edition. McGraw\hbox{--}Hill, 1951. \howmanypages{506~pages.}
3rd~edition. McGraw\hbox{--}Hill, 1970. \howmanypages{567~pages.}
\emph{Перевод:}
\bookauthor{Тимошенко Степан П.}, \bookauthor{Джеймс Гудьер}.
Теория упругости.
2\hbox{-}е~издание.
\naukapublisher, 1979.
\howmanypages{560~с.}

\publication{truesdell-firstcourse} % A First Course in Rational Continuum Mechanics by Clifford Ambrose Truesdell
% Truesdell, C. A First Course in Rational Continuum Mechanics. The Johns Hopkins University, Baltimore, Maryland, 1972.
\bookauthor{Truesdell,}[Clifford~A.]
A~first course in~rational continuum mechanics.
Volume~1:~General concepts.
2nd~edition.
Academic Press,
1991.
\howmanypages{391~pages.} % (Pure and Applied Mathematics, vol.\:71.)
%
\emph{Перевод:}
\bookauthor{Трусделл}[К.]
Первоначальный курс рациональной механики сплошных сред.
\mirpublisher,
1975.
\howmanypages{592~с.}

\publication{feodosiev-talks}
\bookauthor{Феодосьев}[В.][И.]
Десять лекций\hbox{-}бесед по~сопротивлению материалов.
2\hbox{-}е~издание.
\naukapublisher, 1975.
\howmanypages{173~с.}

\publication{ziegler-structuralstability} % Principles of structural stability by Hans Ziegler, originally published 1968
\emph{Перевод:}
\bookauthor{Циглер}[Г.]
Основы теории устойчивости конструкций.
\mirpublisher, 1971.
\howmanypages{192~с.}

\publication{chernyh-anisotropicelasticity}
\bookauthor{Черн\'{ы}х}[К.][Ф.]
Введение в~анизотропную упругость.
\naukapublisher,
1988.
\howmanypages{192~с.}

\publication{chernyh-nonlinearelasticity}
\bookauthor{Черн\'{ы}х}[К.][Ф.]
Нелинейная теория упругости в~машиностроительных расчетах.
\maschinenbauleningraderverlag,
1986.
\howmanypages{336~с.}

%
% oscillations and waves
%

\en{\subsection*{Oscillations and waves}}

\ru{\subsection*{Колебания и волны}}

\publication{timoshenko.young.weaver}
\bookauthor{Timoshenko,}[Stephen~P.];
\bookauthor{Young,}[Donovan~H.];
\bookauthor{William Weaver,}[jr.]
Vibration problems in engineering.
5th~edition.
John~Wiley~\&~Sons, 1990.
\howmanypages{624~pages.}
\emph{Перевод:} \bookauthor{Тимошенко Степан П.},
\bookauthor{Янг Донован Х.},
\bookauthor{Уильям Уивер}.
Колебания в~инженерном деле.
\maschinenbaumoskauerverlag, 1985.
\howmanypages{472~с.}

\publication{babakov-theoryofoscillations}
\bookauthor{Бабаков}[И.][М.] Теория колебаний. 4\hbox{-}е~издание. <<Дрофа>>, 2004. \howmanypages{592~с.}

\publication{biderman-oscillations}
\bookauthor{Бидерман}[В.][Л.] Теория механических колебаний. М.:\;Высшая школа, 1980. \howmanypages{408~с.}

\publication{bolotin-randomoscillations}
\bookauthor{Болотин}[В.][В.] Случайные колебания упругих систем. \naukapublisher, 1979. \howmanypages{336~с.}

\publication{grinchenko.meleshko}
\bookauthor{Гринченко}[В.][Т.], \bookauthor{Мелешко}[В.][В.] Гармонические колебания и~волны в~упругих телах. Киев:\;Наукова думка, 1981. \howmanypages{284~с.}

\publication{whitham-waves}
\bookauthor{Whitham,}[Gerald~B.] Linear and~nonlinear waves. John~Wiley~\&~Sons, 1974. \howmanypages{636~pages.}
\emph{Перевод:} \bookauthor{Уизем}[Дж.] Линейные и~нелинейные волны. \mirpublisher, 1977. \howmanypages{624~с.}

\publication{kolsky-stresswavesinsolids}
\bookauthor{Kolsky,}[Herbert]. Stress waves in solids. Oxford, Clarendon Press, 1953. \howmanypages{211~p.} 2nd~edition. Dover Publications, 2012. \howmanypages{224~p.}
\emph{Перевод:} \bookauthor{Кольский}[Г.] Волны напряжения в~твёрдых телах. М.:\;ИИЛ, 1955. \howmanypages{192~с.}

\publication{engelbrecht.nigul-nonlineardeformationwaves} % Uno Karlovich Nigul, Jüri Engelbrecht
\bookauthor{Энгельбрехт}[Ю.][К.], \bookauthor{Нигул}[У.][К.] Нелинейные волны деформации. \naukapublisher, 1981. \howmanypages{256~с.}

\publication{slepyan-nonstationeryelasticwaves}
\bookauthor{Слепян}[Л.][И.] Нестационарные упругие волны. Л.:\;Судостроение, 1972. \howmanypages{376~с.}

\publication{grigolyuk.selezov}
\bookauthor{Григолюк}[Э.][И.], \bookauthor{Селезов}[И.][Т.] Неклассические теории колебаний стержней, пластин и~оболочек. (Итоги науки и~техники. Механика твёрдых деформируемых тел. Том~5.) М.:\;ВИНИТИ, 1973. \howmanypages{272~с.}

%
% fracture mechanics
%

\en{\subsection*{Fracture mechanics}}

\ru{\subsection*{Механика трещин (разрушения)}}

\publication{kachanov-fracturemechanics}
\bookauthor{Качанов}[Л.][М.]
Основы механики разрушения.
\naukapublisher, 1974.
\howmanypages{312~с.}

\publication{kerstein.klyushnikov.lomakin.shesterikov-experimentalfracturemechanics}
\bookauthor{Керштейн}[И.][М.],
\bookauthor{Клюшников}[В.][Д.],
\bookauthor{Ломакин}[Е.][В.],
\bookauthor{Шестериков}[С.][А.]
Основы экспериментальной механики разрушения.
Изд\hbox{-}во~МГУ, 1989.
\howmanypages{140~с.}

\publication{morozov-fractures}
\bookauthor{Морозов}[Н.][Ф.]
Математические вопросы теории трещин.
\naukapublisher, 1984.
\howmanypages{256~с.}

\publication{parton-fracturemechanics}
\bookauthor{Партон}[Владимир З.]
Механика разрушения: от~теории к~практике.
\naukapublisher, 1990.
\howmanypages{240~с.}

\publication{parton.morozov-destructionofelastoplastic}
\bookauthor{Партон}[Владимир З.], \bookauthor{Морозов}[Евгений М.]
Механика упругопластического разрушения.
2\hbox{-}е~издание. \naukapublisher, 1985.
\howmanypages{504~с.}

\publication{hellan-fracturemechintro} % Introduction to Fracture Mechanics by Kåre Hellan, originally published 1984
\emph{Перевод:}
\bookauthor{Хеллан}[К.]
Введение в~механику разрушения.
\mirpublisher, 1988.
\howmanypages{364~с.}

\publication{cherepanov-fragilefracture}
\bookauthor{Геннадий П.}[Черепанов].
Механика хрупкого разрушения.
\naukapublisher, 1974.
\howmanypages{640~с.}

%
% composites
%

\en{\subsection*{Composites}}

\ru{\subsection*{Композиты}}

\publication{christensen-compositematerials}
\bookauthor{Christensen,}[Richard~M.] Mechanics of~composite materials. New~York: Wiley, 1979. \howmanypages{348~p.}
\emph{Перевод:} \bookauthor{Кристенсен}[Р.] Введение в~механику композитов. \mirpublisher, 1982. \howmanypages{336~с.}

\publication{kravchuk.mayboroda.urzhumtsev-polymericandcompositematerials}
\bookauthor{Кравчук}[А.][С.], \bookauthor{Майборода}[В.][П.], \bookauthor{Уржумцев}[Ю.][С.] Механика полимерных и~композиционных материалов. Экспериментальные и численные методы. \naukapublisher, 1985. \howmanypages{304~с.}

\publication{pobedrya-composites}
\bookauthor{Борис Е.}[Победря].
Механика композиционных материалов.
Издательство Московского университета,
1984.
\howmanypages{336~с.}

\publication{bakhvalov.panasenko}
\bookauthor{Бахвалов}[Н.][С.], \bookauthor{Панасенко}[Г.][П.] Осреднение процессов в~периодических средах. Математические задачи механики композиционных материалов. \naukapublisher, 1984. \howmanypages{352~с.}

\publication{asymptoticanalysisforperiodicstructures}
\bookauthor{Bensoussan}[A.], % Alain Bensoussan
\bookauthor{Lions}[J.-L.], % Jacques-Louis Lions
\bookauthor{Papanicolaou}[G.]
Asymptotic analysis for periodic structures.
Amsterdam: North\hbox{-}Holland,
1978.
\howmanypages{700~p.}

\publication{cherepanov-destructionofcomposites}
\bookauthor{Геннадий П.}[Черепанов].
Механика разрушения композиционных материалов.
\naukapublisher,
1983.
\howmanypages{296~с.}

\publication{shermergor}
\bookauthor{Тимофей Д.}[Шермергор].
Теория упругости микронеоднородных сред.
\naukapublisher,
1977.
\howmanypages{400~с.}

%
% finite element method
%

\en{\subsection*{The finite element method}}

\ru{\subsection*{Метод конечных элементов}}

\publication{zienkiewicz.morgan-finiteelementsandapproximation}
\bookauthor{Зенкевич}[О.], \bookauthor{Морган}[К.] Конечные элементы и~аппроксимация. \mirpublisher, 1986. \howmanypages{318~с.}

\publication{shabrov-finiteelementmethod}
\bookauthor{Шабров}[Н.][Н.] Метод конечных элементов в~расчётах деталей тепловых двигателей. \maschinenbauleningraderverlag, 1983. \howmanypages{212~с.}

%
% mechanics, thermodynamics, electromagnetism
%

\en{\subsection*{Mechanics, thermodynamics, electromagnetism}}

\ru{\subsection*{Механика, термодинамика, электромагнетизм}}

% The Feynman Lectures on Physics by Richard Phillips Feynman, Robert B. Leighton and Matthew Sands
% Volume I: Mainly mechanics, radiation, and heat
% Volume II: Mainly electromagnetism and matter
% Volume III: Quantum mechanics

\publication{feynman-lecturesonphysics}
\bookauthor{Feynman,}[Richard Ph.] ${\hspace{-0.2ex}\bigdot}$ \bookauthor{Leighton,}[Robert B.] ${\hspace{-0.2ex}\bigdot}$ \bookauthor{Sands,}[Matthew]. The Feynman Lectures on Physics. New millennium edition. Volume II: Mainly electromagnetism and matter. Basic Books, 2011. \howmanypages{566~pages.}
\emph{Online:}
\href{http://www.feynmanlectures.caltech.edu/}{The Feynman Lectures on Physics. Online edition.}
%%\emph{Перевод:}
%%\bookauthor{Фейнман}[Р.], \bookauthor{Лейтон}[Р.], \bookauthor{Сэндс}[М.]
%%Фейнмановские лекции по физике.
%%Том~5: Электричество и магнетизм. Издание~8\hbox{-}е. URSS, 2014. \howmanypages{310~с.}
%%Том~6: Электродинамика. Издание~9\hbox{-}е. URSS, 2016. \howmanypages{352~с.}

% Classical Mechanics by Herbert Goldstein
%
% In 2001, a new (third) edition of the book was released, with the collaboration of Charles P. Poole and John L. Safko

\publication{goldstein-classicalmechanics}
\bookauthor{Goldstein,}[Herbert]; \bookauthor{Poole,}[Charles~P.]; \bookauthor{Safko,}[John L.] Classical Mechanics. 3rd~edition. Addison\hbox{--}Wesley, 2001. \howmanypages{638~pages.}
\emph{Перевод:} \bookauthor{Голдстейн}[Г.], \bookauthor{Пул}[Ч.], \bookauthor{Сафко}[Дж.] Классическая механика. URSS, 2012. \howmanypages{828~с.}

% Leopold Alexander “Alan” Pars is most remembered for his textbooks Introduction to dynamics (1953), Calculus of variations (1962), and his monumental 650 page Treatise on analytical dynamics (1965)

\publication{treatiseonanalyticaldynamics-by-l.a.pars} % A Treatise on Analytical Dynamics by Leopold A. Pars, originally published London: Heinemann, 1965
\bookauthor{Pars,}[Leopold~A.] A~treatise on analytical dynamics.
London: Heinemann, 1965. \howmanypages{641~pages.}
\emph{Перевод:} \bookauthor{Парс}[Л.][А.] Аналитическая динамика. \naukapublisher, 1971. \howmanypages{636~с.}

\publication{terhaar-hamiltonianmechanics} % Elements of Hamiltonian Mechanics by Dirk Ter Haar
\bookauthor{Ter Haar,}[Dirk]. Elements of hamiltonian mechanics. 2nd~edition. Pergamon Press, 1971. \howmanypages{201~pages.}
\emph{Перевод:} \bookauthor{Тер~Хаар}[Д.] Основы гамильтоновой механики. \naukapublisher, 1974. \howmanypages{223~с.}

\publication{belyaev.ryadno}
\bookauthor{Беляев}[Н.][М.], \bookauthor{Рядно}[А.][А.] Методы теории теплопроводности. М.:\;Высшая школа, 1982. В~2\hbox{-}х томах.
\howmanypages{Том~1, 328~с.}
\howmanypages{Том~2, 304~с.}

\publication{classicalelectrodynamics}
\bookauthor{Бредов}[М.][М.], \bookauthor{Румянцев}[В.][В.], \bookauthor{Топтыгин}[И.][Н.] Классическая электродинамика. \naukapublisher, 1985. \howmanypages{400~с.}

\publication{gantmacher-analyticalmechanics}
\bookauthor{Феликс~Р.\;Гантмахер}
Лекции по~аналитической механике.
Издание~2\hbox{-}е.
\naukapublisher, 1966.
\howmanypages{300~с.}

\publication{landau.lifshitz-shortcourse}
\bookauthor{Ландау}[Л.][Д.], \bookauthor{Лифшиц}[Е.][М.] Краткий курс теоретической физики. Книга~1. Механика. Электродинамика. \naukapublisher, 1969. \howmanypages{271~с.}
% Лев Давидович Ландау, Евгений Михайлович Лифшиц

\publication{loitsjanskiy.lurie}
\bookauthor{Лев Г.\:Лойцянский}, \bookauthor{Анатолий И. Лурье}. Курс теоретической механики: В~2\hbox{-}х томах.
<<Дрофа>>, 2006.
%
Том~1:~Статика и~кинематика.
9\hbox{-}е~издание.
\howmanypages{447~с.}
%
Том~2:~Динамика.
7\hbox{-}е~издание.
\howmanypages{719~с.}

\publication{lurie-analyticalmechanics}
\bookauthor{Анатолий И. Лурье}.
Аналитическая механика.
\fizmatgiz, 1961.
\howmanypages{824~с.}

\publication{olkhovskiy-theoreticalmechanicsforphysicists}
\bookauthor{Ольховский}[И.][И.] Курс теоретической механики для~физиков. 3\hbox{-}е~издание. Изд\hbox{-}во~МГУ, 1978. \howmanypages{575~с.}

\publication{tamm-electricity}
\bookauthor{Тамм}[И.][Е.] % Тамм Игорь Евгеньевич
Основы теории электричества. 11\hbox{-}е~издание. М.:\;Физматлит, 2003.
\howmanypages{616~с.}
% http://www.samomudr.ru/d/Tamm%20I.E.%20_Osnovy%20teorii%20Elektrichestva_616str_2003g.pdf

%
% tensors
%

\en{\subsection*{Tensors and tensor calculus}}

\ru{\subsection*{Тензоры и тензорное исчисление}}

\publication{mcconnell-tensoranalysis} % Applications of Tensor Analysis by Albert Joseph McConnell
\bookauthor{McConnell,}[Albert Joseph]. Applications of tensor analysis. New~York: Dover Publications, 1957. \howmanypages{318 pages.}
\emph{Перевод:} \bookauthor{Мак\hbox{-}Коннел}[А.][Дж.] Введение в~тензорный анализ с~приложениями к~геометрии, механике и~физике. \fizmatgiz, 1963. \howmanypages{412~с.}

\publication{dimitrienko-tensorcalculus}
\bookauthor{Димитриенко}[Ю.][И.]
Тензорное исчисление: Учебное пособие для вузов.
М.:\;\inquotes{Высшая школа}, 2001.
\howmanypages{575~с.}

\publication{rashevsky-riemanniangeometry}
\bookauthor{Рашевский}[П.][К.]
Риманова геометрия и~тензорный анализ.
Издание~3\hbox{-}е. \naukapublisher, 1967.
\howmanypages{664~с.}

\publication{schouten-tensoranalysis} % Tensor analysis for physicists by Jan Arnoldus Schouten
\bookauthor{Schouten,}[Jan~A.] Tensor analysis for physicists. 2nd~edition. Dover Publications, 2011. \howmanypages{320~pages.}
\emph{Перевод:} \bookauthor{Схоутен}[Я.][А.] Тензорный анализ для~физиков. \naukapublisher, 1965. \howmanypages{456~с.}

\publication{sokolnikoff-tensoranalysis}
\bookauthor{Sokolnikoff,}[I.][S.] Tensor analysis: Theory and applications to geometry and mechanics of~continua. 2nd~edition. John~Wiley~\&~Sons, 1965. \howmanypages{361~pages.}
\emph{Перевод:} \bookauthor{Сокольников}[И.][С.] Тензорный анализ (с~приложениями к~геометрии и~механике сплошных сред). \naukapublisher, 1971. \howmanypages{376~с.}

%
% variational methods
%

\en{\subsection*{Variational methods}}

\ru{\subsection*{Вариационные методы}}

\publication{rektorys-variationalmethods}
\bookauthor{Karel~Rektorys}. Varia\v{c}ní metody v in\v{z}en\'{y}rsk\'{y}ch probl\'{e}mech a~v~pro\-bl\'{e}\-mech matematick\'{e} fyziky. SNTL (St\'{a}tní nakladatelství technick\'{e} literatury), 1974. \howmanypages{593~s.}
\emph{Translation:}
\bookauthor{Rektorys,}[Karel]. Variational Methods in Mathematics, Science and Engineering. Second edition. D.\,Reidel Publishing Company, 1980. \howmanypages{571~p.}
\emph{Перевод:}
\bookauthor{Ректорис}[К.] Вариационные методы в~математической физике. \mirpublisher, 1985. \howmanypages{590~с.}

\publication{washizubook} % Variational Methods in Elasticity and Plasticity by Kyuichiro Washizu, originally published: 1968
\bookauthor{Washizu,}[Kyuichiro]. Variational methods in elasticity and plasticity. 3rd~edition. Pergamon Press, Oxford, 1982. \howmanypages{630~pages.}
\emph{Перевод:} \bookauthor{Васидзу}[К.] Вариационные методы в~теории упругости и~пластичности. \mirpublisher, 1987. \howmanypages{542~с.}

\publication{berdichevsky}
\bookauthor{Бердичевский}[В.][Л.] Вариационные принципы механики сплошной среды. \naukapublisher, 1983. \howmanypages{448~с.}

\publication{mihlin-variationalmethods}
\bookauthor{Михлин}[С.][Г.] Вариационные методы в~математической физике. Издание~2\hbox{-}е. \naukapublisher, 1970. \howmanypages{512~с.}

%
% perturbation (asymptotic) methods
%

\en{\subsection*{Perturbation methods (asymptotic methods)}}

\ru{\subsection*{Методы возмущений (асимптотические методы)}}

\publication{cole-perturbationmethods}
\bookauthor{Cole,}[Julian~D.] Perturbation methods in applied mathematics. Blaisdell Publishing Co., 1968. \howmanypages{260~pages.}
\emph{Перевод:} \bookauthor{Коул}[Дж.] Методы возмущений в~прикладной математике. \mirpublisher, 1972. \howmanypages{274~с.}

\publication{nayfeh-introtoperturbation} % Introduction to perturbation techniques by Ali Hasan Nayfeh
\bookauthor{Nayfeh,}[Ali~H.] Introduction to perturbation techniques. Wiley, 1981. \howmanypages{536~pages.}
\emph{Перевод:} \bookauthor{Найфэ}[Али~Х.] Введение в~методы возмущений. \mirpublisher, 1984. \howmanypages{535~с.}

\publication{nayfeh-perturbation} % Perturbation methods by Ali Hasan Nayfeh
\bookauthor{Nayfeh,}[Ali~H.] Perturbation methods. Wiley-VCH, 2004. \howmanypages{425~pages.}
%%\emph{Перевод:} \bookauthor{Найфэ}[Али~Х.] Методы возмущений. \mirpublisher, 1976. \howmanypages{456~с.}

\publication{bogolyubovmitropolsky}
\bookauthor{Боголюбов}[Н.][Н.], \bookauthor{Митропольский}[Ю.][А.] Асимптотические методы в~теории нелинейных колебаний. \naukapublisher, 1974. \howmanypages{504~с.}

\publication{vasiljevabutuzov}
\bookauthor{Васильева}[А.][Б.], \bookauthor{Бутузов}[В.][Ф.] Асимптотические методы в~теории сингулярных возмущений. М.:\;Высшая школа, 1990. \howmanypages{208~с.}

\publication{zino.tropp}
\bookauthor{Зино}[И.][Е.], % Зино Игорь Евгеньевич
\bookauthor{Тропп}[Э.][А.] % Тропп Эдуард Абрамович
Асимптотические методы в~задачах теории теплопроводности и~термоупругости. Изд\hbox{-}во~ЛГУ, 1978. \howmanypages{224~с.}

\publication{moiseev-asymptoticalmethods}
\bookauthor{Моисеев}[Н.][Н.] Асимптотические методы нелинейной механики. 2\hbox{-}е~издание. \naukapublisher, 1981. \howmanypages{400~с.}

\publication{tovstik-thinwalledshellsstability}
\bookauthor{Товстик}[П.][Е.] Устойчивость тонких оболочек: асимптотические методы. \naukapublisher, 1995. \howmanypages{319~с.}

%
% other math
%

\en{\subsection*{Other topics of mathematics}}

\ru{\subsection*{Другие темы математики}}

\publication{collatz-eigenwertaufgaben}
\bookauthor{Collatz,}[Lothar]. Eigenwertaufgaben mit technischen Anwendungen. 2.\:Auflage. Akademische Verlagsgesellschaft Geest\;\&\;Portig, Leipzig, 1963. \howmanypages{500~Seiten.}
\emph{Перевод:} \bookauthor{Коллатц}[Л.] Задачи на~собственные значения~(с~техническими приложениями). \naukapublisher, 1968. \howmanypages{504~с.}

\publication{dwight-tables}
\bookauthor{Dwight,}[Herbert Bristol]. Tables of integrals and other mathematical data. 4th~edition. The~Macmillan~Co., 1961. \howmanypages{336~pages.}
\emph{Перевод:} \bookauthor{Двайт}[Г.][Б.] Таблицы интегралов и~другие математические формулы. Издание~4\hbox{-}е. \naukapublisher, 1973. \howmanypages{228~с.}

\publication{kamke-ordinarydifferentialequations}
\bookauthor{Kamke,}[Erich]. Differentialgleichungen, Lösungsmethoden und~Lö\-sun\-gen. Bd.\:I. Gewöhnliche Differentialgleichungen. 10.\:Auflage. Teubner Verlag, 1977. \howmanypages{670~Seiten.}
\emph{Перевод:} \bookauthor{Камке}[Э.] Справочник по~обыкновенным дифференциальным уравнениям. 6\hbox{-}е~издание. <<Лань>>, 2003. \howmanypages{576~с.}
%% 4\hbox{-}е~издание. \naukapublisher, 1971. \howmanypages{576~с.}

\publication{graninokorn.theresakorn-mathematicalhandbook} % Mathematical Handbook for Scientists and Engineers by Granino Arthur Korn and Theresa M. Korn
\bookauthor{Korn,}[Granino~A.] and \bookauthor{Korn,}[Theresa M.]
Mathematical handbook for scientists and engineers: definitions, theorems, and formulas for reference and review.
Revised edition. Dover Publications, 2013. \howmanypages{1152~pages.}
\emph{Перевод:}
\bookauthor{Корн}[Г.],
\bookauthor{Корн}[Т.]
Справочник по~математике для научных работников и~инженеров.
\naukapublisher,
1974.
\howmanypages{832~с.}

\publication{lavrentiev.shabat}
\bookauthor{Лаврентьев}[М.][А.],
\bookauthor{Шабат}[Б.][В.]
Методы теории функций комплексного переменного.
4\hbox{-}е~издание.
\naukapublisher,
1973.
\howmanypages{736~с.}

\publication{pogorelov-differentialgeometry}
\bookauthor{Погорелов}[А.][В.]
Дифференциальная геометрия.
Издание~6\hbox{-}е.
\naukapublisher,
1974.
\howmanypages{176~с.}

\end{otherlanguage}

\normalsize
\end{thebibliography}

