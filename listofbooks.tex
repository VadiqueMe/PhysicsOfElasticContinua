\vspace*{10mm}

\addcontentsline{toc}{chapter}{\bibliographyname}

\newcommandx*{\bibauthor}[3][2=,3=]{\textbf{#1%
\ifthenelse{\equal{#2}{}}{}{~#2}%
\ifthenelse{\equal{#3}{}}{}{\:#3}%
}}

\newcommand\mirpublisher{<<Мир>>} % М.:\;Мир
\newcommand\naukapublisher{<<Наука>>} % М.:\;Наука
\newcommand\fizmatgiz{М.:\;Физ\-мат\-гиз}

\begin{thebibliography}{123}
\interlinepenalty=10000 % inhibit page breaks inside a paragraph
\small

\thispagestyle{empty}

%% “Mechanics of Solids. Volume II. Linear Theories of Elasticity and Thermoelasticity. Linear and Nonlinear Theories of Rods, Plates, and Shells (Editor: C. Truesdell)” originally appeared in hardcover as Volume Vla/2 of Encyclopedia of Physics © by Springer\hbox{-}Verlag Berlin Heidelberg 1972

\begin{otherlanguage}{russian}

\bibitem{alfutov-stabilitycalc}
\bibauthor{Алфутов}[Н.][А.] Основы расчета на~устойчивость упругих систем. Изд.\:2\hbox{-}е. М.:\;Машино\-строение, 1991. 336~с.

\bibitem{stuartantman-theoryofrods} % Stuart S. Antman
\bibauthor{Antman,}[Stuart~S.] The~theory of~rods. In: Truesdell~C.\:(editor) Mechanics of~solids. Volume II. Linear theories of~elasticity and~thermoelasticity. Linear and~nonlinear theories of~rods, plates, and~shells. Springer\hbox{-}Verlag Berlin Heidelberg, 1973. Pages~641\hbox{--}703.

\bibitem{artobolevskiyetother}
\bibauthor{Артоболевский}[И.][И.], \bibauthor{Бобровницкий}[Ю.][И.], \bibauthor{Генкин}[М.][Д.] Введение в~акустическую динамику машин. \naukapublisher, 1979. 296~с.

\bibitem{babakov}
\bibauthor{Бабаков}[И.][М.] Теория колебаний. 4\hbox{-}е~изд. <<Дрофа>>, 2004. 592~с.
% (3\hbox{-}е~изд.~--- 1968~г.)

\bibitem{babichvestnik}
\bibauthor{Бабич}[В.][М.], \bibauthor{Булдырев}[B.][С.] Искусство асимптотики~//~Вестник ЛГУ, 1977, №\,13, вып.\:3, С.\:5\hbox{--}12.

\bibitem{bakhvalovpanasenko}
\bibauthor{Бахвалов}[Н.][С.], \bibauthor{Панасенко}[Г.][П.] Осреднение процессов в~периодических средах. Математические задачи механики композиционных материалов. \naukapublisher, 1984. 352~с.

\bibitem{belyaevryadno}
\bibauthor{Беляев}[Н.][М.], \bibauthor{Рядно}[А.][А.] Методы теории теплопроводности. М.:\;Высшая школа, 1982. Т.\:1. 327~с. Т.\:2. 304~с.

\bibitem{asymptoticanalysisforperiodicstructures}
\bibauthor{Bensoussan}[A.], \bibauthor{Lions}[J.-L.], \bibauthor{Papanicolaou}[G.] Asymptotic analysis for periodic structures. Amsterdam: North\hbox{-}Holland, 1978. 700~p.

\bibitem{berdichevsky}
\bibauthor{Бердичевский}[В.][Л.] Вариационные принципы механики сплошной среды. \naukapublisher, 1983. 448~с.

\bibitem{biderman-thinwalled}
\bibauthor{Бидерман}[В.][Л.] Механика тонкостенных конструкций. М.:\;Машино\-строение, 1977. 488~с.

\bibitem{biderman-oscillations}
\bibauthor{Бидерман}[В.][Л.] Теория механических колебаний. М.:\;Высшая школа, 1980. 408~с.

\bibitem{bogolyubovmitropolsky}
\bibauthor{Боголюбов}[Н.][Н.], \bibauthor{Митропольский}[Ю.][А.] Асимптотические методы в~теории нелинейных колебаний. \naukapublisher, 1974. 504~с.

\bibitem{bolotin-stability}
\bibauthor{Болотин}[В.][В.] Неконсервативные задачи теории упругой устойчивости. \fizmatgiz, 1961. 339~с.

\bibitem{bolotin-randomoscillations}
\bibauthor{Болотин}[В.][В.] Случайные колебания упругих систем. \naukapublisher, 1979. 336~с.

\bibitem{borisenkotarapov}
\bibauthor{Борисенко}[А.][И.], \bibauthor{Тарапов}[И.][Е.] Векторный анализ и~начала тензорного исчисления. Изд.\:6\hbox{-}е. Харьков:~Вища школа. Изд\hbox{-}во при Харь\-ков\-ском ун\hbox{-}те, 1986. 216~с.

\bibitem{classicalelectrodynamics}
\bibauthor{Бредов}[М.][М.], \bibauthor{Румянцев}[В.][В.], \bibauthor{Топтыгин}[И.][Н.] Классическая электродинамика. \naukapublisher, 1985. 400~с.

\bibitem{washizu} % Variational Methods in Elasticity and Plasticity by Kyuichiro Washizu, originally published: 1968
\bibauthor{Washizu,}[Kyuichiro]. Variational methods in elasticity and plasticity. 3rd~edition. Pergamon Press, Oxford, 1982. 630~pages.
\emph{Перевод:} \bibauthor{Васидзу}[К.] Вариационные методы в~теории упругости и~пластичности. \mirpublisher, 1987. 542~с.

\bibitem{vasiljevabutuzov}
\bibauthor{Васильева}[А.][Б.], \bibauthor{Бутузов}[В.][Ф.] Асимптотические методы в~теории сингулярных возмущений. М.:\;Высшая школа, 1990. 208~с.

\bibitem{vlasov-thinwalledrods}
\bibauthor{Власов}[В.][З.] Тонкостенные упругие стержни. \fizmatgiz, 1959. 568~с.

\bibitem{gantmacher}
\bibauthor{Гантмахер}[Ф.][Р.] Лекции по~аналитической механике. Изд.\:2\hbox{-}е. \naukapublisher, 1966. 300~с.

\bibitem{godunov-mathphysics}
\bibauthor{Годунов}[С.][К.] Уравнения математической физики. \naukapublisher, 1971. 416~с.

\bibitem{goldenveizer-thinshells}
\bibauthor{Гольденвейзер}[А.][Л.] Теория упругих тонких оболочек. \naukapublisher, 1976. 512~с.

\bibitem{goldenveizer.lidskiy.tovstik}
\bibauthor{Гольденвейзер}[А.][Л.], \bibauthor{Лидский}[В.][Б.], \bibauthor{Товстик}[П.][Е.] Свободные ко\-леба\-ния тонких упругих оболочек. \naukapublisher, 1979. 383~с.

% Classical Mechanics by Herbert Goldstein
%
% In 2001, a new (third) edition of the book was released, with the collaboration of Charles P. Poole and John L. Safko

\bibitem{goldstein-classicalmechanics}
\bibauthor{Goldstein,}[Herbert]; \bibauthor{Poole,}[Charles~P.]; \bibauthor{Safko,}[John L.] Classical Mechanics. 3rd~edition. Addison\hbox{-}Wesley, 2001. 638~pages.
\emph{Перевод:} \bibauthor{Голдстейн}[Г.], \bibauthor{Пул}[Ч.], \bibauthor{Сафко}[Дж.] Классическая механика. URSS, 2012. 828~с.
%%\bibauthor{Гольдстейн}[Г.] Классическая механика. \naukapublisher, 1975. 416~с.

\bibitem{gordon-constructions}
\bibauthor{Gordon,}[James~E.] Structures, or Why things don’t fall down. Penguin Books, 1978. 395~p.
\emph{Перевод:} \bibauthor{Гордон}[Дж.] Конструкции, или почему не~ломаются вещи. \mirpublisher, 1980. 390~с.

\bibitem{gordon-whyyoudontfallthru}
\bibauthor{Gordon,}[James~E.] The new science of strong materials, or Why you don’t fall through the floor. Penguin Books, 1968. 269~p.
\emph{Перевод:} \bibauthor{Гордон}[Дж.] Почему мы не~проваливаемся сквозь пол. \mirpublisher, 1971. 272~с.

\bibitem{grigolyuk.selezov}
\bibauthor{Григолюк}[Э.][И.], \bibauthor{Селезов}[И.][Т.] Неклассические теории колебаний стержней, пластин и~оболочек. (Итоги науки и~техники. Механика твёрдых деформируемых тел. Том~5.) М.:\;ВИНИТИ, 1973. 272~с.

\bibitem{grinchenko.meleshko}
\bibauthor{Гринченко}[В.][Т.], \bibauthor{Мелешко}[В.][В.] Гармонические колебания и~волны в~упругих телах. Киев:\;Наукова думка, 1981. 284~с.

\bibitem{dwight-tables}
\emph{Перевод:} \bibauthor{Двайт}[Г.][Б.] Таблицы интегралов и~другие математические формулы. Изд.\:4\hbox{-}е. \naukapublisher, 1973. 228~с.

% by deWit, Roland
% Theory of disclinations:
% II. Continuous and discrete disclinations in anisotropic elasticity
% III. Continuous and discrete disclinations in isotropic elasticity
% IV. Straight disclinations https://pdfs.semanticscholar.org/8834/3a6514cd502a5fa0f40f5bae4a87101deef8.pdf

\bibitem{dewit-disclinations}
\emph{Перевод:} \bibauthor{Де~Вит}[Р.] Континуальная теория дисклинаций. \mirpublisher, 1977. 208~с.

\bibitem{janelidzepanovko-thinwalledrods}
\bibauthor{Джанелидзе}[Г.][Ю.], \bibauthor{Пановко}[Я.][Г.] Статика упругих тонкостенных стержней. Л.,\:М.:\;Гостехиздат, 1948. 208~с.

\bibitem{eliseev-models}
\bibauthor{Елисеев}[В.][В.] \href{https://www.researchgate.net/publication/320895320_Odnomernye_i_trehmernye_modeli_v_mehanike_uprugih_sterznej}{Одномерные и~трёхмерные модели в~механике упругих стержней. Дис. ... д\hbox{-}ра физ.\hbox{-}мат. наук. ЛГТУ, 1991. 300~с.}

\bibitem{eshelby-theoryofdislocations} % John Douglas Eshelby
\bibauthor{Eshelby,}[John~D.] The continuum theory of lattice defects~//~Solid State Physics, Academic Press, vol.\:3, 1956, pp.\:79\hbox{--}144.
\emph{Перевод:} \bibauthor{Эшелби}[Дж.] Континуальная теория дислокаций. М.:\;ИИЛ, 1963. 247~с.

\bibitem{haroldalexander-rubberlike}
\bibauthor{Harold Alexander}. \href{https://kundoc.com/pdf-a-constitutive-relation-for-rubber-like-materials-.html}{A~constitutive relation for rubber-like materials~// International Journal of~Engineering Science, volume~6 (September 1968), pages 549\hbox{--}563.}
% https://www.researchgate.net/publication/232329906_A_constitutive_relation_for_rubber-like_materials

\bibitem{zienkiewicz.morgan-finiteelementsandapproximation}
\bibauthor{Зенкевич}[О.], \bibauthor{Морган}[К.] Конечные элементы и~аппроксимация. \mirpublisher, 1986. 318~с.

\bibitem{zino.tropp}
\bibauthor{Зино}[И.][Е.], \bibauthor{Тропп}[Э.][А.] Асимптотические методы в~задачах теории теплопроводности и~термоупругости. Изд\hbox{-}во~ЛГУ, 1978. 224~с.

\bibitem{zubov}
\bibauthor{Зубов}[Л.][М.] Методы нелинейной теории упругости в~теории оболочек.
%%Ростов\hbox{-}на\hbox{-}Дону:\;Изд\hbox{-}во~РГУ,
Изд\hbox{-}во Ростовского ун\hbox{-}та,
1982. 144~с.

++ {Илюхин}[А.][А.] О~построении соотношений теории упругих стержней~//~Механика твердого тела (Киев), 1990, №\,22, С.\:82\hbox{--}92.

\bibitem{kamke-ordinarydifferentialequations}
\bibauthor{Kamke,}[Erich]. Differentialgleichungen, Lösungsmethoden und~Lö\-sun\-gen. Bd.\:I. Gewöhnliche Differentialgleichungen. 10.\:Auflage. Teubner Verlag, 1977. 670~Seiten.
\emph{Перевод:} \bibauthor{Камке}[Э.] Справочник по~обыкновенным дифференциальным уравнениям. 6\hbox{-}е~изд. <<Лань>>, 2003. 576~с.
%% 4\hbox{-}е~изд. \naukapublisher, 1971. 576~с.

\bibitem{kachanov-fracturemechanics}
\bibauthor{Качанов}[Л.][М.] Основы механики разрушения. \naukapublisher, 1974. 312~с.

\bibitem{kerstein.klyushnikov.lomakin.shesterikov-experimentalfracturemechanics}
\bibauthor{Керштейн}[И.][М.], \bibauthor{Клюшников}[В.][Д.], \bibauthor{Ломакин}[Е.][В.], \bibauthor{Шестериков}[С.][А.]
Основы экспериментальной механики разрушения. Изд\hbox{-}во~МГУ, 1989. 140~с.

\bibitem{kirchhoff:crelle56}
\bibauthor{Kirchhoff,}[Gustav~R.] \href{https://opacplus.bsb-muenchen.de/Vta2/bsb10525510/bsb:2960444?page=291}{Über das~Gleichgewicht und die~Bewegung eines unendlich dünnen elastischen Stabes~//~Journal für die reine und angewandte Mathematik (Crelle’s journal), 56. Band (1859). Seiten 285\hbox{--}313.}

\bibitem{collatz-eigenwertaufgaben}
\bibauthor{Collatz,}[Lothar]. Eigenwertaufgaben mit technischen Anwendungen. 2.\:Auflage. Akademische Verlagsgesellschaft Geest\;\&\;Portig, Leipzig, 1963. 500~Seiten.
\emph{Перевод:} \bibauthor{Коллатц}[Л.] Задачи на~собственные значения~(с~техническими приложениями). \naukapublisher, 1968. 504~с.

\bibitem{kolsky-stresswavesinsolids}
\emph{Перевод:} \bibauthor{Кольский}[Г.] Волны напряжения в~твёрдых телах. М.:\;ИИЛ, 1955. 192~с.

\bibitem{graninokorn.theresakorn-mathematicalhandbook} % Mathematical Handbook for Scientists and Engineers by Granino A. Korn and Theresa M. Korn
\emph{Перевод:} \bibauthor{Корн}[Г.], \bibauthor{Корн}[Т.] Справочник по~математике для научных работников и~инженеров. \naukapublisher, 1974. 832~с.

\bibitem{cosserat}
\bibauthor{Cosserat}[E.] et \bibauthor{Cosserat}[F.] \href{https://jscholarship.library.jhu.edu/bitstream/handle/1774.2/34209/31151000327233.pdf}{Théorie des corps déformables. Paris: A.\:Hermann et~Fils, 1909. 226~p.}

\bibitem{cottrell-dislocations} % Theory of crystal dislocations by Alan Cottrell
\emph{Перевод:} \bibauthor{Коттрел}[А.] Теория дислокаций. \mirpublisher, 1969. 96~с.

\bibitem{cole-perturbationmethods} % Cole, Julian D. Perturbation methods in applied mathematics
\emph{Перевод:} \bibauthor{Коул}[Дж.] Методы возмущений в~прикладной математике. \mirpublisher, 1972. 274~с.

\bibitem{kravchuk.mayboroda.urzhumtsev-polymericandcompositematerials}
\bibauthor{Кравчук}[А.][С.], \bibauthor{Майборода}[В.][П.], \bibauthor{Уржумцев}[Ю.][С.] Механика полимерных и~композиционных материалов. Экспериментальные и численные методы. \naukapublisher, 1985. 304~с.

\bibitem{kroener-kontinuumstheorie}
\bibauthor{Kröner,}[Ekkehart] \emph{(i)} Kontinuumstheorie der~Versetzungen und Eigen\-spannung\-en. Springer\hbox{-}Verlag Berlin Heidelberg, 1958. 180~pages.
\emph{(ii)} Allgemeine Kontinuumstheorie der~Versetzungen und Eigen\-span\-nung\-en~//~Archive for Rational Mechanics and Analysis. Volume~4, Issue~1 (January 1959), pp.\:273\hbox{--}334.
\emph{Перевод:} \bibauthor{Крёнер}[Э.] Общая континуальная теория дислокаций и~собственных напряжений. \mirpublisher, 1965. 104~с.

\bibitem{christensen-compositematerials}
\bibauthor{Christensen,}[Richard~M.] Mechanics of~composite materials. New~York: Wiley, 1979. 348~p.
\emph{Перевод:} \bibauthor{Кристенсен}[Р.] Введение в~механику композитов. \mirpublisher, 1982. 336~с.

\bibitem{lavrentiev.shabat}
\bibauthor{Лаврентьев}[М.][А.], \bibauthor{Шабат}[Б.][В.] Методы теории функций комплексного переменного. 4\hbox{-}е~изд. \naukapublisher, 1973. 736~с.

\bibitem{landau.lifshitz-shortcourse}
\bibauthor{Ландау}[Л.][Д.], \bibauthor{Лифшиц}[Е.][М.] Краткий курс теоретической физики. Книга~1. Механика. Электродинамика. \naukapublisher, 1969. 271~с.
% Лев Давидович Ландау, Евгений Михайлович Лифшиц

\bibitem{loitsjanskiy.lurie}
\bibauthor{Лойцянский}[Л.][Г.], \bibauthor{Лурье}[А.][И.] Курс теоретической механики: В~2\hbox{-}х томах. <<Дрофа>>, 2006.
Т.\:1:~Статика и~кинематика. 9\hbox{-}е~изд. 447~с.
Т.\:2:~Динамика. 7\hbox{-}е~изд. 719~с.

\bibitem{lurie-analyticalmechanics}
\bibauthor{Лурье}[А.][И.] Аналитическая механика. \fizmatgiz, 1961. 824~с.

\bibitem{lurie-nonlinearelasticity}
\bibauthor{Лурье}[А.][И.] Нелинейная теория упругости. \naukapublisher, 1980. 512~с.

\bibitem{lurie-spatialproblems}
\bibauthor{Лурье}[А.][И.] Пространственные задачи теории упругости. М.:\;Гос\-тех\-издат, 1955. 492~с.

\bibitem{lurie-thinwalledshells}
\bibauthor{Лурье}[А.][И.] Статика тонкостенных упругих оболочек. М.,\:Л.:\;Гос\-тех\-издат, 1947. 252~с.

\bibitem{lurie-theoryofelasticity}
\bibauthor{Лурье}[А.][И.] Теория упругости. \naukapublisher, 1970. 940~с.

\bibitem{love-mathematicaltheoryofelasticity}
\emph{Перевод:} \bibauthor{Ляв}[А.] Математическая теория упругости. М.:\;ОНТИ, 1935. 674~с.

\bibitem{mcconnell-tensoranalysis} % Applications of Tensor Analysis by Albert Joseph McConnell, originally published New York: Dover Publications, 1957
\emph{Перевод:} \bibauthor{Мак\hbox{-}Коннел}[А.][Дж.] Введение в~тензорный анализ с~приложениями к~геометрии, механике и~физике. \fizmatgiz, 1963. 412~с.

\bibitem{parkus.melan-waermespannungen} % Wärmespannungen infolge Stationärer Temperaturfelder, von Ernst Melan und Heinz Parkus. Wein, Springer-Verlag, 1953
\emph{Перевод:} \bibauthor{Мелан}[Э.], \bibauthor{Паркус}[Г.] Термоупругие напряжения, вызываемые стационарными температурными полями. \fizmatgiz, 1958. 167~с.

\bibitem{merkin-threadmechanics}
\bibauthor{Меркин}[Д.][Р.] Введение в~механику гибкой нити. \naukapublisher, 1980. 240~с.

\bibitem{merkin-stabilityintro}
\bibauthor{Меркин}[Д.][Р.] Введение в~теорию устойчивости движения. 3\hbox{-}е~изд. \naukapublisher, 1987. 304~с.

\bibitem{mindlin.tiersten}
\bibauthor{Mindlin,}[Raymond~David] % Raymond David Mindlin
and~\bibauthor{Tiersten,}[Harry~F.] % Harry F. Tiersten
Effects of couple-stresses in linear elasticity~//~Archive for Rational Mechanics and Analysis. Volume~11, Issue~1 (January 1962), pp.\:415\hbox{--}448.
\emph{Перевод:}
\bibauthor{Миндлин}[Р.][Д.], \bibauthor{Тирстен}[Г.][Ф.]
Эффекты моментных напряжений в~линейной теории упругости~//~Механика: Сб.~переводов. \mirpublisher, 1964. №\,4\:(86). С.\:80\hbox{--}114.
% Механика: Сб. переводов (Сборник переводов и~обзоров иностранной периодической литературы), выходит с~1950\;г.

\bibitem{mihlin-variationalmethods}
\bibauthor{Михлин}[С.][Г.] Вариационные методы в~математической физике. Изд.\:2\hbox{-}е. \naukapublisher, 1970. 512~с.

\bibitem{moiseev-asymptoticalmethods}
\bibauthor{Моисеев}[Н.][Н.] Асимптотические методы нелинейной механики. 2\hbox{-}е~изд. \naukapublisher, 1981. 400~с.

\bibitem{morozov-twodimensionalproblems}
\bibauthor{Морозов}[Н.][Ф.] Избранные двумерные задачи теории упругости. Изд\hbox{-}во~ЛГУ, 1978. 182~с.

\bibitem{morozov-fractures}
\bibauthor{Морозов}[Н.][Ф.] Математические вопросы теории трещин. \naukapublisher, 1984. 256~с.

\bibitem{muskhelishvili-somebasicproblems}
\bibauthor{Мусхелишвили}[Н.][И.] Некоторые основные задачи математической теории упругости. Изд.\:5\hbox{-}е. \naukapublisher, 1966. 708~с.

\bibitem{naghdi-theoryofshellsandplates}
\bibauthor{Naghdi}[P.][M.] The theory of shells and plates. In: Truesdell~C.\:(editor) Mechanics of~solids. Volume II. Linear theories of~elasticity and~thermoelasticity. Linear and~nonlinear theories of~rods, plates, and~shells. Springer\hbox{-}Verlag Berlin Heidelberg, 1973. Pages~425\hbox{--}640.

\bibitem{nayfeh-introtoperturbation} % Introduction to perturbation techniques by Ali Hasan Nayfeh, originally published January 19, 1981
\emph{Перевод:} \bibauthor{Найфэ}[Али~Х.] Введение в~методы возмущений. \mirpublisher, 1984. 535~с.

\bibitem{nayfeh-perturbation} % Perturbation methods by Ali H. Nayfeh, originally published March 2, 1973
\emph{Перевод:} \bibauthor{Найфэ}[Али~Х.] Методы возмущений. \mirpublisher, 1976. 456~с.

\bibitem{nowacki-dynamicalproblems}
\bibauthor{Новацкий}[В.] Динамические задачи термоупругости. \mirpublisher, 1970. 256~с.

\bibitem{nowacki-elasticity}
\bibauthor{Новацкий}[В.] Теория упругости. \mirpublisher, 1975. 872~с.

\bibitem{nowacki-electromagneticeffects}
\bibauthor{Новацкий}[В.] Электромагнитные эффекты в~твёрдых телах. \mirpublisher, 1986. 160~с.

\bibitem{novozhilov-theoryofthinshells}
\bibauthor{Новожилов}[В.][В.] Теория тонких оболочек. 2\hbox{-}е~изд. Л.:\;Судпромгиз, 1962. 431~с.

\bibitem{olkhovskiy-theoreticalmechanicsforphysicists}
\bibauthor{Ольховский}[И.][И.] Курс теоретической механики для~физиков. 3\hbox{-}е~изд. Изд\hbox{-}во~МГУ, 1978. 575~с.

\bibitem{panovko-mechanicsconceptsparadoxes}
\bibauthor{Пановко}[Я.][Г.] Механика деформируемого твердого тела: Современные концепции, ошибки и~парадоксы. Изд.\:2\hbox{-}е. URSS, 2017. 288~с. 
%%Пановко Я. Г. Механика деформируемого твердого тела: Современные концепции, ошибки и~парадоксы. \naukapublisher, 1985. 288~с.

\bibitem{panovko.beylin-thinwalledrods}
\bibauthor{Пановко}[Я.][Г.], \bibauthor{Бейлин}[Е.][А.] Тонкостенные стержни и~системы, составленные из тонкостенных стержней. В~сборнике: Рабинович~И.\:М.\:(редактор) Строительная механика в~СССР 1917\hbox{--}1967. М.:\;Строй\-издат, 1969. С.\:75\hbox{--}98.

++ {Пановко}[Я.][Г.], {Губанова}[И.][И.] Устойчивость и~колебания упругих систем. \naukapublisher, 1979. 384~с.

++ {Паркус}[Г.] Неустановившиеся температурные напряжения. \fizmatgiz, 1963. 252~с.

% Leopold Alexander “Alan” Pars is most remembered for his textbooks Introduction to dynamics (1953), Calculus of variations (1962), and his monumental 650 page Treatise on analytical dynamics (1965)

\bibitem{treatiseonanalyticaldynamics-by-l.a.pars} % A Treatise on Analytical Dynamics by Leopold A. Pars, originally published London: Heinemann, 1965
\emph{Перевод:} \bibauthor{Парс}[Л.][А.] Аналитическая динамика. \naukapublisher, 1971. 636~с.

\bibitem{parton-fracturemechanics}
\bibauthor{Партон}[В.][З.] Механика разрушения: от~теории к~практике. \naukapublisher, 1990. 240~с.

\bibitem{parton-electromagneticelasticity}
\bibauthor{Партон}[В.][З.], \bibauthor{Кудрявцев}[Б.][А.] Электромагнитоупругость пьезоэлектрических и~электропроводных тел. \naukapublisher, 1988. 472~с.

\bibitem{parton.morozov-destructionofelastoplastic}
\bibauthor{Партон}[В.][З.], \bibauthor{Морозов}[Е.][М.] Механика упругопластического разрушения. 2\hbox{-}е~изд. \naukapublisher, 1985. 504~с.

\bibitem{pobedrya-composites}
\bibauthor{Победря}[Б.][Е.] Механика композиционных материалов. Изд\hbox{-}во~Моск.\:ун\hbox{-}та, 1984. 336~с.

\bibitem{pogorelov-differentialgeometry}
\bibauthor{Погорелов}[А.][В.] Дифференциальная геометрия. Изд.\:6\hbox{-}е. \naukapublisher, 1974. 176~с.

\bibitem{podstrigach.burak.kondrat-magnetothermoelasticity}
\bibauthor{Подстригач}[Я.][С.], \bibauthor{Бурак}[Я.][И.], \bibauthor{Кондрат}[В.][Ф.] Магнито\-термо\-упру\-гость электропроводных тел. Киев:\;Наукова~думка, 1982. 296~с.

\bibitem{poniatovsky-thinwallopenprofileasymptotic}
\bibauthor{Понятовский}[В.][В.] Вывод уравнений тонко\-стен\-ных стержней\hbox{--}обо\-ло\-чек открытого профиля из~уравнений теории упругости методом асимптотического интегрирования~//~Исследования по~упругости и~пластичности. Изд\hbox{-}во~ЛГУ, 1980. Вып.\;13. С.\:40\hbox{--}48.

\bibitem{poruchikov-dynamicelasticity}
\bibauthor{Поручиков}[В.][Б.] Методы динамической теории упругости. \naukapublisher, 1986. 328~с.

\bibitem{rabotnov-mechanicsofdeformable}
\bibauthor{Работнов}[Ю.][Н.] Механика деформируемого твёрдого тела. 2\hbox{-}е~изд. \naukapublisher, 1988. 712~с.

\bibitem{rashevsky-riemanniangeometry}
\bibauthor{Рашевский}[П.][К.] Риманова геометрия и~тензорный анализ. Изд.\:3\hbox{-}е. \naukapublisher, 1967. 664~с.

++ {Ректорис}[К.] Вариационные методы в~математической физике. \mirpublisher, 1985. 590~с.

\bibitem{southwell-introductiontotheoryofelasticity} % An Introduction to the Theory of Elasticity for Engineers and Physicists by Richard Vynne Southwell, originally published 1936
\bibauthor{Southwell,}[Richard~V.] An~introduction to the~theory of~elasticity for engineers and physicists. Dover Publications, 1970. 509~pages.
\emph{Перевод:} \bibauthor{Саусвелл}[Р.][В.] Введение в~теорию упругости для~инженеров и~физиков. М.:\;ИИЛ, 1948. 675~с.

\bibitem{sedov-continuummechanics}
\bibauthor{Седов}[Л.][И.] Механика сплошной среды. Том~2. 5\hbox{-}е~изд. \naukapublisher, 1994. 560~с.

++ {Слепян}[Л.][И.] Нестационарные упругие волны. Л.:\;Судостроение, 1972. 376~с.

\bibitem{sokolnikov}
\bibauthor{Сокольников}[И.][С.] Тензорный анализ. \naukapublisher, 1971. 376~с.

\bibitem{schouten-tensoranalysis} % Tensor analysis for physicists by Jan Arnoldus Schouten
\bibauthor{Schouten,}[Jan~A.] Tensor analysis for physicists. 2nd~edition. Dover Publications, 2011. 320~pages.
\emph{Перевод:} \bibauthor{Схоутен}[Я.][А.] Тензорный анализ для~физиков. \naukapublisher, 1965. 456~с.

\bibitem{ciarlet-mathematicalelasticity}
\bibauthor{Ciarlet,}[Philippe~G.] Mathematical elasticity. Volume~1:~Three\hbox{-}di\-men\-sion\-al elasticity. Elsevier Science Publishers B.\hspace{.1ex}V., 1988. xlii\:+\:452~pp. % (Studies in Mathematics and its Applications, vol.\:20.)
\emph{Перевод:} \bibauthor{Сьярле}[Ф.] Математическая теория упругости. \mirpublisher, 1992. 472~с.

\bibitem{tamm-electricity}
\bibauthor{Тамм}[И.][Е.] Основы теории электричества. \naukapublisher, 1989. 504~с.

\bibitem{teodosiu-crystaldefects} % Elastic Models of Crystal Defects by Cristian Teodosiu
\bibauthor{Teodosiu,}[Cristian]. Elastic models of crystal defects. Springer\hbox{-}Verlag Berlin Heidelberg, 1982. 336~pages.
\emph{Перевод:} \bibauthor{Теодосиу}[К.] Упругие модели дефектов в~кристаллах. \mirpublisher, 1985. 352~с.

\bibitem{terhaar-hamiltonianmechanics} % Elements of Hamiltonian Mechanics by Dirk Ter Haar
\bibauthor{Ter Haar,}[Dirk]. Elements of hamiltonian mechanics. 2nd~edition. Pergamon Press, 1971. 201~pages.
\emph{Перевод:} \bibauthor{Тер~Хаар}[Д.] Основы гамильтоновой механики. \naukapublisher, 1974. 223~с.

\bibitem{timoshenko-stability}
\bibauthor{Тимошенко}[С.][П.] Устойчивость стержней, пластин и~оболочек. \naukapublisher, 1971. 808~с.

\bibitem{timoshenko-platesnshells}
\bibauthor{Тимошенко}[С.][П.], \bibauthor{Войновский\hbox{-}Кригер}[С.] Пластинки и~оболочки. \naukapublisher, 1966. 635~с.

\bibitem{timoshenkogoodier}
\bibauthor{Тимошенко}[С.][П.], \bibauthor{Гудьер}[Дж.] Теория упругости. \naukapublisher, 1979. 560~с.

\bibitem{timoshenko.young.weaver}
\bibauthor{Тимошенко}[С.][П.], \bibauthor{Янг}[Д.][Х.], \bibauthor{Уивер}[У.] Колебания в~инженерном деле. М.:\;Машино\-строение, 1985. 472~с.

\bibitem{tihonovsamarsky-mathphysicsequations}
\bibauthor{Тихонов}[А.][Н.], \bibauthor{Самарский}[А.][А.] Уравнения математической физики. 6\hbox{-}е~изд. Изд\hbox{-}во~МГУ, 1999. 798~с.

\bibitem{tovstik-thinwalledshellsstability}
\bibauthor{Товстик}[П.][Е.] Устойчивость тонких оболочек: асимптотические методы. \naukapublisher, 1995. 319~с.

\bibitem{truesdell-firstcourse} % A First Course in Rational Continuum Mechanics by Clifford Ambrose Truesdell
% Truesdell, C. A First Course in Rational Continuum Mechanics. The Johns Hopkins University, Baltimore, Maryland, 1972.
\bibauthor{Truesdell,}[Clifford~A.] A~first course in~rational continuum mechanics. Volume~1:~General concepts. 2nd~edition. Academic Press, 1991. 391~pages. % (Pure and Applied Mathematics, vol.\:71.)
\emph{Перевод:} \bibauthor{Трусделл}[К.] Первоначальный курс рациональной механики сплошных сред. \mirpublisher, 1975. 592~с.

\bibitem{whitham-waves}
\bibauthor{Whitham,}[Gerald~B.] Linear and~nonlinear waves. John~Wiley~\&~Sons, 1974. 636~p.
\emph{Перевод:} \bibauthor{Уизем}[Дж.] Линейные и~нелинейные волны. \mirpublisher, 1977. 624~с.

% The Feynman Lectures on Physics by Richard Phillips Feynman, Robert B. Leighton and Matthew Sands
% Volume I: Mainly mechanics, radiation, and heat
% Volume II: Mainly electromagnetism and matter
% Volume III: Quantum mechanics

\bibitem{feynman-lecturesonphysics}
\bibauthor{Feynman,}[Richard Ph.] ${\hspace{-0.2ex}\bigdot}$ \bibauthor{Leighton,}[Robert B.] ${\hspace{-0.2ex}\bigdot}$ \bibauthor{Sands,}[Matthew]. The Feynman Lectures on Physics. New millennium edition. Volume II: Mainly electromagnetism and matter. Basic Books, 2011. 566~pages.
\emph{Online:}
\href{http://www.feynmanlectures.caltech.edu/}{The Feynman Lectures on Physics. Online edition.}
%%\emph{Перевод:}
%%\bibauthor{Фейнман}[Р.], \bibauthor{Лейтон}[Р.], \bibauthor{Сэндс}[М.]
%%Фейнмановские лекции по физике.
%%Т.\:5: Электричество и магнетизм. Изд.\:8\hbox{-}е. URSS, 2014. 310~с.
%%Т.\:6: Электродинамика. Изд.\:9\hbox{-}е. URSS, 2016. 352~с.

\bibitem{feodosiev-talks}
\bibauthor{Феодосьев}[В.][И.] Десять лекций\hbox{-}бесед по~сопротивлению материалов. 2\hbox{-}е~изд. \naukapublisher, 1975. 173~с.

\bibitem{hellan-fracturemechintro} % Introduction to Fracture Mechanics by Kåre Hellan, originally published 1984
\emph{Перевод:} \bibauthor{Хеллан}[К.] Введение в~механику разрушения. \mirpublisher, 1988. 364~с.

\bibitem{ziegler-structuralstability} % Principles of structural stability by Hans Ziegler, originally published 1968
\emph{Перевод:} \bibauthor{Циглер}[Г.] Основы теории устойчивости конструкций. \mirpublisher, 1971. 192~с.

\bibitem{cherepanov-compositematerialfracture}
\bibauthor{Черепанов}[Г.][П.] Механика разрушения композиционных материалов. \naukapublisher, 1983. 296~с.

\bibitem{cherepanov-fragilefracture}
\bibauthor{Черепанов}[Г.][П.] Механика хрупкого разрушения. \naukapublisher, 1974. 640~с.

\bibitem{chernina-thinwalledshells}
\bibauthor{Чернина}[B.][С.] Статика тонкостенных оболочек вращения. \naukapublisher, 1968. 456~с.

\bibitem{chernyh-anisotropicelasticity}
\bibauthor{Черн\'{ы}х}[К.][Ф.] Введение в~анизотропную упругость. \naukapublisher, 1988. 192~с.

\bibitem{chernyh-nonlinearelasticity}
\bibauthor{Черн\'{ы}х}[К.][Ф.] Нелинейная теория упругости в~машиностроительных расчетах. Л.:\;Машино\-строение, 1986. 336~с.

\bibitem{shabrov-finiteelementmethod}
\bibauthor{Шабров}[Н.][Н.] Метод конечных элементов в~расчётах деталей тепловых двигателей. Л.:\;Машино\-строение, 1983. 212~с.

\bibitem{shermergor}
\bibauthor{Шермергор}[Т.][Д.] Теория упругости микронеоднородных сред. \naukapublisher, 1977. 400~с.

\bibitem{engelbrecht.nigul-nonlineardeformationwaves} % Uno Karlovich Nigul, Jüri Engelbrecht
\bibauthor{Энгельбрехт}[Ю.][К.], \bibauthor{Нигул}[У.][К.] Нелинейные волны деформации. \naukapublisher, 1981. 256~с.

\end{otherlanguage}

\normalsize
\end{thebibliography}
