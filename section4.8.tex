\en{\section{Finding displacements by deformations}}

\ru{\section{Нахождение смещений по деформациям}}

\label{section:displacementsfromdeformations}

\en{Like any}\ru{Как и~всякий}
\en{bivalent tensor}\ru{бивалентный тензор},
\en{the displacement gradient}\ru{градиент смещения}
% ${\boldnabla \fieldofdisplacements}$
\en{can be decomposed}\ru{может быть разложен}
\en{into the sum}\ru{на сумму}
\en{of the symmetric and antisymmetric parts}\ru{симметричной и~антисимметричной частей}

\nopagebreak\vspace{-0.1em}\begin{equation}
\label{lineartheory:displacementgradientdecomposed}
\boldnabla \bm{u} \hspace{.2ex} = \tikzmark{beginSymmNablaU} \hspace{.3ex} \infinitesimaldeformation \hspace{.4ex} \tikzmark{endSymmNablaU} \hspace{-0.16ex} - \hspace{.25ex} \tikzmark{beginAsymmNablaU} \bm{\omega} \times \hspace{-0.1ex} \UnitDyad \tikzmark{endAsymmNablaU} \hspace{.1ex} , \:\;
\bm{\omega} \equiv \displaystyle \onehalf \hspace{.3ex} \boldnabla \hspace{-0.1ex} \times \hspace{-0.1ex} \bm{u} \hspace{.2ex},
\end{equation}%
\AddOverBrace[line width=.75pt][.1ex,0.1ex]%
{beginSymmNablaU}{endSymmNablaU}{${\scriptstyle \boldnabla {\bm{u}}^{\hspace{.1ex}\mathsf{S}}}$}%
\AddOverBrace[line width=.75pt][.2ex,0.1ex]%
{beginAsymmNablaU}{endAsymmNablaU}{${\scriptstyle - \hspace{.1ex} \boldnabla {\bm{u}}^{\hspace{.1ex}\mathsf{A}}}$}

\en{The symmetric part}\ru{Симметричная часть}
${\boldnabla {\bm{u}}^{\hspace{.1ex}\mathsf{S}}}$
\en{is}\ru{есть}
\en{the linear deformation tensor}\ru{тензор линейной деформации}~$\infinitesimaldeformation$.

\en{The antisymmetric part}\ru{Антисимметричную часть} ${\boldnabla {\bm{u}}^{\hspace{.1ex}\mathsf{A}}}$ \en{we will denote as}\ru{обозначим как}~$\bm{\Omega}$ \en{and }\ru{и~}\en{will call it}\ru{назовём} \en{the tensor of small rotations}\ru{тензором малых поворотов}.
\en{Any antisymmetric bivalent tensor}\ru{Любой антисимметричный бивалентный тензор} \en{can be represented by a~vector}\ru{может быть представлен вектором} (\chapterdotsectionref{chapter:mathapparatus}{section:tensors.symmetric+skewsymmetric}).
\en{So}\ru{Итак}, \en{to find displacements}\ru{чтобы найти смещения}~$\bm{u}$ \en{by deformations}\ru{по~деформациям}~${\infinitesimaldeformation}$, \ru{нужно }\en{one more field}\ru{ещё одно поле}\en{ is needed}\:--- \en{the field of rotations}\ru{поле поворотов}~$\bm{\omega}(\bm{r})$.

....

\emph{\en{The compatibility conditions}\ru{Условия совместности}
\en{in the linear elasticity}\ru{в~линейной упругости}}

\en{The }\ru{Условия совместности }Saint\hbox{-\hspace{-0.2ex}}Venant’\en{s}\ru{а}\en{ compatibility conditions} \en{represent}\ru{представляют} \en{the integrability conditions}\ru{условия интегрируемости} \en{for}\ru{для} \en{a~symmetric}\ru{симметричного} \en{bivalent tensor field}\ru{бивалентного тензорного поля}.
\en{When such a~tensor field is compatible}\ru{Когда такое тензорное поле совместно}, \en{then it describes some deformation (strain)}\ru{тогда оно описывает какую-то деформацию}.

\en{In }\ru{В~отношении }\en{the }\en{displacement}\ru{смещение}\hbox{$\hspace{.3ex}\mapsto\hspace{.3ex}$}\en{deformation}\ru{деформация}\en{ relation}~${\infinitesimaldeformation = \hspace{-0.2ex} \boldnabla {\bm{u}}^{\hspace{.1ex}\mathsf{S}}}$\en{,}
\en{the six}\ru{шесть} \en{components}\ru{компонент}~${\infinitesimaldeformationcomponents{i\hspace{-0.1ex}j}}$ \en{of deformation}\ru{деформации}~${\infinitesimaldeformation}$ \en{originate from}\ru{происходят из} \en{only three components}\ru{лишь трёх компонент}~${u_{k}}$ \en{of the displacement vector}\ru{вектора смещений}~${\bm{u}}$.

\en{The compatibility conditions}\ru{Условия совместности} \en{determine}\ru{определяют}\ru{,} \en{whether this deformation does not cause}\ru{не является ли эта деформация причиной} \en{any gaps and/or overlaps}\ru{каких-либо промежутков и/или перекрываний}.

\noindent(~\textcolor{blue}{.... add a~picture here .....}~)


...


\begin{equation*}
\operatorname{inc} \infinitesimaldeformation \hspace{-0.1ex}
\equiv
\hspace{-0.1ex} \boldnabla \hspace{-0.2ex} \times \hspace{-0.3ex} \bigl( \hspace{.1ex} \boldnabla \hspace{-0.2ex} \times \hspace{-0.15ex} \infinitesimaldeformation \hspace{.1ex} \bigr)^{\hspace{-0.2ex}\T}
\end{equation*}

\en{A~contour}\ru{Контур}
\en{here}\ru{здесь}
\en{is arbitrary}\ru{произволен},
\en{so}\ru{так что}
\en{we have the relation}\ru{имеем отношение}

\nopagebreak\vspace{-0.25em}\begin{equation}\label{incompatibilityequalszero}
\operatorname{inc} \infinitesimaldeformation = \hspace{-0.07ex} {^2\bm{0}}
\hspace{.1ex} ,
\end{equation}

\vspace{-0.33em}\noindent
\en{called}\ru{называемому}
\en{the compatibility of deformations equation}\ru{уравнением совместности деформаций}.

...

\en{Expression}\ru{Выражение}~\eqref{incompatibilityequalszero} \en{provides}\ru{предоставляет} \en{constraints}\ru{ограничения} \en{on possible variants}\ru{на возможные варианты} \en{of a~deformation (strain) field}\ru{поля деформации}.

\noindent(~\textcolor{blue}{... the figure with cut squares ...}~)

...

\en{Tensor}\ru{Тензор}~${\operatorname{inc} \infinitesimaldeformation}$ \en{is symmetric}\ru{симметричен} \en{together with the }\ru{вместе с~}${\infinitesimaldeformation}$

...

\en{All equations}\ru{Все уравнения} \en{of the linear theory}\ru{линейной теории} \en{have an~analogue (primary source)}\ru{имеют аналог (первоисточник)} \en{in the nonlinear theory}\ru{в~нелинейной теории}.
\en{To find it for}\ru{Чтобы найти его для}~\eqref{incompatibilityequalszero}, \en{remember}\ru{вспомним} \en{the}\ru{тензор деформации} Cauchy\hbox{--}Green\ru{’а}\en{deformation tensor}~(\chapterdotsectionref{chapter:nonlinearcontinuum}{section:deformationtensors}) \en{and}\ru{и}~\en{curvature tensors}\ru{тензоры кривизны}~(\chapterdotsectionref{chapter:mathapparatus}{section:curvaturetensors})

...................

...................


