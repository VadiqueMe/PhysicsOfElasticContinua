\en{\chapter{Oscillations and waves}}

\ru{\chapter{Колебания и волны}}

\thispagestyle{empty}

\label{chapter:vibrationsnwaves}

\begin{otherlanguage}{russian}

\section{Колебания трёхмерных тел}

\lettrine[lines=2, findent=2pt, nindent=0pt]{Р}{ассмотрим} динамическую задачу классической линейной упругости
\nopagebreak\vspace{-0.5em}\begin{equation}\label{classiclinearelasticity:problem}
\begin{array}{c}
\boldnabla \dotp \mathboldtau \hspace{0.15ex} + \bm{f} = \rho \hspace{0.1ex} \mathdotdotabove{\bm{u}} \hspace{0.2ex}, \:\:
\mathboldtau = \stiffnesstensor \dotdotp \hspace{-0.25ex} \boldnabla {\bm{u}}^{\mathsf{S}} \hspace{-0.2ex},\\[0.32em]
%
\bm{u} \hspace{0.1ex} \bigr|_{o_1} \hspace{-0.64ex} = \hspace{0.2ex} 0 \hspace{0.1ex}, \:\:
\bm{n} \dotp \mathboldtau \hspace{0.25ex} \bigr|_{o_2} \hspace{-0.64ex} = \hspace{0.2ex} \bm{p} \hspace{0.16ex},\\[0.4em]
%
\bm{u} \hspace{0.1ex} \bigr|_{t=0} \hspace{-0.2ex} = \bm{u}^{\hspace{-0.1ex}\circ} \hspace{-0.4ex}, \:\:
\mathdotabove{\bm{u}} \hspace{0.1ex} \bigr|_{t=0} \hspace{-0.2ex} = \mathdotabove{\bm{u}}^{\circ} \hspace{-0.32ex}.
\end{array}
\end{equation}

\vspace{.2em} В~соответствии с~общей теорией~(\chapdotpararef{chapter:genericmechanics}{para:smalloscillations}), начинаем с~анализа главных, или~нормальных, колебаний:

...




\section{Колебания стержней}

В~линейной динамике стержней имеем следующую систему для~сил~$\mathboldQ$, моментов~$\mathboldM$, перемещений~$\bm{u}$ и~поворотов~$\bm{\theta}~(\chapdotpararef{chapter:rods}{para:ooooooooo})$:

...




\section{Малые возмущения параметров}

Рассмотрим задачу об~определении собственных частот и~форм с~малыми возмущениями масс и~жёсткостей:

...




\section{Колебания оболочек}

Динамика оболочек рассматривалась многими

...




\section{Волны в упругой среде}

Рассмотрим линейные уравнения динамики однородной изотропной среды без объёмных сил

...




\section{Волны в стержнях}

Рассмотрим прямой стержень. Продольная деформация описывается уравнениями

...




\section{Нелинейные колебания}

Рассмотрим простой пример: продольные колебания прямого стержня с~м\'{а}лой нелинейной добавкой в~соотношениях упругости

...




\end{otherlanguage}

\vspace{8mm}
\hfill\begin{minipage}[b]{0.95\linewidth}
\fontsize{10}{12}\selectfont

\section*{\wordforbibliography}

\begin{otherlanguage}{russian}

Методы решения динамических задач теории упругости представлены в~книгах ...

\end{otherlanguage}

\end{minipage}
