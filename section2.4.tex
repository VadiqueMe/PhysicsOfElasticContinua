\en{\section{Hamilton’s principle and Lagrange’s equations}}

\ru{\section{Принцип Hamilton’а и уравнения Lagrange’а}}

\en{\dropcap{T}{he} two branches}\ru{\dropcap{Д}{ве} ветви}
\en{of~analytical mechanics}\ru{аналитической механики}\en{ are}\ru{\:---}
\href{https://en.wikipedia.org/wiki/Lagrangian_mechanics}{Lagrang\en{ian}\ru{’ева} \en{mechanics}\ru{механика}}
(\en{operating with}\ru{оперирующая}
\en{generalized coordinates}\ru{обобщёнными координатами}
\en{and}\ru{и}~\en{corresponding}\ru{соответствующими}
\en{generalized velocities}\ru{обобщёнными скоростями}
\en{in}\ru{в}~\href{https://en.wikipedia.org/wiki/Configuration_space_(physics)}{\en{configuration space}\ru{конфигурационном пространстве}})
\en{and}\ru{и}~%
\href{https://en.wikipedia.org/wiki/Hamiltonian_mechanics}{Hamilton\en{ian}\ru{’ова} \en{mechanics}\ru{механика}}
(\en{operating with}\ru{оперирующая}
\en{coordinates}\ru{координатами}
\en{and}\ru{и}~\en{corresponding}\ru{соответствующими}
\en{momenta}\ru{импульсами} %conjugate (for generalized coordinates) generalized momenta
\en{in}\ru{в}~\href{https://en.wikipedia.org/wiki/Phase_space}{\en{phase space}\ru{фазовом пространстве}}).
\en{Both formulations}\ru{Обе формулировки}
\en{are equivalent}\ru{эквивалентны}
\en{by a~}\ru{преобразованием }Legendre\ru{’а}\en{ transformation}
\en{on the~generalized coordinates}\ru{по обобщённым координатам},
\en{velocities}\ru{скоростям}
\en{and}\ru{и}~\en{momenta}\ru{импульсам},
\en{therefore}\ru{поэтому}
\en{both}\ru{обе}
\en{contain}\ru{содержат}
\en{the~same information}\ru{ту~же информацию}
%%\footnote{%
%%In classical mechanics, phase space is the cotangent bundle of configuration space.}\hspace{-0.3ex}
\en{for describing}\ru{для описания}
\en{the~dynamics}\ru{динамики}
\en{of~a~system}\ru{системы}.

\begin{otherlanguage}{russian}

\en{Variational equation}\ru{Вариационное уравнение}~\eqref{discrete:principleofvirtualwork}
\en{is satisfied}\ru{удовлетворяется}
\en{at any moment}\ru{в~любой момент}
\en{of~time}\ru{времени}.
Проинтегрируем
\textcolor{magenta}{его (оттуда лишь равенство F = ma)}\footnote{%
${ \mathcolor{magenta}{\variation{\kineticenergyinmechanics}
= \scalebox{.8}{$ \displaystyle \sum_{\smash{k}} $} \hspace{.2ex} m_k \mathdotabove{\locationvector}_{\hspace{-0.1ex}k} \hspace{-0.16ex} \dotp \variation{\hspace{.1ex}\mathdotabove{\locationvector}_{\hspace{-0.1ex}k}}} \hspace{-0.5ex} }$ , \hspace{-0.1ex}
${%
\biggl(
\scalebox{.8}{$ \displaystyle \sum_{\smash{k}} $} \hspace{.2ex} m_k \mathdotabove{\locationvector}_{\hspace{-0.1ex}k} \hspace{-0.16ex} \dotp \variation{\hspace{.1ex}\locationvector_{\hspace{-0.1ex}k}} \hspace{-0.4ex}
\biggr)^{\hspace{-0.25em}\tikz[baseline=-0.2ex]\draw[black, fill=black] (0,0) circle (.28ex);} \hspace{-0.1ex}
= \hspace{.2ex}
\scalebox{.8}{$ \displaystyle \sum_{\smash{k}} $} \hspace{.2ex} m_k \mathdotdotabove{\locationvector}_{\hspace{-0.1ex}k} \hspace{-0.16ex} \dotp \variation{\hspace{.1ex}\locationvector_{\hspace{-0.1ex}k}}
+ \hspace{-0.2ex} \tikzmark{beginVariationOfKinetic} \hspace{.32ex} \scalebox{.8}{$ \displaystyle \sum_{\smash{k}} $} \hspace{.2ex} m_k \mathdotabove{\locationvector}_{\hspace{-0.1ex}k} \hspace{-0.16ex} \dotp \variation{\hspace{.1ex}\mathdotabove{\locationvector}_{\hspace{-0.1ex}k}} \tikzmark{endVariationOfKinetic}
}$
\\[.5em]
${%
\Rightarrow \hspace{.6em}
\biggl[ \hspace{.2ex}
\scalebox{.8}{$ \displaystyle \sum_{\smash{k}} $} \hspace{.2ex} m_k \mathdotabove{\locationvector}_{\hspace{-0.1ex}k} \hspace{-0.16ex} \dotp \variation{\hspace{.1ex}\locationvector_{\hspace{-0.1ex}k}} \hspace{.16ex}
\biggr]_{\hspace{-0.25ex}t_1}^{\hspace{-0.25ex}t_2}
= \hspace{-0.1ex}
\scalebox{.95}{$ \displaystyle \integral\displaylimits_{\mathclap{t_1}}^{\raisemath{.12em}{\mathclap{t_2}}} $} \scalebox{.8}{$ \displaystyle \sum_{\smash{k}} $} \hspace{.2ex} m_k \mathdotdotabove{\locationvector}_{\hspace{-0.1ex}k} \hspace{-0.16ex} \dotp \variation{\hspace{.1ex}\locationvector_{\hspace{-0.1ex}k}} \hspace{.25ex} dt \hspace{.4ex}
+ \hspace{-0.1ex} \scalebox{.95}{$ \displaystyle \integral\displaylimits_{\mathclap{t_1}}^{\raisemath{.12em}{\mathclap{t_2}}} $} \hspace{-0.1ex} \variation{\kineticenergyinmechanics} \hspace{.1ex} dt \hspace{.16ex}
}$}%
\AddUnderBrace[line width=.75pt][-0.1ex,-0.77em]{beginVariationOfKinetic}{endVariationOfKinetic}{${\scriptstyle \variation{\kineticenergyinmechanics}}$}
%
по~какому\hbox{-}либо промежутку
${\left[\hspace{.15ex} t_1, t_2 \hspace{.15ex}\right]}$

\nopagebreak\vspace{-0.25em}
\begin{equation}
\displaystyle \integral\displaylimits_{t_1}^{\raisemath{.12em}{t_2}}
\hspace{-0.4ex}
\biggl( \hspace{-0.25ex} \variation{\kineticenergyinmechanics}
+ \scalebox{.95}[.98]{$ \displaystyle \sum_{\smash{k}} $} \hspace{.16ex} \bm{F}_k \hspace{-0.1ex} \dotp \variation{\hspace{.1ex}\locationvector_{\hspace{-0.1ex}k}} \hspace{-0.4ex} \biggr) \hspace{-0.2ex} dt \hspace{.1ex}
- \hspace{-0.2ex} \left[ \hspace{.2ex} \scalebox{.95}[.98]{$ \displaystyle \sum_{\smash{k}} $} \hspace{.2ex} m_k \mathdotabove{\locationvector}_{\hspace{-0.1ex}k} \hspace{-0.1ex} \dotp \variation{\hspace{.1ex}\locationvector_{\hspace{-0.1ex}k}} \hspace{.16ex} \right]_{\hspace{-0.32ex}t_1}^{\hspace{-0.32ex}t_2}
\hspace{-0.8ex} = 0
\hspace{.1ex} .
\end{equation}

\vspace{-0.16em}\noindent
Без ущерба для общности
можно принять
${\variation{\hspace{.1ex}\locationvector_{\hspace{-0.1ex}k}}\hspace{.1ex}(t_1) \hspace{-0.1ex} = \variation{\hspace{.1ex}\locationvector_{\hspace{-0.1ex}k}}\hspace{.1ex}(t_2) \hspace{-0.1ex} = \zerovector}$,
\en{then}\ru{тогда}
\en{the~non\hbox{-}integral term}\ru{внеинтегральный член}
\en{vanishes}\ru{исчезает}.

\ru{Вводятся }\en{Generalized coordinates}\ru{обобщённые координаты}~$q^i$~%
(${i = 1, \ldots, n}$\:---
\en{the~number of~degrees o’freedom}\ru{число степеней свободы})\en{ are introduced}.
\en{Location vectors}\ru{Векторы положения}
\en{become functions}\ru{становятся функциями}
\hbox{\en{like}\ru{вида}}
${\locationvector_{k}(q^i, t)}$,
тождественно удовлетворяющими
уравнениям связей~\eqref{holonomicconstraint}.
Если связи стационарны,
то~есть
уравнения~\eqref{holonomicconstraint}
не~содержат~$t$,
то остаётся~${\locationvector_{k}(q^i)}$.
\en{Kinetic energy}\ru{Кинетическая энергия}
превращается
в~функцию
${\kineticenergyinmechanics \hspace{.1ex} (q^i, \mathdotabove{q}^{\hspace{.2ex}i}, t)}$,
где явно входящее~$t$
характерно лишь для нестационарных связей.

\newcommand\partialoflocationbycoordinatewithbothindices[2]{%
\scalebox{.93}{$ \displaystyle%
   \frac{\raisemath{-0.08em}{\partial \hspace{.1ex} \locationvector_{#1}}}%
        {\raisemath{-0.08em}{\partial q^{#2}}} $}}

\en{Hello}\ru{Привет}
\en{to the~essential}\ru{существенному}
\en{concept}\ru{понятию}
\en{of~generalized forces}\ru{обобщённых сил}.
\en{They}\ru{Они}
\en{originate from}\ru{происходят из}
\en{the~virtual work}\ru{виртуальной работы}
${ \bm{F}_k \hspace{-0.2ex} \dotp \variation{\hspace{.1ex}\locationvector_{\hspace{-0.1ex}k}} }$.
\en{With }\ru{С~}\en{variation}\ru{вариацией}~${\variation{\hspace{.1ex}\locationvector_{\hspace{-0.1ex}k}}}$
\ru{вектора положения }\en{of the~}$k$\hbox{-}\en{th}\ru{ой}
\en{point’s}\ru{точки}\en{ location vector},
\en{expanded}\ru{развёрнутой}
\en{for generalized coordinates}\ru{для обобщённых координат}~${q^i}$,

\nopagebreak\vspace{-0.25em}
\begin{equation}\label{locationexpandedforgeneralizedcoordinates}
\variation{\hspace{.1ex}\locationvector_{\hspace{-0.1ex}k}}
= \hspace{-0.2ex}
\scalebox{.95}[.98]{$ \displaystyle \sum_{\smash{i}} $} \hspace{.3ex}
\partialoflocationbycoordinatewithbothindices{\hspace{-0.1ex}k}{i} \hspace{.2ex} \variation{q^i}
\hspace{.2ex} ,
\end{equation}

\vspace{-0.33em}\noindent
\en{the~virtual work}\ru{виртуальная работа}
\en{can be}\ru{может быть}
\en{written as}\ru{записана как}

\nopagebreak\vspace{-0.25em}
\begin{multline}
\scalebox{.8}{$ \displaystyle \sum_{\smash{k}} $} \hspace{.2ex}
\bm{F}_k \hspace{-0.2ex} \dotp \variation{\hspace{.1ex}\locationvector_{\hspace{-0.1ex}k}}
=
\scalebox{.8}{$ \displaystyle \sum_{\smash{k}} $} \hspace{.2ex}
\bm{F}_k \dotp \hspace{-0.2ex} \scalebox{.95}[.98]{$ \displaystyle \sum_{\smash{i}} $} \hspace{.3ex}
\partialoflocationbycoordinatewithbothindices{\hspace{-0.1ex}k}{i} \hspace{.2ex} \variation{q^i}
\hspace{-0.1ex} = \hspace{-0.2ex}
\scalebox{.95}[.98]{$ \displaystyle \sum_{\smash{i,k}} $} \hspace{.2ex}
\bm{F}_k \hspace{-0.1ex} \dotp
\partialoflocationbycoordinatewithbothindices{\hspace{-0.1ex}k}{i} \hspace{.2ex} \variation{q^i}
\\
%
=
\scalebox{.8}{$ \displaystyle \sum_{\smash{i}} $}
\biggl( \scalebox{.95}[.98]{$ \displaystyle \sum_{\smash{k}} $} \hspace{.2ex}
\bm{F}_k \hspace{-0.1ex} \dotp
\partialoflocationbycoordinatewithbothindices{\hspace{-0.1ex}k}{i} \hspace{-0.2ex} \biggr) \hspace{-0.1ex} \variation{q^i}
\hspace{-0.1ex} = \hspace{-0.1ex}
\scalebox{.8}{$\displaystyle \sum_{i}$} \hspace{.16ex} Q_{\hspace{-0.1ex}i} \hspace{.12ex} \variation{q^i}
\hspace{.1ex} ,
\end{multline}

\vspace{-0.25em}\noindent
\en{where}\ru{где}

\nopagebreak\vspace{-1em}
\begin{equation}\label{whatisageneralizedforce.definition}
Q_{\hspace{-0.1ex}i} \equiv
\scalebox{.95}[.98]{$ \displaystyle \sum_{\smash{k}} $} \hspace{.2ex}
\bm{F}_k \hspace{-0.1ex} \dotp
\partialoflocationbycoordinatewithbothindices{\hspace{-0.1ex}k}{i}
\hspace{.2ex} .
\vspace{-0.1em}\end{equation}

\vspace{-0.2em}\noindent
\en{It’s worth}\ru{Ст\'{о}ит}
\en{to~accentuate}\ru{акцентировать}
\en{once more}\ru{ещё раз}
\en{the~origin}\ru{происхождение}
\en{of~generalized forces}\ru{обобщённых сил}
\en{from work}\ru{от~работы}.
\en{Having chosen}\ru{Выбрав}
\en{the~generalized coordinates}\ru{обобщённые координаты}~${q^i}$
\en{for the~problem}\ru{для проблемы},
\en{the~applied forces}\ru{приложенные силы}~${\bm{F}_k}$
\en{are then grouped}\ru{группируются затем}
\en{into the~sets}\ru{в~наборы}
\en{of~generalized forces}\ru{обобщённых сил}~${Q_{\hspace{-0.1ex}i}}$.

\en{The~particular case}\ru{Частный случай}
\en{of~potential forces}\ru{потенциальных сил}
\en{is very relevant}\ru{очень актуален}
\en{for}\ru{для}
\en{this book}\ru{этой книги}.
%
\en{A~force}\ru{Сила}
\en{is }\emph{\en{potential}\ru{потенциальна}}\ru{,}
\en{when}\ru{когда}
\en{the~work done by it}\ru{совершённая ей работа}
\en{depends only}\ru{зависит только}
\en{on locations of~points}\ru{от положения точек},
\en{but not}\ru{но не}
\en{on paths between them}\ru{от путей между ними}.
%
\en{Then}\ru{Тогда}
\en{the~potential energy}\ru{потенциальная энергия}~${\potentialenergyinmechanics}$
\en{can be introduced}\ru{может быть введена}
\en{as}\ru{как}
\en{a~scalar field}\ru{скалярное поле},
\en{also dubbed}\ru{также называемое}
\en{a~}\inquotes{\en{potential field}\ru{потенциальным полем}}
\en{or just}\ru{или просто}
\inquotesx{\en{a~potential}\ru{потенциалом}}[,]
\en{because it is}\ru{потому что это}
\en{a~function}\ru{функция}
\en{of~only}\ru{только}
\en{coordinates}\ru{координат}~${\potentialenergyinmechanics \narroweq \potentialenergyinmechanics (q^i)}$
(\en{and possibly time}\ru{и, возможно, времени},
${\potentialenergyinmechanics (q^i \hspace{-0.3ex} , t)}$\:---
\en{the~explicit dependence}\ru{явная зависимость}
\en{on~time}\ru{от~времени}~$t$
\en{may appear}\ru{может появиться}
\en{due to}\ru{из\hbox{-}за}
\en{non\hbox{-}stationary constraints}\ru{нестационарных связей}
\en{or}\ru{или}
\en{because}\ru{потому, что}
\en{the~physical fields}\ru{физические поля}
\en{themselves}\ru{сами по~себе}
\en{depend}\ru{зависят}
\en{on~time}\ru{от~времени}).
%
%%\en{In generalized coordinates}\ru{В~обобщённых координатах}
${ \variation{\potentialenergyinmechanics}
\hspace{-0.1ex} = \hspace{-0.2ex} \scalebox{.8}[1]{$\displaystyle \sum$} \hspace{.3ex}
\scalebox{.8}{$ \displaystyle \frac{\raisemath{-0.3ex}{\partial \hspace{.2ex} \potentialenergyinmechanics}}{\raisemath{-0.1ex}{\partial q^i}} $} \hspace{.2ex} \variation{q^i} }$\en{ is}\ru{\:---}
\en{a~variation}\ru{вариация}
\en{of~energy}\ru{энергии}~$\potentialenergyinmechanics$.
%
\en{The~function}\ru{Функция}~${\potentialenergyinmechanics}$
\en{is usually}\ru{обыкновенно}
\en{defined}\ru{определяется}
\en{so that}\ru{так, что}
\en{a~positive work}\ru{положительная работа}\ru{\:---}
\en{is a~reduction}\ru{это понижение}
\en{in the~potential}\ru{потенциала}.
%
\en{Thus}\ru{Так},
\en{if}\ru{если}
\en{generalized forces}\ru{обобщённые силы}
\en{are potential}\ru{потенциальны},
\en{then}\ru{то}

\nopagebreak\vspace{-0.2em}
\begin{equation}\label{generalizedforcesarepotential}
\scalebox{.8}{$\displaystyle \sum_{i}$} \hspace{.16ex} Q_{\hspace{-0.1ex}i} \hspace{.12ex} \variation{q^i} \hspace{-0.1ex}
= - \hspace{.2ex} \variation{\potentialenergyinmechanics}
\hspace{.1ex} ,
\hspace{.8em}
Q_{\hspace{-0.1ex}i} \hspace{-0.1ex} = - \hspace{.2ex} \scalebox{.96}{$ \displaystyle \frac{\raisemath{-0.12em}{\partial \hspace{.1ex} \potentialenergyinmechanics}}{\raisemath{-0.1em}{\partial q^i}} $}
\hspace{.2ex} .
\vspace{-0.1em}\end{equation}

...

\subsection*{\en{Lagrange’s equations of the first kind}\ru{Уравнения Lagrange’а первого рода}}

\en{Since}\ru{Поскольку}
\en{there are}\ru{существуют}
\ru{уравнения }Lagrange’\en{s}\ru{а}\en{ equations}
\inquotesx{\en{of~the~second}\ru{второго}
\en{kind}\ru{рода}}[,]
\en{the~reader}\ru{читатель}
\en{may guess that}\ru{может догадаться, что}
\en{equations}\ru{уравнения}
\inquotes{\en{of the~first}\ru{первого}
\en{kind}\ru{рода}}
\en{exist too}\ru{тоже существуют}.
%
\en{Yes}\ru{Да},
\en{they are known}\ru{они известны}.
%
\en{And}\ru{И}~\en{they are worth mentioning}\ru{о~них ст\'{о}ит упомянуть}
\en{at least because}\ru{хотя бы потому, что}
\en{the~derivation method}\ru{метод вывода},
\en{founded}\ru{основанный}
\en{on these equations}\ru{на этих уравнениях},
\en{is used}\ru{используется}
\en{in this book}\ru{в~этой книге}
\en{many times}\ru{много раз}.

\en{When}\ru{Когда}
\en{constraints}\ru{связи}~\eqref{holonomicconstraint}
\en{are imposed}\ru{наложены}
\en{on a~system}\ru{на систему},
\en{the~equality}\ru{равенство}
${\bm{F}_k \hspace{-0.12ex} = m_k \mathdotdotabove{\locationvector}_{\hspace{-0.1ex}k}}$
\en{doesn’t follow}\ru{не~следует}
\en{from}\ru{из}
\en{the~variational equation}\ru{вариационного уравнения}~\eqref{discrete:principleofvirtualwork},
\en{since}\ru{ведь}
\en{virtual displacements}\ru{виртуальные смещения}~${\variation{\hspace{.1ex}\locationvector_{\hspace{-0.1ex}k}}}$
\en{are then not independent}\ru{тогда не~независимы}.
%
\en{Having}\ru{Имея}
$m$~\en{constraints}\ru{связей}
\en{and therefore}\ru{и~поэтому}
$m$~\en{conditions}\ru{условий}~\eqref{requirementforvirtualdisplacements}
\en{for}\ru{для}
\en{variations}\ru{вариаций},
\en{each}\ru{каждое}
\en{of these conditions}\ru{из этих условий}
\en{is multiplied}\ru{умножается}
\en{by some scalar}\ru{на~некий скаляр}~$\lambda_{a}$~(${a = 1, \ldots, m}$)
\en{and}\ru{и}~\en{added}\ru{добавляется}
\en{to}\ru{к}~\eqref{discrete:principleofvirtualwork},
\en{turning into}\ru{превращаясь~в}

\nopagebreak\vspace{-0.3em}\begin{equation}
\scalebox{.92}[.96]{$ \displaystyle \sum_{k=1}^{N} $} \hspace{-0.1ex}
\biggl( \hspace{-0.22ex}
   \activeforcewithindex{k}
   \hspace{-0.1ex} + \hspace{-0.2ex}
   \scalebox{.92}[.96]{$ \displaystyle \sum_{a=1}^{m} $} \hspace{.1ex} \lambda_{a} \hspace{.2ex}
   \scalebox{.9}{$\displaystyle \frac{\raisemath{-0.12em}{\partial \hspace{.1ex} c_{a}}}{\partial \hspace{.1ex} \locationvector_{\hspace{-0.1ex}k}}$}
   - m_k \mathdotdotabove{\locationvector}_{\hspace{-0.1ex}k}
\hspace{-0.22ex} \biggr) \hspace{-0.4ex}
\dotp \variation{\hspace{.1ex}\locationvector_{\hspace{-0.1ex}k}}
\hspace{-0.2ex} = 0
\hspace{.2ex} .
\end{equation}

\vspace{-0.1em}\noindent
\en{Among}\ru{Среди}
${3N \hspace{-0.2ex}}$~\en{components}\ru{компонент}
\en{of~variations}\ru{вариаций}~${\variation{\hspace{.1ex}\locationvector_{\hspace{-0.1ex}k}}}$,
$m$~\en{are dependent}\ru{зависимых}.
\en{Aha}\ru{Ага},
\en{and }\ru{а~}%
\en{the~number}\ru{число}
\en{of~}\ru{множителей }Lagrange\ru{’а}\en{ multipliers}~$\lambda_{a}$
\en{is}\ru{тоже}~$m$\en{ too}.
%
Если выбрать $\lambda_{\alpha}$
такие, что
\en{coefficients}\ru{коэффициенты}\textcolor{red}{(??как\'{и}е?)}
\en{for}\ru{для}
\en{dependent variations}\ru{зависимых вариаций}
обращаются в~нуль,
то тогда
у~остальных вариаций
коэффициенты\textcolor{red}{(??)}
также будут нулевые
из\hbox{-}за независимости.
Следовательно,
все выражения
в~скобках~${(\cdots\hspace{-0.2ex})}$
равны нулю\:---
это и~есть
\ru{уравнения }Lagrange’\en{s}\ru{а}\en{ equations}
\en{of~the~first kind}\ru{первого рода}.

\en{Since}\ru{Поскольку}
\en{for each particle}\ru{для каждой частицы}

...



\end{otherlanguage}

