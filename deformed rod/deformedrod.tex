\documentclass[tikz,margin=5]{standalone}

\usepgfmodule{nonlineartransformations}
\usepgflibrary{curvilinear}

\usepackage{tikz}
\usepackage{tikz-3dplot} % needs tikz-3dplot.sty in same folder
\usetikzlibrary{calc}
\usetikzlibrary{arrows, arrows.meta}

\usepackage{bm}

\begin{document}

%%\begin{center}

\def\cameraangle{100}
\tdplotsetmaincoords{66}{\cameraangle} % orientation of camera

\def\rodheight{8}
\def\rodradius{0.2}

\pgfmathsetmacro{\beginangle}{\cameraangle}
\pgfmathsetmacro{\endangle}{\cameraangle - 180}

\tikzset{pics/rod/.style={code={

	\coordinate (O) at ( 0, 0, 0 ) ;
	\coordinate (rodTopCenter) at ($ (O) + ( 0, 0, \rodheight ) $) ;

	% draw rod

	%%\foreach \height in { 0, 0.02, ..., \rodheight }
		%%\draw [line width=0.8pt, color=yellow, fill=yellow]
			%%($ (O) + ( 0, 0, \height ) $) circle ( \rodradius ) ;

	\pgfmathsetmacro{\stepangle}{\beginangle - 4}
	\foreach \angle in { \beginangle, \stepangle, ..., \endangle }
		\draw [line width=0.8pt, color=yellow, opacity=.9]
			( \angle:\rodradius ) -- ($ ( \angle:\rodradius ) + ( 0, 0, \rodheight ) $) ;

	\draw [line width=0.8pt, color=black, domain=\beginangle:\endangle]
		plot ( {\rodradius*cos(\x)}, {\rodradius*sin(\x)}, 0 ) ;

	\draw [line width=0.85pt, color=black, line cap=round]
		( \beginangle:\rodradius ) -- ($ ( \beginangle:\rodradius ) + ( 0, 0, \rodheight ) $) ;
	\draw [line width=0.85pt, color=black, line cap=round]
		( \endangle:\rodradius ) -- ($ ( \endangle:\rodradius ) + ( 0, 0, \rodheight ) $) ;

	\draw [line width=0.8pt, color=yellow, fill=yellow, opacity=.9] ( 0, 0, \rodheight ) circle ( \rodradius ) ;

	\draw [line width=0.8pt, color=black, domain=\beginangle:\endangle]
		plot ( {\rodradius*cos(\x)}, {\rodradius*sin(\x)}, \rodheight )
		plot ( {-\rodradius*cos(\x)}, {-\rodradius*sin(\x)}, \rodheight ) ;

}}}

\tikzset{pics/rodaxis/.style={code={

	% draw axis
	\draw [line width=0.5pt, blue, line cap=round, dash pattern=on 12pt off 2pt on \the\pgflinewidth off 2pt]
		( 0, 0, -0.4pt ) -- ($ ( 0, 0, \rodheight ) + ( 0, 0, 0.4pt ) $) ;

}}}

\begin{tikzpicture}[scale=1, tdplot_main_coords] % use 3dplot

	\coordinate (O) at ( 0, 0, 0 ) ;
	\coordinate (rodTopCenter) at ($ (O) + ( 0, 0, \rodheight ) $) ;

	% draw circle
	\def\circleradius{0.8}
	\def\heightofhatch{0.5}

	\pgfmathsetmacro{\stepangleforcircle}{\beginangle - 10}
	\foreach \angle in { \beginangle, \stepangleforcircle, ..., \endangle }
		\draw [line width=0.4pt, color=black]
			( \angle:\circleradius ) -- ($ ( \angle:\circleradius ) - ( 0, 0, \heightofhatch ) $) ;

	\draw [line width=0.8pt, color=black, fill=white] (O) circle ( \circleradius ) ;

	% draw rod
	\pic (initial) {rod} ;
	\pic (initial) {rodaxis} ;

	% draw force
	\def\forcelength{1.2}

	\draw [line width=1.4pt, blue, line cap=round, -{Triangle[round, length=3.6mm, width=2.4mm]}]
		($ (rodTopCenter) + ( 0, 0, \forcelength ) $) -- (rodTopCenter)
		node [pos=0.5, above left, inner sep=0, outer sep=3.2pt]
			{\scalebox{1.2}[1.2]{${\bm{F}}$}} ;

	\scoped {
		\pgfsetcurvilinearbeziercurve
			{\pgfpointxyz{0}{0}{0}}
			{\pgfpointxyz{0.1}{0.1}{1.5}}
			{\pgfpointxyz{0.25}{0.25}{1.75}}
			{\pgfpointxyz{0.5}{0.5}{2.5}}
		\pgftransformnonlinear{\pgfgetlastxy\x\y\pgfpointcurvilinearbezierorthogonal{\y}{\x}}
			\pic (deformed) {rod} ;
			\pic (deformed) {rodaxis} ;
	}

\end{tikzpicture}
%%\end{center}

\end{document}
