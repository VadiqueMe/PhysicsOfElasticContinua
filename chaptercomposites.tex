\en{\chapter{Composites}}

\ru{\chapter{Композиты}}

\thispagestyle{empty}

\label{chapter:composites}

\en{\section{Introductory thoughts}}

\ru{\section{Вводные размышления}}

\dropcap{\en{U}\ru{И}}{\en{sing}\ru{спользуя}} \en{clay}\ru{глину} \en{as a~building material}\ru{как строительный материал}, \ru{в~неё добавляют }\en{shredded straw}\ru{измельчённую солому}\en{ is added to it}.
\en{Working}\ru{Работая} \en{with }\ru{с~}\en{epoxy gum}\ru{эпоксидной смолой}, \en{it’s useful}\ru{полезно} \en{before solidifying}\ru{до~отвердевания} \en{to~blend with a~filler}\ru{смешать с~наполнителем}: \en{powder}\ru{порошком}, \en{fibers}\ru{волокнами}, \en{pieces o’\;fabric}\ru{кусочками ткани}.
\en{These are examples}\ru{Это примеры} \en{of composites}\ru{композитов}~(\en{composite materials}\ru{композитных материалов}, \en{composite mixtures}\ru{композитных смесей}).
\en{New types of~composites}\ru{Новые типы композитов} \en{are~used}\ru{применяются} \en{more and~more widely}\ru{всё~шире}, \en{displacing}\ru{вытесняя} \en{steel}\ru{сталь}, \en{aluminum alloys}\ru{алюминиевые сплавы} \en{and}\ru{и}~\en{other popular homogeneous materials}\ru{другие популярные однородные материалы}.

\begin{otherlanguage}{russian}

\en{Composites}\ru{Композиты} \en{can be defined}\ru{могут быть определены} \en{as}\ru{как} (\hbox{\en{micro}\ru{микро}-\hspace{-0.1ex}})\en{inhomogeneous materials}\ru{неоднородные материалы}, в~которых происходит некое осреднение с~возникновением новых свойств.
\inquotes{\en{Usual}\ru{Обычная}} \en{mechanics of~continua}\ru{механика контину\kern-0.11exумов} \en{is applicable}\ru{применима}, \en{surely}\ru{конечно~же}, \en{to~composites too}\ru{и~к~композитам тоже}.
\en{But}\ru{Но} \en{it’s barely possible}\ru{едва~ли возможно} \en{to model}\ru{смоделировать} \en{each aspect}\ru{каждый аспект} \en{of a~composite material}\ru{композитного материала}\:--- \en{and rather}\ru{и~весьма} \en{absurd}\ru{абсурдно}.
\en{Some new approach is needed}\ru{Нужен некий новый подход}\ru{,} \en{that will deal}\ru{который будет иметь дело} \en{with}\ru{со} \en{the~complexity of~material structure}\ru{сложностью структуры материала}.
\en{As for a~gas}\ru{Как для газа}, \en{when}\ru{когда} \en{instead of describing}\ru{вместо описания} \en{dynamics of individual molecules}\ru{динамики отдельных молекул} \en{we introduce}\ru{мы вводим} \en{parameters}\ru{параметры} \inquotesx{\en{pressure}\ru{давление}}[,] \inquotes{\en{temperature}\ru{температура}} \en{and}\ru{и}~\en{others}\ru{другие}.

\en{Mechanics of a~homogeneous non\hbox{-}composite continuum}\ru{Механика однородного некомпозитного контину\kern-0.11exума} \en{features}\ru{являет} \en{only single}\ru{лишь одну}\:--- \en{macroscopic}\ru{макроскопическую}\:--- \en{length}\ru{длину}~$\elcursive$\hbox{\hspace{-0.2ex},} \en{the~characteristic dimension}\ru{характерный размер} \en{of~a~body}\ru{тела} (\en{volume divided by surface area}\ru{объём, делённый на пл\'{о}щадь поверхности}), \en{assuming that}\ru{полагая что} \en{volumes of~any smallness}\ru{объёмы любой малости} \en{have the~same properties as}\ru{имеют такие~же свойства, как~и} \en{finite ones}\ru{конечные}.

\en{For a~composite}\ru{Для композита}\:--- \en{divergently}\ru{по\hbox{-}иному},
\en{giving here}\ru{давая здесь} \en{three scopes}\ru{три диапазона} \en{of~length}\ru{длины}:
${\elcursive \gg \hspace{-0.1ex} \elcursive\smash{\raisemath{.22em}{\hspace{.25ex}'}}\hspace{-0.45ex} \gg \hspace{-0.1ex} \elcursivesub{0}}$.
\en{Largest}\ru{Наибольший}~$\elcursive$ \en{presents}\ru{представляет} \en{dimensions of a~body}\ru{размеры тела}.
\en{Smallest}\ru{Наименьший}~$\elcursivesub{0}$\en{ is}\ru{\:---} \en{around}\ru{вокруг} \en{the~size}\ru{размера} \en{of~elements}\ru{элементов} \en{of~material structure}\ru{структуры материала} (\en{for example}\ru{например}, \en{particulates}\ru{частичек} \en{of a~filler powder}\ru{порошка\hbox{-}наполнителя}).
\en{Intermediate}\ru{Промежуточный} \en{scope}\ru{диапазон} ${\elcursive\smash{\raisemath{.22em}{\hspace{.25ex}'}}\hspace{-0.5ex}}$ \en{exposes}\ru{показывает} \en{the~size}\ru{размер} \en{of a~so-called}\ru{так-называемого} \href{https://en.wikipedia.org/wiki/Representative_elementary_volume}{\inquotes{\en{representative}\ru{представительного}} \en{volume}\ru{объёма}, \inquotes{\en{unit cell}\ru{единичной ячейки}}}: того минимального объёма, в~котором проявляются \en{specific properties}\ru{специфические свойства} \en{of this composite}\ru{этого композита}.

\en{In}\ru{В}~\en{composites}\ru{композитах}\en{,} \en{the~complex problem}\ru{сложная проблема} \en{for an~inhomogeneous body}\ru{для неоднородного тела} \en{is split into two}\ru{расщепляется на~две}: \en{for a~body in a~whole}\ru{для тела в~целом} (\en{macrolevel}\ru{макроуровень}) \en{and for}\ru{и~для} \en{a~}\inquotes{\en{representative}\ru{представительного}} \en{volume}\ru{объёма} (\en{microlevel}\ru{микроуровень}).
На~макроуровне~(\en{scope}\ru{диапазон}~${\elcursive\hspace{.25ex}}$) композит моделируется однородной средой с~\inquotes{эффективными} свойствами, \en{where}\ru{где} \inquotes{\en{representative}\ru{представительные}} \en{volumes}\ru{объёмы} играют роль частиц.
На~микроуровне~(\en{scope}\ru{диапазон}~$\elcursivesub{0}$) поля очень неоднородны, но некое осреднение по \inquotes{представительному} объёму выводит на~макроуровень.

% mesoscopic, on a scale between microscopic and macroscopic
% мезоскопический

...



\end{otherlanguage}

\en{\section{Effective fields}}

\ru{\section{Эффективные поля}}

\begin{otherlanguage}{russian}

Любое поле в~композите обычно представляется суммой

...



\end{otherlanguage}

\en{\section{Boundary value problems for representative volume}}

\ru{\section{Краевые задачи для представительного объёма}}

\begin{otherlanguage}{russian}

Как определяются упругие модели для~однородной среды?
\en{Without the~real possibility}\ru{Без реальной возможности} \en{to get}\ru{получить} \en{a~relation}\ru{отношение} \en{between}\ru{между} ${\hspace{-0.1ex}\linearstress\hspace{.1ex}}$ \en{and}\ru{и}~$\infinitesimaldeformation$ \en{for a~point of infinitesimal size}\ru{для точки бесконечно-малого размера}, \en{experiments are carried out}\ru{эксперименты проводятся} \en{with bodies of~finite volumes}\ru{с~телами конечных объёмов} \en{under}\ru{при} \en{a~so\hbox{-}called}\ru{так называемом} \inquotesx{\en{homogeneous stress}\ru{однородном напряжении}}[---] \en{when}\ru{когда} \en{stress}\ru{напряжение} \en{is the~same}\ru{одно и~то~же} \en{at all points of a~body}\ru{во~всех точках тела}.
В~композитах роль точки играет \en{a~}\inquotes{\en{representative}\ru{представительный}} \en{volume}\ru{объём}.

...

\end{otherlanguage}

\en{\section{Hill’s fork}}

\ru{\section{Вилка Hill’а}}

\begin{otherlanguage}{russian}

{\small
\noindent
Using Voigt and Reuss theories, Hill derived upper and lower bounds on the effective properties of a composite material [\bibauthor{Hill,}[R.][W.] The~elastic behaviour of a~crystalline aggregate. \emph{Proceedings of the~Physical Society}, Section~A, Volume~65, Issue~5 (May 1952). Pages 349\hbox{--}354.]

The scale separation is motivated by the material properties, at both scales continuum mechanics models the underlying system.
Such an approach uses energy equivalence at both scales as proposed in Hill (1972).

Hill R (1972) On constitutive macro-variables for heterogeneous solids at finite strain (pages 131–147)

For a composite material, at least two different materials with known material models and parameters, generate a homogenized material modeled with a predetermined constitutive equation. Determination of material parameters of the homogenized material is a challenging task.
\par}

Отметив, что

...



\end{otherlanguage}

\en{\section{Eshelby formulas}}

\ru{\section{Формулы Eshelby}}

\begin{otherlanguage}{russian}

Итак, эффективные модули определяются энергией представительного объёма в~первой или~второй задачах:

...



\end{otherlanguage}

\ru{\section{Эффективные модули для материала со сферическими включениями}}

\en{\section{Effective moduli for material with spherical inclusions}}

\begin{otherlanguage}{russian}

В~однородной матрице случайным образом, но достаточно равномерно, распределены сферические включения радиусом~$a$. Получившийся композит на~макроуровне будет изотропным, его упругие свойства полностью определяются ...

...



\end{otherlanguage}

\en{\section{Self-consistent method}}

\ru{\section{Метод самосогласования}}

\begin{otherlanguage}{russian}

Мы опирались на~две задачи для представительного объёма и~определяли эффективные модули из~равенства энергий. В~\hbox{основе} метода самосогласования лежит новая идея: представительный объём помещается в~безграничную среду с~эффективными свойствами, на~бесконечности состояние считается однородным, эффективные модули находятся из~некоторых дополнительных условий самосогласования.

Обратимся снова к~вопросу об~объёмном модуле среды со~сферическими включениями. Задача сферически симметрична; для~включения по\hbox{-}прежнему

...



\end{otherlanguage}

\en{\section{Hashin\hbox{--}Shtrikman principle}}

\ru{\section{Принцип Хашина\hbox{--}Штрикмана}}

{\small
\noindent
Hashin and Shtrikman derived upper and lower bounds for the effective elastic properties of quasi-isotropic and quasi-homogeneous multiphase materials using a variational approach [\bibauthor{Hashin,}[Z.]; \bibauthor{Shtrikman,}[S.] A~variational approach to the~theory of the~elastic behaviour of multiphase materials~// Journal of the Mechanics and Physics of Solids. Volume~11, Issue~2 (March--April 1963). Pages 127\hbox{--}140.]

Hashin Z, Shtrikman S (1962) On some variational principles in anisotropic and nonhomogeneous elasticity. Journal of the Mechanics and Physics of Solids 10(4): pages 335–342
\par}

\begin{otherlanguage}{russian}

Вилка Hill’а основана на~обычных экстремальных принципах теории упругости. Специально для~механики композитов Hashin и~Shtrikman построили очень своеобразный функционал, который на~некотором точном решении может иметь как~максимум, так~и~минимум, давая возможность с~двух сторон оценивать эффективные модули~\cite{shermergor}.

Рассмотрим первую из~двух задач для представительного объёма

...



\end{otherlanguage}

\section*{\small \wordforbibliography}

\begin{changemargin}{\parindent}{0pt}
\fontsize{10}{12}\selectfont

\begin{otherlanguage}{russian}

Книги R.\:Christensen’а~\cite{christensen-compositematerials} и Б.\,Е.\:Победри~\cite{pobedrya-composites} содержат и~основы механики композитов, и~постановку не~теряющих актуальности проблем. Для~с\'{а}мого требовательного читателя представляет интерес монография Т.\,Д.\:Шермергора~\cite{shermergor}. Немало книг посвящено механике разрушения композитов, здесь ст\'{о}ит отметить труд Г.\,П.\:Черепанова~\cite{cherepanov-compositematerialfracture}.

\end{otherlanguage}

\end{changemargin}
