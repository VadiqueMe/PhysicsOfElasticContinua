\en{\chapter{Composites}}

\ru{\chapter{Композиты}}

\thispagestyle{empty}

\label{chapter:composites}

\en{\section{Introductory thoughts}}

\ru{\section{Вводные размышления}}

\begin{otherlanguage}{russian}

\lettrine[lines=2, findent=2pt, nindent=0pt]{И}{спользуя} глину как строительный материал, в~неё добавляют измельчённую солому. Работая с~эпоксидной смолой, полезно до~затвердевания ввести наполнитель: порошок, волокна, кусочки ткани. Это примеры композитов~(композиционных материалов). Новые виды композитов применяются всё~шире, вытесняя сталь, алюминиевые сплавы и~другие распространённые материалы.

Композиты могут быть определены как неоднородные материалы, в~которых происходит некое осреднение с~возникновением новых свойств. Обычная механика сплошной среды применима, разумеется, и~к~композитам. Но~едва~ли возможно учесть все детали структуры~--- и~неразумно. Необходимы новые подходы, опирающиеся именно на~сложность структуры. Ведь, например, в~газе

...



\end{otherlanguage}

\en{\section{Effective fields}}

\ru{\section{Эффективные поля}}

\begin{otherlanguage}{russian}

Любое поле в~композите обычно представляется суммой

...



\end{otherlanguage}

\en{\section{Boundary value problems for representative volume}}

\ru{\section{Краевые задачи для представительного объёма}}

\begin{otherlanguage}{russian}

Как определяются упругие модели для~\inquotes{обычной} среды?

...



\end{otherlanguage}

\en{\section{Hill’s fork}}

\ru{\section{Вилка Хилла}}

\begin{otherlanguage}{russian}

{\small \noindent Using Voigt and Reuss theories, Hill derived upper and lower bounds on the effective properties of a composite material [\bibauthor{Hill,}[R.][W.] The~elastic behaviour of a~crystalline aggregate~// Proceedings of the~Physical Society. Section~A, Volume~65, Issue~5 (May 1952). Pages 349\hbox{--}354.]
\par}

Отметив, что

...



\end{otherlanguage}

\en{\section{Eshelby formulas}}

\ru{\section{Формулы Эшелби}}

\begin{otherlanguage}{russian}

Итак, эффективные модули определяются энергией представительного объёма в~первой или~второй задачах:

...



\end{otherlanguage}

\ru{\section{Эффективные модули для материала со сферическими включениями}}

\en{\section{Effective moduli for material with spherical inclusions}}

\begin{otherlanguage}{russian}

В~однородной матрице случайным образом, но достаточно равномерно, распределены сферические включения радиусом~$a$. Получившийся композит на~макроуровне будет изотропным, его упругие свойства полностью определяются ...

...



\end{otherlanguage}

\en{\section{Self-consistent method}}

\ru{\section{Метод самосогласования}}

\begin{otherlanguage}{russian}

Мы опирались на~две задачи для представительного объёма и~определяли эффективные модули из~равенства энергий. В~\hbox{основе} метода самосогласования лежит новая идея: представительный объём помещается в~безграничную среду с~эффективными свойствами, на~бесконечности состояние считается однородным, эффективные модули находятся из~некоторых дополнительных условий самосогласования.

Обратимся снова к~вопросу об~объёмном модуле среды со~сферическими включениями. Задача сферически симметрична; для~включения по\hbox{-}прежнему

...



\end{otherlanguage}

\en{\section{Hashin\hbox{--}Shtrikman principle}}

\ru{\section{Принцип Хашина\hbox{--}Штрикмана}}

{\small \noindent Hashin and Shtrikman derived upper and lower bounds for the effective elastic properties of quasi-isotropic and quasi-homogeneous multiphase materials using a variational approach [\bibauthor{Hashin,}[Z.]; \bibauthor{Shtrikman,}[S.] A~variational approach to the~theory of the~elastic behaviour of multiphase materials~// Journal of the Mechanics and Physics of Solids. Volume~11, Issue~2 (March--April 1963). Pages 127\hbox{--}140.]
\par}

\begin{otherlanguage}{russian}

Вилка Hill’а основана на~обычных экстремальных принципах теории упругости. Специально для~механики композитов Hashin и~Shtrikman построили очень своеобразный функционал, который на~некотором точном решении может иметь как~максимум, так~и~минимум, давая возможность с~двух сторон оценивать эффективные модули~\cite{shermergor}.

Рассмотрим первую из~двух задач для представительного объёма

...



\end{otherlanguage}

\section*{\small \wordforbibliography}

\begin{changemargin}{\parindent}{0pt}
\fontsize{10}{12}\selectfont

\begin{otherlanguage}{russian}

Книги R.\:Christensen’а~\cite{christensen-compositematerials} и Б.\,Е.\:Победри~\cite{pobedrya-composites} содержат и~основы механики композитов, и~постановку не~теряющих актуальности проблем. Для~с\'{а}мого требовательного читателя представляет интерес монография Т.\,Д.\:Шермергора~\cite{shermergor}. Немало книг посвящено механике разрушения композитов, здесь ст\'{о}ит отметить труд Г.\,П.\:Черепанова~\cite{cherepanov-compositematerialfracture}.

\end{otherlanguage}

\end{changemargin}
