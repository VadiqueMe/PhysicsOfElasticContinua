\en{\chapter{Composites}}

\ru{\chapter{Композиты}}

\thispagestyle{empty}

\label{chapter:composites}

\en{\section{Introductory thoughts}}

\ru{\section{Вводные размышления}}

\lettrine[lines=2, findent=2pt, nindent=0pt]{\en{U}\ru{И}}{\en{sing}\ru{спользуя}} \en{clay}\ru{глину} \en{as a~building material}\ru{как строительный материал}, \ru{в~неё добавляют }\en{shredded straw}\ru{измельчённую солому}\en{ is added to it}.
\en{Working}\ru{Работая} \en{with }\ru{с~}\en{epoxy gum}\ru{эпоксидной смолой}, \en{it’s useful}\ru{полезно} \en{before solidifying}\ru{до~отвердевания} \en{to~blend with a~filler}\ru{смешать с~наполнителем}: \en{powder}\ru{порошком}, \en{fibers}\ru{волокнами}, \en{pieces o’\;fabric}\ru{кусочками ткани}.
\en{These are examples}\ru{Это примеры} \en{of composites}\ru{композитов}~(\en{composite materials}\ru{композиционных материалов}).
\en{New types of~composites}\ru{Новые типы композитов} \en{are~used}\ru{применяются} \en{more and~more widely}\ru{всё~шире}, \en{displacing}\ru{вытесняя} \en{steel}\ru{сталь}, \en{aluminum alloys}\ru{алюминиевые сплавы} \en{and}\ru{и}~\en{other popular homogeneous materials}\ru{другие популярные однородные материалы}.

\begin{otherlanguage}{russian}

%%\en{micro\hbox{-}inhomogeneous materials}\ru{микронеоднородные материалы}

Композиты могут быть определены как неоднородные материалы, в~которых происходит некое осреднение с~возникновением новых свойств. \inquotes{Обычная} механика сплошной среды применима, разумеется, и~к~композитам. Но~едва~ли возможно учесть все детали структуры\:--- и~неразумно. Нужны новые подходы, опирающиеся именно на~сложность структуры. Ведь, например, в~газе вместо описания динамики отдельных молекул вводят параметры \inquotesx{давление}[,] \inquotes{температура} и~другие.

В~композитах выделяются три масштаба длины: ${\elcursive \gg \hspace{-0.1ex} \elcursive\smash{\raisemath{.22em}{\hspace{.25ex}'}}\hspace{-0.45ex} \gg \hspace{-0.1ex} \elcursivesub{0}}$.
Наибольший\:--- характерный размер $\elcursive$ тела.
Наименьший\:--- размер $\elcursivesub{0}$ \en{of~elements}\ru{элементов} \en{of~material structure}\ru{структуры материала} (например, частиц порошка\hbox{-}наполнителя).
Промежуточный масштаб ${\elcursive\smash{\raisemath{.22em}{\hspace{.25ex}'}}\hspace{-0.5ex}}$ определяет размер так называемого \inquotes{представительного} объёма: того минимального объёма, в~котором проявляются характерные свойства композита.

В~механике некомпозитного однородного контину\kern-0.11exума существует лишь один масштаб~$\elcursive$~(размеры тела), а~любые малые объёмы рассматриваются так~же, как и конечные.
Для~композитов\:--- иначе, сложная задача для неоднородного тела ращепляется на~две: для тела в~целом (макроуровень) и~для \inquotes{представительного} объёма (микроуровень).
На~макроуровне~(масштаб~${\elcursive\hspace{.25ex}}$) композит моделируется однородной средой с~\inquotes{эффективными} свойствами, где \inquotes{представительные} объёмы играют роль частиц.
На~микроуровне~(масштаб~$\elcursivesub{0}$) поля очень неоднородны, но некое осреднение по \inquotes{представительному} объёму выводит на~макроуровень.

...



\end{otherlanguage}

\en{\section{Effective fields}}

\ru{\section{Эффективные поля}}

\begin{otherlanguage}{russian}

Любое поле в~композите обычно представляется суммой

...



\end{otherlanguage}

\en{\section{Boundary value problems for representative volume}}

\ru{\section{Краевые задачи для представительного объёма}}

\begin{otherlanguage}{russian}

Как определяются упругие модели для~однородной среды?
Без реальной возможности найти связь между ${\hspace{-0.1ex}\cauchystress\hspace{.1ex}}$ и~$\mathboldepsilon$ в~точке бесконечно\-малых размеров, \en{experiments are carried out}\ru{эксперименты проводятся} \en{with bodies of~finite volumes}\ru{с~телами конечных объёмов} \en{under}\ru{при} \en{a~so\hbox{-}called}\ru{так называемом} \inquotesx{\en{homogeneous stress}\ru{однородном напряжении}}[---] когда напряжение во~всех точках тела одно и~то~же.
В~композитах роль точки играет \inquotes{представительный} объём.

...

\end{otherlanguage}

\en{\section{Hill’s fork}}

\ru{\section{Вилка Хилла}}

\begin{otherlanguage}{russian}

{\small \noindent Using Voigt and Reuss theories, Hill derived upper and lower bounds on the effective properties of a composite material [\bibauthor{Hill,}[R.][W.] The~elastic behaviour of a~crystalline aggregate~// Proceedings of the~Physical Society. Section~A, Volume~65, Issue~5 (May 1952). Pages 349\hbox{--}354.]
\par}

Отметив, что

...



\end{otherlanguage}

\en{\section{Eshelby formulas}}

\ru{\section{Формулы Эшелби}}

\begin{otherlanguage}{russian}

Итак, эффективные модули определяются энергией представительного объёма в~первой или~второй задачах:

...



\end{otherlanguage}

\ru{\section{Эффективные модули для материала со сферическими включениями}}

\en{\section{Effective moduli for material with spherical inclusions}}

\begin{otherlanguage}{russian}

В~однородной матрице случайным образом, но достаточно равномерно, распределены сферические включения радиусом~$a$. Получившийся композит на~макроуровне будет изотропным, его упругие свойства полностью определяются ...

...



\end{otherlanguage}

\en{\section{Self-consistent method}}

\ru{\section{Метод самосогласования}}

\begin{otherlanguage}{russian}

Мы опирались на~две задачи для представительного объёма и~определяли эффективные модули из~равенства энергий. В~\hbox{основе} метода самосогласования лежит новая идея: представительный объём помещается в~безграничную среду с~эффективными свойствами, на~бесконечности состояние считается однородным, эффективные модули находятся из~некоторых дополнительных условий самосогласования.

Обратимся снова к~вопросу об~объёмном модуле среды со~сферическими включениями. Задача сферически симметрична; для~включения по\hbox{-}прежнему

...



\end{otherlanguage}

\en{\section{Hashin\hbox{--}Shtrikman principle}}

\ru{\section{Принцип Хашина\hbox{--}Штрикмана}}

{\small \noindent Hashin and Shtrikman derived upper and lower bounds for the effective elastic properties of quasi-isotropic and quasi-homogeneous multiphase materials using a variational approach [\bibauthor{Hashin,}[Z.]; \bibauthor{Shtrikman,}[S.] A~variational approach to the~theory of the~elastic behaviour of multiphase materials~// Journal of the Mechanics and Physics of Solids. Volume~11, Issue~2 (March--April 1963). Pages 127\hbox{--}140.]
\par}

\begin{otherlanguage}{russian}

Вилка Hill’а основана на~обычных экстремальных принципах теории упругости. Специально для~механики композитов Hashin и~Shtrikman построили очень своеобразный функционал, который на~некотором точном решении может иметь как~максимум, так~и~минимум, давая возможность с~двух сторон оценивать эффективные модули~\cite{shermergor}.

Рассмотрим первую из~двух задач для представительного объёма

...



\end{otherlanguage}

\section*{\small \wordforbibliography}

\begin{changemargin}{\parindent}{0pt}
\fontsize{10}{12}\selectfont

\begin{otherlanguage}{russian}

Книги R.\:Christensen’а~\cite{christensen-compositematerials} и Б.\,Е.\:Победри~\cite{pobedrya-composites} содержат и~основы механики композитов, и~постановку не~теряющих актуальности проблем. Для~с\'{а}мого требовательного читателя представляет интерес монография Т.\,Д.\:Шермергора~\cite{shermergor}. Немало книг посвящено механике разрушения композитов, здесь ст\'{о}ит отметить труд Г.\,П.\:Черепанова~\cite{cherepanov-compositematerialfracture}.

\end{otherlanguage}

\end{changemargin}
