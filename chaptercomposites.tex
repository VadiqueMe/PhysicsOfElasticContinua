\en{\chapter{Composites}}

\ru{\chapter{Композиты}}

\thispagestyle{empty}

\label{chapter:composites}

\en{\section{Introductory thoughts}}

\ru{\section{Вводные размышления}}

\label{para:composites.thoughts}

\dropcap{\en{W}\ru{К}}{\en{hen}\ru{огда}}
\en{the clay}\ru{глина}
\en{is used}\ru{используется}
\en{as a~building material}\ru{как строительный материал},
\ru{в~неё добавляют }\en{a~shredded straw}\ru{измельчённую солому}\en{ is added to it}.
\en{Working}\ru{Работая} \en{with }\ru{с~}\en{an~epoxy gum}\ru{эпоксидной смолой},
\en{it’s useful}\ru{полезно}
\en{to~blend with a~filler}\ru{смешать с~наполнителем}
\en{before solidifying}\ru{до~отвердевания},
\en{which}\ru{который}
\en{can be}\ru{может быть}
\en{a~powder}\ru{порошком}, \en{fibers}\ru{волокнами}, \en{pieces o’\;fabric}\ru{кусочками ткани}.
\en{These were}\ru{Это были}
\en{examples}\ru{примеры}
\en{of composites}\ru{композитов}
(\en{composite materials}\ru{композитных материалов},
\en{composite mixtures}\ru{композитных смесей}).
\en{The new types of~composites}\ru{Новые типы композитов}
\en{are~used}\ru{применяются}
\en{more and more widely}\ru{всё шире и~шире},
\en{replacing}\ru{заменяя}
\en{the steel}\ru{сталь},
\en{the aluminum alloys}\ru{алюминиевые сплавы}
\en{and}\ru{и}~\en{another popular homogeneous materials}\ru{другие популярные однородные материалы}.

\en{The composites}\ru{Композиты}
\en{can be defined}\ru{могут быть определены}
\en{as}\ru{как}
\en{the micro}\ru{микро}-\hspace{-0.1ex}\en{inhomogeneous}\ru{неоднородные}
\en{materials}\ru{материалы},
\en{where happens}\ru{где происходит}
\en{some averaging}\ru{некое осреднение}
\en{together with the new}\ru{вместе с~новыми}
\en{properties}\ru{свойствами}.
\en{The~}\inquotes{\en{usual}\ru{Обычная}}
\en{mechanics of~continua}\ru{механика \rucontinuum{}ов}
\en{is applicable}\ru{применима},
\en{surely}\ru{конечно~же},
\en{to the~composite materials too}\ru{и~к~композитным материалам тоже}.
\en{But}\ru{Но}
\en{it’s barely possible}\ru{едва~ли возможно}
\en{and is absurd}\ru{и~абсурдно}
\en{to model}\ru{смоделировать}
\en{the~every aspect}\ru{каждый аспект}
\en{of a~composite material}\ru{композитного материала}.
\en{The new approach is needed}\ru{Нужен новый подход}\ru{,} \en{that will deal}\ru{который будет иметь дело}
\en{with}\ru{со}
\en{the~complexity of~material structure}\ru{сложностью структуры материала}.
\en{As for a~gas}\ru{Как для газа}, \en{when}\ru{когда} \en{instead of describing}\ru{вместо описания} \en{the dynamics of individual molecules}\ru{динамики отдельных молекул}
\en{we introduce}\ru{мы вводим}
\en{the parameters}\ru{параметры},
\en{such as}\ru{такие как}
\inquotesx{\en{the pressure}\ru{давление}}[,]
\inquotes{\en{the temperature}\ru{температура}}
\en{and}\ru{и}
\en{the others}\ru{другие}.

\en{The mechanics of a~homogeneous non\hbox{-}composite continuum}\ru{Механика однородного некомпозитного \rucontinuum{}а}
\en{uses}\ru{использует}
\en{the only single}\ru{лишь одну},
\en{macroscopic}\ru{макроскопическую}
\en{length}\ru{длину}~$\elcursive$\hbox{\hspace{-0.2ex}.}
\en{Thus}\ru{Потому}
\en{the~characteristic dimension}\ru{характерный размер}
\en{is}\ru{это}
\en{the~volume}\ru{объём}\ru{,}
\en{divided}\ru{делённый}
\en{by the~surface area}\ru{на пл\'{о}щадь поверхности}),
\en{assuming that}\ru{полагая что}
\en{any small}\ru{любые м\'{а}лые}
\en{volumes}\ru{объёмы}
\en{have}\ru{имеют}
\en{the~same properties as}\ru{те~же свойства, как~и}
\en{the finite}\ru{конечные}
\en{volumes}\ru{объёмы}.

\en{For a~composite}\ru{Для композита}
\en{it’s}\ru{это}
\en{different}\ru{иначе},
\en{there we have}\ru{там у~нас есть}
\en{the three scopes}\ru{три диапазона}
\en{of~length}\ru{длины}:
${\elcursive \gg \hspace{-0.1ex} \elcursive\smash{\raisemath{.22em}{\hspace{.25ex}'}}\hspace{-0.45ex} \gg \hspace{-0.1ex} \elcursivesub{0}}$.
\en{The largest}\ru{Наибольший}~$\elcursive$
\en{presents}\ru{представляет}
\en{the macroscopic}\ru{макроскопические}
\en{dimensions of a~body}\ru{размеры тела}.
\en{The smallest}\ru{Наименьший}~$\elcursivesub{0}$
\en{is near}\ru{близок}
\en{to the~size}\ru{к~размеру}
\en{of~elements}\ru{элементов}
\en{of the material structure}\ru{структуры материала},
\en{for example}\ru{например}
\en{particles}\ru{частиц}
\en{of a~filler powder}\ru{порошка\hbox{-}наполнителя}).
\en{The intermediate}\ru{Промежуточный}
\en{scope}\ru{диапазон}
${\elcursive\smash{\raisemath{.22em}{\hspace{.25ex}'}}\hspace{-0.5ex}}$
\en{shows}\ru{показывает}
\en{the~size}\ru{размер}
\en{of a~so-called}\ru{так-называемого}
\href{https://en.wikipedia.org/wiki/Representative_elementary_volume}{\inquotes{\en{representative}\ru{представительного}}
\en{volume}\ru{объёма},
\en{of }\inquotes{\en{the unit cell}\ru{единичной ячейки}}}:
\en{that small enough}\ru{того довольно м\'{а}лого}
\en{volume}\ru{объёма},
\en{where}\ru{где}
\en{we can see}\ru{мы можем увидеть}
\en{the specific properties}\ru{специфические свойства}
\en{of this composite}\ru{этого композита}.

\en{In}\ru{В}~\en{composites}\ru{композитах}\en{,}
\en{the~complex problem}\ru{сложная проблема}
\en{with an~inhomogeneous body}\ru{с~неоднородным телом}
\en{is split into the two}\ru{расщепляется на~две}:
\en{for a~body as a~whole}\ru{для тела как целого}
(\en{the macrolevel}\ru{макроуровень})
\en{and for}\ru{и~для}
\en{a~}\inquotes{\en{representative}\ru{представительного}}
\en{volume}\ru{объёма}
(\en{the microlevel}\ru{микроуровень}).
%
\en{At the macrolevel}\ru{На макроуровне}~(\en{scope}\ru{диапазон}~${\elcursive\hspace{.25ex}}$)\en{,}
\en{a~composite}\ru{композит}
\en{is modeled as}\ru{моделируется как}
\en{a~homogeneous medium}\ru{однородная среда}
\en{with }\ru{с}
\inquotes{\en{the~\textcolor{magenta}{effective} properties}\ru{\textcolor{magenta}{эффективными} свойствами}},
\en{and}\ru{и}~\en{the~}\inquotes{\en{representative}\ru{представительные}}
\en{volumes}\ru{объёмы}
\en{play}\ru{играют}
\en{there}\ru{там}
\en{the role of particles}\ru{роль частиц}.
%
\en{At the microlevel}\ru{На микроуровне}~(\en{scope}\ru{диапазон}~$\elcursivesub{0}$)
\en{fields are very inhomogeneous}\ru{пол\'{я} очень неоднородны},
\en{but}\ru{но}
\en{some}\ru{некое}
\en{averaging}\ru{осреднение}
\en{by}\ru{по}
\en{the }\inquotes{\en{representative}\ru{представительному}}
\en{volume}\ru{объёму}
\en{leads}\ru{ведёт}
\en{to the macrolevel}\ru{на макроуровень}.
%
\en{The complex problem}\ru{Сложная задача}
\en{for a~composite}\ru{для композита}
\en{as an~inhomogeneous body}\ru{как неоднородного тела}
\en{splits}\ru{расщепляется}
\en{into the two}\ru{на две}:
\en{for a~representative volume}\ru{для представительного объёма}~(\en{the microlevel}\ru{микроуровень})
\en{and}\ru{и}
\en{for a~body as a~whole}\ru{для тела в~целом}~(\en{the macrolevel}\ru{макроуровень}).

% mesoscopic, on a scale between microscopic and macroscopic
% мезоскопический

\en{These}\ru{Эти}
\en{thoughts}\ru{мысли},
\en{however}\ru{однако},
\en{are not perfectly convincing}\ru{не идеально убедительны}.
\en{To convince more}\ru{Чтобы убедить больше},
\en{we can use}\ru{мы можем использовать}
\inquotes{\en{the random field theory}\ru{теорию случайных полей}}

...

\en{The mechanics of composites}\ru{Механика композитов}
\en{appeared}\ru{появилась}
\en{not very long ago}\ru{не~очень давно},
\en{but yet}\ru{но нынче}
\en{it is developing}\ru{она развивается}
\en{pretty fast}\ru{весьма быстро}.
%
\en{Due to the high fracture toughness}\ru{Из-за высокой трещиностойкости},
\en{the fracture mechanics}\ru{механика \textcolor{magenta}{разрушения}}
\en{of~composites}\ru{композитов}
\en{is very popular}\ru{очень популярна}.

\en{\section{Effective fields}}

\ru{\section{Эффективные поля}}

\label{para:composites.effectivefields}

\en{Any field}\ru{Любое поле} \en{in a~composite}\ru{в~композите} \en{is usually represented}\ru{обычно представляется} \en{by the sum}\ru{суммой}
${u = u* + u'}$,
\en{where}\ru{где} ${u*}$\en{ is}\ru{\:---} \en{some}\ru{некоторое} \en{smoothed}\ru{сглаженное} \inquotes{\en{effective}\ru{эффективное}} \en{field}\ru{поле}, \en{and}\ru{а}~${u'}$\en{ is}\ru{\:---} \en{fast oscillating fluctuation}\ru{быстро осциллирующая флюктуация}.
\en{It’s often assumed}\ru{Часто предполагается}

\nopagebreak\vspace{-0.2em}\begin{equation}
\label{composites.averaging}
u* (A) = <u>
\equiv \mathcal{V}^{-1}
\scalebox{0.85}{$ \displaystyle\integral\displaylimits_{\mathcal{V}} $} \hspace{-0.4ex}
u d\mathcal{V}
\hspace{.1ex} ,
\end{equation}

\vspace{-0.2em}\noindent
\en{where}\ru{где}~${<u>}$ \en{is}\ru{есть} \en{an~average}\ru{среднее} \en{within}\ru{внутри} \en{a~representative volume}\ru{представительного объёма} \en{centered}\ru{с~центром} \en{at point}\ru{в~точке}~$A$.
\en{Averaging}\ru{Осреднение}~\eqref{composites.averaging}

...

%%\begin{otherlanguage}{russian}
%%\end{otherlanguage}

\en{\section{Boundary value problems for representative volume}}

\ru{\section{Краевые задачи для представительного объёма}}

\label{para:composites.boundaryvalueproblems}

\en{How}\ru{Как} \en{the elastic moduli}\ru{упругие модули} \en{are determined}\ru{определяются} \en{for a~homogeneous medium}\ru{для~однородной среды}?
\en{Without the~real possibility}\ru{Без реальной возможности} \en{to get}\ru{получить} \en{a~relation}\ru{отношение} \en{between}\ru{между} ${\hspace{-0.1ex}\linearstress\hspace{.1ex}}$ \en{and}\ru{и}~$\infinitesimaldeformation$ \en{for a~point of infinitesimal size}\ru{для точки бесконечно-малого размера}, \en{experiments are carried out}\ru{эксперименты проводятся} \en{with bodies of~finite volumes}\ru{с~телами конечных объёмов} \en{under}\ru{при} \en{a~so\hbox{-}called}\ru{так называемом} \inquotesx{\en{homogeneous stress}\ru{однородном напряжении}}[---] \en{when}\ru{когда} \en{stress}\ru{напряжение} \en{is the~same}\ru{одно и~то~же} \en{at all points of a~body}\ru{во~всех точках тела}.
\en{In~composites}\ru{В~композитах}\en{,} \en{the~role of a~point}\ru{роль точки} \en{plays}\ru{играет} \en{a~}\inquotes{\en{representative}\ru{представительный}} \en{volume}\ru{объём}.

...

\en{\section{Hill’s fork}}

\ru{\section{Вилка Hill’а}}

\label{para:composites.hillsfork}

\begin{otherlanguage}{russian}

{\small
\noindent
Using Voigt and Reuss theories, Hill derived upper and lower bounds on the effective properties of a composite material [\bookauthor{Hill,}[R.][W.] The~elastic behaviour of a~crystalline aggregate. \emph{Proceedings of the~Physical Society}, Section~A, Volume~65, Issue~5 (May 1952). Pages 349\hbox{--}354.]

The scale separation is motivated by the material properties, at both scales continuum mechanics models the underlying system.
Such an approach uses energy equivalence at both scales as proposed in Hill (1972).

Hill R (1972) On constitutive macro-variables for heterogeneous solids at finite strain (pages 131–147)

For a composite material, at least two different materials with known material models and parameters, generate a homogenized material modeled with a predetermined constitutive equation. Determination of material parameters of the homogenized material is a challenging task.
\par}

Отметив, что

...



\end{otherlanguage}

\en{\section{Eshelby formulas}}

\ru{\section{Формулы Eshelby}}

\label{para:composites.eshelbyformulas}

\begin{otherlanguage}{russian}

Итак, эффективные модули определяются энергией представительного объёма в~первой или~второй задачах:

...



\end{otherlanguage}

\ru{\section{Эффективные модули для материала со сферическими включениями}}

\en{\section{Effective moduli for material with spherical inclusions}}

\label{para:composites.materialwithsphericalinclusions}

\begin{otherlanguage}{russian}

В~однородной матрице случайным образом, но достаточно равномерно, распределены сферические включения радиусом~$a$. Получившийся композит на~макроуровне будет изотропным, его упругие свойства полностью определяются ...

...



\end{otherlanguage}

\en{\section{Self-consistent method}}

\ru{\section{Метод самосогласования}}

\label{para:composites.selfconsistentmethod}

\begin{otherlanguage}{russian}

Выше мы опирались на~две задачи \en{for}\ru{для} \en{a~representative volume}\ru{представительного объёма} и~определяли \en{effective modules}\ru{эффективные модули} \en{from}\ru{из} \en{the equality of~energies}\ru{равенства энергий}.
В~основе метода самосогласования лежит новая идея: представительный объём помещается в~безграничную среду с~эффективными свойствами, на~бесконечности состояние считается однородным, эффективные модули находятся из~некоторых дополнительных условий самосогласования.

Обратимся снова к~вопросу об~объёмном модуле среды со~сферическими включениями.
Задача сферически симметрична; для~включения по\hbox{-}прежнему

...



\end{otherlanguage}

\en{\section{Hashin\hbox{--}Shtrikman principle}}

\ru{\section{Принцип Хашина\hbox{--}Штрикмана}}

\label{para:composites.hashin-shtrikman-principle}

{\small
\noindent
Hashin \en{and}\ru{и} Shtrikman derived upper and lower bounds for the effective elastic properties of quasi-isotropic and quasi-homogeneous multiphase materials using a~variational approach [\bookauthor{Hashin,}[Z.]; \bookauthor{Shtrikman,}[S.] A~variational approach to the~theory of the~elastic behaviour of multiphase materials. // Journal of the Mechanics and Physics of Solids. Volume~11, Issue~2 (March--April 1963). Pages 127\hbox{--}140.]

Hashin Z., Shtrikman S. (1962) On some variational principles in anisotropic and nonhomogeneous elasticity. Journal of the Mechanics and Physics of Solids 10(4): pages 335–342
\par}

\begin{otherlanguage}{russian}

Вилка Hill’а основана на~обычных экстремальных принципах теории упругости.
Специально для~механики композитов Hashin и~Shtrikman построили очень своеобразный функционал, который на~некотором точном решении может иметь как~максимум, так~и~минимум, давая возможность с~двух сторон оценивать эффективные модули~\cite{shermergor}.

Рассмотрим первую из~двух задач для представительного объёма

...



\end{otherlanguage}

\section*{\small \wordforbibliography}

\begin{changemargin}{\parindent}{0pt}
\fontsize{10}{12}\selectfont

\begin{otherlanguage}{russian}

Книги R.\:Christensen’а~\cite{christensen-compositematerials} \en{and}\ru{и} Б.\,Е.\:Победри~\cite{pobedrya-composites} содержат и~основы механики композитов, и~постановку не~теряющих актуальности проблем.
Для~с\'{а}мого требовательного читателя представляет интерес монография Т.\,Д.\:Шермергора~\cite{shermergor}.
Немало книг посвящено механике разрушения композитов, здесь ст\'{о}ит отметить труд Г.\,П.\:Черепанова~\cite{cherepanov-compositematerialfracture}.

\end{otherlanguage}

\end{changemargin}
