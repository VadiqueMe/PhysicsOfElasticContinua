%%\begin{center}

\begin{wrapfigure}{o}{0.4\textwidth}
\makebox[0.45\textwidth][c]{\begin{minipage}[t]{0.5\textwidth}
\vspace{-1em}
\scalebox{0.95}{

\def\cameraangle{166}
\tdplotsetmaincoords{44}{\cameraangle} % orientation of camera

\def\rodheight{10}
\def\rodradius{0.2}

\pgfmathsetmacro{\beginangle}{\cameraangle}
\pgfmathsetmacro{\endangle}{\cameraangle - 180}

\tikzset{pics/rod/.style={code={

	% draw rod

	\draw [line width=1pt, color=black, fill=yellow!50!white, opacity=.9]
		plot [domain=\beginangle:\endangle]
			( {\rodradius*cos(\x)}, {\rodradius*sin(\x)}, 0 )
		-- plot [domain=\endangle:\beginangle]
			( {\rodradius*cos(\x)}, {\rodradius*sin(\x)}, \rodheight )
		-- cycle ;

	\draw [line width=1pt, color=black, fill=yellow!50!white, opacity=.9, domain=0:360]
		plot ( {-\rodradius*cos(\x)}, {-\rodradius*sin(\x)}, \rodheight ) ;

}}}

\tikzset{pics/dottedrod/.style={code={

	% draw rod dotted

	\draw [line width=1pt, color=black, line cap=round, dash pattern=on 0pt off 1.6\pgflinewidth]
		plot [domain=\beginangle:\endangle]
			( {\rodradius*cos(\x)}, {\rodradius*sin(\x)}, 0 )
		-- plot [domain=\endangle:\beginangle]
			( {\rodradius*cos(\x)}, {\rodradius*sin(\x)}, \rodheight )
		-- cycle ;

	\draw [line width=1pt, color=black, line cap=round, dash pattern=on 0pt off 1.6\pgflinewidth, opacity=.9, domain=0:360]
		plot ( {-\rodradius*cos(\x)}, {-\rodradius*sin(\x)}, \rodheight ) ;

}}}

\tikzset{pics/rodaxis/.style={code={

	% draw axis
	\draw [line width=0.5pt, blue, line cap=round, dash pattern=on 12pt off 2pt on \the\pgflinewidth off 2pt]
		( 0, 0, -0.4pt ) -- ( 0, 0, \rodheight + 0.4pt ) ;

}}}

\tikzset{pics/externalforce/.style={code={

	% draw force
	\def\forcelength{1.2}

	\draw [line width=1.5pt, red, line cap=round, -{Triangle[round, length=3.6mm, width=2.4mm]}]
		( 0, 0, \rodheight + \forcelength ) -- ( 0, 0, \rodheight )
		node [pos=0, below left, inner sep=0, outer sep=3.2pt]
			{\scalebox{1.2}[1.2]{${\bm{F}}$}} ;

}}}

\begin{tikzpicture}[scale=1, tdplot_main_coords] % use 3dplot

	\coordinate (O) at ( 0, 0, 0 ) ;
	\coordinate (rodTopCenter) at ($ (O) + ( 0, 0, \rodheight ) $) ;

	% draw circle
	\def\circleradius{0.8}
	\def\heightofhatch{0.5}

	\pgfmathsetmacro{\stepangleforcircle}{\beginangle - 10}
	\foreach \angle in { \beginangle, \stepangleforcircle, ..., \endangle }
		\draw [line width=0.4pt, color=black]
			( \angle:\circleradius ) -- ($ ( \angle:\circleradius ) - ( 0, 0, \heightofhatch ) $) ;

	\draw [line width=1pt, color=black, fill=white] (O) circle ( \circleradius ) ;

	% draw rod, axis and force
	\pic (initial) {rod} ;
	\pic (initial) {rodaxis} ;
	\pic (initial) {externalforce} ;

	% draw deformed rod
	\scoped {
		\pgfsetcurvilinearbeziercurve
			{\pgfpointxyz{0}{0}{0}}
			{\pgfpointxyz{0}{0}{0.5cm}}
			{\pgfpointxyz{0.25cm}{0}{1cm}}
			{\pgfpointxyz{1.25cm}{0}{1.25cm}}
		\pgftransformnonlinear{\pgfgetlastxy\x\y\pgfpointcurvilinearbezierorthogonal{\y}{\x}}
			\pic (deformed) {dottedrod} ;
			\pic (deformed) {rodaxis} ;
	}

\end{tikzpicture}
}
\vspace{-0.2em}\caption{}\label{fig:deformedrod}
\end{minipage}}
\end{wrapfigure}

%%\end{center}
