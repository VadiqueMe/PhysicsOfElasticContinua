\section*{\small \wordforbibliography}

\begin{changemargin}{\parindent}{0pt}
\fontsize{10}{12}\selectfont

\begin{otherlanguage}{russian}

Глубина изложения нелинейной безмоментной упругости характерна для книг А.\,И.\:Лурье~\cite{lurie-nonlinearelasticity, lurie-theoryofelasticity}.
%
Оригинальность как~основных идей, так~и~стиля присуща книге Clifford’а Truesdell’а~\cite{truesdell-firstcourse}.
%
Монография Юрия Работнова~\cite{rabotnov-mechanicsofdeformable}, где напряжения представлены как множители Лагранжа, очень интересна и~своеобразна.
%
Много ценной информации можно найти у~К.\,Ф.\:Черн\'{ы}х~\cite{chernyh-nonlinearelasticity}.
%
\en{The}\ru{Книгу}
\russianlanguage{Л.\,М.\:Зубов}\ru{а}\en{’s book}~\cite{zubov}
\en{is worthy of~mention too}\ru{тоже ст\'{о}ит упомянуть}.
%
О~применении нелинейной теории упругости в~смежных областях рассказано в~книге Cristian’а Teodosiu~\cite{teodosiu-crystaldefects}.
%
Повышенным математическим уровнем отличается монография Philippe’а Ciarlet~\cite{ciarlet-mathematicalelasticity}.
\par

\end{otherlanguage}

\end{changemargin}
