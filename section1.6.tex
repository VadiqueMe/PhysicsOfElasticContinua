
\en{\section{The cross product}}

\ru{\section{Векторное произведение}}

\label{para:crossproduct}

%%en{the~}\hbox{\hspace{-0.2ex}\inquotes{${\hspace{-0.25ex}\times\hspace{-0.1ex}}$}\hspace{-0.2ex}-\en{product}\ru{произведение}}

\en{By common notions}\ru{По~привычным представлениям},
\en{the~}\hbox{\hspace{-0.2ex}\inquotes{${\hspace{-0.25ex}\times\hspace{-0.1ex}}$}\hspace{-0.2ex}-\en{product}\ru{произведение}}
(\en{the }\inquotesx{cross product},
\en{the }\inquotesx{\en{vector product}\ru{векторное произведение}},
\en{sometimes}\ru{иногда}
\en{the }\inquotes{oriented area product})
\en{of the two vectors}\ru{двух векторов}
\en{is the vector}\ru{есть вектор},
\en{heading perpendicular}\ru{направленный перпендикулярно}
\en{to the~plane of~multipliers}\ru{плоскости сомножителей},
\en{whose length}\ru{длина которого}
\en{is equal to}\ru{равна}
\en{the~area}\ru{пл\'{о}\-ща\-ди}
\en{of the parallelogram}\ru{параллелограмма},
\en{spanned by}\ru{охватываемого}
\en{the multipliers}\ru{сомножителями}

\nopagebreak\vspace{-0.2em}\begin{equation*}
\| \hspace{.33ex} \bm{a} \times \bm{b} \hspace{.4ex} \| \hspace{.1ex}
= \hspace{.1ex} \| \bm{a} \| \hspace{.1em} \| \bm{b} \| \hspace{.1em} \operatorname{sin} \measuredangle (\bm{a}, \bm{b})
\hspace{.1ex} .
\end{equation*}

\vspace{-0.2em}\noindent
\en{However}\ru{Однако},
\en{a~}\hbox{\hspace{-0.2ex}\inquotes{${\hspace{-0.25ex}\times\hspace{-0.1ex}}$}\hspace{-0.2ex}-\en{product}\ru{произведение}}\en{ isn’t quite}\ru{\:--- не~вполн\'{е}}
\en{a~vector}\ru{вектор},
\en{since}\ru{поскольку}
\en{it is not completely invariant}\ru{оно не~полностью инвариантно}.

\vspace{.4em}
\hspace*{-\parindent}\begin{minipage}{\linewidth}
\setlength{\parindent}{\horizontalindent}
\setlength{\parskip}{\spacebetweenparagraphs}

\begin{wrapfigure}[8]{R}{.32\textwidth}
\makebox[.36\textwidth][c]{\hspace{.5em} \begin{minipage}[t]{.36\textwidth}
\vspace{-1.5em}
\hspace{1.25em}\scalebox{.9}{%

\begin{tikzpicture}[scale=.8]

   \draw [line width=1.6pt, black, -{Stealth[round, length=4.5mm, width=3mm]}]
      (0, 0) -- (2.4, -0.8)
      node[ above=1.7mm, xshift=-1.8mm ] {$\bm{a}$} ;
      \draw [ line width=1.6pt, black, -{Stealth[round, length=4.5mm, width=3mm]} ]
	(0, 0) -- (1.6, 0.8)
	node[midway, above=0.8mm] {$\bm{b}$};

\draw [line width=1.77pt, blue, -{Stealth[round, length=4.5mm, width=3mm]}]
	(0, 0) -- (0, -1.93);
\draw [line width=1.77pt, blue, -{Stealth[round, length=4.5mm, width=3mm]}]
	(0, 0) -- (0, -2.2)
	node[pos=0.8, right, inner sep=1pt, outer sep=5pt] {$\bm{c}$};

\draw [line width=1.77pt, blue, -{Stealth[round, length=4.5mm, width=3mm]}]
	(0, 0) -- (0, 1.93);
\draw [line width=1.77pt, blue, -{Stealth[round, length=4.5mm, width=3mm]}]
	(0, 0) -- (0, 2.2)
	node[pos=.82, right, inner sep=1pt, outer sep=5pt] {$\bm{c}$};

\draw [line width=0.5pt, black!50] (2.4, -0.8) -- (4, 0);
\draw [line width=0.5pt, black!50] (1.6, 0.8) -- (4, 0);

\end{tikzpicture}
}\vspace{-1.6em}\caption{}\label{fig:crossproduct}
\end{minipage}}
\end{wrapfigure}

\en{The multipliers}\ru{Сомножители}
\en{of the~}\hbox{ \hspace{-0.2ex}\inquotes{$ {
   \hspace{-0.25ex} \times \hspace{-0.1ex}
} $}\hspace{-0.2ex}-\en{product}\ru{произведения}}
${\bm{c} = \bm{a} \times \bm{b}}$
\en{determine}\ru{определяют}
\en{the~result’s direction}\ru{направление результата}
\en{in space}\ru{в~пространстве},
\en{with an~accuracy}\ru{с~точностью}
\en{up to the sign}\ru{до знака}
\figureref{fig:crossproduct}.

\en{Once you pick}\ru{Как только ты выбираешь}
\en{as}\ru{как}
\en{the positive}\ru{положительное}
\en{the }\inquotes{ \en{right-chiral}\ru{правостороннюю} }
\en{(\inquotes{right-handed})}\ru{}
\en{or}\ru{или}
\en{the }\inquotes{ \en{left-chiral}\ru{левостороннюю} }
\en{(\inquotes{left-handed})}
\en{orientation}\ru{ориентацию}
\en{of space}\ru{пространства},
\en{the one direction from the possible two}\ru{одно направление из двух возможных},
\en{then}\ru{тогда}
\en{the results}\ru{результаты}
\en{of the }\hbox{ \hspace{-0.2ex}\inquotes{${ \hspace{-0.25ex} \times \hspace{-0.1ex} }$ }\hspace{-0.2ex}-\en{products}\ru{произведений}}
\en{become}\ru{становятся}
\en{completely determined}\ru{полностью определёнными}.

\inquotes{\en{The chiral}\ru{Хиральный}}
\en{means}\ru{значит}
\en{asymmetric}\ru{ассиметричный}
in such a~way that.
\en{the thing}\ru{вещица}
\en{and}\ru{и}
\en{its mirror image}\ru{её зеркальное отражение}
\en{are not superimposable}\ru{не совмещаются},
a~picture cannot be superposed
on its mirror image
by any combination
of rotations and translations.

\en{An object}\ru{Объект}
\en{is chiral}\ru{хирален}\ru{,}
\en{if}\ru{если}
\en{it is distinguishable}\ru{он отличим}
\en{from}\ru{от}
\en{its mirror image}\ru{своего отражения}.

\end{minipage}
\vspace{.2em}

\en{Vectors}\ru{Векторы}
\en{are usually measured}\ru{обычно измеряются}\ru{,}
\en{via some a~basis}\ru{используя какой-нибудь базис}~$\bm{e}_i$.
\en{They}\ru{Они} \en{are decomposed}\ru{раскладываются} \en{into linear combinations}\ru{на линейные комбинации} \en{like}\ru{вида} ${\bm{a} = a_i \bm{e}_i}$.
\en{So}\ru{Так что} \en{the orientation}\ru{ориентация} \en{of space}\ru{пространства} \en{is equivalent}\ru{эквивалентна} \en{to the orientation}\ru{ориентации} \en{of the~sequential triple of~basis vectors}\ru{последовательной тройки базисных векторов} $\bm{e}_1$, $\bm{e}_2$, $\bm{e}_3$.
\en{It}\ru{Это} \en{means}\ru{означает}\ru{,} \en{that}\ru{что} \en{the~sequence of~basis vectors}\ru{последовательность базисных векторов} \en{becomes}\ru{становится} \en{significant}\ru{значимой} (\en{for linear combinations}\ru{для линейных комбинаций}\en{,} \en{the sequence of addends}\ru{последовательность слагаемых} \en{doesn’t affect anything}\ru{ни на что не~влияет}).

\en{If}\ru{Если} \en{two bases}\ru{два базиса} \en{consist of different sequences}\ru{состоят из разных последовательностей} \en{of the~same vectors}\ru{одних и~тех~же векторов} \en{within an oriented space}\ru{в~ориентированном пространстве}, \en{then}\ru{то} \en{their orientations}\ru{их ориентации} \en{differ}\ru{отличаются} \en{by some permutation}\ru{некоторой перестановкой}.

\en{The~orientation}\ru{Ориентация}
\en{of~the~space}\ru{пространства}
\en{is}\ru{есть}
\en{a }(\en{kind of}\ru{нечто вроде})
\en{asymmetry}\ru{асимметрии}.
\en{This asymmetry}\ru{Эта асимметрия}
\en{makes it impossible}\ru{делает невозможным}
\en{to replicate a~reflection}\ru{повторение отражения}
\en{by the means of}\ru{посредством}
\en{any rotations}\ru{любых вращений}%
\footnote{
   \en{Applying only rotations}\ru{Применяя лишь повороты},
   \en{it’s impossible}\ru{невозможно}
   \en{to replace}\ru{заменить}
   \en{the left hand}\ru{левую руку}
   \en{of a~human figure}\ru{фигуры человека}
   \en{into the right hand}\ru{на правую руку}.
   \en{But}\ru{Но}
   \en{it is possible}\ru{это возможно}
   \en{by reflection}\ru{отражением}
   \en{of~a~figure}\ru{фигуру}
   \en{in a~mirror}\ru{в~зеркале}.
}

%%\textcolor{blue}{A~nonsingular linear mapping is orientation-preserving if the determinant of its matrix is positive.}

\begin{figure}[htb!]
\begin{center}

\tdplotsetmaincoords{46}{140} % orientation of camera

\begin{tikzpicture}[scale=2.5, tdplot_main_coords]

\pgfmathsetmacro{\spiralradius}{.69}
\pgfmathsetmacro{\verticaladvance}{.5}

\pgfmathsetmacro{\unitvectorlength}{1.2}

\begin{scope}[xshift=-2.8em]

\coordinate (O) at (0, 0, 0);

% basis vectors
\draw [line width=1.25pt, blue, -{Latex[round, length=3.6mm, width=2.4mm]}]
	(O) -- (\unitvectorlength, 0, 0)
	node[pos=0.9, above, xshift=-0.8em] {${\bm{e}}_2$};
\draw [line width=1.25pt, blue, -{Latex[round, length=3.6mm, width=2.4mm]}]
	(O) -- (0, \unitvectorlength, 0)
	node[pos=0.9, above, xshift=0.2em, yshift=0.1em] {${\bm{e}}_1$};
\draw [line width=1.25pt, blue, -{Latex[round, length=3.6mm, width=2.4mm]}]
	(O) -- (0, 0, \unitvectorlength)
	node[pos=0.9, left, yshift=0.5em] {${\bm{e}}_3$};

\pgfmathsetmacro\verticalshift{-0.1}
\pgfmathsetmacro\initialangle{-35}
\def\lengthincircles{2.15}
\pgfmathsetmacro\endangle{\lengthincircles * 360 + \initialangle}
\def\howmanysamples{120}

\draw[ line width=.8pt ]
	plot [ domain=\initialangle:\endangle, variable=\t, samples=\howmanysamples ]
	( {\spiralradius*sin(\t)}, {\spiralradius*cos(\t)}, {\verticaladvance*\t/360 + \verticalshift} )
	[ arrow inside = {}{0.1, 1} ] ;

\end{scope}

\begin{scope}[xshift=2.8em]

\coordinate (O) at (0, 0, 0);

% basis vectors
\draw [line width=1.25pt, blue, -{Latex[round, length=3.6mm, width=2.4mm]}]
	(O) -- (\unitvectorlength, 0, 0)
	node[pos=0.9, above, xshift=-0.8em] {${\bm{e}}_1$};
\draw [line width=1.25pt, blue, -{Latex[round, length=3.6mm, width=2.4mm]}]
	(O) -- (0, \unitvectorlength, 0)
	node[pos=0.9, above, xshift=0.2em, yshift=0.1em] {${\bm{e}}_2$};
\draw [line width=1.25pt, blue, -{Latex[round, length=3.6mm, width=2.4mm]}]
	(O) -- (0, 0, \unitvectorlength)
	node[pos=0.9, left, yshift=0.5em] {${\bm{e}}_3$};

\pgfmathsetmacro\verticalshift{.02}
\pgfmathsetmacro\initialangle{-125}
\def\lengthincircles{2.15}
\pgfmathsetmacro\endangle{\lengthincircles * 360 + \initialangle}
\def\howmanysamples{120}

\draw[ line width=.8pt ]
	plot [ domain=\initialangle:\endangle, variable=\t, samples=\howmanysamples ]
	( {-\spiralradius*sin(\t)}, {\spiralradius*cos(\t)}, {\verticaladvance*\t/360 + \verticalshift} )
	[ arrow inside = {}{0.1, 1} ] ;

\end{scope}

\end{tikzpicture}

\end{center}
\vspace{-1.2em}\caption{}\label{fig:leftrightspirals}
\vspace{-1.1em}
\end{figure}

%\en{an axial}\ru{аксиальный} \en{vector}\ru{вектор}

\en{A~pseudo\-vector}\ru{Псевдо\-вектор}
\en{is}\ru{это}
\en{a~vector\hbox{-}like}\ru{похожий на~вектор}
\en{object}\ru{объект},
\en{invariant under any rotation}\ru{инвариантный при любом повороте}.
\footnote{\ru{Повороты}\en{Rotations}
\en{cannot change}\ru{не~могут поменять}
\en{the orientation}\ru{ориентацию}
\en{of a~triple of~basis vectors}\ru{тройки векторов базиса},
\en{only}\ru{только}
\en{a~reflection}\ru{отражение}
\en{can}\ru{может}.
}\hbox{\hspace{-.5ex}.}

\textcolor{magenta}{... put the figure here ...}






\en{Except in the rare cases}\ru{Кроме редких случаев},
\en{the direction}\ru{направление}
\en{of a~fully invariant (polar) vector}\ru{полностью инвариантного (полярного) вектора}
\en{will change}\ru{поменяется}
\en{with a~reflection}\ru{с~отражением}.

\en{A~pseudovector}\ru{Псевдовектор}
(\en{an axial vector}\ru{аксиальный вектор}),
\en{unlike}\ru{в~отличие от}
\en{a~}\en{polar}\ru{полярного} \en{vector}\ru{вектора},
\en{doesn’t change}\ru{не~меняет}
\ru{ортогональную плоскости отражения компоненту}\en{the component orthogonal to the plane of~reflection},
\en{and}\ru{и}
\en{turns out to be flipped}\ru{оказывается перевёрнутым}
\en{relatively to}\ru{относительно}
\en{the polar vectors}\ru{полярных векторов} \en{and}\ru{и} \en{the geometry}\ru{геометрии} \en{of the entire space}\ru{всего пространства}.
\en{This happens}\ru{Это случается} \en{because}\ru{из\hbox{-}за того, что} \en{the sign}\ru{знак} (\en{and}\ru{и}, \en{accordingly}\ru{соответственно}, \en{the direction}\ru{направление}) \en{of each axial vector}\ru{каждого аксиального вектора} \en{changes}\ru{меняется} \en{along with changing}\ru{вместе c~изменением} \en{the sign}\ru{знака} \en{of the~}\hbox{\hspace{-0.2ex}\inquotes{${\hspace{-0.25ex}\times\hspace{-0.1ex}}$}\hspace{-0.2ex}-\en{product}\ru{произведения}}\:--- \en{which corresponds}\ru{что соответствует} \en{to reflection}\ru{отражению}.

%%таким образом, что
%%in such a way that

\en{The~otherness}\ru{Инаковость} \en{of~pseudovectors}\ru{псевдовекторов} \en{narrows}\ru{сужает} \en{the~variety}\ru{разнообразие} \en{of~formulas}\ru{формул}:
\en{a~pseudovector is not additive with a~vector}\ru{псевдовектор не~складывается с~вектором}.
\en{Formula}\ru{Формула}
${\bm{v} = \bm{v}_{\raisemath{-0.1em}{0}} + \hspace{.15ex} \bm{\omega} \hspace{-0.15ex}\times\hspace{-0.15ex} \locationvector}$
%\en{for an~absolutely rigid undeformable body’s kinematics}\ru{для кинематики абсолютно жёсткого недеформируемого тела}
\en{is correct}\ru{корректна}, \en{because}\ru{поскольку}
$\bm{\omega}$ \en{is pseudovector there}\ru{там\:--- псевдовектор},
\en{and}\ru{и} \en{with the~cross product}\ru{с~векторным произведением} \en{two}\ru{два} \inquotes{\en{pseudo}\ru{псевдо}} \en{give}\ru{дают} ${(-1)^2 = 1}$ (\inquotes{\en{mutually compensate each other}\ru{взаимно компенсируют друг друга}}).

\en{The tensor}\ru{Тензор}
\en{of the permutations parity}\ru{чётности перестановок},
\en{the volumetric}\ru{объёмометрический}
\en{tensor of third complexity}\ru{тензор третьей сложности}

\nopagebreak\vspace{-0.15em}\begin{equation}\label{permutationsparityintro}
\permutationsparitytensor = \permutationsparitysymbols{i\hspace{-0.1ex}j\hspace{-0.1ex}k} \hspace{.2ex} \bm{e}_i \bm{e}_j \bm{e}_k
\hspace{.1ex}, \:\:
\permutationsparitysymbols{i\hspace{-0.1ex}j\hspace{-0.1ex}k} \hspace{-0.12ex} \equiv \hspace{.1ex} \bm{e}_i \hspace{-0.2ex} \times \hspace{-0.2ex} \bm{e}_j \hspace{-0.1ex} \dotp \hspace{.1ex} \bm{e}_k
\end{equation}

\nopagebreak\vspace{-0.1em}\noindent
\en{with components}\ru{с~компонентами}~${\permutationsparitysymbols{i\hspace{-0.1ex}j\hspace{-0.1ex}k}}$\ru{,} \en{equal to}\ru{равными} \inquotes{\en{triple}\ru{трой\-ным}}~(\inquotesx{\en{mixed}\ru{смешан\-ным}}[,] \inquotes{\en{cross\hbox{-}dot}\ru{векторно\hbox{-}скалярным}}) \en{products}\ru{произведениям} \en{of~basis vectors}\ru{векторов базиса}.

\en{The~absolute value}\ru{Абсолютная величина}~(\en{modulus}\ru{модуль}) \en{of~each nonzero component}\ru{всякой ненулевой компоненты} \en{of~tensor}\ru{тензора}~$\permutationsparitytensor$
\en{is equal to}\ru{\hbox{равна}} \en{the~volume}\ru{объёму}~${\hspace{-0.25ex}\sqrt{\hspace{-0.33ex}\mathstrut{\textsl{g}}}}$ \en{of~a~parallelopiped}\ru{параллелепипеда}, \en{drew upon a~basis}\ru{натянутого на \hbox{базис}}.
\en{For basis}\ru{Для базиса}~${\bm{e}_i}$ \en{of pairwise perpendicular one\hbox{-}unit long vectors}\ru{из попарно перпендикулярных векторов единичной длины}
${\hspace{-0.25ex}\sqrt{\hspace{-0.33ex}\mathstrut{\textsl{g}\hspace{.12ex}}} = \hspace{-0.1ex} 1}$.

\en{Tensor}\ru{Тензор}~$\permutationsparitytensor$ \en{is isotropic}\ru{изотропен}, \en{its components}\ru{его компоненты} \en{are constant}\ru{постоянны} en{and}\ru{и} \en{independent}\ru{независимы} \en{of rotations of a~basis}\ru{от поворотов базиса}.
\en{But}\ru{Но}
\en{a~reflection}\ru{отражение}\:---
\en{a~change in}\ru{изменение}
\en{the orientation}\ru{ориентации}
\en{of a~triple of basis vectors}\ru{тройки базисных векторов}
(\en{a~change in}\ru{перемена}
\inquotes{\en{the~direction of a~screw}\ru{направления винта}})\:---
\en{changes the~sign of}\ru{меняет знак}~$\permutationsparitytensor$,
\en{so}\ru{так что}
\en{this is pseudotensor}\ru{это псевдотензор}
(\en{axial tensor}\ru{аксиальный тензор}).

\begin{otherlanguage}{russian}

\en{If}\ru{Если} ${\bm{e}_1 \hspace{-0.2ex} \times \bm{e}_2 \hspace{-0.15ex} = \bm{e}_3}$,
\en{then}\ru{то} ${\bm{e}_i}$ \en{is}\ru{есть} ориентированная положительно тройка, произвольно выбираемая из двух вариантов (\figureref{fig:crossproduct}).
В~таком случае
компоненты~$\permutationsparitytensor$
равны символу чётности перестановок
${\permutationsparitysymbols{i\hspace{-0.1ex}j\hspace{-0.1ex}k} \hspace{-0.1ex} = \hspace{.1ex} e_{i\hspace{-0.1ex}j\hspace{-0.1ex}k}}$.
Когда~же ${\bm{e}_1 \hspace{-0.2ex} \times \bm{e}_2 \hspace{-0.15ex} = - \hspace{.25ex} \bm{e}_3}$, \en{then}\ru{тогда} \en{triple}\ru{тройка}~${\bm{e}_i}$ \en{is oriented negatively}\ru{ориентирована отрицательно} (\en{or}\ru{или} \inquotes{\en{mirrored}\ru{зеркальная}}).
\en{For}\ru{Для} \en{mirrored triples}\ru{зеркальных троек} ${\permutationsparitysymbols{i\hspace{-0.1ex}j\hspace{-0.1ex}k} \hspace{-0.1ex} = - \hspace{.2ex} e_{i\hspace{-0.1ex}j\hspace{-0.1ex}k}}$
(\en{and}\ru{а}~${e_{i\hspace{-0.1ex}j\hspace{-0.1ex}k} \hspace{-0.12ex} = - \hspace{.33ex} \bm{e}_i \hspace{-0.2ex} \times \hspace{-0.2ex} \bm{e}_j \hspace{-0.1ex} \dotp \hspace{.12ex} \bm{e}_k}$).

\end{otherlanguage}




\en{With}\ru{С}~\en{the Levi\hbox{-}Civita tensor}\ru{тензором Л\'{е}ви\hbox{-\!}Чив\'{и}ты}~$\permutationsparitytensor$ \en{it is possible}\ru{возможно} \en{to take a~fresh look}\ru{по\hbox{-}новому взглянуть} \en{at the cross product}\ru{на~векторное произведение}:

\nopagebreak\vspace{-0.3em}\begin{equation*}
\permutationsparitysymbols{i\hspace{-0.1ex}j\hspace{-0.1ex}k} \hspace{-0.15ex} = \hspace{.15ex}
\hspace{.1ex} \bm{e}_i \hspace{-0.2ex} \times \hspace{-0.2ex} \bm{e}_j \hspace{-0.1ex} \dotp \hspace{.12ex} \bm{e}_k
\:\Leftrightarrow\:
\bm{e}_i \hspace{-0.1ex} \times \bm{e}_j \hspace{-0.12ex}
= \permutationsparitysymbols{i\hspace{-0.1ex}j\hspace{-0.1ex}k} \hspace{.2ex} \bm{e}_k
\hspace{.1ex} ,
\end{equation*}\vspace{-1.6em}
\begin{multline}
\bm{a} \mathcolor{blue}{\times} \bm{b} \hspace{.2ex}
= \hspace{.1ex} a_i \bm{e}_i \times b_j \bm{e}_j
= \hspace{.1ex} a_i \hspace{.1ex} b_j \bm{e}_i \hspace{-0.2ex} \times \hspace{-0.2ex} \bm{e}_j
= \hspace{.1ex} a_i \hspace{.1ex} b_j \hspace{.1ex} \permutationsparitysymbols{i\hspace{-0.1ex}j\hspace{-0.1ex}k} \hspace{.2ex} \bm{e}_k =
\\[-0.12em]
\shoveright{= \hspace{.2ex} b_j \hspace{.1ex} a_i \hspace{.1ex} \bm{e}_j \bm{e}_i \dotdotp \permutationsparitysymbols{mnk} \hspace{.2ex} \bm{e}_m \bm{e}_n \bm{e}_k
= \hspace{.2ex} \bm{b} \hspace{.1ex} \bm{a} \hspace{.1ex} \dotdotp \permutationsparitytensor \hspace{.1ex} ,}
\\[-0.1em]
= \hspace{.1ex} a_i \hspace{.1ex} \permutationsparitysymbols{i\hspace{-0.1ex}j\hspace{-0.1ex}k} \hspace{.2ex} \bm{e}_k \hspace{.1ex} b_j
= - \hspace{.2ex} a_i \hspace{.1ex} \permutationsparitysymbols{ikj} \hspace{.2ex} \bm{e}_k \hspace{.1ex} b_j
= \mathcolor{blue}{-} \hspace{.2ex} \bm{a} \hspace{.1ex} \mathcolor{blue}{\dotp \permutationsparitytensor \dotp} \hspace{.12ex} \bm{b}
\hspace{.1ex} .
\end{multline}

\vspace{-0.1em}\noindent
\en{So that}\ru{Так что}, \en{the~cross product}\ru{векторное произведение} \en{is not}\ru{не~есть} \en{another new}\ru{ещё одно новое}, \en{entirely distinct operation}\ru{полностью отдельное действие}.
\en{With}\ru{С} \en{the}\ru{тензором} Levi\hbox{-}Civita\ru{\hspace{-0.1ex}’ы}\en{ tensor} \en{it reduces}\ru{оно сводится} \en{to the four}\ru{к~четырём} \en{already described}\ru{уже описанным}~(\sectionref{para:operationswithtensors}) \en{and}\ru{и} \en{is applicable}\ru{примен\'{и}мо} \en{to tensors}\ru{к~тензорам} \en{of any complexity}\ru{любой сложности}.

\inquotes{\en{The cross product}\ru{Векторное произведение}} \en{is just}\ru{это всего лишь} \en{the }dot product\:--- \en{the combination}\ru{комбинация} \en{of multiplication and contraction}\ru{умножения и~свёртки} (\sectionref{para:operationswithtensors})\:--- \en{involving}\ru{с~участием} \en{tensor}\ru{тензора}~$\permutationsparitytensor$.
\en{Such combinations}\ru{Такие комбинации} \en{are possible}\ru{возможны} \en{with any tensors}\ru{с~любыми тензорами}:

\nopagebreak\vspace{-0.1em}\begin{equation*}
\begin{array}{c}
\bm{a} \hspace{.1ex} \mathcolor{blue}{\times} {^2\hspace{-0.16em}\bm{B}} = a_i \bm{e}_i \times B_{\hspace{-0.1ex}j\hspace{-0.1ex}k} \bm{e}_j \bm{e}_k \hspace{-0.1ex} = \tikzmark{BeginVectorCrossTensor} a_i B_{\hspace{-0.1ex}j\hspace{-0.1ex}k} \hspace{.1ex} \permutationsparitysymbols{i\hspace{-0.1ex}jn} \tikzmark{EndVectorCrossTensor} \bm{e}_n \bm{e}_k = \mathcolor{blue}{-} \hspace{.2ex} \bm{a} \hspace{.15ex} \mathcolor{blue}{\dotp \permutationsparitytensor \dotp} {^2\!\bm{B}} ,
\\[1.6em]
%
{^2\hspace{-0.05ex}\bm{C}} \mathcolor{blue}{\times} \hspace{-0.1ex} \bm{d} \bm{b} = C_{i\hspace{-0.1ex}j} \bm{e}_i \bm{e}_j \hspace{-0.1ex} \times d_p b_q \bm{e}_p \bm{e}_q \hspace{-0.1ex} = \bm{e}_i C_{i\hspace{-0.1ex}j} d_p \tikzmark{BeginTensorCrossTensor} \permutationsparitysymbols{j\hspace{-0.1ex}pk} \tikzmark{EndTensorCrossTensor} \bm{e}_k b_q \bm{e}_q =
\\[1.6em]
\hspace{12.5em} =
- \hspace{.2ex} {^2\hspace{-0.05ex}\bm{C}} \hspace{-0.1ex} \bm{d} \hspace{.12ex} \dotdotp \permutationsparitytensor \hspace{.1ex} \bm{b} \hspace{.2ex} =
\mathcolor{blue}{-} \hspace{.2ex} {^2\hspace{-0.05ex}\bm{C}} \mathcolor{blue}{\dotp \permutationsparitytensor \dotp} \bm{d} \bm{b} ,
\\
\end{array}
\end{equation*}%
\AddUnderBrace[line width=.75pt][0,-0.2ex]%
{BeginVectorCrossTensor}{EndVectorCrossTensor}{${\scriptstyle - a_i \permutationsparitysymbols{in\hspace{-0.1ex}j} B_{\hspace{-0.1ex}j\hspace{-0.1ex}k}}$}%
\AddUnderBrace[line width=.75pt][.2ex,-0.2ex]%
{BeginTensorCrossTensor}{EndTensorCrossTensor}{${\scriptstyle \;- \permutationsparitysymbols{pj\hspace{-0.1ex}k} \:=\: - \permutationsparitysymbols{j\hspace{-0.1ex}kp}}$}%
%
\vspace{-0.32em}\begin{equation}\label{iso:twothree}
\UnitDyad \times \hspace{-0.16ex} \UnitDyad = \bm{e}_i \bm{e}_i \times \bm{e}_j \bm{e}_j = \hspace{-0.4ex} \tikzmark{BeginECrossE} - \hspace{-0.2ex} \permutationsparitysymbols{i\hspace{-0.1ex}j\hspace{-0.1ex}k} \bm{e}_i \bm{e}_j \bm{e}_k \tikzmark{EndECrossE} = - \hspace{.2ex} \permutationsparitytensor
\hspace{.1ex} .
\end{equation}
%
\AddUnderBrace[line width=.75pt][.2ex,-0.2ex]%
{BeginECrossE}{EndECrossE}{${\scriptstyle \;\;+\permutationsparitysymbols{i\hspace{-0.1ex}j\hspace{-0.1ex}k} \bm{e}_i \bm{e}_k \bm{e}_j}$}

\vspace{-0.5em}\noindent
\en{The last}\ru{Последнее} \en{equation}\ru{равенство} \en{connects}\ru{связывает} \en{the isotropic tensors}\ru{изотропные тензоры} \en{of second and third complexities}\ru{второй и~третьей сложностей}.

\en{Generalizing to}\ru{Обобщая на} \en{all tensors}\ru{все тензоры} \en{of nonzero complexity}\ru{ненулевой сложности}

\nopagebreak\vspace{-0.2em}\begin{equation}\label{crossproductforanytwotensors}
{^\mathrm{n}\hspace{-0.12ex}\bm{\xi}} \times \hspace{-0.12ex} {^\mathrm{m}\hspace{-0.12ex}\bm{\zeta}}
=
- \hspace{.25ex} {^\mathrm{n}\hspace{-0.12ex}\bm{\xi}} \dotp \permutationsparitytensor \dotp \hspace{-0.1ex} {^\mathrm{m}\hspace{-0.12ex}\bm{\zeta}}
\;\;\:
%
\forall \hspace{.4ex} {^\mathrm{n}\hspace{-0.12ex}\bm{\xi}} ,
\hspace{-0.12ex} {^\mathrm{m}\hspace{-0.12ex}\bm{\zeta}}
\;\;
\forall \hspace{.25ex} n \hspace{-0.25ex} > \hspace{-0.25ex} 0 , \,
m \hspace{-0.25ex} > \hspace{-0.25ex} 0
\hspace{.1ex} .
\end{equation}

\vspace{-0.1em}\noindent
\en{When one of the operands}\ru{Когда один из операндов}\en{ is}\ru{\:---} \en{the }\en{unit}\ru{единичный}~(\en{metric}\ru{метрический}) \en{tensor}\ru{тензор}, \en{from}\ru{из}~\eqref{crossproductforanytwotensors} \en{and}\ru{и}~\eqref{definingpropertyofidentitytensor} ${\forall \, {^\mathrm{n}\hspace{-0.12ex}\bm{\Upsilon}}}$ ${\forall \,\mathrm{n \!>\! 0}}$

\nopagebreak\vspace{-0.2em}\begin{equation*}\begin{array}{c}
\UnitDyad \hspace{.1ex} \times \hspace{-0.1ex} {^\mathrm{n}\hspace{-0.12ex}\bm{\Upsilon}}
= - \hspace{.2ex} \UnitDyad \hspace{.1ex} \dotp \permutationsparitytensor \dotp {^\mathrm{n}\hspace{-0.12ex}\bm{\Upsilon}}
= - \hspace{.2ex} \permutationsparitytensor \dotp {^\mathrm{n}\hspace{-0.12ex}\bm{\Upsilon}} \hspace{-0.15ex},
\\[.2em]
%
{^\mathrm{n}\hspace{-0.12ex}\bm{\Upsilon}} \times \UnitDyad
= - \hspace{.2ex} {^\mathrm{n}\hspace{-0.12ex}\bm{\Upsilon}} \hspace{-0.1ex} \dotp \permutationsparitytensor \dotp \hspace{-0.1ex} \UnitDyad
= - \hspace{.2ex} {^\mathrm{n}\hspace{-0.12ex}\bm{\Upsilon}} \hspace{-0.1ex} \dotp \permutationsparitytensor
\hspace{.1ex} .
\end{array}\end{equation*}

\vspace{-0.1em}
\en{The~cross product}\ru{Векторное произведение}
\en{of the two vectors}\ru{двух векторов}
\en{is not~commutative}\ru{не~коммутативно},
\en{but}\ru{но}
\en{is anti\-commutative}\ru{анти\-коммутативно}:

\nopagebreak
\begin{equation}\label{crossproductoftwovectors}
\begin{array}{c}
\bm{a} \times \bm{b}
= \bm{a} \dotp \hspace{-0.1ex}
\left(
   \bm{b}
   \hspace{-0.1ex} \times \hspace{-0.25ex}
   \UnitDyad
   \hspace{.1ex} \right)
= \left( \bm{a} \hspace{-0.1ex} \times \hspace{-0.25ex} \UnitDyad \hspace{.1ex} \right) \hspace{-0.1ex} \dotp \bm{b}
= - \hspace{.25ex} \bm{a} \bm{b} \dotdotp \permutationsparitytensor
= - \hspace{.2ex} \permutationsparitytensor \dotdotp \bm{a} \bm{b}
\hspace{.15ex} ,
\\[.15em]
%
\bm{b} \times \bm{a}
= \bm{b} \dotp \hspace{-0.1ex} \left( \bm{a} \hspace{-0.1ex} \times \hspace{-0.25ex} \UnitDyad \hspace{.1ex} \right)
= \left( \bm{b} \hspace{-0.1ex} \times \hspace{-0.25ex} \UnitDyad \hspace{.1ex} \right)  \hspace{-0.1ex} \dotp \bm{a}
= - \hspace{.25ex} \bm{b} \bm{a} \dotdotp \permutationsparitytensor
= - \hspace{.2ex} \permutationsparitytensor \dotdotp \bm{b} \bm{a}
\hspace{.15ex} ,
\\[.2em]
%
\bm{a} \times \bm{b} \hspace{.16ex}
= - \hspace{.33ex} \bm{a} \bm{b} \dotdotp \permutationsparitytensor
= \bm{b} \bm{a} \dotdotp \permutationsparitytensor
\:\Rightarrow\:
\bm{a} \times \bm{b} \hspace{.16ex} = - \hspace{.32ex} \bm{b} \times \bm{a}
\hspace{.2ex} .
\end{array}
\end{equation}

\vspace{-0.2em} \noindent
\en{For}\ru{Для} \en{any}\ru{любого} \en{bivalent tensor}\ru{бивалентного тензора}~${\hspace{-0.2ex}^2\hspace{-0.16em}\bm{B}}$
\en{and}\ru{и}~\en{a~tensor of first complexity}\ru{тензора первой сложности}~(\en{vector}\ru{вектора})~$\bm{a}$

\nopagebreak\vspace{-0.3em}\begin{equation*}\label{crossproductisnotcommutative}
{^2\hspace{-0.16em}\bm{B}} \hspace{-0.1ex} \times \bm{a}
= \bm{e}_i B_{i\hspace{-0.1ex}j} \bm{e}_{\hspace{-0.1ex}j} \hspace{-0.25ex} \times \hspace{-0.1ex} a_k \bm{e}_k \hspace{-0.2ex}
= \bigl( \hspace{-0.1ex} - \hspace{.25ex} a_k \bm{e}_k \hspace{-0.25ex} \times \hspace{-0.1ex} \bm{e}_{\hspace{-0.1ex}j} B_{i\hspace{-0.1ex}j} \bm{e}_i \hspace{.16ex} \bigr)^{\hspace{-0.25ex}\T} \hspace{-0.4ex}
= - \hspace{.25ex} \bigl( \hspace{.12ex} \bm{a} \times \hspace{-0.2ex} {^2\hspace{-0.16em}\bm{B}^{\T}} \hspace{.1ex} \bigr)^{\hspace{-0.3ex}\T}
\hspace{-0.4ex} .
\end{equation*}

\vspace{-0.15em} \noindent
\en{However}\ru{Однако}, \en{in~the~particular case of~}\ru{в~частном случае }\en{the~unit tensor}\ru{единичного тензора}~${\hspace{-0.2ex}\UnitDyad}$ \en{and}\ru{и}~\en{a~vector}\ru{вектора}

%%\bigl( \UnitDyad \times \bm{a} \bigr)^{\hspace{-0.25ex}\T} \hspace{-0.4ex}
%%= - \hspace{.33ex} \bm{a} \times \UnitDyad
%%= - \hspace{.25ex} \UnitDyad \times \bm{a}
%%= \hspace{-0.1ex} \bigl( \bm{a} \times \UnitDyad \hspace{.1ex} \bigr)^{\hspace{-0.25ex}\T}

\nopagebreak\vspace{-0.25em}\begin{equation}\begin{array}{c}
\UnitDyad \times \bm{a}
= - \hspace{.25ex} \bigl( \hspace{.1ex} \bm{a} \times \hspace{-0.25ex} \UnitDyad^{\hspace{.1ex}\T} \hspace{.1ex} \bigr)^{\hspace{-0.3ex}\T} \hspace{-0.4ex}
= - \hspace{.25ex} \bigl( \bm{a} \times \hspace{-0.25ex} \UnitDyad \hspace{.15ex} \bigr)^{\hspace{-0.25ex}\T} \hspace{-0.4ex}
= \bm{a} \times \hspace{-0.25ex} \UnitDyad
\hspace{.15ex} ,
\\[.25em]
%
\UnitDyad \times \bm{a} = \hspace{.1ex} \bm{a} \times \hspace{-0.25ex} \UnitDyad
= - \hspace{.25ex} \bm{a} \dotp \permutationsparitytensor
= - \hspace{.2ex} \permutationsparitytensor \dotp \bm{a}
\hspace{.2ex} .
\end{array}\end{equation}

\begin{otherlanguage}{russian}

Справедливо такое соотношение

\nopagebreak\vspace{-0.1em}\begin{equation}\label{doubleveblen}
e_{i\hspace{-0.1ex}j\hspace{-0.1ex}k} \hspace{.1ex} e_{pqr}
\hspace{-0.1ex} = \hspace{.1ex}
\operatorname{det} \hspace{-0.25ex} \left[
\begin{array}{ccc}
\delta_{ip} & \delta_{iq} & \delta_{ir} \\
\delta_{\hspace{-0.1ex}j\hspace{-0.1ex}p} & \delta_{\hspace{-0.1ex}j\hspace{-0.1ex}q} & \delta_{\hspace{-0.1ex}j\hspace{-0.1ex}r} \\
\delta_{kp} & \delta_{kq} & \delta_{kr}
\end{array}\hspace{-0.1ex}
\right]
\end{equation}

\noindent
${ \tikz[baseline=-1ex] \draw [line width=.5pt, color=black, fill=white] (0, 0) circle (.8ex);
\hspace{.6em} }$
Доказательство
начнём
с~представления
символов чётности перестановки
как определителей~\eqref{permutationsymbolasdeterminant}.
${ e_{i\hspace{-0.1ex}j\hspace{-0.1ex}k} \hspace{-0.2ex}
= \pm \hspace{.4ex}
\bm{e}_i \hspace{-0.36ex} \times \hspace{-0.2ex} \bm{e}_j \hspace{-0.15ex} \dotp \hspace{.1ex} \bm{e}_k }$ по строкам,
${ e_{pqr} \hspace{-0.2ex}
= \pm \hspace{.4ex}
\bm{e}_p \hspace{-0.36ex} \times \hspace{-0.2ex} \bm{e}_q \hspace{-0.15ex} \dotp \hspace{.1ex} \bm{e}_r }$ по~столбцам,
с~\inquotes{$-$} для~\inquotes{левой} тройки

\nopagebreak\vspace{-0.25em}\begin{equation*}
e_{i\hspace{-0.1ex}j\hspace{-0.1ex}k} \hspace{-0.1ex} = \hspace{.1ex}
\operatorname{det}\hspace{-0.25ex} \scalebox{0.96}[0.96]{$\left[\begin{array}{ccc}
\delta_{i1} & \delta_{i2} & \delta_{i3} \\
\delta_{\hspace{-0.1ex}j1} & \delta_{\hspace{-0.1ex}j2} & \delta_{\hspace{-0.1ex}j3} \\
\delta_{k1} & \delta_{k2} & \delta_{k3}
\end{array}\right]$} \hspace{-0.25ex}, \:\:
e_{pqr} \hspace{-0.1ex} = \hspace{.1ex}
\operatorname{det}\hspace{-0.25ex} \scalebox{0.96}[0.96]{$\left[\begin{array}{ccc}
\delta_{p1} & \delta_{q1} & \delta_{r1} \\
\delta_{p2} & \delta_{q2} & \delta_{r2} \\
\delta_{p3} & \delta_{q3} & \delta_{r3}
\end{array}\right]$}
\hspace{-0.2ex} .
\end{equation*}

\vspace{-0.1em}\noindent
Левая часть~\eqref{doubleveblen} есть произведение ${e_{i\hspace{-0.1ex}j\hspace{-0.1ex}k} \hspace{.1ex} e_{pqr}}$ этих определителей.
Но~${\operatorname{det} \hspace{.2ex} (\hspace{-0.1ex}AB\hspace{.1ex}) = (\operatorname{det} A)(\operatorname{det} B)}$\:--- определитель произведения матриц равен произведению определителей~\eqref{determinantofmatrixproduct}.
В~матрице\hbox{-}произведении элемент~${\left[{\,\cdots\,}\right]}_{1\hspace{-0.1ex}1}$ равен~${\delta_{is} \delta_{ps} \hspace{-0.2ex} = \delta_{ip}}$, как~и~в~\eqref{doubleveblen}; \textcolor{magenta}{легко проверить и~другие фрагменты}.
${ \hspace{.6em}
\tikz[baseline=-0.6ex] \draw [color=black, fill=black] (0, 0) circle (.8ex); }$

\en{The contraction of}\ru{Свёртка}~\eqref{doubleveblen} приводит к~полезным формулам

\nopagebreak\vspace{-0.2em}\begin{equation*}\begin{array}{c}
e_{i\hspace{-0.1ex}j\hspace{-0.1ex}k} \hspace{.1ex} e_{pqk} \hspace{-0.1ex} = \hspace{.2ex}
\operatorname{det} \hspace{-0.25ex} \left[
\begin{array}{ccc}
\delta_{ip} & \delta_{iq} & \delta_{ik} \\
\delta_{\hspace{-0.1ex}j\hspace{-0.1ex}p} & \delta_{\hspace{-0.1ex}j\hspace{-0.1ex}q} & \delta_{\hspace{-0.1ex}j\hspace{-0.1ex}k} \\
\delta_{kp} & \delta_{kq} & \delta_{kk}
\end{array}
\right] \hspace{-0.5ex} = \hspace{.2ex}
\operatorname{det} \hspace{-0.25ex} \left[
\begin{array}{ccc}
\delta_{ip} & \delta_{iq} & \delta_{ik} \\
\delta_{\hspace{-0.1ex}j\hspace{-0.1ex}p} & \delta_{\hspace{-0.1ex}j\hspace{-0.1ex}q} & \delta_{\hspace{-0.1ex}j\hspace{-0.1ex}k} \\
\delta_{kp} & \delta_{kq} & 3
\end{array}
\right] \hspace{-0.5ex} =
\\[1.5em]
%
= \hspace{.1ex} 3 \hspace{.2ex} \delta_{ip} \delta_{\hspace{-0.1ex}j\hspace{-0.1ex}q} \hspace{-0.2ex}
+ \delta_{iq} \delta_{\hspace{-0.1ex}j\hspace{-0.1ex}k} \delta_{kp} \hspace{-0.2ex}
+ \delta_{ik} \delta_{\hspace{-0.1ex}j\hspace{-0.1ex}p} \delta_{kq} \hspace{-0.2ex}
- \delta_{ik} \delta_{\hspace{-0.1ex}j\hspace{-0.1ex}q} \delta_{kp} \hspace{-0.2ex}
- \hspace{.1ex} 3 \hspace{.2ex} \delta_{iq} \delta_{\hspace{-0.1ex}j\hspace{-0.1ex}p} \hspace{-0.2ex}
- \delta_{ip} \delta_{\hspace{-0.1ex}j\hspace{-0.1ex}k} \delta_{kq} \hspace{-0.2ex} =
\\[.25em]
%
= \hspace{.1ex} 3 \hspace{.2ex} \delta_{ip} \delta_{\hspace{-0.1ex}j\hspace{-0.1ex}q} \hspace{-0.2ex}
+ \delta_{iq} \delta_{\hspace{-0.1ex}j\hspace{-0.1ex}p} \hspace{-0.2ex}
+ \delta_{iq} \delta_{\hspace{-0.1ex}j\hspace{-0.1ex}p} \hspace{-0.2ex}
- \delta_{ip} \delta_{\hspace{-0.1ex}j\hspace{-0.1ex}q} \hspace{-0.2ex}
- \hspace{.1ex} 3 \hspace{.2ex} \delta_{iq} \delta_{\hspace{-0.1ex}j\hspace{-0.1ex}p} \hspace{-0.2ex}
- \delta_{ip} \delta_{\hspace{-0.1ex}j\hspace{-0.1ex}q} \hspace{-0.2ex} =
\\[.25em]
%
= \delta_{ip} \delta_{\hspace{-0.1ex}j\hspace{-0.1ex}q} \hspace{-0.1ex}
- \hspace{.1ex} \delta_{iq} \delta_{\hspace{-0.1ex}j\hspace{-0.1ex}p}
\hspace{.2ex} ,
\end{array}\end{equation*}

\nopagebreak\vspace{-0.15em}\begin{equation*}
e_{i\hspace{-0.1ex}j\hspace{-0.1ex}k} \hspace{.1ex} e_{pj\hspace{-0.1ex}k} \hspace{-0.15ex}
= \delta_{ip} \delta_{\hspace{-0.1ex}j\hspace{-0.12ex}j} \hspace{-0.2ex} - \hspace{.1ex} \delta_{i\hspace{-0.1ex}j} \delta_{\hspace{-0.1ex}j\hspace{-0.1ex}p} \hspace{-0.15ex}
= \hspace{.1ex} 3 \hspace{.2ex} \delta_{ip} \hspace{-0.2ex} - \delta_{ip} \hspace{-0.15ex}
= \hspace{.1ex} 2 \hspace{.1ex} \delta_{ip}
\hspace{.2ex} ,
\end{equation*}

\nopagebreak\vspace{-0.15em}\begin{equation*}
e_{i\hspace{-0.1ex}j\hspace{-0.1ex}k} \hspace{.1ex} e_{i\hspace{-0.1ex}j\hspace{-0.1ex}k} \hspace{-0.15ex}
= \hspace{.1ex} 2 \hspace{.2ex} \delta_{ii} \hspace{-0.15ex}
= \hspace{.1ex} 6
\hspace{.2ex} .
\end{equation*}

\vspace{-0.33em}\noindent
\en{Or}\ru{Или} \en{in short}\ru{к\'{о}ротко}

\nopagebreak\vspace{-0.3em}\begin{equation}\label{veblencontraction}
e_{i\hspace{-0.1ex}j\hspace{-0.1ex}k} \hspace{.1ex} e_{pqk} \hspace{-0.25ex} = \delta_{ip} \delta_{\hspace{-0.1ex}j\hspace{-0.1ex}q} \hspace{-0.25ex} - \hspace{.1ex} \delta_{iq} \delta_{\hspace{-0.1ex}j\hspace{-0.1ex}p}
\hspace{.2ex} ,
\hspace{.5em}
%
e_{i\hspace{-0.1ex}j\hspace{-0.1ex}k} \hspace{.1ex} e_{pj\hspace{-0.1ex}k} \hspace{-0.25ex} = 2 \hspace{.2ex} \delta_{ip}
\hspace{.2ex} ,
\hspace{.5em}
%
e_{i\hspace{-0.1ex}j\hspace{-0.1ex}k} \hspace{.1ex} e_{i\hspace{-0.1ex}j\hspace{-0.1ex}k} \hspace{-0.25ex} = 6
\hspace{.2ex} .
\end{equation}

\en{The first}\ru{Первая} \en{of~these formulas}\ru{из~этих формул} даёт представление двойного векторного произведения

\nopagebreak\vspace{-0.5em}\begin{multline}
\bm{a} \times \hspace{-0.15ex} \left(\hspace{.2ex}{\bm{b} \times \bm{c}}\hspace{.15ex}\right)
= a_i \bm{e}_i \times \permutationsparitysymbols{pqj} \hspace{.2ex} b_p c_q \bm{e}_j \hspace{-0.2ex}
= \permutationsparitysymbols{ki\hspace{-0.1ex}j} \permutationsparitysymbols{pqj} \hspace{.2ex} a_i b_p c_q \bm{e}_k \hspace{-0.2ex} =
\\
= \left({\delta_{kp}\delta_{iq} \hspace{-0.2ex} - \delta_{kq}\delta_{ip}}\right) a_i b_p c_q \bm{e}_k \hspace{-0.2ex}
= a_i b_k c_i \bm{e}_k \hspace{-0.15ex} - a_i b_i c_k \bm{e}_k \hspace{-0.2ex} =
\\
= \bm{a} \dotp \bm{c} \bm{b} - \bm{a} \dotp \bm{b} \bm{c}
= \bm{a} \dotp \hspace{-0.1ex} \bigl( \bm{c} \bm{b} - \bm{b} \bm{c} \hspace{.15ex} \bigr) \hspace{-0.3ex}
= \bm{a} \dotp \bm{c} \bm{b} - \bm{c} \bm{b} \dotp \bm{a}
%%= \bm{b} \bm{a} \dotp \bm{c} - \bm{c} \bm{a} \dotp \bm{b}
\hspace{.1ex} .
\end{multline}

\end{otherlanguage}

\vspace{-0.1em}\noindent
\en{By~another interpretation}\ru{По~другой интерпретации}, \en{the~}dot product \en{of~a~dyad}\ru{диады} \en{and}\ru{и}~\en{a~vector}\ru{вектора} \en{is not~commutative}\ru{не~коммутативен}:\ru{\hspace{.2ex}}
${\bm{b} \bm{d} \hspace{.1ex} \dotp \bm{c} \hspace{.25ex} \neq \hspace{.25ex} \bm{c} \hspace{.1ex} \dotp \bm{b} \bm{d}}$,
\en{and}\ru{и}~\en{this difference}\ru{эта разница} \en{can be expressed as}\ru{может быть выражена как}

\nopagebreak\vspace{-1.2em}\begin{equation}
\bm{b} \bm{d} \hspace{.1ex} \dotp \bm{c} \hspace{.25ex} - \hspace{.25ex} \bm{c} \hspace{.1ex} \dotp \bm{b} \bm{d}
\hspace{.25ex} = \hspace{.25ex}
\bm{c} \times \hspace{-0.2ex} \bigl( \hspace{.1ex} \bm{b} \times \bm{d} \hspace{.25ex} \bigr)
\hspace{-0.1ex} .
\end{equation}

\noindent ${
\bm{a} \dotp \bm{b} \bm{c} = \bm{c} \bm{b} \dotp \bm{a} = \bm{c} \bm{a} \dotp \bm{b} = \bm{b} \dotp \bm{a} \bm{c}
}$

\noindent ${
\left(\hspace{.1ex} \bm{a} \times \bm{b} \hspace{.2ex}\right) \hspace{-0.2ex} \times \bm{c}
= - \hspace{.33ex} \bm{c} \times \hspace{-0.15ex} \left(\hspace{.2ex}{\bm{a} \times \bm{b}}\hspace{.15ex}\right) \hspace{-0.1ex}
= \bm{c} \times \hspace{-0.2ex} \left(\hspace{.2ex}{\bm{b} \times \bm{a}}\hspace{.15ex}\right)
}$

\vspace{-0.15em}\noindent
\textcolor{magenta}{\en{The~same way}\ru{Тем~же путём} \en{it may~be derived that}\ru{выводится}}

\nopagebreak\vspace{-0.2em}\begin{equation}
\left(\hspace{.1ex} \bm{a} \times \bm{b} \hspace{.2ex}\right) \hspace{-0.2ex} \times \bm{c}
= \hspace{-0.15ex} \bigl( \hspace{.1ex} \bm{b} \bm{a} - \bm{a} \bm{b} \hspace{.15ex} \bigr) \hspace{-0.3ex} \dotp \bm{c}
= \bm{b} \bm{a} \dotp \bm{c} - \bm{a} \bm{b} \dotp \bm{c}
\hspace{.2ex} .
\end{equation}

\vspace{-0.15em}\noindent
\en{And}\ru{И}~\en{following}\ru{следующие} \en{identities}\ru{тождества} \en{for any two vectors}\ru{для~любых двух векторов}~$\bm{a}$ \en{and}\ru{и}~$\bm{b}$

\nopagebreak\vspace{-0.5em}
\begin{multline}\label{vectorcrossvectorcrossidentity}
\bigl( \bm{a} \times \bm{b} \hspace{.15ex}\bigr) \hspace{-0.33ex} \times \hspace{-0.22ex} \UnitDyad
= \permutationsparitysymbols{i\hspace{-0.1ex}j\hspace{-0.1ex}k} \hspace{.2ex} a_i \hspace{.1ex} b_j \hspace{.1ex} \bm{e}_k \hspace{-0.15ex} \times \hspace{-0.1ex} \bm{e}_n \bm{e}_n \hspace{-0.2ex}
= a_i \hspace{.1ex} b_j \hspace{.1ex} \permutationsparitysymbols{i\hspace{-0.1ex}j\hspace{-0.1ex}k} \permutationsparitysymbols{knq} \hspace{.2ex} \bm{e}_q \bm{e}_n \hspace{-0.2ex} =
\\[-0.1em]
%
= a_i b_j \bigl( \delta_{in} \delta_{j\hspace{-0.1ex}q} \hspace{-0.22ex} - \delta_{iq} \delta_{jn} \bigr) \bm{e}_q \bm{e}_n \hspace{-0.2ex}
= a_i b_j \bm{e}_j \bm{e}_i \hspace{-0.12ex} - a_i b_j \bm{e}_i \bm{e}_j \hspace{-0.2ex} =
\\[-0.1em]
%
= \bm{b} \bm{a} - \bm{a} \bm{b}
\hspace{.2ex},
\end{multline}

\vspace{-1.1em}\begin{multline}\label{vectorcrossidentitydotvectorcrossidentity}
\bigl( \bm{a} \hspace{-0.1ex}\times\hspace{-0.25ex} \UnitDyad \hspace{.2ex}\bigr) \hspace{-0.25ex} \dotp \hspace{-0.15ex} \bigl(\hspace{.1ex} \bm{b} \hspace{-0.1ex}\times\hspace{-0.25ex} \UnitDyad \hspace{.2ex}\bigr) \hspace{-0.2ex}
= \bigl( \bm{a} \hspace{.1ex} \dotp \hspace{-0.1ex} \permutationsparitytensor \hspace{.1ex}\bigr) \hspace{-0.25ex} \dotp \hspace{-0.15ex} \bigl(\hspace{.1ex} \bm{b} \hspace{.1ex} \dotp \hspace{-0.1ex} \permutationsparitytensor \hspace{.1ex}\bigr) \hspace{-0.2ex} =
\\[-0.1em]
%
= a_i \permutationsparitysymbols{ipn} \hspace{.1ex} \bm{e}_p \bm{e}_n \hspace{-0.15ex} \dotp \hspace{.2ex} b_j \permutationsparitysymbols{jsk} \hspace{.1ex} \bm{e}_s \bm{e}_k \hspace{-0.2ex}
= a_i b_j \permutationsparitysymbols{ipn} \permutationsparitysymbols{nkj} \hspace{.05ex} \bm{e}_p \bm{e}_k \hspace{-0.2ex} =
\\[-0.1em]
%
= a_i b_j \bigl( \delta_{ik} \delta_{pj} \hspace{-0.22ex} - \delta_{i\hspace{-0.1ex}j} \delta_{pk} \bigr) \bm{e}_p \bm{e}_k \hspace{-0.2ex}
= a_i b_j \bm{e}_j \bm{e}_i \hspace{-0.12ex} - a_i b_i \bm{e}_k \bm{e}_k \hspace{-0.2ex} =
\\[-0.1em]
%
= \hspace{.1ex} \bm{b} \bm{a} - \bm{a} \hspace{-0.1ex} \dotp \bm{b} \hspace{.1ex} \UnitDyad
\hspace{.1ex} .
\end{multline}

\vspace{-0.3em}
\en{Finally}\ru{Наконец},
\en{the one more}\ru{ещё одно} \en{correlation}\ru{соотношение}
\en{between}\ru{между}
\en{the isotropic tensors}\ru{изотропными тензорами}
\en{of the second and third}\ru{второй и~третьей}
\en{complexities}\ru{сложностей}:

\nopagebreak\vspace{-0.2em}\begin{equation}
\permutationsparitytensor \hspace{.2ex} \dotdotp \permutationsparitytensor
= \permutationsparitysymbols{i\hspace{-0.1ex}j\hspace{-0.1ex}k} \bm{e}_i \hspace{.2ex} \permutationsparitysymbols{kjn} \bm{e}_n \hspace{-0.2ex}
= - \hspace{.2ex} 2 \hspace{.2ex} \delta_{in} \hspace{.12ex} \bm{e}_i \bm{e}_n \hspace{-0.2ex}
= - \hspace{.1ex} 2 \hspace{.1ex} \UnitDyad
\hspace{.1ex} .
\end{equation}
