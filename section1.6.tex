\section{\en{Matrices}\ru{Матрицы}, \en{permutations}\ru{перестановки} \en{and}\ru{и} \en{determinants}\ru{определители}}

\label{section:matrices+permutations+determinants}

\en{Matrices}\ru{Матрицы}
\en{are the~convenient tool}\ru{это удобный инструмент}
\en{for}\ru{для}
\en{arranging}\ru{расположения}
\en{of~elements}\ru{элементов}
\en{and for}\ru{и~для}
\en{solving}\ru{решения}
\en{systems}\ru{систем}
\en{of linear equations}\ru{линейных уравнений}.

\en{Does the~reader know}\ru{Знает~ли читатель}\ru{,}
\en{that}\ru{что}
\en{matrices}\ru{матрицы}
\en{are sometimes called}\ru{иногда называют}
\inquotes{\en{arrays}\ru{массивами}}?
\en{Does someone need}\ru{Кому-то нужны}
\en{two-dimensional arrays}\ru{двумерные массивы}?
\en{Matrices}\ru{Матрицы}
\en{can be presented}\ru{могут быть представлены}
\en{as tables}\ru{как таблицы}
\en{full of rows and columns}\ru{полные строк и~колонок}.
\en{Any matrix has}\ru{У~любой матрицы}
\en{the~same number of elements}\ru{одно и то~же число элементов}
\en{in each row}\ru{в~каждой строке}
\en{and }\ru{и~}%
\en{the~same number of elements}\ru{одно и~то~же число элементов}
\en{in each column}\ru{в~каждой колонке}.
\en{Rectangular}\ru{Прямоугольное}
\en{arrangement}\ru{расположение}
\en{of~items}\ru{элементов},
\en{anyone}\ru{кому-нибудь}?
\en{Matrices}\ru{Матрицы}
\en{are full}\ru{полн\'{ы}}
\en{of~numbers}\ru{чисел}
\en{and }\ru{и~}\en{expressions}\ru{выражений}
\en{in rows}\ru{в~строках}
\en{and columns}\ru{и~колонках}.
\en{A~column}\ru{Колонка~(столбец)}
\en{arranges elements}\ru{располагает элементы}
\en{vertically}\ru{вертикально}
\en{from top to bottom}\ru{сверху вниз},
\en{while}\ru{тогда как}
\en{a~row}\ru{строка~(ряд)}
\en{arranges}\ru{располагает}
\en{horizontally}\ru{горизонтально}
\en{from left to right}\ru{слева направо}.

%--------------------

\subsection*{\en{Matrix dimensions}\ru{Размерности матрицы}}

\en{Matrices}\ru{Матрицы}
\en{come in all sizes}\ru{бывают всех размеров},
\en{or}\ru{или}
\inquotesx{\en{dimensions}\ru{размерностей}}[.]

\en{By convention}\ru{По соглашению},
\en{to avoid confusion}\ru{чтобы избежать путаницы},
\en{the~rows are listed first}\ru{строки перечисляются первыми},
\en{and the~columns second}\ru{а~колонки вторыми}.
%
\en{The~dimension of a~matrix}\ru{Размерность матрицы}
\en{is written}\ru{пишется}
\en{as}\ru{как}
\en{the~number of~rows}\ru{число строк},
\en{then}\ru{затем}
\en{a~multiplication sign}\ru{знак умножения}
(\inquotes{×} \en{is used}\ru{используется}
\en{the~most often}\ru{чаще всего}),
\en{and then}\ru{а~затем}
\en{the~number of~columns}\ru{число колонок}.

%%{\small
\noindent
\en{Here are examples}\ru{Вот примеры}

\nopagebreak\vspace{-0.2em}
\begin{alignat*}{2}
\asmatrixwithdimensions{\mathcal{A}}{3}{3}
= \hspace{-0.3ex} \scalebox{.9}{$%
\left[ \begin{array}{c@{\hspace{1em}}c@{\hspace{1em}}c}
A_{1\hspace{-0.1ex}1} & A_{12} & A_{13} \\
A_{21} & A_{22} & A_{23} \\
A_{31} & A_{32} & A_{33}
\end{array} \hspace{.2ex}\right]%
$}
& \hspace{1em}
\scalebox{.8}{ \begin{minipage}{.3\textwidth}\small
   $3$~\en{rows}\ru{строки} \en{and}\ru{и} $3$~\en{columns}\ru{колонки} \\
   \en{a}\ru{матрица}~$3{×}3$\en{~matrix} \\
   \en{it’s}\ru{это}
   \en{a\:}\href{https://en.wikipedia.org/wiki/Square_matrix}{\emph{\inquotes{\en{square}\ru{квадратная}}} \en{matrix}\ru{матрица}}
\end{minipage} }
&&
\scalebox{.8}{ \begin{minipage}{.25\textwidth}\centering\small
   \emph{\en{the~number}\ru{число} \en{of~rows}\ru{строк}} \\
   \emph{\en{is the~same}\ru{такое~же}} \\
   \emph{\en{as of~columns}\ru{как и колонок}}
\end{minipage} }
\\[.4em]
%
\asmatrixwithdimensions{\mathcal{B}}{2}{4}
= \hspace{-0.3ex} \scalebox{.9}{$%
\left[ \begin{array}{c@{\hspace{.9em}}c@{\hspace{.9em}}c@{\hspace{.9em}}c}
B_{1\hspace{-0.1ex}1} & B_{12} & B_{13} & B_{14} \\
B_{21} & B_{22} & B_{23} & B_{24}
\end{array} \hspace{.2ex}\right]%
$}
& \hspace{1em}
\scalebox{.8}{ \begin{minipage}{.3\textwidth}\small
   $2$~\en{rows}\ru{строки} \en{and}\ru{и} $4$~\en{columns}\ru{колонки} \\
   \en{dimension}\ru{размерность}~$2{×}4$
\end{minipage} }
&& %%\hspace{1em}
\\[.4em]
%
\asmatrixwithdimensions{\mathcal{C}}{3}{1}
= \hspace{-0.3ex} \scalebox{.9}{$%
\left[ \begin{array}{c}
C_{1\hspace{-0.1ex}1} \\
C_{21} \\
C_{31}
\end{array} \hspace{-0.2ex}\right]%
$}
& \hspace{1em}
\scalebox{.8}{ \begin{minipage}{.3\textwidth}\small
   $3$~\en{rows}\ru{строки}, $1$~\en{column}\ru{колонка} \\
   $3{×}1$ \\
   \en{a\:}\emph{\inquotes{\en{column matrix}\ru{матрица-колонка}}}
\end{minipage} }
&&
\scalebox{.8}{ \begin{minipage}{.2\textwidth}\centering\small
   \emph{\en{a~matrix}\ru{матрица}} \\
   \emph{\en{with just one}\ru{лишь с~одной}} \\
   \emph{\en{column}\ru{колонкой}}
\end{minipage} }
\\[.4em]
%
\asmatrixwithdimensions{\mathcal{D}}{1}{6}
= \hspace{-0.3ex} \scalebox{.9}{$%
\left[ \begin{array}{c@{\hspace{.5em}}c@{\hspace{.5em}}c@{\hspace{.5em}}c@{\hspace{.5em}}c@{\hspace{.5em}}c}
D_{1\hspace{-0.1ex}1} & D_{12} & D_{13} & D_{14} & D_{15} & D_{16}
\end{array} \hspace{-0.2ex}\right]%
$}
& \hspace{1em}
\scalebox{.8}{ \begin{minipage}{.3\textwidth}\small
   $1$~\en{row}\ru{строка}, $6$~\en{columns}\ru{колонок} \\
   $1{×}6$ \\
   \en{a\:}\emph{\inquotes{\en{row matrix}\ru{матрица-строка}}}
\end{minipage} }
&&
\scalebox{.8}{ \begin{minipage}{.2\textwidth}\centering\small
   \emph{\en{a~matrix}\ru{матрица}} \\
   \emph{\en{with just one}\ru{лишь с~одной}} \\
   \emph{\en{row}\ru{строкой}}
\end{minipage} }
\end{alignat*}
%%\par}

%--------------------

\subsection*{\en{The matrix algebra}\ru{Матричная алгебра}}

\en{The matrix algebra}\ru{Матричная алгебра}
\en{includes}\ru{включает}
\en{the linear operations}\ru{линейные операции}\:---
\en{the addition of matrices}\ru{сложение матриц}
\en{and}\ru{и}
\en{the multiplication by scalar}\ru{умножение на скаляр}.

\en{The~dimension of a~matrix}\ru{Размерность матрицы}
\en{is essential}\ru{существенна}
\en{for binary operations}\ru{для бинарных операций},
\en{that is}\ru{то есть}
\en{for operations}\ru{для операций}
\en{involving}\ru{с~участием}
\en{two matrices}\ru{двух матриц}.

\en{An~addition}\ru{Сложение}
\en{or}\ru{или}
\en{subtraction}\ru{вычитание}
\en{of the two matrices}\ru{двух матриц}
\en{is possible}\ru{возможно}
\en{only}\ru{только}
\en{if}\ru{если}
\en{they have}\ru{они имеют}
\en{the same sizes}\ru{те же размеры}.

%----

\subsection*{\en{The multiplication of matrices}\ru{Умножение матриц}}


.................


\nopagebreak\vspace{.1em}\begin{gather*}
\asmatrixwithdimensions{\mathcal{A}}{m}{n} \hspace{.1ex}
= \hspace{-0.2ex} \ldots
\end{gather*}

\en{The matrix of the result}\ru{Матрица результата},
\en{known as}\ru{известная как}
\inquotes{\en{the matrix product}\ru{матричное произведение}},
\en{has}\ru{имеет}
\en{the number of~rows}\ru{число строк}
\en{of the first multiplier matrix}\ru{первой матрицы-сомножителя}
\en{and the number of~columns}\ru{и~число колонок}
\en{of the second matrix}\ru{второй матрицы}.


....................


\subsection*{\en{Square matrices}\ru{Квадратные матрицы}}

....


\subsection*{\en{Matrices and one-dimensional arrays}\ru{Матрицы и одномерные массивы}}

\en{Two indices}\ru{Два индекса}
\en{of~a~table}\ru{таблицы}\en{ are}\ru{\:---}
\en{more than}\ru{больше, чем}
\en{the single index}\ru{единственный индекс}
\en{of a~one-dimensional array}\ru{одномерного массива}.
\en{Due to this}\ru{Из-за этого}\en{,}
\en{a~one-dimensional array}\ru{одномерный массив}
\en{could be}\ru{может быть}
\en{presented}\ru{представлен}
\inquotes{\en{vertically}\ru{вертикально}}
\en{or}\ru{или}
\inquotesx{\en{horizontally}\ru{горизонтально}}[,]
\en{either}\ru{либо}
\en{as}\ru{как}
\en{a~table}\ru{таблица}
\en{of~rows}\ru{строк}
\en{in~one column}\ru{в~одну колонку}

\nopagebreak\vspace{-0.5em}
\begin{equation*}
\scalebox{.9}{$
\left[ \hspace{-0.1ex}
\begin{array}{c}
v_{1\hspace{-0.1ex}1} \\[-0.1em]
v_{21} \\[-0.1em]
v_{31}
\end{array}
\hspace{-0.1ex} \right]
$}
,
\end{equation*}

\noindent
\en{or}\ru{либо}
\en{as}\ru{как}
\en{a~table}\ru{таблица}
\en{of~columns}\ru{колонок}
\en{in~one row}\ru{в~одну строку}

\nopagebreak\vspace{-0.3em}
\begin{equation*}
\scalebox{.9}{$
\left[ \hspace{-0.1ex}
\begin{array}{c@{\hspace{.5em}}c@{\hspace{.5em}}c}
h_{1\hspace{-0.1ex}1} & h_{12} & h_{13}
\end{array}
\hspace{-0.1ex} \right]
$}
\hspace{.1ex} .
\end{equation*}

....

\subsection*{\en{Permutation parity symbols}\ru{Символы чётности перестановки}}

\en{To write permutations}\ru{Чтобы записывать перестановки},
\ru{вводятся }\en{the~}\inquotes{\ru{символы чётности}\en{parity symbols}}~${e_{i\hspace{-0.1ex}j\hspace{-0.1ex}k}}$\en{ are introduced}.
\en{They}\ru{Их}
\en{are often}\ru{часто}
\en{associated}\ru{связывают}
\en{with the~names}\ru{с~именами}
\en{of~}\href{https://en.wikipedia.org/wiki/Oswald_Veblen}{Oswald\ru{’а} Veblen\ru{’а}}
\en{and}\ru{и}~%
\href{https://en.wikipedia.org/wiki/Tullio_Levi-Civita}{Tullio Levi\hbox{-}Civita\ru{’ы}}.

...

\en{the~permutation parity symbols}\ru{символы чётности перестановки}
\en{via}\ru{через}
\en{the determinant}\ru{детерминант}

\nopagebreak\vspace{-0.2em}\begin{equation*}
e_{pqr} \hspace{-0.2ex}
= e_{i\hspace{-0.1ex}j\hspace{-0.1ex}k} \hspace{.1ex} \delta_{pi} \delta_{\hspace{-0.1ex}qj} \delta_{rk} \hspace{-0.2ex}
= e_{i\hspace{-0.1ex}j\hspace{-0.1ex}k} \hspace{.1ex} \delta_{ip} \delta_{\hspace{-0.15ex}j\hspace{-0.1ex}q} \delta_{kr}
\hspace{.1ex} ,
\end{equation*}

\nopagebreak\vspace{-0.1em}
\begin{equation}\label{permutationsymbolasdeterminant}
e_{pqr} \hspace{-0.1ex}
= \hspace{.1ex}
\determinant\hspace{-0.25ex} \scalebox{.92}{$\left[ \begin{array}{ccc}
\delta_{1p} & \delta_{1q} & \delta_{1r} \\
\delta_{2p} & \delta_{2q} & \delta_{2r} \\
\delta_{3p} & \delta_{3q} & \delta_{3r}
\end{array} \hspace{-0.1ex} \right]$} \hspace{-0.2ex}
= \hspace{.1ex}
\determinant\hspace{-0.25ex} \scalebox{.92}{$\left[ \begin{array}{ccc}
\delta_{p1} & \delta_{p2} & \delta_{p3} \\
\delta_{q1} & \delta_{q2} & \delta_{q3} \\
\delta_{r1} & \delta_{r2} & \delta_{r3}
\end{array} \hspace{-0.1ex} \right]$}
\hspace{-0.2ex} .
\end{equation}

....

\en{A~determinant}\ru{Определитель}
\en{is not sensitive}\ru{не~чувствителен}
\en{to transposing}\ru{к~транспонированию},

\nopagebreak\vspace{-0.25em}
\begin{equation*}
\determinantby{i}{\hspace{.15ex}j} \, A_{i\hspace{-0.1ex}j} \hspace{-0.16ex}
= \hspace{.1ex} \determinantby{i}{\hspace{.15ex}j} \, A_{j\hspace{-0.06ex}i} \hspace{-0.16ex}
= \hspace{.1ex} \determinantby{j\hspace{-0.2ex}}{\hspace{.1ex}i} \, A_{i\hspace{-0.1ex}j}
\hspace{.1ex} .
\end{equation*}

....

\en{The~determinant}\ru{Определитель}
\en{of the~matrix product}\ru{матричного произведения}
\en{of two matrices}\ru{двух матриц}
\en{is equal}\ru{равен}
\en{to the~product}\ru{произведению}
\en{of the~determinants}\ru{определителей}
\en{of~each}\ru{каждой}
\en{of~these}\ru{из~этих}
\en{matrices}\ru{матриц}

\nopagebreak\vspace{-0.2em}
\begin{equation}\label{determinantofmatrixproduct}
\determinantby{i}{k} \, \somematrixelements{ik} \hspace{.4ex} \determinantby{k}{\hspace{.15ex}j} \, C_{kj} \hspace{-0.15ex}
= \hspace{.1ex} \determinantby{i}{\hspace{.15ex}j} \, \somematrixelements{ik} \hspace{.1ex} C_{kj}
\end{equation}

\[
e_{\hspace{-0.25ex}f\hspace{-0.2ex}gh} \hspace{.33ex} \determinantby{m}{n} \, \somematrixelements{m\mathcolor{blue}{s}} \hspace{.1ex} C_{\mathcolor{blue}{s}n} \hspace{-0.2ex}
= e_{pqr} \hspace{.1ex} \somematrixelements{\hspace{-0.25ex}f\hspace{-0.1ex}\mathcolor{blue}{i}} C_{\mathcolor{blue}{i}p} \hspace{.1ex} \somematrixelements{\hspace{-0.1ex}g\mathcolor{blue}{j}} C_{\hspace{-0.1ex}\mathcolor{blue}{j}\hspace{-0.1ex}q} \hspace{.1ex} \somematrixelements{h\mathcolor{blue}{k}} C_{\mathcolor{blue}{k}r}
\hspace{-0.2ex}
\]

\[
e_{\hspace{-0.25ex}f\hspace{-0.2ex}gh} \hspace{.33ex} \determinantby{m}{s} \, \somematrixelements{ms} \hspace{-0.2ex}
= e_{i\hspace{-0.1ex}j\hspace{-0.1ex}k} \hspace{.1ex} \somematrixelements{\hspace{-0.25ex}f\hspace{-0.1ex}i} \somematrixelements{\hspace{-0.1ex}gj} \somematrixelements{hk}
\hspace{-0.2ex}
\]

\[
e_{i\hspace{-0.1ex}j\hspace{-0.1ex}k} \hspace{.33ex} \determinantby{s}{\hspace{.1ex}n} \, C_{sn} \hspace{-0.2ex}
= e_{pqr} \hspace{.1ex} C_{ip} C_{\hspace{-0.1ex}j\hspace{-0.1ex}q} C_{kr}
\hspace{-0.2ex}
\]

\[
e_{\hspace{-0.25ex}f\hspace{-0.2ex}gh} \hspace{.2ex} \mathcolor{black!50}{e_{i\hspace{-0.1ex}j\hspace{-0.1ex}k}} \hspace{.33ex} \determinantby{m}{s} \, \somematrixelements{ms} \hspace{.33ex} \determinantby{s}{\hspace{.1ex}n} \, C_{sn} \hspace{-0.2ex}
= \mathcolor{black!50}{e_{i\hspace{-0.1ex}j\hspace{-0.1ex}k}} \hspace{.2ex} e_{pqr} \hspace{.1ex} \somematrixelements{\hspace{-0.25ex}f\hspace{-0.1ex}i} \somematrixelements{\hspace{-0.1ex}gj} \somematrixelements{hk} \hspace{.1ex} C_{ip} C_{\hspace{-0.1ex}j\hspace{-0.1ex}q} C_{kr}
\hspace{-0.2ex}
\]

...

\en{There’s}\ru{Есть}
\en{the~following}\ru{следующее}
\en{equality}\ru{равенство}

\nopagebreak\vspace{-0.1em}
\begin{equation}\label{theproductoftwopermutations}
e_{i\hspace{-0.1ex}j\hspace{-0.1ex}k} \hspace{.1ex} e_{pqr}
\hspace{-0.1ex} = \hspace{.1ex}
\determinant \hspace{-0.25ex} \left[
\begin{array}{ccc}
\delta_{ip} & \delta_{iq} & \delta_{ir} \\
\delta_{\hspace{-0.1ex}j\hspace{-0.1ex}p} & \delta_{\hspace{-0.1ex}j\hspace{-0.1ex}q} & \delta_{\hspace{-0.1ex}j\hspace{-0.1ex}r} \\
\delta_{kp} & \delta_{kq} & \delta_{kr}
\end{array}\hspace{-0.1ex}
\right]
\end{equation}

\noindent
${ \tikz[baseline=-1ex] \draw [line width=.5pt, color=black, fill=white] (0, 0) circle (.8ex);
\hspace{.6em} }$
\en{Representing}\ru{Представляя}
\en{the~permutation parity symbols}\ru{символы чётности перестановки}
\en{via}\ru{через}
\en{determinants}\ru{определители}~\eqref{permutationsymbolasdeterminant},
${ e_{i\hspace{-0.1ex}j\hspace{-0.1ex}k} }$~\en{by~rows}\ru{по~строкам}
\en{and}\ru{и}~${ e_{pqr} }$~\en{by~columns}\ru{по~колонкам}

\nopagebreak\vspace{-0.2em}
\begin{equation*}
e_{i\hspace{-0.1ex}j\hspace{-0.1ex}k} \hspace{-0.1ex} = \hspace{.1ex}
\determinant\hspace{-0.25ex} \scalebox{0.96}[0.96]{$\left[\begin{array}{ccc}
\delta_{i1} & \delta_{i2} & \delta_{i3} \\
\delta_{\hspace{-0.1ex}j1} & \delta_{\hspace{-0.1ex}j2} & \delta_{\hspace{-0.1ex}j3} \\
\delta_{k1} & \delta_{k2} & \delta_{k3}
\end{array}\right]$}
\hspace{-0.2ex} , \hspace{.5em}
%
e_{pqr} \hspace{-0.1ex} = \hspace{.1ex}
\determinant\hspace{-0.25ex} \scalebox{0.96}[0.96]{$\left[\begin{array}{ccc}
\delta_{p1} & \delta_{q1} & \delta_{r1} \\
\delta_{p2} & \delta_{q2} & \delta_{r2} \\
\delta_{p3} & \delta_{q3} & \delta_{r3}
\end{array}\right]$}
\hspace{-0.2ex} ,
\end{equation*}

\vspace{-0.25em}\noindent
${e_{i\hspace{-0.1ex}j\hspace{-0.1ex}k} \hspace{.1ex} e_{pqr}}$
\en{on the~left side of}\ru{в~левой части}~\eqref{theproductoftwopermutations}
\en{appears as}\ru{предстаёт как}
\en{the~product}\ru{произведение}
\en{of~these determinants}\ru{этих определителей}.
%
\en{And then}\ru{И~тогда}~${\determinant \hspace{.2ex} (\hspace{-0.1ex}AB\hspace{.1ex}) = (\determinant A)(\determinant B)}$\:---
\en{the~determinant}\ru{определитель}
\en{of~matrix product}\ru{матричного произведения}
\en{is equal}\ru{равен}
\en{to the~product}\ru{произведению}
\en{of~determinants}\ru{определителей}~\eqref{determinantofmatrixproduct}.
%
\en{In~the~product matrix}\ru{В~матрице-произведении}\en{,}
\en{the~item}\ru{элемент}~${\left[{\,\cdots\,}\right]}_{1\hspace{-0.1ex}1}$
\en{equals}\ru{равен}~${\delta_{is} \delta_{ps} \hspace{-0.2ex} = \delta_{ip}\hspace{.1ex}}$,
\en{just like}\ru{совсем как}
\en{on}\ru{в}~\en{the~right side}\ru{правой части}\en{ of}~\eqref{theproductoftwopermutations}\,;
\en{same}\ru{так~же}
\en{for all}\ru{и~для всех}
\en{the~other items}\ru{других элементов}.
${ \hspace{.6em}
\tikz[baseline=-0.6ex] \draw [color=black, fill=black] (0, 0) circle (.8ex); }$

\en{The contraction of}\ru{Свёртка}~\eqref{theproductoftwopermutations}
\en{leads to}\ru{ведёт к}~%
\en{the~useful}\ru{полезным}
\en{formulas}\ru{формулам}

\nopagebreak\vspace{-0.2em}
\begin{multline*}
e_{i\hspace{-0.1ex}j\hspace{-0.1ex}k} \hspace{.1ex} e_{pqk} \hspace{-0.1ex} = \hspace{.2ex}
\determinant \hspace{-0.25ex} \left[
\begin{array}{ccc}
\delta_{ip} & \delta_{iq} & \delta_{ik} \\
\delta_{\hspace{-0.1ex}j\hspace{-0.1ex}p} & \delta_{\hspace{-0.1ex}j\hspace{-0.1ex}q} & \delta_{\hspace{-0.1ex}j\hspace{-0.1ex}k} \\
\delta_{kp} & \delta_{kq} & \delta_{kk}
\end{array}
\right] \hspace{-0.5ex} = \hspace{.2ex}
\determinant \hspace{-0.25ex} \left[
\begin{array}{ccc}
\delta_{ip} & \delta_{iq} & \delta_{ik} \\
\delta_{\hspace{-0.1ex}j\hspace{-0.1ex}p} & \delta_{\hspace{-0.1ex}j\hspace{-0.1ex}q} & \delta_{\hspace{-0.1ex}j\hspace{-0.1ex}k} \\
\delta_{kp} & \delta_{kq} & 3
\end{array}
\right] %% \hspace{-0.5ex} =
\\[.2em]
%
= \hspace{.1ex} 3 \hspace{.2ex} \delta_{ip} \delta_{\hspace{-0.1ex}j\hspace{-0.1ex}q} \hspace{-0.1ex}
+ \delta_{iq} \delta_{\hspace{-0.1ex}j\hspace{-0.1ex}k} \delta_{kp} \hspace{-0.1ex}
+ \delta_{ik} \delta_{\hspace{-0.1ex}j\hspace{-0.1ex}p} \delta_{kq} \hspace{-0.1ex}
- \delta_{ik} \delta_{\hspace{-0.1ex}j\hspace{-0.1ex}q} \delta_{kp} \hspace{-0.1ex}
- \hspace{.1ex} 3 \hspace{.2ex} \delta_{iq} \delta_{\hspace{-0.1ex}j\hspace{-0.1ex}p} \hspace{-0.1ex}
- \delta_{ip} \delta_{\hspace{-0.1ex}j\hspace{-0.1ex}k} \delta_{kq} %% \hspace{-0.1ex} =
\\
%
= \hspace{.1ex} 3 \hspace{.2ex} \delta_{ip} \delta_{\hspace{-0.1ex}j\hspace{-0.1ex}q} \hspace{-0.1ex}
+ \delta_{iq} \delta_{\hspace{-0.1ex}j\hspace{-0.1ex}p} \hspace{-0.1ex}
+ \delta_{iq} \delta_{\hspace{-0.1ex}j\hspace{-0.1ex}p} \hspace{-0.1ex}
- \delta_{ip} \delta_{\hspace{-0.1ex}j\hspace{-0.1ex}q} \hspace{-0.1ex}
- \hspace{.1ex} 3 \hspace{.2ex} \delta_{iq} \delta_{\hspace{-0.1ex}j\hspace{-0.1ex}p} \hspace{-0.1ex}
- \delta_{ip} \delta_{\hspace{-0.1ex}j\hspace{-0.1ex}q} %% \hspace{-0.1ex} =
\\
%
= \delta_{ip} \delta_{\hspace{-0.1ex}j\hspace{-0.1ex}q} \hspace{-0.1ex}
- \hspace{.1ex} \delta_{iq} \delta_{\hspace{-0.1ex}j\hspace{-0.1ex}p}
\hspace{.2ex} ,
\end{multline*}

\nopagebreak\vspace{-0.8em}
\begin{equation*}
e_{i\hspace{-0.1ex}j\hspace{-0.1ex}k} \hspace{.1ex} e_{pj\hspace{-0.1ex}k} \hspace{-0.15ex}
= \delta_{ip} \delta_{\hspace{-0.1ex}j\hspace{-0.12ex}j} \hspace{-0.2ex} - \hspace{.1ex} \delta_{i\hspace{-0.1ex}j} \delta_{\hspace{-0.1ex}j\hspace{-0.1ex}p} \hspace{-0.15ex}
= \hspace{.1ex} 3 \hspace{.2ex} \delta_{ip} \hspace{-0.2ex} - \delta_{ip} \hspace{-0.15ex}
= \hspace{.1ex} 2 \hspace{.1ex} \delta_{ip}
\hspace{.2ex} ,
\end{equation*}

\nopagebreak\vspace{-0.8em}
\begin{equation*}
e_{i\hspace{-0.1ex}j\hspace{-0.1ex}k} \hspace{.1ex} e_{i\hspace{-0.1ex}j\hspace{-0.1ex}k} \hspace{-0.15ex}
= \hspace{.1ex} 2 \hspace{.2ex} \delta_{ii} \hspace{-0.15ex}
= \hspace{.1ex} 6
%%\hspace{.2ex} .
\end{equation*}

\vspace{-0.4em}\noindent
\en{or}\ru{или}
\en{in short}\ru{вкратце}

\nopagebreak\vspace{-0.3em}
\begin{equation}\label{doublepermutationscontracted}
e_{i\hspace{-0.1ex}j\hspace{-0.1ex}k} \hspace{.1ex} e_{pqk} \hspace{-0.25ex} = \delta_{ip} \delta_{\hspace{-0.1ex}j\hspace{-0.1ex}q} \hspace{-0.25ex} - \hspace{.1ex} \delta_{iq} \delta_{\hspace{-0.1ex}j\hspace{-0.1ex}p}
\hspace{.2ex} ,
\hspace{.5em}
%
e_{i\hspace{-0.1ex}j\hspace{-0.1ex}k} \hspace{.1ex} e_{pj\hspace{-0.1ex}k} \hspace{-0.25ex} = 2 \hspace{.2ex} \delta_{ip}
\hspace{.2ex} ,
\hspace{.5em}
%
e_{i\hspace{-0.1ex}j\hspace{-0.1ex}k} \hspace{.1ex} e_{i\hspace{-0.1ex}j\hspace{-0.1ex}k} \hspace{-0.25ex} = 6
\hspace{.2ex} .
\end{equation}

....

${
\determinantby{i}{\hspace{.15ex}j} \hspace{.4ex} \delta_{i\hspace{-0.1ex}j} \hspace{-0.15ex} = 3
}$

.....

\en{The~determinant}\ru{Определитель}
\en{of~components}\ru{компонент}
\en{of~a~bivalent tensor}\ru{бивалентного тензора}
\en{is invariant}\ru{инвариантен},
\en{it}\ru{он}
\en{doesn’t change}\ru{не~меняется}
\en{with a~rotation}\ru{с~поворотом}
\en{of~the~basis}\ru{базиса}

\nopagebreak\vspace{-0.2em}
\begin{equation*}
A'_{i\hspace{-0.1ex}j} \hspace{-0.16ex} = \cosinematrix{i'\hspace{-0.1ex}m} \hspace{.1ex} \cosinematrix{j'\hspace{-0.1ex}n} \hspace{.16ex} A_{mn}
\end{equation*}

....

