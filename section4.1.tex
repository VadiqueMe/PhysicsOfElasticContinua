\en{\section{The complete set of equations}}

\ru{\section{Полный набор уравнений}}

\label{section:wholesetofequations.lineartheory}

\en{\dropcap{E}{quations}}\ru{\dropcap{У}{равнения}}
\en{of the nonlinear elasticity}\ru{нелинейной упругости},
\en{even in their simplest cases}\ru{даже в~самых простых случаях},
\en{lead to the mathematically complex problems}\ru{приводят к~математически сложным задачам}.
\en{Therefore}\ru{Поэтому}
\en{the~linear theory}\ru{линейная теория}
\en{of~infinitesimal displacements}\ru{бесконечно м\'{а}лых смещений}
\en{is applied everywhere}\ru{повсеместно применяется}.
\en{This theory’s equations}\ru{Уравнения этой теории}
\en{were derived}\ru{были выведены}
\en{in the~first half}\ru{в~первой половине}
\en{of~the~}\hbox{XIX$^{\textrm{\en{th}\ru{го}}}$\hspace{-0.2ex}}~\en{century}\ru{века}\en{ by}
\href{https://en.wikipedia.org/wiki/Augustin-Louis_Cauchy}{Cauchy},
\href{https://en.wikipedia.org/wiki/Claude-Louis_Navier}{Navier},
\href{https://en.wikipedia.org/wiki/Gabriel_Lam\%C3\%A9}{Lam\'{e}},
\href{https://en.wikipedia.org/wiki/Beno\%C3\%AEt_Paul_\%C3\%89mile_Clapeyron}{Clapeyron}\ru{’ом},
\href{https://en.wikipedia.org/wiki/Sim\%C3\%A9on_Denis_Poisson}{Poisson}\ru{’ом},
\href{https://en.wikipedia.org/wiki/Adh\%C3\%A9mar_Jean_Claude_Barr\%C3\%A9_de_Saint-Venant}{Saint\hbox{-\hspace{-0.2ex}}Venant}\ru{’ом},
\href{https://en.wikipedia.org/wiki/George_Green_(mathematician)}{George Green}\ru{’ом}
\en{and the other scientists}\ru{и~другими учёными}.

\en{The~complete closed set}\ru{Полный замкнутый набор}
\en{of~equations}\ru{уравнений}
\en{of the classical linear theory}\ru{классической линейной теории}
\en{in~the~direct invariant tensor notation}\ru{в~прямой инвариантной тензорной з\'{а}писи},
\en{including}\ru{включающий}
\begin{itemize}
\item \en{the~balance of~forces}\ru{баланс сил}~(\en{of~momentum}\ru{импульса},
\en{of~}\ru{количества движения, }\emph{vis~viva}),
\item \ru{соотношения}\en{the} \en{stress}\ru{напряжение}--\en{strain}\ru{деформация}\en{ relations}
\en{for a~material}\ru{для материала},
\item ${\bm{u} \mapsto \infinitesimaldeformation}$,
\end{itemize}

\noindent
\en{is}\ru{есть}

\nopagebreak\vspace{-0.8em}
%%%\hspace*{-\parindent}\begin{minipage}{\linewidth}
\begin{equation}
\label{lineartheory:wholesetofequations}
\boldnabla \dotp \linearstress \hspace{.15ex} + \hspace{.15ex} \bm{v}
\hspace{.1ex} = \hspace{.1ex}
\bm{0}
\hspace{.1ex} ,
\hspace{1em}
%
\linearstress = \scalebox{.92}{$
   \displaystyle
   \frac{ \raisemath{-0.133em}{
      \partial \hspace{.1ex} \potentialenergydensity^{\mathstrut}
   }%close \raisemath
   }%close 1st argument of \frac
   { \raisemath{-0.07em}{\partial \infinitesimaldeformation} }
$}
=
\stiffnesstensor
\hspace{-0.1ex} \dotdotp
\infinitesimaldeformation
\hspace{.1ex} ,
\hspace{1em}
%
\infinitesimaldeformation
= \hspace{-0.2ex}
\boldnabla {\bm{u}}^{\hspace{.1ex}\mathsf{S}}
\hspace{-0.2ex} .
\end{equation}

\nopagebreak\noindent
\en{Here}\ru{Здесь}
$\linearstress$~\ru{это}\en{is}
\en{the~linear stress tensor}\ru{линейный тензор напряжений},
$\bm{v}$~\en{is}\ru{это}
\en{the~resultant vector}\ru{результирующий вектор}
\en{of volume loads}\ru{объёмных нагрузок},
$\infinitesimaldeformation$~\ru{это}\en{is}
\en{the~tensor}\ru{тензор}
 \en{of infinitesimal relative deformation~(strain)}\ru{бесконечномалой относительной деформации},
${%
   \potentialenergydensity (
      \hspace{-0.1ex}\infinitesimaldeformation\hspace{-0.1ex}
   )\hspace{-0.1ex}
}$~\ru{это}\en{is}
\en{the potential energy}\ru{потенциальная энергия}
\en{of deformation}\ru{деформации}
\en{per volume unit}\ru{на единицу объёма}
\en{and}\ru{и}~$\stiffnesstensor$~\ru{это}\en{is}
\en{the stiffness tensor}\ru{тензор жёсткости}.
\en{The~latter}\ru{Последний}
\en{is tetravalent}\ru{четырёхвалентен}
\en{with the following symmetry}\ru{со~следующей симметрией}
\begin{equation*}
   \stiffnesstensor_{\hspace{.12ex} \indexjuggling{12}{34}}
   \hspace{-0.4ex} = \hspace{-0.1ex}
   \stiffnesstensor
\hspace{.2ex} , \hspace{.8em}
   \stiffnesstensor_{\hspace{.12ex} \indexjuggling{1}{2}}
   \hspace{-0.33ex} = \hspace{-0.1ex}
   \stiffnesstensor
\hspace{.2ex} , \hspace{.8em}
   \stiffnesstensor_{\hspace{.12ex} \indexjuggling{3}{4}}
   \hspace{-0.33ex} = \hspace{-0.1ex}
   \stiffnesstensor
\hspace{.2ex} .
\end{equation*}

\noindent
\en{But}\ru{Но}
\en{where does}\ru{откуда}
\en{this set}\ru{этот набор}~(\en{system}\ru{система})
\en{of~equations}\ru{уравнений}
\en{follow from}\ru{следует}?

\en{The equations}\ru{Уравнения}~\eqref{lineartheory:wholesetofequations}
\en{are exact}\ru{точные},
\en{they}\ru{они}
\en{can be derived}\ru{могут быть получены}
\en{by varying}\ru{варьированием}
\en{the~equations}\ru{уравнений}
\en{of the~nonlinear theory}\ru{нелинейной теории}.
\en{Varying}\ru{Варьирование}
\en{from an~arbitrary configuration}\ru{от~произвольной конфигурации}
\en{is~described in}\ru{описано в}~\chapterdotsectionref{chapter:nonlinearcontinuum}{section:variationofconfiguration}.
\en{The~linear theory}\ru{Линейная теория}
\ru{это}\en{is}
\en{the~result of varying}\ru{результат варьирования}
\en{from the~initial unstressed configuration}\ru{от~начальной ненапряжённой конфигурации},
\en{when}\ru{когда}

\nopagebreak
\begin{equation}\label{variationfrominitialconfiguration}
\begin{array}{c}
\bm{F} = \UnitDyad \hspace{.1ex} ,
\:\;
\bm{C} \hspace{-0.1ex} = {\hspace{-0.2ex}^2\bm{0}}
\hspace{.1ex} ,
\:\;
\variation{\hspace{.12ex}\bm{C}} \hspace{-0.1ex} = \hspace{-0.2ex} \boldnabla \hspace{.12ex} \variation{\currentlocationvector}^{\hspace{.2ex}\mathsf{S}}
%%\hspace{-0.15ex} \equiv \infinimentpetitdeformationdevariation
\hspace{-0.1ex} ,
\\[.2em]
%
\cauchystress = {\hspace{-0.2ex}^2\bm{0}}
\hspace{.1ex} ,
\:\;
\varbivalent{\hspace{-0.2ex}\cauchystress}
= \variation{\hspace{.1ex}\firstpiolakirchhoffstress}
= \scalebox{.9}{$
\displaystyle \frac{\raisemath{-0.125em}{ \partial^2 \hspace{.1ex} \potentialenergydensity }}{\raisemath{-0.1em}{\partial \hspace{.1ex} \bm{C} \hspace{.1ex} \partial \hspace{.1ex} \bm{C}}} $}
\hspace{-0.1ex} \dotdotp
\variation{\hspace{.12ex}\bm{C}}
\hspace{.1ex} ,
\:\:
\boldnabla \dotp \varbivalent{\hspace{-0.2ex}\cauchystress}
\hspace{.1ex} + \hspace{.1ex}
\rho \hspace{.25ex} \variation{\hspace{-0.2ex}\bm{f}}
= \hspace{.1ex}
\bm{0}
\hspace{.11ex} .
\end{array}
\end{equation}

\vspace{-0.1em}\noindent
\en{It remains to~change}\ru{Остаётся поменять}
\begin{itemize}
\item
   ${\variation{\currentlocationvector}}$
   \en{to}\ru{на}
   $\bm{u}$,
\item
   ${\variation{\hspace{.12ex}\bm{C}}}$
   \en{to}\ru{на}
   $\infinitesimaldeformation$,
\item
   $\varbivalent{\hspace{-0.15ex}\cauchystress}$
   \en{to}\ru{на}
   $\linearstress$,
\item ${%
   \scalebox{.95}{$%
      \raisemath{.15em}{ \scalebox{.92}{$ \partial^2 \hspace{.1ex} \potentialenergydensity $} } $}
   /
   \hspace{-0.1ex}
   \raisemath{-0.3em}{\scalebox{.92}{$
      \partial \hspace{.1ex} \bm{C}
      \hspace{.1ex}
      \partial \hspace{.1ex} \bm{C}
   $}} }$
   \en{to}\ru{на}
   $\stiffnesstensor$,
\item
   ${ \hspace{-0.1ex} \rho \hspace{.25ex}
   \variation{\hspace{-0.2ex}\bm{f}} }$
   \en{to}\ru{на}
   ${ \hspace{-0.1ex} \bm{v} \hspace{-0.2ex} }$.
\end{itemize}

\en{If}\ru{Если}
\en{the derivation of}\ru{вывод}~\eqref{variationfrominitialconfiguration}
\en{seems abstruse to the~reader}\ru{кажется читателю малопонятным},
\en{it’s possible}\ru{возможно}
\en{to proceed}\ru{исходить}
\en{from}\ru{из}
\en{the~following equations}\ru{следующих уравнений}

\nopagebreak\vspace{-0.25em}
\begin{equation}\label{nonlinear:setofequations}
\begin{array}{c}
\boldnabla \dotp \cauchystress \hspace{.15ex} + \rho \bm{f} = \hspace{.1ex} \bm{0} \hspace{.1ex},
\:\:
\boldnabla = \bm{F}^{-\T} \hspace{-0.2ex} \dotp \boldnablacircled ,
\:\:
\bm{F} = \UnitDyad \hspace{.1ex} + \hspace{-0.2ex} \boldnablacircled {\bm{u}}^{\T} \hspace{-0.3ex},
\\[.2em]
%
\cauchystress \hspace{.1ex} = J^{-1} \hspace{.2ex} \bm{F} \dotp \scalebox{0.9}{$ \displaystyle \frac{\raisemath{-0.125em}{\partial\hspace{.1ex} \potentialenergydensity}}{\raisemath{-0.1em}{\partial\hspace{.1ex} \bm{C}}} $} \dotp \bm{F}^{\hspace{.1ex}\T} \hspace{-0.4ex},
\:\:
\bm{C} = \boldnablacircled {\bm{u}}^{\hspace{.1ex}\mathsf{S}} \hspace{-0.3ex} + \smalldisplaystyleonehalf \boldnablacircled \bm{u} \dotp \hspace{-0.25ex} \boldnablacircled \bm{u}^{\T} \hspace{-0.25ex}.
\end{array}
\end{equation}

\vspace{-0.1em}\noindent
\en{Assuming}\ru{Полагая}
\en{the displacement}\ru{смещение}~$\bm{u}$
\en{is small}\ru{м\'{а}лым}
(\en{infinitesimal}\ru{бесконечно м\'{а}лым}),
\en{we’ll move}\ru{мы перейдём}
\en{from}\ru{от}~\eqref{nonlinear:setofequations}
\en{to}\ru{к}~\eqref{lineartheory:wholesetofequations}.

\en{Or so}\ru{Или так}.
\en{Instead of}\ru{Вместо}~$\bm{u}$
\en{to take}\ru{взять}
\en{some}\ru{некоторый}
\en{small enough}\ru{достаточно малый}
\en{parameter}\ru{параметр} $\smallparameter \bm{u}$,
${\smallparameter \hspace{-0.1ex}\to 0}$.
\en{And to represent thereafter}\ru{И~представить после этого}
\en{the unknowns}\ru{неизвестные}
\en{by the series}\ru{рядами}
\en{in the integer exponents}\ru{в~целых показателях}
\en{of parameter}\ru{параметра}~$\smallparameter$

\nopagebreak\vspace{-0.1em}\begin{equation*}
\begin{array}{c}
\cauchystress \hspace{.1ex} = \cauchystress^{\hspace{.2ex}\scalebox{0.66}{(0)}} \hspace{-0.2ex} + \smallparameter \cauchystress^{\hspace{.2ex}\scalebox{0.66}{(1)}} \hspace{-0.1ex} + \ldots \hspace{.1ex},
\:\:
\bm{C} = \bm{C}^{\hspace{.2ex}\scalebox{0.66}{(0)}} \hspace{-0.2ex} + \smallparameter \bm{C}^{\hspace{.2ex}\scalebox{0.66}{(1)}} \hspace{-0.1ex} + \ldots \hspace{.1ex} ,
\\[.1em]
%
\boldnabla \hspace{.1ex} = \boldnablacircled \hspace{.1ex} + \smallparameter \boldnabla^{\hspace{.2ex}\scalebox{0.66}{(1)}} \hspace{-0.1ex} + \ldots \hspace{.1ex} , \:\:
\bm{F} = \UnitDyad + \hspace{-0.1ex} \smallparameter \boldnablacircled {\bm{u}}^{\T} \hspace{-0.3ex},
\:\:
J = 1 + \hspace{-0.1ex} \smallparameter J^{\hspace{.1ex}\scalebox{0.66}{(1)}} \hspace{-0.1ex} + \ldots
\end{array}
\end{equation*}

\vspace{-0.1em}\noindent
\en{The complete set of equations}\ru{Полный набор уравнений}~\eqref{lineartheory:wholesetofequations}
\en{comes}\ru{выходит}
\en{from the first}\ru{из первых}~(\en{zeroth}\ru{нулевых})
\en{terms}\ru{членов}
\en{of these series}\ru{этих рядов}.
\en{In}\ru{В}~\en{the~book}\ru{книге}~\cite{truesdell-firstcourse}
\en{this is called}\ru{это названо}
\inquotesx{\en{formal approximation}\ru{формальным приближением}}[.]

\en{It is impossible}\ru{Невозможно}
\en{to tell}\ru{сказать}
\en{unambiguously}\ru{однозначно}
\en{how small}\ru{насколько малым}
\en{the parameter}\ru{параметр}~$\smallparameter$
\en{should be}\ru{должен быть}\:---
\en{the answer}\ru{ответ}
\en{depends on the situation}\ru{зависит от~ситуации}
\en{and}\ru{и}
\en{is determined}\ru{определяется}
\en{by whether the linear model describes}\ru{тем, описывает ли линейная модель}
\en{the effect we are interested in}\ru{интересующий нас эффект}
\en{or not}\ru{или нет}.
\en{When}\ru{Когда},
\en{as example}\ru{как пример},
\en{I’m interested in}\ru{меня интересует}
\en{a~relation}\ru{связь}
\en{between}\ru{между}
\en{the~frequency}\ru{частотой}
\en{of a~freely vibrating}\ru{свободно вибрирующего}
\en{motion}\ru{движения}
\en{after}\ru{после}
\en{the initial displacement}\ru{начального смещения},
\en{then}\ru{то}
\en{a~nonlinear model is needed}\ru{нужна нелинейная модель}.

\en{A~linear problem}\ru{Линейная задача} \en{is posed}\ru{ставится}
\en{in the~initial volume}\ru{в~начальном объёме}
{$ \mathcal{V} \hspace{-0.2ex} = \hspace{-0.2ex} \mathcircabove{\mathcal{V}} $}\hspace{-0.12em},
\en{bounded by the surface}\ru{ограниченном поверхностью}~$o$
\en{with the area vector}\ru{с~вектором пл\'{о}щади}~${\unitnormalvector do}$
(\inquotes{\en{the~principle}\ru{принцип}
\en{of initial dimensions}\ru{начальных размеров}}).

\en{The boundary}\ru{Краевые~(граничные)} \en{conditions}\ru{условия}
\en{most often}\ru{чаще всего}
\en{are}\ru{такие}:
\en{on}\ru{на}
\hbox{\en{the part}\ru{части}}~${o_1}$
\en{of~the~surface}\ru{поверхности}
\ru{известны }\en{displacements}\ru{смещения}\en{ are known},
\en{and}\ru{а}
\en{on}\ru{на}
\en{another}\ru{другой}
\hbox{\en{part}\ru{части}}~${o_2}$
\ru{известны }\en{the forces}\ru{силы}\en{ are known}.

\nopagebreak\vspace{-0.2em}
\begin{equation}\label{lineartheory:boundaryconditions}
\bm{u} \hspace{.1ex} \bigr|_{o_1}
\hspace{-0.64ex} = \hspace{.2ex}
\bm{u}_{\raisemath{-0.1em}{0}}
\hspace{.2ex} ,
\hspace{.8em}
\unitnormalvector \dotp \linearstress \hspace{.2ex} \bigr|_{o_2}
\hspace{-0.64ex} = \hspace{.2ex}
\bm{p}
\hspace{.2ex} .
\end{equation}

\en{The more complex combinations happen too}\ru{Бывают и~более сложные комбинации},
\en{if we know}\ru{если мы знаем}
\en{the certain}\ru{некоторые}
\en{components}\ru{компоненты}
\en{of the both}\ru{как}~$\bm{u}$\ru{,} \en{and}\ru{так~и}~${ \tractionvector{n} \hspace{-0.2ex} = \unitnormalvector \dotp \linearstress }$
\en{simultaneously}\ru{одновременно}.
\en{For example}\ru{Для примера},
\en{on a~flat face}\ru{на~плоской грани}~${x = \constant}$
\en{when pressing a~stamp}\ru{при~вдавливании штампа}
\en{with a~smooth surface}\ru{с~гладкой поверхностью}
${u_x \hspace{-0.25ex} = \nu \hspace{.1ex} (y , \hspace{-0.1ex} z)}$,
${\mathtau_{xy} \hspace{-0.2ex} = \mathtau_{xz} \hspace{-0.2ex} = 0}$
(\en{the function}\ru{функция}~$\nu$ \en{is determined}\ru{определяется} \en{by the stamp’s shape}\ru{формой штампа}).

\en{For}\ru{Для}
\en{the dynamic problems}\ru{динамических проблем}
\en{we have}\ru{мы имеем}
${
   \bm{f}
   \hspace{-0.1ex} - \hspace{-0.1ex}
   \rho \hspace{.2ex} \mathdotdotabove{\bm{u}}
}$  
\en{instead of just}\ru{вместо просто}~$\bm{f}$.
\en{And}\ru{А}
\en{the initial conditions}\ru{начальные условия}
\en{for}\ru{для}
\en{the dynamic problems}\ru{динамических задач}
\en{are set}\ru{ставятся}
\en{as it’s common}\ru{как обычно}
\en{in~mechanics}\ru{в~механике}\:---
\en{on the positions}\ru{на положения}
\en{and}\ru{и}
\en{on the velocities}\ru{на скорости}:
\en{at the given moment}\ru{в~данный момент}
\en{of~time}\ru{времени}~${t \narroweq \hspace{.1ex} 0}$
\ru{известны }$\bm{u}$ \en{and}\ru{и}~$\mathdotabove{\bm{u}}$\en{ are known}.
%
\en{The linearity of the problems}\ru{Линейность задач} 

\en{The linearity}\ru{Линейность}
\en{gives}\ru{даёт}
\en{the principle}\ru{принцип}
\en{of superposition}\ru{суперпозиции}
\en{(or independence)}\ru{(или независимости)}
\en{of the action}\ru{действия}
\en{of loads}\ru{нагрузок}.
\en{When}\ru{Когда}
\en{there are several loads}\ru{нагрузок несколько},
\en{the problem}\ru{проблема}
\en{can be solved}\ru{может быть решена}
\en{for}\ru{для}
\en{the each load}\ru{для каждой нагрузки}
\en{separately}\ru{отдельно}.
\en{And then}\ru{И~тогда}
\en{the complete}\ru{полное}
\en{solution}\ru{решение}
\en{can be}\ru{может быть}
\en{obtained}\ru{получено}
\en{by the summation}\ru{суммированием}.
\en{For}\ru{Для}
\en{statics}\ru{статики}
\en{this}\ru{это}
\en{means}\ru{значит},
\en{for example}\ru{например},
\en{the following}\ru{следующее}:
\en{if}\ru{если}
\en{external loads}\ru{внешние нагрузки}~${\bm{f}}$ \en{and}\ru{и}~${\bm{p}}$
\en{increase}\ru{растут}
\en{by}\ru{в}~$m$
\en{times}\ru{раз}
(\en{body}\ru{тело}
\en{is fixed on}\ru{закреплено на}~${o_1}$),
\en{then}\ru{то}
$\bm{u}$,
$\infinitesimaldeformation$
\en{and}\ru{и}~$\linearstress$
\en{will increase}\ru{вырастут}
\ru{тоже }\en{by}\ru{в}~$m$ \en{times}\ru{раз}\en{ too}.
\en{Potential energy density}\ru{Плотность потенциальной энергии}~$\potentialenergydensity$
\en{will increase by}\ru{вырастет в}~$m^2$ \en{times}\ru{раз}.
\en{In~reality}\ru{В~реальности}
\en{such}\ru{такое}
\en{is observed}\ru{наблюдается}
\en{only}\ru{лишь}
\en{when}\ru{когда}
\en{the loads}\ru{нагрузки}
\en{are small}\ru{малые}.

\en{The density}\ru{Плотность}
\en{of the potential energy}\ru{потенциальной энергии}
\en{of the elastic deformation}\ru{упругой деформации}~$\potentialenergydensity$

\nopagebreak\vspace{-0.1em}
\begin{equation*}
\potentialenergydensity ( \hspace{-0.1ex}
\infinitesimaldeformation
\hspace{-0.1ex} ) \hspace{-0.2ex}
= \hspace{.1ex}
\smash{\smalldisplaystyleonehalf} \hspace{.2ex}
\infinitesimaldeformation
\hspace{-0.1ex} \dotdotp \hspace{-0.1ex}
\stiffnesstensor
\dotdotp \hspace{-0.1ex}
\infinitesimaldeformation
\end{equation*}

\vspace{-0.4em}\noindent
\en{and}\ru{и}
\en{its variation}\ru{её вариация}

\nopagebreak\vspace{.3em}\begin{equation*}
\begin{gathered}
\variation{\potentialenergydensity}
= \hspace{.1ex}
\smash{\smalldisplaystyleonehalf} \hspace{.25ex}
\variation{ \bigl(
   \infinitesimaldeformation
   \hspace{-0.1ex} \dotdotp \hspace{-0.1ex}
   \stiffnesstensor
   \dotdotp \hspace{-0.1ex}
   \infinitesimaldeformation
\bigr)
} \hspace{-0.25ex}
= \hspace{.1ex}
\smash{\smalldisplaystyleonehalf}
\hspace{-0.1ex} \bigl(
\scalebox{.95}{$
   \variation{\infinitesimaldeformation}
   \hspace{-0.1ex} \dotdotp \hspace{-0.1ex}
   \stiffnesstensor
   \dotdotp \hspace{-0.1ex}
   \infinitesimaldeformation
   +
   \infinitesimaldeformation
   \hspace{-0.1ex} \dotdotp \hspace{-0.1ex}
   \stiffnesstensor
   \dotdotp
   \variation{\infinitesimaldeformation}
$}
\bigr)
\hspace{-0.25ex} =
\tikzmark{beginLinearStressAsDeformationAndStiffness}
\infinitesimaldeformation
\hspace{-0.1ex} \dotdotp \hspace{-0.1ex}
\stiffnesstensor
\tikzmark{endLinearStressAsDeformationAndStiffness}
\dotdotp
\variation{\infinitesimaldeformation}
\\[.3em]
%
\variation{\hspace{.1ex}\potentialenergydensity}
( \hspace{-0.1ex}
    \infinitesimaldeformation
\hspace{-0.1ex} ) \hspace{-0.2ex}
= \scalebox{.9}{$
    \displaystyle\frac{
        \raisemath{-0.16em}
        {\partial \hspace{.1ex} \potentialenergydensity}}
        {\partial \infinitesimaldeformation}
    $} \hspace{-0.1ex}
\dotdotp
\variation{\infinitesimaldeformation}
= \linearstress
\dotdotp
\variation{\infinitesimaldeformation}
= \infinitesimaldeformation
\hspace{-0.1ex} \dotdotp \hspace{-0.1ex}
\stiffnesstensor
\dotdotp
\variation{\infinitesimaldeformation}
\\
%
\secondvariation{\hspace{.1ex}\potentialenergydensity}(\hspace{-0.1ex}
\infinitesimaldeformation
\hspace{-0.1ex}) \hspace{-0.2ex}
= \variation{\infinitesimaldeformation}
\hspace{-0.1ex} \dotdotp
\scalebox{.9}{$ \displaystyle\frac{\raisemath{-0.16em}{\partial^2 \hspace{.1ex} \potentialenergydensity}}{ \partial \infinitesimaldeformation \hspace{.1ex} \partial \infinitesimaldeformation } $}
\hspace{-0.1ex} \dotdotp
\variation{\infinitesimaldeformation}
= \variation{\infinitesimaldeformation}
\hspace{-0.1ex} \dotdotp
\stiffnesstensor
\dotdotp
\variation{\infinitesimaldeformation}
= 2 \potentialenergydensity (\variation{\infinitesimaldeformation})
\end{gathered}
\end{equation*}%
\AddUnderBrace[line width=.75pt][0, .1ex][yshift = .2ex]%
{beginLinearStressAsDeformationAndStiffness}{endLinearStressAsDeformationAndStiffness}%
{${
   \scalebox{0.8}{$ \linearstress $}
}$}
\vspace{-0.5em}

\en{As was noted}\ru{Как отмечалось}
\en{in}\ru{в}~\chapterref{chapter:genericmechanics},
\en{the principle}\ru{принцип}
\en{of the virtual work}\ru{виртуальной работы}
(\ru{принцип }\en{the~}\hbox{d’\hspace{-0.2ex}Alembert\ru{’а}--Lagrange\ru{’а}}\en{ principle})
\en{can be put}\ru{может быть положен}
\en{into the foundation}\ru{в~основу}
\en{of mechanics}\ru{механики}.
\en{This principle}\ru{Этот принцип}
\en{is true}\ru{справедлив}
\en{for the linear theory too}\ru{и~для линейной теории}
(\en{the internal forces}\ru{внутренние силы}
\en{in an~elastic medium}\ru{в~упругой среде}
\en{are potential}\ru{потенциальны}
${ \variation{\internalwork} = - \hspace{.2ex} \variation{\potentialenergydensity} }$)

\nopagebreak\vspace{-0.2em}
\begin{equation}\label{lineartheory:principleofvirtualwork}
\displaystyle\integral\displaylimits_{\mathcal{V}}
\hspace{-0.2ex} \Bigl(
    \left(
        \bm{f}
        \hspace{-0.1ex} - \hspace{-0.1ex}
        \rho \hspace{.1ex}
        \mathdotdotabove{\bm{u}}
        \hspace{.2ex}
    \right)
    \hspace{-0.1ex}
    \dotp
    \variation{\bm{u}} - \variation{\potentialenergydensity}
\hspace{.1ex} \Bigr) d\mathcal{V}
+ \hspace{-0.3ex}
\displaystyle\integral\displaylimits_{o_2} \hspace{-0.4ex} \bm{p} \dotp \variation{\bm{u}} \hspace{.25ex} do \hspace{.1ex}
= \hspace{.1ex} 0
\hspace{.1ex} ,
\hspace{.6em}
%
\bm{u} \hspace{.1ex}
\bigr|_{o_1}
\hspace{-0.64ex} = \hspace{.1ex}
\bm{0}
\hspace{.1ex} ,
\vspace{-0.2em}
\end{equation}

\noindent\vspace{-0.2em}
\en{because}\ru{потому что}

\nopagebreak\vspace{-0.2em}\begin{equation*}
\begin{gathered}
\variation{\potentialenergydensity}
= \linearstress \dotdotp \variation{\infinitesimaldeformation}
= \linearstress \dotdotp \hspace{-0.12ex} \boldnabla \variation{\bm{u}}^{\hspace{.1ex}\mathsf{S}} \hspace{-0.16ex}
= \boldnabla \hspace{-0.1ex} \dotp \left( \hspace{.1ex} \linearstress \hspace{-0.1ex} \dotp \variation{\bm{u}} \hspace{.1ex} \right) - \boldnabla \dotp \linearstress \dotp \variation{\bm{u}}
\hspace{.1ex} ,
\\[.1em]
%
\displaystyle \integral\displaylimits_{\mathcal{V}} \hspace{-0.5ex} \variation{\potentialenergydensity} \hspace{.2ex} d\mathcal{V} =
\ointegral\displaylimits_{\mathclap{o\hspace{.1ex}(\boundary \mathcal{V})}} \hspace{-0.2ex} \unitnormalvector \hspace{.1ex} \dotp \linearstress \dotp \variation{\bm{u}} \hspace{.32ex} do \hspace{.2ex} - \hspace{-0.1ex}
\integral\displaylimits_{\mathcal{V}} \hspace{-0.5ex} \boldnabla \dotp \linearstress \dotp \variation{\bm{u}} \hspace{.32ex} d\mathcal{V}
\end{gathered}
\end{equation*}

\vspace{-0.2em}\noindent
\en{and}\ru{и} \en{the left part of}\ru{левая часть}~\eqref{lineartheory:principleofvirtualwork}
\en{becomes}\ru{становится}
\begin{equation*}
\displaystyle\integral\displaylimits_{\mathcal{V}} \hspace{-0.4ex} \Bigl( \hspace{-0.1ex} \boldnabla \dotp \linearstress \hspace{.1ex} + \bm{f} \hspace{-0.1ex} - \hspace{-0.1ex} \rho \hspace{.1ex} \mathdotdotabove{\bm{u}} \hspace{.1ex} \Bigr) \hspace{-0.3ex} \dotp \variation{\bm{u}} \hspace{.25ex} d\mathcal{V}
+ \hspace{-0.3ex}
\displaystyle\integral\displaylimits_{o_2} \hspace{-0.4ex} \Bigl( \bm{p} - \unitnormalvector \dotp \linearstress \Bigr) \hspace{-0.3ex} \dotp \variation{\bm{u}} \hspace{.25ex} do
\hspace{.1ex} ,
\end{equation*}

\vspace{-0.4em}\noindent
\en{that}\ru{что}
\en{is equal to zero}\ru{равно нулю}.
\en{Notice}\ru{Отметим}
\en{the boundary condition}\ru{краевое условие}
${\bm{u} \hspace{.1ex} \bigr|_{o_1} \hspace{-0.64ex} = \hspace{.2ex} \bm{0}}$:
\en{the virtual displacements}\ru{виртуальные смещения}
\en{are compatible}\ru{совместимы}
\en{with this constraint}\ru{с~этой связью}
${ \variation{\bm{u}} \hspace{.1ex} \bigr|_{o_1} \hspace{-0.64ex} = \hspace{.2ex} \bm{0} }$.

