\usepackage[utf8]{inputenc}

\usepackage[T2A,LGR,OT1,T1]{fontenc}
\usepackage{ucs}

\usepackage[english, greek, russian]{babel}

\RequirePackage{ifthen}
\newif\ifen
\newif\ifru
\newcommand{\en}[1]{\ifen#1\fi}
\newcommand{\ru}[1]{\ifru#1\fi}

\PrerenderUnicode{æ}

\usepackage{newunicodechar}
\newunicodechar{×}{\ensuremath{\times}}

\usepackage{multicol}

% no more "Too many math alphabets used in version normal"
%%\newcommand\hmmax{0}
%%\newcommand\bmmax{0}

\usepackage{gensymb}

\usepackage{xargs}
\usepackage{xparse}

\usepackage{soul} % for \st that's strikethrough

\usepackage{comfortaa}
%%\usepackage{calligra}
\usepackage{frcursive}

\usepackage[usenames,dvipsnames]{xcolor}

\definecolor{arsenic}{rgb}{0.23, 0.27, 0.29}
\definecolor{charcoal}{rgb}{0.21, 0.27, 0.31}
\definecolor{chocolate}{rgb}{0.48, 0.25, 0.0}
\definecolor{darkcerulean}{rgb}{0.03, 0.27, 0.49}
\definecolor{darkmagenta}{rgb}{0.55, 0.0, 0.55}
\definecolor{darksienna}{rgb}{0.24, 0.08, 0.08}

\makeatletter
\def\mathcolor#1#{\@mathcolor{#1}}
\def\@mathcolor#1#2#3{%
	\protect\leavevmode
	\begingroup\color#1{#2}#3\endgroup
}
\makeatother

\makeatletter
\newcommand{\pushright}[1]{\ifmeasuring@#1\else\omit\hfill$\displaystyle#1$\fi\ignorespaces}
\newcommand{\pushleft}[1]{\ifmeasuring@#1\else\omit$\displaystyle#1$\hfill\fi\ignorespaces}
\makeatother

\usepackage[unicode=true, pdfusetitle,%
	bookmarks=true, bookmarksnumbered=false, bookmarksopen=true, linktocpage=true,%
	breaklinks=false, pdfborder={0 0 0}, backref=false,%
	colorlinks=true, linkcolor=darkcerulean, citecolor=darkmagenta, urlcolor=darksienna]%
{hyperref}

\usepackage{bookmark}

\usepackage{microtype}

\usepackage{lettrine}
\renewcommand{\LettrineFontHook}{\fontseries{b}}
\renewcommand{\LettrineTextFont}{\normalfont}

\newcommand\dropcap[2]{\lettrine[lines=2, findent=2pt, nindent=0pt]{#1}{#2}}

\usepackage{calc}
\usepackage{ulem}

\usepackage{xintexpr}

%%\usepackage{pdflscape}

\usepackage{layouts}

\usepackage{bigstrut}

\usepackage{wrapfig}

\usepackage{multirow}

\usepackage{longtable}

\newcommand\textbold[1]{{\fontfamily{cmr}\fontseries{b}\selectfont{}#1}}
\newcommand\textboldextended[1]{{\fontfamily{cmr}\fontseries{bx}\selectfont{}#1}}

%%\renewcommand{\bfdefault}{b} % make bold font narrower
%%\renewcommand{\bfdefault}{sbc}

\DeclareTextFontCommand{\textboldoblique}{\fontseries{bx}\itshape}

\usepackage{bm} % bold math symbols

\usepackage{amsmath}
\usepackage{amssymb}
\usepackage{amsfonts}

%%\AtBeginDocument{\let\boldmathbb\mathbb}
%%\usepackage{bbm}
%%\AtBeginDocument{\let\boldmathbb\mathbbm}
\usepackage{dsfont}
\AtBeginDocument{\let\boldmathbb\mathds}

\usepackage{nccmath}
\usepackage{mathtools}

\numberwithin{equation}{section}

\usepackage{wasysym}
\usepackage{mathrsfs}

\usepackage{mathdesign}

\usepackage{graphicx}

\DeclareFontFamily{U}{mymathpazo}{}
\DeclareFontShape{U}{mymathpazo}{m}{n}{<-> fplmri}{}
\DeclareFontShape{U}{mymathpazo}{b}{n}{<-> fplmbi}{}

\usepackage{pifont}
\makeatletter
\newcommand\Pimathsymbol[3][\mathord]{%
	#1{\@Pimathsymbol{#2}{#3}}}
\def\@Pimathsymbol#1#2{\mathchoice
	{\@Pim@thsymbol{#1}{#2}\tf@size}
	{\@Pim@thsymbol{#1}{#2}\tf@size}
	{\@Pim@thsymbol{#1}{#2}\sf@size}
	{\@Pim@thsymbol{#1}{#2}\ssf@size}}
\def\@Pim@thsymbol#1#2#3{%
	\mbox{\fontsize{#3}{#3}\Pisymbol{#1}{#2}}}
\makeatother
\newcommand{\varpartial}{\Pimathsymbol[\mathord]{mymathpazo}{182}}

\newcommand\englishlanguage[1]{\foreignlanguage{english}{#1}}
\newcommand\russianlanguage[1]{\foreignlanguage{russian}{#1}}

\newcommand\boundary{\hspace{.1ex}\varpartial}
%%\newcommand\boundary{\reflectbox{\textsf{6}}}

\newcommand\oneone{1\hspace{-0.1ex}1}

\newcommand\mathgray[1]{\mathcolor{gray}{#1}}

\newcommand\asmatrixwithoutdimensions[1]{\bigl[ \hspace{.2ex} #1 \hspace{.2ex} \bigr]}

\newcommand\asmatrixwithdimensions[3]{%
\underset%
{\raisemath{.1em}{\scriptscriptstyle{\ensuremath{#2 \times #3}}}}%
{\asmatrixwithoutdimensions{#1}}%
}

\DeclareMathOperator*{\integral}{\scalerel*{\rotatebox{12}{$\!\textstyle\int\!$}}{\int}} % \rotatebox{8} for vertical line
\DeclareMathOperator*{\ointegral}{\scalerel*{\rotatebox{12}{$\!\textstyle\oint\!$}}{\oint}} % \rotatebox{8} for vertical line

\newcommand{\boldnabla}{\bm{\nabla}\hspace{-0.2ex}}
\newcommand{\boldnablacircled}{\mathcircabove{\bm{\nabla}}\hspace{-0.2ex}}

\newcommand*\Laplacian{\mathop{}\!\mathbin\bigtriangleup\hspace{.04ex}}

\newcommand\narroweq{\hspace{.25ex}\scalebox{.8}[1]{$=$}\hspace{.25ex}}

\newcommand\constant{\scalebox{.92}[.98]{\ensuremath{\mathsf{constant}}}}
\newcommand\boldconstant{\mathcolor{green!50!black}{\boldmath \constant}}

\newcommand\eigenvalue{\eta\hspace{.1ex}}

\newcommand\smallparameter{\upchi}

\let\vec\relax

\usepackage{nicefrac}
\usepackage{xfrac}

\usepackage{upgreek}

\usepackage{fontawesome}

\newcommand\variation[1]{\ensuremath\updelta\hspace{-0.1ex}#1}
\newcommand\secondvariation[1]{\ensuremath\updelta^{\hspace{-0.1ex}2}#1}

\makeatletter
\newcommand\boldvari@tion[2]{\ensuremath\mathcolor{#2}{\bm{\updelta}\hspace{-0.125ex}\bm{\mathrm{#1}}}}
\newcommand\varvector[1]{\boldvari@tion{#1}{teal}}
\newcommand\constvarvector[1]{\boldvari@tion{#1}{green!50!black}}
\newcommand\varbivalent[1]{\boldvari@tion{#1}{cyan!66!black}}
\makeatother

\makeatletter
\newcommand*\dotp{\mathpalette\dotp@{.55}}
\newcommand*\bigdot{\mathpalette\dotp@{.64}}
\newcommand*\dotp@[2]{\mathbin{\vcenter{\hbox{\scalebox{#2}{$\m@th#1\bullet$}}}}}
\makeatother
\newcommand\dotdotp{\dotp\hspace{-0.16em}\dotp\hspace{.20em}}
\newcommand\narrowdotp{\hspace{-0.12em}\dotp\hspace{-0.12em}}
\newcommand\narrowdotdotp{\hspace{-0.12em}\dotdotp\hspace{-0.12em}}

\newcommand\dotdotdotp{\dotp\hspace{-0.16em}\dotp\hspace{-0.16em}\dotp\hspace{.08em}}

\newcommand\T{{\mathpalette\raiseT\intercal}} % symbol for transpose
\newcommand\raiseT[2]{\hspace{-0.02em}\raisebox{.4ex}{$#1#2$}}

\newcommand\indexjuggling[2]{#1\hspace{.1ex}\scalebox{.6}[.8]{$\rightleftarrows$}\hspace{.1ex}#2}

% symbol for X-companion
\newcommand\Xcompanion{{\mathpalette\lowerXcompanion\times}}
\newcommand\lowerXcompanion[2]{\hspace{-0.16ex}\raisebox{-0.2ex}{$#1\bm{#2}$}\hspace{-0.24ex}}

\makeatletter
\newcommand{\raisemath}[1]{\mathpalette{\raisem@th{#1}}}
\newcommand{\raisem@th}[3]{\raisebox{#1}{$#2#3$}}
\makeatother

\newcommand\constantbivalent{\raisemath{-0.1em}{\mathcolor{green!50!black}{^2}}\boldconstant}

\newcommand*{\lineover}[1]{\overline{\raisebox{0pt}[1.2\height]{\ensuremath{#1}}}}
\newcommand*{\lineoverlower}[1]{\overline{\raisebox{0pt}[1.1\height]{\ensuremath{#1}}}}

% symbol for trace dot
\newcommand\tracedot{\hspace{-0.1ex}\raisemath{-0.4ex}{\scalebox{.64}{$\bullet$}}\hspace{-0.1ex}}

%the Kronecker delta
%named after Leopold Kronecker
\newcommand\KroneckerDelta[1]{\delta_{#1}}

\newcommand\permutationsparitysymbolslower[1]{\raisemath{.2ex}{\scalebox{.93}{$\mathrm{\in}$}_{\raisemath{-0.2ex}{#1}}}\hspace{.08ex}}
\newcommand\permutationsparitysymbols[1]{\permutationsparitysymbolslower{#1}}
\newcommand\permutationsparitysymbolsupper[1]{\raisemath{.2ex}{\scalebox{.93}{$\mathrm{\in}$}^{\raisemath{-0.2ex}{#1}}}\hspace{.08ex}}
\newcommand\permutationsparitytensor{{^3\hspace{-0.25ex}\bm{\epsilon}}\hspace{.05ex}}

\newcommand\dotproductinquotes{\hbox{\hspace{-0.2ex}\inquotes{${\dotp\hspace{.22ex}}$}\hspace{-0.2ex}}}
\newcommand\crossproductinquotes{\hbox{\hspace{-0.2ex}\inquotes{${\hspace{-0.25ex}\times\hspace{-0.1ex}}$}\hspace{-0.2ex}}}

\newcommand\rotationtensor{\bm{O}}
\newcommand\orthogonaltensor{\bm{Q}}

\newcommand\somebivalenttensor{{^2\!\bm{B}}}
\newcommand\somebivalenttensortransposed{{^2\!\bm{B}^{\T}}}
\newcommand\somebivalenttensorcomponents[1]{B_{#1}}
\newcommand\somebivalenttensorcomponentsupper[1]{B^{\hspace{.1ex}{#1}}}

\newcommand\somematrixelements[1]{B_{#1}}

%~%~%

\newcommand\anyfirstinvariant{\mathrm{I}}
\newcommand\anysecondinvariant{\mathrm{II}}
\newcommand\anythirdinvariant{\mathrm{III}}
%
\newcommand\anyfirstinvariantof[1]{\anyfirstinvariant\hspace{.16ex}(#1)}
\newcommand\anysecondinvariantof[1]{\anysecondinvariant\hspace{.16ex}(#1)}
\newcommand\anythirdinvariantof[1]{\anythirdinvariant\hspace{.16ex}(#1)}

\newcommand\firstalgebraicinvariant{\mathrm{algI}}
\newcommand\secondalgebraicinvariant{\mathrm{algII}}
\newcommand\thirdalgebraicinvariant{\mathrm{algIII}}
%
\newcommand\firstalgebraicinvariantof[1]{\firstalgebraicinvariant\hspace{.16ex}(#1)}
\newcommand\secondalgebraicinvariantof[1]{\secondalgebraicinvariant\hspace{.16ex}(#1)}
\newcommand\thirdalgebraicinvariantof[1]{\thirdalgebraicinvariant\hspace{.16ex}(#1)}

\newcommand\firstcharacteristicinvariant{\mathrm{chaI}}
\newcommand\secondcharacteristicinvariant{\mathrm{chaII}}
\newcommand\thirdcharacteristicinvariant{\mathrm{chaIII}}
%
\newcommand\firstcharacteristicinvariantof[1]{\firstcharacteristicinvariant\hspace{.16ex}(#1)}
\newcommand\secondcharacteristicinvariantof[1]{\secondcharacteristicinvariant\hspace{.16ex}(#1)}
\newcommand\thirdcharacteristicinvariantof[1]{\thirdcharacteristicinvariant\hspace{.16ex}(#1)}

\newcommand\firstresearchinvariant{\mathrm{resI}}
\newcommand\secondresearchinvariant{\mathrm{resII}}
\newcommand\thirdresearchinvariant{\mathrm{resIII}}
%
\newcommand\firstresearchinvariantof[1]{\firstresearchinvariant\hspace{.16ex}(#1)}
\newcommand\secondresearchinvariantof[1]{\secondresearchinvariant\hspace{.16ex}(#1)}
\newcommand\thirdresearchinvariantof[1]{\thirdresearchinvariant\hspace{.16ex}(#1)}

%~%

\newcommand\mathfrakI{\scalebox{1.03}[1]{$\mathfrak{I}$}}
\newcommand\boldmathfrakI{\scalebox{.98}[1]{$\bm{\mathfrak{I}}$}}
\newcommand\mathfrakJ{\scalebox{1.03}[1]{$\mathfrak{J}$}}
\newcommand\boldmathfrakJ{\scalebox{.98}[1]{$\bm{\mathfrak{J}}$}}

\newcommand\inertiatensor{{\hspace{-0.1ex}^2}\boldmathfrakI}
\newcommand\inertiascalar{\mathfrakI}
\newcommand\inertiatensorcomponents[1]{\mathfrakI_{\hspace{-0.1ex}#1}\hspace{.1ex}}
\newcommand\inertiatensordotabove{\ensuremath{{\hspace{-0.1ex}^2}\mathdotabove{\boldmathfrakI}}}
\newcommand\inertiatensorcircabove{\ensuremath{{\hspace{-0.1ex}^2}\mathcircabove{\boldmathfrakI}}}

\newcommand\stiffnesstensorA{\hspace{.08ex}{^4\hspace{-0.7ex}\bm{\mathcal{A}}}\hspace{.08ex}}
\newcommand\stiffnesstensorB{{^4\hspace{-0.2ex}\bm{\mathcal{B}}}}
\newcommand\stiffnesstensorC{{^4\bm{\mathcal{C}}}}

\newcommand\stiffnesstensor{\stiffnesstensorA}
\newcommand\pliabilitytensor{\stiffnesstensorB}

\newcommand{\rateofdeformationtensor}{\bm{\mathcal{D}}}
\newcommand{\rateofdeformationcomponents}[1]{\mathcal{D}_{#1}}
\newcommand{\vorticitytensor}{\scalebox{.88}[1.0]{$\bm{\mathcal{W}}$}}
\newcommand{\vorticityvector}{\bm{w}}

\newcommand\potentialenergyfunctional{{\mathscr{E}\hspace{-0.2ex}}}
\newcommand\complementaryenergyfunctional{{\mathscr{D}}}

\usepackage{accents}

\makeatletter

\newcommand{\sapproxabove}{%
	\hbox{\fontfamily{lmr}\fontsize{.64\dimexpr(\f@size pt)}{0}\selectfont$\,approx$}}
\DeclareRobustCommand{\mathapproxabove}{\accentset{\sapproxabove}}

\newcommand{\sasteriskabove}{%
	\hbox{\fontfamily{lmr}\fontsize{.64\dimexpr(\f@size pt)}{0}\selectfont$\,*$}}
\DeclareRobustCommand{\mathasteriskabove}{\accentset{\sasteriskabove}}

\newcommand{\scircabove}{%
	\hbox{\fontfamily{lmr}\fontsize{.64\dimexpr(\f@size pt)}{0}\selectfont$\,\circ$}}
\DeclareRobustCommand{\mathcircabove}{\accentset{\scircabove}}

\newcommand{\sdotabove}{%
	\hbox{\fontfamily{lmr}\fontsize{.5\dimexpr(\f@size pt)}{0}\selectfont$\,\bullet$}}
\DeclareRobustCommand{\mathdotabove}{\accentset{\sdotabove}}
\newcommand{\sdotdotabove}{%
	\hbox{\fontfamily{lmr}\fontsize{.5\dimexpr(\f@size pt)}{0}\selectfont$\,\bullet\hspace{-0.25ex}\bullet$}}
\DeclareRobustCommand{\mathdotdotabove}{\accentset{\sdotdotabove}}

\makeatother

\newcommand\mathdotbelow[1]{\underaccent{\scalebox{.5}{$\bullet$}}{#1}}

\newcommand\cosinematrix[1]{o_{#1}} % or \alpha
\newcommand\cosinematrixdotted[1]{\mathdotabove{o}_{#1}}

\newcommand\mathboldN{\scalebox{.9}[1]{\ensuremath{\bm{N}\hspace{-0.1em}}}}
\newcommand\mathboldM{\scalebox{.9}[1]{\ensuremath{\bm{M}\hspace{-0.1em}}}}
\newcommand\mathboldQ{\scalebox{.9}[1]{\ensuremath{\bm{Q}\hspace{-0.1ex}}}}
\newcommand\mathboldU{\scalebox{.9}[1]{\ensuremath{\bm{U}\hspace{-0.1ex}}}}

\newcommand\mathboldae{\ensuremath{\pmb{\scalebox{1.2}[1]{$\textit{æ}$}}}}
\newcommand\mathboldoe{\ensuremath{\pmb{\scalebox{1.2}[1]{$\textit{œ}$}}\hspace{.1ex}}}

\newcommand\mathboldalpha{\scalebox{.88}[1]{\ensuremath{\bm{\alpha}}}}

\newcommand\mathboldrcursive{\ensuremath{\pmb{\text{\textcursive{\slshape r}}}}}

\newcommand\hcursive{\smash{\scalebox{.9}{$\text{\textcursive{\slshape h}}$}\hspace{.5ex}}}

\newcommand\elcursive{\smash{\scalebox{.9}{$\text{\textcursive{\slshape l}}$}\hspace{.5ex}}}
\newcommand\elcursivesub[1]{\smash{\scalebox{.9}{$\text{\textcursive{\slshape l}}$}_{\raisemath{-0.22ex}{\hspace{.1ex}#1}}}}

\newcommand\trace[1]{\operatorname{trace}\hspace{.16ex}{#1}}

\newcommand\realpart[1]{\operatorname{{\Re}e}\hspace{.16ex}{#1}}
\newcommand\imaginarypart[1]{\operatorname{{\Im}m}\hspace{.16ex}{#1}}

\newcommand\expminusone{\hspace{.16ex}\scalebox{.8}[1]{\hbox{--}}\raisemath{.1ex}{1}}
\newcommand\expminusT{\hspace{.16ex}\scalebox{.8}[1]{\hbox{--}}\T}
\newcommand\expminustwo{\hspace{.16ex}\scalebox{.8}[1]{\hbox{--}}\hspace{.1ex}\raisemath{.1ex}{2}}

\newcommand\smthactive{{\raisemath{.1ex}{\smash{\hspace{.12ex}(\hspace{-0.1ex}a\hspace{-0.1ex})}\hspace{-0.16ex}}}}

\newcommand\smthexternal{{\raisemath{.1ex}{\smash{\hspace{.12ex}(\hspace{-0.1ex}e\hspace{-0.1ex})}\hspace{-0.16ex}}}}
\newcommand\smthinternal{{\raisemath{.1ex}{\smash{\hspace{.12ex}(i\hspace{.1ex})}\hspace{-0.16ex}}}}

\newcommand\internalwork{W^{\hspace{-0.12ex}\smthinternal}}
\newcommand\externalwork{W^{\hspace{-0.12ex}\smthexternal}}

\newcommand\potentialenergyinmechanics{\raisebox{-0.1ex}{\scalebox{1.1}[1]{$\Uppi$}}}

\newcommand\complementaryenergyinmechanics{\raisebox{-0.1ex}{\rotatebox[origin=c]{180}{\scalebox{1.1}[1]{$\Uppi$}}}} %% previously \widehat{\potentialenergyinmechanics}

\newcommand\kineticenergyinmechanics{\mathrm{K}\hspace{-0.12ex}}

\newcommand\potentialenergydensity{\Pi}
\newcommand\kineticenergydensity{\mathrm{K}\hspace{-0.12ex}}

\newcommand\complementaryenergydensity{\rotatebox[origin=c]{180}{$\Pi$}} %% previously \widehat{\potentialenergydensity}

\newcommand\differentialindex[1]{{\hspace{-0.1ex}\scalebox{.63}{$\mathcolor{black!70}{\partial}$} #1}}

\newcommand\locationvector{\bm{r}}
%%\newcommand\newlocationvector{\bm{R}}
\newcommand\currentlocationvector{\locationvector}
\newcommand\initiallocationvector{\mathcircabove{\locationvector}}

\newcommand\UnitDyad{\bm{E}}

\newcommand\unitnormalvector{\bm{n}}
\newcommand\currentunitnormal{\unitnormalvector} % \mathboldN
\newcommand\initialunitnormal{\mathcircabove{\unitnormalvector}}

\newcommand\verynicefrac[2]{\raisebox{.11ex}{$ \raisebox{.3em}{\scalebox{.7}{$#1$}} \hspace{-0.4ex} / \hspace{-0.46ex} \raisebox{-0.25em}{\scalebox{.7}{$#2$}}\hspace{.15ex} $}}

\newcommand\onehalf{\raisemath{.02em}{\frac{\raisemath{-0.16em}{1}}{2}}}
\newcommand\smalldisplaystyleonehalf{\scalebox{.8}{$\displaystyle \onehalf$}\hspace{.1ex}}
\newcommand\smallerdisplaystyleonehalf{\scalebox{.7}{$\raisemath{.1em}{\displaystyle \onehalf}$}\hspace{.1ex}}

\newcommand\onefourth{\raisemath{.02em}{\frac{\raisemath{-0.16em}{1}}{4}}}
\newcommand\smalldisplaystyleonefourth{\scalebox{.8}{$\displaystyle \onefourth$}\hspace{.1ex}}

\def\upTeeSymbol{}

% put the symbol of perpendicularity into the index of an argument
\newcommand\withtheindexofperpendicularity[1]{{#1}_{\hspace{-.25ex}\bot}}

% put the symbol of parallelism into the index of an argument
\newcommand\withtheindexofparallelism[1]{{#1}_{\parallel}}

\usepackage{scalerel}

\usepackage{paralist}
\usepackage{booktabs}

%%\usepackage{indentfirst}

\usepackage{setspace}
%%\onehalfspacing
\setstretch{1}

\def\spacebetweenparagraphs{.8mm minus0.2mm}
\setlength{\parskip}{\spacebetweenparagraphs}

\def\horizontalindent{3.3ex}
\setlength{\parindent}{\horizontalindent} % offset of the first line
\newlength\negparindent
\setlength{\negparindent}{-\parindent}

\def\horizontalindentforchapterintro{2.7ex}

\def\changemargin#1#2{\list{}{\rightmargin#2\leftmargin#1}\item[]}
\let\endchangemargin=\endlist

\usepackage{blindtext}
\usepackage{fancybox}
\usepackage{rotating}
\usepackage{placeins}
\usepackage{wasysym}
\usepackage{etoolbox}

\usepackage{tikz}
\usepackage{tikz-3dplot} % it needs tikz-3dplot.sty in same folder
\usetikzlibrary{calc}
%%\usetikzlibrary{topaths}
\usetikzlibrary{angles, quotes}
\usetikzlibrary{arrows, arrows.meta}
\usetikzlibrary{positioning}
\usetikzlibrary{shapes.misc}
\usetikzlibrary{decorations.markings, decorations.pathmorphing, decorations.pathreplacing}

\tikzset {
	set arrow inside/.code={\pgfqkeys{/tikz/arrow inside}{#1}},
	set arrow inside={end/.initial=>, opt/.initial=},
	/pgf/decoration/Mark/.style={
		mark/.expanded=at position #1 with {
			\noexpand\arrow[\pgfkeysvalueof{/tikz/arrow inside/opt}]{\pgfkeysvalueof{/tikz/arrow inside/end}}
		}
	},
	arrow inside/.style 2 args={
		set arrow inside={#1},
		postaction={
			decorate,decoration={
				markings,Mark/.list={#2}
			}
		}
	},
}

\usepackage{tkz-euclide}

\usepgfmodule{nonlineartransformations}
\usepgflibrary{curvilinear}

\newcommand{\tikzsl}[1]{\tikz[baseline=(sl.base)] \node [xslant=0.21, inner sep=0pt, outer sep=0pt] (sl) {#1};}

\newcommand\nullofthefirstcomplexity{{\bm{0}}}
\newcommand\zerovector{\nullofthefirstcomplexity}
%\newcommand\nullvector{\nullofthefirstcomplexity}
%\newcommand\nilvector{\nullofthefirstcomplexity}

\newcommand\nullofthesecondcomplexity{{^2\bm{0}}}
\newcommand\zerobivalent{\nullofthesecondcomplexity}
%\newcommand\nullbivalent{\nullofthesecondcomplexity}
%\newcommand\nilbivalent{\nullofthesecondcomplexity}

\newcommand\nullofthefourthcomplexity{{^4\bm{0}}}
\newcommand\zerotetravalent{\nullofthefourthcomplexity}

\newcommand\mathboldPhi{\scalebox{.98}[1]{\tikzsl{\ensuremath{\hspace{-0.16ex}\bm{\Phi}\hspace{-0.2ex}}}}}

\newcommand\mathsigma{\scalebox{1}[1.06]{\tikzsl{\ensuremath{\hspace{-0.15ex}\upsigma\hspace{-0.1ex}}}}}
\newcommand\mathboldsigma{\scalebox{1}[1.06]{\tikzsl{\ensuremath{\hspace{-0.1ex}\bm{\upsigma}\hspace{-0.1ex}}}}}

\newcommand\mathtau{\scalebox{1}[1.06]{\tikzsl{\ensuremath{\uptau\hspace{-0.1ex}}}}}
\newcommand\mathboldtau{\scalebox{1}[1.06]{\tikzsl{\ensuremath{\bm{\uptau}\hspace{-0.1ex}}}}}

\newcommand\cauchystress{\mathboldtau}
\newcommand\cauchystresscomponents[1]{\mathtau_{#1}}

\newcommand\tractionvector[1]{{\bm{t}{\raisebox{-0.1em}{$\scriptstyle (#1)$}}\hspace{-0.1ex}}}
\newcommand\tractionvectoritself{\bm{t}}

\newcommand\linearstress{\mathboldsigma}

\newcommand\firstpiolakirchhoffstress{\bm{T}}
\newcommand\secondpiolakirchhoffstress{\bm{S}}

\newcommand\mathepsilon{\scalebox{1}[1.06]{\hspace{-0.2ex}\tikzsl{\scalebox{1.2}[1]{\ensuremath{\upvarepsilon\hspace{-0.1ex}}}}}}
\newcommand\mathboldepsilon{\scalebox{1}[1.06]{\hspace{-0.2ex}\tikzsl{\scalebox{1.2}[1]{\ensuremath{\bm{\upvarepsilon}\hspace{-0.1ex}}}}}}

\newcommand\infinitesimaldeformation{\mathboldepsilon}
\newcommand\infinitesimaldeformationcomponents[1]{\hspace{.123ex}\varepsilon_{#1}}
\newcommand\infinitesimalstrain{\infinitesimaldeformation}
\newcommand\infinitesimalstraincomponents[1]{\infinitesimaldeformationcomponents{#1}}
\newcommand\infinimentpetitdeformationdevariation{\varbivalent{\infinitesimaldeformation}}
\newcommand\infinitesimaldeformationeigenvalue[1]{\mathepsilon_{#1}}

\newcommand{\fieldofdisplacements}{\bm{u}}
\newcommand{\fieldofdisplacementscomponents}[1]{u_{[#1]}}

\newcommand{\fieldofrotations}{\bm{\varphi}} %% {\bm{\theta}}

\newcommandx*{\tikzcancel}[2][1=black]{%
	\tikz[baseline=(tocancel.base)]{%
		\node [inner sep=-1pt, outer sep=0pt] (tocancel) {#2};
		\draw [line width=0.25mm, #1] (tocancel.south west) -- (tocancel.north east);
	}%
}

\newcommandx*{\tikzbackcancel}[2][1=black]{%
	\tikz[baseline=(tocancel.base)]{%
		\node [inner sep=-1pt, outer sep=0pt] (tocancel) {#2};
		\draw [line width=0.25mm, #1] (tocancel.south east) -- (tocancel.north west);
	}%
}

\newcommandx*{\tikzcrosscancel}[2][1=black]{%
	\tikz[baseline=(tocancel.base)]{%
		\node [inner sep=-1pt, outer sep=0pt] (tocancel) {#2};
		\draw [line width=0.25mm, #1] (tocancel.south west) -- (tocancel.north east);
		\draw [line width=0.25mm, #1] (tocancel.south east) -- (tocancel.north west);
	}%
}

\newcommand\widearc[1]{%
	\tikz[baseline=(wideArcAnchor.base)]{%
		\node [inner sep=0] (wideArcAnchor) {$#1$};
		\coordinate (wideArcAnchorA) at ($ (wideArcAnchor.north west) + (0.2ex, 0.33ex) $);
		\coordinate (wideArcAnchorB) at ($ (wideArcAnchor.north east) + (-0.2ex, 0.33ex) $);
		\draw [line width=.5pt, line cap=round, out=60, in=120]
			(wideArcAnchorA) to (wideArcAnchorB) ;
}}

\newcommand\widearctoo[1]{%
	\tikz[baseline=(wideArcAnchor.base)]{%
		\node [inner sep=0] (wideArcAnchor) {$#1$};
		\coordinate (wideArcAnchorA) at ($ (wideArcAnchor.north west) + (0.2ex, 0.66ex) $);
		\coordinate (wideArcAnchorB) at ($ (wideArcAnchor.north east) + (-0.2ex, 0.66ex) $);
		\draw [line width=.5pt, line cap=round, out=-60, in=-120]
			(wideArcAnchorA) to (wideArcAnchorB) ;
}}

\newcommand\widearclowered[1]{%
	\tikz[baseline=(wideArcAnchor.base)]{%
		\node [inner sep=0] (wideArcAnchor) {$#1$};
		\coordinate (wideArcAnchorA) at ($ (wideArcAnchor.north west) + (0.2ex, 0.2ex) $);
		\coordinate (wideArcAnchorB) at ($ (wideArcAnchor.north east) + (-0.2ex, 0.2ex) $);
		\draw [line width=.5pt, line cap=round, out=45, in=135]
			(wideArcAnchorA) to (wideArcAnchorB) ;
}}

\pgfdeclaredecoration{plain brace}{brace}
{%
  \state{brace}[width=+\pgfdecoratedremainingdistance, next state=final]
  {%
    \pgfpathmoveto{\pgfpointorigin}%
    {%
      \pgfpathcurveto%
        {\pgfqpoint{.15\pgfdecorationsegmentamplitude}{.3\pgfdecorationsegmentamplitude}}%
        {\pgfqpoint{.5\pgfdecorationsegmentamplitude}{.5\pgfdecorationsegmentamplitude}}%
        {\pgfqpoint{\pgfdecorationsegmentamplitude}{.5\pgfdecorationsegmentamplitude}}%
    }%
    \pgftransformxshift{+\pgfdecoratedremainingdistance}%
    \pgfpathlineto{\pgfqpoint{-\pgfdecorationsegmentamplitude}{.5\pgfdecorationsegmentamplitude}}%
    {%
      \pgfpathcurveto%
        {\pgfqpoint{-.5\pgfdecorationsegmentamplitude}{.5\pgfdecorationsegmentamplitude}}%
        {\pgfqpoint{-.15\pgfdecorationsegmentamplitude}{.3\pgfdecorationsegmentamplitude}}%
        {\pgfqpoint{0\pgfdecorationsegmentamplitude}{0\pgfdecorationsegmentamplitude}}%
    }%
  }%
  \state{final}{}%
}

\newcommand{\tikzmark}[1]{\tikz[overlay, remember picture] \node (#1) {};}

% tweak these if you wish
\newcommand*{\BraceAmplitude}{0.4em}
\newcommand*{\VerticalOffset}{0.64ex}
\newcommand*{\HorizontalOffset}{0em}
\newcommand*{\TextOffset}{0.4ex}

\NewDocumentCommand{\AddUnderBrace}{%
	O{}	% #1 = draw options
	O{0,0} % #2 = [x,y] offset for brace
	O{}	% #3 = node options
	O{}	% #4 = optional brace options
	m	% #5 = left tikzmark
	m	% #6 = right tikzmark
	m	% #7 = text to place underbrace
}{%
\begin{tikzpicture}[overlay, remember picture]
	\draw [decoration={plain brace, amplitude=\BraceAmplitude, mirror, #4}, decorate, thick, draw=black!80, text=black, #1]
		($ (#5.base) + (-\HorizontalOffset,-\VerticalOffset) + (#2) $) --
		($ (#6.base) + (\HorizontalOffset,-\VerticalOffset) + (#2) $)
	node [below, yshift=-\TextOffset, midway, #3] {#7};
\end{tikzpicture}%
}

\NewDocumentCommand{\AddOverBrace}{%
	O{}	% #1 = draw options
	O{0,0} % #2 = [x,y] offset for brace
	O{}	% #3 = node options
	O{}	% #4 = optional brace options
	m	% #5 = left tikzmark
	m	% #6 = right tikzmark
	m	% #7 = text to place underbrace
}{%
\begin{tikzpicture}[overlay, remember picture]
	\draw [decoration={plain brace, amplitude=\BraceAmplitude, #4}, decorate, thick, draw=black!80, text=black, #1]
		($ (#5.north) + (-\HorizontalOffset,\VerticalOffset+0.25em) + (#2) $) --
		($ (#6.north) + (\HorizontalOffset,\VerticalOffset+0.25em) + (#2) $)
	node [above, yshift=\TextOffset, midway, #3] {#7};
\end{tikzpicture}%
}

\newcommandx*\circled[2][1=black!5]{\tikz[baseline=(char.base)]{%
	\node[shape=circle, fill=#1, inner sep=2pt] (char) {#2};}}

% tune headers and footers

\setlength{\headheight}{5mm}
\setlength{\footskip}{0pt}

\usepackage{fancyhdr}

\newlength\pagenumberoffset
\addtolength{\pagenumberoffset}{9mm}

\fancypagestyle{plain}{%
\fancyhf{} % clear all header and footers
\fancyhead[LE,RO]{\circled[yellow!12]{\scalebox{.75}{\textcursive\thepage}}}
\fancyheadoffset[LE]{\pagenumberoffset} % move page numbers left on even pages
\fancyheadoffset[RO]{\pagenumberoffset} % move page numbers right on odd pages
\renewcommand{\headrulewidth}{0pt} % no line in header
\renewcommand{\footrulewidth}{0pt} % no line in footer
}

\pagestyle{fancy}
\fancyhf{}
\fancyhead[LE,RO]{\circled[yellow!12]{\scalebox{.75}{\textcursive\thepage}}}
\fancyheadoffset[LE]{\pagenumberoffset}
\fancyheadoffset[RO]{\pagenumberoffset}
\renewcommand{\headrulewidth}{0pt}
\renewcommand{\footrulewidth}{0pt}

\usepackage{array}
\newcolumntype{C}[1]{>{\centering\arraybackslash}m{#1}<{}} % equal-width centered column

% tune captions for figures and tables

\usepackage[format=plain]{caption}

\captionsetup[figure]{%
font={small,it},labelfont=small,%
labelsep=newline,justification=centering,singlelinecheck=off,%
aboveskip=4mm,belowskip=2.5mm}

\usepackage{chngcntr}
\counterwithout{footnote}{chapter}
\counterwithout{figure}{chapter}
\counterwithout{table}{chapter}

\addto\captionsenglish{\renewcommand{\chaptertitlename}{Chapter}}
\addto\captionsrussian{\renewcommand{\chaptertitlename}{Глава}}
%%\addto\captionsukrainian{\renewcommand{\chaptertitlename}{Розділ}}

\addto\captionsenglish{\def\chaptershortname{chapter\hspace{.28ex}}}
\addto\captionsrussian{\def\chaptershortname{гл.}}
%%\addto\captionsukrainian{\def\chaptershortname{розд.}}

\addto\captionsenglish{\renewcommand{\contentsname}{Contents}}
\addto\captionsrussian{\renewcommand{\contentsname}{Содержание}}
%%\addto\captionsukrainian{\renewcommand{\contentsname}{Змiст}}

\addto\captionsenglish{\renewcommand{\figurename}{figure}}
%%\addto\captionsrussian{\renewcommand{\figurename}{фигура}}
\addto\captionsrussian{\renewcommand{\figurename}{рисунок}}
%%\addto\captionsukrainian{\renewcommand{\figurename}{малюнок}}

\addto\captionsenglish{\def\figureshortname{\figurename}} %%\def\figureshortname{fig.}
\addto\captionsrussian{\def\figureshortname{\figurename}} %%\def\figureshortname{фиг.} %%\def\figureshortname{рис.}
%%\addto\captionsukrainian{\def\figureshortname{\figurename}} %%\def\figureshortname{мал.}

\addto\captionsenglish{\def\wordforbibliography{Bibliography}}
\addto\captionsrussian{\def\wordforbibliography{Библиография}}

\newcommand\publication[1]{\bibitem{#1}}

\newcommand\ruyy{у\kern-0.11exу}
%
\newcommand\rucontinuum{контин\ruyy{}м}
\newcommand\ruContinuum{Контин\ruyy{}м}
%
\newcommand\ruvacuum{вак\ruyy{}м}
\newcommand\ruVacuum{Вак\ruyy{}м}

\newcommand\sectionref[1]{\hyperref[#1]{\hbox{\S\,\ref*{#1}}}}
\newcommand\chapterdotsectionref[2]{\hbox{\hyperref[#2]{\S\,\ref*{#1}.\ref*{#2}}}}
\newcommand\chapterref[1]{\hyperref[#1]{\hbox{\chaptershortname\:\ref*{#1}}}}

\newcommand\figureref[1]{\figureshortname\:\ref{#1}}

\newcommandx*\customref[2][1=]{\hyperref[#2]{#1\ref*{#2}}}

\newcommand\eqrefwithchapterdotsection[3]{\text{(\hyperref[#1]{\ref*{#1}},\:\chapterdotsectionref{#2}{#3})}}

\captionsetup[table]{%
%labelsep=newline,
labelsep=period,justification=centering,singlelinecheck=off}

\newcommand*\boldauthor[1]{\textbold{#1}}

\newcommandx*{\bookauthor}[3][2=,3=]{\mbox{\boldauthor{#1}}%
\ifthenelse{\equal{#2}{}}{}{~\mbox{\textbold{#2}}}%
\ifthenelse{\equal{#3}{}}{}{\:\mbox{\textbold{#3}}}%
}

\newcommand\howmanypages[1]{\textcolor{gray}{#1}}

\sloppy

%\usepackage{showframe} % uncomment to show frames

% tune lists

\usepackage[shortlabels]{enumitem}
\setlist{nolistsep} % no vertical space between list items

%%\renewcommand{\labelitemi}{\raise .36ex\hbox{\tiny${\bullet}$}}
\renewcommand{\labelitemi}{\pmb{\checkmark}}

\newlength\mylistlabelsep
\setlength{\mylistlabelsep}{1.2mm}

\newlength\listlabelwithsep
\setlength{\listlabelwithsep}{\mylistlabelsep}
\addtolength{\listlabelwithsep}{\widthof{\labelitemi}}

\newlength\listleftindent
\setlength{\listleftindent}{\parindent}
\addtolength{\listleftindent}{\listlabelwithsep}

\makeatletter
\renewcommand{\@listI}{%
	\leftmargin=\listleftindent
	\rightmargin=0mm
	\labelsep=\mylistlabelsep
	\itemindent=0mm
	\listparindent=0mm
	\parskip=0mm
	\topsep=1mm
	\partopsep=0mm
	\parsep=0mm
	\itemsep=.1em % vertical space between items
}
\makeatother

%%\usepackage[defaultlines=2,all]{nowidow}
\raggedbottom
\widowpenalty=10000 % avoid widows
\clubpenalty=10000 % avoid orphans (clubs)

\makeatletter
\@beginparpenalty=10000 % avoid page breaks before lists
\makeatother

\hyphenpenalty=800 % reduce number of hyphenation

% --------------------------------------------------

% tune footnotes

%%\usepackage{footnotebackref}

\renewcommand*\footnoterule{} % no line before footnotes

%%\renewcommand{\thefootnote}{\arabic{footnote})}

\renewcommand{\thempfootnote}{\arabic{mpfootnote}}

\usepackage[symbol, bottom]{footmisc} % use symbols instead of numbers and place footnotes at the very bottom of page
\setlength{\skip\footins}{1em plus1em} % \footins is the space between the text body and the footnotes

\setlength{\footnotesep}{\baselineskip} % space between footnotes

\makeatletter
\def\@fnsymbol#1{\ensuremath{\ifcase #1 %
	\or {*} %
	\or {*}\hspace{-0.2ex}{*} %
	\or {*}\hspace{-0.2ex}{*}\hspace{-0.2ex}{*} %
	\or {*}\hspace{-0.2ex}{*}\hspace{-0.2ex}{*}\hspace{-0.2ex}{*} %
	\else {\infty} %% \else \@ctrerr
\fi}}
\def\@makefnmark{\raisebox{.8ex}{\hbox{\normalfont\@thefnmark}}}
\makeatother

\usepackage{perpage}
\MakePerPage{footnote} % reset footnote numbering on every page

% pair of counters for something like
% \stepcounter{footnote}\setcounter{auxfootnotecounter}{\value{footnote}}\footnotemark[\value{auxfootnotecounter}]
\newcounter{auxfootnotecounter}
\newcounter{auxfootnotecountertoo}

\makeatletter
\renewcommand\footnotesize{%
   \@setfontsize\footnotesize\@ixpt{11}%
   \abovedisplayskip 8\p@ \@plus2\p@ \@minus4\p@
   \abovedisplayshortskip \z@ \@plus\p@
   \belowdisplayshortskip 4\p@ \@plus2\p@ \@minus2\p@
   \def\@listi{\leftmargin\leftmargini
               \topsep 4\p@ \@plus2\p@ \@minus2\p@
               \parsep 2\p@ \@plus\p@ \@minus\p@
               \itemsep \parsep}%
   \belowdisplayskip \abovedisplayskip
}
\makeatother

\let\originalfootnote\footnote
\renewcommand{\footnote}[1]{%
	\originalfootnote{\hspace{.25em}\parbox[t]{.88\textwidth}{#1}}}

\newcommand{\nicefootnotetext}[2]{%
	\footnotetext[\value{#1}]{\hspace{.25em}\parbox[t]{.88\textwidth}{#2}}}

% --------------------------------------------------

% tune the list of publications, mistakenly referred to as "the bibliography"

\makeatletter
\renewcommand*{\@biblabel}[1]{#1.}
\g@addto@macro\@openbib@code{\setlength{\itemsep}{.2em}}
\makeatother

\addto\captionsenglish{\def\bibliographyname{List of publications}}
\addto\captionsenglish{\renewcommand{\bibname}{\bibliographyname}}

\addto\captionsrussian{\def\bibliographyname{Список публикаций}}
\addto\captionsrussian{\renewcommand{\bibname}{\bibliographyname}}

%%\addto\captionsukrainian{\def\bibliographyname{Список публікацій}}
%%\addto\captionsukrainian{\renewcommand{\bibname}{\bibliographyname}}

% --------------------------------------------------

\renewcommand{\labelenumi}{\arabic{enumi}) }
\renewcommand{\labelenumii}{\alph{enumii}) }
\renewcommand{\labelenumiii}{\roman{enumiii}) }

% tune titles

\makeatletter
\patchcmd{\chapter}{\if@openright\cleardoublepage\else\clearpage\fi}{}{}{}
\makeatother

\usepackage{titlesec}

\renewcommand{\thechapter}{\arabic{chapter}}
\renewcommand{\thesection}{\arabic{section}}
\renewcommand{\thesubsection}{\arabic{section}.\arabic{subsection}}

\newcommand{\mychapterfont}{\normalfont\fontsize{12}{13}}
\newcommand{\mysectionfont}{\normalfont\normalsize\bfseries}
\newcommand{\mysubsectionfont}{\normalfont\normalsize\itshape}

\titleformat{\chapter}[display]
	{\mychapterfont\filleft}
	{\vspace{0pt}\scalebox{1.11}[1.1]{\textit{\MakeLowercase{\chaptertitlename}~\thechapter}}}
	{.4mm}
	{\comfortaa\bfseries\MakeUppercase}

\titlespacing*{\chapter}
	{\parindent}% left
	{12mm}% before
	{8mm}% after
	[\parindent]%right

\titleformat{\section}
	{\mysectionfont\filleft}{\mdseries\comfortaa{\S\:\thesection\hspace{.1ex}.}}{.8ex}{\comfortaa}

\titlespacing*{\section}
	{\parindent}% left
	{8mm}% before
	{2mm}% after
	[\parindent]%right

\titleformat{\subsection}
	{\mysubsectionfont\filleft}{\mdseries\itshape(\thesubsection)}{.8ex}{}

\titlespacing*{\subsection}
	{\parindent}% left
	{4mm}% before
	{1mm}% after
	[\parindent]%right

%%%\usepackage[format=hang,labelsep=period,justification=raggedright,singlelinecheck=true]{caption}

% --------------------------------------------------

% tune table of contents

\usepackage{tocloft}

\setcounter{tocdepth}{1} % chapters & sections, no subsections

\renewcommand{\cfttoctitlefont}{\hfill\mychapterfont\comfortaa\bfseries\MakeUppercase}
\renewcommand{\cftaftertoctitle}{\hspace{\horizontalindent}}
\setlength\cftbeforetoctitleskip{4em} % 2.5em
\setlength\cftaftertoctitleskip{3.3em} % 2.5em

\setlength{\cftbeforechapskip}{.64em}
\setlength{\cftbeforesecskip}{.1em} % {\parskip}

\renewcommand{\cftchapafterpnum}{\nopagebreak\vspace{.2em}}
\preto\section{%
	\ifnum\value{section}=0\addtocontents{toc}{\vspace{.2em}}\fi
}

\makeatletter
\patchcmd{\l@chapter}
	{\leavevmode}
	{\leavevmode\renewcommand{\numberline}[1]{\hbox{\normalfont\emph{\@cftbsnum ##1\@cftasnum}}\@cftasnumb}}
	{}{}
\makeatother

\newlength\raisetocchap
\setlength{\raisetocchap}{.11ex}
\renewcommand{\cftchappresnum}{\fontsize{10}{11}\selectfont\color{black!66}\raisebox{\raisetocchap}\chaptertitlename~\color{black!66}\raisebox{\raisetocchap}}
\renewcommand{\cftchapaftersnum}{~\hspace{1.5ex}}

\renewcommand{\cftsecpresnum}{\hfill\hspace{-0.5em}\S\:}
\renewcommand{\cftsecaftersnum}{.~\hspace{.15ex}}
\newlength\tocsectionlength
\settowidth\tocsectionlength{\cftsecpresnum\cftsecaftersnum}
\addtolength\cftsecnumwidth{\tocsectionlength}
\addtolength\cftsecnumwidth{-2.4ex}

\renewcommand{\cftpnumalign}{l} % align page numbers left

\makeatletter
\renewcommand{\@pnumwidth}{3ex}
\renewcommand{\@tocrmarg}{1.5cm}
\makeatother

\renewcommand{\cftdotsep}{1.2}

\renewcommand{\cftchapleader}{\hfill}
\renewcommand{\cftchapfont}{\fontsize{9}{10}\bfseries\comfortaa}
\renewcommand{\cftchappagefont}{\fontsize{9}{10}\bfseries\comfortaa}

\renewcommand{\cftsecleader}{\textcolor{black!40}{\cftdotfill{\cftdotsep}}\hspace{.5ex}}
\renewcommand{\cftsecfont}{\fontsize{10}{11}\mdseries}
\renewcommand{\cftsecpagefont}{\fontsize{10}{11}\mdseries}

% --------------------------------------------------

\usepackage[many]{tcolorbox} %%[most]{tcolorbox}
\tcbset{
	frame code={}
	center title,
	left = 0pt,
	right = 0pt,
	top = 0pt,
	bottom = 0pt,
	colback = gray!10,
	colframe = white,
	width = \dimexpr\textwidth\relax,
	enlarge left by = 0mm,
	boxsep = 5pt,
	arc = 0pt, outer arc = 0pt}

\newtcolorbox{leftverticalbar}[1][]{%
	blanker, %%breakable,
	left=10pt, right=0pt, top=1pt, bottom=1pt,
	borderline west={5pt}{0pt}{cyan!50},
	before upper=\indent, parbox=false, #1}

\setcounter{totalnumber}{100} % increase number of floats on one page

\newcommand{\lsinglequote}[0]{‘}
\newcommand{\rsinglequote}[0]{’}

\newcommand\insinglequotes[1]{\lsinglequote{#1}\rsinglequote}

\newcommand{\lquote}[0]{“} % {``}
\newcommand{\rquote}[0]{”} % {''}

\newcommand{\altlquote}[0]{«} % {<<}
\newcommand{\altrquote}[0]{»} % {>>}

\newcommand{\inquotes}[1]{\lquote{#1}\rquote}
\newcommand{\inaltquotes}[1]{\altlquote{#1}\altrquote}

\newcommandx*{\inquotesx}[2][2=]{\lquote{#1}\hbox{\rquote%
\ifthenelse{\equal{#2}{.}}{\hspace{-0.7ex}.\hspace{.3ex}}{}%
\ifthenelse{\equal{#2}{,}}{\hspace{-0.7ex},\hspace{.3ex}}{}%
\ifthenelse{\equal{#2}{---}}{\hspace{-0.5ex}~---}{}%
}}

\newcommand{\sectionunnumbered}[1]%
{%
	\phantomsection
	\section*{#1}
	\addcontentsline{toc}{section}{#1}
}

\newcommand{\chapterunnumbered}[1]%
{%
	\phantomsection
	\chapter*{#1}
	\addcontentsline{toc}{chapter}{#1}
}

\newcommandx*{\loosetexttr}[2][1=50]{\textls[#1]{#2}}

\newcommand{\replica}[1]{\noindent ---~{#1}}
\newcommand{\remark}[1]{\emph{({#1})}}

\newcommand{\bigvspace}[0]{\vspace{0.75\vsize}}

\setlength{\arraycolsep}{3.2pt} % default is 5 points

% space above and below equations
\setlength{\abovedisplayskip}{2pt}
\setlength{\belowdisplayskip}{2pt}
