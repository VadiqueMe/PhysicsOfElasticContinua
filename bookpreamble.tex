\usepackage[utf8]{inputenc}

\usepackage[T2A,OT1,T1]{fontenc}
\usepackage{ucs}

\usepackage[english, ukrainian, russian]{babel}

\RequirePackage{ifthen}
\newif\ifen
\newif\ifru
\newcommand{\en}[1]{\ifen#1\fi}
\newcommand{\ru}[1]{\ifru#1\fi}

\PrerenderUnicode{æ}

\usepackage{newunicodechar}
\newunicodechar{×}{\ensuremath{\times}}

% no more "Too many math alphabets used in version normal"
%%\newcommand\hmmax{0}
%%\newcommand\bmmax{0}

\usepackage{gensymb}

\usepackage{xargs}
\usepackage{xparse}

%%\usepackage{calligra}
\usepackage{frcursive}

\usepackage[usenames,dvipsnames]{xcolor}

\definecolor{arsenic}{rgb}{0.23, 0.27, 0.29}
\definecolor{charcoal}{rgb}{0.21, 0.27, 0.31}
\definecolor{chocolate}{rgb}{0.48, 0.25, 0.0}
\definecolor{darkcerulean}{rgb}{0.03, 0.27, 0.49}
\definecolor{darkmagenta}{rgb}{0.55, 0.0, 0.55}
\definecolor{darksienna}{rgb}{0.24, 0.08, 0.08}

\makeatletter
\def\mathcolor#1#{\@mathcolor{#1}}
\def\@mathcolor#1#2#3{%
	\protect\leavevmode
	\begingroup\color#1{#2}#3\endgroup
}
\makeatother

\usepackage[unicode=true,pdfusetitle,%
	bookmarks=true,bookmarksnumbered=false,bookmarksopen=true,linktocpage=true,%
	breaklinks=false,pdfborder={0 0 0},backref=false,%
	colorlinks=true,linkcolor=darkcerulean,citecolor=darkmagenta,urlcolor=darksienna]%
{hyperref}

\usepackage{bookmark}

\usepackage{microtype}

\usepackage{lettrine}
\renewcommand{\LettrineFontHook}{\fontseries{b}}
\renewcommand{\LettrineTextFont}{\normalfont}

\usepackage{calc}
\usepackage{ulem}

\usepackage{xintexpr}

%%\usepackage{pdflscape}

\usepackage{layouts}

\usepackage{bigstrut}

\usepackage{wrapfig}

\usepackage{multirow}
%%\usepackage{tabularx}
%%\usepackage{tabulary}
\usepackage{longtable}

\usepackage{bm} % bold math symbols

% from newtxmath
\DeclareFontFamily{U}{ntxmia}{}
\DeclareFontShape{U}{ntxmia}{m}{it}{<-> ntxmia }{}
\DeclareFontShape{U}{ntxmia}{b}{it}{<-> ntxbmia }{}
\DeclareSymbolFont{lettersA}{U}{ntxmia}{m}{it}
\SetSymbolFont{lettersA}{bold}{U}{ntxmia}{b}{it}

%%\AtBeginDocument{\let\mathbb\boldmathbb}

\ExplSyntaxOn
\NewDocumentCommand{\boldmathbb}{m}
{
	\tl_map_inline:nn { #1 }
	{
		\use:c { varbb##1 }
	}
}
\tl_map_inline:nn { ABCDEFGHIJKLMNOPQRSTUVWXYZ }
{
	\exp_args:Nc \DeclareMathSymbol{varbb#1}{\mathord}{lettersA}{\int_eval:n { `#1+64 }}
}
\exp_args:Nc \DeclareMathSymbol{varbbk}{\mathord}{lettersA}{169}
\ExplSyntaxOff

\usepackage{amsmath}
\usepackage{nccmath}
\usepackage{mathtools}

\numberwithin{equation}{section}

\usepackage{abraces}

\usepackage{amssymb}
\usepackage{wasysym}
\usepackage{mathrsfs}

\usepackage{mathdesign}

\usepackage{graphicx}

\DeclareFontFamily{U}{mymathpazo}{}
\DeclareFontShape{U}{mymathpazo}{m}{n}{<-> fplmri}{}
\DeclareFontShape{U}{mymathpazo}{b}{n}{<-> fplmbi}{}

\usepackage{pifont}
\makeatletter
\newcommand\Pimathsymbol[3][\mathord]{%
	#1{\@Pimathsymbol{#2}{#3}}}
\def\@Pimathsymbol#1#2{\mathchoice
	{\@Pim@thsymbol{#1}{#2}\tf@size}
	{\@Pim@thsymbol{#1}{#2}\tf@size}
	{\@Pim@thsymbol{#1}{#2}\sf@size}
	{\@Pim@thsymbol{#1}{#2}\ssf@size}}
\def\@Pim@thsymbol#1#2#3{%
	\mbox{\fontsize{#3}{#3}\Pisymbol{#1}{#2}}}
\makeatother
\newcommand{\varpartial}{\Pimathsymbol[\mathord]{mymathpazo}{182}}

\newcommand\boundary{\hspace{.1ex}\varpartial}
%%\newcommand\boundary{\reflectbox{\textsf{6}}}

\DeclareMathOperator*{\integral}{\scalerel*{\rotatebox{12}{$\!\textstyle\int\!$}}{\int}} % \rotatebox{8} for vertical line
\DeclareMathOperator*{\ointegral}{\scalerel*{\rotatebox{12}{$\!\textstyle\oint\!$}}{\oint}} % \rotatebox{8} for vertical line

\newcommand{\boldnabla}{\bm{\nabla}\hspace{-0.2ex}}
\newcommand{\boldnablacircled}{\mathcircabove{\bm{\nabla}}\hspace{-0.2ex}}
\newcommand{\boldnablaflat}{\mathtwoabove{\bm{\nabla}}\hspace{-0.2ex}}

\newcommand\narroweq{\hspace{0.25ex}\scalebox{0.8}[1]{$=$}\hspace{0.25ex}}

\newcommand\const{\scalebox{0.92}[0.98]{\ensuremath{\mathsf{const}}}}

\newcommand\boldconst{\mathcolor{green!50!black}{\boldmath \const}}

\newcommand\eigenvalue{\eta\hspace{.1ex}}

\newcommand\smallparameter{\upchi}

\let\vec\relax

\usepackage{nicefrac}

\usepackage{upgreek}

\newcommand\variation[1]{\ensuremath\updelta\hspace{-0.1ex}#1}

\makeatletter
\newcommand\boldvari@tion[2]{\ensuremath\mathcolor{#2}{\bm{\updelta}\hspace{-0.125ex}\bm{\mathrm{#1}}}}
\newcommand\varvector[1]{\boldvari@tion{#1}{teal}}
\newcommand\constvarvector[1]{\boldvari@tion{#1}{green!50!black}}
\newcommand\varbivalent[1]{\boldvari@tion{#1}{cyan!66!black}}
\makeatother

\makeatletter
\newcommand*\dotp{\mathpalette\dotp@{.55}}
\newcommand*\bigdot{\mathpalette\dotp@{.64}}
\newcommand*\dotp@[2]{\mathbin{\vcenter{\hbox{\scalebox{#2}{$\m@th#1\bullet$}}}}}
\makeatother
\newcommand\dotdotp{\dotp\hspace{-0.16em}\dotp\hspace{0.20em}}
\newcommand\narrowdotp{\hspace{-0.12em}\dotp\hspace{-0.12em}}
\newcommand\narrowdotdotp{\hspace{-0.12em}\dotdotp\hspace{-0.12em}}

\newcommand\dotdotdotp{\dotp\hspace{-0.16em}\dotp\hspace{-0.16em}\dotp\hspace{0.08em}}

\newcommand\T{{\mathpalette\raiseT\intercal}} % symbol for transpose
\newcommand\raiseT[2]{\hspace{-0.02em}\raisebox{0.4ex}{$#1#2$}}

% symbol for X-companion
\newcommand\Xcompanion{{\mathpalette\lowerXcompanion\times}}
\newcommand\lowerXcompanion[2]{\hspace{-0.16ex}\raisebox{-0.2ex}{$#1\bm{#2}$}\hspace{-0.24ex}}

\makeatletter
\newcommand{\raisemath}[1]{\mathpalette{\raisem@th{#1}}}
\newcommand{\raisem@th}[3]{\raisebox{#1}{$#2#3$}}
\makeatother

% symbol for trace dot
\newcommand\tracedot{\hspace{-0.1ex}\raisemath{-0.4ex}{\scalebox{.64}{$\bullet$}}\hspace{-0.1ex}}

\newcommand\levicivita{\raisemath{0.2ex}{\mathrm{\in}}}
%%\newcommand\levicivitatensor{{^3\hspace{-0.05ex}\pmb{\bm{\levicivita}}}}
%%\newcommand\levicivitatensor{{^3\hspace{-0.05ex}\pmb{\scalebox{1.25}[1]{$\textsf{Є}$}}}\hspace{.05ex}}
\newcommand\levicivitatensor{{^3\hspace{-0.25ex}\bm{\epsilon}}\hspace{.05ex}}

\newcommand\stiffnesstensorA{\hspace{.08ex}{^4\hspace{-0.72ex}\bm{\mathcal{A}}}\hspace{.08ex}}
\newcommand\stiffnesstensorB{{^4\hspace{-0.2ex}\bm{\mathcal{B}}}}
\newcommand\stiffnesstensorC{{^4\bm{\mathcal{C}}}}

\newcommand\stiffnesstensor{\stiffnesstensorA}
\newcommand\pliabilitytensor{\stiffnesstensorB}

\newcommand\rotationtensor{\bm{O}}

\usepackage{accents}

\makeatletter

\newcommand{\scircabove}{%
	\hbox{\fontfamily{lmr}\fontsize{.64\dimexpr(\f@size pt)}{0}\selectfont$\,\circ$}}
\DeclareRobustCommand{\mathcircabove}{\accentset{\scircabove}}

\newcommand{\sdotabove}{%
	\hbox{\fontfamily{lmr}\fontsize{.5\dimexpr(\f@size pt)}{0}\selectfont$\,\bullet$}}
\DeclareRobustCommand{\mathdotabove}{\accentset{\sdotabove}}
\newcommand{\sdotdotabove}{%
	\hbox{\fontfamily{lmr}\fontsize{.5\dimexpr(\f@size pt)}{0}\selectfont$\,\bullet\hspace{-0.25ex}\bullet$}}
\DeclareRobustCommand{\mathdotdotabove}{\accentset{\sdotdotabove}}

\newcommand{\stwoabove}{%
	\hbox{\fontfamily{lmr}\fontsize{.5\dimexpr(\f@size pt)}{0}\selectfont$\hspace{.25ex}2$}}
\DeclareRobustCommand{\mathtwoabove}{\accentset{\stwoabove}}

\makeatother

\newcommand\mathdotbelow[1]{\underaccent{\scalebox{.5}{$\bullet$}}{#1}}

\newcommand\cosinematrix[1]{o_{#1}} % or \alpha
\newcommand\cosinematrixdotted[1]{\mathdotabove{o}_{#1}}

\newcommand\mathboldN{\scalebox{0.9}[1]{\ensuremath{\bm{N}\hspace{-0.1em}}}}
\newcommand\mathboldM{\scalebox{0.9}[1]{\ensuremath{\bm{M}\hspace{-0.1em}}}}
\newcommand\mathboldU{\scalebox{0.9}[1]{\ensuremath{\bm{U}\hspace{-0.1ex}}}}
\newcommand\mathboldQ{\scalebox{0.9}[1]{\ensuremath{\bm{Q}\hspace{-0.1ex}}}}

\newcommand\mathboldae{\pmb{\textit{æ}}}

\newcommand\expminusone{\hspace{.16ex}\scalebox{0.8}[1]{\hbox{--}}\raisemath{.1ex}{1}}
\newcommand\expminusT{\hspace{.16ex}\scalebox{0.8}[1]{\hbox{--}}\T}

\newcommand\expexternal{\raisemath{.1ex}{\smash{\hspace{.12ex}(\hspace{-0.1ex}e\hspace{-0.1ex})}\hspace{-0.16ex}}}
\newcommand\expinternal{\raisemath{.1ex}{\smash{\hspace{.12ex}(i)}\hspace{-0.16ex}}}

\newcommand\internalwork{W^{\hspace{-0.12ex}\expinternal}}
\newcommand\externalwork{W^{\hspace{-0.12ex}\expexternal}}

\newcommand\onehalf{\raisemath{.02em}{\frac{\raisemath{-0.16em}{1}}{2}}}
\newcommand\smalldisplaystyleonehalf{\scalebox{0.8}{$\displaystyle \onehalf$}\hspace{.1ex}}

\newcommand\onefourth{\raisemath{.02em}{\frac{\raisemath{-0.16em}{1}}{4}}}
\newcommand\smalldisplaystyleonefourth{\scalebox{0.8}{$\displaystyle \onefourth$}\hspace{.1ex}}

\usepackage{scalerel}

\usepackage{paralist}
\usepackage{booktabs}

%%\usepackage{indentfirst}

\usepackage{setspace}
%%\onehalfspacing
\setstretch{1}

\setlength{\parskip}{0.8mm minus0.2mm} % space between paragraphs
\setlength{\parindent}{4ex} % offset of first line
\newlength\negparindent
\setlength{\negparindent}{-\parindent}

\def\changemargin#1#2{\list{}{\rightmargin#2\leftmargin#1}\item[]}
\let\endchangemargin=\endlist

\usepackage{blindtext}
\usepackage{fancybox}
\usepackage{rotating}
\usepackage{placeins}
\usepackage{wasysym}
\usepackage{etoolbox}

\usepackage{tikz}
\usepackage{tikz-3dplot} % it needs tikz-3dplot.sty in same folder
\usetikzlibrary{calc}
%%\usetikzlibrary{topaths}
\usetikzlibrary{angles, quotes}
\usetikzlibrary{arrows, arrows.meta}
\usetikzlibrary{decorations.markings, decorations.pathmorphing, decorations.pathreplacing}

\usepackage{tkz-euclide}
\usetkzobj{all}

\usepgfmodule{nonlineartransformations}
\usepgflibrary{curvilinear}

\newcommand{\tikzsl}[1]{\tikz[baseline=(sl.base)] \node [xslant=0.21, inner sep=0pt, outer sep=0pt] (sl) {#1};}

\newcommand\mathtau{\scalebox{1}[1.06]{\tikzsl{\ensuremath{\uptau\hspace{-0.1ex}}}}}
\newcommand\mathboldtau{\scalebox{1}[1.06]{\tikzsl{\ensuremath{\bm{\uptau}\hspace{-0.1ex}}}}}

\newcommand\cauchystress{\mathboldtau}

\newcommand\firstpiolakirchhoffstress{\bm{T}}
\newcommand\secondpiolakirchhoffstress{\bm{S}}

\newcommand\mathboldepsilon{\scalebox{1}[1.06]{\hspace{-0.2ex}\tikzsl{\scalebox{1.2}[1]{\ensuremath{\bm{\upvarepsilon}\hspace{-0.1ex}}}}}}

\newcommand\mathboldPhi{\scalebox{1.06}[1]{\tikzsl{\ensuremath{\hspace{-0.16ex}\bm{\Upphi}\hspace{-0.32ex}}}}}

\newcommand{\fieldofdisplacements}{\bm{u}}
\newcommand{\fieldofrotations}{\bm{\varphi}} %% {\bm{\theta}}

\newcommandx*{\tikzcancel}[2][1=black]{%
	\tikz[baseline=(tocancel.base)]{%
		\node [inner sep=-1pt, outer sep=0pt] (tocancel) {#2};
		\draw [line width=0.25mm, #1] (tocancel.south west) -- (tocancel.north east);
	}%
}

\newcommandx*{\tikzbackcancel}[2][1=black]{%
	\tikz[baseline=(tocancel.base)]{%
		\node [inner sep=-1pt, outer sep=0pt] (tocancel) {#2};
		\draw [line width=0.25mm, #1] (tocancel.south east) -- (tocancel.north west);
	}%
}

\newcommandx*{\tikzcrosscancel}[2][1=black]{%
	\tikz[baseline=(tocancel.base)]{%
		\node [inner sep=-1pt, outer sep=0pt] (tocancel) {#2};
		\draw [line width=0.25mm, #1] (tocancel.south west) -- (tocancel.north east);
		\draw [line width=0.25mm, #1] (tocancel.south east) -- (tocancel.north west);
	}%
}

\newcommand\widearc[1]{%
	\tikz[baseline=(wideArcAnchor.base)]{%
		\node [inner sep=0] (wideArcAnchor) {$#1$};
		\coordinate (wideArcAnchorA) at ($ (wideArcAnchor.north west) + (0.2ex, 0.32ex) $);
		\coordinate (wideArcAnchorB) at ($ (wideArcAnchor.north east) + (-0.2ex, 0.32ex) $);
		\draw [line width=.5pt, line cap=round, out=45, in=135]
			(wideArcAnchorA) to (wideArcAnchorB) ;
}}

\pgfdeclaredecoration{plain brace}{brace}
{%
  \state{brace}[width=+\pgfdecoratedremainingdistance, next state=final]
  {%
    \pgfpathmoveto{\pgfpointorigin}%
    {%
      \pgfpathcurveto%
        {\pgfqpoint{.15\pgfdecorationsegmentamplitude}{.3\pgfdecorationsegmentamplitude}}%
        {\pgfqpoint{.5\pgfdecorationsegmentamplitude}{.5\pgfdecorationsegmentamplitude}}%
        {\pgfqpoint{\pgfdecorationsegmentamplitude}{.5\pgfdecorationsegmentamplitude}}%
    }%
    \pgftransformxshift{+\pgfdecoratedremainingdistance}%
    \pgfpathlineto{\pgfqpoint{-\pgfdecorationsegmentamplitude}{.5\pgfdecorationsegmentamplitude}}%
    {%
      \pgfpathcurveto%
        {\pgfqpoint{-.5\pgfdecorationsegmentamplitude}{.5\pgfdecorationsegmentamplitude}}%
        {\pgfqpoint{-.15\pgfdecorationsegmentamplitude}{.3\pgfdecorationsegmentamplitude}}%
        {\pgfqpoint{0\pgfdecorationsegmentamplitude}{0\pgfdecorationsegmentamplitude}}%
    }%
  }%
  \state{final}{}%
}

\newcommand{\tikzmark}[1]{\tikz[overlay, remember picture] \node (#1) {};}

% tweak these if you wish
\newcommand*{\BraceAmplitude}{0.4em}
\newcommand*{\VerticalOffset}{0.64ex}
\newcommand*{\HorizontalOffset}{0em}
\newcommand*{\TextOffset}{0.4ex}

\NewDocumentCommand{\AddUnderBrace}{%
	O{}	% #1 = draw options
	O{0,0} % #2 = [x,y] offset for brace
	O{}	% #3 = node options
	O{}	% #4 = optional brace options
	m	% #5 = left tikzmark
	m	% #6 = right tikzmark
	m	% #7 = text to place underbrace
}{%
\begin{tikzpicture}[overlay, remember picture]
	\draw [decoration={plain brace, amplitude=\BraceAmplitude, mirror, #4}, decorate, thick, draw=black!80, text=black, #1]
		($ (#5.base) + (-\HorizontalOffset,-\VerticalOffset) + (#2) $) --
		($ (#6.base) + (\HorizontalOffset,-\VerticalOffset) + (#2) $)
	node [below, yshift=-\TextOffset, midway, #3] {#7};
\end{tikzpicture}%
}

\NewDocumentCommand{\AddOverBrace}{%
	O{}	% #1 = draw options
	O{0,0} % #2 = [x,y] offset for brace
	O{}	% #3 = node options
	O{}	% #4 = optional brace options
	m	% #5 = left tikzmark
	m	% #6 = right tikzmark
	m	% #7 = text to place underbrace
}{%
\begin{tikzpicture}[overlay, remember picture]
	\draw [decoration={plain brace, amplitude=\BraceAmplitude, #4}, decorate, thick, draw=black!80, text=black, #1]
		($ (#5.north) + (-\HorizontalOffset,\VerticalOffset+0.25em) + (#2) $) --
		($ (#6.north) + (\HorizontalOffset,\VerticalOffset+0.25em) + (#2) $)
	node [above, yshift=\TextOffset, midway, #3] {#7};
\end{tikzpicture}%
}

\newcommandx*\circled[2][1=black!5]{\tikz[baseline=(char.base)]{%
	\node[shape=circle, fill=#1, inner sep=2pt] (char) {#2};}}

% tune headers and footers

\setlength{\headheight}{12pt}
\setlength{\footskip}{0pt}

\usepackage{fancyhdr}

\fancypagestyle{plain}{%
\fancyhf{} % clear all header and footers
\fancyhead[LE,RO]{\vspace*{.66em}\circled[yellow!5]{\small\thepage}}
\fancyheadoffset[LE]{7.5mm} % move page numbers left on even pages
\fancyheadoffset[RO]{7.5mm} % move page numbers right on odd pages
\renewcommand{\headrulewidth}{0pt} % no line in header
\renewcommand{\footrulewidth}{0pt} % no line in footer
}

\pagestyle{fancy}
\fancyhf{}
\fancyhead[LE,RO]{\vspace*{.66em}\circled[yellow!5]{\small\thepage}}
\fancyheadoffset[LE]{7.5mm}
\fancyheadoffset[RO]{7.5mm}
\renewcommand{\headrulewidth}{0pt}
\renewcommand{\footrulewidth}{0pt}

\usepackage{array}
\newcolumntype{C}[1]{>{\centering\arraybackslash}m{#1}<{}} % equal-width centered column

% setup captions for figures and tables
\usepackage[format=plain]{caption}

\captionsetup[figure]{%
font={small,it},labelfont=small,%
labelsep=newline,justification=centering,singlelinecheck=off,%
aboveskip=4mm,belowskip=2.5mm}

\usepackage{chngcntr}
\counterwithout{figure}{chapter}
\counterwithout{table}{chapter}

\addto\captionsenglish{\renewcommand{\chaptertitlename}{Chapter}}
\addto\captionsrussian{\renewcommand{\chaptertitlename}{Глава}}
\addto\captionsukrainian{\renewcommand{\chaptertitlename}{Розділ}}

\addto\captionsenglish{\def\chaptershortname{chapter}}
\addto\captionsrussian{\def\chaptershortname{гл.}}
\addto\captionsukrainian{\def\chaptershortname{розд.}}

\addto\captionsenglish{\renewcommand{\contentsname}{Contents}}
\addto\captionsrussian{\renewcommand{\contentsname}{Содержание}}
\addto\captionsukrainian{\renewcommand{\contentsname}{Змiст}}

\addto\captionsenglish{\renewcommand{\figurename}{figure}}
%%\addto\captionsrussian{\renewcommand{\figurename}{фигура}}
\addto\captionsrussian{\renewcommand{\figurename}{рисунок}}
\addto\captionsukrainian{\renewcommand{\figurename}{малюнок}}

\addto\captionsenglish{\def\figureshortname{fig.}}
%%\addto\captionsrussian{\def\figureshortname{фиг.}}
\addto\captionsrussian{\def\figureshortname{рис.}}
\addto\captionsukrainian{\def\figureshortname{мал.}}

\addto\captionsenglish{\def\wordforbibliography{Bibliography}}
\addto\captionsrussian{\def\wordforbibliography{Библиография}}

\newcommand\pararef[1]{\hyperref[#1]{\S\,\ref*{#1}}}
\newcommand\chapdotpararef[2]{\hyperref[#2]{\S\,\ref*{#1}.\ref*{#2}}}
\newcommand\chapref[1]{\hyperref[#1]{\chaptershortname\:\ref*{#1}}}

\newcommand\figref[1]{\figureshortname\:\ref{#1}}

\newcommandx*\customref[2][1=]{\hyperref[#2]{#1\ref*{#2}}}

\newcommand\eqrefwithchapdotpara[3]{\text{(\hyperref[#1]{\ref*{#1}},\:\chapdotpararef{#2}{#3})}}

\captionsetup[table]{%
%labelsep=newline,
labelsep=period,justification=centering,singlelinecheck=off}

\newcommandx*{\bibauthor}[3][2=,3=]{\textbf{\hbox{#1}%
\ifthenelse{\equal{#2}{}}{}{~\hbox{#2}}%
\ifthenelse{\equal{#3}{}}{}{\:\hbox{#3}}%
}}

\sloppy

%%\usepackage{showframe} % uncomment to show frames

% tune lists

\usepackage[shortlabels]{enumitem}
\setlist{nolistsep} % no vertical space between list items

%%\renewcommand{\labelitemi}{\raise .36ex\hbox{\tiny${\bullet}$}}
\renewcommand{\labelitemi}{\pmb{\checkmark}}

\newlength\mylistlabelsep
\setlength{\mylistlabelsep}{1.2mm} %% 2mm

\newlength\listlabelwithsep
\addtolength{\listlabelwithsep}{\mylistlabelsep} % add \labelsep
\addtolength{\listlabelwithsep}{\widthof{\labelitemi}}

\newlength\listleftindent
\setlength{\listleftindent}{\parindent}
\addtolength{\listleftindent}{\listlabelwithsep}

\makeatletter
\renewcommand{\@listI}{%
	\leftmargin=\listleftindent
	\rightmargin=0mm
	\labelsep=\mylistlabelsep
	\itemindent=0mm
	\listparindent=0mm
	\parskip=0mm
	\topsep=1mm
	\partopsep=0mm
	\parsep=0mm
	\itemsep=0.5mm}
\makeatother

%%\usepackage[defaultlines=2,all]{nowidow}
\raggedbottom
\widowpenalty=10000 % avoid widows
\clubpenalty=10000 % avoid orphans (clubs)

\makeatletter
\@beginparpenalty=10000 % avoid page breaks before lists
\makeatother

\hyphenpenalty=800 % reduce number of hyphenation

% --------------------------------------------------

% tune footnotes

\renewcommand*\footnoterule{} % no line before footnotes

%%\renewcommand{\thefootnote}{\arabic{footnote})}

\usepackage[symbol, bottom]{footmisc} % use symbols instead of numbers and place footnotes at the very bottom of page
\setlength{\skip\footins}{1em plus1em} % \footins is the space between the text body and the footnotes

\setlength{\footnotesep}{\baselineskip} % space between footnotes

\makeatletter
\def\@fnsymbol#1{\ensuremath{\ifcase #1 %
	\or {*} %
	\or {*}{*} %
	\or {*}{*}{*} %
	\or {*}{*}{*}{*} %
	\else {\infty} %% \else \@ctrerr
\fi}}
\def\@makefnmark{\raisebox{.8ex}{\hbox{\normalfont\@thefnmark}}}
\makeatother

\usepackage{perpage}
\MakePerPage{footnote} % restart \footnote numbering on every page

\let\originalfootnote\footnote
\renewcommand{\footnote}[1]{%
	\renewcommand\footnotesize\small%
	\originalfootnote{\hspace{.25em}\parbox[t]{0.88\textwidth}{#1}}}

% --------------------------------------------------

% tune bibliography

\makeatletter
\renewcommand*{\@biblabel}[1]{#1.}
\g@addto@macro\@openbib@code{\setlength{\itemsep}{.2em}}
\makeatother

\addto\captionsenglish{\def\bibliographyname{List of publications}}
\addto\captionsenglish{\renewcommand{\bibname}{List of publications}}

\addto\captionsrussian{\def\bibliographyname{Список публикаций}}
\addto\captionsrussian{\renewcommand{\bibname}{Список публикаций}}

\addto\captionsukrainian{\def\bibliographyname{Список публікацій}}
\addto\captionsukrainian{\renewcommand{\bibname}{Список публікацій}}

% --------------------------------------------------

\renewcommand{\bfdefault}{b} % make bold font narrower
%%\renewcommand{\bfdefault}{sbc}

\DeclareTextFontCommand{\textboldoblique}{\fontseries{bx}\itshape}

\renewcommand{\labelenumi}{\arabic{enumi}) }
\renewcommand{\labelenumii}{\alph{enumii}) }
\renewcommand{\labelenumiii}{\roman{enumiii}) }

% tune titles

\makeatletter
\patchcmd{\chapter}{\if@openright\cleardoublepage\else\clearpage\fi}{}{}{}
\makeatother

\usepackage{titlesec}

\renewcommand{\thechapter}{\arabic{chapter}}
\renewcommand{\thesection}{\arabic{section}}
\renewcommand{\thesubsection}{\arabic{section}.\arabic{subsection}.}

\newcommand{\mychapterfont}{\normalfont\fontsize{12}{13}}
\newcommand{\mysectionfont}{\normalfont\normalsize\bfseries}
\newcommand{\mysubsectionfont}{\normalfont\normalsize\itshape}

\titleformat{\chapter}[display]
	{\mychapterfont\filleft}
	{\vspace{0pt}\textit{\chaptertitlename~\thechapter}}
	{0.4mm}
	{\bfseries\MakeUppercase}

\titlespacing*{\chapter}
	{\parindent}% left
	{12mm}% before
	{8mm}% after
	[\parindent]%right

\titleformat{\section}
	{\mysectionfont\filleft}{\S\:\thesection.}{.8ex}{}

\titlespacing*{\section}
	{\parindent}% left
	{8mm}% before
	{2mm}% after
	[\parindent]%right

\titleformat{\subsection}
	{\mysubsectionfont\filleft}{\thesubsection}{.8ex}{}

\titlespacing*{\subsection}
	{\parindent}% left
	{4mm}% before
	{1mm}% after
	[\parindent]%right

%%%\usepackage[format=hang,labelsep=period,justification=raggedright,singlelinecheck=true]{caption}

% --------------------------------------------------

% tune table of contents

\usepackage{tocloft}

\setcounter{tocdepth}{1} % chapters & sections, no subsections

\renewcommand{\cfttoctitlefont}{\hfill\mychapterfont\bfseries\MakeUppercase}
\renewcommand{\cftaftertoctitle}{\hspace*{5ex}}
\setlength\cftaftertoctitleskip{2em}

\setlength{\cftbeforechapskip}{.64em}
\setlength{\cftbeforesecskip}{.1em} % {\parskip}

\renewcommand{\cftchapafterpnum}{\nopagebreak\vspace{.2em}}
\preto\section{%
	\ifnum\value{section}=0\addtocontents{toc}{\vspace{.2em}}\fi
}

\makeatletter
\patchcmd{\l@chapter}
	{\leavevmode}
	{\leavevmode\renewcommand{\numberline}[1]{\hbox{\normalfont\emph{\@cftbsnum ##1\@cftasnum}}\@cftasnumb}}
	{}{}
\makeatother

\renewcommand{\cftchappresnum}{\chaptertitlename~}
\renewcommand{\cftchapaftersnum}{.~\hspace{.16ex}}

\renewcommand{\cftsecpresnum}{\hfill\hspace{-0.5em}\S\:}
\renewcommand{\cftsecaftersnum}{.~\hspace{.16ex}}
\newlength\tocsectionlength
\settowidth\tocsectionlength{\cftsecpresnum\cftsecaftersnum}
\addtolength\cftsecnumwidth{\tocsectionlength}
\addtolength\cftsecnumwidth{-2.4ex}

\renewcommand{\cftpnumalign}{l} % align page numbers left

\makeatletter
\renewcommand{\@pnumwidth}{3ex}
\renewcommand{\@tocrmarg}{1.5cm}
\makeatother

\renewcommand{\cftdotsep}{1.2}

%%\renewcommand{\cftchapleader}{\cftdotfill{\cftdotsep}\hspace{0.5ex}}
\renewcommand{\cftchapleader}{\hfill}
\renewcommand{\cftchapfont}{\fontsize{10}{11}\bfseries}
\renewcommand{\cftchappagefont}{\fontsize{11}{12}\bfseries}

\renewcommand{\cftsecleader}{\textcolor{black!40}{\cftdotfill{\cftdotsep}}\hspace{0.5ex}}
\renewcommand{\cftsecfont}{\fontsize{10}{11}\mdseries}
\renewcommand{\cftsecpagefont}{\fontsize{10}{11}\mdseries}

% --------------------------------------------------

\usepackage[many]{tcolorbox} %%[most]{tcolorbox}
\tcbset{
	frame code={}
	center title,
	left = 0pt,
	right = 0pt,
	top = 0pt,
	bottom = 0pt,
	colback = gray!10,
	colframe = white,
	width = \dimexpr\textwidth\relax,
	enlarge left by = 0mm,
	boxsep = 5pt,
	arc = 0pt, outer arc = 0pt}

\setcounter{totalnumber}{100} % increase number of floats on one page

\newcommand{\lquote}[0]{``}
\newcommand{\rquote}[0]{''}

\newcommand{\altlquote}[0]{<<}
\newcommand{\altrquote}[0]{>>}

\newcommand{\inquotes}[1]{\lquote{#1}\rquote}
\newcommand{\inaltquotes}[1]{\altlquote{#1}\altrquote}

\newcommandx*{\inquotesx}[2][2=]{\lquote{#1}\hbox{\rquote%
\ifthenelse{\equal{#2}{.}}{\hspace{-0.4ex}.}{}%
\ifthenelse{\equal{#2}{,}}{\hspace{-0.4ex},}{}%
\ifthenelse{\equal{#2}{---}}{\hspace{-0.4ex}~---}{}%
}}

\newcommand{\sectionunnumbered}[1]%
{%
	\phantomsection
	\section*{#1}
	\addcontentsline{toc}{section}{#1}
}

\newcommand{\chapterunnumbered}[1]%
{%
	\phantomsection
	\chapter*{#1}
	\addcontentsline{toc}{chapter}{#1}
}

\newcommandx*{\loosetexttr}[2][1=50]{\textls[#1]{#2}}

\newcommand{\replica}[1]{\noindent ---~{#1}}
\newcommand{\remark}[1]{\emph{({#1})}}

\newcommand{\bigvspace}[0]{\vspace{0.75\vsize}}

\setlength{\arraycolsep}{3.2pt} % default: 5pt
