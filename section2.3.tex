\en{\section{Balance of momentum, rotational momentum, and~energy}}

\ru{\section{Баланс импульса, момента импульса и~энергии}}

\begin{otherlanguage}{russian}

Эти уравнения баланса
могут быть связаны
со~свойствами
пространства
\en{and}\ru{и}~времени~\cite{landau.lifshitz-shortcourse}.
Сохранение импульса
(количества движения)
в~з\'{а}мкнутой~(изолированной)%
\footnote{\emph{З\'{а}мкнутая~(изолированная) система}
это такая система частиц,
которые
взаимодействуют
\en{only}\ru{только}
друг с~другом,
\en{but}\ru{но}
ни с~какими другими телами.}\hspace{-0.25ex}
системе выводится из~однородности пространства \emph{(при любом параллельном переносе\:--- трансляции\:--- замкнутой системы как целого свойства этой системы не~меняются)}.
Сохранение момента импульса\:--- следствие изотропии пространства \emph{(свойства замкнутой системы не~меняются при любом повороте этой системы как целого)}.
Энергия~же
изолированной системы
сохраняется,
так~как
время однородно%
%
\footnote{%
Характеристики
\inquotes{однородность}
\en{and}\ru{и}~\inquotes{изотропность}
пространства,
\inquotes{однородность}
времени
не~фигурируют среди аксиом
классической механики.
}\hspace{-0.25ex} % end of footnote
(\en{energy}\ru{энергия}
${\mathrm{E} \hspace{.1ex} \equiv \kineticenergyinmechanics \hspace{.1ex} (q, \mathdotabove{q} \hspace{.2ex}) \hspace{-0.2ex} + \hspace{-0.1ex} \potentialenergyinmechanics (q)}$
такой системы
не~зависит явно от~времени).

\en{The~balance equations}\ru{Уравнения баланса}
\en{can be}\ru{могут быть}
\en{derived}\ru{выведены}
\en{from}\ru{из}
\en{the~principle of~virtual work}\ru{принципа виртуальной работы}~\eqref{discrete:principleofvirtualwork}.
Перепишем его,
выделив
внешние силы~${\bm{F}^{\smthexternal}_{\hspace{-0.16ex}k}}$
и~виртуальную работу
внутренних сил
${\variation{\internalwork} \hspace{-0.1ex} = \hspace{-0.2ex} \scalebox{.8}{$ \displaystyle \underset{\raisemath{.25ex}{\smash{k,\hspace{.1ex}j}}}{\sum} $} \hspace{.2ex} \bm{F}^{\smthinternal}_{\hspace{-0.16ex}kj} \hspace{-0.1ex} \dotp \variation{\hspace{.1ex}\locationvector_{\hspace{-0.1ex}k}} \hspace{.1ex}}$

\nopagebreak\vspace{-1em}
\begin{equation}\label{discrete:principleofvirtualwork.externalinternal}
\scalebox{.9}{$ \displaystyle \sum_{\smash{k}} $}
\Bigl( \hspace{-0.25ex} \bm{F}^{\smthexternal}_{\hspace{-0.16ex}k} \hspace{-0.1ex} - m_k \mathdotdotabove{\locationvector}_{\hspace{-0.1ex}k} \hspace{-0.2ex} \Bigr) \hspace{-0.32ex} \dotp \variation{\hspace{.1ex}\locationvector_{\hspace{-0.1ex}k}} \hspace{-0.1ex}
+ \variation{\internalwork} \hspace{-0.1ex} = 0
\hspace{.1ex} .
\vspace{-0.25em}\end{equation}

\end{otherlanguage}

\vspace{-0.1em}
\en{It’s assumed that}\ru{Предполагается, что}
\en{internal forces}\ru{внутренние силы}
\en{don’t do any work}\ru{не~делают никакой работы}
\en{on virtual displacements}\ru{на виртуальных смещениях}
\en{of~a~system}\ru{системы}
\en{as a~rigid whole}\ru{как жёсткого целого}
(${\constvarvector{\hspace{-0.1ex}\bm{\rho}}}$
\en{and}\ru{и}~${\constvarvector{\bm{o}}}$\en{ are}\ru{\:---}
\en{some}\ru{некоторые}
\en{constant vectors}\ru{постоянные векторы}\ru{,}
\en{describing}\ru{описывающие}
\en{translation}\ru{трансляцию}
\en{and}\ru{и}~\en{rotation}\ru{поворот})

\nopagebreak\vspace{-0.2em}
\begin{equation}\label{assumptionforvirtualwork}
\begin{array}{l}
\variation{\hspace{.1ex}\locationvector_{\hspace{-0.1ex}k}} \hspace{-0.16ex}
= \constvarvector{\hspace{-0.1ex}\bm{\rho}} \hspace{.2ex} + \hspace{.12ex} \constvarvector{\bm{o}} \hspace{-0.2ex} \times \hspace{-0.1ex} \locationvector_{\hspace{-0.1ex}k}
\hspace{.1ex} ,
\\
\constvarvector{\hspace{-0.1ex}\bm{\rho}} = \boldconstant \hspace{.1ex} , \:
\constvarvector{\bm{o}} = \boldconstant
\end{array}
\hspace{.3ex} \Rightarrow \hspace{.6ex}
\variation{\internalwork} \hspace{-0.1ex} = 0 \hspace{.1ex}.
\end{equation}

\vspace{-0.1em}
\en{Premises}\ru{Предпосылки}
\en{and}\ru{и}~\en{considerations}\ru{соображения}
\en{for this assumption}\ru{для~этого предположения}
\en{are as follows}\ru{следующие}.
%
\en{The~first}\ru{Первое}\:---
\en{for}\ru{для}
\en{the~case}\ru{случая}
\en{of~potential}\ru{потенциальных},
\en{such as elastic}\ru{таких как упругие},
\en{forces}\ru{сил}.
\en{A~variation}\ru{Вариация}
\en{of~the~work}\ru{работы}
\en{of~potential internal forces}\ru{потенциальных внутренних сил}~$\internalwork$
\en{is}\ru{есть}
\en{a~variation}\ru{вариация}
\en{of~the~potential}\ru{потенциала}~$\potentialenergyinmechanics$
\en{with the~opposite sign}\ru{с~противоположным знаком},

\nopagebreak\vspace{-0.4em}
\begin{equation}\label{variationofworkofpotentialinternalforces}
\variation{\internalwork} = - \hspace{.1ex} \variation{\potentialenergyinmechanics}
\hspace{.1ex} .
\end{equation}

\vspace{-0.4em}\noindent
\en{And it’s quite obvious that}\ru{И~весьма очевидно, что}
$\potentialenergyinmechanics$~\en{alters}\ru{меняется}
\en{only}\ru{только}
\en{by deforming}\ru{деформированием}.
%
\en{The~second consideration}\ru{Второе соображение}\:---
\en{the~internal forces}\ru{внутренние силы}
\en{are balanced}\ru{сбалансированы}
\en{in the~sense that}\ru{в~том смысле, что}
\en{the~net vector}\ru{суммарный \inquotes{нетто} вектор}~(\en{the~resultant force}\ru{результирующая сила})
\en{and }\ru{и~}%
\en{the~net moment}\ru{суммарный \inquotes{нетто} момент}~(\en{the~resultant couple}\ru{результирующая пара})
\en{are}\ru{равны}~$\zerovector$\hbox{\hspace{.1ex},}

\eqref{actionreactionprinciple.fordiscretepoints}~\&~\eqref{iternalinteractionsarecentral.betweenparticles}

\begin{equation*}
\sum \ldots
\end{equation*}

...

\begin{otherlanguage}{russian}

Принимая~\eqref{assumptionforvirtualwork}
и~подставляя в~\eqref{discrete:principleofvirtualwork.externalinternal}
сначала ${\variation{\hspace{.1ex}\locationvector_{\hspace{-0.1ex}k}} \hspace{-0.16ex} = \hspace{-0.08ex} \constvarvector{\hspace{-0.1ex}\bm{\rho}}}$
(трансляция),
а~затем ${\variation{\hspace{.1ex}\locationvector_{\hspace{-0.1ex}k}} \hspace{-0.16ex} = \hspace{-0.08ex} \constvarvector{\bm{o}} \hspace{-0.2ex} \times \hspace{-0.1ex} \locationvector_{\hspace{-0.1ex}k}}$
(поворот),
получаем баланс импульса~(...)
\en{and}\ru{и}~%
баланс момента импульса~(...).

\end{otherlanguage}

...

