\subsection{What is abstract? What does math do?}

\en{Mathematics}\ru{Математика}\en{, or math for short,}
\en{is abstract}\ru{абстрактна}.
\en{Abstract is the~adjective of~math}\ru{Абстрактная\:--- прилагательное математики},
\en{math is the~noun of~abstract}\ru{математика\:--- существительное абстрактной}.
\inquotesx{Abstract}[,]
\inquotes{theoretical}
\en{and}\ru{и}~\inquotes{mathematical}
\en{are}\ru{это}
\en{synonyms}\ru{синонимы}.
When we do math, we are playing a~game in a~world of imagination.

\en{Numbers}\ru{Ч\'{и}сла}
\en{are not}\ru{это не}
\en{real entities}\ru{реальные сущности}.
\en{They are}\ru{Они\:---}
\en{purely}\ru{чисто}
\en{imaginary}\ru{воображаемые}
\en{concepts}\ru{понятия}.
\en{We}\ru{Мы}
\en{cannot}\ru{не~можем}
\en{experience, sensate}\ru{почувствовать, ощутить}
\en{numbers}\ru{ч\'{и}сла}.
\en{One}\ru{Кто\hbox{-}то}
\en{can}\ru{может}
\en{compose}\ru{сочинять}
\en{stories}\ru{истории}
\en{about them}\ru{про них},
\en{such as}\ru{такие как}
${1+1=2}$.
\en{But}\ru{Но}
\en{no one can ever}\ru{никто никогда не~сможет}
\en{feel, perceive}\ru{чувствовать, ощущать}
\en{such an~operation}\ru{такую операцию}\ru{,}
\en{since}\ru{поскольку}
\en{there are no such thing}\ru{не~существует такого}
\en{as}\ru{как}
\emph{\en{one}\ru{один}}
\en{or}\ru{или}
\emph{\en{two}\ru{два}}.

.................


\section{\en{The ancient but intuitive geometry}\ru{Древняя, но интуитивно понятная геометрия}}

\vspace{.5em}

\subsection{Points}

{\small
\setlength{\parindent}{0pt}

\nopagebreak%
\begin{multicols}{2}
%
\makebox[\columnwidth]{\href{http://www.physics.ntua.gr/~mourmouras/euclid/book1/elements1.html}{\textgreek{Στοιχεῖα Εὐκλείδου α΄}}}\\
\makebox[\columnwidth]{\href{http://www.physics.ntua.gr/~mourmouras/euclid/book1/elements1.html}{\textgreek{Βιβλίον I}}}\\[.3em]
\textgreek{Ὅροι κγ΄} \textgreek{α΄}\:(1)

\textgreek{Σημεῖόν ἐστιν, οὗ μέρος οὐθέν.}

\columnbreak
%
\makebox[\columnwidth]{Euclid’s Elements}\\
\makebox[\columnwidth]{Book I}\\[.3em]
Definition \textgreek{α΄}\:(1)

A point is that which has no part.
%
\end{multicols}
\par}

The description of a~point, \inquotesx{\emph{that which has no part}}[,] shows that Euclid imagines a~point as having no width, length, or breadth, but as an indivisible location.

\subsection{Linear lines, curved and straight}

{\small
\setlength{\parindent}{0pt}

\nopagebreak%
\begin{multicols}{2}
%
\makebox[\columnwidth]{\href{http://www.physics.ntua.gr/~mourmouras/euclid/book1/elements1.html}{\textgreek{Στοιχεῖα Εὐκλείδου α΄}}}\\
\makebox[\columnwidth]{\href{http://www.physics.ntua.gr/~mourmouras/euclid/book1/elements1.html}{\textgreek{Βιβλίον I}}}\\[.3em]
\textgreek{Ὅροι κγ΄} \textgreek{β΄}\:(2)

\textgreek{Γραμμὴ δὲ μῆκος ἀπλατές.}

\columnbreak
%
\makebox[\columnwidth]{Euclid’s Elements}\\
\makebox[\columnwidth]{Book I}\\[.3em]
Definition \textgreek{β΄}\:(2)

A line is breadthless length.
%
\end{multicols}
\par}

\inquotes{Line} is the second primitive term in the Elements.
\inquotes{Breadthless length} says that a~line will have one dimension, length, but it won’t have breadth.
The terms \inquotes{length} and \inquotes{breadth} are not defined in the Elements.

\subsection{A~relation between lines and points}

{\small
\setlength{\parindent}{0pt}

\nopagebreak%
\begin{multicols}{2}
%
\makebox[\columnwidth]{\href{http://www.physics.ntua.gr/~mourmouras/euclid/book1/elements1.html}{\textgreek{Στοιχεῖα Εὐκλείδου α΄}}}\\
\makebox[\columnwidth]{\href{http://www.physics.ntua.gr/~mourmouras/euclid/book1/elements1.html}{\textgreek{Βιβλίον I}}}\\[.3em]
\textgreek{Ὅροι κγ΄} \textgreek{γ΄}\:(3)

\textgreek{Γραμμῆς δὲ πέρατα σημεῖα.}

\columnbreak
%
\makebox[\columnwidth]{Euclid’s Elements}\\
\makebox[\columnwidth]{Book I}\\[.3em]
Definition \textgreek{γ΄}\:(3)

The ends of a~line are points.
%
\end{multicols}
\par}

This statement doesn’t mention how many ends a~line can have.

\subsection{Do straight lines exists?}

\en{The hypothesis}\ru{Гипотеза}
\en{of the existence}\ru{существования}
\en{of straight lines}\ru{прямых линий}.

\noindent
The existence of
Euclidean straight lines
in space.

{\small
\setlength{\parindent}{0pt}

\nopagebreak%
\begin{multicols}{2}
%
\makebox[\columnwidth]{\href{http://www.physics.ntua.gr/~mourmouras/euclid/book1/elements1.html}{\textgreek{Στοιχεῖα Εὐκλείδου α΄}}}\\
\makebox[\columnwidth]{\href{http://www.physics.ntua.gr/~mourmouras/euclid/book1/elements1.html}{\textgreek{Βιβλίον I}}}\\[.3em]
\textgreek{Ὅροι κγ΄} \textgreek{δ΄}\:(4)

\textgreek{Εὐθεῖα γραμμή ἐστιν, ἥτις ἐξ ἴσου τοῖς ἐφ' ἑαυτῆς σημείοις κεῖται.}

\columnbreak
%
\makebox[\columnwidth]{Euclid’s Elements}\\
\makebox[\columnwidth]{Book I}\\[.3em]
Definition \textgreek{δ΄}\:(4)

A straight line is a~line which lies evenly with the points on itself.
%
\end{multicols}
\par}

\en{To draw a~straight line}\ru{Начертить прямую линию}
\en{by hand}\ru{рукой}
\en{is absolutely impossible}\ru{абсолютно невозможно}.

\subsection{The existence of vectors. Do vectors exist?}

\subsection{Continuity of line}

\subsection{A~point of reference}

\subsection{Translation as the easiest kind of motion. Translations and vectors}

\subsection{Straight line and vector}

\en{A~(geometric) vector}\ru{Вектор~(геометрический)} \en{may be}\ru{может быть} \en{like}\ru{как} \en{a~straight line}\ru{прямая линия} \en{with an~arrow}\ru{со стрелкой} \en{at one of its ends}\ru{на одном из её концов}.
\textcolor{red}{Then} it is fully described (characterized) by the~magnitude and the~direction.

\en{Within}\ru{в}~\href{https://en.wikipedia.org/wiki/Abstract_algebra}{\en{the abstract algebra}\ru{абстрактной алгебре}}\en{,} \en{the word}\ru{слово} \emph{\en{vector}\ru{вектор}}\en{ is}\ru{\:---} \en{about any object}\ru{про любой объект}\ru{,} \en{which}\ru{который} \en{can be}\ru{может быть} \en{summed}\ru{суммирован} \en{with similar objects}\ru{с~подобными объектами} \en{and}\ru{и}~\en{scaled~(multiplied)}\ru{умножен~(scaled)} \en{by scalars}\ru{на скаляры}, \en{and }\ru{а~}\en{vector space}\ru{векторное пространство}\en{ is}\ru{\:---} \en{a~synonym}\ru{синоним} \en{of~}\en{linear space}\ru{линейного пространства}.
\en{Therefore}\ru{Поэтому} \en{I clarify that}\ru{я поясняю, что} \en{in this book}\ru{в~этой книге} \emph{\en{vector}\ru{вектор}} \en{is}\ru{есть} \en{nothing else than}\ru{не что иное, как} \en{three-dimensional}\ru{трёхмерный} \en{geometric}\ru{геометрический} (\textgreek{Ευκλείδειος}, \ru{Евклидов, }Euclidean) \en{vector}\ru{вектор}.

\textcolor{magenta}{Why are vectors always straight (linear)?}

(a) \textcolor{magenta}{Vectors are linear (straight), they cannot be curved.}

(b) \textcolor{magenta}{Vectors are neither straight nor curved.}
A~vector has the~magnitude and the~direction.
A~vector is not a~line or a~curve, albeit it can be represented by a~straight line.

\textcolor{red}{Vector can’t be thought of as a~line.}

\subsection{\en{Line which figures real numbers}\ru{Изображающая реальные числа линия}}

often just \inquotes{\en{number line}\ru{числовая прямая}}

\subsection{What is a distance?}

\subsection{\en{Plane and more dimensional space}\ru{Плоскость и более многомерное пространство}}

\subsection{\en{Distance on plane or more dimensional space}\ru{Расстояние на плоскости или в более-мерном пространстве}}

\subsection{What is an angle?}

angle~$\equiv$~\textcolor{magenta}{inclination /slope, slant/ of two lines}

two lines sharing a~common point are usually called intersecting lines

angle~$\equiv$~\textcolor{magenta}{the amount of rotation of line or plane within space}

angle~$\equiv$~\textcolor{blue}{the result of the dot product of two unit vectors gives angle’s cosine}

\subsection{\en{Differentiation of continuous into small differential chunks}\ru{Дифференциация непрерывного на м\'{а}лые дифференциальные кусочки}}

\en{small differential chunks}\ru{м\'{а}лые дифференциальные кусочки}

infinitesimal (infinitely small)

\begin{changemargin}{\parindent}{\parindent}
%%\vspace{-2.5em}
{\noindent\small
\setlength{\parskip}{\spacebetweenparagraphs}

\en{A~mention}\ru{Упоминание}
\en{of~tensors}\ru{тензоров}
\en{may scare away the~reader}\ru{может отпугнуть читателя},
\en{commonly avoiding needless complications}\ru{обычно избегающего ненужных сложностей}.
\en{Don’t be afraid}\ru{Не бойся}:
\en{tensors}\ru{тензоры}
\en{are used}\ru{используются}
\en{just}\ru{просто}
\en{due to their wonderful property}\ru{из\hbox{-}за своего чудесного свойства}
\en{of the invariance}\ru{инвариантности}\:---
\en{the independence}\ru{независимости}
\en{from a~coordinate system}\ru{от системы координат}.
%
\par}
\vspace{-1.4em}
\end{changemargin}

