\subsection{What is abstract? What does math do?}

Math is abstract.

Numbers are not real entities.
They are purely imaginary concepts.

When we do math, we are playing a~game in a~world of imagination.

We cannot experience numbers.
One can make up stories about them, such as ${1+1=2}$.
But no one can ever experience such an operation since there’s no such thing as \emph{one}.

\subsection{Points}

.................

{\small
\setlength{\parindent}{0pt}

\begin{leftverticalbar}%%[oversize]

Euclid’s Elements\\
Book I\\
Definition 1

A point is that which has no part.

\end{leftverticalbar}
\par}

The description of a~point, \inquotes{that which has no part}, indicates that Euclid will be treating a~point as having no width, length, or breadth, but as an indivisible location.

\subsection{Lines, curved and straight (linear)}

{\small
\setlength{\parindent}{0pt}

\begin{leftverticalbar}%%[oversize]

Euclid’s Elements\\
Book I\\
Definition 2

A line is breadthless length.

\end{leftverticalbar}
\par}

\inquotes{Line} is the second primitive term in the Elements.
\inquotes{Breadthless length} says that a~line will have one dimension, length, but it won’t have breadth.
The terms \inquotes{length} and \inquotes{breadth} are not defined in the Elements.

\subsection{A~relation between lines and points}

{\small
\setlength{\parindent}{0pt}

\begin{leftverticalbar}%%[oversize]

Euclid’s Elements\\
Book I\\
Definition 3

The ends of a~line are points.

\end{leftverticalbar}
\par}

This statement doesn’t mention how many ends a~line can have.

\subsection{Do straight lines exists?}

\en{The hypothesis}\ru{Гипотеза} \en{of the existence}\ru{существования} \en{of straight lines}\ru{прямых линий}.

The existence of
Euclidean straight lines
in space.

{\small
\setlength{\parindent}{0pt}

\begin{leftverticalbar}%%[oversize]

Euclid’s Elements\\
Book I\\
Definition 4

A straight line is a~line which lies evenly with the points on itself.

\end{leftverticalbar}
\par}

\en{To draw a~straight line}\ru{Начертить прямую линию} \en{by hand}\ru{рукой} \en{is absolutely impossible}\ru{абсолютно невозможно}.

\subsection{The existence of vectors. Do vectors exists?}

\subsection{Continuity of line}

\subsection{A~point of reference}

\subsection{Translation as the easiest kind of motion. Translations and vectors}

\subsection{Straight line and vector}

\en{A~(geometric) vector}\ru{Вектор~(геометрический)} \en{may be}\ru{может быть} \en{like}\ru{как} \en{a~straight line}\ru{прямая линия} \en{with an~arrow}\ru{со стрелкой} \en{at one of its ends}\ru{на одном из её концов}.
\textcolor{red}{Then} it is fully described (characterized) by the~magnitude and the~direction.

\en{Within}\ru{в}~\href{https://en.wikipedia.org/wiki/Abstract_algebra}{\en{the abstract algebra}\ru{абстрактной алгебре}}\en{,} \en{the word}\ru{слово} \emph{\en{vector}\ru{вектор}}\en{ is}\ru{\:---} \en{about any object}\ru{про любой объект}\ru{,} \en{which}\ru{который} \en{can be}\ru{может быть} \en{summed}\ru{суммирован} \en{with similar objects}\ru{с~подобными объектами} \en{and}\ru{и}~\en{scaled~(multiplied)}\ru{умножен~(scaled)} \en{by scalars}\ru{на скаляры}, \en{and }\ru{а~}\en{vector space}\ru{векторное пространство}\en{ is}\ru{\:---} \en{a~synonym}\ru{синоним} \en{of~}\en{linear space}\ru{линейного пространства}.
\en{Therefore}\ru{Поэтому} \en{I clarify that}\ru{я поясняю, что} \en{in this book}\ru{в~этой книге} \emph{\en{vector}\ru{вектор}} \en{is}\ru{есть} \en{nothing else than}\ru{не что иное, как} \en{three-dimensional}\ru{трёхмерный} \en{geometric}\ru{геометрический} (\textgreek{Ευκλείδειος}, \ru{Евклидов, }Euclidean) \en{vector}\ru{вектор}.

\textcolor{magenta}{Why are vectors always straight (linear)?}

(a) \textcolor{magenta}{Vectors are linear (straight), they cannot be curved.}

(b) \textcolor{magenta}{Vectors are neither straight nor curved.}
A~vector has the~magnitude and the~direction.
A~vector is not a~line or a~curve, albeit it can be represented by a~straight line.

\textcolor{red}{Vector can’t be thought of as a~line.}

\subsection{\en{Line which figures real numbers}\ru{Изображающая реальные числа линия}}

often just \inquotes{\en{number line}\ru{числовая прямая}}

\subsection{What is a distance?}

\subsection{\en{Plane and more dimensional space}\ru{Плоскость и более многомерное пространство}}

\subsection{\en{Distance on plane or more dimensional space}\ru{Расстояние на плоскости или в более-мерном пространстве}}

\subsection{What is an angle?}

angle~$\equiv$~\textcolor{magenta}{inclination /slope, slant/ of two lines}

two lines sharing a~common point are usually called intersecting lines

angle~$\equiv$~\textcolor{magenta}{the amount of rotation of line or plane within space}

angle~$\equiv$~\textcolor{blue}{the result of the dot product of two unit vectors gives angle’s cosine}

\subsection{\en{Differentiation of continuous into small differential chunks}\ru{Дифференциация непрерывного на м\'{а}лые дифференциальные кусочки}}

\en{small differential chunks}\ru{м\'{а}лые дифференциальные кусочки}

infinitesimal (infinitely small)

\begin{changemargin}{\parindent}{\parindent}
%%\vspace{-2.5em}
{\noindent\small
\setlength{\parskip}{\spacebetweenparagraphs}

\en{A~mention}\ru{Упоминание} \en{of~tensors}\ru{тензоров} \en{may scare away the~reader}\ru{может отпугнуть читателя}, \en{commonly avoiding needless complications}\ru{обычно избегающего ненужных сложностей}.
\en{Don’t be afraid}\ru{Не бойся}: \en{tensors}\ru{тензоры} \en{are used}\ru{используются} \en{just}\ru{просто} \en{due to their wonderful property}\ru{из\hbox{-}за своего чудесного свойства} \en{of invariance}\ru{инвариантности}\:--- \en{independence from coordinate systems}\ru{независимости от систем координат}.

\par}
\vspace{-1.4em}
\end{changemargin}









\en{\section{Vector}}

\ru{\section{Вектор}}

\label{para:vector}
