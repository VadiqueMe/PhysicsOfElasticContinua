\en{\chapter{Oscillations and waves}}

\ru{\chapter{Колебания и волны}}

\thispagestyle{empty}

\label{chapter:vibrationsnwaves}

\en{\section{Vibrations of three-dimensional bodies}}

\ru{\section{Вибрации трёхмерных тел}}

\label{section:vibrations.3dbodies}

\en{\dropcap{D}{ynamic}}\ru{\dropcap{Д}{инамическая}} \en{problem}\ru{задача} \en{of the classical linear elasticity}\ru{классической линейной упругости} \en{is}\ru{есть}

\nopagebreak\vspace{-0.1em}\begin{equation}\label{classiclinearelasticity:problem}
\begin{array}{c}
\boldnabla \dotp \linearstress \hspace{.15ex} + \bm{f} = \rho \hspace{.1ex} \mathdotdotabove{\bm{u}} \hspace{.2ex} ,
\:\:
\linearstress = \stiffnesstensor \dotdotp \hspace{-0.25ex} \boldnabla {\bm{u}} \hspace{.1ex} ,
\\[.32em]
%
\bm{u} \hspace{.1ex} \bigr|_{o_1} \hspace{-0.64ex} = \hspace{.2ex} \zerovector \hspace{.1ex},
\:\:
\bm{n} \dotp \linearstress \hspace{0.25ex} \bigr|_{o_2} \hspace{-0.64ex} = \hspace{.2ex} \bm{p} \hspace{.16ex} ,
\\[.4em]
%
\bm{u} \hspace{.1ex} \bigr|_{t=0} \hspace{-0.2ex} = \bm{u}^{\hspace{-0.1ex}\circ} \hspace{-0.4ex},
\:\:
\mathdotabove{\bm{u}} \hspace{.1ex} \bigr|_{t=0} \hspace{-0.2ex} = \mathdotabove{\bm{u}}^{\circ} \hspace{-0.4ex} .
\end{array}
\end{equation}

\vspace{.2em}
\en{According to the common theory}\ru{Согласно общей теории}~(\chapterdotsectionref{chapter:classicalmechanics}{section:smalloscillations}), \en{we begin}\ru{мы начинаем} \en{with the analysis}\ru{с~анализа} \en{of~harmonics}\ru{гармоник}~(\en{ortho\-gonal oscillations}\ru{орто\-гональ\-ных колебаний}):

\nopagebreak\vspace{-0.25em}\begin{equation*}
\bm{f} = \zerovector \hspace{.1ex} ,
\:\;
\bm{p} = \zerovector \hspace{.1ex} ,
\:\;
\bm{u}(\locationvector, t) \hspace{-0.2ex} = \mathboldU(\locationvector) \sine \omega t
\hspace{.1ex} ,
\end{equation*}

\nopagebreak\vspace{-0.22em}\begin{equation}\label{oscillations:equationsforharmonics}
\boldnabla \dotp \hspace{-0.12ex} \left( \hspace{.1ex} \stiffnesstensor \dotdotp \hspace{-0.2ex} \boldnabla \hspace{.1ex} \mathboldU \hspace{.2ex} \right) \hspace{-0.16ex}
+ \rho \hspace{.3ex} \omega^2 \hspace{.25ex} \mathboldU \hspace{-0.1ex}
= \hspace{.1ex} \zerovector
\hspace{.1ex} .
\end{equation}

\vspace{-0.2em}\noindent
\en{If}\ru{Если}
\en{a~homogeneous problem}\ru{однородная задача}
\en{has}\ru{имеет}
\en{a~nontrivial solution}\ru{нетривиальное решение},
\en{then}\ru{то}
\en{the~values of}\ru{значения}~$\omega$
\en{are}\ru{это}
\en{natural}\ru{натуральные}
\en{resonant}\ru{резонансные}
\en{frequencies}\ru{частоты},
\en{and}\ru{а}~${\mathboldU(\locationvector)}$\ru{\:---}\en{ are}
\en{ortho\-gonal}\ru{орто\-гональ\-ные}~(\en{normal}\ru{нормальные})
\inquotesx{\en{modes}\ru{моды}}[.]

\en{The~time independent}\ru{Независимое от~времени}
\en{equation}\ru{уравнение}~\eqref{oscillations:equationsforharmonics}
\en{looks like}\ru{выглядит как}
\en{equation}\ru{уравнение}
\en{of~linear elasto\-statics}\ru{линейной эласто\-статики}
\eqrefwithchapterdotsection{lineartheory:equationsindisplacements}{chapter:linearclassicalelasticity}{section:equationsindisplacements.linearelasticity}\ru{,}
\en{when}\ru{когда}
\en{the~volume load}\ru{объемная нагрузка}
\en{is equal to}\ru{равна}~${\omega^2 \hspace{-0.2ex} \rho \hspace{.25ex} \mathboldU \hspace{-0.1ex}}$.
\en{The surface load}\ru{Поверхностная нагрузка}
\en{on}\ru{на}~${o_2}$
\en{is equal to zero}\ru{равна нулю}.
%
\ru{Тождество }\en{The~}Clapeyron’\en{s}\ru{а}\en{ identity}~\eqrefwithchapterdotsection{clapeyron:elasticitytheorem}{chapter:linearclassicalelasticity}{section:theoremsofstatics}
\en{gives}\ru{даёт}

\nopagebreak\vspace{-0.2em}\begin{equation}
\omega^2 \hspace{-0.2em}
\integral\displaylimits_{\mathcal{V}} \hspace{-0.5ex}
\rho \hspace{.4ex} \mathboldU \hspace{-0.2ex} \dotp \mathboldU
\hspace{.1ex} d\mathcal{V}
= \hspace{.1ex}
2 \hspace{-0.2em}
\integral\displaylimits_{\mathcal{V}} \hspace{-0.5ex}
\potentialenergydensity \bigl( \scalebox{0.9}{$\boldnabla \hspace{.1ex} \mathboldU \hspace{.2ex}^{\mathsf{S}}$} \hspace{.1ex} \bigr) \hspace{.1ex} d\mathcal{V} .
\end{equation}

\vspace{-0.2em}\noindent
\en{It also means that}\ru{Это также значит, что}
${\omega^2 \hspace{-0.2ex} \geq 0}$,
\en{and}\ru{и}~${{=}\hspace{.4ex}0}$
\en{only if}\ru{только если}
\en{a~continuum}\ru{\rucontinuum}
\en{moves}\ru{движется}
(\en{with}\ru{с}~${\mathboldU\hspace{.2ex}}$)
\en{as a~rigid whole}\ru{как жёсткое целое}.
%
\en{When}\ru{Когда}
\en{even}\ru{даже}
\en{a~small part}\ru{малая часть}
\en{of a~surface}\ru{поверхности}
\en{is fixed}\ru{закреплена},
\en{then}\ru{тогда}
\en{all}\ru{все}~${\omega_i \hspace{-0.1ex} > 0}$.

\en{And}\ru{И}~\en{here}\ru{тут}
\en{we assume that}\ru{мы предполагаем, что}
${\omega^2\hspace{-0.1ex}}$
\en{and}\ru{и}~$\mathboldU$
\en{are real numbers}\ru{это вещественные числа}.
\en{This can be proven}\ru{Это может быть доказано} \inquotesx{\en{by~contradiction}\ru{от~противного}}[.]
\en{If}\ru{Если}~${\imaginarypart{\omega^2} \hspace{-0.2ex} \neq 0}$, \en{then}\ru{то} \en{conjugate}\ru{сопряжённая} \en{frequency}\ru{часто\-та}~${\lineover{\omega}^{\hspace{.2ex}2}\hspace{-0.2ex}}$\ru{\:---}\en{ is} \ru{тоже }\en{part}\ru{часть} \en{of the~oscillation spectrum}\ru{спектра колебания}\en{ too}, \en{and}\ru{и} \inquotes{\en{mode}\ru{мода}}~${\lineoverlower{\mathboldU}\hspace{-0.1ex}}$ \en{for this frequency}\ru{для этой часто\-ты} \en{has}\ru{имеет} \en{conjugate components}\ru{сопряжённые компоненты}.
\en{Using then}\ru{Используя далее}
\en{the~reciprocal work theorem}\ru{теорему о~взаимности работ}~\eqrefwithchapterdotsection{betti:reciprocalworktheorem}{chapter:linearclassicalelasticity}{section:theoremsofstatics}
\en{for}\ru{для}
$\mathboldU$
\en{and}\ru{и}~$\lineoverlower{\mathboldU}$,
\en{we have}\ru{мы имеем}

\nopagebreak\vspace{-0.2em}\begin{equation*}
\omega^2 \hspace{-0.1em}
\scalebox{.92}{$
   \displaystyle \integral\displaylimits_{\mathcal{V}}
$}
\hspace{-0.25ex}
\rho
\hspace{.33ex}
\mathboldU
\hspace{-0.15ex} \dotp
\lineoverlower{\mathboldU}
\hspace{.1ex}
d\mathcal{V}
= \hspace{.2ex}
\lineover{\omega}^{\hspace{.2ex}2}
\hspace{-0.1em}
\scalebox{.92}{$
   \displaystyle \integral\displaylimits_{\mathcal{V}}
$}
\hspace{-0.25ex}
\rho
\hspace{.33ex}
\lineoverlower{\mathboldU}
\hspace{-0.15ex} \dotp
\mathboldU
\hspace{.1ex}
d\mathcal{V}
%
\hspace{.3em} \Rightarrow \hspace{.3em}
%
\omega^2
\hspace{-0.25ex}
= \lineover{\omega}^{\hspace{.2ex}2}
\hspace{-0.15ex}
%
\hspace{.3em} \Rightarrow \hspace{.3em}
%
\imaginarypart{\omega^2}
\hspace{-0.2ex}
= 0
\hspace{.15ex} .
\end{equation*}

...

\en{However}\ru{Однако}, \en{a~bright picture}\ru{яркая картина} \en{with decomposition}\ru{с~разложением} \en{by modes}\ru{по модам} \en{is of little use}\ru{малопригодна} \en{for}\ru{для} \en{practical calculations}\ru{практических расчётов} \en{of~}\en{oscillations}\ru{колебаний}~(\en{vibrations}\ru{вибраций}) \en{of a~three-dimensional elastic body}\ru{трёхмерного упругого тела}.
\en{The reason}\ru{Причина}\en{ is}\ru{\:---} \en{density}\ru{густота} \en{of the spectrum}\ru{спектра}, \en{driven oscillations}\ru{вынужденные колебания} \en{excite}\ru{возбуждают} \en{many modes}\ru{много мод}.
\en{When}\ru{Когда} \en{the natural frequency density}\ru{плотность собственных частот} \en{is high}\ru{высокая}, \en{even}\ru{даже} \en{a~small friction}\ru{малое трение} \en{qualitatively}\ru{качественно} \en{changes}\ru{меняет} \en{the resonance curve}\ru{резонансную кривую}.
\en{Damping}\ru{Демпфирование} (\en{decrease in~amplitudes}\ru{уменьшение амплитуд}) \en{in real bodies}\ru{в~реальных телах} \en{is also important}\ru{тоже важно}.
\en{In addition}\ru{Вдобавок}, \en{wave nature}\ru{волновая природа} \en{of non-stationary processes}\ru{нестационарных процессов} \en{hinders}\ru{мешает} \en{to just transfer}\ru{просто перенести} \en{the theory of oscillations of discrete systems}\ru{теорию колебаний дискретных систем} \en{to the continuum}\ru{на~\rucontinuum}: \en{in case of}\ru{в~случае} \en{sudden}\ru{внезапного} \en{local}\ru{локального} \en{excitation}\ru{возбуждения} \en{it’s more correct}\ru{корректнее} \en{to consider waves}\ru{рассматривать волны} \en{instead of superposing modes}\ru{вместо наложения мод}.

\en{The way}\ru{Путь} \en{from}\ru{от} \en{a~continuous dynamic model}\ru{непрерывной динамической модели} \en{to a~discrete one}\ru{к~дискретной} \en{goes}\ru{проходит} \en{through}\ru{через} \en{the variational approach}\ru{вариационный подход}.

\nopagebreak\vspace{-0.2em}\begin{equation}\label{oscillations.variationalapproach}
\displaystyle\integral\displaylimits_{\mathcal{V}} \hspace{-0.4ex}
\Bigl( \bigl(
\bm{f} - \rho \mathdotdotabove{\bm{u}}
\bigr) \hspace{-0.1ex} \dotp \variation{\bm{u}}
- \linearstress \dotdotp \variation{\infinitesimaldeformation} \Bigr) d\mathcal{V} \hspace{.1ex}
+ \hspace{-0.3ex}
\displaystyle\integral\displaylimits_{o_2} \hspace{-0.4ex}
\bm{p} \dotp \variation{\bm{u}} \hspace{.3ex} do \hspace{.2ex}
= 0
\hspace{.1ex} .
\end{equation}

\vspace{-0.2em}\noindent
\en{This is}\ru{Это} \en{the principle of virtual work}\ru{принцип виртуальной работы} \en{with forces of inertia}\ru{с~силами инерции}.
\en{Looking for}\ru{Разыскивая} \en{an approximate solution}\ru{приближённое решение} \en{in series}\ru{в~рядах}

\nopagebreak\vspace{-0.2em}\begin{equation*}
\bm{u}(\bm{r}, t) = \scalebox{0.8}{$ \displaystyle\sum_{\scriptstyle k=1}^{\scriptstyle N} $}
\alpha_{k} ( t ) \varphi_{k} ( \bm{r} )
\hspace{.1ex} ,
\end{equation*}

\vspace{-0.2em}\noindent
\en{where}\ru{где}~$\varphi_{k}$ \en{are given}\ru{даются} ${ \Bigl( \varphi_{k} \hspace{.1ex} \bigr|_{o_1} \hspace{-0.64ex} = \hspace{.2ex} \zerovector \Bigr) }$,
\en{and}\ru{а}~${\alpha_{k} ( t )}$ \en{are varying}\ru{варьируются}.
\en{The solution}\ru{Решение}????\en{ is}\ru{\:---} \en{\textcolor{red}{a/?????the system}}\ru{система} \en{of ordinary equations}\ru{обыкновенных уравнений}

....

\en{In place}\ru{Вместо}
\en{of the~principle}\ru{принципа}
\en{of virtual work}\ru{виртуальной работы}~\eqref{oscillations.variationalapproach},
\ru{смешанная формулировка }\en{the~}Hellinger\ru{’а}--Reissner’\en{s}\ru{а}\en{ mixed formulation}
\en{with }\ru{с~}%
\en{an independent approximation}\ru{независимой аппроксимацией}
\en{of~stresses}\ru{напряжений}
\en{can be}\ru{может быть}
\en{applied}\ru{применена}.

\en{In the dynamic theory of elasticity}\ru{В~динамической теории упругости}\en{,}
\ru{часто применяется }\href{https://en.wikipedia.org/wiki/Laplace_transform}{\ru{интегральное преобразование }\en{the }Laplace\ru{’а}\en{ integral transform}}\en{ is often applied}.
\en{For simple shaped bodies}\ru{Для тел простой формы} \en{it is }\en{sometimes}\ru{иногда} \en{possible}\ru{возможно} \en{to find}\ru{найти} \en{an analytical solution}\ru{аналитическое решение} \en{in transforms}\ru{в~трансформах}.
\en{The original}\ru{Оригинал} \en{can be found}\ru{может быть найден} \en{by the numerical inversion}\ru{численным обращением}, \en{but sometimes}\ru{но иногда} \en{it’s possible}\ru{возможно} \en{to take}\ru{взять} \en{the }\ru{интеграл }Riemann\ru{’а}\hbox{--}Mellin\ru{’а}\en{ integral}\footnote{%
\ru{Интеграл }\en{The }Riemann\ru{’а}\hbox{--}Mellin\ru{’а}\en{ integral}
\en{does}\ru{делает}
\en{the inverse}\ru{обратное}
\ru{преобразование }Laplace\ru{’а}\en{ transform}
${F(s) \mapsto f(t)}$.
}\ru{\hbox{\hspace{-0.5ex},}}
\en{using}\ru{используя}
\en{the }\inquotes{\en{saddle-point method}\ru{метод перевала}}
(\en{or}\ru{или}
\en{the }\inquotes{\en{method of steepest descents}\ru{метод крутых спусков}})
\en{with a~deformation}\ru{с~деформацией}
\en{of~contour}\ru{контура}
\en{in the~complex plane}\ru{в~комплексной плоскости}~\cite{poruchikov-dynamicelasticity, slepyan-nonstationeryelasticwaves}.

\section{%
\en{Vibrations}\ru{Вибрации}
\en{of rods}\ru{стержней}%
}

\label{section:vibrations.rods}

\en{In the linear dynamics}\ru{В~линейной динамике} \en{of~rods}\ru{стержней} \en{we have}\ru{мы имеем} \en{the following system}\ru{следующую систему} \en{for}\ru{для} \en{forces}\ru{сил}~$\mathboldQ$, \en{force couples}\ru{пар сил}~(\en{moments}\ru{моментов})~$\mathboldM$, \en{displacements}\ru{смещений}~$\bm{u}$ \en{and}\ru{и}~\en{rotations}\ru{поворотов}~$\bm{\theta}~(\chapterdotsectionref{chapter:rods}{section:rods-lineartheory})$:

\nopagebreak\vspace{-0.2em}\begin{gather}
\bm{Q}' + \bm{q}
= \rho \bigl( \mathdotdotabove{\bm{u}} + \mathdotdotabove{\bm{\theta}} \hspace{-0.2ex} \times \hspace{-0.2ex} \infinitesimaldeformation \bigr)
\hspace{.1ex} ,
\hspace{.3em}
\bm{M}' + \bm{r}' \hspace{-0.2ex} \times \hspace{-0.2ex} \bm{Q} + \bm{m} =
\bm{J} \dotp \mathdotdotabove{\bm{\theta}} + \rho \infinitesimaldeformation \hspace{-0.2ex} \times \hspace{-0.2ex} \mathdotdotabove{\bm{u}}
\hspace{.1ex} ,
\\[-0.2em]
%
...
\end{gather}

...

%%\begin{otherlanguage}{russian}
%%\end{otherlanguage}

\en{\section{Small perturbations of parameters}}

\ru{\section{Малые возмущения параметров}}

\label{section:vibrations.smallperturbations}

\begin{otherlanguage}{russian}

\en{With }\ru{С~}\en{small perturbations}\ru{малыми возмущениями}
\en{of mass}\ru{массы}
\en{and }\ru{и~}\en{rigidity}\ru{жёсткости},
рассмотрим
задачу
об~определении
собственных частот и~форм\::

\nopagebreak\vspace{-0.5em}\begin{equation}\begin{array}{c}
\bigl( C_{i\hspace{-0.1ex}j} \hspace{-0.1ex} - \omega^2 \hspace{-0.2ex} A_{i\hspace{-0.1ex}j} \bigr) \hspace{.1ex} U_{\hspace{-0.2ex}j} \hspace{-0.15ex} = 0 \hspace{.1ex} ,
\\[.4em]
%
C_{i\hspace{-0.1ex}j} \hspace{-0.2ex} = C_{i\hspace{-0.1ex}j}^{\hspace{.2ex}\scalebox{.66}{(0)}} \hspace{-0.3ex} + \smallparameter \hspace{.2ex} C_{i\hspace{-0.1ex}j}^{\hspace{.2ex}\scalebox{.66}{(1)}}
\hspace{-0.4ex} ,
\hspace{1ex}
A_{i\hspace{-0.1ex}j} \hspace{-0.2ex} = \hspace{-0.2ex} A_{i\hspace{-0.1ex}j}^{\hspace{.2ex}\scalebox{.66}{(0)}} \hspace{-0.3ex} + \smallparameter \hspace{-0.1ex} A_{i\hspace{-0.1ex}j}^{\hspace{.2ex}\scalebox{.66}{(1)}}
\hspace{-0.4ex} ,
\hspace{1.5ex}
\smallparameter \hspace{-0.12ex} \to 0
\hspace{.1ex} .
\end{array}\end{equation}

\vspace{-0.5em}\noindent
Наход\'{я} решение в~виде

\nopagebreak\vspace{-0.2em}\begin{equation*}
\omega = \omega^{\hspace{.1ex}\scalebox{.66}{(0)}} \hspace{-0.25ex} + \smallparameter \hspace{.2ex} \omega^{\hspace{.1ex}\scalebox{.66}{(1)}} \hspace{-0.25ex} + \ldots \hspace{.1ex} ,
\;\;
U_{\hspace{-0.2ex}j} \hspace{-0.15ex} = U_{\hspace{-0.2ex}j}^{\hspace{.1ex}\scalebox{.66}{(0)}} \hspace{-0.25ex} + \smallparameter \hspace{.2ex} U_{\hspace{-0.2ex}j}^{\hspace{.1ex}\scalebox{.66}{(1)}} \hspace{-0.25ex} + \ldots \hspace{.1ex} ,
\end{equation*}

\vspace{-0.25em}\noindent
получаем последовательность задач

...




\end{otherlanguage}

\en{\section{Vibrations of shells}}

\ru{\section{Вибрации оболочек}}

\label{section:vibrations.shells}

\begin{otherlanguage}{russian}

Динамика оболочек рассматривалась многими

...




\end{otherlanguage}

\en{\section{Waves in an elastic continuum}}

\ru{\section{Волны в упругом \rucontinuum{}е}}

\label{section:wavesinelastic}

\begin{otherlanguage}{russian}

Рассмотрим линейные уравнения динамики однородной изотропной среды без объёмных сил

...




\end{otherlanguage}

\en{\section{Waves in a rod}}

\ru{\section{Волны в стержне}}

\label{section:wavesinrod}

\begin{otherlanguage}{russian}

Рассмотрим прямой стержень.
Продольная деформация описывается уравнениями

...




\end{otherlanguage}

\en{\section{Nonlinear oscillations}}

\ru{\section{Нелинейные колебания}}

\label{section:oscillations.nonlinear}

\begin{otherlanguage}{russian}

Рассмотрим простой пример: продольные колебания прямого стержня с~м\'{а}лой нелинейной добавкой в~соотношениях упругости

...

\end{otherlanguage}

\section*{\small \wordforbibliography}

\begin{changemargin}{\parindent}{0pt}
\fontsize{10}{12}\selectfont

\begin{otherlanguage}{russian}

Методы решения динамических задач упругости представлены в~книгах
Л.\,И.\;Слепяна~\cite{slepyan-nonstationeryelasticwaves}
и~В.\,Б.\;Поручикова~\cite{poruchikov-dynamicelasticity}.
О~м\'{а}лых линейных колебаниях~(вибрациях) написано
у~С.\,П.\;Тимошенко, D.\,H.\:Young’а и~W.\,Weaver’а~\cite{timoshenko.young.weaver},
И.\,М.\;Бабакова~\cite{babakov-theoryofoscillations},
В.\,Л.\;Бидермана~\cite{biderman-oscillations},
В.\,Т.\;Гринченко и~В.\,В.\;Мелешко~\cite{grinchenko.meleshko}.
Асимптотические проблемы колебаний оболочек освещены
у~...

\end{otherlanguage}

\end{changemargin}
