\en{\chapter{Oscillations and waves}}

\ru{\chapter{Колебания и волны}}

\thispagestyle{empty}

\label{chapter:vibrationsnwaves}

\en{\section{Vibrations of a three-dimensional body}}

\ru{\section{Вибрации трёхмерного тела}}

\begin{otherlanguage}{russian}

\lettrine[lines=2, findent=2pt, nindent=0pt]{Д}{инамическая} задача классической линейной упругости

\nopagebreak\vspace{-0.1em}\begin{equation}\label{classiclinearelasticity:problem}
\begin{array}{c}
\boldnabla \dotp \linearstress \hspace{.15ex} + \bm{f} = \rho \hspace{.1ex} \mathdotdotabove{\bm{u}} \hspace{.2ex} ,
\:\:
\linearstress = \stiffnesstensor \dotdotp \hspace{-0.25ex} \boldnabla {\bm{u}} \hspace{.1ex} ,
\\[.32em]
%
\bm{u} \hspace{.1ex} \bigr|_{o_1} \hspace{-0.64ex} = \hspace{.2ex} \bm{0} \hspace{.1ex},
\:\:
\bm{n} \dotp \linearstress \hspace{0.25ex} \bigr|_{o_2} \hspace{-0.64ex} = \hspace{.2ex} \bm{p} \hspace{.16ex} ,
\\[.4em]
%
\bm{u} \hspace{.1ex} \bigr|_{t=0} \hspace{-0.2ex} = \bm{u}^{\hspace{-0.1ex}\circ} \hspace{-0.4ex},
\:\:
\mathdotabove{\bm{u}} \hspace{.1ex} \bigr|_{t=0} \hspace{-0.2ex} = \mathdotabove{\bm{u}}^{\circ} \hspace{-0.4ex} .
\end{array}
\end{equation}

\vspace{.2em} Согласно общей теории~(\chapdotpararef{chapter:genericmechanics}{para:smalloscillations}), начинаем с~анализа \en{of~harmonics}\ru{гармоник}~(\en{ortho\-gonal oscillations}\ru{орто\-гональ\-ных колебаний}):

\nopagebreak\vspace{-0.25em}\begin{equation*}
\bm{f} = \bm{0} \hspace{.1ex} ,
\:\;
\bm{p} = \bm{0} \hspace{.1ex} ,
\:\;
\bm{u}(\bm{r}, t) \hspace{-0.2ex} = \mathboldU(\bm{r}) \operatorname{sin} \omega t
\hspace{.1ex} ,
\end{equation*}

\nopagebreak\vspace{-0.22em}\begin{equation}\label{oscillations:equationsforharmonics}
\boldnabla \dotp \hspace{-0.12ex} \left( \hspace{.1ex} \stiffnesstensor \dotdotp \hspace{-0.2ex} \boldnabla \hspace{.1ex} \mathboldU \hspace{.2ex} \right) \hspace{-0.16ex}
+ \rho \hspace{.3ex} \omega^2 \hspace{.25ex} \mathboldU \hspace{-0.1ex}
= \hspace{.1ex} \bm{0}
\hspace{.1ex} .
\end{equation}

\vspace{-0.2em} \noindent Значения~$\omega$, при которых однородная задача имеет нетривиальное решение\:--- это \en{natural}\ru{натуральные} \en{resonant}\ru{резонансные} \en{frequencies}\ru{частоты}, \en{and}\ru{а}~${\mathboldU(\bm{r})}$\ru{\:---}\en{ are} \en{ortho\-gonal}\ru{орто\-гональ\-ные}~(\en{normal}\ru{нормальные}) \inquotesx{\en{modes}\ru{моды}}[.]

\en{Time independent}\ru{Независимое от~времени}~\eqref{oscillations:equationsforharmonics} \en{looks like}\ru{выглядит как} \en{equation}\ru{уравнение} \en{of~linear elasto\-statics}\ru{линейной эласто\-статики} \eqrefwithchapdotpara{lineartheory:equationsfordisplacement}{chapter:linearclassicalelasticity}{para:equationsfordisplacement} \en{with the~volume load equal to}\ru{с~объёмной нагрузкой, равной}~${\omega^2 \hspace{-0.2ex} \rho \hspace{.25ex} \mathboldU \hspace{-0.1ex}}$.
Поверхностная нагрузка на~${o_2}$\:--- нуль.
Тождество Clapeyron’а~\eqrefwithchapdotpara{clapeyron:elasticitytheorem}{chapter:linearclassicalelasticity}{para:theoremsofstatics} даёт

\nopagebreak\vspace{-0.2em}\begin{equation}
\omega^2 \hspace{-0.2em}
\integral\displaylimits_{\mathcal{V}} \hspace{-0.5ex}
\rho \hspace{.33ex} \mathboldU \hspace{-0.15ex} \dotp \mathboldU
\hspace{.1ex} d\mathcal{V}
= \hspace{.1ex}
2 \hspace{-0.2em}
\integral\displaylimits_{\mathcal{V}} \hspace{-0.5ex}
\Pi \bigl( \scalebox{0.9}{$\boldnabla \hspace{.1ex} \mathboldU \hspace{.2ex}^{\mathsf{S}}$} \hspace{.1ex} \bigr) \hspace{.1ex} d\mathcal{V} .
\end{equation}

\vspace{-0.2em} \noindent \en{It also means}\ru{Это также значит} ${\omega^2 \hspace{-0.2ex} \geq 0}$, \en{with zero}\ru{с~нулём} \en{for}\ru{для} \en{motion}\ru{движения}~$\mathboldU$ \en{of~a~continuum}\ru{контину\kern-0.11exума} \en{as a~rigid whole}\ru{как жёсткого целого}.
С~закреплением хотя~бы малой части поверхности \en{all}\ru{все}~${\omega_i \hspace{-0.1ex} > 0}$.

\en{But}\ru{Но} \en{here was assumed that}\ru{тут предполагалось, что} ${\omega^2\hspace{-0.1ex}}$ \en{and}\ru{и}~$\mathboldU$ \en{are real}\ru{вещественны}.
\en{This can be proven}\ru{Это может быть доказано} \inquotesx{\en{by~contradiction}\ru{от~противного}}[.] \en{If}\ru{Если}~${\imaginarypart{\omega^2} \hspace{-0.2ex} \neq 0}$, \en{then}\ru{то} \en{conjugate}\ru{сопряжённая} \en{frequency}\ru{часто\-та}~${\lineover{\omega}^{\hspace{.2ex}2}\hspace{-0.2ex}}$\ru{\:---}\en{ is} \ru{тоже }\en{part}\ru{часть} \en{of the~oscillation spectrum}\ru{спектра колебания}\en{ too}, \en{and}\ru{и} \inquotes{\en{mode}\ru{мода}}~${\lineoverlower{\mathboldU}\hspace{-0.1ex}}$ \en{for this frequency}\ru{для этой часто\-ты} \en{has}\ru{имеет} \en{conjugate components}\ru{сопряжённые компоненты}.
\en{Using then}\ru{Используя далее} \en{the~reciprocal work theorem}\ru{теорему о~взаимности работ}~\eqrefwithchapdotpara{betti:reciprocalworktheorem}{chapter:linearclassicalelasticity}{para:theoremsofstatics} \en{for}\ru{для} $\mathboldU$ \en{and}\ru{и}~$\lineoverlower{\mathboldU}$ \en{as two static variants}\ru{как двух статических вариантов},

\nopagebreak\vspace{-0.2em}\begin{equation*}
\omega^2 \hspace{-0.1em}
\scalebox{0.92}{$\displaystyle \integral\displaylimits_{\mathcal{V}}$} \hspace{-0.25ex}
\rho \hspace{.33ex} \mathboldU \hspace{-0.15ex} \dotp \lineoverlower{\mathboldU}
\hspace{.1ex} d\mathcal{V} \hspace{.05ex}
= \hspace{.2ex}
\lineover{\omega}^{\hspace{.2ex}2} \hspace{-0.1em}
\scalebox{0.92}{$\displaystyle \integral\displaylimits_{\mathcal{V}}$} \hspace{-0.25ex}
\rho \hspace{.33ex} \lineoverlower{\mathboldU} \hspace{-0.15ex} \dotp \mathboldU
\hspace{.1ex} d\mathcal{V}
%
\hspace{.22em} \Rightarrow \hspace{.22em}
%
\omega^2 \hspace{-0.25ex} = \lineover{\omega}^{\hspace{.2ex}2} \hspace{-0.15ex}
%
\hspace{.22em} \Rightarrow \hspace{.22em}
%
\imaginarypart{\omega^2} \hspace{-0.2ex} = 0
\hspace{.15ex} .
\end{equation*}

...




\end{otherlanguage}

\en{\section{Vibrations of a rod}}

\ru{\section{Вибрации стержня}}

\begin{otherlanguage}{russian}

В~линейной динамике стержней имеем следующую систему для~сил~$\mathboldQ$, моментов~$\mathboldM$, перемещений~$\bm{u}$ и~поворотов~$\bm{\theta}~(\chapdotpararef{chapter:rods}{para:ooooooooo})$:

...



\end{otherlanguage}

\en{\section{Small perturbations of parameters}}

\ru{\section{Малые возмущения параметров}}

\begin{otherlanguage}{russian}

Рассмотрим задачу об~определении собственных частот и~форм с~малыми возмущениями масс и~жёсткостей:

\nopagebreak\vspace{-0.1em}\begin{equation}\begin{array}{c}
\bigl( C_{i\hspace{-0.1ex}j} \hspace{-0.15ex} - \omega^2 \hspace{-0.2ex} A_{i\hspace{-0.1ex}j} \bigr) \hspace{.1ex} U_{\hspace{-0.2ex}j} \hspace{-0.15ex} = 0 \hspace{.1ex} ,
\\[.2em]
%
C_{i\hspace{-0.1ex}j} \hspace{-0.2ex} = C_{i\hspace{-0.1ex}j}^{\hspace{.2ex}\scalebox{0.66}[0.66]{(0)}} \hspace{-0.2ex} + \smallparameter \hspace{.2ex} C_{i\hspace{-0.1ex}j}^{\hspace{.2ex}\scalebox{0.66}[0.66]{(1)}} \hspace{-0.4ex} ,
\;\:
A_{i\hspace{-0.1ex}j} \hspace{-0.2ex} = \hspace{-0.2ex} A_{i\hspace{-0.1ex}j}^{\hspace{.2ex}\scalebox{0.66}[0.66]{(0)}} \hspace{-0.2ex} + \smallparameter \hspace{-0.1ex} A_{i\hspace{-0.1ex}j}^{\hspace{.2ex}\scalebox{0.66}[0.66]{(1)}} \hspace{-0.4ex} ,
\;\:
\smallparameter \hspace{-0.12ex} \to 0 \hspace{.1ex} .
\end{array}\end{equation}

\vspace{-0.2em} \noindent Наход\'{я} решение в~виде

\nopagebreak\vspace{-0.2em}\begin{equation*}
\omega = \omega^{\hspace{.1ex}\scalebox{0.66}[0.66]{(0)}} \hspace{-0.25ex} + \smallparameter \hspace{.2ex} \omega^{\hspace{.1ex}\scalebox{0.66}[0.66]{(1)}} \hspace{-0.25ex} + \ldots \hspace{.1ex} ,
\;\;
U_{\hspace{-0.2ex}j} \hspace{-0.15ex} = U_{\hspace{-0.2ex}j}^{\hspace{.1ex}\scalebox{0.66}[0.66]{(0)}} \hspace{-0.25ex} + \smallparameter \hspace{.2ex} U_{\hspace{-0.2ex}j}^{\hspace{.1ex}\scalebox{0.66}[0.66]{(1)}} \hspace{-0.25ex} + \ldots \hspace{.1ex} ,
\end{equation*}

\vspace{-0.25em} \noindent получаем последовательность задач

...




\end{otherlanguage}

\en{\section{Vibrations of a shell}}

\ru{\section{Вибрации оболочки}}

\begin{otherlanguage}{russian}

Динамика оболочек рассматривалась многими

...




\end{otherlanguage}

\en{\section{Waves in an elastic continuum}}

\ru{\section{Волны в упругой среде}}

\begin{otherlanguage}{russian}

Рассмотрим линейные уравнения динамики однородной изотропной среды без объёмных сил

...




\end{otherlanguage}

\en{\section{Waves in a rod}}

\ru{\section{Волны в стержне}}

\begin{otherlanguage}{russian}

Рассмотрим прямой стержень. Продольная деформация описывается уравнениями

...




\end{otherlanguage}

\en{\section{Nonlinear oscillations}}

\ru{\section{Нелинейные колебания}}

\begin{otherlanguage}{russian}

Рассмотрим простой пример: продольные колебания прямого стержня с~м\'{а}лой нелинейной добавкой в~соотношениях упругости

...




\end{otherlanguage}

\section*{\small \wordforbibliography}

\begin{changemargin}{\parindent}{0pt}
\fontsize{10}{12}\selectfont

\begin{otherlanguage}{russian}

Методы решения динамических задач упругости представлены в~книгах
Л.\,И.\;Слепяна~\cite{slepyan-nonstationeryelasticwaves}
и~В.\,Б.\;Поручикова~\cite{poruchikov-dynamicelasticity}.
О~м\'{а}лых линейных колебаниях~(вибрациях) написано
у~С.\,П.\;Тимошенко, D.\,H.\:Young’а и~W.\,Weaver’а~\cite{timoshenko.young.weaver},
И.\,М.\;Бабакова~\cite{babakov-theoryofoscillations},
В.\,Л.\;Бидермана~\cite{biderman-oscillations},
В.\,Т.\;Гринченко и~В.\,В.\;Мелешко~\cite{grinchenko.meleshko}.
Асимптотические проблемы колебаний оболочек освещены
у~...

\end{otherlanguage}

\end{changemargin}
