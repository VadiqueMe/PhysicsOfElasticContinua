<<<<<<< HEAD
\en{\section{Vector}}

\ru{\section{Вектор}}

\label{para:vector}

\en{\dropcap{I}{}}\ru{\dropcap{Я}{}}
\hspace{-0.1ex}\en{propose}\ru{предлагаю} \en{to~begin familiarizing with tensors}\ru{начать знакомство с~тензорами} \en{via memoirs}\ru{через мемуары} \en{about such a~phenomenon as a~vector}\ru{о~таком феномене как вектор}%,......................
\hbox{\hspace{-0.5ex}.}

\begin{itemize}
\item A \emph{point} has position in space. The only characteristic that distinguishes one point from another is its position.
\item A \emph{vector} has both magnitude and direction, but no specific position in space.
\end{itemize}

\subsection{\en{What is a~vector?}\ru{Что такое вектор?}}

What is \inquotes{linear}?
\\
(1) straight
\\
(2) relating to, resembling, or having \textcolor{teal}{a~graph} that is a~straight line

All vectors are linear objects.

Examples of vectors:

\begin{itemize}
\item \en{A~force acts on an~object}\ru{Сила действует на объект}.
%
\item \en{The~velocity of an~object}\ru{Скорость объекта} \en{describes}\ru{описывает} \en{what’s happening}\ru{происходящее} \en{with this object}\ru{с~этим объектом} \en{at an~instant}\ru{за мгновение}.
\end{itemize}

\subsection{\en{What is a~vector?}\ru{Что такое вектор?}}

What is \inquotes{linear}?
\\
(1) straight
\\
(2) relating to, resembling, or having \textcolor{teal}{a~graph} that is a~straight line

All vectors are linear objects.

Examples of vectors:

\begin{itemize}
\item \en{A~force acts on an~object}\ru{Сила действует на объект}.
%
\item \en{The~velocity of an~object}\ru{Скорость объекта} \en{describes}\ru{описывает} \en{what’s happening}\ru{происходящее} \en{with this object}\ru{с~этим объектом} \en{at an~instant}\ru{за мгновение}.
\end{itemize}

The sum (combination) of two or more vectors is the new (\inquotes{resultant}) vector.
There are two similar methods to calculate the resultant vector geometrically.

The \emph{\inquotes{head to tail} method} involves lining up the head of the one vector with the tail of the other.
Here the resultant goes from the initial point (\inquotes{tail}) of the first addend to the end point (\inquotes{head}) of the second addend when the tail (initial point) of the second one coincides with the head (end point) of the first one.

[ .... figure here .... ]

The \emph{\inquotes{parallelogram} method}.

Vector addition is commutative
\begin{equation*}
\bm{v} + \bm{w} = \bm{w} + \bm{v}
\hspace{.1ex} .
\end{equation*}

\en{Multiplication of a~vector by a~scalar.}\ru{Умножение вектора на скаляр.}

\en{Multiplication by minus one.}\ru{Умножение на минус единицу.}

\ru{Принцип }\en{The }Newton’\en{s}\ru{а} \en{action}\ru{действия}--\en{reaction}\ru{противодействия}\en{ principle} \inquotes{\foreignlanguage{russian}{действие равно противодействию по магнитуде и обратно ему по направлению}}.

In every mechanical interaction, there’s a~pair of forces acting on the two objects that interact. These forces can be represented as vectors, they are equal in magnitude and reverse in direction.

Multiplying a~vector by the negative one~${ -1}$ reverses the direction but doesn’t change the magnitude.

\subsection{The addition and subtraction}

\nopagebreak\vspace{-0.1em}\begin{gather*}
\bm{p} + \bm{q}
\hspace{.1ex} ,
\\
\bm{p} - \bm{q} = \bm{p} + \bigl( - \bm{q} \bigr)
= \bm{p} + \bigl( - 1 \bigr) \bm{q}
\hspace{.1ex} .
\end{gather*}

For every action, there’s an equal (in magnitude) and opposite (in direction) reaction force.

A~vector may be also represented as the sum (combination) of some trio of~other vectors, called \inquotes{basis}, when the each of the three is scaled by a~number (coefficient).
Such a~representation is called a~\inquotes{linear combination} of basis vectors.
A~list (array, tuple) of coefficients alone, without basis vectors, is not enough and can’t represent a~vector.

{\small
\foreignlanguage{russian}{У элементов векторного пространства\:--- векторов\:--- компонент нет.
Компоненты появляются тогда, когда выбран базис.
И~в~разных базисах компоненты одного и~того~же вектора будут разные.}
\par}

\en{Here it is}\ru{Вот он}\:--- \en{the }\en{vector}\ru{вектор}, $\bm{v}$ \en{seems}\ru{кажется} \en{like a~suitable name}\ru{подходящим именем} \en{for it}\ru{для него}.

\en{Like all geometric vectors}\ru{Как и~все геометрические векторы}, $\bm{v}$ \en{is pretty well characterized}\ru{вполне характеризуется} \en{by the }\en{two mutually independent properties}\ru{двумя взаимно независимыми свойствами}: \en{its length}\ru{своей длиной}~(\en{magnitude}\ru{магнитудой}, \en{norm}\ru{нормой}, \en{modulus}\ru{модулем}) \en{and }\ru{и~}\en{its direction}\ru{своим направлением} \en{in space}\ru{в~пространстве}.
\en{This characterization is complete}\ru{Эта характеристика полная}, \en{so}\ru{так что} \en{two vectors}\ru{два вектора} \en{with the same magnitude and the same direction}\ru{с~одинаковой магнитудой и~одинаковым направлением} \en{are considered equal}\ru{считаются равными}.

\en{Every vector}\ru{Каждый вектор} \en{exists}\ru{существует} \en{objectively by itself}\ru{объективно сам по~себе}, \en{independently}\ru{независимо} \en{of}\ru{от}~\en{methods}\ru{методов} \en{and}\ru{и}~\en{units}\ru{единиц} \en{of~measurement}\ru{измерения} \en{of~both lengths and directions}\ru{и~длин, и~направлений} (\en{including}\ru{включая} \en{any abstractions}\ru{любые абстракции} \en{of such units and methods}\ru{таких единиц и~методов}).

\textgreek{Εὐκλείδης}
Eὐkleídēs
\ru{Евклид}

\textgreek{ευκλείδειος}
euclidean
\ru{евклидово}

plane geometry is the two-dimensional Euclidean geometry

The Elements (\textgreek{Στοιχεῖα}) (Stoikheîa)

\subsection{\en{The method of coordinates}\ru{Метод координат}}

......

\vspace{.2em}
\hspace*{-\parindent}\begin{minipage}{\linewidth}
\setlength{\parindent}{\horizontalindent}
\setlength{\parskip}{\spacebetweenparagraphs}

\begin{wrapfigure}[12]{o}{.4\textwidth}
\makebox[.32\textwidth][c]{\begin{minipage}[t]{.5\textwidth}
\vspace{-1.3em}
\scalebox{.93}{
\tdplotsetmaincoords{45}{125}


% three parameters for the vector
\pgfmathsetmacro{\lengthofvector}{2.5}
\pgfmathsetmacro{\anglefromz}{44}
\pgfmathsetmacro{\anglefromx}{70}

\begin{tikzpicture}[scale=2.5, tdplot_main_coords]

   \coordinate (O) at (0, 0, 0);

   % draw axes
   \draw [line width=.4pt, blue] (O) -- (1.22, 0, 0);
   \draw [line width=1.25pt, blue, -{Latex[round, length=3.6mm, width=2.4mm]}]
      (O) -- (1, 0, 0)
      node[pos=.9, above, xshift=-0.6em, yshift=.1em]
      {$ \bm{e}_1 $};

      \draw [line width=.4pt, blue] (O) -- (0, 1.88, 0);
      \draw [line width=1.25pt, blue, -{Latex[round, length=3.6mm, width=2.4mm]}]
         (O) -- (0, 1, 0)
         node[pos=.9, above, xshift=.2em, yshift=.1em]
         {$ \bm{e}_2 $};

      \draw [line width=.4pt, blue] (O) -- (0, 0, 1.22);
      \draw [line width=1.25pt, blue, -{Latex[round, length=3.6mm, width=2.4mm]}]
         (O) -- (0, 0, 1)
         node[pos=.9, below left, xshift=-0.1em, yshift=.5em]
         {$ \bm{e}_3 $};

      % draw vector
      \tdplotsetcoord{V}{\lengthofvector}{\anglefromz}{\anglefromx} % {length}{angle from z}{angle from x}
      % it also defines projections of point
      \draw [line width=1.6pt, black, -{Stealth[round, length=5mm, width=2.8mm]}]
      (O) -- (V)
      node[pos=.8, above, yshift=.2em] { \scalebox{1.2}{$\bm{v}$} };

      % draw components of vector
      %%\draw [line width=.4pt, dotted, color=black] (O) -- (Vxy);
      \draw [line width=.4pt, dotted, color=black] (Vxy) -- (Vy);

      \draw [color=black, line width=1.6pt, line cap=round, dash pattern=on 0pt off 1.6\pgflinewidth,
         -{Stealth[round, length=4mm, width=2.4mm]}]
         (O) -- (Vx)
         node[pos=.5, above, xshift=-0.9em]
         {${ v_1 \hspace{-0.1ex} \bm{e}_1 }$};

      \draw [color=black, line width=1.6pt, line cap=round, dash pattern=on 0pt off 1.6\pgflinewidth,
         -{Stealth[round, length=4mm, width=2.4mm]}]
         (Vx) -- (Vxy)
         node[pos=.53, below left, xshift=.3em] {${ v_2 \bm{e}_2 }$};


      \draw [color=black, line width=1.6pt, line cap=round, dash pattern=on 0pt off 1.6\pgflinewidth,
            -{Stealth[round, length=4mm, width=2.4mm]}]
            (Vxy) -- (V)
            node[pos=.5, above right, yshift=-0.2em] {${ v_3 \bm{e}_3 }$};

\end{tikzpicture}


\vspace{-2em}\caption{}\label{fig:componentsofvector}
\end{minipage}}
\end{wrapfigure}

\en{Introduce rectangular}\ru{Введём прямоугольные}~(\inquotes{\en{cartesian}\ru{декартовы}}) \en{coordinates}\ru{координаты} \en{by picking}\ru{выбором} \en{some}\ru{каких-либо} \en{three}\ru{трёх} \en{mutually perpendicular unit vectors}\ru{взаимно перпендикулярных единичных векторов} ${\bm{e}_1}$,\;${\bm{e}_2}$,\;${\bm{e}_3}$ \en{as}\ru{как} \en{a~\hbox{basis}}\ru{основы~(\hbox{базиса})} \en{for measurements}\ru{для измерений}.
\en{Within such}\ru{В~так\'{о}й}
\en{a~}\hbox{\en{system}\ru{системе}}\en{,}
\hbox{\hspace{-0.2ex}\inquotes{${\dotp\hspace{.22ex}}$}\hspace{-0.2ex}}-\en{products}\ru{произведения} \en{of the basis vectors}\ru{базисных векторов}
\en{are equal to}\ru{равны}
\en{the}\ru{дельте} Kronecker\ru{’а}\en{ delta}

\nopagebreak\vspace{-0.15em}\begin{equation*}
\bm{e}_i \dotp \hspace{.1ex} \bm{e}_j \hspace{-0.15ex}
= \hspace{.1ex} \delta_{i\hspace{-0.1ex}j} \hspace{-0.15ex}
= \hspace{.1ex} \scalebox{.95}{$
\left\{\hspace{.3ex}\begin{gathered}
1, \; i = j
\\[-0.25em]
0, \; i \neq j
\end{gathered}\right.
$}
\end{equation*}

\nopagebreak\vspace{-0.2em}\noindent
\hfill\hbox{\en{for any}\ru{для любого} \en{orthonormal}\ru{ортонормального} \en{basis}\ru{базиса}}.


\en{Decomposing vector}\ru{Разлагая вектор}~$\bm{v}$ \en{in some ortho\-normal basis}\ru{в~некотором орто\-нормаль\-ном базисе}~${\bm{e}_i}$ (${i = 1, 2, 3}$), \en{we get}\ru{получаем} \en{coefficients}\ru{коэффициенты}~$v_i$\:--- \en{components}\ru{компоненты} \en{of~vector}\ru{вектора}~$\bm{v}$ \en{in that basis}\ru{в~том базисе}~(\figref{fig:componentsofvector})

\end{minipage}

\nopagebreak\vspace{-0.33em}\begin{equation}\label{vectorcomponents}
\bm{v}
= v_1 \bm{e}_1 \hspace{-0.2ex} + v_2 \bm{e}_2 \hspace{-0.2ex} + v_3 \bm{e}_3 \hspace{-0.1ex}
\equiv \hspace{-0.1ex} \scalebox{.84}{$\displaystyle\sum_{i=1}^3$} \hspace{.2ex} {v_i \bm{e}_i} \hspace{-0.1ex}
\equiv \hspace{.1ex} v_i \bm{e}_i
\hspace{.1ex} ,
\hspace{.4em}
v_i \hspace{-0.2ex} = \bm{v} \dotp \bm{e}_i
\hspace{.1ex} .
\end{equation}

{\small
\setlength{\parindent}{0pt}

\begin{leftverticalbar}%%[oversize]
\en{Here and hereinafter,}\ru{Здесь и~далее} \ru{принимается соглашение о~суммировании }\en{the~}Einstein’\en{s}\ru{а}\en{ summation convention is accepted}:
\en{an~index repeated twice~(and no~more than twice) in a~single term}\ru{повторённый дважды~(и~не~более чем дважды) в~одночлене индекс} \en{implies summation over this index}\ru{подразумевает суммирование по~этому индексу}.
\en{And }\ru{А~}\en{a~non-repeating index}\ru{неповторяющийся индекс} \en{is called}\ru{называется} \inquotesx{\en{free}\ru{свободным}}[,] \en{it is identical in all parts of~the~equation}\ru{он одинаков во всех частях равенства}.
\en{These are examples}\ru{Это примеры}:

\nopagebreak\vspace{-0.2em}
\begin{equation*}
\sigma = \mathtau_{ii} \hspace{.1ex}, \;\;
p_{\hspace{-0.1ex}j} \hspace{-0.25ex} = n_i \mathtau_{i\hspace{-0.1ex}j}
\hspace{.1ex} , \;\;
m_i \hspace{-0.2ex} = e_{i\hspace{-0.1ex}j\hspace{-0.1ex}k} \hspace{.15ex} x_{\hspace{-0.1ex}j} \hspace{-0.1ex} f_{\hspace{-0.1ex}k}
\hspace{.1ex} , \;\;
a_i \hspace{-0.2ex} = \hspace{-0.1ex} \lambda b_i \hspace{-0.2ex} + \hspace{-0.1ex} \mu c_i
\hspace{.1ex} .
\end{equation*}





\vspace{-1.1em}
(\en{But equations}\ru{Равенства~же}
${a = b_{k\hspace{-0.1ex}k\hspace{-0.1ex}k} \hspace{.1ex}}$,
${c = f_{\hspace{-0.1ex}i} \hspace{-0.2ex} + \hspace{-0.1ex} g_k \hspace{.1ex}}$,
${a_{i\hspace{-0.1ex}j} \hspace{-0.25ex} = k_i \hspace{.2ex} \gamma_{i\hspace{-0.1ex}j}}$
\en{are incorrect}\ru{некорректны}.)
\end{leftverticalbar}
\par}

\en{Having components}\ru{Имея компоненты} \en{of a~vector}\ru{вектора} \en{in an~orthonormal basis}\ru{в~ортонормальном базисе},
\en{the~length}\ru{длина} \en{of this vector}\ru{этого вектора} \en{is retrieved}\ru{возвращается} \en{by }\en{\hbox{the~\hspace{-0.1ex}}}\inquotesx{\ru{равенством }\href{https://el.wikipedia.org/wiki/\%CE\%A0\%CF\%85\%CE\%B8\%CE\%B1\%CE\%B3\%CF\%8C\%CF\%81\%CE\%B1\%CF\%82}{\en{\textgreek{Πυθαγόρας}’\hspace{-0.2ex}}\ru{Пифагора~(\textgreek{Πυθαγόρας})}}\en{ equation}}

\nopagebreak\vspace{-0.15em}\begin{equation}\label{vectorlength}
\bm{v} \hspace{-0.1ex} \dotp \hspace{-0.1ex} \bm{v} \hspace{-0.1ex}
= v_i \bm{e}_i \hspace{-0.1ex} \dotp v_{\hspace{-0.1ex}j} \bm{e}_{\hspace{-0.1ex}j} \hspace{-0.25ex}
= v_i \hspace{.1ex} \delta_{i\hspace{-0.1ex}j} \hspace{.1ex} v_{\hspace{-0.1ex}j} \hspace{-0.25ex}
= v_i v_i
\hspace{.1ex} , \hspace{.5em}
\| \bm{v} \| \hspace{-0.1ex}
= \hspace{-0.1ex} \sqrt{\bm{v} \hspace{-0.1ex} \dotp \hspace{-0.1ex} \bm{v}} \hspace{-0.1ex}
= \hspace{-0.2ex} \sqrt{\vphantom{i} v_i v_i}
\hspace{.2ex} .
\end{equation}

\vspace{-0.1em}\noindent
\en{The~direction of~a~vector in space}\ru{Направление вектора в~пространстве} \en{is measured}\ru{измеряется} \en{by the three angles~(co\-sines of~angles)}\ru{тремя углами~(ко\-синусами углов)} \en{between}\ru{между} \en{this~vector}\ru{этим вектором} \en{and}\ru{и}~\en{each of the basis ones}\ru{каждым из базисных}:

\nopagebreak\vspace{-0.2em}\begin{equation}\label{anglesbetweenvectorandbasisvectors}
\operatorname{cos} \measuredangle \bigl( \bm{v} \widehat{\phantom{w}} \bm{e}_i \bigr) %%( \bm{v}, \bm{e}_i )
\hspace{-0.1ex}
= \scalebox{.9}{$ \displaystyle\frac{\hspace{.2ex}\raisemath{-0.16em}{\bm{v}}}{\raisemath{-0.05em}{\|\bm{v}\|}} $} \hspace{-0.1ex} \dotp \hspace{.1ex} \bm{e}_i \hspace{-0.1ex}
= \hspace{-0.1ex}\scalebox{.9}{$ \displaystyle\frac{\hspace{.2ex}\raisemath{-0.12em}{v_i}}{\raisemath{-0.1em}{\sqrt{\vphantom{i} v_{\hspace{-0.1ex}j} v_{\hspace{-0.1ex}j}}}} $}
\hspace{.4em}\Leftrightarrow\hspace{.4em}
\tikzmark{beginComponentOfVectorThruLengthAndCosine} v_i \hspace{.3ex} \tikzmark{endComponentOfVectorThruLengthAndCosine} \hspace{-0.4ex} = \|\bm{v}\| \hspace{-0.1ex} \operatorname{cos} \measuredangle ({\bm{v}, \bm{e}_i})
\hspace{.1ex} .
\end{equation}%
\AddUnderBrace[line width=.75pt][-0.1ex, 0][xshift=.12em, yshift=.1em]%
{beginComponentOfVectorThruLengthAndCosine}{endComponentOfVectorThruLengthAndCosine}%
{\scalebox{.75}{$ \bm{v} \hspace{-0.1ex} \dotp \hspace{-0.1ex} \bm{e}_i $}}
\vspace{-0.6em}

\emph{\en{Measurement of angles}\ru{Измерение углов}.}
\en{The cosine of the angle}\ru{Косинус угла} \en{between two vectors}\ru{между двумя векторами} is the same as the dot product of these vectors when they are normalized to both have the magnitude equal to the one unit of length

\nopagebreak\vspace{-0.2em}\begin{gather*}
\operatorname{cos} \measuredangle ({\bm{v}, \bm{w}}) \hspace{-0.1ex}
= \scalebox{.9}{$ \displaystyle\frac{\hspace{.2ex}\raisemath{-0.16em}{\bm{v}}}{\raisemath{-0.05em}{\|\bm{v}\|}} $} \hspace{-0.1ex} \dotp \scalebox{.9}{$ \displaystyle\frac{\hspace{.2ex}\raisemath{-0.16em}{\bm{w}}}{\raisemath{-0.05em}{\|\bm{w}\|}} $}
\hspace{.2ex} .
\end{gather*}

{\small
To accompany the magnitude, which represents the length independent of direction, there’s a~way to represent the direction of a~vector independent of its length.
For this purpose, the unit vectors are used, which are vectors with a~magnitude of~1.

A rotation matrix is just a transform that expresses the basis vectors of the input space in a different orientation.
The length of the basis vectors will be the same, and the origin will not change.
Also, the angle between the basis vectors will not change.
All that changes is the relative direction of all of the basis vectors.

Therefore, a rotation matrix is not really just a~“rotation” matrix;
it is an orientation matrix.

\en{There are also pseudovectors}\ru{Бывают ещё и~псевдовекторы}, \en{waiting for the~reader}\ru{ждущие читателя} \en{below}\ru{ниже} \en{in}\ru{в}~\pararef{para:crossproduct}.}\hbox{\hspace{-0.5ex}.
\par}

\emph{\en{The angle}\ru{Угол} \en{between two random vectors}\ru{между двумя случайными векторами}.}
\en{According to}\ru{Согласно}~\eqref{anglesbetweenvectorandbasisvectors}

\nopagebreak\vspace{-0.2em}\begin{gather*}
\operatorname{cos} \measuredangle ({\bm{v}, \bm{e}_m}) \hspace{-0.1ex}
= \scalebox{.9}{$ \displaystyle\frac{\hspace{.2ex}\raisemath{-0.16em}{\bm{v}}}{\raisemath{-0.05em}{\|\bm{v}\|}} $} \hspace{-0.1ex} \dotp \bm{e}_m \hspace{-0.1ex}
= \hspace{-0.1ex}\scalebox{.9}{$ \displaystyle\frac{\hspace{.2ex}\raisemath{-0.12em}{v_m}}{\raisemath{-0.1em}{\sqrt{\vphantom{i} v_{\hspace{-0.1ex}j} v_{\hspace{-0.1ex}j}}}} $}
\hspace{.1ex} ,
\\
%
\operatorname{cos} \measuredangle ({\bm{w}, \bm{e}_n}) \hspace{-0.1ex}
= \scalebox{.9}{$ \displaystyle\frac{\hspace{.2ex}\raisemath{-0.16em}{\bm{w}}}{\raisemath{-0.05em}{\|\bm{w}\|}} $} \hspace{-0.1ex} \dotp \bm{e}_n \hspace{-0.1ex}
= \hspace{-0.1ex}\scalebox{.9}{$ \displaystyle\frac{\hspace{.2ex}\raisemath{-0.12em}{w_n}}{\raisemath{-0.1em}{\sqrt{\vphantom{i} w_{\hspace{-0.1ex}k} w_{\hspace{-0.1ex}k}}}} $}
\hspace{.1ex} .
\end{gather*}

\en{The~length}\ru{Длина}~\eqref{vectorlength}
\en{and}\ru{и}~\en{the~direction in~space}\ru{направление в~пространстве
\eqref{anglesbetweenvectorandbasisvectors}\:---
\en{measurable}\ru{измер\'{и}мые}
\en{by the means of}\ru{посредством}
\en{the~trio}\ru{трио}
\en{of~basic vectors}\ru{базисных векторов}\:---
\en{describe}\ru{описывают}
\en{a~vector}\ru{вектор},
\en{and}\ru{и}
\en{every vector}\ru{каждый вектор}
\en{possesses these properties}\ru{обладает этими свойствами}%
\footnote{ \en{And what is the direction}\ru{И~как\'{о}е~же направление}
\en{of the~null vector}\ru{у~нуль\hbox{-}вектора}
$\bm{0}$
\en{with zero length}\ru{с~нулевой длиной}
${ \| \bm{0} \| \hspace{-0.25ex} = \hspace{-0.1ex} 0 }$?
(\en{The zero vector}\ru{Нулевой вектор}
\en{with no magnitude}\ru{без магнитуды}
\en{ends exactly where it begins}\ru{кончается точно там~же, где он начинается}
\en{and}\ru{и}
\en{is not directed anywhere}\ru{никуда не направлен},
\en{thus}\ru{поэтому}
\en{its direction}\ru{его направление}
\en{is }\emph{\en{undefined}\ru{не определено}}.) }%
\hbox{\hspace{-0.5ex}.}
\en{However}\ru{Однако},
\en{this is not~enough}\ru{этого м\'{а}ло} (\inquotes{\href{https://en.wikipedia.org/wiki/Necessity_and_sufficiency}{\en{not}\ru{не}~\en{sufficient}\ru{достаточно}}}
\en{in jargon}\ru{на жаргоне}
\en{of the math books}\ru{книг по~математике}).

\vspace{.1em}
\hspace*{-\parindent}\begin{minipage}{\linewidth}
\setlength{\parindent}{\horizontalindent}
\setlength{\parskip}{\spacebetweenparagraphs}

\begin{wrapfigure}[17]{o}{.5\textwidth}
\makebox[.36\textwidth][c]{\begin{minipage}[t]{.5\textwidth}
\vspace{-0.8em}
\scalebox{.93}{
\tdplotsetmaincoords{35}{75}


\begin{tikzpicture}[scale=3.2, tdplot_main_coords]

	\coordinate (O) at (0, 0, 0) ;

	% three coordinates of vector
	\pgfmathsetmacro{\Vx}{0.88}
	\pgfmathsetmacro{\Vy}{1.33}
	\pgfmathsetmacro{\Vz}{2.2}
	\coordinate (Vx) at (\Vx, 0, 0) ;
	\coordinate (Vy) at (0, \Vy, 0) ;
	\coordinate (Vz) at (0, 0, \Vz) ;
	\coordinate (V) at ($ (Vx) + (Vy) + (Vz) $) ;

	% draw vector
	\draw [line width=1.6pt, black, -{Stealth[round, length=5mm, width=2.8mm]}]
		(O) -- (V)
		node[pos=0.93, below right, inner sep=0pt, outer sep=4.4pt] {\scalebox{1.2}[1.2]{${\bm{v}}$}};

	\tdplotsetrotatedcoords{21}{-15}{-43} % 3-2-3 rotation sequence
	\begin{scope}[tdplot_rotated_coords]

	% draw axes

	\draw [line width=1.25pt, blue, -{Latex[round, length=3.6mm, width=2.4mm]}]
		(O) -- (1,0,0)
		node[pos=0.86, above left, inner sep=0pt, outer sep=2.5pt] {${\bm{e}}_1$};

	\draw [line width=1.25pt, blue, -{Latex[round, length=3.6mm, width=2.4mm]}]
		(O) -- (0,1,0)
		node[pos=0.9, above, inner sep=0pt, outer sep=5pt] {${\bm{e}}_3$};

	\draw [line width=1.25pt, blue, -{Latex[round, length=3.6mm, width=2.4mm]}]
		(O) -- (0,0,1)
		node[pos=0.92, below left, inner sep=0pt, outer sep=3.3pt] {${\bm{e}}_2$};

	%%\draw [line width=.4pt, dotted, color=blue] (1,0,0) -- (0,1,0);
	%%\draw [line width=.4pt, dotted, color=blue] (0,1,0) -- (0,0,1);
	%%\draw [line width=.4pt, dotted, color=blue] (0,0,1) -- (1,0,0);

	\end{scope}

	% get projections of vector
	\tdplottransformmainrot{\Vx}{\Vy}{\Vz}
	\pgfmathsetmacro{\Vrotx}{\tdplotresx}
	\pgfmathsetmacro{\Vroty}{\tdplotresy}
	\pgfmathsetmacro{\Vrotz}{\tdplotresz}
	%%\draw [tdplot_rotated_coords, line width=1.6pt, blue, -{Stealth[round, length=5mm, width=2.8mm]}]
		%%(O) -- (\Vrotx, \Vroty, \Vrotz)
		%%node[pos=0.8, above left, inner sep=0pt, outer sep=2.5pt] {\scalebox{1.2}[1.2]{${\bm{v}}$}};

	% draw components of vector

	\draw [tdplot_rotated_coords, color=blue!50!black, line width=1.6pt, line cap=round, dash pattern=on 0pt off 1.6\pgflinewidth,
		-{Stealth[round, length=4mm, width=2.4mm]}]
		(O) -- (\Vrotx, 0, 0)
		node[pos=0.48, below right, inner sep=0pt, outer sep=2.5pt] {${v_1 \hspace{-0.1ex} \bm{e}_1}$};

	\draw [tdplot_rotated_coords, color=blue!50!black, line width=1.6pt, line cap=round, dash pattern=on 0pt off 1.6\pgflinewidth,
		-{Stealth[round, length=4mm, width=2.4mm]}]
		(\Vrotx, 0, 0) -- (\Vrotx, 0, \Vrotz)
		node[pos=0.47, below left, inner sep=0pt, outer sep=3.3pt] {${v_2 \bm{e}_2}$};

	\draw [tdplot_rotated_coords, color=blue!50!black, line width=1.6pt, line cap=round, dash pattern=on 0pt off 1.6\pgflinewidth,
		-{Stealth[round, length=4mm, width=2.4mm]}]
		(\Vrotx, 0, \Vrotz) -- (\Vrotx, \Vroty, \Vrotz)
		node[pos=0.5, above, inner sep=0pt, outer sep=3.5pt] {${v_3 \bm{e}_3}$};

	\tdplotsetrotatedcoords{2}{-8}{28} % 3-2-3 rotation sequence

	% draw axes

	\draw [tdplot_rotated_coords, line width=1.25pt, red, -{Latex[round, length=3.6mm, width=2.4mm]}]
		(O) -- (1, 0, 0)
		node[pos=.86, above right, inner sep=0pt, outer sep=1.3pt] {${\bm{e}'\hspace{-0.6ex}}_{\raisemath{-0.15ex}{1}}$};

	\draw [tdplot_rotated_coords, line width=1.25pt, red, -{Latex[round, length=3.6mm, width=2.4mm]}]
		(O) -- (0, 1, 0)
		node[pos=.9, above left, inner sep=0pt, outer sep=2.7pt] {${\bm{e}'\hspace{-0.6ex}}_{\raisemath{-0.15ex}{3}}$};

	\draw [tdplot_rotated_coords, line width=1.25pt, red, -{Latex[round, length=3.6mm, width=2.4mm]}]
		(O) -- (0, 0, 1)
		node[pos=.98, below right, inner sep=0pt, outer sep=4.4pt] {${\bm{e}'\hspace{-0.6ex}}_{\raisemath{-0.15ex}{2}}$};

	%%\draw [tdplot_rotated_coords, line width=.4pt, dotted, color=red] (1, 0, 0) -- (0, 1, 0);
	%%\draw [tdplot_rotated_coords, line width=.4pt, dotted, color=red] (0, 1, 0) -- (0, 0, 1);
	%%\draw [tdplot_rotated_coords, line width=.4pt, dotted, color=red] (0, 0, 1) -- (1, 0, 0);

	% get projections of vector
	\tdplottransformmainrot{\Vx}{\Vy}{\Vz}
	\pgfmathsetmacro{\Vrotx}{\tdplotresx}
	\pgfmathsetmacro{\Vroty}{\tdplotresy}
	\pgfmathsetmacro{\Vrotz}{\tdplotresz}

	% draw components of vector

	\draw [tdplot_rotated_coords, color=red!50!black, line width=1.6pt, line cap=round, dash pattern=on 0pt off 1.6\pgflinewidth,
		-{Stealth[round, length=4mm, width=2.4mm]}]
		(O) -- (\Vrotx, 0, 0)
		node[pos=.81, below left, inner sep=0pt, outer sep=2pt] {${v'_{\raisemath{-0.15ex}{\hspace{-0.1ex}1}} \hspace{-0.1ex} \bm{e}'_{\raisemath{-0.15ex}{1}}}$};

	\draw [tdplot_rotated_coords, color=red!50!black, line width=1.6pt, line cap=round, dash pattern=on 0pt off 1.6\pgflinewidth,
		-{Stealth[round, length=4mm, width=2.4mm]}]
		(\Vrotx, 0, 0) -- (\Vrotx, 0, \Vrotz)
		node[pos=.4, above right, inner sep=0pt, outer sep=3.3pt] {${v'_{\raisemath{-0.15ex}{\hspace{-0.1ex}2}} \bm{e}'_{\raisemath{-0.15ex}{2}}}$};

	\draw [tdplot_rotated_coords, color=red!50!black, line width=1.6pt, line cap=round, dash pattern=on 0pt off 1.6\pgflinewidth,
		-{Stealth[round, length=4mm, width=2.4mm]}]
		(\Vrotx, 0, \Vrotz) -- (\Vrotx, \Vroty, \Vrotz)
		node[pos=.26, below right, inner sep=0pt, outer sep=1pt] {${v'_{\raisemath{-0.15ex}{\hspace{-0.1ex}3}} \bm{e}'_{\raisemath{-0.15ex}{3}}}$};

\end{tikzpicture}



}
\vspace{-2.2em}\caption{}\label{fig:vectorintwoorthonormalbases}
\end{minipage}}
\end{wrapfigure}

\en{A~vector}\ru{Вектор} \en{is}\ru{ведь} \en{not just}\ru{не~просто} \en{a~collection}\ru{совокупность} \en{of components}\ru{компонент} \en{in some basis}\ru{в~ка\-к\'{о}м\hbox{-}то базисе}.

\vspace{-0.1em}
\en{A~triple}\ru{Тройка} \en{of pairwise perpendicular}\ru{попарно перпендикулярных} \en{unit vectors}\ru{единичных векторов} \en{can only rotate}\ru{может только поворачиваться} \en{and thereby}\ru{и~тем самым} \en{characterize}\ru{характеризовать} \en{the angular orientation}\ru{угловую ориентацию}.

\en{The decomposition}\ru{Разложение} \en{of the~same vector}\ru{одного и~того~же вектора}~${\bm{v}}$
\en{in the two cartesian systems}\ru{в~двух декартовых системах}
\en{with basis unit vectors}\ru{с~базисными ортами}
${ \bm{e}_{i}}$ \en{and}\ru{и}~${\bm{e}'_{i} }$
(\figref{fig:vectorintwoorthonormalbases})
\en{gives}\ru{даёт}

\nopagebreak\vspace{-0.2em}\begin{equation*}
\bm{v} = v_{i} \hspace{.1ex} \bm{e}_{i} \hspace{-0.16ex} = v'_{\hspace{-0.1ex}i} \hspace{.1ex} \bm{e}'_{i}
\hspace{.15ex} ,
\end{equation*}

\vspace{-0.8em}\noindent
\en{where}\ru{где}

\nopagebreak\vspace{-0.2em}\ru{\vspace{-0.3em}}\begin{equation*}
\begin{gathered}
v_i \hspace{-0.2ex} = \bm{v} \dotp \bm{e}_i \hspace{-0.2ex} = v'_{\hspace{-0.1ex}k} \hspace{.1ex} \bm{e}'_{k} \hspace{-0.2ex} \dotp \bm{e}_i
\hspace{.1ex} ,
\\[-0.1em]
%
v'_{\hspace{-0.1ex}i} \hspace{-0.2ex} = \bm{v} \dotp \bm{e}'_{i} \hspace{-0.2ex} = v_k \hspace{.1ex} \bm{e}_k \hspace{-0.15ex} \dotp \bm{e}'_{i}
\hspace{.1ex} .
\end{gathered}
\end{equation*}

\end{minipage}

\vspace{.1em}\noindent
\en{Appeared here}\ru{Возникшие тут}
\en{two\hbox{-}index}\ru{двухиндексные} \en{objects}\ru{объекты}
(\en{the two\hbox{-}dimensional arrays}\ru{двумерные массивы})
${
  \cosinematrix{k'\hspace{-0.1ex}i}
  \hspace{-0.2ex} \equiv
  \bm{e}'_{k}
  \hspace{-0.2ex} \dotp
  \bm{e}_i
}$ \en{and}\ru{и} ${
  \cosinematrix{ki'}
  \hspace{-0.25ex} \equiv
  \bm{e}_k
  \hspace{-0.15ex} \dotp
  \bm{e}'_{i}
}$
\en{are used}\ru{используются}
\en{to shorten formulas}\ru{для укорочения формул}.

%%\en{are called}\ru{называются} \inquotes{\en{matrices of~cosines}\ru{матрицы косинусов}} \en{or}\ru{или} \inquotes{\en{rotation/orientation matrices}\ru{матрицы поворота/ориентации}}

\en{The~}\inquotes{${\dotp\hspace{.25ex}}$}\hbox{\hspace{-0.2ex}-}\en{product}\ru{произведение}
(dot product)
\en{of two vectors}\ru{двух векторов}
\en{is commutative}\ru{коммутативно}\:---
\en{that is}\ru{то есть},
\en{the swapping}\ru{обмен местами}
\en{of multipliers}\ru{множителей}
\en{doesn’t change}\ru{не~меняет}
\en{the~result}\ru{результат}.
\en{Thus}\ru{Так что}

\nopagebreak\vspace{-0.2em}\en{\vspace{-0.3em}}
\begin{gather*}
\cosinematrix{k'\hspace{-0.1ex}i} \hspace{-0.2ex}
= \bm{e}'_{k} \hspace{-0.2ex} \dotp \bm{e}_i
= \operatorname{cos} \measuredangle (
    \bm{e}'_{k}
    \hspace{.1ex} , \hspace{-0.1ex}
    \bm{e}_i
) \hspace{-0.1ex}
= \operatorname{cos} \measuredangle (
    \bm{e}_{i}
    \hspace{.1ex} , \hspace{-0.1ex}
    \bm{e}'_{k}
\hspace{.1ex}) \hspace{-0.1ex}
= \bm{e}_{i} \hspace{-0.15ex} \dotp \bm{e}'_k \hspace{-0.25ex}
= \cosinematrix{ik'} ,
\tag{\theequation a}\label{eq:firstlineofcosines}
\\[-0.1em]
%
\cosinematrix{ki'} \hspace{-0.25ex}
= \bm{e}_k \hspace{-0.15ex} \dotp \bm{e}'_{i} \hspace{-0.25ex}
= \operatorname{cos} \measuredangle (
    \bm{e}_k
    \hspace{.1ex} , \hspace{-0.1ex}
    \bm{e}'_{i}
\hspace{.1ex} ) \hspace{-0.1ex}
= \operatorname{cos} \measuredangle (
    \bm{e}'_{i}
    \hspace{.1ex} , \hspace{-0.1ex}
    \bm{e}_k
) \hspace{-0.1ex}
= \bm{e}'_{i} \hspace{-0.2ex} \dotp \bm{e}_k \hspace{-0.2ex}
= \cosinematrix{i'\hspace{-0.1ex}k}
\hspace{.1ex} .
\tag{\theequation b}\label{eq:secondlineofcosines}
\end{gather*}

\noindent
\en{The lines of~equalities}\ru{Строки равенств}~\eqref{eq:firstlineofcosines}
\en{and}\ru{и}~\eqref{eq:secondlineofcosines}
\en{are mutually reciprocal}\ru{взаимно-обратны}
\en{by multiplication}\ru{по умножению}

\nopagebreak\vspace{-0.2em}\begin{gather*}
\cosinematrix{k'\hspace{-0.1ex}i} \hspace{.1ex} \cosinematrix{ki'}
\hspace{-0.2ex}
= \cosinematrix{ki'} \hspace{.1ex} \cosinematrix{k'\hspace{-0.1ex}i} \hspace{-0.2ex}
= 1
\hspace{.1ex} ,
\hspace{.4em}
%
\cosinematrix{k'\hspace{-0.1ex}i} \hspace{.1ex} \cosinematrix{i'\hspace{-0.1ex}k} \hspace{-0.2ex}
= \cosinematrix{i'\hspace{-0.1ex}k} \hspace{.1ex} \cosinematrix{k'\hspace{-0.1ex}i} \hspace{-0.2ex}
= 1
\hspace{.1ex} .
\end{gather*}

\en{Multiplying}\ru{Умножение}
\en{of an orthogonal matrix}\ru{ортогональной матрицы}
\en{by the components}\ru{на компоненты}
\en{of any vector}\ru{любого вектора}
\en{retains}\ru{сохраняет}
\en{the~length}\ru{длину}
\en{of this vector}\ru{этого вектора}:

\nopagebreak\vspace{-0.25em}\begin{equation*}
\|\bm{v}\|^2 \hspace{-0.22ex} = \bm{v} \dotp \bm{v}
= v'_i \hspace{.1ex} v'_i \hspace{-0.2ex}
= \cosinematrix{i'\hspace{-0.1ex}k} \hspace{.1ex} v_k \hspace{.2ex} \cosinematrix{i'\hspace{-0.1ex}n} \hspace{.1ex} v_n \hspace{-0.2ex}
= v_n v_n
\end{equation*}

\nopagebreak\vspace{-0.25em}\noindent
--- \en{this conclusion}\ru{этот вывод} \textcolor{magenta}{\en{leans on}\ru{опирается на}}~\eqref{orthogonalityofcosinematrix}.

\vspace{.2em}
\href{https://en.wikipedia.org/wiki/Orthogonal_transformation}{\en{Orthogonal transformation}\ru{Ортогональное преобразование}}
\en{of the vector components}\ru{компонент вектора}

\nopagebreak\vspace{-0.3em}
\begin{equation}\label{orthotransform:1}
\bm{v} \dotp \bm{e}'_i \hspace{-0.2ex}
= v_k \hspace{.1ex} \bm{e}_k \hspace{-0.15ex} \dotp \bm{e}'_i \hspace{-0.2ex}
= \bm{e}'_i \hspace{-0.15ex} \dotp \bm{e}_k \hspace{.1ex} v_k \hspace{-0.2ex}
= \cosinematrix{i'\hspace{-0.1ex}k} \hspace{.1ex} v_k \hspace{-0.2ex}
= v'_i
\end{equation}

\vspace{-0.2em}\noindent
\en{is sometimes used}\ru{иногда используется}
\en{for}\ru{для}
\en{defining}\ru{определения}
\en{a~vector itself}\ru{самог\'{о} вектора}.
\en{If}\ru{Если}
\en{in each}\ru{в~каждом}
\en{orthonormal}\ru{ортонормальном}
\en{basis}\ru{базисе}~${\bm{e}_i}$
\ru{известна }\en{a~triplet of~numbers}\ru{тройка чисел}~${v_i}$\en{ is known},
\en{and}\ru{и}
\en{with}\ru{с}
\en{a~rotation}\ru{вращением}
\en{of the basis}\ru{базиса}
\en{as a~whole}\ru{как целого}
\en{it is transformed}\ru{он преобразуется}
\en{according to}\ru{согласно}~\eqref{orthotransform:1}.
\en{then}\ru{тогда}
\en{this triplet of~components}\ru{эта тройка компонент}
\en{represents}\ru{представляет}
\en{an~invariant object}\ru{инвариантный объект}\:---
\en{vector}\ru{вектор}~${\bm{v}}$.

=======
--- theparagraph1.1.tex
+++ theparagraph1.1.tex
@@ -139,11 +139,15 @@
 \end{wrapfigure}
 
 \en{Introduce rectangular}\ru{Введём прямоугольные}~(\inquotes{\en{cartesian}\ru{декартовы}}) \en{coordinates}\ru{координаты} \en{by picking}\ru{выбором} \en{some}\ru{каких-либо} \en{three}\ru{трёх} \en{mutually perpendicular unit vectors}\ru{взаимно перпендикулярных единичных векторов} ${\bm{e}_1}$,\;${\bm{e}_2}$,\;${\bm{e}_3}$ \en{as}\ru{как} \en{a~\hbox{basis}}\ru{основы~(\hbox{базиса})} \en{for measurements}\ru{для измерений}.
-\en{Within such}\ru{В~так\'{о}й} \en{a~}\hbox{\en{system}\ru{системе}}\en{,} \hbox{\hspace{-0.2ex}\inquotes{${\dotp\hspace{.22ex}}$}\hspace{-0.2ex}}-\en{products}\ru{произведения} \en{of~\hbox{basis} vectors}\ru{базисных векторов} \en{are equal to}\ru{равны} \en{the~\hbox{Kronecker} delta}\ru{\hbox{дельте} \hbox{Kronecker’а}}:
+\en{Within a~such}\ru{В~так\'{о}й}
+\en{system}\ru{системе}\en{,} \hbox{\hspace{-0.2ex}\inquotes{${\dotp\hspace{.22ex}}$}\hspace{-0.2ex}}-\en{products}\ru{произведения}
+\en{of the basis vectors}\ru{базисных векторов}
+\en{are equal to}\ru{равны}
+\en{the}\ru{дельте} Kronecker’\en{s}\ru{а}\en{ delta}
 
 \nopagebreak\vspace{-0.15em}\begin{equation*}
 \bm{e}_i \dotp \hspace{.1ex} \bm{e}_j \hspace{-0.15ex}
-= \hspace{.1ex} \delta_{i\hspace{-0.1ex}j} \hspace{-0.15ex}
+= \hspace{.1ex} \KroneckerDelta{i\hspace{-0.1ex}j} \hspace{-0.15ex}
 = \hspace{.1ex} \scalebox{.95}{$
 \left\{\hspace{.3ex}\begin{gathered}
 1, \; i = j
@@ -156,8 +160,14 @@
 \nopagebreak\vspace{-0.2em}\noindent
 \hfill\hbox{\en{for any}\ru{для любого} \en{orthonormal}\ru{ортонормального} \en{basis}\ru{базиса}}.
 
-
-\en{Decomposing vector}\ru{Разлагая вектор}~$\bm{v}$ \en{in some ortho\-normal basis}\ru{в~некотором орто\-нормаль\-ном базисе}~${\bm{e}_i}$ (${i = 1, 2, 3}$), \en{we get}\ru{получаем} \en{coefficients}\ru{коэффициенты}~$v_i$\:--- \en{components}\ru{компоненты} \en{of~vector}\ru{вектора}~$\bm{v}$ \en{in that basis}\ru{в~том базисе}~(\figref{fig:componentsofvector})
+\en{Decomposing vector}\ru{Разлагая вектор}~$\bm{v}$
+\en{in some ortho\-normal basis}\ru{в~некотором орто\-нормаль\-ном базисе}
+${\bm{e}_i}$ (${i = 1, 2, 3}$),
+\en{we get}\ru{получаем}
+\en{coefficients}\ru{коэффициенты}~$v_i$\:---
+\en{the components}\ru{компоненты} \en{of~vector}\ru{вектора}~$\bm{v}$
+\en{in that basis}\ru{в~том базисе}
+(\figref{fig:componentsofvector}).
 
 \end{minipage}
 
@@ -211,7 +221,7 @@
 \nopagebreak\vspace{-0.15em}\begin{equation}\label{vectorlength}
 \bm{v} \hspace{-0.1ex} \dotp \hspace{-0.1ex} \bm{v} \hspace{-0.1ex}
 = v_i \bm{e}_i \hspace{-0.1ex} \dotp v_{\hspace{-0.1ex}j} \bm{e}_{\hspace{-0.1ex}j} \hspace{-0.25ex}
-= v_i \hspace{.1ex} \delta_{i\hspace{-0.1ex}j} \hspace{.1ex} v_{\hspace{-0.1ex}j} \hspace{-0.25ex}
+= v_i \hspace{.1ex} \KronekerDelta{i\hspace{-0.1ex}j} \hspace{.1ex} v_{\hspace{-0.1ex}j} \hspace{-0.25ex}
 = v_i v_i
 \hspace{.1ex} , \hspace{.5em}
 \| \bm{v} \| \hspace{-0.1ex}
>>>>>>> 63698db (edit 1.1 and 4.4)
