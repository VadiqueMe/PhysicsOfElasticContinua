\section*{\small \wordforbibliography}

\begin{changemargin}{\parindent}{0pt}
\fontsize{10}{12}\selectfont

\en{There are}\ru{Существует}
\en{several}\ru{несколько}
\en{dozen}\ru{дюжин}
\en{books}\ru{книг}
\en{on}\ru{по}
\en{the classical}\ru{классической}
\en{linear}\ru{линейной}
\en{theory of~elasticity}\ru{теории упругости}\ru{,}
\en{that}\ru{которые}
\en{haven’t lost}\ru{не~потеряли}
\en{their relevance}\ru{своей актуальности}
\en{over time}\ru{со~временем}.
%
%
\en{First of all}\ru{Прежде всего},
\en{the fundamental mono\-graph}\ru{фундаментальная моно\-графия}
\en{by }\russianlanguage{Анатоли\en{й}\ru{я} Лурье}\en{~(Anatoliy Lurie)}
\cite{lurie-theoryofelasticity}
\en{and}\ru{и}~%
\en{his earlier}\ru{более ранняя его}
\en{book}\ru{книга}~\cite{lurie-spatialproblems}
\en{about solving}\ru{про решение}
\en{spatial problems}\ru{пространственных проблем}.
%
%
\en{Quite rich}\ru{Весьма богата}
\en{in content}\ru{по содержанию}
\en{is the}\ru{книга} Witold\ru{’а} Nowacki\en{’s}\en{ book}~\cite{nowacki-elasticity}.
%
\en{There}\ru{Там}
\en{the~author}\ru{автор}
\en{spent}\ru{потратил}
\en{many pages}\ru{много страниц}
\en{describing}\ru{на описание}
\en{the problems}\ru{проблем}
\en{of~both statics}\ru{и~статики,}
\en{and}\ru{и}\hspace{1ex}\inquotes{\en{elastokinetics}\ru{эластокинетики}}
(\en{that is dynamics}\ru{то есть динамики}),
\en{and}\ru{а}~%
\en{the~last chapter}\ru{последняя глава}
\en{of~this book}\ru{этой книги}
\en{describes}\ru{описывает}
\en{the linear}\ru{линейный}
\ru{\rucontinuum }Cosserat\en{ continuum}\:---
\en{that’s what the~next chapter is about}\ru{это то, о~чём следующая глава}.
%
%
\en{Being}\ru{Будучи}
\en{mathematically}\ru{математически}
\en{capacious and saturated}\ru{ёмкой и~насыщенной},
\en{the theory of~elasticity}\ru{теория упругости}
\en{attracts}\ru{привлекает}
\en{mathematicians}\ru{математиков},
\en{as it happened}\ru{как это случилось}
\en{with the~monograph}\ru{с~монографией}~\cite{ciarlet-mathematicalelasticity}
\en{by }Philippe\ru{’а} Ciarlet.
%
%
\en{The}\ru{Книга}~Augustus\ru{’а} Love’\en{s}\ru{а}\en{ book}~\cite{love-mathematicaltheoryofelasticity}
\ru{также }\en{cannot}\ru{не может}
\en{go unmentioned}\ru{остаться неупомянутой}\en{ as well}.
%
%
\russianlanguage{Климентий Черн\'{ы}х}\en{ (Klimentiy Chernih)}
\en{described}\ru{описал}
\en{in}\ru{в}~\cite{chernyh-anisotropicelasticity}
\en{how}\ru{как}
\en{to~model}\ru{моделировать}
\en{linear elastic}\ru{линейно упругие}
\en{media}\ru{среды}
\en{with anisotropy}\ru{с~анизотропией}.

\end{changemargin}
