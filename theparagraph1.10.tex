\en{\section{Rotation tensor}}

\ru{\section{Тензор поворота}}

\label{para:rotationtensor}

\en{The relation}\ru{Соотношение} \en{between}\ru{между} \en{two}\ru{двумя} \inquotes{\en{right}\ru{правыми}} (\en{or}\ru{или} \en{two}\ru{двумя} \inquotes{\en{left}\ru{левыми}}) \en{orthonormal}\ru{ортонормальными} \en{bases}\ru{базисами} ${\bm{e}_i}$ \en{and}\ru{и}~${\mathcircabove{\bm{e}}_i}$ \en{can be described}\ru{может быть описано} \en{by }\en{a~two\hbox{-}index array}\ru{двухиндексным массивом}\ru{,} \en{represented as a~matrix}\ru{представленным в~виде матрицы}~(\pararef{para:vector}, \pararef{para:matrices+permutations+determinants})

\nopagebreak\vspace{-0.2em}\begin{equation*}
\bm{e}_i \hspace{-0.2ex} = \bm{e}_i \dotp \hspace{.1ex} \tikzmark{beginEqualsE} \mathcircabove{\bm{e}}_j \mathcircabove{\bm{e}}_j \tikzmark{endEqualsE} \hspace{-0.1ex} = \hspace{.1ex} \cosinematrix{\hspace{-0.2ex}i\mathcircabove{j}} \, \mathcircabove{\bm{e}}_j
\hspace{.1ex} ,
\hspace{.4em}
%
\cosinematrix{\hspace{-0.2ex}i\mathcircabove{j}} \hspace{.1ex} \equiv \hspace{.1ex} \bm{e}_i \dotp \mathcircabove{\bm{e}}_j
%%\hspace{.1ex} .
\end{equation*}
\AddUnderBrace[line width=.75pt][0,-0.2ex]%
{beginEqualsE}{endEqualsE}{${\scriptstyle \UnitDyad}$}

\vspace{-0.3em}\noindent
(\inquotes{\en{a~matrix of~cosines}\ru{матрицы косинусов}}).

\en{Also}\ru{Также}, \en{a~rotation}\ru{поворот} \en{of a~tensor}\ru{тензора} \en{can be described}\ru{может быть описан} \en{by another tensor}\ru{другим тензором}, \en{called}\ru{называемым} \en{rotation tensor}\ru{тензором поворота}~$\rotationtensor$

\nopagebreak\vspace{-0.2em}\begin{equation}\label{introductionofrotationtensor}
\bm{e}_i = \bm{e}_j \hspace{.2ex} \tikzmark{beginEqualsKroneckerDelta} \mathcircabove{\bm{e}}_j \hspace{-0.2ex} \dotp \mathcircabove{\bm{e}}_i \tikzmark{endEqualsKroneckerDelta} = \rotationtensor \hspace{-0.2ex} \dotp \mathcircabove{\bm{e}}_i \hspace{.1ex}, \:\,
\rotationtensor \equiv \bm{e}_j \mathcircabove{\bm{e}}_j = \scalebox{.96}[1]{$ \bm{e}_1 \hspace{-0.1ex} \mathcircabove{\bm{e}}_1 + \bm{e}_2 \mathcircabove{\bm{e}}_2 + \bm{e}_3 \mathcircabove{\bm{e}}_3 $}
\hspace{.1ex} .
\end{equation}
\AddUnderBrace[line width=.75pt][.1ex,-0.2ex]%
{beginEqualsKroneckerDelta}{endEqualsKroneckerDelta}{${\scriptstyle \delta_{j\hspace{-0.06ex}i}}$}

\en{Components of}\ru{Компоненты}~$\rotationtensor$ \en{both }\ru{и~в~начальном}\en{in an~initial}~${\mathcircabove{\bm{e}}_i}$\ru{,} \en{and}\ru{и}~\en{in a~rotated}\ru{в~повёрнутом}~${\bm{e}_i}$ \en{bases}\ru{базисах} \en{are the same}\ru{одни и~те~же}

\nopagebreak\vspace{-0.2em}\begin{equation}\label{componentsofrotationtensor}
\begin{array}{c}
\bm{e}_{i} \dotp \rotationtensor \hspace{-0.1ex} \dotp \bm{e}_{j} =
\hspace{.2ex} \tikzmark{beginEqualsKroneckerDeltaPresent} \bm{e}_{i} \hspace{-0.1ex} \dotp \bm{e}_{k} \tikzmark{endEqualsKroneckerDeltaPresent} \hspace{.2ex} \mathcircabove{\bm{e}}_{k} \hspace{-0.1ex} \dotp \bm{e}_{j} =
\hspace{.2ex} \mathcircabove{\bm{e}}_{i} \hspace{-0.1ex} \dotp \bm{e}_{j}
\hspace{.1ex} ,
\\[1.4em]
%
\mathcircabove{\bm{e}}_{i} \dotp \rotationtensor \hspace{-0.1ex} \dotp \mathcircabove{\bm{e}}_{j} =
\hspace{.2ex} \mathcircabove{\bm{e}}_{i} \hspace{-0.1ex} \dotp \bm{e}_{k} \hspace{.2ex} \tikzmark{beginEqualsKroneckerDeltaPast} \mathcircabove{\bm{e}}_{k} \hspace{-0.1ex} \dotp \mathcircabove{\bm{e}}_{j} \tikzmark{endEqualsKroneckerDeltaPast} =
\hspace{.2ex} \mathcircabove{\bm{e}}_{i} \hspace{-0.1ex} \dotp \bm{e}_{j}
\hspace{.1ex} .
\end{array}
\end{equation}
\AddUnderBrace[line width=.75pt][.1ex, -0.2ex]%
{beginEqualsKroneckerDeltaPresent}{endEqualsKroneckerDeltaPresent}{${\scriptstyle \delta_{ik}}$}
\AddUnderBrace[line width=.75pt][0, -0.2ex]%
{beginEqualsKroneckerDeltaPast}{endEqualsKroneckerDeltaPast}{${\scriptstyle \delta_{kj}}$}

\vspace{-0.1em}\noindent
\en{In matrix notation}\ru{В матричной записи}\en{,}
\en{these components}\ru{эти компоненты} \en{present}\ru{представляют} \en{the }\textcolor{red}{???\en{transposed}\ru{транспонированную}????} \en{matrix of~cosines}\ru{матрицу косинусов}~${\cosinematrix{\!j\mathcircabove{i}} = \hspace{.1ex} \mathcircabove{\bm{e}}_i \dotp \bm{e}_j}$\hspace{.1ex}:

\nopagebreak\vspace{-0.3em}\begin{equation*}
\rotationtensor = \cosinematrix{\!j\mathcircabove{i}} \hspace{.5ex} {\bm{e}}_{i} {\bm{e}}_{j} = \cosinematrix{\!j\mathcircabove{i}} \hspace{.5ex} \mathcircabove{\bm{e}}_{i} \mathcircabove{\bm{e}}_{j}
\hspace{.1ex} .
\end{equation*}

{\small
Spatial transformations in the 3-dimensional Euclidean space ${\mathbb{R}^{3}}$ are distinguished into active or alibi transformations, and passive or alias transformations.
An active transformation is a transformation which actually changes the physical position (alibi, elsewhere) of objects, which can be defined in the absence of a coordinate system; whereas a passive transformation is merely a change in the coordinate system in which the object is described (alias, other name) (change of coordinates, or change of basis). By transformation, math texts usually refer to active transformations.
\par}

\en{Tensor}\ru{Тензор}~$\rotationtensor$ \en{relates}\ru{соотносит} \en{the two vectors}\ru{два вектора}\:--- \inquotes{\en{before rotation}\ru{до~поворота}}~${\initiallocationvector = \hspace{-0.1ex} \rho_i \hspace{.1ex} \mathcircabove{\bm{e}}_i}$ \en{and}\ru{и}~\inquotes{\en{after rotation}\ru{после поворота}}~${\locationvector = \hspace{-0.1ex} \rho_i \hspace{.1ex} \bm{e}_i}$.
\en{Components}\ru{Компоненты}~${\rho_i \hspace{-0.2ex} = \hspace{-0.1ex} \constant}$ \en{of}\ru{у}~$\locationvector$ \en{in rotated basis}\ru{в~повёрнутом базисе}~${\bm{e}_i}$\en{ are}\ru{\:---} \en{the same}\ru{такие~же} \en{as}\ru{как} \en{of}\ru{у}~${\initiallocationvector}$ \en{in immobile basis}\ru{в~неподвижном базисе}~${\mathcircabove{\bm{e}}_i}$.
\en{So that}\ru{Так что} \en{the rotation tensor}\ru{тензор поворота} \en{describes}\ru{описывает} \en{the rotation of the vector together with the basis}\ru{вращение вектора вместе с~базисом}.
\en{And since}\ru{И~поскольку} ${\bm{e}_i = \bm{e}_j \mathcircabove{\bm{e}}_j \hspace{-0.15ex} \dotp \mathcircabove{\bm{e}}_i
\:\Leftrightarrow\,
\rho_i \hspace{.1ex} \bm{e}_i \hspace{-0.2ex} = \bm{e}_j \mathcircabove{\bm{e}}_j \hspace{-0.15ex} \dotp \rho_i \hspace{.1ex} \mathcircabove{\bm{e}_i}}$, \en{then}\ru{то}

\nopagebreak\vspace{-0.25em}\begin{equation}\label{rodriguesrotationformula}
\locationvector = \rotationtensor \dotp\hspace{.2ex} \initiallocationvector
\end{equation}

\vspace{-0.4em}\noindent
(\en{this is}\ru{это} \en{the }\href{https://fr.wikipedia.org/wiki/Rotation_vectorielle#Cas_g%C3%A9n%C3%A9ral}{\ru{формула поворота }Rodrigues\ru{’а}\en{ rotation formula}}).

\bibauthor{Olinde Rodrigues}.
\href{http://sites.mathdoc.fr/JMPA/PDF/JMPA_1840_1_5_A39_0.pdf}{Des lois géométriques qui régissent les déplacements d’un système solide dans l’espace, et de la variation des coordonnées provenant de ces déplacements considérés indépendants des causes qui peuvent les produire. \emph{Journal de mathématiques pures et appliquées}, tome~5~(1840), pages~380\hbox{--}440.}

\vspace{-0.1em}
\en{For}\ru{Для} \en{a~second complexity tensor}\ru{тензора второй сложности} ${\mathcircabove{\bm{C}} = C_{i\hspace{-0.1ex}j} \hspace{.1ex} \mathcircabove{\bm{e}}_i \mathcircabove{\bm{e}}_j}$\en{,} \en{a~rotation}\ru{поворот} \en{into the current position}\ru{в~текущую позицию} ${\bm{C} = C_{i\hspace{-0.1ex}j} \hspace{.1ex} \bm{e}_i \bm{e}_j}$ \en{looks like}\ru{выглядит как}

\nopagebreak\vspace{-0.25em}\begin{equation}
\bm{e}_i C_{i\hspace{-0.1ex}j} \bm{e}_j \hspace{-0.2ex}
= \bm{e}_i \mathcircabove{\bm{e}}_i \hspace{-0.1ex} \dotp \mathcircabove{\bm{e}}_p C_{pq} \mathcircabove{\bm{e}}_q \hspace{-0.1ex} \dotp \mathcircabove{\bm{e}}_j \bm{e}_j
\,\;\Leftrightarrow\;\:
\bm{C} = \rotationtensor \dotp\hspace{.1ex} \mathcircabove{\bm{C}} \dotp \rotationtensor^{\T}
\hspace{-0.3ex} .
\end{equation}

\en{Essential property}\ru{Существенное свойство} \en{of a~rotation tensor}\ru{тензора поворота}\:--- \en{orthogonality}\ru{ортогональность}\:--- \en{is expressed as}\ru{выражается как}

\nopagebreak\en{\vspace{-0.8em}}\ru{\vspace{-0.25em}}
\begin{equation}\label{orthogonalityofrotationtensor}
\aunderbrace[l1r]{\rotationtensor}_{\bm{e}_i \mathcircabove{\bm{e}}_i} \hspace{.1em} \dotp \hspace{.1em} \aunderbrace[l1r]{\rotationtensor^{\T}\hspace{-0.2em}}_{\mathcircabove{\bm{e}}_j \bm{e}_j}
\hspace{.2ex} = \hspace{.2ex}
\aunderbrace[l1r]{\rotationtensor^{\T}\hspace{-0.2em}}_{\mathcircabove{\bm{e}}_i \bm{e}_i} \hspace{.2em} \dotp \aunderbrace[l1r]{\rotationtensor}_{\bm{e}_j \mathcircabove{\bm{e}}_j}
\hspace{-0.2ex} = \hspace{.3ex}
\aunderbrace[l1r]{\hspace{.2ex}\UnitDyad\hspace{.2ex}}_{\mathclap{\begin{subarray}{l} \mathcircabove{\bm{e}}_i \mathcircabove{\bm{e}}_i \\ \bm{e}_i \bm{e}_i \end{subarray}}}
\hspace{.2em} ,
\end{equation}

\vspace{-0.1em}\noindent
\en{that~is}\ru{то~есть} \en{the~transposed tensor}\ru{транспонированный тензор} \en{coincides}\ru{совпадает} \en{with}\ru{с}~\en{the~reciprocal tensor}\ru{обратным тензором}:
${\rotationtensor^{\T} \hspace{-0.32ex} = \rotationtensor^{\expminusone} \Leftrightarrow \rotationtensor = \rotationtensor^{\expminusT}\hspace{-0.25ex}}$.

\en{An~orthogonal tensor}\ru{Ортогональный тензор} \en{retains}\ru{сохраняет} \en{lengths and~angles}\ru{дл\'{и}ны и~углы} (\en{the metric}\ru{метрику})\ru{,}
\en{because}\ru{потому что} \en{it}\ru{он} \en{does not change}\ru{не~меняет} \en{the }\hbox{\hspace{-0.2ex}\inquotes{${\dotp\hspace{.22ex}}$}\hspace{-0.2ex}}-\en{product}\ru{произведение} \en{of vectors}\ru{векторов}

\nopagebreak\vspace{-0.1em}\begin{equation}
\label{orthogonaltensorkeepsmetrics}
\left( \rotationtensor \hspace{-0.2ex} \dotp \bm{a}\hspace{.2ex} \right) \dotp \left( \rotationtensor \hspace{-0.2ex} \dotp \bm{b}\hspace{.2ex} \right)
= \bm{a} \dotp \rotationtensor^{\T} \hspace{-0.4ex} \dotp \rotationtensor \hspace{-0.1ex} \dotp \hspace{.1ex} \bm{b}
= \bm{a} \dotp \UnitDyad \dotp \bm{b}
= \bm{a} \dotp \bm{b}
\hspace{.1ex} .
\end{equation}

\en{For all}\ru{Для всех} \en{orthogonal tensors}\ru{ортогональных тензоров} ${\left(\operatorname{det} \orthogonaltensor\hspace{.1ex}\right)^2 \hspace{-0.1ex} = 1}$:

\nopagebreak\vspace{-0.1em}\begin{equation*}
1 = \operatorname{det} \UnitDyad = \operatorname{det} \left({\hspace{-0.1ex} \orthogonaltensor \dotp \orthogonaltensor^{\T} \hspace{.2ex}}\right) \hspace{-0.3ex}
= \left({\operatorname{det} \orthogonaltensor \hspace{.1ex}}\right) \left({\operatorname{det} \orthogonaltensor^{\T} \hspace{.2ex}}\right) \hspace{-0.3ex}
= \left(\operatorname{det} \orthogonaltensor \hspace{.1ex} \right)^{2}
\hspace{.1ex} .
\end{equation*}

\en{A~rotation tensor is an~orthogonal tensor}\ru{Тензор поворота это ортогональный тензор} \en{with}\ru{с}~${\operatorname{det} \rotationtensor \hspace{-0.1ex} = 1}$.

\en{But}\ru{Но} \en{not only}\ru{не~только лишь} \en{rotation tensors}\ru{тензоры поворота} \en{possess}\ru{обладают} \en{the property of~orthogonality}\ru{свойством ортогональности}.
\en{When}\ru{Когда} \en{in}\ru{в}~\eqref{introductionofrotationtensor} \en{the first basis}\ru{первый базис} \en{is }\inquotesx{\en{left}\ru{левый}}[,] \en{and}\ru{а}~\en{the second one}\ru{второй} \en{is }\inquotesx{\en{right}\ru{правый}}[,] \en{then}\ru{тогда} \en{there’s}\ru{это} \en{a~combination of a~rotation and a~reflection}\ru{комбинация поворота и~отражения} (\inquotes{rotoreflection}) ${\rotationtensor = -\UnitDyad \dotp \rotationtensor}$ \en{with}\ru{с}~${\operatorname{det} \left( -\UnitDyad \dotp \rotationtensor \hspace{.15ex} \right) \hspace{-0.1ex} = -1}$.

\begin{otherlanguage}{russian}

У~любого бивалентного тензора в~трёхмерном пространстве как минимум одно собственное число\:--- \en{the root of}\ru{корень}~\eqref{chardetequation}\en{ is}\ru{\:---} \en{non-complex}\ru{некомпл\'{е}ксное}~(\ru{действительное, }real).
\en{For a~rotation tensor}\ru{Для тензора поворота}\en{,} \en{it is equal to one}\ru{оно равно единице}

\nopagebreak\vspace{1em}\begin{equation*}
\begin{array}{c}
\rotationtensor \dotp \bm{a} = \eigenvalue \bm{a}
\:\Rightarrow\:
\tikzmark{BeginPaBrace} \bm{a} \hspace{.16ex} \dotp \tikzmark{BeginEBrace} \rotationtensor^{\T} \tikzmark{EndPaBrace} \hspace{-0.4ex} \dotp \rotationtensor \hspace{-0.32ex}\tikzmark{EndEBrace}\hspace{.32ex} \dotp \hspace{.16ex} \bm{a} = \eigenvalue \bm{a} \dotp \eigenvalue \bm{a}
\:\Rightarrow\: \eigenvalue^{2} \hspace{-0.2ex} = 1 \hspace{.1ex} .
\end{array}
\end{equation*}
\AddOverBrace[line width=.75pt][0.1ex,0.4ex]{BeginPaBrace}{EndPaBrace}{${\scriptstyle \rotationtensor \:\dotp\; \bm{a}}$}
\AddUnderBrace[line width=.75pt][-0.1ex,0.1ex]{BeginEBrace}{EndEBrace}{${\scriptstyle \UnitDyad}$}

\vspace{-0.5em}\noindent
Соответствующая собственная ось называется осью поворота.
Теорема Euler’а о~конечном повороте в~том и~состоит, что такая ось существует.
Если ${\bm{k}}$\:--- орт этой оси, а~${\vartheta}$\:--- конечная величина угла поворота, то тензор поворота представ\'{и}м как

\nopagebreak\vspace{-0.1em}\begin{equation}\label{eulerfiniterotation}
\rotationtensor\hspace{.1ex}(\bm{k}, \vartheta) = \UnitDyad \operatorname{cos} \vartheta + \bm{k} \times\hspace{-0.2ex} \UnitDyad \operatorname{sin} \vartheta + \bm{k} \bm{k} \left( {1 - \operatorname{cos} \vartheta} \right)
\hspace{-0.2ex} .
\end{equation}

\vspace{-0.1em}
Доказывается эта формула так.
Направление~${\bm{k}}$ при~повороте не~меняется~(${\rotationtensor \hspace{-0.2ex}\dotp \bm{k} = \bm{k}\hspace{.12ex}}$), поэтому на~оси поворота ${\mathcircabove{\bm{e}}_3 \hspace{-0.2ex} = \bm{e}_3 \hspace{-0.2ex} = \bm{k}}$.
В~перпендикулярной плоскости~(\figref{fig:eulerfiniterotation}) ${\mathcircabove{\bm{e}}_1 \hspace{-0.2ex} = \bm{e}_1 \operatorname{cos} \vartheta - \bm{e}_2 \operatorname{sin} \vartheta}$, ${\mathcircabove{\bm{e}}_2 \hspace{-0.2ex} = \bm{e}_1 \operatorname{sin} \vartheta + \bm{e}_2 \operatorname{cos} \vartheta}$, ${\rotationtensor = \bm{e}_i \mathcircabove{\bm{e}}_i \,\Rightarrow\hspace{.2ex}}$~\eqref{eulerfiniterotation}.

% ~ ~ ~ ~ ~
\begin{figure}[!htbp]

\vspace*{-0.5em}\[
\mathcircabove{\bm{e}}_i = \mathcircabove{\bm{e}}_i \dotp \bm{e}_j \bm{e}_j
\]

\vspace{-1.5em}\[
\left[ \begin{array}{c} \mathcircabove{\bm{e}}_1 \\ \mathcircabove{\bm{e}}_2 \\ \mathcircabove{\bm{e}}_3 \end{array} \right] =
\left[ \begin{array}{ccc}
\mathcircabove{\bm{e}}_1 \dotp \bm{e}_1 & \mathcircabove{\bm{e}}_1 \dotp \bm{e}_2 & \mathcircabove{\bm{e}}_1 \dotp \bm{e}_3 \\
\mathcircabove{\bm{e}}_2 \dotp \bm{e}_1 & \mathcircabove{\bm{e}}_2 \dotp \bm{e}_2 & \mathcircabove{\bm{e}}_2 \dotp \bm{e}_3 \\
\mathcircabove{\bm{e}}_3 \dotp \bm{e}_1 & \mathcircabove{\bm{e}}_3 \dotp \bm{e}_2 & \mathcircabove{\bm{e}}_3 \dotp \bm{e}_3
\end{array} \right] \hspace{-0.5ex}
\left[ \hspace{-0.12ex} \begin{array}{c} {\bm{e}}_1 \\ {\bm{e}}_2 \\ {\bm{e}_3} \end{array} \right]
\]

\vspace{-1.25em}

\begin{center}
\tdplotsetmaincoords{60}{120} % set orientation of axes
\pgfmathsetmacro{\angletheta}{42}
% three parameters for vector
\pgfmathsetmacro{\lengthofvector}{0.55}
\pgfmathsetmacro{\anglefromz}{40}
\pgfmathsetmacro{\anglefromx}{240}

\begin{tikzpicture}[scale=4, tdplot_main_coords] % tdplot_main_coords style to use 3dplot

	\coordinate (O) at (0,0,0);

	% draw initial axes
	\draw [line width=1.2pt, black, -{Stealth[round, length=4mm, width=2.4mm]}]
		(O) -- (1,0,0)
		node[pos=0.9, above, xshift=-0.8em] {$\mathcircabove{\bm{e}}_1$};

	\draw [line width=1.2pt, black, -{Stealth[round, length=4mm, width=2.4mm]}]
		(O) -- (0,1,0)
		node[pos=0.9, above, xshift=1em, yshift=-0.2em] {$\mathcircabove{\bm{e}}_2$};

	\draw [line width=1.2pt, red, -{Stealth[round,length=4mm,width=2.4mm]}]
		(O) -- (0,0,0.9)
		node[anchor=south] {$\mathcircabove{\bm{e}}_3 = \bm{e}_3 = \bm{k}$};

	% draw initial vector
	\tdplotsetcoord{point}{\lengthofvector}{\anglefromz}{\anglefromx} % {length}{angle from z}{angle from x}
		% it also defines (pointxy), (pointxz), and (pointyz) projections of point
	\draw [line width=1.2pt, black, -{Stealth[round, length=4mm, width=2.4mm]}]
		(O) -- (point)
		node[anchor=south] {$\mathcircabove{\bm{r}}$};
	% draw its projection on xy plane
	\draw [line width=0.4pt, dotted, color=black] (O) -- (pointxy);
	\draw [line width=0.4pt, dotted, color=black] (pointxy) -- (point);

	% draw the angle, and label it
	% syntax: \tdplotdrawarc[coordinate frame, draw options]{center point}{r}{angle}{end angle}{label options}{label}
	\tdplotdrawarc [line width=0.5pt, red, ->]
		{(O)}{0.4}{0}{\angletheta}{anchor=north}{$\vartheta$}
	\tdplotdrawarc [line width=0.5pt, red, ->]
		{(O)}{0.4}{90}{90+\angletheta}{anchor=west}{$\vartheta$}

	% rotate coordinates using Euler angles "z(\alpha)y(\beta)z(\gamma)"
	\tdplotsetrotatedcoords{\angletheta}{0}{0}

	% draw rotated axes
	\draw [line width=1.2pt, blue, tdplot_rotated_coords, -{Stealth[round, length=4mm, width=2.4mm]}]
		(O) -- (1,0,0)
		node[pos=0.9, left, xshift=-0.1em] {$\bm{e}_1$};

	\draw [line width=1.2pt, blue, tdplot_rotated_coords, -{Stealth[round, length=4mm, width=2.4mm]}]
		(O) -- (0,1,0)
		node[pos=0.9, above, xshift=0.2em, yshift=0.2em] {$\bm{e}_2$};

	%%\draw [line width=1.2pt, blue, tdplot_rotated_coords, -{Stealth[round, length=4mm, width=2.4mm]}]
		%%(O) -- (0,0,0.8) ;

	% draw rotated vector
	\tdplotsetcoord{rotatedpoint}%
		{\lengthofvector}{\anglefromz}{\anglefromx+\angletheta}
	\draw [line width=1.2pt, blue, tdplot_rotated_coords, -{Stealth[round, length=4mm, width=2.4mm]}]
		(O) -- (rotatedpoint)
		node[anchor=south] {$\bm{r}$};
	% draw its projection on xy plane
	\draw [line width=0.4pt, dotted, color=blue, tdplot_rotated_coords] (O) -- (rotatedpointxy);
	\draw [line width=0.4pt, dotted, color=blue, tdplot_rotated_coords] (rotatedpointxy) -- (rotatedpoint);

	\tdplotdrawarc [line width=0.5pt, red, ->]
		{(O)}{0.28}{\anglefromx}{\anglefromx+\angletheta}{anchor=south east, xshift=0.3em, yshift=-0.1em}{$\vartheta$}

\end{tikzpicture}
\end{center}

\vspace{-1em}\[
\scalebox{0.8}[0.85]{$\left[ \begin{array}{ccc}
\mathcircabove{\bm{e}}_1 \dotp \bm{e}_1 & \mathcircabove{\bm{e}}_1 \dotp \bm{e}_2 & \mathcircabove{\bm{e}}_1 \dotp \bm{e}_3 \\
\mathcircabove{\bm{e}}_2 \dotp \bm{e}_1 & \mathcircabove{\bm{e}}_2 \dotp \bm{e}_2 & \mathcircabove{\bm{e}}_2 \dotp \bm{e}_3 \\
\mathcircabove{\bm{e}}_3 \dotp \bm{e}_1 & \mathcircabove{\bm{e}}_3 \dotp \bm{e}_2 & \mathcircabove{\bm{e}}_3 \dotp \bm{e}_3
\end{array} \right]$} \hspace{-0.32ex} = \hspace{-0.2ex}
%
\scalebox{0.8}[0.85]{$\left[ \hspace{-0.2ex} \begin{array}{ccc}
\operatorname{cos} \vartheta & \hspace{-1ex} \operatorname{cos} \left( 90\degree \!+ \vartheta \right) & \operatorname{cos} 90\degree \\
\operatorname{cos} \left( 90\degree \!- \vartheta \right) & \operatorname{cos} \vartheta & \operatorname{cos} 90\degree \\
\operatorname{cos} 90\degree & \operatorname{cos} 90\degree & \operatorname{cos} 0\degree
\end{array} \right]$} \hspace{-0.32ex} = \hspace{-0.2ex}
%
\scalebox{0.8}[0.85]{$\left[ \hspace{-0.1ex} \begin{array}{ccc}
\operatorname{cos} \vartheta & - \operatorname{sin} \vartheta & 0 \\
\operatorname{sin} \vartheta & \operatorname{cos} \vartheta & 0 \\
0 & 0 & 1
\end{array} \right]$}
\]

\vspace{-0.8em}
\[\begin{array}{c}
\mathcircabove{\bm{e}}_1 \hspace{-0.16ex} = \bm{e}_1 \operatorname{cos} \vartheta \hspace{0.1ex} - \hspace{0.1ex} \bm{e}_2 \operatorname{sin} \vartheta \\[0.1em]
\mathcircabove{\bm{e}}_2 \hspace{-0.16ex} = \bm{e}_1 \operatorname{sin} \vartheta \hspace{0.1ex} + \hspace{0.1ex} \bm{e}_2 \operatorname{cos} \vartheta \\[0.1em]
\mathcircabove{\bm{e}}_3 \hspace{-0.16ex} = \bm{e}_3 = \bm{k}
\end{array}\]

\vspace{-1em}
\begin{multline*}
\shoveleft{ \bm{P} = \bm{e}_1 \hspace{-0.1ex} \mathcircabove{\bm{e}}_1 + \bm{e}_2 \mathcircabove{\bm{e}}_2 + \bm{e}_3 \mathcircabove{\bm{e}}_3 = \hfill }\\[1.5em]
%
= \hspace{0.2ex} \tikzmark{StartBraceE1E1} {\bm{e}_1 \bm{e}_1 \operatorname{cos} \vartheta - \bm{e}_1 \bm{e}_2 \operatorname{sin} \vartheta \hspace{0.2em}} \tikzmark{EndBraceE1E1} \hspace{-0.1ex} + \hspace{0.1ex} \tikzmark{StartBraceE2E2} {\bm{e}_2 \bm{e}_1 \operatorname{sin} \vartheta + \bm{e}_2 \bm{e}_2 \operatorname{cos} \vartheta \hspace{0.2em}} \tikzmark{EndBraceE2E2} \hspace{-0.1ex} + \tikzmark{StartBraceE3E3} {\hspace{0.25ex} \bm{k} \bm{k} \hspace{0.1ex}} \tikzmark{EndBraceE3E3} \hspace{0.1ex} =\\[0.32em]
%
= \hspace{0.1ex} \bm{E} \operatorname{cos} \vartheta - \hspace{-0.1ex} \tikzmark{StartBraceKk} {\hspace{0.1ex}\bm{e}_3 \bm{e}_3\hspace{0.1ex}} \tikzmark{EndBraceKk} \hspace{-0.25ex} \operatorname{cos} \vartheta \hspace{0.1ex} + \tikzmark{StartBraceLeviCivita} {\left( \bm{e}_2 \bm{e}_1 - \bm{e}_1 \bm{e}_2 \right)} \tikzmark{EndBraceLeviCivita} \operatorname{sin} \vartheta + \bm{k} \bm{k} \hspace{0.1ex} =\\[1.5em]
%
\shoveright{ \hfill = \bm{E} \operatorname{cos} \vartheta + \bm{k} \times\hspace{-0.2ex} \bm{E} \operatorname{sin} \vartheta + \bm{k} \bm{k} \left({1 - \operatorname{cos} \vartheta}\right) }
\end{multline*}

\AddOverBrace[line width=0.75pt]{StartBraceE1E1}{EndBraceE1E1}{${\scriptstyle \bm{e}_1 \mathcircabove{\bm{e}}_1}$}
\AddOverBrace[line width=0.75pt]{StartBraceE2E2}{EndBraceE2E2}{${\scriptstyle \bm{e}_2 \mathcircabove{\bm{e}}_2}$}
\AddOverBrace[line width=0.75pt]{StartBraceE3E3}{EndBraceE3E3}{${\scriptstyle \bm{e}_3 \mathcircabove{\bm{e}}_3}$}
\AddUnderBrace[line width=0.75pt][-0.1ex,-0.2ex]{StartBraceKk}{EndBraceKk}{${\scriptstyle \bm{k}\bm{k}}$}
\AddUnderBrace[line width=0.75pt][-0.1ex,-0.2ex][xshift=0.4ex]{StartBraceLeviCivita}{EndBraceLeviCivita}{${\scriptstyle \bm{e}_3 \times \bm{e}_i \bm{e}_i \:=\: \levicivita_{3ij} \bm{e}_j \bm{e}_i}$}

\vspace{-0.5em}
\caption{\inquotes{\en{Finite rotation}\ru{Конечный поворот}}}\label{fig:eulerfiniterotation}
\end{figure}

% ~ ~ ~ ~ ~

Из~\eqref{eulerfiniterotation} и~\eqref{rodriguesrotationformula} получаем формулу поворота Родрига в~параметрах~$\bm{k}$ и~$\vartheta$:

\nopagebreak\vspace{-0.3em}\begin{equation*}
\locationvector \hspace{.3ex}
= \hspace{.4ex} \initiallocationvector \operatorname{cos} \vartheta \hspace{.3ex} + \hspace{.3ex} \bm{k} \times \initiallocationvector \hspace{.2ex} \operatorname{sin} \vartheta \hspace{.3ex} + \hspace{.4ex} \bm{k} \bm{k} \dotp \hspace{.1ex} \initiallocationvector \left({1 - \operatorname{cos} \vartheta}\right)
\hspace{-0.25ex} .
\end{equation*}

\vspace{-0.2em}
В~параметрах конечного поворота транспонирование, оно~же обращение, тензора~$\rotationtensor$ эквивалентно перемене направления поворота\:--- знака угла~$\vartheta$
\[
\rotationtensor^{\T} \hspace{-0.1ex}=\hspace{.1ex} \rotationtensor \hspace{.1ex} \bigr|_{\vartheta \,=\hspace{.1ex} -\vartheta} \hspace{-0.1ex} = \UnitDyad \operatorname{cos} \vartheta - \bm{k} \times\hspace{-0.2ex} \UnitDyad \operatorname{sin} \vartheta + \bm{k} \bm{k} \left({1 - \operatorname{cos} \vartheta}\right)
\hspace{-0.3ex} .
\]

Пусть теперь тензор поворота меняется со~временем:
${\rotationtensor \narroweq \rotationtensor(t)}$.
Псевдовектор угловой скорости~${\bm{\omega}}$ вводится через тензор поворота~$\rotationtensor$ таким путём.
Дифференцируем тождество ортогональности~\eqref{orthogonalityofrotationtensor} по~времени%
\footnote{\en{Various notations are used}\ru{Используются различные записи} \en{to designate the~time derivative}\ru{для обозначения производной по времени}.
\en{In~addition}\ru{Вдобавок} \en{to}\ru{к~записи} \en{the }Leibniz’\en{s}\ru{а}\en{ notation} ${\scalebox{.9}{$ \raisemath{.3em}{dx} $} \hspace{-0.3ex} / \hspace{-0.4ex} \scalebox{.9}{$ \raisemath{-.3em}{dt} $}\hspace{.1ex}}$, \en{the~very popular one is}\ru{очень популярна запись} \en{the~}\inquotes{\en{dot-above}\ru{точка-сверху}} Newton’\en{s}\ru{а}\en{ notation}~${\mathdotabove{x}}$.}

\nopagebreak\vspace{-0.1em}\begin{equation*}
\mathdotabove{\rotationtensor} \dotp \rotationtensor^{\T} \hspace{-0.1ex} + \hspace{.25ex} \rotationtensor \dotp \mathdotabove{\rotationtensor}^{\T} \hspace{-0.1ex} = \hspace{.1ex} {^2\bm{0}}
\hspace{.1ex} .
\end{equation*}

Тензор ${\mathdotabove{\rotationtensor} \dotp \rotationtensor^{\T\!}}$ (по~\eqref{transposeofdotproduct} ${\left({ \mathdotabove{\rotationtensor} \dotp \rotationtensor^{\T} }\right)^{\raisemath{-0.25em}{\!\T}} \hspace{-0.4ex} = \rotationtensor \dotp \mathdotabove{\rotationtensor}^{\T}}$) оказался анти\-сим\-метрич\-ным.
Поэтому согласно~\eqref{companionvector} он представ\'{и}м сопутствующим вектором как ${\mathdotabove{\rotationtensor} \dotp \rotationtensor^{\T\!} = \bm{\omega} \times \UnitDyad = \bm{\omega} \times \rotationtensor \dotp \rotationtensor^{\T}}$\!.
То~есть

\nopagebreak\vspace{-0.1em}\begin{equation}\label{angularvelocityvector}
\mathdotabove{\rotationtensor} = \bm{\omega} \times \rotationtensor, \;\:\:
\bm{\omega} \equiv -\, \displaystyle \onehalf \left( \mathdotabove{\rotationtensor} \dotp \rotationtensor^{\T} \right)_{\hspace{-0.2em}\Xcompanion}
\vspace{-0.25em}\end{equation}

Помимо этого общего представления~вектора~${\bm{\omega}}$, для~него есть и~другие. Например, через параметры конечного поворота.

Производная~${\mathdotabove{\rotationtensor}}$ в~параметрах конечного поворота в~общем случае (оба параметра\:--- и~единичный вектор~$\bm{k}$, и~угол~$\vartheta$\:--- переменны во~времени):
\vspace{.3em}%
\[
\begin{array}{r@{\hspace{.33em}}c@{\hspace{.25em}}l}
\mathdotabove{\rotationtensor} \hspace{.2em} & = & \hspace{.1em} \left(\rotationtensor^{\mathsf{\hspace{.12ex}S}} \hspace{-0.2ex} +^{\mathstrut} \rotationtensor^{\mathsf{\hspace{.12ex}A}}\right)^{\hspace{-0.2ex}\tikz[baseline=-0.5ex]\draw[black, fill=black] (0,0) circle (.266ex);} =
\hspace{.1em} \left(\hspace{.2ex} \tikzmark{StartBracePs} {\UnitDyad \operatorname{cos} \vartheta + \bm{k} \bm{k} \left({1 \!-\! \operatorname{cos} \vartheta}\right)} \hspace{-0.2ex} \tikzmark{EndBracePs} \hspace{.32ex} +^{\mathstrut} \hspace{.2ex}
\tikzmark{StartBracePa} {\bm{k} \hspace{-0.24ex}\times\hspace{-0.4ex} \UnitDyad \operatorname{sin} \vartheta \hspace{.2ex}} \tikzmark{EndBracePa} {} \hspace{.16ex}\right)^{\hspace{-0.2ex}\tikz[baseline=-0.5ex]\draw[black, fill=black] (0,0) circle (.266ex);} \hspace{-0.05em} =
\\[0.4em]
%
& = & \hspace{.2em} \tikzmark{StartBraceDotPs} {\left( \hspace{.1ex} \bm{k} \bm{k} \hspace{-0.1ex} - \hspace{-0.2ex} \UnitDyad \hspace{.1ex} \right) \hspace{-0.1ex} \mathdotabove{\vartheta} \operatorname{sin} \vartheta + \hspace{-0.2ex} ( \bm{k} \mathdotabove{\bm{k}} + \mathdotabove{\bm{k}} \bm{k} ) \hspace{-0.2ex} \left({1 \!-\! \operatorname{cos} \vartheta}\right)} \tikzmark{EndBraceDotPs} \hspace{.64ex} + \\[0.64em]
& & \hspace{13.2em} + \hspace{.72ex} \tikzmark{StartBraceDotPa} {\hspace{.12ex} \bm{k} \hspace{-0.24ex}\times\hspace{-0.4ex} \UnitDyad \hspace{.4ex} \mathdotabove{\vartheta} \operatorname{cos} \vartheta + \mathdotabove{\bm{k}} \hspace{-0.24ex}\times\hspace{-0.4ex} \UnitDyad \operatorname{sin} \vartheta} \tikzmark{EndBraceDotPa}
\hspace{.1em} .
\end{array}\]%
\vspace{-1.2em}

\AddOverBrace[line width=0.75pt][0.12ex,0]{StartBracePs}{EndBracePs}{${\scriptstyle \rotationtensor^{\mathsf{\hspace{.12ex}S}}}$}
\AddOverBrace[line width=0.75pt][-0.12ex,0]{StartBracePa}{EndBracePa}{${\scriptstyle \rotationtensor^{\mathsf{\hspace{.12ex}A}}}$}
\AddUnderBrace[line width=0.75pt]{StartBraceDotPs}{EndBraceDotPs}{${\scriptstyle \mathdotabove{\rotationtensor}^{\mathsf{\hspace{.12ex}S}}}$}
\AddUnderBrace[line width=0.75pt][-0.1ex,0.1ex]{StartBraceDotPa}{EndBraceDotPa}{${\scriptstyle \mathdotabove{\rotationtensor}^{\mathsf{\hspace{.12ex}A}}}$}

\vspace{-1.32em} \noindent Находим
\vspace{.2em}\[\begin{array}{r@{\hspace{.25em}}c@{\hspace{.4em}}l}
\mathdotabove{\rotationtensor} \dotp \rotationtensor^{\T} & = & (\hspace{.1em} \mathdotabove{\rotationtensor}^{\mathsf{\hspace{.12ex}S}} \hspace{-0.16ex} + \mathdotabove{\rotationtensor}^{\mathsf{\hspace{.12ex}A}} \hspace{.05em}) \hspace{-0.1ex} \dotp \hspace{-0.1ex} (\hspace{.1em} \rotationtensor^{\mathsf{\hspace{.12ex}S}} \hspace{-0.16ex} - \rotationtensor^{\mathsf{\hspace{.12ex}A}} \hspace{.1ex} \hspace{.05em}) =
\\[0.25em]
& = & \mathdotabove{\rotationtensor}^{\mathsf{\hspace{.12ex}S}} \hspace{-0.2ex}\dotp \rotationtensor^{\mathsf{\hspace{.12ex}S}}
+ \hspace{.2ex} \mathdotabove{\rotationtensor}^{\mathsf{\hspace{.12ex}A}} \hspace{-0.2ex}\dotp \rotationtensor^{\mathsf{\hspace{.12ex}S}}
- \hspace{.2ex} \mathdotabove{\rotationtensor}^{\mathsf{\hspace{.12ex}S}} \hspace{-0.2ex}\dotp \rotationtensor^{\mathsf{\hspace{.12ex}A}}
- \hspace{.2ex} \mathdotabove{\rotationtensor}^{\mathsf{\hspace{.12ex}A}} \hspace{-0.2ex}\dotp \rotationtensor^{\mathsf{\hspace{.12ex}A}} ,
\end{array}\]

\vspace{-0.5em}\noindent
\en{using}\ru{используя}
\[\scalebox{0.95}[0.96]{$\begin{array}{c}
\bm{k} \dotp \bm{k} = 1 = \constant \,\Rightarrow\:
\bm{k} \dotp \mathdotabove{\bm{k}} + \mathdotabove{\bm{k}} \dotp \bm{k} = 0 \;\Leftrightarrow\; \mathdotabove{\bm{k}} \dotp \bm{k} = \bm{k} \dotp \mathdotabove{\bm{k}} = 0 \hspace{.1ex} ,
\\[.08em]
%
\bm{k} \bm{k} \hspace{-0.2ex}\dotp\hspace{-0.2ex} \bm{k} \bm{k} = \bm{k} \bm{k}, \:\:
\mathdotabove{\bm{k}} \bm{k} \hspace{-0.2ex}\dotp\hspace{-0.2ex} \bm{k} \bm{k} = \mathdotabove{\bm{k}} \bm{k} , \:\:
\bm{k} \mathdotabove{\bm{k}} \hspace{-0.2ex}\dotp\hspace{-0.2ex} \bm{k} \bm{k} = {\hspace{-0.2ex}^2\bm{0}} \hspace{.1ex},
\\[.16em]
%
\left( \hspace{.1ex} \bm{k} \bm{k} \hspace{-0.1ex} - \hspace{-0.2ex} \UnitDyad \hspace{.1ex} \right) \hspace{-0.2ex} \dotp \bm{k} = \bm{k} - \bm{k} = {\bm{0}} \hspace{.1ex}, \,\,
\left( \hspace{.1ex} \bm{k} \bm{k} \hspace{-0.1ex} - \hspace{-0.2ex} \UnitDyad \hspace{.1ex} \right) \hspace{-0.2ex} \dotp \bm{k} \bm{k} = \bm{k} \bm{k} - \bm{k} \bm{k} = {\hspace{-0.2ex}^2\bm{0}} \hspace{.1ex} ,
\\[.08em]
%
\bm{k} \dotp \hspace{-0.1ex} ( \bm{k} \hspace{-0.24ex}\times\hspace{-0.4ex} \UnitDyad ) \hspace{-0.1ex}
= \hspace{-0.1ex} ( \bm{k} \hspace{-0.24ex}\times\hspace{-0.4ex} \UnitDyad ) \hspace{-0.2ex} \dotp \bm{k}
= \bm{k} \hspace{-0.24ex}\times\hspace{-0.2ex} \bm{k} = \bm{0} \hspace{.1ex} , \,\,
\bm{k} \bm{k} \dotp \hspace{-0.1ex} ( \bm{k} \hspace{-0.24ex}\times\hspace{-0.4ex} \UnitDyad ) \hspace{-0.1ex}
= \hspace{-0.1ex} ( \bm{k} \hspace{-0.24ex}\times\hspace{-0.4ex} \UnitDyad ) \hspace{-0.2ex} \dotp \bm{k} \bm{k}
= {\hspace{-0.2ex}^2\bm{0}} \hspace{0.1ex},
\\[.08em]
%
\left( \hspace{0.1ex} \bm{k} \bm{k} \hspace{-0.1ex} - \hspace{-0.2ex} \UnitDyad \hspace{0.1ex} \right) \hspace{-0.2ex} \dotp \hspace{-0.1ex} ( \bm{k} \hspace{-0.24ex}\times\hspace{-0.4ex} \UnitDyad ) \hspace{-0.1ex} = \hspace{-0.1ex}
- \hspace{0.2ex} \bm{k} \hspace{-0.24ex}\times\hspace{-0.4ex} \UnitDyad \hspace{0.1ex},
\\
%
( \bm{a} \hspace{-0.24ex}\times\hspace{-0.4ex} \UnitDyad ) \hspace{-0.2ex} \dotp \bm{b} =
\bm{a} \hspace{-0.2ex}\times\hspace{-0.36ex} ( \UnitDyad \hspace{-0.1ex} \dotp \bm{b} ) \hspace{-0.16ex} =
\bm{a} \hspace{-0.16ex}\times\hspace{-0.12ex} \bm{b} \:\,\Rightarrow\,
( \mathdotabove{\bm{k}} \hspace{-0.24ex}\times\hspace{-0.4ex} \UnitDyad ) \hspace{-0.2ex} \dotp \bm{k} \bm{k} =
\mathdotabove{\bm{k}} \hspace{-0.16ex}\times\hspace{-0.2ex} \bm{k} \bm{k} \hspace{0.1ex},
\end{array}$}\]

\vspace{-0.4em} \noindent \eqref{vectorcrossidentitydotvectorcrossidentity} $\,\Rightarrow\,$
\scalebox{0.95}[0.96]{${( \bm{k} \hspace{-0.24ex}\times\hspace{-0.4ex} \UnitDyad ) \hspace{-0.2ex} \dotp \hspace{-0.2ex} ( \bm{k} \hspace{-0.24ex}\times\hspace{-0.4ex} \UnitDyad ) \hspace{-0.16ex} = \bm{k} \bm{k} \hspace{-0.1ex} - \hspace{-0.2ex} \UnitDyad}$},\hspace{0.4ex}
%
\scalebox{0.95}[0.96]{${\displaystyle ( \mathdotabove{\bm{k}} \hspace{-0.24ex}\times\hspace{-0.4ex} \UnitDyad ) \hspace{-0.2ex} \dotp \hspace{-0.2ex} ( \bm{k} \hspace{-0.24ex}\times\hspace{-0.4ex} \UnitDyad ) \hspace{-0.16ex} = \bm{k} \mathdotabove{\bm{k}} \hspace{-0.1ex} - \tikzbackcancel[black!25]{$\mathdotabove{\bm{k}} \hspace{-0.2ex}\dotp\hspace{-0.2ex} \bm{k} \hspace{.16ex} \UnitDyad$}}$\hspace{.16ex}},

\noindent \eqref{vectorcrossvectorcrossidentity} $\,\Rightarrow\,$
\scalebox{0.95}[0.96]{$\mathdotabove{\bm{k}} \bm{k} \hspace{-0.1ex} - \hspace{-0.1ex} \bm{k} \mathdotabove{\bm{k}} = \hspace{-0.16ex} ( \bm{k} \hspace{-0.2ex} \times \hspace{-0.24ex} \mathdotabove{\bm{k}} ) \hspace{-0.32ex} \times \hspace{-0.32ex} \UnitDyad$},\hspace{.4ex}
%
\scalebox{0.95}[0.96]{${\displaystyle ( \mathdotabove{\bm{k}} \hspace{-0.2ex}\times\hspace{-0.24ex} \bm{k} ) \hspace{.2ex} \bm{k} \hspace{-0.1ex} - \hspace{-0.1ex} \bm{k} \hspace{.16ex} ( \mathdotabove{\bm{k}} \hspace{-0.2ex}\times\hspace{-0.24ex} \bm{k} ) \hspace{-0.16ex} = \bm{k} \hspace{-0.2ex} \times \hspace{-0.32ex} ( \mathdotabove{\bm{k}} \hspace{-0.2ex}\times\hspace{-0.24ex} \bm{k} ) \hspace{-0.32ex} \times \hspace{-0.32ex} \UnitDyad}$\hspace{.16ex}}

\begin{fleqn}[0pt]
\begin{multline*}
\shoveleft{\scalebox{0.94}[0.96]{$\mathdotabove{\bm{P}}^{\mathsf{\hspace{0.12ex}S}} \hspace{-0.2ex}\dotp \bm{P}^{\mathsf{\hspace{.12ex}S}} \hspace{-0.25ex} = $} \hspace{2em} \hfill}
\\[-0.25em]
%
\shoveleft{\scalebox{0.8}[0.82]{$= \hspace{.2ex} \left( \hspace{.1ex} \bm{k} \bm{k} \hspace{-0.1ex} - \hspace{-0.2ex} \UnitDyad \hspace{.1ex} \right) \hspace{-0.1ex} \mathdotabove{\vartheta} \operatorname{sin} \vartheta \dotp \UnitDyad \operatorname{cos} \vartheta +
( \bm{k} \mathdotabove{\bm{k}} + \mathdotabove{\bm{k}} \bm{k} ) \hspace{-0.2ex} \left({1 \!-\! \operatorname{cos} \vartheta}\right) \dotp \UnitDyad \operatorname{cos} \vartheta \hspace{.32em} +$} \hfill}
\\[-0.2em]
\shoveright{\hfill \scalebox{0.8}[0.82]{$+\; \tikzbackcancel[black!25]{$\left( \hspace{.1ex} \bm{k} \bm{k} \hspace{-0.1ex} - \hspace{-0.2ex} \UnitDyad \hspace{.1ex} \right) \hspace{-0.1ex} \mathdotabove{\vartheta} \operatorname{sin} \vartheta \dotp \bm{k} \bm{k} \left({1 \!-\! \operatorname{cos} \vartheta}\right)$} \hspace{.2ex} +
( \bm{k} \mathdotabove{\bm{k}} + \mathdotabove{\bm{k}} \bm{k} ) \hspace{-0.2ex} \left({1 \!-\! \operatorname{cos} \vartheta}\right) \hspace{-0.2ex} \dotp \bm{k} \bm{k} \left({1 \!-\! \operatorname{cos} \vartheta}\right) =$}}
\\
%
\scalebox{0.8}[0.82]{$= \left( \hspace{.1ex} \bm{k} \bm{k} \hspace{-0.1ex} - \hspace{-0.2ex} \UnitDyad \hspace{.1ex} \right) \hspace{-0.1ex} \mathdotabove{\vartheta} \operatorname{sin} \vartheta \operatorname{cos} \vartheta
+ ( \bm{k} \mathdotabove{\bm{k}} + \mathdotabove{\bm{k}} \bm{k} ) \hspace{-0.1ex} \operatorname{cos} \vartheta \left({1 \!-\! \operatorname{cos} \vartheta}\right) + ( \tikzbackcancel[black!25]{$\bm{k} \mathdotabove{\bm{k}} \hspace{-0.1ex}\dotp\hspace{-0.1ex} \bm{k} \bm{k}$} + \mathdotabove{\bm{k}} \bm{k} \hspace{-0.1ex}\dotp\hspace{-0.1ex} \bm{k} \bm{k} ) \left({1 \!-\! \operatorname{cos} \vartheta}\right)^{\hspace{-0.12ex}2} \hspace{-0.25ex} =$}
\\
%
\shoveleft{\scalebox{0.8}[0.82]{$= \left( \hspace{.1ex} \bm{k} \bm{k} \hspace{-0.1ex} - \hspace{-0.2ex} \UnitDyad \hspace{.1ex} \right) \hspace{-0.1ex} \mathdotabove{\vartheta} \operatorname{sin} \vartheta \operatorname{cos} \vartheta + \bm{k} \mathdotabove{\bm{k}} \operatorname{cos} \vartheta \left({1 \!-\! \operatorname{cos} \vartheta}\right) +$} \hfill}
\\[-0.2em]
\shoveright{\hfill \scalebox{0.8}[0.82]{$+ \hspace{.24em} \mathdotabove{\bm{k}} \bm{k} \operatorname{cos} \vartheta - \mathdotabove{\bm{k}} \bm{k} \operatorname{cos}^{2\hspace{-0.4ex}} \vartheta + \mathdotabove{\bm{k}} \bm{k} - 2 \, \mathdotabove{\bm{k}} \bm{k} \operatorname{cos} \vartheta + \mathdotabove{\bm{k}} \bm{k} \operatorname{cos}^{2\hspace{-0.4ex}} \vartheta =$}}\\
%
%% \shoveright{\hfill \scalebox{0.8}[0.82]{$= \left( \hspace{.1ex} \bm{k} \bm{k} \hspace{-0.1ex} - \hspace{-0.2ex} \UnitDyad \hspace{.1ex} \right) \hspace{-0.1ex} \mathdotabove{\vartheta} \operatorname{sin} \vartheta \operatorname{cos} \vartheta \hspace{.1ex}
%% + \hspace{.1ex} \bm{k} \mathdotabove{\bm{k}} \operatorname{cos} \vartheta \left({1 \!-\! \operatorname{cos} \vartheta}\right)
%% + \mathdotabove{\bm{k}} \bm{k}
%% - \mathdotabove{\bm{k}} \bm{k} \operatorname{cos} \vartheta =$}}\\
%
\shoveright{\hfill \hspace{4.8em} \scalebox{0.94}[0.96]{$= \left( \hspace{.1ex} \bm{k} \bm{k} \hspace{-0.1ex} - \hspace{-0.2ex} \UnitDyad \hspace{.1ex} \right) \hspace{-0.1ex} \mathdotabove{\vartheta} \operatorname{sin} \vartheta \operatorname{cos} \vartheta \hspace{.1ex}
%% + \hspace{-0.1ex} ( \bm{k} \mathdotabove{\bm{k}} \operatorname{cos} \vartheta + \mathdotabove{\bm{k}} \bm{k} ) \hspace{-0.2ex} \left({1 \!-\! \operatorname{cos} \vartheta}\right)
+ \bm{k} \mathdotabove{\bm{k}} \operatorname{cos} \vartheta
- \bm{k} \mathdotabove{\bm{k}} \operatorname{cos}^{2\hspace{-0.4ex}} \vartheta
+ \mathdotabove{\bm{k}} \bm{k} \left({1 \!-\! \operatorname{cos} \vartheta}\right) \hspace{-0.16ex},$}}
\end{multline*}
\begin{multline*}
\shoveleft{\scalebox{0.94}[0.96]{$\mathdotabove{\bm{P}}^{\mathsf{\hspace{.12ex}A}} \hspace{-0.2ex}\dotp \bm{P}^{\mathsf{\hspace{.12ex}S}} \hspace{-0.25ex} = $} \hfill} \\[-0.25em]
%
\shoveleft{\scalebox{0.8}[0.82]{$= ( \hspace{.12ex} \bm{k} \hspace{-0.24ex}\times\hspace{-0.4ex} \UnitDyad \hspace{.1ex} ) \hspace{-0.2ex} \dotp \hspace{-0.2ex} \UnitDyad \hspace{.4ex} \mathdotabove{\vartheta} \operatorname{cos}^{2\hspace{-0.4ex}} \vartheta +
( \hspace{.12ex} \mathdotabove{\bm{k}} \hspace{-0.24ex}\times\hspace{-0.4ex} \UnitDyad \hspace{.1ex} ) \hspace{-0.2ex} \dotp \hspace{-0.2ex} \UnitDyad \hspace{.1ex} \operatorname{sin} \vartheta \operatorname{cos} \vartheta \:+$} \hfill} \\[-0.2em]
\shoveright{\hfill \scalebox{0.8}[0.82]{$+\; \tikzbackcancel[black!25]{$ ( \hspace{.12ex} \bm{k} \hspace{-0.24ex}\times\hspace{-0.4ex} \UnitDyad \hspace{.1ex} ) \hspace{-0.2ex} \dotp \hspace{-0.1ex} \bm{k} \bm{k} \hspace{.5ex} \mathdotabove{\vartheta} \operatorname{cos} \vartheta \left({1 \!-\! \operatorname{cos} \vartheta}\right) $} \hspace{.2ex} +
( \hspace{.12ex} \mathdotabove{\bm{k}} \hspace{-0.24ex}\times\hspace{-0.4ex} \UnitDyad \hspace{.1ex} ) \hspace{-0.25ex} \dotp \hspace{-0.1ex} \bm{k} \bm{k} \hspace{.2ex} \operatorname{sin} \vartheta \left({1 \!-\! \operatorname{cos} \vartheta}\right) =$}} \\
%
\hspace{3.85em} \scalebox{0.94}[0.96]{$= \hspace{.2ex} \bm{k} \hspace{-0.24ex}\times\hspace{-0.4ex} \UnitDyad \hspace{.4ex} \mathdotabove{\vartheta} \operatorname{cos}^{2\hspace{-0.4ex}} \vartheta +
\mathdotabove{\bm{k}} \hspace{-0.24ex}\times\hspace{-0.4ex} \UnitDyad \hspace{.1ex} \operatorname{sin} \vartheta \operatorname{cos} \vartheta +
%%\hspace{-0.12ex} ( \hspace{.12ex} \mathdotabove{\bm{k}} \hspace{-0.24ex}\times\hspace{-0.4ex} \UnitDyad \hspace{.1ex} ) \hspace{-0.25ex} \dotp \hspace{-0.1ex} \bm{k} \bm{k} \hspace{.2ex}
\mathdotabove{\bm{k}} \hspace{-0.24ex}\times\hspace{-0.32ex} \bm{k} \bm{k} \hspace{.2ex}
\operatorname{sin} \vartheta \left({1 \!-\! \operatorname{cos} \vartheta}\right) \hspace{-0.16ex},$}
\end{multline*}
\begin{multline*}
\shoveleft{\scalebox{0.94}[0.96]{$\mathdotabove{\bm{P}}^{\mathsf{\hspace{.12ex}S}} \hspace{-0.2ex}\dotp \bm{P}^{\mathsf{\hspace{.12ex}A}} \hspace{-0.25ex} = $} \hfill} \\[-0.25em]
%
\scalebox{0.8}[0.82]{$= ( \hspace{.1ex} \bm{k} \bm{k} \hspace{-0.1ex} - \hspace{-0.2ex} \UnitDyad \hspace{.1ex} ) \hspace{.25ex} \mathdotabove{\vartheta} \operatorname{sin} \vartheta \dotp ( \bm{k} \hspace{-0.24ex}\times\hspace{-0.4ex} \UnitDyad ) \operatorname{sin} \vartheta +
( \bm{k} \mathdotabove{\bm{k}} + \mathdotabove{\bm{k}} \bm{k} ) \hspace{-0.2ex} \left({1 \!-\! \operatorname{cos} \vartheta}\right) \hspace{-0.1ex} \dotp ( \bm{k} \hspace{-0.24ex}\times\hspace{-0.4ex} \UnitDyad ) \operatorname{sin} \vartheta =$} \\
%
\scalebox{0.78}[0.82]{$= \hspace{.2ex} \tikzbackcancel[black!25]{$\bm{k} \bm{k} \hspace{-0.12ex} \dotp \hspace{-0.12ex} ( \bm{k} \hspace{-0.24ex}\times\hspace{-0.4ex} \UnitDyad \hspace{.1ex} ) \hspace{.32ex} \mathdotabove{\vartheta} \operatorname{sin}^{\hspace{-0.1ex}2\hspace{-0.4ex}} \vartheta$} \hspace{.12ex}
- \hspace{-0.1ex} \UnitDyad \hspace{-0.16ex} \dotp \hspace{-0.12ex} ( \bm{k} \hspace{-0.24ex}\times\hspace{-0.4ex} \UnitDyad \hspace{.1ex} ) \hspace{.25ex} \mathdotabove{\vartheta} \operatorname{sin}^{\hspace{-0.1ex}2\hspace{-0.4ex}} \vartheta
+ \hspace{-0.2ex} \left( \hspace{-0.1ex} \bm{k} \mathdotabove{\bm{k}} \dotp \hspace{-0.1ex} ( \bm{k} \hspace{-0.32ex}\times\hspace{-0.4ex} \UnitDyad ) \hspace{-0.16ex} + \hspace{.1ex} \tikzbackcancel[black!25]{$\mathdotabove{\bm{k}} \bm{k} \dotp \hspace{-0.1ex} ( \bm{k} \hspace{-0.32ex}\times\hspace{-0.4ex} \UnitDyad$} ) \hspace{-0.12ex} \right) \hspace{-0.2ex} \operatorname{sin} \vartheta \left({1 \!-\! \operatorname{cos} \vartheta} \right) =$} \\[-0.25em]
%
\hspace{13em} \scalebox{0.94}[0.96]{$= \hspace{-0.16ex} - \hspace{.2ex} \bm{k} \hspace{-0.24ex}\times\hspace{-0.4ex} \UnitDyad \hspace{.4ex} \mathdotabove{\vartheta} \operatorname{sin}^{\hspace{-0.1ex}2\hspace{-0.4ex}} \vartheta
+ \bm{k} \mathdotabove{\bm{k}} \hspace{-0.24ex}\times\hspace{-0.32ex} \bm{k} \hspace{.2ex} \operatorname{sin} \vartheta \left({1 \!-\! \operatorname{cos} \vartheta}\right) \hspace{-0.16ex},$}
\end{multline*}
\begin{multline*}
\shoveleft{\scalebox{0.94}[0.96]{$\mathdotabove{\bm{P}}^{\mathsf{\hspace{.12ex}A}} \hspace{-0.2ex}\dotp \bm{P}^{\mathsf{\hspace{.12ex}A}} \hspace{-0.25ex} =
%
( \bm{k} \hspace{-0.24ex}\times\hspace{-0.4ex} \UnitDyad ) \hspace{.32ex} \mathdotabove{\vartheta} \operatorname{cos} \vartheta \dotp ( \bm{k} \hspace{-0.24ex}\times\hspace{-0.4ex} \UnitDyad ) \operatorname{sin} \vartheta + \hspace{-0.1ex}
( \mathdotabove{\bm{k}} \hspace{-0.24ex}\times\hspace{-0.4ex} \UnitDyad ) \hspace{-0.16ex} \dotp \hspace{-0.16ex} ( \bm{k} \hspace{-0.24ex}\times\hspace{-0.4ex} \UnitDyad ) \hspace{.1ex} \operatorname{sin}^{\hspace{-0.1ex}2\hspace{-0.4ex}} \vartheta = $} \hfill} \\
%
\shoveright{\hfill \hspace{16em} \scalebox{0.94}[0.96]{$= ( \bm{k} \bm{k} \hspace{-0.1ex} - \hspace{-0.2ex} \UnitDyad ) \hspace{.32ex} \mathdotabove{\vartheta} \operatorname{sin} \vartheta \operatorname{cos} \vartheta
+ \bm{k} \mathdotabove{\bm{k}} \operatorname{sin}^{\hspace{-0.1ex}2\hspace{-0.4ex}} \vartheta \hspace{.2ex};$}}
\end{multline*}
\end{fleqn}

\begin{multline*}
\scalebox{0.94}[0.96]{$\mathdotabove{\bm{P}} \dotp \bm{P}^{\T}$} \hspace{.25ex}
\scalebox{0.92}[0.96]{$= \hspace{.1ex}
\mathdotabove{\bm{P}}^{\mathsf{\hspace{.12ex}S}} \hspace{-0.2ex}\dotp \bm{P}^{\mathsf{\hspace{.12ex}S}}
+ \hspace{.2ex} \mathdotabove{\bm{P}}^{\mathsf{\hspace{.12ex}A}} \hspace{-0.25ex}\dotp \bm{P}^{\mathsf{\hspace{.12ex}S}}
- \hspace{.2ex} \mathdotabove{\bm{P}}^{\mathsf{\hspace{.12ex}S}} \hspace{-0.25ex}\dotp \bm{P}^{\mathsf{\hspace{.12ex}A}}
- \hspace{.2ex} \mathdotabove{\bm{P}}^{\mathsf{\hspace{.12ex}A}} \hspace{-0.25ex}\dotp \bm{P}^{\mathsf{\hspace{.12ex}A}} \hspace{-0.25ex} =$} \\[-0.25em]
%
\scalebox{0.8}[0.82]{$= {\color{black!50}{\left( \hspace{.1ex} \bm{k} \bm{k} \hspace{-0.1ex} - \hspace{-0.2ex} \UnitDyad \hspace{.1ex} \right) \hspace{-0.1ex} \mathdotabove{\vartheta} \operatorname{sin} \vartheta \operatorname{cos} \vartheta}}
+ \bm{k} \mathdotabove{\bm{k}} \operatorname{cos} \vartheta
- {\color{magenta!80!black}{\bm{k} \mathdotabove{\bm{k}}}} {\color{black!50}{\hspace{.4ex}\operatorname{cos}^{2\hspace{-0.4ex}} \vartheta}}
+ \mathdotabove{\bm{k}} \bm{k} \left({1 \!-\! \operatorname{cos} \vartheta}\right) +$} \\[-0.2em]
%
\scalebox{0.8}[0.82]{$+\; {\color{blue!80!black}{\bm{k} \hspace{-0.24ex}\times\hspace{-0.4ex} \UnitDyad \hspace{.4ex} \mathdotabove{\vartheta}}} {\color{black!50}{\hspace{.4ex}\operatorname{cos}^{2\hspace{-0.4ex}} \vartheta}}
+ \mathdotabove{\bm{k}} \hspace{-0.24ex}\times\hspace{-0.4ex} \UnitDyad \hspace{.1ex} \operatorname{sin} \vartheta \operatorname{cos} \vartheta
+ \mathdotabove{\bm{k}} \hspace{-0.24ex}\times\hspace{-0.32ex} \bm{k} \bm{k} \hspace{.2ex} {\color{green!50!black}{\hspace{.5ex}\operatorname{sin} \vartheta \left({1 \!-\! \operatorname{cos} \vartheta}\right)\hspace{.4ex}}} +$} \\[-0.25em]
%
\scalebox{0.8}[0.82]{$+\; {\color{blue!80!black}{\bm{k} \hspace{-0.24ex}\times\hspace{-0.4ex} \UnitDyad \hspace{.4ex} \mathdotabove{\vartheta}}} {\color{black!50}{\hspace{.4ex}\operatorname{sin}^{\hspace{-0.1ex}2\hspace{-0.4ex}} \vartheta}}
- \bm{k} \mathdotabove{\bm{k}} \hspace{-0.24ex}\times\hspace{-0.32ex} \bm{k} {\color{green!50!black}{\hspace{.5ex}\operatorname{sin} \vartheta \left({1 \!-\! \operatorname{cos} \vartheta}\right)\hspace{.4ex}}}
{\color{black!50}{-\hspace{.5ex} ( \bm{k} \bm{k} \hspace{-0.1ex} - \hspace{-0.2ex} \UnitDyad ) \hspace{.32ex} \mathdotabove{\vartheta} \operatorname{sin} \vartheta \operatorname{cos} \vartheta}}
- {\color{magenta!80!black}{\bm{k} \mathdotabove{\bm{k}}}} {\color{black!50}{\hspace{.4ex}\operatorname{sin}^{\hspace{-0.1ex}2\hspace{-0.4ex}} \vartheta}} =$} \\[-0.08em]
%
\scalebox{0.79}[0.82]{$= \bm{k} \hspace{-0.24ex}\times\hspace{-0.4ex} \UnitDyad \hspace{.4ex} \mathdotabove{\vartheta} \hspace{-0.1ex}
+ \hspace{-0.1ex} \hspace{-0.16ex} ( \hspace{.2ex} \mathdotabove{\bm{k}} \bm{k} \hspace{-0.2ex} - \hspace{-0.2ex} \bm{k} \mathdotabove{\bm{k}} \hspace{.2ex} ) \hspace{.1ex} ({1 \!-\! \operatorname{cos} \vartheta}) \hspace{-0.2ex}
+ \mathdotabove{\bm{k}} \hspace{-0.24ex}\times\hspace{-0.4ex} \UnitDyad \hspace{.1ex} \operatorname{sin} \vartheta \operatorname{cos} \vartheta \hspace{-0.1ex}
+ \hspace{-0.1ex} ( \mathdotabove{\bm{k}} \hspace{-0.28ex}\times\hspace{-0.32ex} \bm{k} \bm{k} \hspace{-0.12ex} - \hspace{-0.12ex} \bm{k} \mathdotabove{\bm{k}} \hspace{-0.28ex}\times\hspace{-0.32ex} \bm{k} ) \operatorname{sin} \vartheta \left({1 \!-\! \operatorname{cos} \vartheta}\right) = $} \\[-0.08em]
%
\scalebox{0.8}[0.82]{$= \bm{k} \hspace{-0.24ex}\times\hspace{-0.4ex} \UnitDyad \hspace{.4ex} \mathdotabove{\vartheta} \hspace{-0.1ex}
+ \hspace{-0.1ex} \bm{k} \hspace{-0.2ex}\times\hspace{-0.2ex}  \mathdotabove{\bm{k}} \hspace{-0.2ex}\times\hspace{-0.4ex} \UnitDyad \hspace{.25ex} ({1 \!-\! \operatorname{cos} \vartheta}) \hspace{-0.2ex}
+ \mathdotabove{\bm{k}} \hspace{-0.24ex}\times\hspace{-0.4ex} \UnitDyad \hspace{.1ex} \operatorname{sin} \vartheta \operatorname{cos} \vartheta \hspace{-0.1ex}
+ \bm{k} \hspace{-0.25ex} \times \hspace{-0.32ex} ( \mathdotabove{\bm{k}} \hspace{-0.2ex}\times\hspace{-0.2ex} \bm{k} ) \hspace{-0.4ex}\times\hspace{-0.4ex} \UnitDyad \hspace{.1ex} \operatorname{sin} \vartheta \left({1 \!-\! \operatorname{cos} \vartheta}\right) = $} \\[-0.08em]
%
\scalebox{0.78}[0.82]{$= \bm{k} \hspace{-0.24ex}\times\hspace{-0.4ex} \UnitDyad \hspace{.4ex} \mathdotabove{\vartheta} \hspace{-0.1ex}
+ \mathdotabove{\bm{k}} \hspace{-0.24ex}\times\hspace{-0.4ex} \UnitDyad \hspace{.1ex} \operatorname{sin} \vartheta \operatorname{cos} \vartheta \hspace{-0.1ex}
+ \hspace{-0.16ex} ( \mathdotabove{\bm{k}} \bm{k} \hspace{-0.1ex} \dotp \hspace{-0.16ex} \bm{k} \hspace{-0.1ex} - \tikzbackcancel[black!25]{$\bm{k} \mathdotabove{\bm{k}} \hspace{-0.16ex} \dotp \hspace{-0.1ex} \bm{k} \hspace{.1ex}$} ) \hspace{-0.4ex}\times\hspace{-0.4ex} \UnitDyad \hspace{.1ex} \operatorname{sin} \vartheta \left({1 \!-\! \operatorname{cos} \vartheta}\right) \hspace{-0.1ex}
+ \hspace{-0.1ex} \bm{k} \hspace{-0.32ex}\times\hspace{-0.25ex}  \mathdotabove{\bm{k}} \hspace{-0.25ex}\times\hspace{-0.42ex} \UnitDyad \hspace{.32ex} ({1 \!-\! \operatorname{cos} \vartheta}) = $} \\
%
\shoveright{\hfill \hspace{11.2em}\scalebox{.96}[.96]{$= \bm{k} \hspace{-0.24ex}\times\hspace{-0.4ex} \UnitDyad \hspace{.4ex} \mathdotabove{\vartheta}
+ \mathdotabove{\bm{k}} \hspace{-0.24ex}\times\hspace{-0.4ex} \UnitDyad \hspace{.1ex} \operatorname{sin} \vartheta
+ \hspace{-0.1ex} \bm{k} \hspace{-0.2ex}\times\hspace{-0.2ex}  \mathdotabove{\bm{k}} \hspace{-0.2ex}\times\hspace{-0.4ex} \UnitDyad \hspace{.32ex} ({1 \hspace{-0.2ex} - \hspace{-0.2ex} \operatorname{cos} \vartheta})
\hspace{.1ex} .
$}}
\end{multline*}

Этот результат, подставленный в~определение~\eqref{angularvelocityvector} псевдо\-вектора~$\bm{\omega}$, даёт

\nopagebreak\vspace{-0.5em}\begin{equation}
\bm{\omega} = \bm{k} \hspace{.2ex} \mathdotabove{\vartheta}
+ \mathdotabove{\bm{k}} \operatorname{sin} \vartheta
+ \bm{k} \hspace{-0.1ex}\times\hspace{-0.1ex} \mathdotabove{\bm{k}} \left( 1 - \operatorname{cos} \vartheta \right) \hspace{-0.4ex}.
\end{equation}

\vspace{-0.3em}\noindent
Вектор~$\bm{\omega}$ получился разложенным по~трём взаимно ортогональным направлениям\:--- $\bm{k}$, $\mathdotabove{\bm{k}}$ и~${\bm{k} \hspace{-0.1ex}\times\hspace{-0.1ex} \mathdotabove{\bm{k}}}$. При~неподвижной оси поворота ${\mathdotabove{\bm{k}} = \bm{0} \,\Rightarrow\, \bm{\omega} = \bm{k} \hspace{.1ex} \mathdotabove{\vartheta}}$.

Ещё одно представление~$\bm{\omega}$ связано с~компонентами тензора поворота~\eqref{componentsofrotationtensor}. Поскольку ${\bm{P} = \cosinematrix{\!j\mathcircabove{i}} \hspace{.4ex} \mathcircabove{\bm{e}}_i \mathcircabove{\bm{e}}_j}$, ${\bm{P}^{\T} \hspace{-0.32ex} = \cosinematrix{\hspace{-0.2ex}i\mathcircabove{j}} \hspace{.4ex} \mathcircabove{\bm{e}}_i \mathcircabove{\bm{e}}_j}$, а~векторы начального базиса~${\mathcircabove{\bm{e}}_i}$ неподвижны (со~временем не~меняются), то
\nopagebreak\vspace{.25em}\[ \mathdotabove{\bm{P}} = \cosinematrixdotted{\!j\mathcircabove{i}} \hspace{.4ex} \mathcircabove{\bm{e}}_i \mathcircabove{\bm{e}}_j
\hspace{.1ex} , \:\,
\mathdotabove{\bm{P}} \dotp \bm{P}^{\T} \hspace{-0.32ex} = \hspace{.1ex} \cosinematrixdotted{\hspace{-0.4ex}n\mathcircabove{i}} \hspace{.4ex} \cosinematrix{\hspace{-0.4ex}n\mathcircabove{j}} \hspace{.4ex} \mathcircabove{\bm{e}}_i \mathcircabove{\bm{e}}_j
\hspace{.1ex}, \]
\nopagebreak\vspace{-0.64em}\begin{equation}
\bm{\omega} = - \hspace{.1ex} \smalldisplaystyleonehalf \hspace{.4ex} \cosinematrixdotted{\hspace{-0.4ex}n\mathcircabove{i}} \hspace{.4ex} \cosinematrix{\hspace{-0.4ex}n\mathcircabove{j}} \hspace{.5ex} \mathcircabove{\bm{e}}_i \hspace{-0.3ex}\times\hspace{-0.3ex} \mathcircabove{\bm{e}}_j \hspace{-0.1ex} =
\smalldisplaystyleonehalf \hspace{.2ex} \levicivita_{j\hspace{-0.06ex}ik} \hspace{.32ex} \cosinematrix{\hspace{-0.4ex}n\mathcircabove{j}} \hspace{.4ex} \cosinematrixdotted{\hspace{-0.4ex}n\mathcircabove{i}} \hspace{.4ex} \mathcircabove{\bm{e}}_k
\hspace{.2ex} .
\end{equation}

\vspace{-0.25em}
Отметим и~формулы
\nopagebreak\vspace{.16em}\begin{equation}\label{angularvelocityandbasisvectors}
\begin{array}{c}
\eqref{angularvelocityvector} \hspace{.32ex} \Rightarrow\:
\mathdotabove{\bm{e}}_i \mathcircabove{\bm{e}}_i \hspace{-0.1ex} = \bm{\omega} \times \hspace{-0.1ex} \bm{e}_i \mathcircabove{\bm{e}}_i \:\Rightarrow\:
\mathdotabove{\bm{e}}_i = \bm{\omega} \times \bm{e}_i \hspace{.12ex},
\\[.32em]
%
\eqref{angularvelocityvector} \hspace{.32ex} \Rightarrow\:
\bm{\omega} = \hspace{-0.1ex} - \hspace{.16ex} \smalldisplaystyleonehalf \hspace{-.2ex} \left( \mathdotabove{\bm{e}}_i \mathcircabove{\bm{e}}_i \hspace{-0.1ex} \dotp \mathcircabove{\bm{e}}_j \bm{e}_j \right)_{\hspace{-0.25ex}\Xcompanion} \hspace{-0.32ex}
= \hspace{-0.1ex} - \hspace{.16ex} \smalldisplaystyleonehalf \hspace{-.2ex} \left( \mathdotabove{\bm{e}}_i \bm{e}_i \hspace{.1ex} \right)_{\hspace{-0.1ex}\Xcompanion} \hspace{-0.25ex}
= \smalldisplaystyleonehalf \hspace{.4ex} \bm{e}_i \hspace{-0.16ex} \times \hspace{-0.1ex} \mathdotabove{\bm{e}}_i
\hspace{.12ex} .
\end{array}
\end{equation}



\textcolor{magenta}{Не всё то вектор, что имеет величину и направление. Поворот тела вокруг оси представляет, казалось~бы, вектор, имеющий численное значение, равное углу поворота, и~направление, совпадающее с~направлением оси вращения.} Однако, два таких поворота не~складываются как векторы, когда углы поворота не~бесконечно-м\'{а}лые. На~с\'{а}мом~же деле последовательные повороты не~складываются, а~умножаются.

Можно~ли складывать угловые скорости?\:--- Да, ведь угол поворота в~$\mathdotabove{\vartheta}$ бесконечномалый.\:--- Но только при вращении вокруг неподвижной оси?

...



\end{otherlanguage}


Варьируя тождество~\eqref{orthogonalityofrotationtensor}, получим ${\variation{\rotationtensor} \hspace{-0.2ex} \dotp \rotationtensor^{\T} \hspace{-0.2ex} = - \hspace{.2ex} \rotationtensor \dotp \variation{\rotationtensor}^{\T}\!}$.
Этот тензор антисимметричен, и~потому выражается через свой сопутствующий вектор~${\varvector{o}}$ как~${\variation{\rotationtensor} \hspace{-0.1ex} \dotp \rotationtensor^{\T} \hspace{-0.3ex} = \varvector{o} \hspace{-0.2ex} \times \hspace{-0.2ex} \UnitDyad}$.
Приходим к~соотношениям

\nopagebreak\vspace{-0.5em}\begin{equation}
\variation{\rotationtensor} \hspace{-0.1ex} = \varvector{o} \hspace{-0.1ex} \times \hspace{-0.1ex} \rotationtensor , \:\:
\varvector{o} = - \hspace{.2ex} \scalebox{.93}{$ \displaystyle\onehalf $} \hspace{-0.1ex} \Bigl( \hspace{-0.1ex} \variation{\rotationtensor} \hspace{-0.1ex} \dotp \rotationtensor^{\T} \Bigr)_{\hspace{-0.25em}\Xcompanion}
\hspace{-0.1ex} ,
\end{equation}

\vspace{-0.5em}\noindent
аналогичным~\eqref{angularvelocityvector}.
Вектор бесконечно малого поворота~${\varvector{o}}$ это не~\inquotesx{вариация $\bm{\mathrm{o}}$}[,] но единый символ (в~отличие от~${\variation{\rotationtensor}}$).

Малый поворот определяется вектором~${\varvector{o}}$, но конечный поворот тоже возможно представить как вектор.

...














\en{\section{Polar decomposition}}

\ru{\section{Полярное разложение}}

\label{para:polardecomposition}

\en{Any tensor}\ru{Любой тензор} \en{of second complexity}\ru{второй сложности}~${\bm{F}}$ \en{with}\ru{с}~${\operatorname{det} F_{i\hspace{-0.1ex}j} \hspace{-0.2ex} \neq 0}$~(\en{not singular}\ru{не сингулярный} \en{tensor}\ru{тензор}) \en{can be}\ru{может быть} \en{decomposed}\ru{разложен} \en{as}\ru{как}

...

\begin{tcolorbox}
\small\setlength{\abovedisplayskip}{2pt}\setlength{\belowdisplayskip}{2pt}

\emph{Example.} Polar decompose tensor~${\bm{C} = C_{i\hspace{-0.1ex}j} \bm{e}_i \bm{e}_j}$, where $\bm{e}_k$ are mutually perpendicular unit vectors of basis, and $C_{i\hspace{-0.1ex}j}$ are tensor’s components

\begin{equation*}
C_{i\hspace{-0.1ex}j} =
\scalebox{0.92}[0.92]{$\left[\hspace{-0.2ex}\begin{array}{c@{\hspace{.6em}}c@{\hspace{.6em}}c}
-5 & 20 & 11 \\
10 & -15 & 23 \\
-3 & -5 & 10
\end{array}\hspace{-0.2ex}\right]$}
\end{equation*}

\begin{equation*}
\rotationtensor = O_{i\hspace{-0.1ex}j} \bm{e}_i \bm{e}_j \hspace{-0.2ex}
= \bm{O}_1 \hspace{-0.2ex} \dotp \bm{O}_2
\end{equation*}

\begin{equation*}
O_{i\hspace{-0.1ex}j} =
\scalebox{0.92}[0.92]{$\left[\hspace{-0.2ex}\begin{array}{c@{\hspace{.6em}}c@{\hspace{.6em}}c}
0 & \nicefrac{3}{5} & \nicefrac{4}{5} \\
0 & \nicefrac{4}{5} & - \hspace{.2ex} \nicefrac{3}{5} \\
-1 & 0 & 0
\end{array}\hspace{-0.4ex}\right]$}
\hspace{-0.25ex} = \hspace{-0.25ex}
\scalebox{0.92}[0.92]{$\left[\hspace{-0.2ex}\begin{array}{c@{\hspace{.6em}}c@{\hspace{.6em}}c}
0 & \mathcolor{red!33!white}{0} & 1 \\
\mathcolor{red!33!white}{0} & \mathcolor{red!77!black}{1} & \mathcolor{red!33!white}{0} \\
-1 & \mathcolor{red!33!white}{0} & 0
\end{array}\hspace{-0.2ex}\right]$}
\scalebox{0.92}[0.92]{$\left[\hspace{-0.2ex}\begin{array}{c@{\hspace{.6em}}c@{\hspace{.6em}}c}
\mathcolor{red!77!black}{1} & \mathcolor{red!33!white}{0} & \mathcolor{red!33!white}{0} \\
\mathcolor{red!33!white}{0} & \nicefrac{4}{5} & - \hspace{.2ex} \nicefrac{3}{5} \\
\mathcolor{red!33!white}{0} & \nicefrac{3}{5} & \nicefrac{4}{5}
\end{array}\hspace{-0.4ex}\right]$}
\end{equation*}

\begin{equation*}\begin{array}{c}
\bm{C} = \rotationtensor \hspace{-0.1ex} \dotp \bm{S_{\smash{\mathsf{R}}}}
\hspace{.1ex} , \:\:
\rotationtensor^{\T}\hspace{-0.5ex} \dotp \bm{C} = \bm{S_{\smash{\mathsf{R}}}}
\end{array}\end{equation*}

\begin{equation*}\begin{array}{c}
\bm{C} = \bm{S_{\smash{\mathsf{L}}}} \hspace{-0.25ex} \dotp \rotationtensor
\hspace{.1ex} , \:\:
\bm{C} \dotp \rotationtensor^{\T}\hspace{-0.5ex} = \bm{S_{\smash{\mathsf{L}}}}
\end{array}\end{equation*}

\begin{equation*}
S_{\smash{\mathsf{R}}\hspace{.15ex}i\hspace{-0.1ex}j} \hspace{-0.1ex} =
\scalebox{0.92}[0.92]{$\left[\hspace{-0.2ex}\begin{array}{c@{\hspace{.6em}}c@{\hspace{.6em}}c}
3 & 5 & -10 \\
5 & 0 & 25 \\
-10 & 25 & -5
\end{array}\hspace{-0.2ex}\right]$}
\end{equation*}

\begin{equation*}
S_{\smash{\mathsf{L}}\hspace{.15ex}i\hspace{-0.1ex}j} \hspace{-0.1ex} =
\scalebox{0.92}[0.92]{$\left[\hspace{-0.2ex}\begin{array}{c@{\hspace{.6em}}c@{\hspace{.6em}}c}
\nicefrac{104}{5} & \nicefrac{47}{5} & 5 \\
\nicefrac{47}{5} & - \hspace{.2ex} \nicefrac{129}{5} & -10 \\
5 & -10 & 3
\end{array}\hspace{-0.2ex}\right]$}
\end{equation*}

\par\end{tcolorbox}

...

\newpage

\en{\section{Tensors in oblique basis}}

\ru{\section{Тензоры в косоугольном базисе}}

\en{Until now}\ru{До~сих~пор} \ru{использовался }\en{a~basis}\ru{базис} \en{of~three mutually perpendicular unit vectors}\ru{трёх взаимно перпендикулярных единичных векторов}~${\bm{e}_i}$\en{ was used}.
\en{Presently}\ru{Теперь} \en{take}\ru{возьмём} \en{a~basis}\ru{базис} \en{of~any three}\ru{из~любых трёх} \en{linearly independent}\ru{линейно независимых}~(\en{non\hbox{-}coplanar}\ru{некомпланарных}) \en{vectors}\ru{векторов}~${\bm{a}_i}$.

%%\begin{figure}[!htbp]
%%\begin{center}
\begin{wrapfigure}[20]{R}{.48\textwidth}
\makebox[.5\textwidth][c]{\begin{minipage}[t]{.5\textwidth}

% some vector to draw
\pgfmathsetmacro{\lengthofvector}{2.66}
	\pgfmathsetmacro{\vectoranglefromz}{33}
	\pgfmathsetmacro{\vectoranglefromx}{59}

\tdplotsetmaincoords{33}{109} % orientation of camera

% vectors of basis
\pgfmathsetmacro{\firstlength}{0.8}
	\pgfmathsetmacro{\firstanglefromz}{90} % first and second are xy plane
	\pgfmathsetmacro{\firstanglefromx}{77} % first is not orthogonal to second
\pgfmathsetmacro{\secondlength}{1.1}
	\pgfmathsetmacro{\secondanglefromz}{90} % first and second are xy plane
	\pgfmathsetmacro{\secondanglefromx}{0}
\pgfmathsetmacro{\thirdlength}{1}
	\pgfmathsetmacro{\thirdanglefromz}{-15}
	\pgfmathsetmacro{\thirdanglefromx}{50}

\vspace{-0.9em}
\hspace{1.1em}\scalebox{0.96}[0.96]{%
\begin{tikzpicture}[scale=2.5, tdplot_main_coords] % tdplot_main_coords style to use 3dplot

	\coordinate (O) at (0,0,0);

	% draw axes and vectors of basis
	\tdplotsetcoord{A1}{\firstlength}{\firstanglefromz}{\firstanglefromx}
	\tdplotsetcoord{A2}{\secondlength}{\secondanglefromz}{\secondanglefromx}
	\tdplotsetcoord{A3}{\thirdlength}{\thirdanglefromz}{\thirdanglefromx}

	\draw [line width=0.4pt, blue] (O) -- ($ 2.2*(A1) $);
	\draw [line width=1.25pt, blue, -{Latex[round, length=3.6mm, width=2.4mm]}]
		(O) -- (A1)
		node[pos=0.93, above, inner sep=0pt, outer sep=6pt] {${\bm{a}}_1$};

	\draw [line width=0.4pt, blue] (O) -- ($ 1.08*(A2) $);
	\draw [line width=1.25pt, blue, -{Latex[round, length=3.6mm, width=2.4mm]}]
		(O) -- (A2)
		node[pos=0.97, above left, inner sep=0pt, outer sep=3.3pt] {${\bm{a}}_2$};

	\draw [line width=0.4pt, blue] (O) -- ($ 1.12*(A3) $);
	\draw [line width=1.25pt, blue, -{Latex[round, length=3.6mm, width=2.4mm]}]
		(O) -- (A3)
		node[pos=1.02, below right, inner sep=0pt, outer sep=7pt] {${\bm{a}}_3$};

	% define vector by sperical coordinates {length}{angle from z}{angle from x}
	% (plus it defines its projections)
	\tdplotsetcoord{V}{\lengthofvector}{\vectoranglefromz}{\vectoranglefromx}

	% get components of vector
	\coordinate (ParallelToThird) at ($ (V) - (A3) $);
	\coordinate (VcomponentXY) at (intersection of V--ParallelToThird and O--Vxy);

	\coordinate (ParallelToSecond) at ($ (VcomponentXY) - (A2xy) $);
	\coordinate (ParallelToFirst) at ($ (VcomponentXY) - (A1xy) $);

	\coordinate (Vcomponent1) at (intersection of VcomponentXY--ParallelToSecond and O--A1);
	\coordinate (Vcomponent2) at (intersection of VcomponentXY--ParallelToFirst and O--A2);

	\draw [line width=0.4pt, dotted, color=black] (O) -- (VcomponentXY); % projection on first & second vectors’ plane
	\draw [line width=0.4pt, dotted, color=black] (V) -- (VcomponentXY);
	\draw [line width=0.4pt, dotted, color=black] (VcomponentXY) -- (Vcomponent1);
	\draw [line width=0.4pt, dotted, color=black] (VcomponentXY) -- (Vcomponent2);

	% draw parallelopiped
	\coordinate (onPlane23) at ($ (Vcomponent2) + (V) - (VcomponentXY) $);
	\draw [line width=0.4pt, dotted, color=black] (Vcomponent2) -- (onPlane23);
	\draw [line width=0.4pt, dotted, color=black] (V) -- (onPlane23);

	\coordinate (onPlane13) at ($ (Vcomponent1) + (V) - (VcomponentXY) $);
	\draw [line width=0.4pt, dotted, color=black] (Vcomponent1) -- (onPlane13);
	\draw [line width=0.4pt, dotted, color=black] (V) -- (onPlane13);

	\coordinate (onAxis3) at ($ (V) - (VcomponentXY) $);
	\draw [line width=0.4pt, dotted, color=black] (O) -- (onAxis3);
	\draw [line width=0.4pt, dotted, color=black] (onPlane13) -- (onAxis3);
	\draw [line width=0.4pt, dotted, color=black] (onPlane23) -- (onAxis3);

	\draw [line width=0.4pt, dotted, color=black] (O) -- (onPlane13);
	\draw [line width=0.4pt, dotted, color=black] (O) -- (onPlane23);

	% draw components of vector
	\draw [color=black, line width=1.6pt, line cap=round, dash pattern=on 0pt off 1.6\pgflinewidth,
		-{Stealth[round, length=4mm, width=2.4mm]}]
		(O) -- (Vcomponent1)
		node[pos=0.6, below, shape=circle, fill=white, inner sep=-2pt, outer sep=2pt] {${v^1 \hspace{-0.1ex} \bm{a}_1}$};

	\draw [color=black, line width=1.6pt, line cap=round, dash pattern=on 0pt off 1.6\pgflinewidth,
		-{Stealth[round, length=4mm, width=2.4mm]}]
		(Vcomponent1) -- (VcomponentXY)
		node[pos=0.4, below right, fill=white, shape=circle, inner sep=0pt, outer sep=4pt] {${v^2 \hspace{-0.1ex} \bm{a}_2}$};

	\draw [color=black, line width=1.6pt, line cap=round, dash pattern=on 0pt off 1.6\pgflinewidth,
		-{Stealth[round, length=4mm, width=2.4mm]}]
		(VcomponentXY) -- (V)
		node[pos=0.55, above right, shape=circle, fill=white, inner sep=0.3pt, outer sep=6.2pt] {${v^3 \hspace{-0.1ex} \bm{a}_3}$};

	% draw vector
	\draw [line width=1.6pt, black, -{Stealth[round, length=5mm, width=2.8mm]}]
		(O) -- (V)
		node[pos=0.68, above, fill=white, inner sep=1pt, outer sep=5pt] {\scalebox{1.2}[1.2]{${\bm{v}}$}};

\end{tikzpicture}}
\vspace{-1.6em}\caption{}\label{fig:ObliqueCoordinates}
\end{minipage}}
\end{wrapfigure}
%%\end{center}
%%\end{figure}\vspace{-1.5em}

\en{Decomposition}\ru{Декомпозиция~(разложение)} \en{of~vector}\ru{вектора}~$\bm{v}$ \en{in~basis}\ru{в~базисе}~${\bm{a}_i}$~(\figref{fig:ObliqueCoordinates}) \en{is linear combination}\ru{есть линейная комбинация}

\nopagebreak\vspace{-0.2em}\en{\vspace{-0.8em}}\begin{equation}\label{decompositionbyobliquebasis}
\bm{v} = v^{i} \hspace{-0.1ex} \bm{a}_i
\hspace{.1ex} .
\end{equation}

{\small\setlength{\parindent}{0pt}
\begin{leftverticalbar}[oversize]
%% if an~index is repeated once at the upper (superscript) level and once at the lower (subscript) level in the~same monomial, it implies summation over this index
%% если индекс повторяется один раз на верхнем (надстрочном) уровне и один раз на нижнем (подстрочном) уровне в~том~же одночлене, это подразумевает суммирование по этому индексу
%%
\en{The~summation convention}\ru{Соглашение о~суммировании} \en{gains}\ru{обретает} \en{the~new conditions}\ru{новые положения}:
\en{a~summation index}\ru{индекс суммирования} \en{is repeated}\ru{повторяется} \en{at different levels}\ru{на разных уровнях} \en{of the~same monomial}\ru{того~же одночлена},
\en{and}\ru{а} \en{a~free}\ru{свободный} \en{index}\ru{индекс} \en{stays}\ru{остаётся} \en{at the~equal height}\ru{на одинаковой высоте} \en{in every part of the~expression}\ru{в~каждой части выражения}
(${a_i \hspace{-0.2ex} = \hspace{.1ex} b_{i\hspace{-0.1ex}j} c^{\hspace{.2ex}j}}$\ru{\:---}\en{ is} \en{correct}\ru{корректно},
${a_i \hspace{-0.2ex} = \hspace{.1ex} b_{kk}^{\hspace{.1ex}i}}$\ru{\:---}\en{ is} \en{wrong twice}\ru{дважды ошибочно}).
\end{leftverticalbar}
\par}

\begin{otherlanguage}{russian}

\vspace{-0.16em}
В~таком базисе уж\'{е} ${\bm{v} \dotp \bm{a}_i \hspace{-0.1ex} = \hspace{.1ex} v^{k} \bm{a}_k \hspace{-0.1ex} \dotp \bm{a}_i \neq\vspace{.2ex} v^{i}}$\hspace{-0.25ex}, ведь~тут ${\bm{a}_i \hspace{-0.1ex} \dotp \bm{a}_k \neq\vspace{.2ex} \delta_{ik}}$.

Дополним~же \hbox{базис}~${\bm{a}_i}$ ещё другой тройкой векторов~\hbox{${\bm{a}^{\hspace{-0.05ex}i}}$\hspace{-0.25ex},} \hbox{называемых} кобазисом или~взаимным базисом, чтобы

\nopagebreak\vspace{-0.3em}\begin{equation}\label{fundamentalpropertyofcobasis}
\begin{array}{c}
\bm{a}_i \dotp \bm{a}^{\hspace{.1ex}j} \hspace{-0.1ex} = \hspace{.1ex} \delta_i^{\hspace{.1ex}j} , \:\:
\bm{a}^{\hspace{-0.05ex}i} \hspace{-0.1ex} \dotp \bm{a}_j \hspace{-0.1ex} = \hspace{.1ex} \delta_{\hspace{-0.1ex}j}^{\hspace{.1ex}i}
\hspace{.16ex} ,
\\[.2em]
\UnitDyad = \bm{a}^{\hspace{-0.1ex}i} \hspace{-0.1ex} \bm{a}_i \hspace{-0.1ex} = \bm{a}_i \hspace{.1ex} \bm{a}^{\hspace{-0.1ex}i}
\hspace{-0.2ex} .
\end{array}
\end{equation}

\vspace{-0.1em}\noindent
Это\:--- основное свойство кобазиса.
Орто\-нормирован\-ный~(орто\-нормаль\-ный) базис может быть определён как совпад\'{а}ющий со~своим кобазисом: ${\bm{e}^{\hspace{.05ex}i} \hspace{-0.2ex} = \bm{e}_{i}}$.

%%\inquotes{В~декартовых координатах}, когда базис\:--- отронормальный правый:
%%\begin{itemize}
%%\item компоненты единичного~(\inquotes{метрического}) тензора\:--- дельта Кронекера,
%%\item компоненты (псевдо)тензора Л\'{е}ви\hbox{-\!}Чив\'{и}ты\:--- символ Веблена c~\inquotes{+} для \inquotes{правой} и с~\inquotes{-} для \inquotes{левой} тройки базисных векторов.
%%\end{itemize}

\begin{comment} %%
\vspace{-0.5em}\[
\bm{a}_i \dotp \bm{a}^{\hspace{.1ex}j} \hspace{-0.1ex} = \hspace{-0.2ex}
\scalebox{0.8}[0.8]{$\left[ \begin{array}{ccc}
\bm{a}_1 \hspace{-0.1ex} \dotp \bm{a}^{\hspace{-0.1ex}1} & \bm{a}_1 \hspace{-0.1ex} \dotp \bm{a}^2 & \bm{a}_1 \hspace{-0.1ex} \dotp \bm{a}^3 \\
\bm{a}_2 \hspace{-0.1ex} \dotp \bm{a}^{\hspace{-0.1ex}1} & \bm{a}_2 \hspace{-0.1ex} \dotp \bm{a}^2 & \bm{a}_2 \hspace{-0.1ex} \dotp \bm{a}^3 \\
\bm{a}_3 \hspace{-0.1ex} \dotp \bm{a}^{\hspace{-0.1ex}1} & \bm{a}_3 \hspace{-0.1ex} \dotp \bm{a}^2 & \bm{a}_3 \hspace{-0.1ex} \dotp \bm{a}^3
\end{array} \right]$} \!=\!
\scalebox{0.8}[0.8]{$\left[ \begin{array}{ccc}
1 & 0 & 0 \\
0 & 1 & 0 \\
0 & 0 & 1
\end{array} \right]$} \!=
\hspace{.1ex} \delta_i^{\hspace{.1ex}j}
\]
\end{comment} %%

Для, к~примеру, первого вектора кобазиса~${\bm{a}^{\hspace{-0.1ex}1}\hspace{-0.1ex}}$

\nopagebreak\vspace{-0.1em}\begin{equation*}
\scalebox{0.9}[0.9]{$\left\{\hspace{-0.16em}\begin{array}{l}
\bm{a}^{\hspace{-0.1ex}1} \hspace{-0.2ex} \dotp \bm{a}_{1} = 1 \\
\bm{a}^{\hspace{-0.1ex}1} \hspace{-0.2ex} \dotp \bm{a}_{2} = 0 \\
\bm{a}^{\hspace{-0.1ex}1} \hspace{-0.2ex} \dotp \bm{a}_{3} = 0 \\[.05em]
\end{array}\right.$}
\Rightarrow\hspace{.33em}
%
\scalebox{0.92}[0.92]{$\left\{\hspace{-0.12em}\begin{array}{l}
\bm{a}^{\hspace{-0.1ex}1} \hspace{-0.2ex} \dotp \hspace{.2ex} \bm{a}_{1} = 1 \\[.1em]
\gamma \hspace{.1ex} \bm{a}^{\hspace{-0.1ex}1} \hspace{-0.2ex} = \hspace{.1ex} \bm{a}_2 \hspace{-0.1ex} \times \bm{a}_3 \\[.08em]
\end{array}\right.$}
\Rightarrow\hspace{.33em}
%
\scalebox{0.96}[0.96]{$\left\{\hspace{-0.11em}\begin{array}{l}
\bm{a}^{\hspace{-0.1ex}1} \hspace{-0.2ex} =
\displaystyle \nicefrac{\scalebox{0.95}{$1$}\hspace{.1ex}}{\scalebox{1.02}{$\gamma$}} \hspace{.5ex} \bm{a}_2 \hspace{-0.1ex} \times \bm{a}_3 \hspace{.1ex} \\[.08em]
\gamma = \hspace{.1ex} \bm{a}_2 \hspace{-0.1ex} \times \bm{a}_3 \hspace{.1ex} \dotp \hspace{.25ex} \bm{a}_{1} \\[.08em]
\end{array}\right.$}
\end{equation*}

\vspace{-0.1em}\noindent
Коэффициент~$\gamma$ получился равным (с~точностью до~знака для \inquotes{левой} тройки ${\bm{a}_i}$) объёму параллелепипеда, построенного на~векторах~$\bm{a}_i$.
\en{In}\ru{В}~\pararef{para:crossproduct+levicivita} \en{the~same volume}\ru{тот~же объём} \en{was presented as}\ru{был представлен как}~${\hspace{-.22ex}\sqrt{\hspace{-0.36ex}\mathstrut{\textsl{g}}}\hspace{.16ex}}$, \en{and this is}\ru{и~это} \en{not without reason}\ru{не без причины}, \en{because}\ru{ведь} \en{it coincides with the~square root of}\ru{он совпад\'{а}ет с~квадратным корнем из}~\href{https://en.wikipedia.org/wiki/Gramian_matrix}{\en{gramian}\ru{грамиана}}~\hbox{$\textsl{g} \hspace{.1ex} \hspace{.25ex} \equiv \hspace{.2ex} \operatorname{det} \textsl{g}_{i\hspace{-0.1ex}j}$\hspace{-0.12ex}}\:--- \en{determinant of}\ru{определителя} \en{the~symmetric}\ru{симметричной} \ru{матрицы }\hbox{\href{https://en.wikipedia.org/wiki/J\%C3\%B8rgen_Pedersen_Gram}{J.\,P.\:Gram}\ru{’а}}\en{ matrix}~${\textsl{g}_{i\hspace{-0.1ex}j} \hspace{-0.1ex} \equiv \hspace{.2ex} \bm{a}_i \dotp \bm{a}_j}$.

\noindent
${ \tikz[baseline=-1ex] \draw [line width=.5pt, color=black, fill=white] (0, 0) circle (.8ex);
\hspace{.6em} }$
\en{The~proof}\ru{Доказательство} \en{resembles the~derivation of}\ru{похоже на вывод}~\eqref{doubleveblen}.
\en{The }\inquotes{\en{triple product}\ru{Тройное произведение}} ${\bm{a}_i \hspace{-0.25ex} \times \hspace{-0.25ex} \bm{a}_j \dotp \hspace{.1ex} \bm{a}_k}$ \en{in some orthonormal basis}\ru{в~каком\hbox{-}нибудь орто\-нормаль\-ном базисе}~${\bm{e}_i}$ вычисл\'{и}мо как~детерминант (с~\inquotes{$-$} для \inquotes{левой} тройки ${\bm{a}_i}$) по~строкам

\nopagebreak\vspace{-0.2em}\begin{equation*}
\levicivita_{i\hspace{-0.1ex}j\hspace{-0.1ex}k} \hspace{-0.1ex}
\equiv \hspace{.16ex} \bm{a}_i \hspace{-0.25ex} \times \hspace{-0.25ex} \bm{a}_j \dotp \hspace{.1ex} \bm{a}_k =
\hspace{.1ex} \pm \hspace{.1ex} \operatorname{det}\hspace{-0.1ex}
\scalebox{0.92}[0.92]{$\left[\hspace{-0.2ex}\begin{array}{c@{\hspace{.64em}}c@{\hspace{.64em}}c}
\bm{a}_i \narrowdotp\hspace{.12ex} \bm{e}_1 & \bm{a}_i \narrowdotp\hspace{.12ex} \bm{e}_2 & \bm{a}_i \narrowdotp\hspace{.12ex} \bm{e}_3 \\
\bm{a}_j \narrowdotp\hspace{.12ex} \bm{e}_1 & \bm{a}_j \narrowdotp\hspace{.12ex} \bm{e}_2 & \bm{a}_j \narrowdotp\hspace{.12ex} \bm{e}_3 \\
\bm{a}_k \narrowdotp\hspace{.12ex} \bm{e}_1 & \bm{a}_k \narrowdotp\hspace{.12ex} \bm{e}_2 & \bm{a}_k \narrowdotp\hspace{.12ex} \bm{e}_3
\end{array}\hspace{-0.12ex}\right]$}
\end{equation*}

\vspace{-0.2em}\noindent
или по~столбцам

\nopagebreak\vspace{-0.4em}\begin{equation*}
\levicivita_{pqr} \hspace{-0.15ex}
\equiv \hspace{.16ex} \bm{a}_p \hspace{-0.25ex} \times \hspace{-0.25ex} \bm{a}_q \dotp \hspace{.1ex} \bm{a}_r =
\hspace{.1ex} \pm \hspace{.1ex} \operatorname{det}\hspace{-0.1ex}
\scalebox{0.92}[0.92]{$\left[\hspace{-0.2ex}\begin{array}{c@{\hspace{.64em}}c@{\hspace{.64em}}c}
\bm{a}_p \narrowdotp\hspace{.12ex} \bm{e}_1 & \bm{a}_q \narrowdotp\hspace{.12ex} \bm{e}_1 & \bm{a}_r \narrowdotp\hspace{.12ex} \bm{e}_1 \\
\bm{a}_p \narrowdotp\hspace{.12ex} \bm{e}_2 & \bm{a}_q \narrowdotp\hspace{.12ex} \bm{e}_2 & \bm{a}_r \narrowdotp\hspace{.12ex} \bm{e}_2 \\
\bm{a}_p \narrowdotp\hspace{.12ex} \bm{e}_3 & \bm{a}_q \narrowdotp\hspace{.12ex} \bm{e}_3 & \bm{a}_r \narrowdotp\hspace{.12ex} \bm{e}_3
\end{array}\hspace{-0.12ex}\right]$} .
\end{equation*}

\vspace{.1em}\noindent
Произведение определителей~${\levicivita_{i\hspace{-0.1ex}j\hspace{-0.1ex}k} \levicivita_{pqr}}$ равно определителю произведения матриц, \en{and}\ru{а}~\en{elements of~the~latter}\ru{элементы последнего}\ru{\:---}\en{ are} \en{sums}\ru{суммы} \en{like}\ru{вида}
${\bm{a}_i \hspace{-0.1ex} \dotp \bm{e}_s \bm{a}_p \hspace{-0.2ex} \dotp \bm{e}_s \hspace{-0.2ex}
= \bm{a}_i \hspace{-0.1ex} \dotp \bm{e}_s \bm{e}_s \hspace{-0.2ex} \dotp \bm{a}_p \hspace{-0.15ex}
= \bm{a}_i \hspace{-0.1ex} \dotp \hspace{-0.1ex} \UnitDyad \dotp \bm{a}_p \hspace{-0.15ex}
= \bm{a}_i \hspace{-0.1ex} \dotp \bm{a}_p}$, в~результате

\nopagebreak\vspace{-0.2em}\begin{equation*}
\levicivita_{i\hspace{-0.1ex}j\hspace{-0.1ex}k} \levicivita_{pqr} = \hspace{0.25ex}
\operatorname{det}\hspace{-0.1ex}
\scalebox{0.92}[0.92]{$\left[\hspace{-0.16ex}\begin{array}{c@{\hspace{.64em}}c@{\hspace{.64em}}c}
\bm{a}_i \narrowdotp\hspace{.12ex} \bm{a}_p & \bm{a}_i \narrowdotp\hspace{.12ex} \bm{a}_q & \bm{a}_i \narrowdotp\hspace{.12ex} \bm{a}_r \\
\bm{a}_j \narrowdotp\hspace{.12ex} \bm{a}_p & \bm{a}_j \narrowdotp\hspace{.12ex} \bm{a}_q & \bm{a}_j \narrowdotp\hspace{.12ex} \bm{a}_r \\
\bm{a}_k \narrowdotp\hspace{.12ex} \bm{a}_p & \bm{a}_k \narrowdotp\hspace{.12ex} \bm{a}_q & \bm{a}_k \narrowdotp\hspace{.12ex} \bm{a}_r
\end{array}\hspace{-0.12ex}\right]$} ;
\end{equation*}

\vspace{-0.1em} \noindent
${i \narroweq p \narroweq 1}$, ${j \narroweq q \narroweq 2}$, ${k \narroweq r \narroweq 3}$ ${\,\Rightarrow}$ ${\levicivita_{123} \hspace{.2ex} \levicivita_{123} \hspace{-0.15ex} = \underset{\raisebox{.15em}{\scalebox{.7}{$i$,$\hspace{.15ex}j$}}}{\operatorname{det}} \left( \bm{a}_i \dotp \bm{a}_j \right) \hspace{-0.1ex} = \underset{\raisebox{.15em}{\scalebox{.7}{$i$,$\hspace{.15ex}j$}}}{\operatorname{det}} \, \textsl{g}_{i\hspace{-0.1ex}j}}$.
${ \hspace{.6em}
\tikz[baseline=-0.6ex] \draw [color=black, fill=black] (0, 0) circle (.8ex); }$

\en{Representing}\ru{Представляя}~${\bm{a}^{\hspace{-0.1ex}1}\hspace{-0.1ex}}$ \en{and other}\ru{и~другие} \en{cobasis vectors}\ru{векторы кобазиса} \en{as the sum}\ru{как сумму}

\nopagebreak\vspace{.8em}\begin{equation*}
\begin{array}{l@{\hspace{.3em}}c@{\hspace{.36em}}r}
\pm \hspace{.33ex} 2 \hspace{.1ex} \scalebox{0.95}[0.96]{$\sqrt{\hspace{-0.36ex}\mathstrut{\textsl{g}}}$} \hspace{.5ex} \bm{a}^{\hspace{-0.1ex}1} & = & \bm{a}_2 \times \bm{a}_3 \hspace{.2ex} \tikzmark{BeginPlusToMinus} - \hspace{.2ex} \bm{a}_3 \times \bm{a}_2 \tikzmark{EndPlusToMinus}
\hspace{.2ex} ,
\end{array}
\end{equation*}%
\AddOverBrace[line width=.75pt][0,-0.2ex]{BeginPlusToMinus}{EndPlusToMinus}%
{${\scriptstyle {+ \hspace{.4ex} \bm{a}_2 \hspace{.1ex} \times \hspace{.2ex} \bm{a}_3}}$}

\vspace{-1.3em}\noindent
приходим к~общей формуле (с~\inquotes{$-$} для \inquotes{левой} тройки ${\bm{a}_i}$)

\nopagebreak\begin{equation}\label{basisvectorstocobasisvectors}
\bm{a}^{\hspace{-0.05ex}i} \hspace{-0.2ex}
= \displaystyle \pm \hspace{.2ex} \frac{\raisemath{-0.4ex}{1}}{2 \hspace{.1ex} \scalebox{0.95}[0.96]{$\sqrt{\hspace{-0.36ex}\mathstrut{\textsl{g}}}$}} \hspace{.5ex} e^{i\hspace{-0.1ex}j\hspace{-0.1ex}k} \hspace{.1ex} \bm{a}_j \hspace{-0.1ex} \times \bm{a}_k \hspace{.1ex},
\:\:
\sqrt{\hspace{-0.36ex}\mathstrut{\textsl{g}}} \hspace{.25ex}
\equiv \hspace{.15ex} \pm \hspace{.4ex} \bm{a}_1 \hspace{-0.2ex} \times \bm{a}_2 \hspace{.1ex} \dotp \hspace{.25ex} \bm{a}_3
> 0
\hspace{.2ex} .
\end{equation}

\vspace{-0.2em}\noindent
Здесь ${e^{i\hspace{-0.1ex}j\hspace{-0.1ex}k}}$ по\hbox{-}прежнему символ перестановки Veblen’а (${\pm 1}$ или~$0$):
${e^{i\hspace{-0.1ex}j\hspace{-0.1ex}k} \hspace{-0.1ex} \equiv e_{i\hspace{-0.1ex}j\hspace{-0.1ex}k}}$.
Произведение~${\bm{a}_j \hspace{-0.1ex} \times \bm{a}_k = \hspace{.1ex} \levicivita_{j\hspace{-0.1ex}kn} \hspace{.2ex} \bm{a}^n\hspace{-0.12ex}}$, компоненты тензора Л\'{е}ви\hbox{-\!}Чив\'{и}ты~${ \levicivita_{j\hspace{-0.1ex}kn} \hspace{-0.2ex} = \pm \hspace{.33ex} e_{j\hspace{-0.1ex}kn} \hspace{-0.1ex} \sqrt{\hspace{-0.36ex}\mathstrut{\textsl{g}}} }$,
\en{and}\ru{а}~\en{by}\ru{по}~\eqref{veblencontraction} ${e^{i\hspace{-0.1ex}j\hspace{-0.1ex}k} e_{j\hspace{-0.1ex}kn} \hspace{-0.25ex} = 2 \hspace{.1ex} \delta_n^{\hspace{.1ex}i}}$.
\en{Thus}\ru{Так что}

\nopagebreak\vspace{-0.1em}\begin{equation*}\scalebox{0.96}[0.96]{$%
\begin{array}{l@{\hspace{.25em}}c@{\hspace{.33em}}r}
\bm{a}^{\hspace{-0.1ex}1} & = & \pm \hspace{.25ex} \displaystyle \nicefrac{\scalebox{0.95}{$1$}}{\hspace{-0.25ex}\sqrt{\hspace{-0.2ex}\scalebox{0.96}{$\mathstrut{\textsl{g}}$}}} \hspace{.2ex} \left( \hspace{.1ex} \bm{a}_2 \hspace{-0.1ex} \times \hspace{-0.1ex} \bm{a}_3 \hspace{.1ex} \right) \hspace{-0.3ex},
\end{array}
\begin{array}{l@{\hspace{.25em}}c@{\hspace{.33em}}r}
\bm{a}^2 & = & \pm \hspace{.25ex} \displaystyle \nicefrac{\scalebox{0.95}{$1$}}{\hspace{-0.25ex}\sqrt{\hspace{-0.2ex}\scalebox{0.96}{$\mathstrut{\textsl{g}}$}}} \hspace{.2ex} \left( \hspace{.1ex} \bm{a}_3 \hspace{-0.1ex} \times \hspace{-0.1ex} \bm{a}_1 \hspace{.1ex} \right) \hspace{-0.3ex},
\end{array}
\begin{array}{l@{\hspace{.25em}}c@{\hspace{.33em}}r}
\bm{a}^3 & = & \pm \hspace{.25ex} \displaystyle \nicefrac{\scalebox{0.95}{$1$}}{\hspace{-0.25ex}\sqrt{\hspace{-0.2ex}\scalebox{0.96}{$\mathstrut{\textsl{g}}$}}} \hspace{.2ex} \left( \hspace{.1ex} \bm{a}_1 \hspace{-0.16ex} \times \hspace{-0.1ex} \bm{a}_2 \hspace{.1ex} \right) \hspace{-0.28ex}.
\end{array}%
$}\end{equation*}

\begin{tcolorbox}
\small\setlength{\abovedisplayskip}{2pt}\setlength{\belowdisplayskip}{2pt}

\emph{Example.} Get cobasis for basis~$\bm{a}_i$ when
\[ \begin{array}{l}
\bm{a}_1 \hspace{-0.2ex} = \bm{e}_1 \hspace{-0.2ex} + \bm{e}_2 \hspace{.1ex} , \\
\bm{a}_2 \hspace{-0.2ex} = \bm{e}_1 \hspace{-0.2ex} + \bm{e}_3 \hspace{.1ex} , \\
\bm{a}_3 \hspace{-0.2ex} = \bm{e}_2 \hspace{-0.2ex} + \bm{e}_3 \hspace{.1ex} .
\end{array} \]

\[
\sqrt{\hspace{-0.36ex}\mathstrut{\textsl{g}}} \hspace{.32ex} =
- \hspace{.4ex} \bm{a}_1 \hspace{-0.2ex} \times \bm{a}_2 \hspace{.1ex} \dotp \hspace{.25ex} \bm{a}_3 \hspace{.2ex} =
- \operatorname{det}\hspace{-0.1ex}
\scalebox{0.92}[0.92]{$\left[\hspace{-0.16ex}\begin{array}{c@{\hspace{.64em}}c@{\hspace{.64em}}c}
1 & 1 & 0 \\
1 & 0 & 1 \\
0 & 1 & 1
\end{array}\hspace{-0.12ex}\right]$} \hspace{-0.5ex} = 2 \hspace{.25ex};
\]
\[
- \hspace{.4ex} \bm{a}_2 \hspace{-0.2ex} \times \bm{a}_3 = \operatorname{det}\hspace{-0.1ex}
\scalebox{0.92}[0.92]{$\left[\hspace{-0.16ex}\begin{array}{c@{\hspace{.6em}}c@{\hspace{.5em}}c}
1 & \bm{e}_1 & 0 \\
0 & \bm{e}_2 & 1 \\
1 & \bm{e}_3 & 1
\end{array}\hspace{-0.2ex}\right]$} \hspace{-0.5ex} = \bm{e}_1 \hspace{-0.2ex} + \bm{e}_2 \hspace{-0.2ex} - \bm{e}_3 \hspace{.1ex},
\]
\[
- \hspace{.4ex} \bm{a}_3 \hspace{-0.2ex} \times \bm{a}_1 = \operatorname{det}\hspace{-0.1ex}
\scalebox{0.92}[0.92]{$\left[\hspace{-0.16ex}\begin{array}{c@{\hspace{.6em}}c@{\hspace{.5em}}c}
0 & \bm{e}_1 & 1 \\
1 & \bm{e}_2 & 1 \\
1 & \bm{e}_3 & 0
\end{array}\hspace{-0.2ex}\right]$} \hspace{-0.5ex} = \bm{e}_1 \hspace{-0.2ex} + \bm{e}_3 \hspace{-0.2ex} - \bm{e}_2 \hspace{.1ex},
\]
\[
- \hspace{.4ex} \bm{a}_1 \hspace{-0.2ex} \times \bm{a}_2 = \operatorname{det}\hspace{-0.1ex}
\scalebox{0.92}[0.92]{$\left[\hspace{-0.16ex}\begin{array}{c@{\hspace{.6em}}c@{\hspace{.5em}}c}
1 & \bm{e}_1 & 1 \\
1 & \bm{e}_2 & 0 \\
0 & \bm{e}_3 & 1
\end{array}\hspace{-0.2ex}\right]$} \hspace{-0.5ex} = \bm{e}_2 \hspace{-0.2ex} + \bm{e}_3 \hspace{-0.2ex} - \bm{e}_1 \hspace{.1ex}
\]

\vspace{-0.4em}and finally
\vspace{-0.4em}\[\begin{array}{l}
\bm{a}^1 \hspace{-0.2ex}=\hspace{.1ex} \smalldisplaystyleonehalf \hspace{-0.2ex} \left(^{\mathstrut} \bm{e}_1 \hspace{-0.2ex} + \bm{e}_2 \hspace{-0.2ex} - \bm{e}_3 \right)
\hspace{-0.5ex} ,
\\[.5em]
\bm{a}^2 \hspace{-0.2ex}=\hspace{.1ex} \smalldisplaystyleonehalf \hspace{-0.2ex} \left(^{\mathstrut} \bm{e}_1 \hspace{-0.2ex} - \bm{e}_2 \hspace{-0.2ex} + \bm{e}_3 \right)
\hspace{-0.5ex} ,
\\[.5em]
\bm{a}^3 \hspace{-0.2ex}=\hspace{.1ex} \smalldisplaystyleonehalf \hspace{-0.2ex} \left(^{\mathstrut} \hspace{-0.2ex} {- \bm{e}_1} \hspace{-0.2ex} + \bm{e}_2 \hspace{-0.2ex} + \bm{e}_3 \right)
\hspace{-0.5ex} .
\end{array}\]

\par\end{tcolorbox}

Имея кобазис, возможно не~только разложить по~нему любой вектор~(\figref{fig:DecompositionOfVector}), но~и найти коэффициенты разложения~\eqref{decompositionbyobliquebasis}:
\begin{equation}\begin{array}{c}
\bm{v} = v^{i} \hspace{-0.1ex} \bm{a}_i = v_{i} \hspace{.1ex} \bm{a}^{\hspace{-0.05ex}i} \hspace{-0.25ex},
\\[.16em]
\bm{v} \dotp \bm{a}^{\hspace{-0.05ex}i} = v^{k} \hspace{-0.1ex} \bm{a}_k \hspace{-0.1ex} \dotp \bm{a}^{\hspace{-0.05ex}i} = v^{i} \hspace{-0.25ex}, \:\;
v_{i} \hspace{-0.1ex} = \bm{v} \dotp \bm{a}_i \hspace{.1ex} .
\end{array}\end{equation}
\noindent
Коэффициенты~${v_i}$ называются ко\-вариант\-ными компонентами вектора~$\bm{v}$, а~${v^i \hspace{-0.25ex}}$\:--- его контра\-вариант\-ными%
\footnote{Потому что они меняются обратно~(contra) изменению длин базисных векторов~${\bm{a}_i}$.}\hspace{-0.2ex}
компонентами.

Есть литература о~тензорах, где introducing existence and различают ко\-вариант\-ные и~контра\-вариант\-ные... векторы~(\en{and}\ru{и}~\inquotes{\en{covectors}\ru{ковекторы}}, \inquotes{dual vectors}).
Не~ст\'{о}ит вводить читателя в~заблуждение: \href{https://www.physicsforums.com/threads/is-a-vector-itself-contra-covariant-or-just-its-components.994318/}{вектор\hbox{-}то один и~тот~же}, просто разложение по~двум разным базисам даёт два набора компонент.

% ~ ~ ~ ~ ~
% converts spherical coordinates to cartesian
\newcommand{\tdsphericaltocartesian}[6]{%
\def\thecostheta{cos(#2)}%
\def\thesintheta{sin(#2)}%
\def\thecosphi{cos(#3)}%
\def\thesinphi{sin(#3)}%
\pgfmathsetmacro{#4}{ #1 * \thesintheta * \thecosphi }%
\pgfmathsetmacro{#5}{ #1 * \thesintheta * \thesinphi }%
\pgfmathsetmacro{#6}{ #1 * \thecostheta }%
}

% takes two points as cartesian {x}{y}{z} and calculates cross product of their location vectors
% placing the result into last three arguments
\newcommand{\tdcrossproductcartesian}[9]{%
\def\crossz{ #1 * #5 - #2 * #4 }%
\def\crossx{ #2 * #6 - #3 * #5 }%
\def\crossy{ #3 * #4 - #1 * #6 }%
\pgfmathsetmacro{#7}{\crossx}%
\pgfmathsetmacro{#8}{\crossy}%
\pgfmathsetmacro{#9}{\crossz}%
}

% takes two points as spherical {length}{anglefromz}{anglefromx} and calculates cross product of their location vectors
% placing the result as cartesian {x}{y}{z} into last three arguments
\newcommand{\tdcrossproductspherical}[9]{%
%
\tdplotsinandcos{\firstsintheta}{\firstcostheta}{#2}%
\tdplotsinandcos{\firstsinphi}{\firstcosphi}{#3}%
\def\firstx{ #1 * \firstsintheta * \firstcosphi }%
\def\firsty{ #1 * \firstsintheta * \firstsinphi }%
\def\firstz{ #1 * \firstcostheta }%
%
\tdplotsinandcos{\secondsintheta}{\secondcostheta}{#5}%
\tdplotsinandcos{\secondsinphi}{\secondcosphi}{#6}%
\def\secondx{ #4 * \secondsintheta * \secondcosphi }%
\def\secondy{ #4 * \secondsintheta * \secondsinphi }%
\def\secondz{ #4 * \secondcostheta }%
%
\def\crossz{ \firstx * \secondy - \firsty * \secondx }%
\def\crossx{ \firsty * \secondz - \firstz * \secondy }%
\def\crossy{ \firstz * \secondx - \firstx * \secondz }%
\pgfmathsetmacro{#7}{\crossx}%
\pgfmathsetmacro{#8}{\crossy}%
\pgfmathsetmacro{#9}{\crossz}%
}

% calculates dot product of location vectors of two 3D points specified by cartesian coordinates
\newcommand{\tddotproductcartesian}[7]{%
\edef\tddotproductcartesianxint{ \xinttheexpr round( #1 * #4 + #2 * #5 + #3 * #6 , 10 ) \relax }%
\pgfmathsetmacro{#7}{\tddotproductcartesianxint}%
}

% calculates dot product of location vectors of two 3D points specified by spherical coordinates
\newcommand{\tddotproductspherical}[7]{%
%
\tdplotsinandcos{\firstsintheta}{\firstcostheta}{#2}%
\tdplotsinandcos{\firstsinphi}{\firstcosphi}{#3}%
\def\firstx{ ( #1 * \firstsintheta * \firstcosphi ) }%
\def\firsty{ ( #1 * \firstsintheta * \firstsinphi ) }%
\def\firstz{ ( #1 * \firstcostheta ) }%
%
\tdplotsinandcos{\secondsintheta}{\secondcostheta}{#5}%
\tdplotsinandcos{\secondsinphi}{\secondcosphi}{#6}%
\def\secondx{ ( #4 * \secondsintheta * \secondcosphi ) }%
\def\secondy{ ( #4 * \secondsintheta * \secondsinphi ) }%
\def\secondz{ ( #4 * \secondcostheta ) }%
%
\edef\tddotproductsphericalxint{ \xinttheexpr round( \firstx * \secondx + \firsty * \secondy + \firstz * \secondz , 10 ) \relax }%
\pgfmathsetmacro{#7}{\tddotproductsphericalxint}%
}

% takes three points as spherical {length}{anglefromz}{anglefromx}
% and calculates triple product r1 × r2 • r3 of their location vectors
% the result is placed into \LastThreeDTripleProduct
\newcommand{\tdtripleproductspherical}[9]{%
%
\tdsphericaltocartesian{#1}{#2}{#3}{\firstx}{\firsty}{\firstz}
\tdsphericaltocartesian{#4}{#5}{#6}{\secondx}{\secondy}{\secondz}
\tdsphericaltocartesian{#7}{#8}{#9}{\thirdx}{\thirdy}{\thirdz}
%
\def\crossz{ ( \firstx * \secondy - \firsty * \secondx ) }%
\def\crossx{ ( \firsty * \secondz - \firstz * \secondy ) }%
\def\crossy{ ( \firstz * \secondx - \firstx * \secondz ) }%
%
\edef\LastThreeDTripleProduct{ \xinttheexpr round( \crossx * \thirdx + \crossy * \thirdy + \crossz * \thirdz , 10 ) \relax }%
}

% orientation of camera
\def\cameraTheta{36} \def\cameraPhi{98}
%% \def\cameraTheta{89.99} \def\cameraPhi{120}
	% 90 gives “You asked me to calculate `1/0.0', but I cannot divide any number by zero.”
\tdplotsetmaincoords{\cameraTheta}{\cameraPhi}

% vectors of basis
\pgfmathsetmacro{\firstlength}{0.69}
	\pgfmathsetmacro{\firstanglefromz}{71}
	\pgfmathsetmacro{\firstanglefromx}{-16}
\pgfmathsetmacro{\secondlength}{0.88}
	\pgfmathsetmacro{\secondanglefromz}{86}
	\pgfmathsetmacro{\secondanglefromx}{77}
\pgfmathsetmacro{\thirdlength}{0.96}
	\pgfmathsetmacro{\thirdanglefromz}{-19}
	\pgfmathsetmacro{\thirdanglefromx}{45}

\tdsphericaltocartesian%
	{\firstlength}{\firstanglefromz}{\firstanglefromx}%
	{\firstcartesianx}{\firstcartesiany}{\firstcartesianz}
\tdsphericaltocartesian%
	{\secondlength}{\secondanglefromz}{\secondanglefromx}%
	{\secondcartesianx}{\secondcartesiany}{\secondcartesianz}
\tdsphericaltocartesian%
	{\thirdlength}{\thirdanglefromz}{\thirdanglefromx}%
	{\thirdcartesianx}{\thirdcartesiany}{\thirdcartesianz}

% some but very important vector
\pgfmathsetmacro{\lengthofvector}{3.33}
	\pgfmathsetmacro{\vectoranglefromz}{33}
	\pgfmathsetmacro{\vectoranglefromx}{44}

\tdsphericaltocartesian%
	{\lengthofvector}{\vectoranglefromz}{\vectoranglefromx}%
	{\vectorcartesianx}{\vectorcartesiany}{\vectorcartesianz}

%%\begin{comment} %%
\begin{minipage}{\textwidth}
\hfill\[\scalebox{0.9}[0.9]{$\begin{array}{l@{\hspace{0.2\textwidth}}l@{\hspace{1.2em}}l@{\hspace{0.8em}}l}
\theta = \pgfmathprintnumber{\cameraTheta}\degree \hspace{0.8em}
\phi = \pgfmathprintnumber{\cameraPhi}\degree
& \scalebox{1.05}{$\bm{v}^{\varrho} = \pgfmathprintnumber{\lengthofvector}$} &
	\scalebox{1.05}{$\bm{v}^{\theta} = \pgfmathprintnumber{\vectoranglefromz}\degree$} &
	\scalebox{1.05}{$\bm{v}^{\phi} = \pgfmathprintnumber{\vectoranglefromx}\degree$} \\[0.25em]
%
& {\bm{a}_{1}^{\varrho} = \pgfmathprintnumber{\firstlength}} &
	{\bm{a}_{1}^{\theta} = \pgfmathprintnumber{\firstanglefromz}\degree} &
		{\bm{a}_{1}^{\phi} = \pgfmathprintnumber{\firstanglefromx}\degree} \\[0.1em]
& {\bm{a}_{2}^{\varrho} = \pgfmathprintnumber{\secondlength}} &
	{\bm{a}_{2}^{\theta} = \pgfmathprintnumber{\secondanglefromz}\degree} &
		{\bm{a}_{2}^{\phi} = \pgfmathprintnumber{\secondanglefromx}\degree} \\[0.1em]
& {\bm{a}_{3}^{\varrho} = \pgfmathprintnumber{\thirdlength}} &
	{\bm{a}_{3}^{\theta} = \pgfmathprintnumber{\thirdanglefromz}\degree} &
		{\bm{a}_{3}^{\phi} = \pgfmathprintnumber{\thirdanglefromx}\degree}
\end{array}$}\]
\end{minipage}
%%\end{comment} %%

\begin{figure}[!htbp]
\begin{center}

\vspace{0.1em}
\begin{tikzpicture}[scale=3.2, tdplot_main_coords] % tdplot_main_coords style to use 3dplot

	\coordinate (O) at (0,0,0);

	% define axes
	\tdplotsetcoord{A1}{\firstlength}{\firstanglefromz}{\firstanglefromx}
	\tdplotsetcoord{A2}{\secondlength}{\secondanglefromz}{\secondanglefromx}
	\tdplotsetcoord{A3}{\thirdlength}{\thirdanglefromz}{\thirdanglefromx}

	% define vector
	\tdplotsetcoord{V}{\lengthofvector}{\vectoranglefromz}{\vectoranglefromx} % {length}{angle from z}{angle from x}

	% square root of Gram matrix’ determinant is a1 × a2 • a3
	\tdtripleproductspherical%
		{\firstlength}{\firstanglefromz}{\firstanglefromx}%
		{\secondlength}{\secondanglefromz}{\secondanglefromx}%
		{\thirdlength}{\thirdanglefromz}{\thirdanglefromx}
	\edef\sqrtGramian{\xinttheexpr round( \LastThreeDTripleProduct, 10 )\relax}
	\edef\inverseOfSqrtGramian{\xinttheexpr round( 1 / \sqrtGramian, 10 )\relax}

	\node[fill=white!50, inner sep=0pt, outer sep=2pt] at (1.2,0,-1.45)
		{$\scalebox{0.9}{$\begin{array}{r}\bm{a}_1 \hspace{-0.4ex} \times \hspace{-0.3ex} \bm{a}_2 \dotp \hspace{0.2ex} \bm{a}_3 \hspace{-0.2ex} = \hspace{-0.2ex} \sqrt{\hspace{-0.36ex}\mathstrut{\textsl{g}}} \hspace{0.1ex} = \hspace{-0.2ex} \pgfmathprintnumber[fixed, precision=5]{\sqrtGramian} \\[0.25em]
		\displaystyle \nicefrac{\scalebox{0.95}{$1$}}{\hspace{-0.25ex}\sqrt{\hspace{-0.2ex}\scalebox{0.96}{$\mathstrut{\textsl{g}}$}}} \hspace{0.1ex} = \hspace{-0.2ex} \pgfmathprintnumber[fixed, precision=5]{\inverseOfSqrtGramian}\end{array}$}$};

	% calculate vectors of cobasis
	\tdcrossproductspherical%
		{\firstlength}{\firstanglefromz}{\firstanglefromx}%
		{\secondlength}{\secondanglefromz}{\secondanglefromx}%
		{\firstsecondcrossx}{\firstsecondcrossy}{\firstsecondcrossz}
	\coordinate (cross12) at (\firstsecondcrossx, \firstsecondcrossy, \firstsecondcrossz);
	\draw [line width=1.25pt, orange, -{Latex[round, length=3.6mm, width=2.4mm]}]
		(O) -- (cross12)
		node[pos=0.64, above right, inner sep=0pt, outer sep=6pt]
		{$\scalebox{0.8}{$\bm{a}_1 \hspace{-0.4ex} \times \hspace{-0.3ex} \bm{a}_2$}$};

	\tdcrossproductspherical%
		{\thirdlength}{\thirdanglefromz}{\thirdanglefromx}%
		{\firstlength}{\firstanglefromz}{\firstanglefromx}%
		{\thirdfirstcrossx}{\thirdfirstcrossy}{\thirdfirstcrossz}
	\coordinate (cross31) at (\thirdfirstcrossx, \thirdfirstcrossy, \thirdfirstcrossz);
	\draw [line width=1.25pt, orange, -{Latex[round, length=3.6mm, width=2.4mm]}]
		(O) -- (cross31)
		node[pos=0.86, above, inner sep=0pt, outer sep=5pt]
		{$\scalebox{0.8}{$\bm{a}_3 \hspace{-0.4ex} \times \hspace{-0.3ex} \bm{a}_1$}$};

	\tdcrossproductspherical%
		{\secondlength}{\secondanglefromz}{\secondanglefromx}%
		{\thirdlength}{\thirdanglefromz}{\thirdanglefromx}%
		{\secondthirdcrossx}{\secondthirdcrossy}{\secondthirdcrossz}
	\coordinate (cross23) at (\secondthirdcrossx, \secondthirdcrossy, \secondthirdcrossz);
	\draw [line width=1.25pt, orange, -{Latex[round, length=3.6mm, width=2.4mm]}]
		(O) -- (cross23)
		node[pos=0.88, below right, inner sep=0pt, outer sep=2.5pt]
		{$\scalebox{0.8}{$\bm{a}_2 \hspace{-0.4ex} \times \hspace{-0.3ex} \bm{a}_3$}$};

	\coordinate (coA3) at ($ \inverseOfSqrtGramian*(cross12) $);
	\coordinate (coA2) at ($ \inverseOfSqrtGramian*(cross31) $);
	\coordinate (coA1) at ($ \inverseOfSqrtGramian*(cross23) $);

	% get vector’s projection on a1 & a2 plane (third co-vector a^3 is normal to that plane)
	% it’s as deep down parallel to a3 as v^3 = v • a^3 in units of a3
	\tddotproductcartesian%
		{\vectorcartesianx}{\vectorcartesiany}{\vectorcartesianz}%
		{\inverseOfSqrtGramian*\firstsecondcrossx}%
			{\inverseOfSqrtGramian*\firstsecondcrossy}%
				{\inverseOfSqrtGramian*\firstsecondcrossz}%
		{\vectorthirdcoco}
	% get third co-component and translate it to vector’s head
	\coordinate (Vcomponent3) at ($ \vectorthirdcoco*(A3) $);
	\coordinate (VcomponentXY) at ($ (V) - (Vcomponent3) $);

	% decompose vector via initial basis
	\coordinate (ParallelToSecond) at ($ (VcomponentXY) - (A2) $);
	\coordinate (ParallelToFirst) at ($ (VcomponentXY) - (A1) $);
	\coordinate (Vcomponent1) at (intersection of VcomponentXY--ParallelToSecond and O--A1);
	\coordinate (Vcomponent2) at (intersection of VcomponentXY--ParallelToFirst and O--A2);

	\draw [line width=0.4pt, dotted, color=blue] (O) -- (VcomponentXY); % projection on first & second vectors’ plane

	\draw [line width=0.4pt, dotted, color=blue] (V) -- (VcomponentXY);
	\draw [line width=0.4pt, dotted, color=blue] (VcomponentXY) -- (Vcomponent1);
	\draw [line width=0.4pt, dotted, color=blue] (VcomponentXY) -- (Vcomponent2);

	% check a^1 × a^2 direction to be the same as a3
	\tdcrossproductcartesian%
		{\inverseOfSqrtGramian*\secondthirdcrossx}%
			{\inverseOfSqrtGramian*\secondthirdcrossy}%
				{\inverseOfSqrtGramian*\secondthirdcrossz}%
		{\inverseOfSqrtGramian*\thirdfirstcrossx}%
			{\inverseOfSqrtGramian*\thirdfirstcrossy}%
				{\inverseOfSqrtGramian*\thirdfirstcrossz}%
		{\CofirstCosecondOrthoX}{\CofirstCosecondOrthoY}{\CofirstCosecondOrthoZ}
	\coordinate (co1co2ortho) at (\CofirstCosecondOrthoX, \CofirstCosecondOrthoY, \CofirstCosecondOrthoZ);
	\draw [line width=1.25pt, blue!50, -{Latex[round, length=3.6mm, width=2.4mm]}]
		(O) -- ($ \sqrtGramian*(co1co2ortho) $);

	% length of a^3
	\tddotproductcartesian%
		{\inverseOfSqrtGramian*\thirdfirstcrossx}%
			{\inverseOfSqrtGramian*\thirdfirstcrossy}%
				{\inverseOfSqrtGramian*\thirdfirstcrossz}%
		{\inverseOfSqrtGramian*\thirdfirstcrossx}%
			{\inverseOfSqrtGramian*\thirdfirstcrossy}%
				{\inverseOfSqrtGramian*\thirdfirstcrossz}%
		{\squaredlengthofthirdcovector}
	%%\node[fill=white!50, inner sep=0pt, outer sep=4pt] at (0,0,-2.25)
		%%{$\scalebox{0.9}{$ | \hspace{0.1ex} \bm{a}^{\hspace{-0.1ex}3} \hspace{0.06ex} | \hspace{0.1ex} =
			%%\sqrt{\pgfmathprintnumber[fixed, precision=5]{\squaredlengthofthirdcovector}} $}$};

	% get vector’s projection on a^1 & a^2 plane (third basis vector a3 is normal to that plane)
	% it’s as deep down parallel to a^3 as v3 = v • a3 in units of a^3
	\tddotproductspherical%
		{\lengthofvector}{\vectoranglefromz}{\vectoranglefromx}%
		{\thirdlength}{\thirdanglefromz}{\thirdanglefromx}%
		{\vectorthirdcomponent}
	% get third co-component and translate it to vector’s head
	\coordinate (Vcoco3) at ($ \vectorthirdcomponent*(coA3) $);
	\coordinate (VcocoXY) at ($ (V) - (Vcoco3) $);

	% decompose vector via cobasis
	%%\coordinate (ParallelToCothird) at ($ (V) - (coA3) $);
	\coordinate (ParallelToCosecond) at ($ (VcocoXY) - (coA2) $);
	\coordinate (ParallelToCofirst) at ($ (VcocoXY) - (coA1) $);
	\coordinate (Vcoco1) at (intersection of VcocoXY--ParallelToCosecond and O--coA1);
	\coordinate (Vcoco2) at (intersection of VcocoXY--ParallelToCofirst and O--coA2);

	\draw [line width=0.4pt, red] (O) -- (Vcoco2);

	\draw [line width=0.4pt, dotted, color=red] (O) -- (VcocoXY);

	\draw [line width=0.4pt, dotted, color=red] (V) -- (VcocoXY);
	\draw [line width=0.4pt, dotted, color=red] (VcocoXY) -- (Vcoco1);
	\draw [line width=0.4pt, dotted, color=red] (VcocoXY) -- (Vcoco2);

	% draw parallelepiped of decomposition
	\coordinate (onPlane23) at ($ (Vcomponent2) + (V) - (VcomponentXY) $);
	\draw [line width=0.4pt, dotted, color=blue] (Vcomponent2) -- (onPlane23);
	\draw [line width=0.4pt, dotted, color=blue] (V) -- (onPlane23);

	\coordinate (onPlane13) at ($ (Vcomponent1) + (V) - (VcomponentXY) $);
	\draw [line width=0.4pt, dotted, color=blue] (Vcomponent1) -- (onPlane13);
	\draw [line width=0.4pt, dotted, color=blue] (V) -- (onPlane13);

	\coordinate (onAxis3) at ($ (V) - (VcomponentXY) $);
	\draw [line width=0.4pt, dotted, color=blue] (O) -- (onAxis3);
	\draw [line width=0.4pt, dotted, color=blue] (onPlane13) -- (onAxis3);
	\draw [line width=0.4pt, dotted, color=blue] (onPlane23) -- (onAxis3);

	\draw [line width=0.4pt, dotted, color=blue] (O) -- (onPlane13);
	\draw [line width=0.4pt, dotted, color=blue] (O) -- (onPlane23);

	% draw co-parallelepiped of co-decomposition
	\coordinate (onCoplane23) at ($ (Vcoco2) + (V) - (VcocoXY) $);
	\draw [line width=0.4pt, dotted, color=red] (Vcoco2) -- (onCoplane23);
	\draw [line width=0.4pt, dotted, color=red] (V) -- (onCoplane23);

	\coordinate (onCoplane13) at ($ (Vcoco1) + (V) - (VcocoXY) $);
	\draw [line width=0.4pt, dotted, color=red] (Vcoco1) -- (onCoplane13);
	\draw [line width=0.4pt, dotted, color=red] (V) -- (onCoplane13);

	\coordinate (onCoAxis3) at ($ (V) - (VcocoXY) $);
	\draw [line width=0.4pt, dotted, color=red] (O) -- (onCoAxis3);
	\draw [line width=0.4pt, dotted, color=red] (onCoplane13) -- (onCoAxis3);
	\draw [line width=0.4pt, dotted, color=red] (onCoplane23) -- (onCoAxis3);

	\draw [line width=0.4pt, dotted, color=red] (O) -- (onCoplane13);
	\draw [line width=0.4pt, dotted, color=red] (O) -- (onCoplane23);

	% draw vectors of cobasis
	\draw [line width=0.4pt, red] (O) -- ($ 1.01*(coA3) $);
	\draw [line width=1.25pt, red, -{Latex[round, length=3.6mm, width=2.4mm]}]
		(O) -- (coA3)
		node[pos=0.8, above right, shape=circle, fill=white, inner sep=-1pt, outer sep=11pt]
		{${\bm{a}}^{\hspace{-0.1ex}3}$};

	\draw [line width=0.4pt, red] (O) -- ($ 1.01*(coA2) $);
	\draw [line width=1.25pt, red, -{Latex[round, length=3.6mm, width=3mm]}]
		(O) -- (coA2)
		node[pos=0.88, above, shape=circle, fill=white, inner sep=-1pt, outer sep=4pt]
		{${\bm{a}}^{\hspace{-0.1ex}2}$};

	\draw [line width=0.4pt, red] (O) -- ($ 1.01*(coA1) $);
	\draw [line width=1.25pt, red, -{Latex[round, length=3.6mm, width=2.4mm]}]
		(O) -- (coA1)
		node[pos=0.92, below right, shape=circle, fill=white, inner sep=-1pt, outer sep=4pt]
		{${\bm{a}}^{\hspace{-0.16ex}1}$};

	% draw vectors of basis
	\draw [line width=0.4pt, blue] (O) -- ($ 1.01*(A1) $);
	\draw [line width=1.25pt, blue, -{Latex[round, length=3.6mm, width=2.4mm]}]
		(O) -- (A1)
		node[pos=0.84, above left, shape=circle, fill=white, inner sep=-1pt, outer sep=6pt]
		{${\bm{a}}_{\hspace{-0.08ex}1}$};

	\draw [line width=0.4pt, blue] (O) -- ($ 1.01*(A2) $);
	\draw [line width=1.25pt, blue, -{Latex[round, length=3.6mm, width=2.4mm]}]
		(O) -- (A2)
		node[pos=0.88, below, shape=circle, fill=white, inner sep=-1pt, outer sep=6pt]
		{${\bm{a}}_2$};
	\draw [line width=0.4pt, blue] (O) -- (Vcomponent2);

	\draw [line width=0.4pt, blue] (O) -- ($ 1.01*(A3) $);
	\draw [line width=1.25pt, blue, -{Latex[round, length=3.6mm, width=2.4mm]}]
		(O) -- (A3)
		node[pos=0.71, above left, shape=circle, fill=white, inner sep=-1pt, outer sep=16pt]
		{${\bm{a}}_3$};

	% draw components of vector
	\draw [color=blue!50!black, line width=1.6pt, line cap=round, dash pattern=on 0pt off 1.6\pgflinewidth,
		-{Stealth[round, length=4mm, width=2.4mm]}]
		(O) -- (Vcomponent1)
		node[pos=0.67, above left, fill=white, shape=circle, inner sep=0pt, outer sep=4pt]
	{${v^{\hspace{-0.08ex}1} \hspace{-0.1ex} \bm{a}_{\hspace{-0.08ex}1}}$};

	\draw [color=blue!50!black, line width=1.6pt, line cap=round, dash pattern=on 0pt off 1.6\pgflinewidth,
		-{Stealth[round, length=4mm, width=2.4mm]}]
		(Vcomponent1) -- (VcomponentXY)
		node[pos=0.48, above, shape=circle, fill=white, inner sep=-2pt, outer sep=1pt]
	{${v^2 \hspace{-0.1ex} \bm{a}_2}$};

	\draw [color=blue!50!black, line width=1.6pt, line cap=round, dash pattern=on 0pt off 1.6\pgflinewidth,
		-{Stealth[round, length=4mm, width=2.4mm]}]
		(VcomponentXY) -- (V)
		node[pos=0.77, above right, shape=circle, fill=white, inner sep=-1pt, outer sep=7pt]
	{${v^3 \hspace{-0.1ex} \bm{a}_3}$};

	% draw co-components of vector
	\draw [color=red!50!black, line width=1.6pt, line cap=round, dash pattern=on 0pt off 1.6\pgflinewidth,
		-{Stealth[round, length=4mm, width=2.4mm]}]
		(O) -- (Vcoco1)
		node[pos=1.02, above left, fill=white, shape=circle, inner sep=-1pt, outer sep=5pt]
	{${v_{\raisemath{-0.2ex}{1}} \bm{a}^{\hspace{-0.16ex}1}}$};

	\draw [color=red!50!black, line width=1.6pt, line cap=round, dash pattern=on 0pt off 1.6\pgflinewidth,
		-{Stealth[round, length=4mm, width=2.4mm]}]
		(Vcoco1) -- (VcocoXY)
		node[pos=0.5, below right, shape=circle, fill=white, inner sep=-2pt, outer sep=7pt]
	{${v_{\raisemath{-0.2ex}{2}}  \hspace{0.1ex} \bm{a}^{\hspace{-0.1ex}2}}$};

	\draw [color=red!50!black, line width=1.6pt, line cap=round, dash pattern=on 0pt off 1.6\pgflinewidth,
		-{Stealth[round, length=4mm, width=2.4mm]}]
		(VcocoXY) -- (V)
		node[pos=0.37, above left, shape=circle, fill=white, inner sep=-1pt, outer sep=9pt]
	{${v_{\raisemath{-0.2ex}{3}} \hspace{0.1ex} \bm{a}^{\hspace{-0.1ex}3}}$};

	% draw vector
	\draw [line width=1.6pt, black, -{Stealth[round, length=5mm, width=2.8mm]}]
		(O) -- (V)
		node[pos=0.69, above, shape=circle, fill=white, inner sep=0pt, outer sep=3.33pt]
			{\scalebox{1.2}[1.2]{${\bm{v}}$}};

	% calculate a_i • a^j
	\tddotproductcartesian%
		{\firstcartesianx}{\firstcartesiany}{\firstcartesianz}%
		{\inverseOfSqrtGramian*\secondthirdcrossx}%
			{\inverseOfSqrtGramian*\secondthirdcrossy}%
				{\inverseOfSqrtGramian*\secondthirdcrossz}%
		{\FirstDotCofirst}
	\tddotproductcartesian%
		{\secondcartesianx}{\secondcartesiany}{\secondcartesianz}%
		{\inverseOfSqrtGramian*\thirdfirstcrossx}%
			{\inverseOfSqrtGramian*\thirdfirstcrossy}%
				{\inverseOfSqrtGramian*\thirdfirstcrossz}%
		{\SecondDotCosecond}
	\tddotproductcartesian%
		{\thirdcartesianx}{\thirdcartesiany}{\thirdcartesianz}%
		{\inverseOfSqrtGramian*\firstsecondcrossx}%
			{\inverseOfSqrtGramian*\firstsecondcrossy}%
				{\inverseOfSqrtGramian*\firstsecondcrossz}%
		{\ThirdDotCothird}
	%
	\tddotproductcartesian%
		{\secondcartesianx}{\secondcartesiany}{\secondcartesianz}%
		{\inverseOfSqrtGramian*\secondthirdcrossx}%
			{\inverseOfSqrtGramian*\secondthirdcrossy}%
				{\inverseOfSqrtGramian*\secondthirdcrossz}%
		{\SecondDotCofirst}
	\tddotproductcartesian%
		{\firstcartesianx}{\firstcartesiany}{\firstcartesianz}%
		{\inverseOfSqrtGramian*\thirdfirstcrossx}%
			{\inverseOfSqrtGramian*\thirdfirstcrossy}%
				{\inverseOfSqrtGramian*\thirdfirstcrossz}%
		{\FirstDotCosecond}
	\tddotproductcartesian%
		{\thirdcartesianx}{\thirdcartesiany}{\thirdcartesianz}%
		{\inverseOfSqrtGramian*\secondthirdcrossx}%
			{\inverseOfSqrtGramian*\secondthirdcrossy}%
				{\inverseOfSqrtGramian*\secondthirdcrossz}%
		{\ThirdDotCofirst}
	\tddotproductcartesian%
		{\firstcartesianx}{\firstcartesiany}{\firstcartesianz}%
		{\inverseOfSqrtGramian*\firstsecondcrossx}%
			{\inverseOfSqrtGramian*\firstsecondcrossy}%
				{\inverseOfSqrtGramian*\firstsecondcrossz}%
		{\FirstDotCothird}
	\tddotproductcartesian%
		{\secondcartesianx}{\secondcartesiany}{\secondcartesianz}%
		{\inverseOfSqrtGramian*\firstsecondcrossx}%
			{\inverseOfSqrtGramian*\firstsecondcrossy}%
				{\inverseOfSqrtGramian*\firstsecondcrossz}%
		{\SecondDotCothird}
	\tddotproductcartesian%
		{\thirdcartesianx}{\thirdcartesiany}{\thirdcartesianz}%
		{\inverseOfSqrtGramian*\thirdfirstcrossx}%
			{\inverseOfSqrtGramian*\thirdfirstcrossy}%
				{\inverseOfSqrtGramian*\thirdfirstcrossz}%
		{\ThirdDotCosecond}

	% show a_i • a^j as matrix
	\node[fill=white!50, inner sep=0pt, outer sep=2pt] at (0,0.45,-3.8)
		{$\scalebox{0.9}[0.9]{$
			\bm{a}_i \dotp \hspace{0.1ex} \bm{a}^{\hspace{0.1ex}j} \hspace{-0.1ex} =
			\hspace{-0.2ex}\scalebox{0.9}[0.9]{$\left[ \begin{array}{ccc}
				\bm{a}_1 \hspace{-0.1ex} \dotp \bm{a}^{\hspace{-0.1ex}1} &
					\bm{a}_1 \hspace{-0.1ex} \dotp \bm{a}^{\hspace{-0.06ex}2} &
						\bm{a}_1 \hspace{-0.1ex} \dotp \bm{a}^{\hspace{-0.06ex}3} \\
				\bm{a}_2 \hspace{-0.1ex} \dotp \bm{a}^{\hspace{-0.1ex}1} &
					\bm{a}_2 \hspace{-0.1ex} \dotp \bm{a}^{\hspace{-0.06ex}2} &
						\bm{a}_2 \hspace{-0.1ex} \dotp \bm{a}^{\hspace{-0.06ex}3} \\
				\bm{a}_3 \hspace{-0.1ex} \dotp \bm{a}^{\hspace{-0.1ex}1} &
					\bm{a}_3 \hspace{-0.1ex} \dotp \bm{a}^{\hspace{-0.06ex}2} &
						\bm{a}_3 \hspace{-0.1ex} \dotp \bm{a}^{\hspace{-0.06ex}3}
			\end{array} \right]$} \!=\!
			\scalebox{0.9}[0.9]{$\left[ \begin{array}{ccc}
				\pgfmathprintnumber[fixed, precision=3]{\FirstDotCofirst} &
					\pgfmathprintnumber[fixed, precision=3]{\FirstDotCosecond} &
						\pgfmathprintnumber[fixed, precision=3]{\FirstDotCothird} \\
				\pgfmathprintnumber[fixed, precision=3]{\SecondDotCofirst} &
					\pgfmathprintnumber[fixed, precision=3]{\SecondDotCosecond} &
						\pgfmathprintnumber[fixed, precision=3]{\SecondDotCothird} \\
				\pgfmathprintnumber[fixed, precision=3]{\ThirdDotCofirst} &
					\pgfmathprintnumber[fixed, precision=3]{\ThirdDotCosecond} &
						\pgfmathprintnumber[fixed, precision=3]{\ThirdDotCothird}
			\end{array} \right]$}
			\hspace{-0.1em} = %%\approx
			\hspace{0.1ex} \delta_i^{\hspace{0.1ex}j}
		$}$};

\end{tikzpicture}

\vspace{0.2em}\caption{\inquotes{Decomposition of vector}}\label{fig:DecompositionOfVector}

\end{center}
\end{figure}

% ~ ~ ~ ~ ~

От~векторов перейдём к~тензорам второй сложности.
Имеем четыре комплекта диад:
${\bm{a}_i \hspace{.1ex} \bm{a}_j}$,
\hbox{$\bm{a}^{\hspace{-0.05ex}i} \hspace{-0.1ex} \bm{a}^{\hspace{0.05ex}j}$\hspace{-0.25ex},}
\hbox{$\bm{a}_i \hspace{.1ex} \bm{a}^{\hspace{0.05ex}j}$\hspace{-0.25ex},}
${\bm{a}^{\hspace{-0.05ex}i} \hspace{-0.1ex} \bm{a}_j}$.
Согласующиеся коэффициенты в~декомпозиции тензора называются его контра\-вариант\-ными, ко\-вариант\-ными и~смешан\-ными компонентами:
\vspace{.1em}\begin{equation}\begin{array}{c}
{^2\!\bm{B}} \hspace{.1ex} =
B^{\hspace{.1ex}i\hspace{-0.1ex}j} \hspace{-0.1ex} \bm{a}_i \hspace{.1ex} \bm{a}_j \hspace{-0.05ex} =
B_{i\hspace{-0.1ex}j} \hspace{.1ex} \bm{a}^{\hspace{-0.05ex}i} \hspace{-0.1ex} \bm{a}^{\hspace{0.05ex}j} \hspace{-0.15ex} =
B_{\hspace{-0.2ex} \cdot j}^{\hspace{.1ex}i} \hspace{.1ex} \bm{a}_i \hspace{.1ex} \bm{a}^{\hspace{0.05ex}j} \hspace{-0.15ex} =
B_{\hspace{-0.1ex}i}^{\hspace{-0.05ex} \cdot j} \hspace{-0.2ex} \bm{a}^{\hspace{-0.05ex}i} \hspace{-0.1ex} \bm{a}_j \hspace{.1ex}, \\[0.4em]
%
B^{\hspace{.1ex}i\hspace{-0.1ex}j} \hspace{-0.25ex} = \bm{a}^{\hspace{-0.05ex}i} \dotp {^2\!\bm{B}} \dotp \hspace{.1ex} \bm{a}^{\hspace{.05ex}j} \hspace{-0.2ex}, \:\,
B_{i\hspace{-0.1ex}j} = \bm{a}_i \dotp {^2\!\bm{B}} \dotp \hspace{.1ex} \bm{a}_j \hspace{.1ex}, \\[0.25em]
%
B_{\hspace{-0.2ex} \cdot j}^{\hspace{.1ex}i} = \bm{a}^{\hspace{-0.05ex}i} \dotp {^2\!\bm{B}} \dotp \hspace{.1ex} \bm{a}_j \hspace{.1ex}, \:\,
B_{\hspace{-0.1ex}i}^{\hspace{-0.05ex} \cdot j} \hspace{-0.2ex} = \bm{a}_i \dotp {^2\!\bm{B}} \dotp \hspace{.1ex} \bm{a}^{\hspace{.1ex}j} \hspace{-0.1ex}.
\end{array}\end{equation}

\vspace{-0.1em}\noindent Для~двух видов смешанных компонент точка в~индексе это просто свободное место: у~${B_{\hspace{-0.2ex} \cdot j}^{\hspace{.1ex}i}}$ верхний индекс~\inquotesx{$i\hspace{.25ex}$}[---] первый, а~ниж\-ний~\inquotesx{$\hspace{-0.1ex}j\hspace{.25ex}$}[---] второй.

Компоненты единичного~(\inquotes{метрического}) тензора~$\UnitDyad$

\nopagebreak\vspace{-0.1em}\begin{equation}\begin{array}{c}
\UnitDyad = \bm{a}^{k} \hspace{-0.1ex} \bm{a}_{k} \hspace{-0.1ex} = \bm{a}_{k} \hspace{.1ex} \bm{a}^{k} \hspace{-0.2ex} = \textsl{g}_{j\hspace{-0.1ex}k} \hspace{.1ex} \bm{a}^{\hspace{.1ex}j} \hspace{-0.1ex} \bm{a}^{k} \hspace{-0.2ex} = \textsl{g}^{\hspace{0.25ex}j\hspace{-0.1ex}k} \hspace{-0.1ex} \bm{a}_j \bm{a}_k \hspace{-0.32ex}: \\[0.2em]
%
\bm{a}_i \dotp \UnitDyad \dotp \hspace{.1ex} \bm{a}^{\hspace{.1ex}j} \hspace{-0.2ex} = \bm{a}_i \hspace{.1ex} \dotp \hspace{.1ex} \bm{a}^{\hspace{.1ex}j} \hspace{-0.2ex} = \delta_i^{\hspace{.1ex}j} , \hspace{0.32em}
\bm{a}^{\hspace{-0.05ex}i} \dotp \UnitDyad \dotp \hspace{.1ex} \bm{a}_j = \bm{a}^{\hspace{-0.05ex}i} \hspace{-0.2ex} \dotp \hspace{.1ex} \bm{a}_j \hspace{-0.2ex} = \delta_{\hspace{-0.1ex}j}^{\hspace{.1ex}i} \hspace{0.2ex},
\\[0.2em]
%
\bm{a}_i \dotp \UnitDyad \dotp \hspace{.1ex} \bm{a}_j = \bm{a}_i \dotp \hspace{.1ex} \bm{a}_j \equiv \hspace{.16ex} \textsl{g}_{i\hspace{-0.1ex}j} \hspace{.1ex} , \hspace{0.32em}
\bm{a}^{\hspace{-0.05ex}i} \dotp \UnitDyad \dotp \hspace{.1ex} \bm{a}^{\hspace{.1ex}j} \hspace{-0.15ex} = \bm{a}^{\hspace{-0.05ex}i} \dotp \hspace{.1ex} \bm{a}^{\hspace{.1ex}j} \hspace{-0.15ex} \hspace{.1ex} \equiv \hspace{.16ex} \textsl{g}^{\hspace{0.2ex}i\hspace{-0.1ex}j} \hspace{.1ex} ;
\\[0.2em]
%
\scalebox{0.96}[0.97]{$\UnitDyad \dotp \hspace{-0.1ex} \UnitDyad = \textsl{g}_{i\hspace{-0.1ex}j} \hspace{.1ex} \bm{a}^{\hspace{-0.05ex}i} \hspace{-0.1ex} \bm{a}^{\hspace{0.05ex}j} \hspace{-0.1ex} \dotp \hspace{.1ex} \textsl{g}^{\hspace{0.2ex}nk} \bm{a}_n \bm{a}_k \hspace{-0.1ex} = \textsl{g}_{i\hspace{-0.1ex}j} \hspace{.1ex} \textsl{g}^{\hspace{0.2ex}j\hspace{-0.1ex}k} \bm{a}^{\hspace{-0.05ex}i} \bm{a}_k \hspace{-0.1ex} = \UnitDyad$}
\:\Rightarrow\: \textsl{g}_{i\hspace{-0.1ex}j} \hspace{.1ex} \textsl{g}^{\hspace{.2ex}j\hspace{-0.1ex}k} \hspace{-0.25ex} = \delta_i^{\hspace{0.05ex}k} \hspace{-0.1ex}.
\end{array}\end{equation}

\vspace{.1em}\noindent
Вдобавок к~\eqref{fundamentalpropertyofcobasis} и~\eqref{basisvectorstocobasisvectors} открылся ещё~один способ найти векторы кобазиса\:--- через матрицу~\hbox{$\textsl{g}^{\hspace{.2ex}i\hspace{-0.1ex}j}$\hspace{-0.3ex}}, обратную матрице Грама~${\textsl{g}_{i\hspace{-0.1ex}j}}$.
И~наоборот:

\nopagebreak\vspace{-0.12em}
\begin{equation}\begin{array}{c}
\bm{a}^{\hspace{-0.05ex}i} \hspace{-0.1ex}
= \UnitDyad \dotp \bm{a}^{\hspace{-0.05ex}i} \hspace{-0.15ex}
= \textsl{g}^{\hspace{.2ex}j\hspace{-0.1ex}k} \bm{a}_j \bm{a}_k \hspace{-0.1ex} \dotp \bm{a}^{\hspace{-0.05ex}i} \hspace{-0.15ex}
= \textsl{g}^{\hspace{.2ex}j\hspace{-0.1ex}k} \bm{a}_j \hspace{.16ex} \delta_{k}^{\hspace{.1ex}i}
= \textsl{g}^{\hspace{.2ex}j\hspace{-0.06ex}i} \bm{a}_j \hspace{.1ex} ,
\\[.25em]
%
\bm{a}_i
= \UnitDyad \dotp \bm{a}_i \hspace{-0.1ex}
= \textsl{g}_{j\hspace{-0.1ex}k} \hspace{.1ex} \bm{a}^{\hspace{.1ex}j} \hspace{-0.1ex} \bm{a}^{k} \hspace{-0.2ex} \dotp \bm{a}_i \hspace{-0.1ex}
= \textsl{g}_{j\hspace{-0.1ex}k} \hspace{.1ex} \bm{a}^{\hspace{.1ex}j} \hspace{.1ex} \delta_{i}^{k}
= \textsl{g}_{j\hspace{-0.06ex}i} \hspace{.16ex} \bm{a}^{\hspace{.1ex}j} \hspace{-0.2ex} .
\end{array}\end{equation}

\begin{tcolorbox}
\small\setlength{\abovedisplayskip}{2pt}\setlength{\belowdisplayskip}{2pt}

\emph{Example.} Using reversed Gram matrix, get cobasis for basis~$\bm{a}_i$ when
\[ \begin{array}{l}
\bm{a}_1 \hspace{-0.2ex} = \bm{e}_1 \hspace{-0.2ex} + \bm{e}_2 \hspace{.1ex} , \\
\bm{a}_2 \hspace{-0.2ex} = \bm{e}_1 \hspace{-0.2ex} + \bm{e}_3 \hspace{.1ex} , \\
\bm{a}_3 \hspace{-0.2ex} = \bm{e}_2 \hspace{-0.2ex} + \bm{e}_3 \hspace{.1ex} .
\end{array} \]

\[\begin{array}{c}
\textsl{g}_{i\hspace{-0.1ex}j} \hspace{-0.32ex} = \bm{a}_i \dotp \bm{a}_j \hspace{-0.12ex} = \hspace{-0.16ex}
\scalebox{0.92}[0.92]{$\left[\hspace{-0.16ex}\begin{array}{c@{\hspace{.64em}}c@{\hspace{.64em}}c}
2 & 1 & 1 \\
1 & 2 & 1 \\
1 & 1 & 2
\end{array}\hspace{-0.12ex}\right]$} \hspace{-0.2ex}, \:\:
\operatorname{det} \hspace{.12ex} \textsl{g}_{i\hspace{-0.1ex}j} \hspace{-0.25ex} = 4 \hspace{.16ex}, \\
%
\operatorname{adj} \hspace{.12ex} \textsl{g}_{i\hspace{-0.1ex}j} \hspace{-0.25ex} = \hspace{-0.16ex}
\scalebox{0.92}[0.92]{$\left[\hspace{-0.4ex}\begin{array}{r@{\hspace{.5em}}r@{\hspace{.5em}}r}
3 & -1 & -1 \\
-1 & 3 & -1 \\
-1 & -1 & 3
\end{array}\hspace{-0.12ex}\right]^{\hspace{-0.4ex}\scalebox{1.02}{$\T$}}$} \hspace{-0.8ex}, \\
%
\textsl{g}^{\hspace{.32ex}i\hspace{-0.1ex}j} \hspace{-0.25ex} = \textsl{g}_{i\hspace{-0.1ex}j}^{\hspace{.4ex}\expminusone} \hspace{-0.12ex} = \displaystyle \frac{\operatorname{adj} \hspace{.12ex} \textsl{g}_{i\hspace{-0.1ex}j}}{\operatorname{det} \hspace{.12ex} \textsl{g}_{i\hspace{-0.1ex}j}} =
\displaystyle \frac{1}{4} \hspace{.12ex}
\scalebox{0.92}[0.92]{$\left[\hspace{-0.4ex}\begin{array}{r@{\hspace{.5em}}r@{\hspace{.5em}}r}
3 & -1 & -1 \\
-1 & 3 & -1 \\
-1 & -1 & 3
\end{array}\hspace{-0.12ex}\right]^{\mathstrut}$} \hspace{-0.25ex}.
\end{array}\]

\vspace{-0.5em}Using ${\bm{a}^i = \textsl{g}^{\hspace{.32ex}i\hspace{-0.1ex}j} \bm{a}_j}$
\[\begin{array}{l}
\bm{a}^1 \hspace{-0.2ex}=\hspace{.1ex} \textsl{g}^{\hspace{.2ex}1\hspace{-0.1ex}1} \bm{a}_1 \hspace{-0.2ex} + \textsl{g}^{\hspace{.2ex}12} \bm{a}_2 \hspace{-0.2ex} + \textsl{g}^{\hspace{.2ex}13} \bm{a}_3 \hspace{-0.2ex} = \smalldisplaystyleonehalf \bm{e}_1 \hspace{-0.2ex} + \smalldisplaystyleonehalf \bm{e}_2 \hspace{-0.2ex} - \smalldisplaystyleonehalf \bm{e}_3
\hspace{.1ex} ,
\\[.5em]
%
\bm{a}^2 \hspace{-0.2ex}=\hspace{.1ex} \textsl{g}^{\hspace{.2ex}21} \bm{a}_1 \hspace{-0.2ex} + \textsl{g}^{\hspace{.2ex}22} \bm{a}_2 \hspace{-0.2ex} + \textsl{g}^{\hspace{.2ex}23} \bm{a}_3 \hspace{-0.2ex} = \smalldisplaystyleonehalf \bm{e}_1 \hspace{-0.2ex} - \smalldisplaystyleonehalf \bm{e}_2 \hspace{-0.2ex} + \smalldisplaystyleonehalf \bm{e}_3
\hspace{.1ex} ,
\\[.5em]
%
\bm{a}^3 \hspace{-0.2ex}=\hspace{.1ex} \textsl{g}^{\hspace{.2ex}31} \bm{a}_1 \hspace{-0.2ex} + \textsl{g}^{\hspace{.2ex}32} \bm{a}_2 \hspace{-0.2ex} + \textsl{g}^{\hspace{.2ex}33} \bm{a}_3 \hspace{-0.2ex} = - \smalldisplaystyleonehalf \bm{e}_1 \hspace{-0.2ex} + \smalldisplaystyleonehalf \bm{e}_2 \hspace{-0.2ex} + \smalldisplaystyleonehalf \bm{e}_3
\hspace{.16ex} .
\end{array}\]

\par\end{tcolorbox}

...


Единичный тензор~(unit tensor, identity tensor, metric tensor)

$\UnitDyad \dotp \hspace{.1ex} \bm{\xi} = \bm{\xi} \hspace{.1ex} \dotp \hspace{-0.12ex} \UnitDyad = \bm{\xi} \quad \forall \bm{\xi}$

$\UnitDyad \dotdotp \bm{a} \bm{b} = \bm{a} \bm{b} \dotdotp \hspace{-0.12ex} \UnitDyad = \bm{a} \dotp \UnitDyad \dotp \bm{b} = \bm{a} \dotp \bm{b}$

$\UnitDyad \dotdotp \bm{A} = \bm{A} \dotdotp \hspace{-0.12ex} \UnitDyad = \trace{\hspace{-0.1ex}\bm{A}}$

$\UnitDyad \dotdotp \hspace{-0.1ex} \bm{A} = \bm{A} \dotdotp \hspace{-0.12ex} \UnitDyad = \trace{\hspace{-0.1ex}\bm{A}} \neq \operatorname{not anymore} A_{j\hspace{-0.12ex}j}$

Thus for, say, trace of some tensor ${\bm{A} = A_{i\hspace{-0.1ex}j} \locationvector^i \locationvector^j}$: ${\bm{A} \dotdotp \hspace{-0.12ex} \UnitDyad = \trace{\hspace{-0.1ex}\bm{A}}}$, you have

${
\bm{A} \dotdotp \hspace{-0.12ex} \UnitDyad
= A_{i\hspace{-0.1ex}j} \locationvector^i \locationvector^j \hspace{-0.25ex} \dotdotp \locationvector_\differentialindex{k} \locationvector^k \hspace{-0.1ex}
= A_{i\hspace{-0.1ex}j} \locationvector^i \hspace{-0.25ex} \dotp \locationvector^j \hspace{-0.1ex}
= A_{i\hspace{-0.1ex}j} \textsl{g}^{\hspace{.33ex}i\hspace{-0.1ex}j}
}$


...

Тензор поворота~(rotation tensor)

$\bm{P} \hspace{-0.2ex} = \hspace{-0.1ex} \bm{a}_i \hspace{.1ex} \mathcircabove{\bm{a}}^i \hspace{-0.2ex} = \hspace{-0.1ex} \bm{a}^{\hspace{-0.16ex}i} \mathcircabove{\bm{a}}_i \hspace{-0.2ex} = \hspace{-0.1ex} \bm{P}^{\expminusT}$

$\bm{P}^{\expminusone} \hspace{-0.4ex} = \hspace{-0.1ex} \mathcircabove{\bm{a}}_i \hspace{.1ex} \bm{a}^{\hspace{-0.16ex}i} \hspace{-0.2ex} = \hspace{-0.1ex} \mathcircabove{\bm{a}}^i \bm{a}_i \hspace{-0.2ex} = \hspace{-0.1ex} \bm{P}^{\T}$

$\bm{P}^{\T} \hspace{-0.4ex} = \hspace{-0.1ex} \mathcircabove{\bm{a}}^i \bm{a}_i \hspace{-0.2ex} = \hspace{-0.1ex} \mathcircabove{\bm{a}}_i \hspace{.1ex} \bm{a}^{\hspace{-0.16ex}i} \hspace{-0.2ex} = \hspace{-0.1ex} \bm{P}^{\expminusone}$

...



... Характеристическое уравнение~\eqref{chardetequation} быстро приводит к~тождеству К\kern-0.04em\'{э}ли\hbox{--}Г\kern-0.08em\'{а}мильтона~(Cayley\hbox{--}Hamilton)

\nopagebreak\vspace{-0.2em}\begin{equation}\label{cayley-hamilton:eq}
\begin{array}{c}
-\bm{B} \hspace{-0.2ex} \dotp \hspace{-0.2ex} \bm{B} \hspace{-0.2ex} \dotp \hspace{-0.2ex} \bm{B}
+ \hspace{.1ex} \mathrm{I} \hspace{.2ex} \bm{B} \hspace{-0.2ex} \dotp \hspace{-0.2ex} \bm{B}
- \hspace{.1ex} \mathrm{II} \hspace{.2ex} \bm{B}
+ \hspace{.1ex} \mathrm{III} \hspace{.2ex} \UnitDyad
= {^2\bm{0}}
\hspace{.1ex} ,
\\[.16em]
-\bm{B}^{3} \hspace{-0.2ex}
+ \hspace{.1ex} \mathrm{I} \hspace{.2ex} \bm{B}^{2} \hspace{-0.2ex}
- \hspace{.1ex} \mathrm{II} \hspace{.2ex} \bm{B}
+ \hspace{.1ex} \mathrm{III} \hspace{.2ex} \UnitDyad
= {^2\bm{0}}
\hspace{.1ex} .
\end{array}
\end{equation}

\end{otherlanguage}

\en{\section{Tensor functions}}

\ru{\section{Тензорные функции}}

\label{para:tensorfunctions}

\noindent
\en{In~the~concept of~function}\ru{В~представлении о~функции}~${y \narroweq \hspace{-0.15ex} f(x)}$
\en{as of mapping (morphism)}\ru{как отображении (морфизме)} ${\smash{f \hspace{-0.2ex}\colon x \mapsto \hspace{-0.16ex} y}}$,
\en{an~input~(argument)}\ru{прообраз~(аргумент)}~$x$ \en{and an~output~(result)}\ru{и~образ~(результат)}~$y$ \en{may be tensors of any complexities}\ru{могут быть тензорами любых сложностей}.

\en{Consider}\ru{Рассмотрим} \en{at~least}\ru{хотя~бы} \en{a~scalar function}\ru{скалярную функцию} \en{of~a~bivalent tensor}\ru{двухвалентного тензора}~${\varphi \narroweq \varphi(\bm{B}\hspace{.1ex})}$.
\en{Examples}\ru{Примеры}\ru{\:---}\en{ are} ${\bm{B} \hspace{-0.2ex} \dotdotp \hspace{-0.25ex} \scalebox{1.1}[1]{$\bm{\mathit{\Phi}}$}}$ (\en{or}\ru{или}~${\bm{p} \dotp \hspace{-0.2ex} \bm{B} \hspace{-0.16ex} \dotp \hspace{-0.1ex} \bm{q}}$) \en{and}\ru{и}~${\bm{B} \hspace{-0.2ex} \dotdotp \hspace{-0.2ex} \bm{B}}$.
\en{Then}\ru{Тогда} \en{in~each basis}\ru{в~каждом базисе}~${\bm{a}_i}$ \en{paired with}\ru{в~паре с}~\en{cobasis}\ru{кобазисом}~${\bm{a}^{\hspace{-0.1ex}i}}$ \en{we have}\ru{имеем} \en{function}\ru{функцию}~${\varphi(B_{i\hspace{-0.1ex}j})}$ \en{of~nine numeric arguments}\ru{девяти числовых аргументов}\:--- \en{components}\ru{компонент}~$B_{i\hspace{-0.1ex}j}$ \en{of~tensor}\ru{тензора}~$\bm{B}$.
\en{For example}\ru{Для примера}

\nopagebreak\vspace{-0.2em}\begin{equation*}
\varphi(\bm{B}\hspace{.1ex}) \hspace{-0.2ex}
= \bm{B} \hspace{-0.2ex} \dotdotp \hspace{-0.28ex} \scalebox{1.1}[1]{$\bm{\mathit{\Phi}}$} \hspace{-0.1ex}
= \hspace{-0.1ex} B_{i\hspace{-0.1ex}j} \hspace{.16ex} \bm{a}^{\hspace{-0.1ex}i} \hspace{-0.16ex} \bm{a}^j \hspace{-0.3ex} \dotdotp \bm{a}_m \bm{a}_n \hspace{-0.2ex} \scalebox{1.2}[1]{$\mathit{\Phi}$}^{mn} \hspace{-0.25ex}
= \hspace{-0.1ex} B_{i\hspace{-0.1ex}j} \hspace{-0.15ex} \scalebox{1.2}[1]{$\mathit{\Phi}$}^{\hspace{.1ex}j\hspace{-0.06ex}i} \hspace{-0.25ex}
= \varphi(B_{i\hspace{-0.1ex}j})
\hspace{.1ex} .
\end{equation*}

\vspace{-0.25em} \noindent
\en{With any transition}\ru{С~любым переходом} \en{to~a~new basis,}\ru{к~новому базису} \en{the~result}\ru{результат} \en{doesn’t change}\ru{не~меняется}:
${\varphi(B_{i\hspace{-0.1ex}j}) \hspace{-0.2ex} = \varphi(B\hspace{.16ex}'_{\hspace{-0.32ex}i\hspace{-0.1ex}j}) \hspace{-0.2ex} = \varphi(\bm{B}\hspace{.1ex})}$.

\en{Differentiation of}\ru{Дифференцирование}~${\varphi(\bm{B}\hspace{.1ex})}$ \en{looks like}\ru{выглядит как}

\nopagebreak\vspace{-0.2em}\begin{equation}
d \hspace{.1ex} \varphi
= \displaystyle \frac{\partial \hspace{.1ex} \varphi}{\partial \hspace{-0.2ex} B_{i\hspace{-0.1ex}j}} \hspace{.2ex} d B_{i\hspace{-0.1ex}j} \hspace{-0.2ex}
= \displaystyle \frac{\partial \hspace{.1ex} \varphi}{\partial \hspace{-0.1ex} \bm{B}} \dotdotp d \bm{B}^{\T}
\hspace{-0.25ex} .
\end{equation}

\en{\vspace{-0.15em}}\ru{\vspace{-0.25em}}\noindent
\en{Tensor}\ru{Тензор}~${\scalebox{0.98}[1]{$\raisemath{.16em}{\scalebox{0.92}[0.92]{$\partial \hspace{.15ex} \varphi$}} \hspace{-0.1ex} / \hspace{-0.1ex} \raisemath{-0.32em}{\scalebox{0.92}[0.92]{$\partial \hspace{-0.1ex} \bm{B}$}}\hspace{.1ex}$}}$
\en{is called}\ru{называется} \en{the~derivative}\ru{производной} \en{of~function}\ru{функции}~$\varphi$ \en{by~argument}\ru{по~аргументу}~${\hspace{-0.15ex}\bm{B}\hspace{.1ex}}$;
${d \bm{B}}$\en{~is}\ru{\:---} \en{the~differential}\ru{дифференциал} \en{of~tensor}\ru{тензора}~$\bm{B}$,
${d \bm{B} \hspace{-0.1ex} = d B_{i\hspace{-0.1ex}j} \hspace{.16ex} \bm{a}^{\hspace{-0.1ex}i} \hspace{-0.16ex} \bm{a}^j}$;
${\smash{\scalebox{0.98}[1]{$\raisemath{.16em}{\scalebox{0.9}{$\partial \hspace{.15ex} \varphi$}} \hspace{-0.1ex} / \hspace{-0.2ex} \raisemath{-0.32em}{\scalebox{0.9}{$\partial \hspace{-0.1ex} B_{i\hspace{-0.1ex}j}$}}$}}}$\ru{\:---}\en{~are} \en{components}\ru{компоненты}~(\en{contra\-variant ones}\ru{контра\-вариант\-ные}) \en{of~\,}${\smash{\scalebox{0.98}[1]{$\raisemath{.16em}{\scalebox{0.92}[0.92]{$\partial \hspace{.15ex} \varphi$}} \hspace{-0.1ex} / \hspace{-0.2ex} \raisemath{-0.32em}{\scalebox{0.92}{$\partial \hspace{-0.1ex} \bm{B}$}}\hspace{.2ex}$}}}$

\nopagebreak\vspace{-0.1em}\begin{equation*}
\bm{a}^{\hspace{-0.1ex}i} \hspace{-0.15ex} \dotp \scalebox{0.92}{$\displaystyle\frac{\partial \hspace{.1ex} \varphi}{\partial \hspace{-0.1ex} \bm{B}}$} \dotp \bm{a}^j \hspace{-0.2ex}
=
\scalebox{0.92}{$\displaystyle\frac{\partial \hspace{.1ex} \varphi}{\partial \hspace{-0.1ex} \bm{B}}$} \dotdotp \bm{a}^j \hspace{-0.2ex} \bm{a}^{\hspace{-0.1ex}i} \hspace{-0.2ex}
=
\scalebox{0.92}{$\displaystyle\frac{\partial \hspace{.1ex} \varphi}{\partial \hspace{-0.15ex} B_{i\hspace{-0.1ex}j}}$}
\;\Leftrightarrow\;
\scalebox{0.92}{$\displaystyle\frac{\partial \hspace{.1ex} \varphi}{\partial \hspace{-0.1ex} \bm{B}}$}
=
\scalebox{0.92}{$\displaystyle\frac{\partial \hspace{.1ex} \varphi}{\partial \hspace{-0.15ex} B_{i\hspace{-0.1ex}j}}$} \hspace{.25ex} \bm{a}_i \bm{a}_{\hspace{-0.1ex}j}
\hspace{.1ex} .
\end{equation*}

...

\nopagebreak\begin{equation*}\begin{array}{c}
\varphi(\bm{B}\hspace{.1ex}) \hspace{-0.2ex}
= \bm{B} \hspace{-0.2ex} \dotdotp \hspace{-0.25ex} \scalebox{1.1}[1]{$\bm{\mathit{\Phi}}$}
\\[.2em]
%
d \hspace{.1ex} \varphi
= d \hspace{.2ex} ( \bm{B} \hspace{-0.2ex} \dotdotp \hspace{-0.25ex} \scalebox{1.1}[1]{$\bm{\mathit{\Phi}}$} \hspace{.1ex} ) \hspace{-0.2ex}
= d \bm{B} \hspace{-0.2ex} \dotdotp \hspace{-0.25ex} \scalebox{1.1}[1]{$\bm{\mathit{\Phi}}$}
= \hspace{-0.1ex} \scalebox{1.1}[1]{$\bm{\mathit{\Phi}}$} \hspace{-0.12ex} \dotdotp d \bm{B}
= \hspace{-0.1ex} \scalebox{1.1}[1]{$\bm{\mathit{\Phi}}$}^{\T} \hspace{-0.4ex} \dotdotp d \bm{B}^{\T}
\\[.2em]
%
d \hspace{.1ex} \varphi
= \scalebox{0.92}{$\displaystyle \frac{\partial \hspace{.1ex} \varphi}{\partial \hspace{-0.1ex} \bm{B}} \dotdotp d \bm{B}^{\T}$} \hspace{-0.3ex} ,
\:\:
\scalebox{0.92}{$\displaystyle \frac{\partial \hspace{-0.1ex} \left( \bm{B} \hspace{-0.2ex} \dotdotp \hspace{-0.25ex} \scalebox{1.1}[1]{$\bm{\mathit{\Phi}}$} \right)}{\partial \hspace{-0.1ex} \bm{B}}$}
= \hspace{-0.1ex} \scalebox{1.1}[1]{$\bm{\mathit{\Phi}}$}^{\T}
\end{array}\end{equation*}

${\bm{p} \dotp \hspace{-0.2ex} \bm{B} \hspace{-0.16ex} \dotp \hspace{-0.1ex} \bm{q} \hspace{.1ex} =
\hspace{-0.1ex} \bm{B} \hspace{-0.16ex} \dotdotp \hspace{-0.1ex} \bm{q} \bm{p}}$

\nopagebreak\begin{equation*}
\scalebox{0.92}[0.92]{$\displaystyle \frac{\partial \hspace{-0.1ex} \left( \hspace{.1ex} \bm{p} \dotp \hspace{-0.2ex} \bm{B} \hspace{-0.16ex} \dotp \hspace{-0.1ex} \bm{q} \right)}{\partial \hspace{-0.1ex} \bm{B}}$}
= \bm{p} \bm{q}
\end{equation*}

...

\nopagebreak\begin{equation*}\begin{array}{c}
\varphi(\bm{B}\hspace{.1ex}) \hspace{-0.2ex}
= \bm{B} \hspace{-0.2ex} \dotdotp \hspace{-0.2ex} \bm{B}
\\[.2em]
%
d \hspace{.1ex} \varphi
= d \hspace{.2ex} ( \bm{B} \hspace{-0.2ex} \dotdotp \hspace{-0.2ex} \bm{B} \hspace{.1ex} ) \hspace{-0.2ex}
= d ...
\end{array}\end{equation*}


...


\begin{otherlanguage}{russian}

Но согласно опять\hbox{-}таки~\eqref{cayley-hamilton:eq}
${\hspace{-0.1ex} -\bm{B}^{2} \hspace{-0.2ex} + \mathrm{I}\hspace{.16ex} \bm{B} \hspace{-0.1ex} - \mathrm{II}\hspace{.16ex} \UnitDyad + \mathrm{III}\hspace{.16ex} \bm{B}^{\expminusone} \hspace{-0.25ex} = {^2\bm{0}}}$, поэтому


...


Скалярная функция~${\varphi(\bm{B})}$ называется изотропной, если она не~чувствительна к~повороту аргумента:
\nopagebreak\vspace{.1em}\begin{equation*}
\varphi(\bm{B}) \hspace{-0.12ex} = \varphi ( \rotationtensor \narrowdotp \smash{\mathcircabove{\bm{B}}} \narrowdotp \hspace{.15ex} \rotationtensor^{\hspace{-0.1ex}\T} ) \hspace{-0.2ex} = \varphi(\smash{\mathcircabove{\bm{B}}}) \;\;\,
\forall \rotationtensor \hspace{-0.2ex} = \hspace{-0.1ex} \bm{a}_i \hspace{.1ex} \mathcircabove{\bm{a}}^i \hspace{-0.25ex} = \hspace{-0.1ex} \bm{a}^{\hspace{-0.2ex}i} \mathcircabove{\bm{a}}_i \hspace{-0.16ex} = \hspace{-0.1ex} \rotationtensor^{\hspace{-0.1ex}\expminusT}
\end{equation*}
\par\vspace{-0.25em}\noindent
для~любого ортогонального тензора~$\rotationtensor$ (тензора поворота, \pararef{para:rotationtensor}).

Симметричный тензор~${\bm{B}^{\mathsf{\hspace{.1ex}S}}}$ полностью определяется тройкой инвариантов и~угловой ориентацией собственных осей (они~же взаимно ортогональны, \pararef{para:eigenvectorseigenvalues}).
Ясно, что изотропная функция~${\varphi(\bm{B}^{\mathsf{\hspace{.1ex}S}})}$ симметричного аргумента является функцией лишь инвариантов ${\mathrm{I}\hspace{.16ex}({\bm{B}^{\mathsf{\hspace{.1ex}S}}})}$, ${\mathrm{II}\hspace{.16ex}({\bm{B}^{\mathsf{\hspace{.1ex}S}}})}$, ${\mathrm{III}\hspace{.16ex}({\bm{B}^{\mathsf{\hspace{.1ex}S}}})}$;
она дифференцируется согласно~\eqref{fonvccbnmxghjsxmnxjsdjhga}, где транспонирование излишне.

\end{otherlanguage}

\en{\section{Spatial differentiation}}

\ru{\section{Пространственное дифференцирование}}

\label{para:spatialdifferentiationoftensorfields} <<<<<< rename: remove fields

\begin{changemargin}{\parindent}{\parindent}
\vspace{-0.1em}
\small
\flushright
\textit{\en{Tensor field}\ru{Тензорное поле}}\ru{\:---}\en{ is} \en{a~tensor}\ru{это тензор}\ru{,} \en{varying from~point to~point}\ru{меняющийся от~точки к~точке} (\en{variable in~space}\ru{переменный в~пространстве}, \en{coordinate dependent}\ru{зависящий от~координат}).

\par\vspace{.3em}
\end{changemargin}

\begin{otherlanguage}{russian}

<<<<<<<

\noindent
Пусть \en{at each point}\ru{в~каждой точке} \en{of some region}\ru{некоторой области} \en{of a~three-dimensional space}\ru{трёхмерного пространства} определена величина~$\varsigma$.
Тогда говорят, что есть тензорное поле~${\varsigma \!=\! \varsigma(\locationvector)}$, \en{where}\ru{где}~$\locationvector$\en{ is}\ru{\:---} \en{location vector}\ru{вектор положения}~(\en{radius vector}\ru{вектор\hbox{-}радиус}) \en{of~a~point}\ru{точки} \en{in~space}\ru{пространства}.

Величина~$\varsigma$ может~быть тензором любой сложности.
Пример скалярного поля\:--- поле температуры в~среде, векторного поля\:--- скорости частиц жидкости.

Концепт тензорного поля никак не~связан с~концептом поля с~операциями $+$ и~$*$ с~11~свойствами этих операций.

\end{otherlanguage}

% ~ ~ ~ ~ ~
\begin{wrapfigure}{R}{0.55\textwidth}
\makebox[0.55\textwidth][c]{%
\hspace{2em}
\begin{minipage}[t]{.55\textwidth}

\begin{tikzpicture}[scale=0.5]

%%\clip (-6, -6) rectangle + (12, 12) ; % crop it
\clip (0, 0) circle (6cm) ; % crop it

\tikzset{%
	tangent/.style={
		decoration={
			markings,% switch on markings
			mark=
			at position #1
			with
			{
				\def\numberoftangent{\pgfkeysvalueof{/pgf/decoration/mark info/sequence number}}
				\coordinate (tangent point-\numberoftangent) at (0, 0);
				\coordinate (tangent unit vector-\numberoftangent) at (1, 0);
				\coordinate (tangent orthogonal unit vector-\numberoftangent) at (0, 1);
			}
		},
		postaction=decorate
	},
	use tangent/.style={
		shift=(tangent point-#1),
		x=(tangent unit vector-#1),
		y=(tangent orthogonal unit vector-#1)
	},
	use tangent/.default=1
}

\tikzset{%
	show curve controls/.style={
		postaction={
			decoration={
				show path construction,
				curveto code={
					\fill [black, opacity=.5]
						(\tikzinputsegmentfirst) circle (.4ex)
						(\tikzinputsegmentlast) circle (.4ex) ;
					\draw [black, opacity=.5, line cap=round, dash pattern=on 0pt off 1.6\pgflinewidth]
						(\tikzinputsegmentfirst) -- (\tikzinputsegmentsupporta)
						(\tikzinputsegmentlast) -- (\tikzinputsegmentsupportb) ;
					\fill [magenta, opacity=.5, line cap=round, dash pattern=on 0pt off 1.6\pgflinewidth]
						(\tikzinputsegmentsupporta) circle [radius=.4ex]
						(\tikzinputsegmentsupportb) circle [radius=.4ex] ;
				}
			},
			decorate
}	}	}

%%\foreach \cycle in {0, 1, ..., 15}
%%	\draw [color=green]
%%		($ (0, 0) - (\cycle, 1.2*\cycle) $)
%%		parabola ($ (4, 3) + 0.5*(1.6*\cycle, \cycle) $);

\foreach \c in {-10, -9.5, ..., 10}
{
	\def\offset{0.2*\c, -0.1*\c}
	\pgfmathsetmacro\bottomoffsetx{-.24 * ( \c )}
	\pgfmathsetmacro\bottomoffsety{-.1 * abs( \c ) + .1 * ( \c )}
	\pgfmathsetmacro\bottomangle{12 - 1.2 * abs( \c )}
	\pgfmathsetmacro\bottomnudge{2}
	\pgfmathsetmacro\midoffsetx{-.1 * abs( \c )}
	\pgfmathsetmacro\midoffsety{.1 * abs( \c )}
	\pgfmathsetmacro\midangle{63 + 1.2 * abs( \c )}
	\pgfmathsetmacro\midnudge{4 + ( .1 * abs( \c ) )}
	\pgfmathsetmacro\topoffsetx{.32 * ( \c ) + 0 * abs( \c )}
	\pgfmathsetmacro\topoffsety{-.16 * ( \c ) + 0 * abs( \c )}
	\pgfmathsetmacro\topangle{166 + 1.6 * ( \c ) + 1.2 * abs( \c )}
	\pgfmathsetmacro\topnudge{5 + ( .25 * abs( \c ) )}
	\draw	[ line width=.4pt
		, color=blue!50
		%%, show curve controls
		]
		($ (-6, -4.5) + 5*(\offset) + (\bottomoffsetx, \bottomoffsety) $)
		.. controls ++(\bottomangle: \bottomnudge) and ++(\midangle: -\midnudge) ..
		($ 4*(\offset) + (\midoffsetx, \midoffsety) $)
		.. controls ++(\midangle: \midnudge) and ++(\topangle: \topnudge) ..
		($ (8, 6) + 2.5*(\offset) + (\topoffsetx, \topoffsety) $) ;
}

\foreach \c in {-10, -9.5, ..., 10}
{
	\def\offset{0.2*\c, 0.1*\c}
	\pgfmathsetmacro\leftoffsetx{- .1 * abs ( \c )}
	\pgfmathsetmacro\leftoffsety{.4 * ( \c )}
	\pgfmathsetmacro\leftangle{33 + .2 * abs( \c )}
	\pgfmathsetmacro\leftnudge{1.6 + .5 * abs( \c )}
	\pgfmathsetmacro\midoffsetx{-.2 * abs( \c )}
	\pgfmathsetmacro\midoffsety{.2 * abs( \c )}
	\pgfmathsetmacro\midangle{111 + 1.2 * abs( \c )}
	\pgfmathsetmacro\midnudge{5}
	\pgfmathsetmacro\rightoffsetx{.25 * abs( \c )}
	\pgfmathsetmacro\rightoffsety{.16 * ( \c )}
	\pgfmathsetmacro\rightangle{177 + 2 * ( \c )}
	\pgfmathsetmacro\rightnudge{abs( 2 - ( .5 * ( \c ) ) )}
	\draw	[ line width=.4pt
		, color=red!50
		%%, show curve controls
		]
		($ (-12, 5) + 2.5*(\offset) + (\leftoffsetx, \leftoffsety) $)
		.. controls ++(\leftangle: \leftnudge) and ++(\midangle: \midnudge) ..
		($ 5*(\offset) + (\midoffsetx, \midoffsety) $)
		.. controls ++(\midangle: -\midnudge) and ++(\rightangle: \rightnudge) ..
		($ (8, -5) + 4*(\offset) + (\rightoffsetx, \rightoffsety) $);
}

\foreach \c in {-10, -9.5, ..., 10}
{
	\def\offset{0*\c, 0.25*\c}
	\pgfmathsetmacro\midnudge{6 + .16 * ( \c )}
	\draw	[ line width=.4pt
		, color=green!50
		%%, show curve controls
		]
		($ (12, 10) + 4*(\offset) $)
		.. controls ++(88: -4) and ++(11: \midnudge) ..
		($ 4*(\offset) $)
		.. controls ++(11: -\midnudge) and ++(99: -4) ..
		($ (-12, 4) + 4*(\offset) $) ;
}

\draw	[ line width=.8pt
	, color=blue!50!black
	%%, show curve controls
	]
	(-6, -4.5)
	.. controls ++(12: 2) and ++(63: -4) ..
	(0, 0);

\draw	[ line width=.8pt
	, color=blue!50!black
	%%, show curve controls
	, tangent=0
	, tangent=0.4
	]
	(0, 0)
	.. controls ++(63: 4) and ++(166: 5) ..
	(8, 6) ;

\path [use tangent=1]
	(0, 0) -- (.4*4, 0)
	node [color=blue, pos=0.86, above left, shape=circle, fill=white, outer sep=4pt, inner sep=1pt]
		{$\bm{r}_3$} ;

\draw [line width=1.25pt, color=blue, use tangent=1, -{Latex[round, length=3.6mm, width=2.4mm]}]
	(0, 0) -- (.4*4, 0) ;

\path [use tangent=2]
	(0, 0) -- (0, -1)
	node [color=blue!50!black, pos=0.48, above, shape=circle, fill=white, outer sep=0pt, inner sep=0.25pt]
		{$q^{\hspace{.1ex}3}$} ;

%%\fill [fill=blue, use tangent=1] (0, 0) circle (1mm);

\draw	[ line width=.8pt
	, color=red!50!black
	%%, show curve controls
	]
	(-12, 5)
	.. controls ++(33: 1.6) and ++(111: 5) ..
	(0, 0);

\draw	[ line width=.8pt
	, color=red!50!black
	%%, show curve controls
	, tangent=0
	, tangent=0.5
	]
	(0, 0)
	.. controls ++(111: -5) and ++(177: 2) ..
	(8, -5) ;

\path [use tangent=1]
	(0, 0) -- (.4*5, 0)
	node [color=red, pos=0.86, below left, shape=circle, fill=white, outer sep=4pt, inner sep=1pt]
		{$\bm{r}_1$} ;

\draw [line width=1.25pt, color=red, use tangent=1, -{Latex[round, length=3.6mm, width=2.4mm]}]
	(0, 0) -- (.4*5, 0);

\path [use tangent=2]
	(0, 0) -- (0, 1)
	node [color=red!50!black, pos=0.16, above, shape=circle, fill=white, outer sep=0pt, inner sep=0.25pt]
		{$q^{1}$} ;

%%\fill [fill=red, use tangent=1] (0, 0) circle (1mm);

\draw	[ line width=.8pt
	, color=green!50!black
	%%, show curve controls
	]
	(12, 10)
	.. controls ++(88: -4) and ++(11: 6) ..
	(0, 0) ;

\draw	[ line width=.8pt
	, color=green!50!black
	%%, show curve controls
	, tangent=0
	, tangent=0.36
	]
	(0, 0)
	.. controls ++(11: -6) and ++(99: -4) ..
	(-12, 4) ;

\path [use tangent=1]
	(0, 0) -- (.4*6, 0)
	node [color=green, pos=0.92, below right, shape=circle, fill=white, outer sep=5pt, inner sep=1pt]
		{$\bm{r}_2$} ;

\draw [line width=1.25pt, color=green, use tangent=1, -{Latex[round, length=3.6mm, width=2.4mm]}]
	(0, 0) -- (.4*6, 0);

\path [use tangent=2]
	(0, 0) -- (0, -1)
	node [color=green!50!black, pos=0.12, above, shape=circle, fill=white, outer sep=0pt, inner sep=0.25pt]
		{$q^{\hspace{.1ex}2}$} ;

%%\fill [fill=green, use tangent=1] (0, 0) circle (1mm);

\coordinate (theOrigin) at (5, -2) ;
\path (0, 0) circle (1mm) node [shape=circle, inner sep=.5mm, outer sep=0] (theCircleOfO) {} ;

\draw [line width=1.5pt, black, -{Stealth[round,length=4mm,width=2.8mm]}] (theOrigin) -- (theCircleOfO)
		node [pos=0.64, above right, shape=circle, fill=white, outer sep=2pt, inner sep=1.2pt]
			{$\bm{r}$} ;

\draw [line width=1.2pt, color=black, fill=white] (0, 0) circle (1ex);

\draw [line width=1.2pt, color=black, fill=white] (theOrigin) circle (1ex);

\end{tikzpicture}

\vspace{0.1em}\caption{}\label{fig:curvilinearcoordinates}
\end{minipage}}
\end{wrapfigure}

% ~ ~ ~ ~ ~

\begin{otherlanguage}{russian}

Не~только для~решения прикладных задач, но нередко и в~\inquotes{чистой тео\-рии} вместо аргумента~$\locationvector$ ис\-поль\-зу\-ет\-ся набор (какая-либо трой\-ка) криво\-линей\-ных координат~${q^{\hspace{.1ex}i}\hspace{-0.2ex}}$.
Если непрерывно менять лишь одну координату из~трёх, получается координатная линия.
Каждая точка трёхмерного пространства лежит на~пересечении трёх координатных линий (\figref{fig:curvilinearcoordinates}).
Вектор положения точки выражается через набор координат \en{as}\ru{как} \en{relation}\ru{отношение} ${\locationvector \hspace{-0.4ex} = \hspace{-0.4ex} \locationvector(q^{\hspace{.1ex}i}\hspace{.1ex})}$.

\end{otherlanguage}

Commonly used \en{sets of~coordinates}\ru{наборы координат}<<<<<<
\en{Rectangular}\ru{Прямоугольные} (\inquotes{\en{cartesian}\ru{декартовы}}), \en{spherical}\ru{сферические} \en{and }\ru{и~}\en{cylindrical}\ru{цилиндрические} \en{coordinates}\ru{координаты}\en{ are}\ru{\:---}

Curvilinear coordinates may be derived from a~set of~rectangular~(\inquotes{cartesian}) coordinates by using a~transformation that is locally invertible (a~one-to-one map) at~each point.
\en{Therefore}\ru{Поэтому} \en{rectangular coordinates}\ru{прямоугольные координаты} \en{of~any point}\ru{любой точки} \en{of~space}\ru{пространства} \en{can be converted}\ru{могут быть преобразованы} \en{to }\ru{в~}\en{some}\ru{какие-либо} \en{curvilinear coordinates}\ru{криволинейные координаты} \en{and}\ru{и}~\en{vice versa}\ru{обратно}.

...

The~differential of a~function presents a~change in the~linearization of this function.

...

\en{partial derivative}\ru{частная производная}

\nopagebreak\vspace{-0.4em}\begin{equation*}
\partial_i \equiv \scalebox{0.9}{$ \displaystyle\frac{\raisebox{-0.2em}{$\partial$}}{\raisebox{-0.1em}{$\partial q^i$}} $}
\end{equation*}

...

\en{differential}\ru{дифференциал} \en{of~}${\varsigma(q^i)}$

\nopagebreak\vspace{-0.4em}\begin{equation}
d\varsigma \hspace{-0.1ex}
= \scalebox{0.9}{$ \displaystyle\frac{\raisebox{-0.2em}{$\partial \hspace{.15ex} \varsigma$}}{\raisebox{-0.1em}{$\partial q^i$}} $} \hspace{.2ex} dq^i \hspace{-0.2ex}
= \partial_i \varsigma \hspace{.15ex} dq^i
\end{equation}

...

\en{Linearity}\ru{Линейность}

\nopagebreak\vspace{-0.4em}\begin{equation}\label{linearityordifferentiation}
\partial_i \bigl( \lambda \hspace{.1ex} p + \hspace{-0.2ex} \mu \hspace{.1ex} q \hspace{.1ex} \bigr) \hspace{-0.2ex}
= \lambda \bigl( \partial_i \hspace{.1ex} p \hspace{.1ex} \bigr) \hspace{-0.2ex} + \hspace{.1ex}
\mu \bigl( \partial_i \hspace{.1ex} q \hspace{.1ex} \bigr)
\end{equation}

\inquotes{Product rule}

\nopagebreak\vspace{-0.4em}\begin{equation}\label{productrulefordifferentiation}
\partial_i \bigl( \hspace{.15ex} p \circ q \hspace{.1ex} \bigr) \hspace{-0.2ex}
= \hspace{-0.2ex} \bigl( \partial_i \hspace{.1ex} p \hspace{.1ex} \bigr) \hspace{-0.25ex} \circ q \hspace{.12ex} +
\hspace{.1ex} p \circ \hspace{-0.25ex} \bigl( \partial_i \hspace{.1ex} q \hspace{.1ex} \bigr)
\end{equation}

...

Local basis ${\locationvector_\differentialindex{i}}$

\en{The~differential}\ru{Дифференциал} \en{of~location vector}\ru{вектора положения}~${\locationvector(q^{\hspace{.1ex}i}\hspace{.1ex})}$ \en{is}\ru{есть}

\nopagebreak\vspace{-0.2em}\begin{equation}\label{differentialoflocationvector}
d\locationvector
=
\scalebox{0.9}{$ \displaystyle\frac{\raisebox{-0.2em}{$ \partial \hspace{.15ex} \locationvector $}}{\partial q^{\hspace{.1ex}i}} $} \hspace{.2ex} dq^i \hspace{-0.1ex}
=
dq^i \locationvector_\differentialindex{i}
\hspace{.1ex} , \hspace{.5em}
\locationvector_\differentialindex{i} \hspace{-0.1ex} \equiv \scalebox{0.9}{$ \displaystyle\frac{\raisebox{-0.2em}{$ \partial \hspace{.15ex} \locationvector $}}{\partial q^{\hspace{.1ex}i}} $} \hspace{-0.1ex}
\equiv \partial_i \hspace{.1ex} \locationvector
\end{equation}

...

Local cobasis ${\locationvector^i}$, ${\locationvector^i \hspace{-0.32ex} \dotp \locationvector_\differentialindex{\hspace{-0.1ex}j} \hspace{-0.22ex} = \delta_{\hspace{-0.15ex}j}^{\hspace{.2ex}i}}$

...

\begin{equation*}
\displaystyle\frac{\raisebox{-0.2em}{$\partial \hspace{.15ex} \varsigma$}}{\partial \locationvector}
=
\displaystyle\frac{\raisebox{-0.2em}{$\partial \hspace{.15ex} \varsigma$}}{\raisebox{-0.1em}{$\partial q^i$}} \hspace{.1ex} \locationvector^i \hspace{-0.25ex}
=
\partial_i \varsigma \hspace{.2ex} \locationvector^i
\end{equation*}

\begin{equation}
d \varsigma \hspace{-0.1ex}
=
\scalebox{0.9}{$ \displaystyle\frac{\raisebox{-0.2em}{$\partial \hspace{.15ex} \varsigma$}}{\partial \locationvector} $} \dotp d\locationvector \hspace{-0.1ex}
=
\partial_i \varsigma \hspace{.2ex} \locationvector^i \hspace{-0.15ex} \dotp dq^{\hspace{.12ex}j} \hspace{-0.1ex} \locationvector_\differentialindex{\hspace{-0.1ex}j} \hspace{-0.2ex}
=
\partial_i \varsigma \hspace{.15ex} dq^i
\end{equation}

...

\en{The bivalent unit tensor}\ru{Бивалентный единичный тензор}~(\en{metric tensor}\ru{метрический тензор})~${\hspace{-0.1ex}\UnitDyad}$,
\en{which}\ru{который} \en{is neutral}\ru{нейтрален}~\eqref{definingpropertyofidentitytensor} \en{to the } \hbox{\hspace{-0.2ex}\inquotes{${\dotp\hspace{.22ex}}$}\hspace{-0.2ex}}-\en{product}\ru{произведению} (dot product\ru{’у}),
\en{can be represented as}\ru{может быть представлен как}

\nopagebreak\vspace{-0.1em}\begin{equation}
\UnitDyad
= \locationvector^i \locationvector_\differentialindex{i} \hspace{-0.15ex}
= \tikzmark{beginOriginOfNabla} \locationvector^i \partial_i \tikzmark{endOriginOfNabla} \hspace{.1ex} \locationvector = \hspace{-0.16ex} \boldnabla \locationvector ,
\end{equation}
\AddUnderBrace[line width=.75pt][0,-0.1ex]%
{beginOriginOfNabla}{endOriginOfNabla}%
{${\scriptstyle \boldnabla}$}

\vspace{-0.4em}\noindent
\en{where appears}\ru{где появляется} \en{the~}\en{differential}\ru{дифференциальный} \ru{оператор }\inquotes{\en{nabla}\ru{набла}}\en{ operator}

\nopagebreak\vspace{-0.2em}\begin{equation}
\boldnabla \equiv \locationvector^i \partial_i \hspace{.1ex} .
\end{equation}

...

\begin{equation}
d \varsigma \hspace{-0.1ex}
=
\scalebox{0.9}{$ \displaystyle\frac{\raisebox{-0.2em}{$\partial \hspace{.15ex} \varsigma$}}{\raisebox{-0.05em}{$\partial \locationvector$}} $} \dotp d\locationvector \hspace{-0.1ex}
=
d\locationvector \dotp \hspace{-0.11ex} \boldnabla \varsigma \hspace{-0.1ex}
=
\partial_i \varsigma \hspace{.15ex} dq^i
\end{equation}

\vspace{1.1em}${
d\locationvector = d\locationvector \dotp \hspace{-0.2ex} \tikzmark{beginItsUnitTensorE} \boldnabla \locationvector \tikzmark{endItsUnitTensorE}
}$%
\AddOverBrace[line width=.75pt][0,0.1ex]{beginItsUnitTensorE}{endItsUnitTensorE}{${\scriptstyle \UnitDyad}$}

...

\en{Divergence}\ru{Дивергенция} \en{of~the~dyadic product}\ru{диадного произведения} \en{of~two vectors}\ru{двух векторов}

\nopagebreak\vspace{-0.3em}\begin{multline}\label{divergenceofdyadicproducoftwovectors}
\boldnabla \hspace{-0.16ex} \dotp \hspace{-0.2ex} \bigl( \hspace{-0.1ex} \bm{a} \bm{b} \hspace{.05ex} \bigr) \hspace{-0.33ex}
= \locationvector^i \partial_i \hspace{-0.1ex} \dotp \hspace{-0.24ex} \bigl( \hspace{-0.1ex} \bm{a} \bm{b} \bigr) \hspace{-0.33ex}
= \locationvector^i \hspace{-0.3ex} \dotp \partial_i \bigl( \hspace{-0.1ex} \bm{a} \bm{b} \bigr) \hspace{-0.3ex}
= \locationvector^i \hspace{-0.3ex} \dotp \hspace{-0.15ex} \bigl( \partial_i \bm{a} \bigr) \bm{b} \hspace{.1ex} + \locationvector^i \hspace{-0.3ex} \dotp \bm{a} \hspace{.1ex} \bigl( \partial_i \bm{b} \bigr) \hspace{-0.33ex} =
\\[-0.1em]
%
= \hspace{-0.15ex} \bigl( \locationvector^i \hspace{-0.3ex} \dotp \partial_i \bm{a} \bigr) \bm{b} \hspace{.1ex} + \bm{a} \dotp \locationvector^i \hspace{-0.15ex} \bigl( \partial_i \bm{b} \bigr) \hspace{-0.33ex}
= \hspace{-0.15ex} \bigl( \locationvector^i \partial_i \hspace{-0.1ex} \dotp \bm{a} \bigr) \bm{b} \hspace{.1ex} + \bm{a} \dotp \hspace{-0.1ex} \bigl( \locationvector^i \partial_i \bm{b} \bigr) \hspace{-0.33ex} =
\\
%
= \hspace{-0.15ex} \bigl( \boldnabla \hspace{-0.15ex} \dotp \hspace{-0.1ex} \bm{a} \bigr) \bm{b} \hspace{.1ex} + \bm{a} \dotp \hspace{-0.12ex} \bigl( \boldnabla \hspace{.1ex} \bm{b} \bigr)
\end{multline}

\vspace{-0.2em}\noindent
--- \en{here’s no~need}\ru{тут нет нужды} \en{to~expand}\ru{разворачивать} \en{vectors}\ru{векторы}~$\bm{a}$ \en{and}\ru{и}~$\bm{b}$, \en{expanding just}\ru{развернув лишь} \en{differential operator}\ru{дифференциальный оператор}~${\hspace{-0.13ex}\boldnabla}$.

...

\en{Gradient of cross product of two vectors}\ru{Градиент векторного произведения двух векторов},
\en{applying}\ru{применяя} \inquotes{product rule}~\eqref{productrulefordifferentiation}
\en{and}\ru{и}~\en{relation}\ru{соотношение}~\eqref{crossproductoftwovectors} \en{for any two vectors}\ru{для любых двух векторов}
(\en{partial derivative}\ru{частная производная}~$\partial_i$ \en{of~some~vector by scalar coordinate}\ru{некоторого вектора по скалярной координате}~$q^i\hspace{-0.1ex}$ \en{is a~vector too}\ru{это тоже вектор})

\nopagebreak\vspace{-0.4em}\begin{multline}\label{gradientofcrossproductoftwovectors}
\boldnabla \hspace{-0.2ex} \left( \bm{a} \hspace{-0.1ex} \times \hspace{-0.1ex} \bm{b} \right) \hspace{-0.2ex}
= \hspace{.1ex} \locationvector^i \partial_i \hspace{-0.3ex} \left( \bm{a} \hspace{-0.2ex} \times \hspace{-0.2ex} \bm{b} \right) \hspace{-0.2ex}
= \locationvector^i \hspace{-0.4ex} \left( \partial_i \bm{a} \hspace{-0.2ex} \times \hspace{-0.2ex} \bm{b} \hspace{.1ex} +
\bm{a} \hspace{-0.2ex} \times \hspace{-0.2ex} \partial_i \bm{b} \right) \hspace{-0.2ex} =
\\[-0.1em]
%
= \locationvector^i \hspace{-0.4ex} \left( \partial_i \bm{a} \hspace{-0.2ex} \times \hspace{-0.2ex} \bm{b} \hspace{.1ex} -
\partial_i \bm{b} \hspace{-0.2ex} \times \hspace{-0.2ex} \bm{a} \right) \hspace{-0.2ex}
= \hspace{.1ex} \locationvector^i \partial_i \hspace{.1ex} \bm{a} \hspace{-0.2ex} \times \hspace{-0.2ex} \bm{b} \hspace{.1ex} - \hspace{.1ex}
\locationvector^i \partial_i \hspace{.1ex} \bm{b} \hspace{-0.2ex} \times \hspace{-0.2ex} \bm{a} =
\\[-0.1em]
%
= \hspace{-0.12ex} \boldnabla \bm{a} \hspace{-0.1ex} \times \hspace{-0.1ex} \bm{b} \hspace{.12ex} - \hspace{-0.12ex}
\boldnabla \hspace{.1ex} \bm{b} \hspace{-0.1ex} \times \hspace{-0.1ex} \bm{a}
\hspace{.2ex} .
\end{multline}

...

\en{Gradient}\ru{Градиент} \en{of }dot product\ru{’а} \en{of two vectors}\ru{двух векторов}

\nopagebreak\vspace{-0.4em}\begin{multline}\label{gradientofdotproductoftwovectors}
\boldnabla \hspace{.1ex} \bigl( \hspace{-0.05ex} \bm{a} \hspace{-0.1ex} \dotp \hspace{-0.1ex} \bm{b} \hspace{.05ex} \bigr) \hspace{-0.3ex}
= \hspace{.1ex} \locationvector^i \partial_i \bigl( \hspace{-0.05ex} \bm{a} \hspace{-0.1ex} \dotp \hspace{-0.1ex} \bm{b} \hspace{.05ex} \bigr) \hspace{-0.33ex}
= \hspace{.1ex} \locationvector^i \bigl( \partial_i \bm{a} \bigr) \hspace{-0.32ex} \dotp \bm{b} + \hspace{.1ex} \locationvector^i \bm{a} \hspace{-0.05ex} \dotp \hspace{-0.15ex} \bigl( \partial_i \bm{b} \bigr) \hspace{-0.33ex} =
\\[-0.1em]
%
= \hspace{-0.2ex} \bigl( \locationvector^i \partial_i \bm{a} \bigr) \hspace{-0.32ex} \dotp \bm{b} \hspace{.1ex} + \hspace{.1ex} \locationvector^i \bigl( \partial_i \bm{b} \bigr) \hspace{-0.33ex} \dotp \bm{a}
= \hspace{-0.16ex} \bigl( \boldnabla \hspace{-0.1ex} \bm{a} \bigr) \hspace{-0.3ex} \dotp \hspace{.1ex} \bm{b} \hspace{.1ex} + \hspace{-0.1ex} \bigl( \boldnabla \hspace{.1ex} \bm{b} \bigr) \hspace{-0.27ex} \dotp \hspace{.1ex} \bm{a}
\hspace{.2ex} .
\end{multline}

\newpage ...

\newpage ...




\en{\section{Integral theorems}}

\ru{\section{Интегральные теоремы}}

\begin{otherlanguage}{russian}

Для векторных полей известны интегральные теоремы Gauss’а и~Stokes’а.

\noindent\leavevmode{\indent}{\small Gauss’ theorem (divergence theorem) enables an~integral taken over a~volume to be replaced by one taken over the closed surface bounding that volume, and vice versa.\par}

\noindent\leavevmode{\indent}{\small Stokes’ theorem enables an~integral taken around a closed curve to be replaced by one taken over \emph{any} surface bounded by that curve. Stokes’ theorem relates a~line integral around a closed path to a surface integral over what is called a~\emph{capping surface} of the path.\par}

Теорема Гаусса о~дивергенции\:--- про~то, как заменить объёмный интеграл поверхностным~(\en{and vice versa}\ru{и~наоборот}). В~этой теореме рассматривается поток (ef)flux вектора через ограничивающую объём~$V$ з\'{а}мкнутую поверхность ${\mathcal{O}(\boundary V)}$ с~единичным вектором внешней нормали~$\bm{n}$

\nopagebreak\vspace{-0.1em}\begin{equation}
\ointegral\displaylimits_{\mathclap{\mathcal{O}(\boundary V)}} \hspace{-0.1ex} \bm{n} \dotp \bm{a} \hspace{.4ex} d\mathcal{O} \hspace{.12ex} = \integral\displaylimits_{V} \hspace{-0.3ex} \boldnabla \hspace{-0.12ex} \dotp \bm{a} \hspace{.4ex} dV \hspace{-0.25ex}.
\end{equation}

Объём~$V$ нарезается тремя семействами координатных поверхностей на~множество бесконечно малых элементов. Поток через поверхность ${\mathcal{O}(\boundary V)}$ равен сумме потоков через края получившихся элементов. В~бесконечной малости каждый такой элемент\:--- маленький локальный дифференциальный кубик~(параллелепипед). ... Поток вектора~$\bm{a}$ через грани малого кубика объёма~$dV$ есть ${\sum_{i = 1}^{6} \bm{n}_i \dotp \bm{a} \hspace{.2ex} \mathcal{O}_i}$, а~через сам этот объём поток равен ${\boldnabla \dotp \bm{a} \hspace{.32ex} dV}$.

Похожая трактовка этой теоремы есть, к примеру, в~курсе Richard’а Feynman’а~\cite{feynman-lecturesonphysics}.

\emph{( рисунок с кубиками )}

to dice\:--- нарез\'{а}ть кубиками

small cube, little cube

локально ортонормальные координаты ${\bm{\xi} = \xi_i \hspace{.2ex} \bm{n}_i \hspace{.1ex}}$, ${d\bm{\xi} = d \xi_i \hspace{.2ex} \bm{n}_i}$, ${\boldnabla = \bm{n}_i \partial_i}$

разложение вектора ${\bm{a} = a_i \bm{n}_i \hspace{.1ex}}$

Теорема Стокса о~циркуляции выражается равенством

...

\newpage ...



\end{otherlanguage}



\newpage

\en{\section{Curvature tensors}}

\ru{\section{Тензоры кривизны}}

\label{para:curvaturetensors}

\begin{changemargin}{2\parindent}{\parindent}
\bgroup % to change \parindent locally
\setlength{\parindent}{\negparindent}
\setlength{\parskip}{\spacebetweenparagraphs}
\small

\leavevmode{\indent}The~\href{https://en.wikipedia.org/wiki/Riemann_curvature_tensor}{\emph{Riemann curvature tensor} or \emph{Riemann\hbox{--}Christoffel tensor}} (after \href{https://en.wikipedia.org/wiki/Bernhard_Riemann}{\textbold{Bernhard Riemann}} and \href{https://en.wikipedia.org/wiki/Elwin_Bruno_Christoffel}{\textbold{Elwin Bruno Christoffel}}) is the most common method used to express the curvature of Riemannian manifolds. It’s a~tensor field, it assigns a~tensor to each point of a~Riemannian manifold, that measures the extent to which the~metric tensor is not locally isometric to that of \inquotes{flat} space. The curvature tensor measures noncommutativity of the covariant derivative, and as such is the~integrability obstruction for the~existence of an~isometry with \inquotes{flat} space.

%%%\vspace{.2em}
%%%\hfill $\sim$\:\emph{from Wikipedia, the free encyclopedia}
\par
\egroup
\nopagebreak\vspace{.12em}
\end{changemargin}

\begin{otherlanguage}{russian}

\noindent
Рассматривая тензорные поля в~криволинейных координатах~(\pararef{para:spatialdifferentiationoftensorfields}), мы исходили из~представления вектора\hbox{-}радиуса~(вектора положения) точки функцией этих координат:
${\locationvector \hspace{-0.4ex} = \hspace{-0.4ex} \locationvector(q^{\hspace{.1ex}i}\hspace{.1ex})}$.
Этим отношением порождаются выражения

\nopagebreak\begin{itemize}
\item векторов локального касательного базиса ${%
\locationvector_\differentialindex{i} \hspace{-0.16ex} \equiv \smash{ \raisemath{.16em}{\scalebox{0.8}{$ \partial \hspace{.1ex} \locationvector $}} \hspace{-0.3ex} / \hspace{-0.4ex} \raisemath{-0.32em}{\scalebox{0.8}{$ \partial q^{\hspace{.1ex}i} $}} } \hspace{-0.15ex} \equiv \partial_i \hspace{.1ex} \locationvector%
}$,
%
\item компонент ${\textsl{g}_{i\hspace{-0.1ex}j} \hspace{-0.24ex} \equiv \locationvector_\differentialindex{i} \hspace{-0.16ex} \dotp \locationvector_\differentialindex{\hspace{-0.1ex}j}}$ и~${\textsl{g}^{\hspace{.25ex}i\hspace{-0.1ex}j} \hspace{-0.32ex} \equiv \locationvector^i \hspace{-0.32ex} \dotp \locationvector^j \hspace{-0.32ex} = \smash{\textsl{g}_{i\hspace{-0.1ex}j}^{\hspace{.33ex}\expminusone}}}$ единичного \inquotes{метрического} тензора~${\UnitDyad = \locationvector_\differentialindex{i} \locationvector^i \hspace{-0.2ex} = \locationvector^i \locationvector_\differentialindex{i} \hspace{-0.15ex} = \textsl{g}_{j\hspace{-0.1ex}k} \hspace{.1ex} \locationvector^{\hspace{.1ex}j} \hspace{-0.1ex} \locationvector^{k} \hspace{-0.25ex} = \textsl{g}^{\hspace{.25ex}j\hspace{-0.1ex}k} \hspace{.1ex} \locationvector_\differentialindex{\hspace{-0.1ex}j} \locationvector_\differentialindex{k}}$,
%
\item векторов локального взаимного кокасательного базиса ${\locationvector^i \hspace{-0.32ex} \dotp \locationvector_\differentialindex{\hspace{-0.1ex}j} \hspace{-0.22ex} = \delta_{\hspace{-0.15ex}j}^{\hspace{.2ex}i}}$, ${\locationvector^i \hspace{-0.25ex} = \textsl{g}^{\hspace{.25ex}i\hspace{-0.1ex}j} \locationvector_\differentialindex{\hspace{-0.1ex}j}}$,
%
\item диф\-ферен\-циаль\-ного набла\hbox{-}оператора Hamilton’а ${\smash{\boldnabla \equiv \locationvector^i \partial_i}}$,
${\UnitDyad = \hspace{-0.25ex} \smash{\boldnabla \locationvector}}$,
%
\item полного дифференциала ${d \bm{\xi} = d \locationvector \dotp \hspace{-0.2ex} \boldnabla \hspace{-0.05ex} \bm{\xi} \hspace{.1ex}}$,
%
\item частных производных касательного \hbox{базиса} (вторых частных производных~$\locationvector$) ${\locationvector_{\differentialindex{i}\hspace{.2ex}\differentialindex{\hspace{-0.1ex}j}} \hspace{-0.2ex} \equiv \partial_i \partial_j \locationvector \hspace{-0.1ex} = \partial_i \hspace{.12ex} \locationvector_\differentialindex{\hspace{-0.1ex}j}}$,
%
\item символов \inquotes{связности} \hbox{Христоффеля}~(\hbox{Christoffel} symbols) ${\Gamma_{\hspace{-0.25ex}i\hspace{-0.1ex}j}^{\hspace{.25ex}k} \hspace{-0.1ex} \equiv \locationvector_{\differentialindex{i}\hspace{.2ex} \differentialindex{\hspace{-0.1ex}j}} \hspace{-0.2ex} \dotp \locationvector^k
%%\hspace{-0.32ex} = \Gamma_{\hspace{-0.25ex}i\hspace{-0.1ex}j\mathdotbelow{n}} \hspace{.25ex} \textsl{g}^{\hspace{.25ex}nk}\hspace{-0.25ex}
}$ и~${\Gamma_{\hspace{-0.25ex}i\hspace{-0.1ex}j\mathdotbelow{k}} \hspace{-0.16ex} \equiv \locationvector_{\differentialindex{i}\hspace{.2ex}\differentialindex{\hspace{-0.1ex}j}} \hspace{-0.2ex} \dotp \locationvector_\differentialindex{k}
%%\hspace{-0.2ex} = \Gamma_{\hspace{-0.25ex}i\hspace{-0.1ex}j}^{\hspace{.25ex}n} \hspace{.16ex} \textsl{g}_{nk}
}$.
\vspace{-0.2em}
\end{itemize}

Представим теперь, что функция~${\locationvector(q^{\hspace{.1ex}k})}$ не~известна, но~\hbox{зат\'{о}} в~каждой точке пространства известны шесть независимых компонент положительно определённой (\en{all}\ru{все} \ru{матрицы }Gram\ru{’а}\en{ matrices} \en{are non-negative definite}\ru{определены неотрицательно}) симметричной метрической матрицы Gram\ru{’а}~${\textsl{g}_{i\hspace{-0.1ex}j}(q^{\hspace{.1ex}k})}$.

the Gram matrix (or Gramian)

Билинейная форма ...

\nopagebreak
...

Поскольку шесть функций~${\textsl{g}_{i\hspace{-0.1ex}j}(q^{\hspace{.1ex}k})}$ происходят от векторной функции~${\locationvector(q^{\hspace{.1ex}k})}$, то между элементами~$\textsl{g}_{i\hspace{-0.1ex}j}$ существуют некие соотношения.

\en{Differential}\ru{Дифференциал} ${d\locationvector}$\;\eqref{differentialoflocationvector}\en{ is}\ru{\:---} \en{exact}\ru{полный~(точный)}.
\en{This is true}\ru{Это истинно} \en{if and only if}\ru{тогда и только тогда, когда} \en{second partial derivatives}\ru{вторые частные производные} \en{commute}\ru{коммутируют}:

\nopagebreak\vspace{-0.2em}\begin{equation*}
d\locationvector \hspace{-0.1ex} = \locationvector_\differentialindex{k} \hspace{.2ex} dq^{\hspace{.1ex}k}
\hspace{.4em}\Leftrightarrow\hspace{.44em}
%%\partial_i \bigl( \partial_j \locationvector \bigr) \hspace{-0.2ex} = \hspace{.1ex} \partial_j \bigl( \partial_i \locationvector \bigr) \hspace{-0.2ex}
%%\hspace{.5em}\text{\en{or}\ru{или}}\hspace{.5em}
\partial_i \hspace{.12ex} \locationvector_\differentialindex{\hspace{-0.1ex}j} \hspace{-0.22ex} = \partial_j \locationvector_\differentialindex{i}
\hspace{.5em}\text{\en{or}\ru{или}}\hspace{.5em}
\locationvector_{\differentialindex{i}\hspace{.2ex}\differentialindex{\hspace{-0.1ex}j}} \hspace{-0.22ex} = \locationvector_{\differentialindex{\hspace{-0.1ex}j}\hspace{.2ex}\differentialindex{i}}
\hspace{.1ex} .
\end{equation*}

\vspace{-0.2em}\noindent
Но это условие уж\'{е} обеспечено симметрией~${\textsl{g}_{i\hspace{-0.1ex}j}}$

...

\en{metric}\ru{метрическая} (\inquotes{\en{affine}\ru{аффинная}}) \en{connection}\ru{связность}~$\nabla_{\hspace{-0.32ex}i\hspace{.1ex}}$, её~же называют \inquotes{\en{covariant derivative}\ru{ковариантная производная}}

\vspace{1.2em}\begin{equation*}
\locationvector_{\differentialindex{i}\hspace{.2ex}\differentialindex{\hspace{-0.1ex}j}} \hspace{-0.1ex} = \hspace{.1ex}
\tikzmark{beginChristoffelSymbolOne} \locationvector_{\differentialindex{i}\hspace{.2ex}\differentialindex{\hspace{-0.1ex}j}} \hspace{-0.15ex} \dotp \tikzmark{beginEtensorUpDown} \locationvector^{k} \hspace{-0.4ex} \tikzmark{endChristoffelSymbolOne} \hspace{.4ex} \locationvector_\differentialindex{k} \tikzmark{endEtensorUpDown} \hspace{-0.2ex}
= \tikzmark{beginChristoffelSymbolOther} \locationvector_{\differentialindex{i}\hspace{.2ex}\differentialindex{\hspace{-0.1ex}j}} \hspace{-0.15ex} \dotp \tikzmark{beginEtensorDownUp} \locationvector_\differentialindex{k} \tikzmark{endChristoffelSymbolOther} \locationvector^k \tikzmark{endEtensorDownUp}
\end{equation*}%
\AddOverBrace[line width=.75pt][0,0.2ex][yshift=-0.1em]{beginEtensorUpDown}{endEtensorUpDown}{${\scriptstyle \UnitDyad}$}%
\AddOverBrace[line width=.75pt][0,0.2ex][yshift=-0.1em]{beginEtensorDownUp}{endEtensorDownUp}{${\scriptstyle \UnitDyad}$}%
\AddUnderBrace[line width=.75pt][0,-0.1ex]{beginChristoffelSymbolOne}{endChristoffelSymbolOne}{${\scriptstyle \Gamma_{\hspace{-0.25ex}i\hspace{-0.1ex}j}^{\hspace{.25ex}k}}$}%
\AddUnderBrace[line width=.75pt][0,-0.1ex]{beginChristoffelSymbolOther}{endChristoffelSymbolOther}{${\scriptstyle \Gamma_{\hspace{-0.25ex}i\hspace{-0.1ex}j\mathdotbelow{k}}}$}

${
\Gamma_{\hspace{-0.25ex}i\hspace{-0.1ex}j}^{\hspace{.25ex}k} \hspace{.2ex} \locationvector_\differentialindex{k} \hspace{-0.2ex} = \locationvector_{\differentialindex{i}\hspace{.2ex}\differentialindex{\hspace{-0.1ex}j}} \hspace{-0.16ex} \dotp \hspace{.1ex} \locationvector^k \locationvector_\differentialindex{k} \hspace{-0.2ex} = \locationvector_{\differentialindex{i}\hspace{.2ex}\differentialindex{\hspace{-0.1ex}j}}
}$

covariant derivative (affine connection) is only defined for vector fields

${
\boldnabla \bm{v} \hspace{-0.15ex}
= \locationvector^{i} \partial_i \hspace{-0.33ex} \left( v^{\hspace{.12ex}j} \locationvector_\differentialindex{\hspace{-0.1ex}j} \right) \hspace{-0.25ex}
= \locationvector^{i} \hspace{-0.4ex} \left( \partial_i v^{\hspace{.12ex}j} \locationvector_\differentialindex{\hspace{-0.1ex}j} \hspace{-0.12ex} + v^{\hspace{.12ex}j} \locationvector_{\differentialindex{i}\hspace{.2ex}\differentialindex{\hspace{-0.1ex}j}} \right)
}$

${
\boldnabla \bm{v} \hspace{-0.15ex}
= \locationvector^{i} \locationvector_\differentialindex{\hspace{-0.1ex}j} \nabla_{\hspace{-0.32ex}i\hspace{.1ex}} v^{\hspace{.12ex}j} \hspace{-0.3ex} , \:\:
\nabla_{\hspace{-0.32ex}i\hspace{.1ex}} v^{\hspace{.12ex}j} \hspace{-0.3ex} \equiv
\partial_i v^{\hspace{.12ex}j} \hspace{-0.33ex} + \Gamma_{\hspace{-0.25ex}in}^{\hspace{.25ex}j} v^{\hspace{.1ex}n}
}$

${
\boldnabla \locationvector_\differentialindex{i} \hspace{-0.2ex}
= \locationvector^k \partial_k \locationvector_\differentialindex{i} \hspace{-0.2ex}
= \locationvector^k \locationvector_{\differentialindex{k}\hspace{.2ex}\differentialindex{i}} \hspace{-0.2ex}
%%= \locationvector^k \hspace{.2ex} \Gamma_{\hspace{-0.25ex}ki}^{\hspace{.25ex}n} \hspace{.2ex} \locationvector_\differentialindex{n} \hspace{-0.2ex}
= \locationvector^k \locationvector_\differentialindex{n} \hspace{.1ex} \Gamma_{\hspace{-0.25ex}ki}^{\hspace{.25ex}n}
\hspace{.2ex} , \:\:
\nabla_{\hspace{-0.32ex}i\hspace{.1ex}} \locationvector_\differentialindex{n} \hspace{-0.25ex}
= \Gamma_{\hspace{-0.25ex}in}^{\hspace{.25ex}k} \hspace{.16ex} \locationvector_\differentialindex{k}
}$

\vspace{.2em}
Christoffel symbols describe a~metric (\inquotes{affine}) connection, that is how the~basis changes from point to~point.

символы Christoffel’я это \inquotes{\en{components of~connection}\ru{компоненты связности}} \en{in local coordinates}\ru{в~локальных координатах}

...

\href{https://en.wikipedia.org/wiki/Torsion_tensor}{\en{torsion tensor}\ru{тензор кручения}}~${^3\bm{\mathfrak{T}}}$ \en{with components}\ru{с~компонентами}

\nopagebreak\vspace{-0.1em}\begin{equation*}
\mathfrak{T}^{k}_{i\hspace{-0.1ex}j} \hspace{-0.15ex} = \Gamma_{\hspace{-0.25ex}i\hspace{-0.1ex}j}^{\hspace{.25ex}k} \hspace{-0.1ex} - \Gamma_{\hspace{-0.33ex}j\hspace{-0.06ex}i}^{\hspace{.25ex}k}
\end{equation*}

\noindent
determines the~antisymmetric part of a~connection

...

\noindent
симметрия ${ \Gamma_{\hspace{-0.25ex}i\hspace{-0.1ex}j\mathdotbelow{k}} = \Gamma_{\hspace{-0.33ex}j\hspace{-0.06ex}i\mathdotbelow{k}} }$, поэтому ${3^3 \hspace{-0.2ex} - 3 \hspace{-0.2ex}\cdot\hspace{-0.2ex} 3 = 18}$ разных~(независимых) ${\Gamma_{\hspace{-0.25ex}i\hspace{-0.1ex}j\mathdotbelow{k}}}$

\begin{multline}
\Gamma_{\hspace{-0.25ex}i\hspace{-0.1ex}j}^{\hspace{.25ex}n} \hspace{.16ex} \textsl{g}_{nk} \hspace{-0.24ex} = \Gamma_{\hspace{-0.25ex}i\hspace{-0.1ex}j\mathdotbelow{k}} \hspace{-0.2ex} = \locationvector_{\differentialindex{i}\hspace{.2ex}\differentialindex{\hspace{-0.1ex}j}} \hspace{-0.2ex} \dotp \locationvector_\differentialindex{k} \hspace{-0.1ex} =
\\[-0.1em]
%
= \smallerdisplaystyleonehalf \hspace{-0.1ex} \bigl( \locationvector_{\differentialindex{i}\hspace{.2ex}\differentialindex{\hspace{-0.1ex}j}} \hspace{-0.16ex} + \locationvector_{\differentialindex{\hspace{-0.1ex}j}\hspace{.2ex}\differentialindex{i}} \bigr) \hspace{-0.2ex} \dotp \locationvector_\differentialindex{k} \hspace{-0.1ex}
+ \smallerdisplaystyleonehalf \hspace{-0.1ex} \bigl( \locationvector_{\differentialindex{\hspace{-0.1ex}j}\hspace{.2ex}\differentialindex{k}} \hspace{-0.16ex} - \locationvector_{\differentialindex{k}\hspace{.2ex}\differentialindex{\hspace{-0.1ex}j}} \bigr) \hspace{-0.2ex} \dotp \locationvector_\differentialindex{i} \hspace{-0.1ex}
+ \smallerdisplaystyleonehalf \hspace{-0.1ex} \bigl( \locationvector_{\differentialindex{i}\hspace{.2ex}\differentialindex{k}} \hspace{-0.16ex} - \locationvector_{\differentialindex{k}\hspace{.1ex}\differentialindex{i}} \bigr) \hspace{-0.2ex} \dotp \locationvector_\differentialindex{\hspace{-0.1ex}j} \hspace{-0.1ex} =
\\[-0.1em]
%
= \smallerdisplaystyleonehalf \hspace{-0.1ex} \bigl( \scalebox{0.93}[1]{$
	\locationvector_{\differentialindex{i}\hspace{.2ex}\differentialindex{\hspace{-0.1ex}j}} \hspace{-0.2ex} \dotp \locationvector_\differentialindex{k} \hspace{-0.16ex}
	+ \locationvector_{\differentialindex{i}\hspace{.2ex}\differentialindex{k}} \hspace{-0.2ex} \dotp \locationvector_\differentialindex{\hspace{-0.1ex}j}
$} \bigr) \hspace{-0.16ex}
+ \smallerdisplaystyleonehalf \hspace{-0.1ex} \bigl( \scalebox{0.93}[1]{$
	\locationvector_{\differentialindex{\hspace{-0.1ex}j}\hspace{.2ex}\differentialindex{i}} \hspace{-0.2ex} \dotp \locationvector_\differentialindex{k} \hspace{-0.16ex}
	+ \locationvector_{\differentialindex{\hspace{-0.1ex}j}\hspace{.2ex}\differentialindex{k}} \hspace{-0.2ex} \dotp \locationvector_\differentialindex{i}
$} \bigr) \hspace{-0.16ex}
- \smallerdisplaystyleonehalf \hspace{-0.1ex} \bigl( \scalebox{0.93}[1]{$
	\locationvector_{\differentialindex{k}\hspace{.2ex}\differentialindex{i}} \hspace{-0.2ex} \dotp \locationvector_\differentialindex{\hspace{-0.1ex}j} \hspace{-0.16ex}
	+ \locationvector_{\differentialindex{k}\hspace{.2ex}\differentialindex{\hspace{-0.1ex}j}} \hspace{-0.2ex} \dotp \locationvector_\differentialindex{i}
$} \bigr) \hspace{-0.2ex} =
\\[-0.25em]
%
= \smalldisplaystyleonehalf \hspace{-0.4ex} \left(^{\mathstrut} \hspace{-0.2ex}
\partial_i ( \locationvector_\differentialindex{\hspace{-0.1ex}j} \hspace{-0.2ex} \dotp \locationvector_\differentialindex{k} ) \hspace{-0.16ex}
+ \partial_j ( \locationvector_\differentialindex{i} \hspace{-0.2ex} \dotp \locationvector_\differentialindex{k} ) \hspace{-0.16ex}
- \partial_k ( \locationvector_\differentialindex{i} \hspace{-0.2ex} \dotp \locationvector_\differentialindex{\hspace{-0.1ex}j} )
\hspace{-0.12ex} \right) \hspace{-0.4ex} =
\\[-0.25em]
%
= \smalldisplaystyleonehalf \hspace{-0.3ex} \left(
\partial_i \hspace{.12ex} \textsl{g}_{j\hspace{-0.1ex}k} \hspace{-0.2ex}
+ \partial_j \hspace{.1ex} \textsl{g}_{ik} \hspace{-0.2ex}
- \partial_k \hspace{.12ex} \textsl{g}_{i\hspace{-0.1ex}j}
\right) \hspace{-0.4ex} .
\end{multline}

Все символы Christoffel’я тождественно равны нулю лишь в~ортонормальной~(декартовой) системе.
\textcolor{magenta}{(А~какие они для косоугольной?)}

Дальше:
${d\locationvector_\differentialindex{i} \hspace{-0.2ex}
= d\locationvector \dotp \hspace{-0.2ex} \boldnabla \locationvector_\differentialindex{i} \hspace{-0.2ex}
= dq^{\hspace{.1ex}k} \partial_k \locationvector_\differentialindex{i} \hspace{-0.2ex}
= \locationvector_{\differentialindex{k}\hspace{.2ex}\differentialindex{i}} \hspace{.2ex} dq^{\hspace{.1ex}k}\hspace{-0.25ex}}$\:--- тоже полные дифференциалы.
\[
d\locationvector_\differentialindex{k} \hspace{-0.2ex}
= \partial_i \locationvector_\differentialindex{k} \hspace{.15ex} dq^i \hspace{-0.3ex}
= \scalebox{0.84}{$ \displaystyle\frac{\raisemath{-0.2ex}{\partial \hspace{.1ex} \locationvector_\differentialindex{k}}}{\raisemath{-0.3ex}{\partial q^1}} $} \hspace{.2ex} dq^1 \hspace{-0.2ex}
+ \scalebox{0.84}{$ \displaystyle\frac{\raisemath{-0.2ex}{\partial \hspace{.1ex} \locationvector_\differentialindex{k}}}{\raisemath{-0.3ex}{\partial q^2}} $} \hspace{.2ex} dq^2 \hspace{-0.2ex}
+ \scalebox{0.84}{$ \displaystyle\frac{\raisemath{-0.2ex}{\partial \hspace{.1ex} \locationvector_\differentialindex{k}}}{\raisemath{-0.3ex}{\partial q^3}} $} \hspace{.2ex} dq^3 \hspace{-0.2ex}
\]
Поэтому ${\partial_i \partial_j \locationvector_\differentialindex{k} \hspace{-0.2ex} = \partial_j \partial_i \locationvector_\differentialindex{k}}$, ${\partial_i \locationvector_{\differentialindex{\hspace{-0.1ex}j}\hspace{.2ex}\differentialindex{k}} \hspace{-0.2ex} = \partial_j \locationvector_{\differentialindex{i}\hspace{.2ex}\differentialindex{k}}}$,
и~трёхиндексный объект из~векторов третьих частных производных

\nopagebreak\vspace{-0.25em}
\begin{equation}
\locationvector_{\differentialindex{i}\hspace{.2ex}\differentialindex{\hspace{-0.1ex}j}\hspace{.2ex}\differentialindex{k}} \hspace{-0.1ex} \equiv \hspace{.1ex} \partial_i \partial_j \partial_k \locationvector
= \partial_i \hspace{.12ex} \locationvector_{\differentialindex{\hspace{-0.1ex}j}\hspace{.2ex}\differentialindex{k}}
\end{equation}

\vspace{-0.24em} \noindent
симметричен по~первому и~второму индексам (а~не~только по~второму и~третьему).
И~тогда равен нулю~${\hspace{-0.16ex}^4\bm{0}}$ следующий тензор четвёртой сложности\:---
\href{https://en.wikipedia.org/wiki/Riemann_curvature_tensor}{ \emph{\ru{тензор кривизны }Riemann\ru{’а}\en{ curvature tensor}} (\en{or}\ru{или}~\emph{\ru{тензор }Riemann\ru{’а}\hbox{--}Christoffel\ru{’я}\en{ tensor}}) }
%% Римана\hbox{--}Христоффеля

\nopagebreak\vspace{-0.1em}\begin{equation}\label{riemanncurvaturetensor}
{^4\bm{\mathfrak{R}}} = \hspace{.12ex} \mathfrak{R}_{\hspace{.1ex}hi\hspace{-0.1ex}j\hspace{-0.1ex}k} \hspace{.12ex} \locationvector^h \locationvector^i \locationvector^j \locationvector^k \hspace{-0.25ex},
\:\:
\mathfrak{R}_{\hspace{.1ex}hi\hspace{-0.1ex}j\hspace{-0.1ex}k} \hspace{-0.12ex}
\equiv
\locationvector_\differentialindex{h} \hspace{-0.15ex} \dotp \left( \hspace{.12ex} \locationvector_{\differentialindex{\hspace{-0.1ex}j}\hspace{.2ex}\differentialindex{i}\hspace{.2ex}\differentialindex{k}} \hspace{-0.2ex} - \locationvector_{\differentialindex{i}\hspace{.2ex}\differentialindex{\hspace{-0.1ex}j}\hspace{.2ex}\differentialindex{k}} \hspace{.12ex} \right)
\hspace{-0.3ex} .
\end{equation}

Выразим компоненты~${\mathfrak{R}_{\hspace{.1ex}i\hspace{-0.1ex}j\hspace{-0.1ex}kn}}$ через метрическую матрицу~${\textsl{g}_{i\hspace{-0.1ex}j}}$.
Начнём с~дифференцирования локального кобазиса:

\[
\locationvector^i \hspace{-0.32ex} \dotp \locationvector_\differentialindex{k} \hspace{-0.16ex} = \delta_k^{\hspace{.1ex}i}
\;\Rightarrow\:
\partial_j \locationvector^i \hspace{-0.32ex} \dotp \locationvector_\differentialindex{k} \hspace{-0.15ex} + \locationvector^i \hspace{-0.32ex} \dotp \locationvector_{\differentialindex{\hspace{-0.1ex}j}\hspace{.2ex}\differentialindex{k}} \hspace{-0.15ex} = 0
\;\Rightarrow\:
\partial_j \locationvector^i \hspace{-0.12ex} = - \hspace{.2ex} \Gamma_{\hspace{-0.25ex}j\hspace{-0.1ex}k}^{\hspace{.25ex}i} \hspace{.2ex} \locationvector^k
\hspace{-0.4ex} .
\]

...

\en{The six}\ru{Шесть}
\en{independent components}\ru{независимых компонент}
${\mathfrak{R}_{\hspace{.1ex}1212}}$, ${\mathfrak{R}_{\hspace{.1ex}1213}}$, ${\mathfrak{R}_{\hspace{.1ex}1223}}$, ${\mathfrak{R}_{\hspace{.1ex}1313}}$, ${\mathfrak{R}_{\hspace{.1ex}1323}}$, ${\mathfrak{R}_{\hspace{.1ex}2323}}$.

...

\en{The symmetric}\ru{Симметричный}
\en{bivalent}\ru{бивалентный} \href{https://en.wikipedia.org/wiki/Ricci_curvature}{ \emph{\ru{тензор кривизны }Ricci\en{’s}\en{ curvature tensor} }

\begin{equation*}
\hspace{.1ex}\pmb{\scalebox{1.2}[1]{$\mathscr{R}$}} \equiv
\smalldisplaystyleonefourth \hspace{.4ex} \mathfrak{R}_{\hspace{.1ex}abi\hspace{-0.1ex}j} \hspace{.2ex} \locationvector^a \hspace{-0.33ex} \times \hspace{-0.1ex} \locationvector^b \locationvector^i \hspace{-0.33ex} \times \hspace{-0.1ex} \locationvector^j \hspace{-0.25ex}
= \smalldisplaystyleonefourth \hspace{.15ex} \levicivita^{abp} \levicivita^{i\hspace{-0.1ex}j\hspace{-0.1ex}q} \hspace{.25ex} \mathfrak{R}_{\hspace{.1ex}abi\hspace{-0.1ex}j} \hspace{.2ex} \locationvector_\differentialindex{p} \locationvector_\differentialindex{q} \hspace{-0.2ex}
= \mathscr{R}^{\hspace{.1ex}pq} \hspace{.1ex} \locationvector_\differentialindex{p} \locationvector_\differentialindex{q}
\end{equation*}

\vspace{-0.2em} \noindent
(\en{the coefficient}\ru{коэффициент}~$\onefourth$
\en{is used here}\ru{используется тут}
\en{for convenience}\ru{для удобства})

\vspace{.1em}\begin{equation*}
\begin{array}{ccc}
\mathscr{R}^{\hspace{.1ex}1\hspace{-0.1ex}1} \hspace{-0.3ex} =
\scalebox{0.8}{$ \displaystyle \frac{\raisemath{-0.2em}{1}}{\raisemath{.15em}{\smash{\textsl{g}}}} $} \hspace{.4ex} \mathfrak{R}_{\hspace{.1ex}2323}
\hspace{.2ex} ,
&
&
\\[.6em]
%
\mathscr{R}^{\hspace{.1ex}21} \hspace{-0.3ex} =
\scalebox{0.8}{$ \displaystyle \frac{\raisemath{-0.2em}{1}}{\raisemath{.15em}{\smash{\textsl{g}}}} $} \hspace{.4ex} \mathfrak{R}_{\hspace{.1ex}1323}
\hspace{.2ex} ,
&
\mathscr{R}^{\hspace{.1ex}22} \hspace{-0.3ex} =
\scalebox{0.8}{$ \displaystyle \frac{\raisemath{-0.2em}{1}}{\raisemath{.15em}{\smash{\textsl{g}}}} $} \hspace{.4ex} \mathfrak{R}_{\hspace{.1ex}1313}
\hspace{.2ex} ,
&
\\[.6em]
%
\mathscr{R}^{\hspace{.1ex}31} \hspace{-0.3ex} =
\scalebox{0.8}{$ \displaystyle \frac{\raisemath{-0.2em}{1}}{\raisemath{.15em}{\smash{\textsl{g}}}} $} \hspace{.4ex} \mathfrak{R}_{\hspace{.1ex}1223}
\hspace{.2ex} ,
&
\mathscr{R}^{\hspace{.1ex}32} \hspace{-0.3ex} =
\scalebox{0.8}{$ \displaystyle \frac{\raisemath{-0.2em}{1}}{\raisemath{.15em}{\smash{\textsl{g}}}} $} \hspace{.4ex} \mathfrak{R}_{\hspace{.1ex}1213}
\hspace{.2ex} ,
&
\mathscr{R}^{\hspace{.1ex}33} \hspace{-0.3ex} =
\scalebox{0.8}{$ \displaystyle \frac{\raisemath{-0.2em}{1}}{\raisemath{.15em}{\smash{\textsl{g}}}} $} \hspace{.4ex} \mathfrak{R}_{\hspace{.1ex}1212}
\hspace{.2ex} .
\end{array}
\end{equation*}

Равенство тензора Риччи нулю
${\hspace{.1ex}\pmb{\scalebox{1.2}[1]{$\mathscr{R}$}} \hspace{-0.16ex} = \hspace{-0.2ex} {^2\bm{0}}}$ (в~компонентах это шесть уравнений ${\hspace{.1ex}\mathscr{R}^{\hspace{.1ex}i\hspace{-0.1ex}j} \hspace{-0.3ex} = \mathscr{R}^{\hspace{.1ex}j\hspace{-0.06ex}i} \hspace{-0.3ex} = 0}$) \en{is}\ru{есть} \en{the~}\textcolor{magenta}{\en{necessary}\ru{необходимое}} \en{condition}\ru{условие} \en{of~integrability}\ru{интегрируемости}~(\ru{\inquotes{совместности}, }\inquotes{compatibility}) для нахождения вектора-радиуса~${\locationvector(q^{\hspace{.1ex}k})}$ по~полю~${\textsl{g}_{i\hspace{-0.1ex}j}(q^{\hspace{.1ex}k})}$.

\end{otherlanguage}

---------------------------------------------------------------------- 
----------------------------------------------------------------------

\en{
( в~отличие от~${\variation{\rotationtensor}} ).

\en{A~small rotation}ru{Малый поворот}
\en{is determined}\ru{определяется}
\en{by the vector}\ru{вектором}~${\varvector{o}}$,




\en{But}\ru{Но и}
\en{the final rotation}\ru{конечный поворот}
\en{is also}\ru{тоже}
\en{possible}\ru{возможно}
\en{to present}\ru{представить}
\en{as a~vector}\ru{как вектор}

\section*{\small \wordforbibliography}

\begin{changemargin}{\parindent}{0pt}
\fontsize{10}{12}\selectfont


\en{Many books exist}\ru{Существует много книг}\ru{,}
\en{which describe}\ru{которые описывают}
\en{only}\ru{только}
\en{the apparatus}\ru{аппарат}
\en{of the tensor calculus}\ru{тензорного исчисления}~\cite{mcconnell-tensoranalysis, schouten-tensoranalysis, sokolnikoff-tensoranalysis, borisenko.tarapov, rashevsky-riemanniangeometry}.
\en{However}\ru{Однако}, \en{an~index notation}\ru{индексная запись}\:--- \en{it’s when}\ru{это когда} \en{tensors are considered}\ru{тензоры рассматриваются} \en{as matrices of~components}\ru{как матрицы компонент}\:--- \en{is still more popular}\ru{всё ещё более популярна}\ru{,} \en{than the direct indexless notation}\ru{чем прямая безиндексная запись}.
\en{The direct notation}\ru{Прямая запись}
\en{is used}\ru{используется},
\en{for example}\ru{например},
\en{in the appendices}\ru{в~приложениях}
\en{of books}\ru{книг}
\en{by }Anatoliy I. Lurie (}\foreignlanguage{russian}{Анатолия И. Лурье}\en{)}
\cite{lurie-nonlinearelasticity, lurie-theoryofelasticity}.



\ru{Лекции }R.\:Feynman’\en{s}\ru{а}\en{ lectures}~\cite{feynman-lecturesonphysics}
\en{contain}\ru{содержат}
\en{a~vivid description}\ru{яркое описание}
\en{of vector fields}\ru{векторных полей}.
\en{Also}\ru{Также},
\en{the information}\ru{информация}
\en{about the tensor calculus}\ru{о~тензорном исчислении}\en{ is}\ru{\:---}
\en{part of}\ru{часть}
\en{original}\ru{оригинальной}
\en{and}\ru{и}
\en{profound}\ru{глубокой}
\en{book}\ru{книги}
\en{by }C.\:Truesdell\ru{’а}~\cite{truesdell-firstcourse}.

\end{changemargin}
