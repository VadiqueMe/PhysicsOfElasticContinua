\en{\section{Stresses as Lagrange multipliers}}

\ru{\section{Напряжения как множители Lagrange’а}}

\label{section:stressesAsLagrangeMultipliers}

\newcommand\bivalentlagrangemultiplier{{^2\hspace{-0.2ex}\bm{\lambda}}}

\en{The~application}\ru{Применению}
\en{of the~principle of~virtual work}\ru{принципа виртуальной работы},
\en{described}\ru{описанному}
\en{in}\ru{в}~\sectionref{section:virtualworkprinciple.elastic},
\en{was preceded by}\ru{предшествовало}
\en{the~introduction}\ru{введение}
\ru{тензора напряжения}\en{of the} Cauchy\en{ stress tensor}
\en{through}\ru{через}
\en{the~balance of~forces}\ru{баланс сил}
\en{for}\ru{для}
\en{an~infinitesimal}\ru{бесконечномалого}
\en{tetrahedron}\ru{тетраэдра}~(\sectionref{section:stressviatetrahedron}).
%
\en{But now}\ru{Но теперь}
\en{the~reader}\ru{читатель}
\en{will see}\ru{увидит}\ru{,}
\en{that}\ru{что}
\en{this principle}\ru{этот принцип}
\en{may be}\ru{может быть}
\en{as well applied without any}\ru{применён и~в\'{о}все без}
\en{tetrahedrons}\ru{тетраэдров}.

\en{Considering}\ru{Рассматривая}
\en{a~continuum}\ru{\rucontinuum}/\en{body}\ru{тело}\:---
\en{not~only}\ru{не~только}
\en{elastic}\ru{упругое},
\en{with any}\ru{с~любой}
\en{virtual work}\ru{виртуальной работой}
\en{of internal forces}\ru{внутренних сил}~${\variation{\internalwork} \hspace{-0.33ex}}$
(\en{per unit mass}\ru{на~единицу массы})\:---
\en{loaded with}\ru{нагруженное}
\en{external forces}\ru{внешними силами},
\en{mass ones}\ru{массовыми}~${\massloadvector dm \hspace{-0.1ex} = \hspace{-0.3ex} \massloadvector \hspace{-0.2ex} \massdensity \hspace{.2ex} d\mathcal{V} \hspace{-0.2ex}}$
(\en{for}\ru{для}
\en{brevity}\ru{краткости}
\en{just}\ru{просто}~${\hspace{-0.2ex} \massloadvector \hspace{-0.33ex}}$,
\en{meaning}\ru{имея в виду}
${\massloadvector \hspace{-0.2ex} \equiv \hspace{-0.2ex} \massloadvector_{\hspace{-0.25ex}*} \hspace{-0.3ex} - \mathdotdotabove{\currentlocationvector}\hspace{.25ex}}$
\en{in}\ru{в}~\en{dynamics}\ru{динамике})
\en{and }\ru{и~}\en{surface ones}\ru{поверхностными}~${\bm{p} \hspace{.25ex} d\mathcal{O}}$.
%
\en{Then}\ru{Тогда}
\en{the variational equa\-tion}\ru{вариационное уравнение}
\en{of the~principle of~virtual work}\ru{принципа виртуальной работы}
\en{is}\ru{есть}

\nopagebreak\vspace{-0.4em}
\begin{equation}\label{stressesAsLagrangeMultipliers:variations}
\integral\displaylimits_{\mathcal{V}} \hspace{-0.4ex}
\massdensity
\Bigl(
   \massloadvector \dotp \variation{\currentlocationvector}
   \hspace{.1ex} +
   \variation{\internalwork}
\hspace{.15ex} \Bigr)
d\mathcal{V}
\hspace{.1ex} + \hspace{-0.4ex}
\integral\displaylimits_{\mathclap{\mathcal{O}(\boundary \mathcal{V})}} \hspace{-0.4ex}
\bm{p} \dotp \variation{\currentlocationvector}
\hspace{.2ex} d\mathcal{O}
= \hspace{.1ex} 0
\hspace{.1ex} .
\end{equation}

\vspace{-0.7em}
\en{Further}\ru{Далее}\en{,}
\en{it’s assumed that}\ru{предполагается, что}
\en{internal forces}\ru{внутренние силы}
(\inquotes{\en{stresses}\ru{напряжения}})
\en{do~not produce work}\ru{не~производят работу}\ru{,}
\en{when}\ru{когда}
\en{a~continuum}\ru{\rucontinuum}/\en{body}\ru{тело}
\en{virtually}\ru{виртуально}
\en{moves}\ru{движется}
(\en{with}\ru{с}~${\variation{\currentlocationvector}}$)
\en{as a~whole}\ru{как целое}
\en{without}\ru{без}
\en{deformations}\ru{деформаций}
(\en{when}\ru{когда}
${\infinimentpetitedeformationvariation \hspace{-0.1ex} \equiv \hspace{-0.1ex} \insideinfinitesimalstrainvariation = \hspace{-0.2ex} \zerobivalent}$),
\en{that is}\ru{то есть}

\nopagebreak\vspace{-0.3em}
\begin{equation}\label{stressesAsLagrangeMultipliers:zerovirtualmovements}
\insideinfinitesimalstrainvariation
= \hspace{-0.2ex}
\zerobivalent
\hspace{1ex} \Rightarrow \hspace{.8ex}
\variation{\internalwork}
\hspace{-0.2ex} =
0
\hspace{.1ex} .
\end{equation}

\vspace{-0.4em}\noindent
\eqref{stressesAsLagrangeMultipliers:variations}
\en{with condition}\ru{с~условием}~\eqref{stressesAsLagrangeMultipliers:zerovirtualmovements}
\en{and }\ru{и~}\en{without}\ru{без}~${\variation{\internalwork}}$
\en{becomes}\ru{становится}
\en{a~variational equation}\ru{вариационным уравнением}
\en{with constraint}\ru{со~связью}.

\en{The~method of Lagrange multipliers}\ru{Метод множителей Lagrange’а}
\en{makes}\ru{делает}~${\variation{\currentlocationvector}}$
\en{random}\ru{случайными}
(\en{independent}\ru{независимыми})
\en{variations}\ru{вариациями}.
%
\en{Since}\ru{Поскольку}
\en{at~each point}\ru{в~каждой точке}
\en{the~constraint}\ru{связь}
\en{appears as}\ru{представляет собой}
\en{a~symmetric}\ru{симметричный}
\en{bivalent}\ru{бивалентный} %\en{of second complexity}\ru{второй сложности}
\en{tensor}\ru{тензор},
\en{the}\ru{множитель} Lagrange\ru{’а}\en{ multiplier}~${\hspace{-0.16ex} \bivalentlagrangemultiplier}$
\en{will likewise be}\ru{тоже будет}
\en{such}\ru{таким~же}
\en{a~tensor}\ru{тензором},
\en{bivalent}\ru{двухвалентным}
\en{and}\ru{и}~\en{symmetric}\ru{симметричным}.
%
\en{The~equation}\ru{Уравнение}
\en{with this multiplier}\ru{с~этим множителем}
\en{looks like}\ru{выглядит как}

\nopagebreak\vspace{-0.4em}
\begin{equation}\label{stressesAsLagrangeMultipliers:equationwiththemultiplier}
\integral\displaylimits_{\mathcal{V}} \hspace{-0.5ex} \Bigl( \massdensity \hspace{-0.1ex} \massloadvector \hspace{-0.15ex} \dotp \variation{\currentlocationvector} \hspace{.1ex} - \hspace{-0.1ex} \bivalentlagrangemultiplier \dotdotp \hspace{-0.15ex} \insideinfinitesimalstrainvariation \hspace{.2ex} \Bigr) \hspace{-0.1ex} d\mathcal{V}
\hspace{.1ex} + \hspace{-0.4ex}
\integral\displaylimits_{\mathclap{\mathcal{O}(\boundary \mathcal{V})}} \hspace{-0.4ex} \bm{p} \dotp \variation{\currentlocationvector} \hspace{.2ex} d\mathcal{O}
= \hspace{.1ex} 0
\hspace{.1ex} .
\vspace{-0.25em}\end{equation}

\vspace{-0.5em}
\en{The~symmetry}\ru{Симметрия}\en{ of}~$\bivalentlagrangemultiplier$
\en{gives}\ru{даёт}\footnote{${%
\bm{\Lambda}^{\hspace{-0.16ex}\mathsf{S}} \hspace{-0.1ex} \dotdotp \hspace{-0.1ex} \bm{X} \hspace{-0.2ex} =
\hspace{.1ex} \bm{\Lambda}^{\hspace{-0.16ex}\mathsf{S}} \hspace{-0.1ex} \dotdotp \hspace{-0.1ex} \bm{X}^{\T} \hspace{-0.4ex} =
\hspace{.1ex} \bm{\Lambda}^{\hspace{-0.16ex}\mathsf{S}} \hspace{-0.1ex} \dotdotp \hspace{-0.1ex} \bm{X}^{\hspace{.1ex}\mathsf{S}}
\hspace{-0.25ex}}$,
\hspace{.66em}
${%
\boldnabla \hspace{-0.1ex} \dotp \hspace{-0.2ex} \bigl( \hspace{.1ex} {\bm{B}} \hspace{-0.1ex} \dotp \bm{a} \hspace{.16ex} \bigr)
\hspace{-0.2ex} = \hspace{-0.2ex}
\bigl( \hspace{.1ex} \boldnabla \dotp \hspace{-0.15ex} \bm{B} \hspace{.1ex} \bigr) \hspace{-0.2ex} \dotp \bm{a} \hspace{.1ex}
+ \bm{B}^{\T} \hspace{-0.3ex} \dotdotp \boldnabla \hspace{-0.15ex} \bm{a}
}$}

\nopagebreak\vspace{-0.2em}\begin{equation*}
\bivalentlagrangemultiplier = \hspace{-0.2ex} \bivalentlagrangemultiplier^{\hspace{-0.33ex}\T}
\hspace{.466em} \Rightarrow \hspace{.4em}
\hspace{.12ex}\bivalentlagrangemultiplier \dotdotp \hspace{-0.15ex} \insideinfinitesimalstrainvariation
= \bivalentlagrangemultiplier \dotdotp \hspace{-0.15ex} \boldnabla \variation{\currentlocationvector}^{\T}
\hspace{-0.4ex} ,
\end{equation*}

\nopagebreak\vspace{-0.2em}\begin{equation*}
\bivalentlagrangemultiplier \dotdotp \hspace{-0.15ex} \insideinfinitesimalstrainvariation
=
\boldnabla \hspace{-0.1ex} \dotp \hspace{-0.1ex} \bigl( \hspace{.1ex} \bivalentlagrangemultiplier \dotp \variation{\currentlocationvector} \hspace{.1ex} \bigr) \hspace{-0.2ex}
- \hspace{-0.1ex} \boldnabla \hspace{.1ex} \dotp \hspace{-0.1ex} \bivalentlagrangemultiplier \dotp \variation{\currentlocationvector}
\hspace{.1ex} .
\end{equation*}

\vspace{-0.2em}\noindent
\en{Substituting}\ru{Подставляя}
\en{this}\ru{это}
\en{into}\ru{в}~\eqref{stressesAsLagrangeMultipliers:equationwiththemultiplier}
\en{and }\ru{и~}%
\en{applying}\ru{применяя}
\en{the~divergence theorem}\ru{теорему о~дивергенции}\footnote{${%
\scalebox{.93}{$ \displaystyle \integral\displaylimits_{\mathcal{V}} $} \hspace{-0.2ex}
\boldnabla \hspace{-0.1ex} \dotp \hspace{-0.2ex} \bigl( \hspace{.1ex} \bivalentlagrangemultiplier \dotp \variation{\currentlocationvector} \bigr)
d\mathcal{V}
\hspace{.1ex} = \hspace{-0.4ex}
\scalebox{.93}{$ \displaystyle \integral\displaylimits_{\mathclap{\mathcal{O}(\boundary \mathcal{V})}} $} \hspace{-0.1ex}
\currentunitnormal \dotp \hspace{-0.2ex} \bigl( \hspace{.1ex} \bivalentlagrangemultiplier \dotp \variation{\currentlocationvector} \bigr)
d\mathcal{O}
\hspace{.1ex} ,
\hspace{.66em}
%
\currentunitnormal \dotp \hspace{-0.2ex} \bigl( \hspace{.1ex} \bivalentlagrangemultiplier \dotp \variation{\currentlocationvector} \bigr)
\hspace{-0.2ex} = \hspace{-0.2ex} \bigl( \currentunitnormal \dotp \bivalentlagrangemultiplier \bigr) \hspace{-0.3ex} \dotp \variation{\currentlocationvector}
= \currentunitnormal \dotp \hspace{-0.1ex} \bivalentlagrangemultiplier \dotp \variation{\currentlocationvector}
}$}\hbox{\hspace{-0.5ex},}
\en{and}\ru{и}
\en{the variational equation with multiplier}\ru{вариационное уравнение с~множителем}~${\hspace{-0.2ex} \bivalentlagrangemultiplier}$
\en{becomes}\ru{становится}

\nopagebreak\vspace{-0.4em}
\begin{equation}\label{stressesAsLagrangeMultipliers:equationwiththemultipliertoo}
\integral\displaylimits_{\mathcal{V}} \hspace{-0.5ex} \Bigl( \massdensity \hspace{-0.1ex} \massloadvector \hspace{.12ex} + \boldnabla \dotp \hspace{-0.1ex} \bivalentlagrangemultiplier \Bigr) \hspace{-0.33ex} \dotp \variation{\currentlocationvector} \hspace{.2ex} d\mathcal{V}
\hspace{.1ex} + \hspace{-0.4ex}
\integral\displaylimits_{\mathclap{\mathcal{O}(\boundary \mathcal{V})}} \hspace{-0.5ex} \Bigl( \bm{p} \hspace{.2ex} - \currentunitnormal \hspace{.1ex} \dotp \hspace{-0.1ex} \bivalentlagrangemultiplier \Bigr) \hspace{-0.33ex} \dotp \variation{\currentlocationvector} \hspace{.2ex} d\mathcal{O}
= \hspace{.1ex} 0
\hspace{.1ex} .
\vspace{-0.25em}\end{equation}

\vspace{-0.4em}\noindent
\en{But}\ru{Но}~$\variation{\currentlocationvector}$
\en{is random}\ru{случайна}
\en{both on a~surface}\ru{и~на~поверхности}\ru{,}
\en{and}\ru{и}~\en{in a~volume}\ru{в~объёме},
\en{thus}\ru{поэтому}

\nopagebreak\vspace{-0.3em}
\begin{equation}\label{stressesAsLagrangeMultipliers:veryfamousequations}
\bm{p} \hspace{.1ex} = \currentunitnormal \hspace{.1ex} \dotp \hspace{-0.1ex} \bivalentlagrangemultiplier \hspace{.2ex},
\hspace{1em}
%
\boldnabla \dotp \hspace{-0.1ex} \bivalentlagrangemultiplier \hspace{.1ex} + \massdensity \hspace{-0.1ex} \massloadvector \hspace{-0.1ex} = \hspace{.1ex} \zerovector
\end{equation}

\vspace{-0.33em}\noindent
---
\en{the symmetric multiplier}\ru{симметричный множитель}~%
${\hspace{-0.2ex} \bivalentlagrangemultiplier}$,
\en{introduced}\ru{введённый}
\en{formally}\ru{формально},
\en{is}\ru{это}
\en{in fact}\ru{на с\'{а}мом деле}
\en{precisely}\ru{именно что}
\ru{тензор напряжения}\en{the} Cauchy\en{ stress tensor}\:!

\en{A~similar}\ru{Похожее}
\en{introduction of~stresses}\ru{введение напряжений}
\en{was presented}\ru{было представлено}
\en{in the~book}\ru{в~книге}~\cite{rabotnov-mechanicsofdeformable}.
\en{Here are no new results}\ru{Тут нет новых результатов},
\en{but}\ru{но}
\en{the~very possibility}\ru{сам\'{а} возможность}
\en{of simultaneously deriving}\ru{одно\-времен\-ного вывода}
\en{those equations}\ru{тех уравнений}
\en{of continuum mechanics}\ru{механики \rucontinuum{}а},
\en{that were previously considered}\ru{которые прежде считались}
\en{independent}\ru{независимыми},
\en{is quite interesting}\ru{весьма интересна}.
%
\en{In}\ru{В}~\en{subsequent chapters}\ru{последующих главах}
\en{this technique}\ru{эта техника}
\en{is used}\ru{используется}
\en{for building}\ru{для построения}
\en{new}\ru{новых}
\en{continuum models}\ru{континуальных моделей}.
