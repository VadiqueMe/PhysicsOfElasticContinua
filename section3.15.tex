\en{\section{Stresses as Lagrange multipliers}}

\ru{\section{Напряжения как множители Lagrange’а}}

\label{section:stressesAsLagrangeMultipliers}

\en{The~application}\ru{Применению}
\en{of the~principle of~virtual work}\ru{принципа виртуальной работы},
\en{described}\ru{описанному}
\en{in}\ru{в}~\sectionref{section:virtualworkprinciple.elastic},
\en{was preceded by}\ru{предшествовало}
\en{the~introduction}\ru{введение}
\ru{тензора напряжения}\en{of the} Cauchy\en{ stress tensor}
\en{through}\ru{через}
\en{the~balance of~forces}\ru{баланс сил}
\en{for}\ru{для}
\en{an~infinitesimal}\ru{бесконечномалого}
\en{tetrahedron}\ru{тетраэдра}~(\sectionref{section:stressviatetrahedron}).
%
\en{But now}\ru{Но теперь}
\en{the~reader}\ru{читатель}
\en{will see}\ru{увидит}\ru{,}
\en{that}\ru{что}
\en{this principle}\ru{этот принцип}
\en{may be}\ru{может быть}
\en{as well applied without any}\ru{применён и~в\'{о}все без}
\en{tetrahedrons}\ru{тетраэдров}.

\begin{otherlanguage}{russian}

Рассмотрим тело\:---
не~только~лишь упругое,
с~любой виртуальной работой внутренних сил~${\variation{\internalwork}}$
\en{per mass unit}\ru{на~единицу массы}\:---
нагруженное массовыми~${\massloadvector dm}$
(для~краткости пишем~$\massloadvector$ вместо~${\massloadvector \hspace{-0.2ex} - \hspace{-0.2ex} \mathdotdotabove{\currentlocationvector}}$,
так что динамика присутствует)
и~поверхностными~${\bm{p} \hspace{.25ex} d\mathcal{O}}$
внешними силами.
Имеем вариационное уравнение

\nopagebreak\vspace{-0.1em}
\begin{equation}\label{stressesAsLagrangeMultipliers:variations}
\integral\displaylimits_{\mathcal{V}} \hspace{-0.4ex} \rho \hspace{-0.2ex} \left(^{\mathstrut} \massloadvector \dotp \variation{\currentlocationvector} \hspace{.1ex} + \variation{\internalwork} \hspace{.15ex} \right) \hspace{-0.25ex} d\mathcal{V}
\hspace{.1ex} + \hspace{-0.4ex}
\integral\displaylimits_{\mathclap{\mathcal{O}(\boundary \mathcal{V})}} \hspace{-0.4ex} \bm{p} \dotp \variation{\currentlocationvector} \hspace{.2ex} d\mathcal{O}
= \hspace{.1ex} 0
\hspace{.1ex} .
\vspace{-0.25em}\end{equation}

\en{It is assumed that}\ru{Предполагается, что}
\en{the internal forces}\ru{внутренние силы}
\en{don’t produce a~work}\ru{не~производят работу}\ru{,}
\en{when}\ru{когда}
\en{a~continuum}\ru{\rucontinuum}
(\en{a~body}\ru{тело})
\en{virtually}\ru{виртуально}
\en{moves}\ru{движется}
\en{with}\ru{с}~${\variation{\currentlocationvector}}$
\en{as a~whole}\ru{как целое}
\en{without}\ru{без}
\en{deformations}\ru{деформаций}
(${\infinimentpetitdeformationdevariation = \zerobivalent}$).
\en{Then}\ru{Тогда}

\nopagebreak\vspace{-0.2em}\begin{equation}\label{stressesAsLagrangeMultipliers:zerovirtualmovements}
\infinimentpetitdeformationdevariation
\hspace{-0.1ex} = \hspace{-0.1ex}
\boldnabla \variation{\currentlocationvector}^{\mathsf{S}}
\hspace{-0.2ex} = \hspace{-0.1ex}
\zerobivalent
\hspace{.5ex} \Rightarrow \hspace{.33ex}
\variation{\internalwork}
\hspace{-0.2ex} =
0
\hspace{.1ex} .
\end{equation}

\vspace{-0.1em}
Отбросив~${\variation{\internalwork}}$ в~\eqref{stressesAsLagrangeMultipliers:variations}
с~условием~\eqref{stressesAsLagrangeMultipliers:zerovirtualmovements},
получим вариационное уравнение со~связью.
Приём с~множителями Lagrange’а даёт возможность считать вариации~${\variation{\currentlocationvector}}$ независимыми.
Поскольку в~каждой точке связь представлена симметричным тензором второй сложности,
то таким~же тензором будут и~множители Lagrange’а~${\hspace{-0.16ex} ^2\hspace{-0.2ex}\bm{\lambda}}$.
Приходим к~уравнению

\nopagebreak\vspace{-0.1em}\begin{equation}\label{stressesAsLagrangeMultipliers:variationstoo}
\integral\displaylimits_{\mathcal{V}} \hspace{-0.5ex} \Bigl( \rho \hspace{-0.1ex} \massloadvector \hspace{-0.15ex} \dotp \variation{\currentlocationvector} \hspace{.1ex} - \hspace{-0.1ex} {^2\hspace{-0.2ex}\bm{\lambda}} \dotdotp \hspace{-0.15ex} \boldnabla \variation{\currentlocationvector}^{\mathsf{S}} \Bigr) \hspace{-0.1ex} d\mathcal{V}
\hspace{.1ex} + \hspace{-0.4ex}
\integral\displaylimits_{\mathclap{\mathcal{O}(\boundary \mathcal{V})}} \hspace{-0.4ex} \bm{p} \dotp \variation{\currentlocationvector} \hspace{.2ex} d\mathcal{O}
= \hspace{.1ex} 0
\hspace{.1ex} .
\vspace{-0.25em}\end{equation}

\vspace{-0.16em}
Благодаря симметрии~${^2\hspace{-0.2ex}\bm{\lambda}}$ имеем\footnote{${%
\bm{\Lambda}^{\hspace{-0.16ex}\mathsf{S}} \hspace{-0.1ex} \dotdotp \hspace{-0.1ex} \bm{X} \hspace{-0.2ex} =
\hspace{.1ex} \bm{\Lambda}^{\hspace{-0.16ex}\mathsf{S}} \hspace{-0.1ex} \dotdotp \hspace{-0.1ex} \bm{X}^{\T} \hspace{-0.4ex} =
\hspace{.1ex} \bm{\Lambda}^{\hspace{-0.16ex}\mathsf{S}} \hspace{-0.1ex} \dotdotp \hspace{-0.1ex} \bm{X}^{\hspace{.1ex}\mathsf{S}}
\hspace{-0.25ex}}$,
\:\:
${%
\boldnabla \hspace{-0.1ex} \dotp \left( \hspace{.1ex} {\bm{B}} \hspace{-0.1ex} \dotp \bm{a} \hspace{.16ex} \right)
= \left( \hspace{.1ex} \boldnabla \dotp \hspace{-0.15ex} \bm{B} \hspace{.1ex} \right) \hspace{-0.1ex} \dotp \bm{a} \hspace{.1ex}
+ \bm{B}^{\T} \hspace{-0.3ex} \dotdotp \boldnabla \hspace{-0.15ex} \bm{a}
}$}

\nopagebreak\vspace{-0.2em}\begin{equation*}
{^2\hspace{-0.2ex}\bm{\lambda}} = \hspace{-0.2ex} {^2\hspace{-0.2ex}\bm{\lambda}}^{\hspace{-0.33ex}\T}
\: \Rightarrow \:
\hspace{.12ex}{^2\hspace{-0.2ex}\bm{\lambda}} \dotdotp \hspace{-0.15ex} \boldnabla \variation{\currentlocationvector}^{\mathsf{S}} \hspace{-0.2ex}
= {^2\hspace{-0.2ex}\bm{\lambda}} \dotdotp \hspace{-0.15ex} \boldnabla \variation{\currentlocationvector}^{\T}
\hspace{-0.4ex} ,
\end{equation*}

\nopagebreak\vspace{-0.2em}\begin{equation*}
{^2\hspace{-0.2ex}\bm{\lambda}} \dotdotp \hspace{-0.15ex} \boldnabla \variation{\currentlocationvector}^{\mathsf{S}} \hspace{-0.2ex} =
\boldnabla^{\mathstrut} \hspace{-0.1ex} \dotp \hspace{-0.1ex} \left( \hspace{.1ex} {^2\hspace{-0.2ex}\bm{\lambda}} \dotp \variation{\currentlocationvector} \hspace{.2ex} \right)
- \boldnabla \hspace{.1ex} \dotp \hspace{-0.1ex} {^2\hspace{-0.2ex}\bm{\lambda}} \dotp \variation{\currentlocationvector}
\hspace{.1ex} .
\end{equation*}

\vspace{-0.2em}\noindent
Подставив это в~\eqref{stressesAsLagrangeMultipliers:variationstoo}
и~применив
теорему о~дивергенции,
получаем

\nopagebreak\vspace{-0.25em}\begin{equation*}
\integral\displaylimits_{\mathcal{V}} \hspace{-0.5ex} \Bigl( \rho \hspace{-0.1ex} \massloadvector \hspace{.12ex} + \boldnabla \hspace{.1ex} \dotp \hspace{-0.1ex} {^2\hspace{-0.2ex}\bm{\lambda}} \Bigr) \hspace{-0.33ex} \dotp \variation{\currentlocationvector} \hspace{.2ex} d\mathcal{V}
\hspace{.1ex} + \hspace{-0.4ex}
\integral\displaylimits_{\mathclap{\mathcal{O}(\boundary \mathcal{V})}} \hspace{-0.5ex} \Bigl( \bm{p} \hspace{.2ex} - \currentunitnormal \hspace{.1ex} \dotp \hspace{-0.1ex} {^2\hspace{-0.2ex}\bm{\lambda}} \Bigr) \hspace{-0.33ex} \dotp \variation{\currentlocationvector} \hspace{.2ex} d\mathcal{O}
= \hspace{.1ex} 0
\hspace{.1ex} .
\vspace{-0.25em}\end{equation*}

\noindent
\en{But}\ru{Но}~$\variation{\currentlocationvector}$
\en{is random}\ru{случайна}
\en{both on a~surface}\ru{и~на~поверхности}\ru{,}
\en{and}\ru{и}~\en{in a~volume}\ru{в~объёме},
\en{thus}\ru{поэтому}

\nopagebreak\vspace{-0.2em}\begin{equation*}
\bm{p} \hspace{.1ex} = \currentunitnormal \hspace{.1ex} \dotp \hspace{-0.1ex} {^2\hspace{-0.2ex}\bm{\lambda}} \hspace{.2ex},
\:\:
\boldnabla \hspace{.1ex} \dotp \hspace{-0.1ex} {^2\hspace{-0.2ex}\bm{\lambda}} \hspace{.2ex} + \hspace{.2ex} \rho \hspace{-0.1ex} \massloadvector \hspace{-0.1ex} = \hspace{.1ex} \zerovector
\end{equation*}

\vspace{-0.25em}\noindent
---
\en{the symmetric multiplier}\ru{симметричный множитель}~%
${\hspace{-0.16ex} ^2\hspace{-0.2ex}\bm{\lambda}}$,
\en{introduced}\ru{введённый}
\en{formally}\ru{формально},
\en{is}\ru{это}
\en{in fact}\ru{на с\'{а}мом деле}
\en{precisely}\ru{именно что}
\ru{тензор напряжения}\en{the} Cauchy\en{ stress tensor}\:!

\en{A~similar}\ru{Похожее}
\en{introduction of~stresses}\ru{введение напряжений}
\en{was presented}\ru{было представлено}
\en{in the~book}\ru{в~книге}~\cite{rabotnov-mechanicsofdeformable}.
\en{Here are no new results}\ru{Тут нет новых результатов},
но сам\'{а} возможность
одно\-времен\-ного вывода
тех уравнений механики \rucontinuum{}а,
которые прежде считались независимыми,
весьма интересна.
В~следующих главах
эта техника
используется
для построения
новых
континуальных моделей.

\end{otherlanguage}
