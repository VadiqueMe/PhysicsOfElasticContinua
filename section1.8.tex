\en{\section{Symmetric and skewsymmetric tensors}}

\ru{\section{Симметричные и кососимметричные тензоры}}

\label{section:tensors.symmetric+skewsymmetric}

\en{A~tensor}\ru{Тензор}\ru{,}
\en{that does not change}\ru{который не изменяется}
\en{upon a~permutation}\ru{при перестановке}
\en{of some pair of its indices}\ru{какой\hbox{-}либо пары своих индексов}\ru{,}
\en{is called}\ru{называется}
\en{symmetric}\ru{симметричным}
\en{for}\ru{для}
\en{that pair}\ru{той пары}
\en{of~indices}\ru{индексов}.
\en{And when}\ru{А~когда}
\en{a~permutation}\ru{перестановка}
\en{of some}\ru{какой\hbox{-}нибудь}
\en{pair}\ru{пары}
\en{of~indices}\ru{индексов}
\en{alternates the~sign}\ru{меняет знак}~\inquotes{$+$/$-$}{}
\en{of a~tensor}\ru{тензора},
\en{then}\ru{тогда}
\en{this tensor}\ru{этот тензор}
\en{is called}\ru{называется}
\en{anti-symmetric}\ru{анти-симметричным}
\en{or}\ru{или}
\en{skew-symmetric}\ru{косо-симметричным}
\en{for that pair of indices}\ru{для той пары индексов}.

\en{As example}\ru{Как пример},
\en{the~tensor of parity of permutations}\ru{тензор чётности перестановок}~$\permutationsparitytensor$~\eqref{permutationsparityintro}
\en{is antisymmetric}\ru{антисимметричен}
\en{for}\ru{для}
\en{any\;\&\;every}\ru{любой и~каждой}
\en{pair of~indices}\ru{пары индексов},
\en{it is completely}\ru{он\:--- полностью}
(\en{absolutely}\ru{соверш\'{е}нно, абсолютно})
\en{skewsymmetric}\ru{кососимметричен}.

\en{Tensor of the~second complexity}\ru{Тензор второй сложности}~$\bm{B}$
\en{is symmetric}\ru{симметричен}\ru{,}
\en{when}\ru{когда}
${\bm{B} = \bm{B}^{\T}\hspace{-0.3ex}}$.
\en{If}\ru{Если}~%
\en{transposing}\ru{транспонирование}
\en{changes the tensor’s sign}\ru{меняет знак тензора},
\en{that is}\ru{то есть}
${ \bm{A}^{\hspace{-0.1em}\T} \hspace{-0.3ex} = - \bm{A} }$,
\en{then}\ru{то}
\en{tensor}\ru{тензор}~${\hspace{-0.2ex}\bm{A}}$
\en{is skewsymmetric}\ru{кососимметричен}
(\en{antisymmetric}\ru{антисимметричен}).

\en{The~sum}\ru{Сумма}
\en{of a~bivalent tensor}\ru{бивалентного тензора}~${\bm{C}}$
\en{with }\ru{с~}\en{the~transpose}\ru{транспонированным}~${\bm{C}^{\hspace{.1ex}\T}\hspace{-0.2ex}}$
\en{is always symmetric}\ru{всегда симметрична}\::
${\bigl( \bm{C} \hspace{-0.1ex} + \bm{C}^{\hspace{.1ex}\T} \bigr)^{\hspace{-0.1ex}\T} \hspace{-0.4ex}
= \bm{C}^{\hspace{.1ex}\T} \hspace{-0.3ex} + \bm{C}
= \bm{C} \hspace{-0.1ex} + \bm{C}^{\hspace{.1ex}\T} \hspace{-0.3ex}}$
${\hspace{.5ex} \forall \bm{C}}$,
\en{while}\ru{тогда как}
\en{the~difference}\ru{разность}
${\bigl( \bm{C} \hspace{-0.1ex} - \bm{C}^{\hspace{.1ex}\T} \bigr)^{\hspace{-0.1ex}\T} \hspace{-0.4ex}
= \bm{C}^{\hspace{.1ex}\T} \hspace{-0.3ex} - \bm{C}
= - \bigl( \bm{C} \hspace{-0.1ex} - \bm{C}^{\hspace{.1ex}\T} \bigr)}$
\en{is always}\ru{всегда}
${\forall \bm{C}}$
\en{antisymmetric}\ru{антисимметрична}.

\en{Denoting}\ru{Обозначая}
%
\begin{equation}\label{symmetricandantisymmetricpartsofbivalent}
\bm{C}^{\mathsf{\hspace{.1ex}S}} \hspace{-0.22ex} \equiv \hspace{.1ex} \displaystyle \smalldisplaystyleonehalf \hspace{-0.2ex} \left( {\bm{C} + \bm{C}^{\hspace{.1ex}\T}}\hspace{.1ex} \right)
\hspace{-0.2ex} , \hspace{.5em}
\bm{C}^{\mathsf{\hspace{.1ex}A}} \hspace{-0.22ex} \equiv \hspace{.1ex} \displaystyle \smalldisplaystyleonehalf \hspace{-0.2ex} \left( {\bm{C} - \bm{C}^{\hspace{.1ex}\T}}\hspace{.1ex} \right)
\end{equation}
%
---
\en{the~symmetric}\ru{симметричная}~${\bm{C}^{\mathsf{\hspace{.1ex}S}}}$
\en{and }\ru{и~}\en{the~antisymmetric}\ru{антисимметричная}~${\bm{C}^{\mathsf{\hspace{.1ex}A}}}$
\en{parts}\ru{части}
\en{of some bivalent tensor}\ru{какого-нибудь бивалентного тензора}~${\bm{C}}$,
\en{any}\ru{любой}
\en{bivalent}\ru{бивалентный}
\en{tensor}\ru{тензор}
\en{can be}\ru{может быть}
\en{presented}\ru{представлен}
\en{as}\ru{как}
\en{the~sum}\ru{сумма}
\en{of these}\ru{этих}
\en{parts}\ru{частей}
%
\begin{equation}\label{bivalentasthesumofsymmetricandantisymmetricparts}
\bm{C} = \bm{C}^{\mathsf{\hspace{.1ex}S}} \hspace{-0.1ex} + \hspace{.1ex} \bm{C}^{\mathsf{\hspace{.1ex}A}}
\hspace{-0.3ex} , \hspace{.5em}
\bm{C}^{\hspace{.1ex}\T} \hspace{-0.33ex} = \bm{C}^{\mathsf{\hspace{.1ex}S}} \hspace{-0.1ex}-\hspace{.1ex} \bm{C}^{\mathsf{\hspace{.1ex}A}}
\hspace{-0.3ex} .
\end{equation}

\noindent
\en{For}\ru{Для}
\en{a~dyad}\ru{диады}
%
\begin{equation*}
\bm{c}\bm{d}
\hspace{.1ex} = \hspace{.1ex}
\tikzmark{beginSymmetricPartOfDyad} \smalldisplaystyleonehalf \hspace{-0.1ex} \left( \bm{c}\bm{d} + \hspace{-0.1ex} \bm{d}\bm{c} \hspace{.1ex} \right)  \hspace{-0.3ex} \tikzmark{endSymmetricPartOfDyad}
+ \hspace{.2ex}
\tikzmark{beginSkewsymmetricPartOfDyad} \smalldisplaystyleonehalf \hspace{-0.1ex} \left( \bm{c} \bm{d} - \hspace{-0.1ex} \bm{d} \bm{c} \hspace{.1ex} \right) \hspace{-0.3ex} \tikzmark{endSkewsymmetricPartOfDyad}
.
\end{equation*}%
\AddOverBrace[line width=.75pt][-0.3ex,0.3em]{beginSymmetricPartOfDyad}{endSymmetricPartOfDyad}{${\scriptstyle%
\bm{c}\bm{d}^{\mathsf{\hspace{.25ex}S}}%
}$}
\AddOverBrace[line width=.75pt][-0.3ex,0.3em]{beginSkewsymmetricPartOfDyad}{endSkewsymmetricPartOfDyad}{${\scriptstyle%
\bm{c}\bm{d}^{\mathsf{\hspace{.25ex}A}}%
}$}

\vspace{-1em}
{\small
\setlength{\parindent}{0pt}
\begin{leftverticalbar}%%[oversize]
%
\en{The~product}\ru{Произведение}
${\bm{C}^{\mathsf{\hspace{.1ex}S}} \hspace{-0.1ex}\dotp \bm{D}^{\mathsf{\hspace{.1ex}S}}}$
\en{of two}\ru{двух}
\en{symmetric tensors}\ru{симметричных тензоров}
${\bm{C}^{\mathsf{\hspace{.1ex}S}}}$
\en{and}\ru{и}~${\bm{D}^{\mathsf{\hspace{.1ex}S}}}$
\en{is symmetric}\ru{симметрично}
\en{not always}\ru{не всегда},
\en{but only when}\ru{но~лишь когда}
${\bm{D}^{\mathsf{\hspace{.1ex}S}} \hspace{-0.1ex}\dotp\hspace{.12ex} \bm{C}^{\mathsf{\hspace{.1ex}S}} \hspace{-0.2ex}=\hspace{.1ex} \bm{C}^{\mathsf{\hspace{.1ex}S}} \hspace{-0.1ex}\dotp \bm{D}^{\mathsf{\hspace{.1ex}S}}\hspace{-0.4ex}}$,
\en{because}\ru{ведь}
\en{by}\ru{по}~\eqref{transposeofdotproductforbivalenttensors}
${\left(\hspace{.08ex} \bm{C}^{\mathsf{\hspace{.1ex}S}} \hspace{-0.1ex}\dotp \bm{D}^{\mathsf{\hspace{.1ex}S}} \hspace{.1ex}\right)^{\hspace{-0.32ex}\T} \hspace{-0.3ex} = \hspace{.1ex} \bm{D}^{\mathsf{\hspace{.1ex}S}} \hspace{-0.1ex}\dotp\hspace{.12ex} \bm{C}^{\mathsf{\hspace{.1ex}S}} \hspace{-0.4ex}}$.

\end{leftverticalbar}
\par}

%%%В~трёхмерном (нечётномерном) пространстве любой антисимметричный тензор второй сложности необрат\'{и}м,
%%%третий инвариант (определитель матрицы компонент) для~него\:--- нулевой.

\en{With}\ru{С}~\eqref{commutativity.unitdyad-cross-vector}
\en{and}\ru{и}~\eqref{theparticularcasefortheunitdyadwithvector},
\en{the~skew symmetry}\ru{косая симметрия}
\en{of the~}\hbox{\hspace{-0.2ex}\inquotes{${\hspace{-0.25ex}\times\hspace{-0.1ex}}$}\hspace{-0.2ex}-\en{product}\ru{произведения}}
\en{for}\ru{для}
\en{the~unit dyad}\ru{единичной диады}
\en{and}\ru{и}~\en{a~vector}\ru{вектора}
\en{is obvious}\ru{очевидна}

\nopagebreak\vspace{-0.5em}
\begin{multline}\label{skewsymmetryofunitdiadcrossvector}
\bigl( \UnitDyad \times \bm{a} \hspace{.1ex} \bigr)^{\hspace{-0.25ex}\T} \hspace{-0.4ex}
= \hspace{-0.2ex} \bigl( \hspace{.1ex} %%\scalebox{.95}[1]{$
    \bm{e}_{\hspace{-0.1ex}j} \bm{e}_{\hspace{-0.1ex}j} \hspace{-0.25ex} \times \hspace{-0.2ex} a_i \bm{e}_i
%%$}
\hspace{.1ex} \bigr)^{\hspace{-0.25ex}\T} \hspace{-0.33ex}
= \hspace{-0.2ex} \bigl( \hspace{-0.1ex} %%\scalebox{.95}[1]{$
    - \hspace{.25ex} \bm{e}_{\hspace{-0.1ex}j} \hspace{.1ex} a_i \bm{e}_i \hspace{-0.3ex} \times \hspace{-0.3ex} \bm{e}_{\hspace{-0.1ex}j}
%%$}
\hspace{.1ex} \bigr)^{\hspace{-0.25ex}\T} \hspace{-0.33ex}
= %%\scalebox{.95}[1]{$
    - \hspace{.25ex} a_i \bm{e}_i \hspace{-0.3ex} \times \hspace{-0.3ex} \bm{e}_{\hspace{-0.1ex}j} \bm{e}_{\hspace{-0.1ex}j}
%%$}
\\[-0.1em]
= - \hspace{.3ex}
\bm{a} \times \UnitDyad
= - \hspace{.25ex} \UnitDyad \times \bm{a}
= \hspace{-0.1ex} \bigl( \bm{a} \times \UnitDyad \hspace{.1ex} \bigr)^{\hspace{-0.25ex}\T}
\hspace{-0.4ex} .
\hspace{1em}
\end{multline}

\en{In search}\ru{В~поисках}
\en{for a~case when}\ru{случая, когда}
\en{a~bivalent tensor}\ru{бивалентный тензор}~${\hspace{-0.2ex}\bm{A}}$
\en{can be}\ru{может быть}
\en{represented}\ru{представлен}
%%(\en{and better}\ru{а~лучше},
%%\emph{\en{reciprocally reversible}\ru{взаимно обратимо}}
%%\en{represented}\ru{представлен})
\en{by just a~single}\ru{всего одним}
\en{vector}\ru{вектором}~$\bm{a}$,
\en{in such a~way that}\ru{таким путём, что}
\en{an~action}\ru{действие}
\en{of~vector}\ru{вектора}~$\bm{a}$
\en{on other objects}\ru{на другие объекты}
\en{is exactly like}\ru{в~точности как}
\en{an~action}\ru{действие}
\en{of~bivalent}\ru{бивалентного}~${\hspace{-0.2ex}\bm{A}}$
\en{on the~same objects}\ru{на те~же объекты},
\en{perhaps}\ru{возможно}
\en{there’s a~chance}\ru{есть шанс}
\en{to find}\ru{найти}
\en{such}\ru{такой}~${\bm{A} \narroweq \bm{A}(\bm{a})}$\ru{,}
\en{that}\ru{чтобы}
\en{for}\ru{для}~${\forall \hspace{.2ex} {^\mathrm{n}\hspace{-0.1ex}\bm{\xi}}}$~${\forall \mathrm{n \!>\! 0}}$
%
\begin{alignat*}{2}
\bm{b} \hspace{.3ex}
= \bm{A} \hspace{.2ex}\dotp\hspace{.1ex} {^\mathrm{n}\hspace{-0.12ex}\bm{\xi}}
& \hspace{.4em} \Leftrightarrow \hspace{.5em} &
\bm{a} \hspace{.1ex} \times {^\mathrm{n}\hspace{-0.12ex}\bm{\xi}}
\hspace{.1ex} = \hspace{.15ex} \bm{b}
& \hspace{.8em} \forall \hspace{.1ex} \bm{b}
\hspace{.2ex} ,
\\[-0.1em]
%
\bm{d} \hspace{.3ex}
= {^\mathrm{n}\hspace{-0.12ex}\bm{\xi}} \hspace{.2ex} \dotp \hspace{-0.1ex} \bm{A}
& \hspace{.4em} \Leftrightarrow \hspace{.5em} &
{^\mathrm{n}\hspace{-0.12ex}\bm{\xi}} \hspace{.1ex} \times \bm{a}
\hspace{.1ex} = \hspace{.15ex} \bm{d}
& \hspace{.8em} \forall \bm{d}
\end{alignat*}
%
\en{or}\ru{или},
\en{in words}\ru{словами},
\en{the~}\hbox{\hspace{-0.2ex}\inquotes{${\dotp\hspace{.22ex}}$}\hspace{-0.2ex}-\en{product}\ru{произведение}}
\en{of~bivalent}\ru{бивалентного}~${\hspace{-0.2ex}\bm{A}}$
\en{and }\ru{и~}\en{some}\ru{какого\hbox{-}нибудь}
\en{other}\ru{другого}
\en{tensor}\ru{тензора}~${\hspace{-0.2ex}{^\mathrm{n}\hspace{-0.12ex}\bm{\xi}}}$
\en{is equal to}\ru{равн\'{о}}
\en{the~}\hbox{\hspace{-0.2ex}\inquotes{${\hspace{-0.25ex}\times\hspace{-0.1ex}}$}\hspace{-0.2ex}-\en{product}\ru{произведению}}
\en{of~pseudovector}\ru{псевдовектора}~$\bm{a}$
\en{and the~same}\ru{и~того~же}
\en{tensor}\ru{тензора}~${\hspace{-0.2ex}{^\mathrm{n}\hspace{-0.12ex}\bm{\xi}}}$.

\en{The~relation}\ru{Отношение}~${\bm{a} \mapsto \hspace{-0.2ex} \bm{A}}$
\en{can be}\ru{может быть}
\en{derived}\ru{получено}
\en{from}\ru{из}~\eqref{definingpropertyoftheidentitytensor}
\en{and}\ru{и}~\eqref{crossproductfortheunitdyadandsometensor}

\nopagebreak\vspace{-0.1em}
\begin{equation}\label{skewsymmetricbivalentfromvector}
\begin{array}{c}
\bm{A} = \bm{A} \hspace{.1ex} \dotp \UnitDyad = \hspace{.1ex} \bm{a} \times \hspace{-0.1ex} \UnitDyad = - \hspace{.3ex} \bm{a} \dotp \permutationsparitytensor
\hspace{.1ex} ,
\\[.1em]
\bm{A} = \UnitDyad \hspace{.1ex} \dotp \bm{A} = \UnitDyad \times \bm{a} = - \hspace{.2ex} \permutationsparitytensor \dotp \bm{a}
\hspace{.2ex} .
\end{array}
\end{equation}

\noindent
\en{Or}\ru{Или},
\en{delving}\ru{копаясь}
\en{into components}\ru{в~компонентах},

\nopagebreak\vspace{-0.1em}
\begin{equation*}
\begin{array}{r@{\hspace{1ex}}c@{\hspace{1ex}}l}
\bm{A} \hspace{.2ex}\dotp\hspace{.1ex} {^\mathrm{n}\hspace{-0.12ex}\bm{\xi}} & = & \bm{a} \hspace{.1ex} \times {^\mathrm{n}\hspace{-0.12ex}\bm{\xi}}
\\
%
A_{hi} \bm{e}_h \bm{e}_i \hspace{.1ex}\dotp\hspace{.25ex} \xi_{j\hspace{-0.1ex}k \ldots q} \hspace{.2ex} \bm{e}_j \bm{e}_k \ldots \bm{e}_q & = & a_{i} \bm{e}_i \times\hspace{.2ex} \xi_{j\hspace{-0.1ex}k \ldots q} \hspace{.2ex} \bm{e}_j \bm{e}_k \ldots \bm{e}_q
\\[.1em]
%
A_{hj} \hspace{.2ex} \xi_{j\hspace{-0.1ex}k \ldots q} \hspace{.2ex} \bm{e}_h \bm{e}_k \ldots \bm{e}_q & = & a_{i} \permutationsparitysymbols{i\hspace{-0.1ex}jh} \hspace{.2ex} \xi_{j\hspace{-0.1ex}k \ldots q} \hspace{.2ex} \bm{e}_h \bm{e}_k \ldots \bm{e}_q
\\[.1em]
%
A_{hj} & = & a_{i} \permutationsparitysymbols{i\hspace{-0.1ex}jh}
\\[.1em]
%
A_{hj} & = & \! - \, a_{i} \permutationsparitysymbols{ihj}
\\[.2em]
%
\bm{A} & = & \! - \, \bm{a} \dotp \permutationsparitytensor
\end{array}
\end{equation*}

\vspace{-0.5em}\noindent
\en{and }\ru{и~}%
\en{by similar way}\ru{похожим путём}
\en{from}\ru{из}
${{^\mathrm{n}\hspace{-0.12ex}\bm{\xi}} \hspace{.2ex} \dotp \bm{A} = {^\mathrm{n}\hspace{-0.12ex}\bm{\xi}} \hspace{.2ex}\times \bm{a}}$
\en{follows}\ru{следует}
${\bm{A} = - \hspace{.2ex} \permutationsparitytensor \dotp \bm{a}\hspace{.1ex}}$.

(\en{Pseudo}\ru{Псевдо})\en{vector}\ru{вектор}~$\bm{a}$
\en{is sometimes}\ru{иногда}
\en{named as}\ru{именуется как}
\inquotes{\en{accompanying}\ru{сопутствующий}}
\en{or}\ru{или}
\inquotes{\en{companion}\ru{компаньон~(companion)}}
\en{for}\ru{для}
\en{tensor}\ru{тензора}~${\hspace{-0.2ex}\bm{A}}$.

\en{To components}\ru{К~компонентам}

\nopagebreak\vspace{-0.8em}
\begin{equation*}
\begin{array}{r@{\hspace{.7ex}}c@{\hspace{.7ex}}l}
\bm{A}
& = &
- \hspace{.2ex} \permutationsparitytensor \dotp \bm{a}
\\[.2em]
A_{i\hspace{-0.1ex}j} \bm{e}_i \bm{e}_{\hspace{-0.1ex}j}
& = &
- \hspace{.1ex} \permutationsparitysymbols{i\hspace{-0.1ex}j\hspace{-0.1ex}k} \hspace{.1ex} \bm{e}_i \bm{e}_{\hspace{-0.1ex}j} a_k
\\[.4em]
A_{i\hspace{-0.1ex}j} ( a_k ) \hspace{.4ex} \text{:} \hspace{.7em}
A_{i\hspace{-0.1ex}j}
& = &
- \hspace{.1ex} \permutationsparitysymbols{i\hspace{-0.1ex}j\hspace{-0.1ex}k} \hspace{.1ex} a_k
\end{array}
\end{equation*}

\vspace{-0.2em}\noindent
\en{or}\ru{или},
\en{written}\ru{записанное}
\en{as}\ru{как}
\en{a~matrix}\ru{матрица},

\nopagebreak\vspace{-0.5em}
\begin{alignat*}{2}
\asmatrixwithdimensions{A_{i\hspace{-0.1ex}j}}{3}{3}
& = \hspace{-0.1em}
\scalebox{.9}{$\left[ \begin{array}{c@{\hspace{1em}}c@{\hspace{1em}}c}
A_{1\hspace{-0.1ex}1} & A_{12} & A_{13} \\
A_{21} & A_{22} & A_{23} \\
A_{31} & A_{32} & A_{33}
\end{array} \hspace{.25ex}\right]$}
\hspace{-0.3ex} && = \hspace{-0.3ex}
\scalebox{.88}{$\left[ \begin{array}{c@{\hspace{1em}}c@{\hspace{1em}}c}
0 & - \permutationsparitysymbols{123} \hspace{.1ex} a_3 & - \permutationsparitysymbols{132} \hspace{.1ex} a_2 \\
- \permutationsparitysymbols{213} \hspace{.1ex} a_3 & 0 & - \permutationsparitysymbols{231} a_1 \\
- \permutationsparitysymbols{312} \hspace{.1ex} a_2 & - \permutationsparitysymbols{321} a_1 & 0
\end{array} \hspace{.3ex}\right]$}
\\[.2em]
& && = \hspace{-0.1em}
\scalebox{.93}{$\left[ \begin{array}{ccc}
0 & - \hspace{-0.1ex} a_3 & a_2 \\
a_3 & 0 & - \hspace{-0.1ex} a_1 \\
- \hspace{-0.1ex} a_2 & a_1 & 0
\end{array} \hspace{.25ex}\right]$}
.
\end{alignat*}

\vspace{-0.1em}
\en{That}\ru{То, что}
\en{the~bivalent}\ru{бивалентный}~${\hspace{-0.2ex}\bm{A} \hspace{-0.1ex} = \bm{a} \hspace{-0.2ex} \times \hspace{-0.3ex} \UnitDyad = \hspace{-0.1ex} \UnitDyad \hspace{-0.2ex} \times \hspace{-0.2ex} \bm{a}}$
\en{is skewsymmetric}\ru{кососимметричен}
\en{was clear}\ru{было ясно}
\en{since}\ru{со~времени}~\eqref{skewsymmetryofunitdiadcrossvector}.
%
\en{In the three-dimensional space}\ru{В~трёхмерном пространстве}\en{,}
\en{any}\ru{у~любого}
\en{antisymmetric}\ru{антисимметричного}
\en{tensor}\ru{тензора}
\en{of the~second complexity}\ru{второй сложности}
\en{has only}\ru{только}
\en{three}\ru{три}
\en{independent}\ru{независимых}
\en{components}\ru{компонента}
\en{out of}\ru{из}~9\::
${A_{ij} \hspace{-0.2ex} = - A_{\hspace{-0.1ex}j\hspace{-0.1ex}i}}$
\en{and}\ru{и}~${A_{\hspace{-0.1ex}j\hspace{-0.1ex}j} = 0}$.

\en{The~uniqueness of}\ru{Уникальность}~$\bm{a}$
\en{for}\ru{для}
\en{the~unique}\ru{уникального}~${\hspace{-0.2ex}\bm{A}}$,
\en{that is}\ru{то есть}
\en{if}\ru{если}
${ \bm{a}' \hspace{-0.4ex} \times \hspace{-0.2ex} \bm{E} = \hspace{-0.2ex} \bm{A} }$
\en{and}\ru{и}~%
${ \bm{a}'\hspace{-0.15ex}' \hspace{-0.4ex} \times \hspace{-0.2ex} \bm{E} = \hspace{-0.2ex} \bm{A} }$
(\en{or}\ru{или}~%
${ \bm{a}' \hspace{-0.4ex} \times \hspace{-0.2ex} \bm{E}
\hspace{.2ex} - \hspace{.2ex} \bm{a}'\hspace{-0.15ex}' \hspace{-0.4ex} \times \hspace{-0.2ex} \bm{E}
= \hspace{-0.2ex} \bm{A} - \hspace{-0.2ex} \bm{A} }$)\ru{,}
\en{then}\ru{то}
${ \bm{a}' \hspace{-0.3ex} = \bm{a}'\hspace{-0.15ex}' \hspace{-0.2ex} }$
\en{or}\ru{или}
${ \bm{a}' \hspace{-0.3ex} - \bm{a}'\hspace{-0.15ex}' \hspace{-0.2ex} = \zerovector }$
%
\begin{equation*}
\begin{array}{r@{\hspace{0ex}}c@{\hspace{.3ex}}r@{\hspace{.7ex}}c@{\hspace{.7ex}}l@{\hspace{0ex}}c@{\hspace{.3ex}}l}
\bigl( \bm{a}' \hspace{-0.2ex} - \bm{a}'\hspace{-0.15ex}' \bigr) & \times & \bm{E}
& = &
\hspace{-0.2ex} \zerobivalent
\\[.2em]
\bigl( \bm{a}' \hspace{-0.2ex} - \bm{a}'\hspace{-0.15ex}' \bigr) & \dotp & \hspace{-0.4ex} \permutationsparitytensor
& = &
\zerovector \dotp \permutationsparitytensor
\end{array}
\end{equation*}
%
\en{follows}\ru{следует}
\en{from}\ru{из}
\en{the~equal}\ru{равных}
\en{zeros}\ru{нулей}~${\zerovector \dotp \permutationsparitytensor = \zerobivalent}$
\en{and }\ru{и~}\en{the~uniqueness}\ru{уникальности}
\ru{результата}\en{of~the~}\hbox{\hspace{-0.2ex}\inquotes{${\dotp\hspace{.22ex}}$}\hspace{-0.2ex}-\en{product}\ru{произведения}\en{’s result}}
(%%${ \bm{c} \dotp \bm{b} = \bm{c} \dotp \bm{d} \hspace{.1ex} , \hspace{.2em} \bm{c} \neq \zerovector
%%\hspace{.3em} \Leftrightarrow \hspace{.25em}
%%\bm{b} = \bm{d} }$,
${\bm{b} \dotp \bm{c} = \bm{d} \dotp \bm{c} \hspace{.1ex} , \hspace{.2em} \bm{c} \neq \zerovector
\hspace{.2em} \Leftrightarrow \hspace{.15em}
\bm{b} = \bm{d} }$,
\en{including}\ru{включая}
${\bm{b} \dotp \bm{c} = \zerovector \dotp \bm{c} \hspace{.1ex} , \hspace{.2em} \bm{c} \neq \zerovector
\hspace{.2em} \Leftrightarrow \hspace{.15em}
\bm{b} = \zerovector }$).
\en{For}\ru{Для}~${\bm{a} = \zerovector}$,
${\hspace{.2ex} \bm{A}(\zerovector) \hspace{-0.2ex} = \zerovector \hspace{-0.2ex} \times \hspace{-0.2ex} \bm{E} = \zerobivalent}$.

\begin{center}
\pmb{\checkmark} $\bm{a}$\en{ is} \en{unique}\ru{уникален} \en{for}\ru{для}~${\hspace{-0.2ex}\bm{A}}$
\end{center}

\en{And yet}\ru{А~вот}
\en{about}\ru{про}
\en{the~reciprocal}\ru{обратное}
\en{relation}\ru{соотношение}~${\hspace{-0.2ex}\bm{A} \mapsto \bm{a}\:: \hspace{.4em} \bm{a} \narroweq \bm{a}(\bm{A})}$.
\en{By}\ru{По}~\eqref{thedirectrelationoftheunitdyadwiththepermutationsparity},
\en{the~unit dyad}\ru{единичная диада}~${\hspace{-0.1ex}\UnitDyad}$
\en{via}\ru{через}~${\hspace{-0.1ex}\permutationsparitytensor}$
%
\begin{equation*}
\UnitDyad = - \hspace{.2ex} \smalldisplaystyleonehalf \hspace{.4ex} \permutationsparitytensor \dotdotp \hspace{-0.1ex} \permutationsparitytensor
\hspace{.2ex} ,
\end{equation*}
%
\en{and}\ru{и}~\en{it is}\ru{это}
\en{neutral}\ru{нейтрально}~\eqref{definingpropertyoftheidentitytensor}
\en{for}\ru{для}
\en{the~}\hbox{\hspace{-0.2ex}\inquotes{${\dotp\hspace{.22ex}}$}\hspace{-0.2ex}-\en{product}\ru{произведения}}
%
\begin{equation*}
\bm{a}
= \hspace{.1ex}
%%\bm{a} \hspace{.2ex} \dotp \UnitDyad
%%\hspace{.1ex} =
%%\UnitDyad \hspace{.1ex} \dotp \hspace{.1ex} \bm{a}
%%\\
%%& = \hspace{.1ex}
\tikzmark{beginUnitTensorViaPermutationsParityIsNeutralOnTheRightSide}
\bm{a} \hspace{.1ex} \dotp \left( \hspace{-0.3ex} - \hspace{.2ex} \smalldisplaystyleonehalf \hspace{.4ex} \permutationsparitytensor \dotdotp \hspace{-0.1ex} \permutationsparitytensor \right)
\hspace{-0.6ex} \tikzmark{endUnitTensorViaPermutationsParityIsNeutralOnTheRightSide}
=
\tikzmark{beginUnitTensorViaPermutationsParityIsNeutralOnTheLeftSide} \hspace{-0.6ex}
\left( \hspace{-0.3ex} - \hspace{.2ex} \smalldisplaystyleonehalf \hspace{.4ex} \permutationsparitytensor \dotdotp \hspace{-0.1ex} \permutationsparitytensor \right) \hspace{-0.3ex} \dotp \hspace{.1ex} \bm{a}
\tikzmark{endUnitTensorViaPermutationsParityIsNeutralOnTheLeftSide}
\end{equation*}%
\AddUnderBrace[line width=.75pt][-0.2ex,-1.5ex]%
{beginUnitTensorViaPermutationsParityIsNeutralOnTheRightSide}{endUnitTensorViaPermutationsParityIsNeutralOnTheRightSide}{${\scriptstyle
- \sfrac{1\hspace{-0.3ex}}{2} \hspace{.3ex}
a_a \permutationsparitysymbols{abc} \permutationsparitysymbols{cbm} \bm{e}_m
}$}%
\AddUnderBrace[line width=.75pt][-0.1ex,-1.5ex]%
{beginUnitTensorViaPermutationsParityIsNeutralOnTheLeftSide}{endUnitTensorViaPermutationsParityIsNeutralOnTheLeftSide}{${\scriptstyle
- \sfrac{1\hspace{-0.3ex}}{2} \hspace{.3ex}
\bm{e}_h \permutationsparitysymbols{hi\hspace{-0.1ex}j} \permutationsparitysymbols{j\hspace{-0.1ex}ik} a_k
}$}

\noindent
\en{or}\ru{или}
\en{without brackets}\ru{без скобок}
%
\begin{equation*}
\bm{a}
= \hspace{.1ex}
- \hspace{.2ex} \smalldisplaystyleonehalf \hspace{.4ex}
\permutationsparitytensor \dotdotp \hspace{-0.1ex} \permutationsparitytensor \dotp \hspace{.1ex} \bm{a}
\hspace{.1ex} = \hspace{.1ex}
- \hspace{.2ex} \smalldisplaystyleonehalf \hspace{.4ex}
\bm{a} \dotp \permutationsparitytensor \dotdotp \hspace{-0.1ex} \permutationsparitytensor
\hspace{.2ex} .
\end{equation*}

\noindent
\en{Bivalent}\ru{Бивалентный}~${\hspace{-0.2ex}\bm{A}}$
\en{can be}\ru{может быть}
\en{introduced}\ru{введён}
\en{here}\ru{сюда}
\en{as in}\ru{как в}~\eqref{skewsymmetricbivalentfromvector},
${- \bm{A} \hspace{-0.12ex}
= \bm{a} \dotp \permutationsparitytensor
= \hspace{-0.2ex} \permutationsparitytensor \dotp \bm{a}}$,
\en{and }\ru{и~}\en{then}\ru{тогда}
%
\begin{equation}\label{pseudovectorfromsomebivalent}
%%\hspace{.6em} \Rightarrow \hspace{.5em}
\bm{a}(\bm{A}) \hspace{-0.2ex}
= \hspace{.1ex}
\smalldisplaystyleonehalf \hspace{.2ex}
\bm{A} \dotdotp \hspace{-0.1ex} \permutationsparitytensor
= \hspace{.1ex}
\smalldisplaystyleonehalf \hspace{.4ex}
\permutationsparitytensor \dotdotp \hspace{-0.1ex} \bm{A}
\hspace{.2ex} .
\end{equation}

...............

\nopagebreak\vspace{-0.5em}
\begin{equation*}\begin{array}{c}
a_{i} \bm{e}_i \hspace{-0.2ex}
= \smalldisplaystyleonehalf \hspace{.25ex} A_{j\hspace{-0.1ex}k} \permutationsparitysymbols{kj\hspace{-0.06ex}i} \hspace{.2ex} \bm{e}_i \hspace{-0.2ex}
= \smalldisplaystyleonehalf \hspace{.3ex} \permutationsparitysymbols{ikj} \hspace{.1ex} A_{j\hspace{-0.1ex}k} \hspace{.2ex} \bm{e}_i
\hspace{.2ex} ,
\\[.5em]
a_{i} \hspace{-0.2ex} = \smalldisplaystyleonehalf \hspace{.3ex} \permutationsparitysymbols{ikj} \hspace{.1ex} A_{j\hspace{-0.1ex}k} = \hspace{.1em}
\displaystyle \onehalf \scalebox{.9}{$\left[\hspace{-0.25ex} \begin{array}{c}
\permutationsparitysymbols{123} \hspace{.1ex} A_{32} + \permutationsparitysymbols{132} \hspace{.1ex} A_{23} \\
\permutationsparitysymbols{213} \hspace{.1ex} A_{31} + \permutationsparitysymbols{231} \hspace{.1ex} A_{13} \\
\permutationsparitysymbols{312} \hspace{.1ex} A_{21} + \permutationsparitysymbols{321} \hspace{.1ex} A_{12}
\end{array} \right]$} \hspace{-0.2em} = \hspace{.1em}
\displaystyle \onehalf \scalebox{.9}{$\left[\hspace{-0.3ex} \begin{array}{c}
A_{32} - A_{23} \\
A_{13} - A_{31} \\
A_{21} - A_{12}
\end{array} \right]$} .
\vspace{.1em}\end{array}\end{equation*}

..............

\vspace{1em}
\begin{alignat*}{2}
-2 \hspace{.2ex} a_{1} \hspace{-0.1ex}
& = \hspace{.4ex} \tikzmark{permutationsparity123for1FROM} \hspace{-0.4ex} \permutationsparitysymbols{123} \hspace{-0.3ex} \tikzmark{permutationsparity123for1TO} \hspace{.3ex}
\hspace{.4ex} \tikzmark{permutationsparity321for1FROM} \hspace{-0.4ex} \permutationsparitysymbols{321} \hspace{-0.3ex} \tikzmark{permutationsparity321for1TO} \hspace{.4ex}
a_{1} \hspace{-0.1ex} \tikzmark{permutationsparity321with1ENDS}
+ \hspace{.4ex} \tikzmark{permutationsparity132for1FROM} \hspace{-0.4ex} \permutationsparitysymbols{132} \hspace{-0.3ex} \tikzmark{permutationsparity132for1TO} \hspace{.3ex}
\hspace{.4ex} \tikzmark{permutationsparity231for1FROM} \hspace{-0.4ex} \permutationsparitysymbols{231} \hspace{-0.3ex} \tikzmark{permutationsparity231for1TO} \hspace{.4ex}
a_{1} \hspace{-0.1ex} \tikzmark{permutationsparity231with1ENDS}
%%\hspace{.2ex} ,
\\[2.5em]
-2 \hspace{.2ex} a_{2} \hspace{-0.1ex}
& = \hspace{.4ex} \tikzmark{permutationsparity213for2FROM} \hspace{-0.4ex} \permutationsparitysymbols{213} \hspace{-0.3ex} \tikzmark{permutationsparity213for2TO} \hspace{.3ex}
\hspace{.4ex} \tikzmark{permutationsparity312for2FROM} \hspace{-0.4ex} \permutationsparitysymbols{312} \hspace{-0.3ex} \tikzmark{permutationsparity312for2TO} \hspace{.5ex}
a_{2} \hspace{-0.1ex} \tikzmark{permutationsparity312with2ENDS}
+ \hspace{.4ex} \tikzmark{permutationsparity231for2FROM} \hspace{-0.4ex} \permutationsparitysymbols{231} \hspace{-0.3ex} \tikzmark{permutationsparity231for2TO} \hspace{.3ex}
\hspace{.4ex} \tikzmark{permutationsparity132for2FROM} \hspace{-0.4ex} \permutationsparitysymbols{132} \hspace{-0.3ex} \tikzmark{permutationsparity132for2TO} \hspace{.5ex}
a_{2} \hspace{-0.1ex} \tikzmark{permutationsparity132with2ENDS}
%%\hspace{.2ex} ,
\\[2.5em]
-2 \hspace{.2ex} a_{3} \hspace{-0.1ex}
& = \hspace{.4ex} \tikzmark{permutationsparity312for3FROM} \hspace{-0.4ex} \permutationsparitysymbols{312} \hspace{-0.3ex} \tikzmark{permutationsparity312for3TO} \hspace{.3ex}
\hspace{.4ex} \tikzmark{permutationsparity213for3FROM} \hspace{-0.4ex} \permutationsparitysymbols{213} \hspace{-0.3ex} \tikzmark{permutationsparity213for3TO} \hspace{.5ex}
a_{3} \hspace{-0.1ex} \tikzmark{permutationsparity213with3ENDS}
+ \hspace{.4ex} \tikzmark{permutationsparity321for3FROM} \hspace{-0.4ex} \permutationsparitysymbols{321} \hspace{-0.3ex} \tikzmark{permutationsparity321for3TO} \hspace{.3ex}
\hspace{.4ex} \tikzmark{permutationsparity123for3FROM} \hspace{-0.4ex} \permutationsparitysymbols{123} \hspace{-0.3ex} \tikzmark{permutationsparity123for3TO} \hspace{.5ex}
a_{3} \hspace{-0.1ex} \tikzmark{permutationsparity123with3ENDS}
%%\hspace{.2ex} ,
\end{alignat*}%
%
\AddOverBrace[line width=.75pt][-0.1ex,-0.3ex][yshift=-0.4ex]%
{permutationsparity123for1FROM}{permutationsparity123for1TO}{${\scalebox{.66}{$ \left( +1 \right) $}}$}%
\AddOverBrace[line width=.75pt][-0.1ex,-0.3ex][yshift=-0.4ex]%
{permutationsparity321for1FROM}{permutationsparity321for1TO}{${\scalebox{.66}{$ \left( -1 \right) $}}$}%
\AddOverBrace[line width=.75pt][-0.1ex,-0.3ex][yshift=-0.4ex]%
{permutationsparity132for1FROM}{permutationsparity132for1TO}{${\scalebox{.66}{$ \left( -1 \right) $}}$}%
\AddOverBrace[line width=.75pt][-0.1ex,-0.3ex][yshift=-0.4ex]%
{permutationsparity231for1FROM}{permutationsparity231for1TO}{${\scalebox{.66}{$ \left( +1 \right) $}}$}%
\AddUnderBrace[line width=.75pt][.1ex,-0.1ex][yshift=.1ex]%
{permutationsparity321for1FROM}{permutationsparity321with1ENDS}{${\scalebox{.8}{$ - A_{32} $}}$}%
\AddUnderBrace[line width=.75pt][.1ex,-0.1ex][yshift=.1ex]%
{permutationsparity231for1FROM}{permutationsparity231with1ENDS}{${\scalebox{.8}{$ - A_{23} $}}$}%
%
\AddOverBrace[line width=.75pt][-0.1ex,-0.3ex][yshift=-0.4ex]%
{permutationsparity213for2FROM}{permutationsparity213for2TO}{${\scalebox{.66}{$ -1 $}}$}%
\AddOverBrace[line width=.75pt][-0.1ex,-0.3ex][yshift=-0.4ex]%
{permutationsparity312for2FROM}{permutationsparity312for2TO}{${\scalebox{.66}{$ +1 $}}$}%
\AddOverBrace[line width=.75pt][-0.1ex,-0.3ex][yshift=-0.4ex]%
{permutationsparity231for2FROM}{permutationsparity231for2TO}{${\scalebox{.66}{$ +1 $}}$}%
\AddOverBrace[line width=.75pt][-0.1ex,-0.3ex][yshift=-0.4ex]%
{permutationsparity132for2FROM}{permutationsparity132for2TO}{${\scalebox{.66}{$ -1 $}}$}%
\AddUnderBrace[line width=.75pt][.1ex,-0.1ex][yshift=.1ex]%
{permutationsparity312for2FROM}{permutationsparity312with2ENDS}{${\scalebox{.8}{$ - A_{31} $}}$}%
\AddUnderBrace[line width=.75pt][.1ex,-0.1ex][yshift=.1ex]%
{permutationsparity132for2FROM}{permutationsparity132with2ENDS}{${\scalebox{.8}{$ - A_{13} $}}$}%
%
\AddOverBrace[line width=.75pt][-0.1ex,-0.3ex][yshift=-0.4ex]%
{permutationsparity312for3FROM}{permutationsparity312for3TO}{${\scalebox{.66}{$ \left( +1 \right) $}}$}%
\AddOverBrace[line width=.75pt][-0.1ex,-0.3ex][yshift=-0.4ex]%
{permutationsparity213for3FROM}{permutationsparity213for3TO}{${\scalebox{.66}{$ \left( -1 \right) $}}$}%
\AddOverBrace[line width=.75pt][-0.1ex,-0.3ex][yshift=-0.4ex]%
{permutationsparity321for3FROM}{permutationsparity321for3TO}{${\scalebox{.66}{$ \left( -1 \right) $}}$}%
\AddOverBrace[line width=.75pt][-0.1ex,-0.3ex][yshift=-0.4ex]%
{permutationsparity123for3FROM}{permutationsparity123for3TO}{${\scalebox{.66}{$ \left( +1 \right) $}}$}%
\AddUnderBrace[line width=.75pt][.1ex,-0.1ex][yshift=.1ex]%
{permutationsparity213for3FROM}{permutationsparity213with3ENDS}{${\scalebox{.8}{$ - A_{21} $}}$}%
\AddUnderBrace[line width=.75pt][.1ex,-0.1ex][yshift=.1ex]%
{permutationsparity123for3FROM}{permutationsparity123with3ENDS}{${\scalebox{.8}{$ - A_{12} $}}$}


...........


${%
\bm{a}' \hspace{-0.4ex} \times \hspace{-0.2ex} \bm{E} = \hspace{-0.2ex} \bm{A}'
\hspace{.6em} \text{\en{and}\ru{и}} \hspace{.6em}
\bm{a}'\hspace{-0.15ex}' \hspace{-0.4ex} \times \hspace{-0.2ex} \bm{E} = \hspace{-0.2ex} \bm{A}'\hspace{-0.15ex}'%
}$

\textcolor{blue}{\textbold{PROVE IT}}
$\bm{A}$\en{ is} \en{unique}\ru{уникален} \en{for}\ru{для}~${\hspace{-0.2ex}\bm{a}}$\
\textcolor{blue}{\textbold{PROVE IT}}

...............

\textcolor{red}{\en{There is}\ru{Существует}}
\textcolor{blue}{\textbold{PROVE THAT ONLY FOR SKEWSYMMETRIC ${\forall \bm{A} \!=\! \bm{A}^{\mathsf{\!\,A}}}$ IT IS BIJECTION}}
\en{a~bijection}\ru{биекция}\footnote{%
\inquotesx{\en{a~bijective}\ru{биективное} \en{relation}\ru{отношение}}[,]
\inquotesx{\en{a~reciprocally}\ru{взаимно} \en{reversible}\ru{обратимое} \en{mapping}\ru{отображение}}[,]
\inquotes{\en{\hbox{a one-to-one} correspondence}\ru{соответствие \hbox{один-к-одному}}}%
}
\en{between}\ru{между}
\en{antisymmetric}\ru{антисимметричными}
\en{bivalent}\ru{бивалентными}
\en{tensors}\ru{тензорами}
\en{and }\ru{и~}(\en{pseudo}\ru{псевдо})\en{vectors}\ru{векторами}.
\en{The~components}\ru{Компоненты}
\en{of a~skewsymmetric}\ru{кососимметричного}
\en{tensor}\ru{тензора}
\en{are fully described}\ru{полностью описываются}
\en{by the~three numbers}\ru{тремя числами}
xxxxxxxxxxxxxxxxxxxxxxxxxxxxxxxxxxxxxxxxxxxxxx

.........

\en{All in all}\ru{В~общем},
\en{here is}\ru{вот}
\en{the~bijection}\ru{биекция}~${\hspace{-0.2ex}\bm{A} \hspace{-0.3ex} \leftrightarrow \hspace{-0.2ex} \bm{a}}$

\nopagebreak\vspace{-0.25em}
\refstepcounter{equation}
\begin{alignat*}{2}
\bm{A}(\bm{a}) &
= \hspace{.1ex}
- \hspace{.3ex} \bm{a} \dotp \permutationsparitytensor
\hspace{.2ex} = \hspace{.2ex}
\bm{a} \times \hspace{-0.1ex} \UnitDyad
\hspace{.2ex} = \hspace{.1ex}
- \hspace{.2ex} \permutationsparitytensor \dotp \bm{a}
\hspace{.2ex} = \hspace{.1ex}
\UnitDyad \times \bm{a}
\tag{$\theequation^{\raisemath{.15em}{A(a)}}$}\label{companionvector.bijection.pseudovector2skewsymmetricbivalent}
\hspace{.2ex} ,
\\[.3em]
%
\bm{a}(\bm{A}) &
= \hspace{.1ex}
\smalldisplaystyleonehalf \hspace{.3ex} \bm{A} \dotdotp \permutationsparitytensor
= \hspace{.1ex}
\smalldisplaystyleonehalf \hspace{.4ex} \permutationsparitytensor \dotdotp \hspace{-0.2ex} \bm{A}
\tag{$\theequation^{\raisemath{.15em}{a(A)}}$}\label{companionvector.bijection.skewsymmetricbivalent2pseudovector}
\hspace{.2ex} .
\end{alignat*}

\en{Easy to memorize}\ru{Легко запомнить},
\en{the }\inquotes{\en{pseudovector invariant}\ru{псевдовекторный инвариант}}
${\!\bm{A}_{\!\bm{\times}}}$
\en{comes from}\ru{происходит из}
\en{the~original tensor}\ru{оригинального тензора}~${\hspace{-0.2ex}\bm{A}}$
\en{by replacing}\ru{заменой}
\en{a~dyadic product}\ru{диадного произведения}
\en{with a~cross product}\ru{на векторное произведение}

\nopagebreak\vspace{-0.15em}
\begin{equation}\label{pseudovectorinvariant}
\begin{array}{c}
\bm{A}_{\Xcompanion} \equiv A_{i\hspace{-0.1ex}j} \hspace{.25ex} \bm{e}_i \times \bm{e}_j = - \hspace{.1ex} \bm{A} \hspace{.1ex} \dotdotp \permutationsparitytensor
\hspace{.2ex}, \\[.3em]
%
\bm{A}_{\Xcompanion}
\hspace{-0.16ex} =
\left(^{\mathstrut} \hspace{-0.1ex} \bm{a} \times\hspace{-0.1ex} \UnitDyad \hspace{.2ex} \right)_{\hspace{-0.25ex}\Xcompanion}
\hspace{-0.25ex} = \hspace{-0.12ex}
- 2 \hspace{.16ex} \bm{a} \hspace{.2ex},\:\:
\bm{a}
=
- \hspace{.2ex} \smalldisplaystyleonehalf \hspace{.25ex} \bm{A}_{\Xcompanion}
=
- \hspace{.2ex} \smalldisplaystyleonehalf \left(^{\mathstrut} \hspace{-0.1ex} \bm{a} \times\hspace{-0.1ex} \UnitDyad \hspace{.2ex} \right)_{\hspace{-0.25ex}\Xcompanion}
\hspace{-0.2ex} .
\end{array}
\end{equation}

\textcolor{magenta}{\en{Explaination}\ru{Объяснение}}:

\nopagebreak\vspace{-0.1em}
\begin{gather*}
\bm{a} \times\hspace{-0.1ex} \UnitDyad
= - \hspace{.2ex} \smalldisplaystyleonehalf \hspace{.4ex} \bm{A}_{\Xcompanion} \times \UnitDyad
= - \hspace{.2ex} \smalldisplaystyleonehalf \, A_{i\hspace{-0.1ex}j} \hspace{-0.2ex}
\left( \hspace{.1ex} \tikzmark{beginFirstCrossProduct} {\bm{e}_i \times \bm{e}_j} \tikzmark{endFirstCrossProduct} \hspace{.16ex} \right)
\times
\bm{e}_k \bm{e}_k
%
\\[1.5em]
%
= - \hspace{.2ex} \smalldisplaystyleonehalf \hspace{.32ex} A_{i\hspace{-0.1ex}j} \hspace{.12ex}
\tikzmark{beginTwoPermutationParities} \permutationsparitysymbols{ni\hspace{-0.1ex}j} \permutationsparitysymbols{nkp} \tikzmark{endTwoPermutationParities}
\hspace{.32ex} \bm{e}_p \bm{e}_k = - \hspace{.2ex} \smalldisplaystyleonehalf \hspace{.32ex} A_{i\hspace{-0.1ex}j} \hspace{-0.1ex} \left( \bm{e}_j \bm{e}_i - \bm{e}_i \bm{e}_j \right)
%
\\[.8em]
%
\hspace{13.2em}= - \hspace{.2ex} \smalldisplaystyleonehalf \left({ \bm{A}^{\hspace{-0.1em}\T} \hspace{-0.25ex} - \hspace{-0.2ex} \bm{A} \hspace{.2ex}}\right) = \bm{A}^{\mathsf{\hspace{-0.1ex}A}} = \bm{A}.
\end{gather*}
\AddUnderBrace[line width=.75pt][0,-0.25ex]%
              {beginFirstCrossProduct}{endFirstCrossProduct}
{${ \scriptstyle \permutationsparitysymbols{i\hspace{-0.1ex}jn} \bm{e}_n }$}
\AddUnderBrace[line width=.75pt][.25ex,-0.25ex]%
              {beginTwoPermutationParities}{endTwoPermutationParities}%
{${ \scriptstyle \hspace{3.2em}
   \delta_{j\hspace{-0.1ex}p}
   \delta_{ik}
   \, - \;
   \delta_{ip}
   \delta_{j\hspace{-0.1ex}k}
}$}

%%the accompanying (companion) vector
%%сопутствующий вектор

\vspace{-0.6em}
\en{The~companion vector}\ru{Сопутствующий вектор}
\en{can be introduced}\ru{может быть введён}
\en{for any}\ru{для любого}
\en{bivalent tensor}\ru{бивалентного тензора}.
\en{But only}\ru{Но только}
\en{the asymmetric part}\ru{антисимметричная часть}
\en{contributes here}\ru{даёт здесь вклад}:
${
   \bm{C}^{\hspace{.2ex}\mathsf{A}}
   \hspace{-0.1ex} = \hspace{-0.1ex}
   - \hspace{.1ex}
   \onehalf \hspace{.32ex} \bm{C}_{\hspace{-0.1ex}\Xcompanion}
   \hspace{-0.16ex} \times \hspace{-0.16ex}
   \UnitDyad
}$.

\en{For}\ru{Для}
\en{a~symmetric tensor}\ru{симметричного тензора}\en{,}
\en{the~companion vector}\ru{сопутствующий вектор}
\en{is zero}\ru{нулевой}\::

\noindent
\begin{equation*}
\bm{B}_{\Xcompanion} \hspace{-0.25ex}
= \zerovector
\hspace{.6em} \Leftrightarrow \hspace{.4em}
\bm{B} = \bm{B}^{\T} \hspace{-0.32ex} = \bm{B}^{\mathsf{\hspace{.1ex}S}}
\hspace{-0.32ex} .
\end{equation*}

\en{With}\ru{С}~\eqref{pseudovectorinvariant}
\en{the~decomposition}\ru{разложение}
\en{of some tensor}\ru{какого\hbox{-}либо тензора}~$\bm{C}$
\en{into}\ru{на}
\en{the~symmetric}\ru{симметричную}
\en{and}\ru{и}
\en{antisymmetric}\ru{антисимметричную}
\en{parts}\ru{части}
\en{looks like}\ru{выглядит как}

\nopagebreak\vspace{-0.1em}
\begin{equation}\label{symmetricantisymmetricdecompositionofsometensor}
\bm{C} = \bm{C}^{\mathsf{\hspace{.1ex}S}} \hspace{-0.32ex} - \hspace{.1ex} \smalldisplaystyleonehalf \hspace{.32ex} \bm{C}_{\hspace{-0.1ex}\Xcompanion} \hspace{-0.16ex} \times \hspace{-0.16ex} \UnitDyad
\hspace{.1ex} .
\end{equation}

\vspace{-0.8em}\noindent
\en{For}\ru{Для}
\en{a~dyad}\ru{диады}
%
\begin{equation*}
\eqref{vectorcrossvectorcrossidentity}
\hspace{.2em} \Rightarrow \hspace{.2em}
\left( \bm{c} \hspace{-0.1ex} \times \hspace{-0.2ex} \bm{d} \hspace{.2ex} \right) \hspace{-0.12ex} \times \hspace{-0.25ex} \UnitDyad
= \bm{d} \bm{c} - \hspace{-0.1ex} \bm{c} \bm{d} \hspace{.1ex}
= \hspace{-0.1ex} - \hspace{.1ex} 2 \hspace{.15ex} \bm{c}\bm{d}^{\mathsf{\hspace{.3ex}A}}
\hspace{-0.2ex} ,
\hspace{.5em}
\left( \bm{c} \bm{d} \hspace{.2ex} \right)_{\hspace{-0.15ex}\Xcompanion}
\hspace{-0.3ex} = \hspace{.1ex}
\bm{c} \hspace{-0.1ex} \times \hspace{-0.15ex} \bm{d}
\hspace{.2ex} ,
\end{equation*}

\nopagebreak\vspace{-0.1em}\noindent
\en{and its decomposition}\ru{и её разложение}

\nopagebreak\vspace{-0.6em}\begin{equation}\label{symmetricantisymmetricdecompositionofdyad}
\hspace*{1em} \bm{c}\bm{d} \hspace{.1ex} = \hspace{.1ex}
%%\smalldisplaystyleonehalf \hspace{-0.1ex} \left( \bm{c}\bm{d} + \hspace{-0.1ex} \bm{d}\bm{c} \hspace{.1ex} \right) - \hspace{.16ex} \smalldisplaystyleonehalf \hspace{-0.1ex} \left( \bm{d} \bm{c} - \hspace{-0.1ex} \bm{c} \bm{d} \hspace{.2ex} \right) = \hspace{.1ex}
\smalldisplaystyleonehalf \hspace{-0.1ex} \left( \bm{c}\bm{d} + \hspace{-0.1ex} \bm{d}\bm{c} \hspace{.1ex} \right)
- \hspace{.16ex} \smalldisplaystyleonehalf \hspace{-0.1ex} \left( \bm{c} \hspace{-0.1ex} \times \hspace{-0.2ex} \bm{d} \hspace{.2ex} \right) \hspace{-0.1ex} \times \hspace{-0.25ex} \UnitDyad
\hspace{.1ex} .
\end{equation}

