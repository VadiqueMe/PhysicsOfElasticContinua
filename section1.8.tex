\en{\section{Symmetric and antisymmetric tensors}}

\ru{\section{Симметричные и антисимметричные тензоры}}

\label{section:tensors.symmetric+skewsymmetric}

\en{A~tensor}\ru{Тензор}\ru{,}
\en{that does not change}\ru{который не изменяется}
\en{upon a~permutation}\ru{при перестановке}
\en{of some pair of its indices}\ru{какой\hbox{-}либо пары своих индексов}\ru{,}
\en{is called}\ru{называется}
\en{symmetric}\ru{симметричным}
\en{for that pair of indices}\ru{для той пары индексов}.
\en{And if}\ru{А~если}
\en{a~tensor}\ru{тензор}
\en{alternates the~sign}\ru{меняет знак}~($+$/$-$)%
\footnote{${{} \cdot (-1)}$}%
\en{upon a~permutation}\ru{при~перестановке}
\en{of some pair of indices}\ru{какой\hbox{-}нибудь пары индексов},
\en{then}\ru{то}
\en{it is called}\ru{он называется}
\en{antisymmetric}\ru{антисимметричным}
\en{or}\ru{или}
\en{skew-symmetric}\ru{кососимметричным}
\en{for that pair of indices}\ru{для той пары индексов}.

\en{The tensor of the parity of permutations}\ru{Тензор чётности перестановок}~$\permutationsparitytensor$
\en{is antisymmetric}\ru{антисимметричен}
\en{by any pair of indices}\ru{по любой паре индексов},
\en{it is completely}\ru{он полностью}
(\en{absolutely}\ru{соверш\'{е}нно, абсолютно})
\en{antisymmetric}\ru{антисимметричен}
(\en{skew-symmetric}\ru{кососимметричен}).

\en{Tensor of the second complexity}\ru{Тензор второй сложности}~$\bm{B}$
\en{is symmetric}\ru{симметричен}\ru{,}
\en{if}\ru{если}
${\bm{B} = \bm{B}^{\T}}$.
\en{When}\ru{Когда}
\en{the transposing}\ru{транспонирование}
\en{changes the sign of a~tensor}\ru{меняет знак тензора}
${ \bm{A}^{\hspace{-0.1em}\T} = - \bm{A} }$,
\en{then}\ru{тогда}
\en{it is antisymmetric}\ru{он антисимметричен}
(\en{skew-symmetric}\ru{кососимметричен}).

\nopagebreak\vspace{-0.1em}\begin{equation}\begin{array}{c}
\bm{C} = \bm{C}^{\mathsf{\hspace{.1ex}S}} \hspace{-0.1ex}+\hspace{.1ex} \bm{C}^{\mathsf{\hspace{.1ex}A}}\hspace{.1ex} , \;
\bm{C}^{\hspace{.1ex}\T} \hspace{-0.33ex} = \bm{C}^{\mathsf{\hspace{.1ex}S}} \hspace{-0.1ex}-\hspace{.1ex} \bm{C}^{\mathsf{\hspace{.1ex}A}}
\hspace{.2ex} ;
\\[0.2em]
\bm{C}^{\mathsf{\hspace{.1ex}S}} \hspace{-0.22ex} \equiv \hspace{.1ex} \displaystyle \onehalf \left( {\bm{C} + \bm{C}^{\hspace{.1ex}\T}}\hspace{.1ex} \right)\!, \;
\bm{C}^{\mathsf{\hspace{.1ex}A}} \hspace{-0.22ex} \equiv \hspace{.1ex} \displaystyle \onehalf \left( {\bm{C} - \bm{C}^{\hspace{.1ex}\T}}\hspace{.1ex} \right)\!.
\end{array}\end{equation}

\begin{otherlanguage}{russian}

\noindent
\en{For}\ru{Для}
\en{a~dyad}\ru{диады}
${\bm{c}\bm{d} \hspace{.1ex} = \bm{c}\bm{d}^{\mathsf{\hspace{.25ex}S}} \hspace{-0.2ex} + \hspace{.1ex} \bm{c}\bm{d}^{\mathsf{\hspace{.25ex}A}} \hspace{-0.16ex} = \hspace{.1ex}
\smalldisplaystyleonehalf \hspace{-0.1ex} \left( \bm{c}\bm{d} + \hspace{-0.1ex} \bm{d}\bm{c} \hspace{.1ex} \right)
+ \hspace{.16ex} \smalldisplaystyleonehalf \hspace{-0.1ex} \left( \bm{c} \bm{d} - \hspace{-0.1ex} \bm{d} \bm{c} \hspace{.1ex} \right)}$.

Произведение двух симметричных тензоров
${\bm{C}^{\mathsf{\hspace{.1ex}S}} \hspace{-0.1ex}\dotp \bm{D}^{\mathsf{\hspace{.1ex}S}}}$
симметрично далеко не~всегда,
а~лишь когда
${\bm{D}^{\mathsf{\hspace{.1ex}S}} \hspace{-0.1ex}\dotp\hspace{.12ex} \bm{C}^{\mathsf{\hspace{.1ex}S}} \hspace{-0.2ex}=\hspace{.1ex} \bm{C}^{\mathsf{\hspace{.1ex}S}} \hspace{-0.1ex}\dotp \bm{D}^{\mathsf{\hspace{.1ex}S}}\hspace{-0.4ex}}$,
ведь по~\eqref{transposeofdotproductforbivalenttensors}
${\left(\hspace{.08ex} \bm{C}^{\mathsf{\hspace{.1ex}S}} \hspace{-0.1ex}\dotp \bm{D}^{\mathsf{\hspace{.1ex}S}} \hspace{.1ex}\right)^{\hspace{-0.32ex}\T} \hspace{-0.16ex} = \hspace{.1ex} \bm{D}^{\mathsf{\hspace{.1ex}S}} \hspace{-0.1ex}\dotp\hspace{.12ex} \bm{C}^{\mathsf{\hspace{.1ex}S}} \hspace{-0.4ex}}$.



\textcolor{magenta}{В~нечётномерных пространствах любой антисимметричный тензор второй сложности необрат\'{и}м,}
\textcolor{red}{определитель матрицы компонент}
\textcolor{magenta}{для~него\:--- нулевой.}

Существует взаимно\hbox{-}однозначное соответствие
между
антисимметричными тензорами
второй сложности
и~(псевдо)векторами.
Компоненты кососимметричного тензора
полностью определяются тройкой чисел
(диагональные элементы матрицы компонент\:--- нули,
недиагональные\:--- попарно противоположны).
Dot product кососимметричного~${\hspace{-0.2ex}\bm{A}}$
и~какого\hbox{-}нибудь тензора~${\hspace{-0.2ex}{^\mathrm{n}\hspace{-0.12ex}\bm{\xi}}}$
однозначно соответствует cross product’у псевдовектора~$\bm{a}$
и~того~же тензора~${\hspace{-0.2ex}{^\mathrm{n}\hspace{-0.12ex}\bm{\xi}}}$

\nopagebreak\vspace{-0.1em}\begin{equation}\begin{array}{c}
\hspace{1.2em} \bm{b} \hspace{.3ex}
= \bm{A} \hspace{.2ex}\dotp\hspace{.1ex} {^\mathrm{n}\hspace{-0.12ex}\bm{\xi}}
\;\,\Leftrightarrow\;
\bm{a} \hspace{.1ex} \times {^\mathrm{n}\hspace{-0.12ex}\bm{\xi}}
\hspace{.1ex} = \hspace{.15ex} \bm{b} \hspace{.1ex} \;\:\:
\forall \bm{A} \!=\! \bm{A}^{\mathsf{\!\,A}} \;\;
\forall \, {^\mathrm{n}\hspace{-0.12ex}\bm{\xi}} \;\; \forall \,\mathrm{n \!>\! 0}
\hspace{.15ex} ,
\\[.2em]
%
\hspace{1.2em} \bm{d} \hspace{.3ex}
= {^\mathrm{n}\hspace{-0.12ex}\bm{\xi}} \hspace{.2ex} \dotp \bm{A}
\;\,\Leftrightarrow\;
{^\mathrm{n}\hspace{-0.12ex}\bm{\xi}} \hspace{.2ex}\times \bm{a}
\hspace{.1ex} = \hspace{.15ex} \bm{d} \hspace{.1ex} \;\:\:
\forall \bm{A} \!=\! \bm{A}^{\mathsf{\!\,A}} \;\;
\forall \, {^\mathrm{n}\hspace{-0.12ex}\bm{\xi}} \;\; \forall \,\mathrm{n \!>\! 0}
\hspace{.15ex} .
\end{array}\end{equation}

Раскроем это соответствие~${\bm{A} \narroweq \bm{A}(\bm{a})}$:

\nopagebreak\vspace{-0.1em}\begin{equation*}
\begin{array}{r@{\hspace{1ex}}c@{\hspace{1ex}}l}
\bm{A} \hspace{.2ex}\dotp\hspace{.1ex} {^\mathrm{n}\hspace{-0.12ex}\bm{\xi}} & = & \bm{a} \hspace{.1ex} \times {^\mathrm{n}\hspace{-0.12ex}\bm{\xi}}
\\
%
A_{hi} \bm{e}_h \bm{e}_i \hspace{.1ex}\dotp\hspace{.25ex} \xi_{j\hspace{-0.1ex}k \ldots q} \hspace{.2ex} \bm{e}_j \bm{e}_k \ldots \bm{e}_q & = & a_{i} \bm{e}_i \times\hspace{.2ex} \xi_{j\hspace{-0.1ex}k \ldots q} \hspace{.2ex} \bm{e}_j \bm{e}_k \ldots \bm{e}_q
\\[.1em]
%
A_{hj} \hspace{.2ex} \xi_{j\hspace{-0.1ex}k \ldots q} \hspace{.2ex} \bm{e}_h \bm{e}_k \ldots \bm{e}_q & = & a_{i} \permutationsparitysymbols{i\hspace{-0.1ex}jh} \hspace{.2ex} \xi_{j\hspace{-0.1ex}k \ldots q} \hspace{.2ex} \bm{e}_h \bm{e}_k \ldots \bm{e}_q
\\[.1em]
%
A_{hj} & = & a_{i} \permutationsparitysymbols{i\hspace{-0.1ex}jh}
\\[.1em]
%
A_{hj} & = & \! - \, a_{i} \permutationsparitysymbols{ihj}
\\[.2em]
%
\bm{A} & = & \! - \, \bm{a} \dotp \permutationsparitytensor
\end{array}
\end{equation*}

Так~же из ${{^\mathrm{n}\hspace{-0.12ex}\bm{\xi}} \hspace{.2ex} \dotp \bm{A} = {^\mathrm{n}\hspace{-0.12ex}\bm{\xi}} \hspace{.2ex}\times \bm{a}}$ получается ${\bm{A} = - \hspace{.2ex} \permutationsparitytensor \dotp \bm{a}}$.

Или проще, согласно~\eqref{crossproductforanytwotensors}

\nopagebreak\vspace{-0.1em}\begin{equation*}\begin{array}{c}
\bm{A} = \bm{A} \hspace{.1ex} \dotp \UnitDyad = \hspace{.1ex} \bm{a} \times \hspace{-0.1ex} \UnitDyad = - \hspace{.3ex} \bm{a} \dotp \permutationsparitytensor
\hspace{.1ex} ,
\\[.1em]
\bm{A} = \UnitDyad \hspace{.1ex} \dotp \bm{A} = \UnitDyad \times \bm{a} = - \hspace{.2ex} \permutationsparitytensor \dotp \bm{a}
\hspace{.2ex} .
\end{array}\end{equation*}

(Псевдо)вектор~$\bm{a}$ называется сопутствующим для тензора~${\hspace{-0.2ex}\bm{A}}$.

В~общем, для взаимно\hbox{-}однозначного соответствия между~${\hspace{-0.2ex}\bm{A}}$ \en{and}\ru{и}~$\bm{a}$ имеем

\end{otherlanguage}

\nopagebreak\vspace{-0.25em}\begin{equation}\label{companionvector}
\begin{array}{c}
\bm{A} \hspace{.2ex} = \hspace{.1ex} - \hspace{.3ex} \bm{a} \dotp \permutationsparitytensor \hspace{.2ex} = \hspace{.2ex} \bm{a} \times\hspace{-0.1ex} \UnitDyad \hspace{.2ex} = \hspace{.1ex} - \hspace{.2ex} \permutationsparitytensor \dotp \bm{a} \hspace{.2ex} = \hspace{.1ex} \UnitDyad \times \bm{a}
\hspace{.1ex} ,
\\[.3em]
%
\bm{a} = \bm{a} \hspace{.1ex} \dotp \UnitDyad = \bm{a} \dotp \left( \hspace{-0.3ex} - \displaystyle \hspace{.2ex} \smalldisplaystyleonehalf \hspace{.4ex} \permutationsparitytensor \dotdotp \hspace{-0.1ex} \permutationsparitytensor \right) \hspace{-0.4ex} = \hspace{.1ex} \smalldisplaystyleonehalf \hspace{.32ex} \bm{A} \dotdotp \permutationsparitytensor
\hspace{.1ex} .
\end{array}
\end{equation}

\en{The components}\ru{Компоненты}
\en{of a~skew-symmetric tensor}\ru{кососимметричного тензора}~${\hspace{-0.2ex}\bm{A}}$
\en{thru}\ru{через}
\en{the components}\ru{компоненты}
\en{of the accompanying}\ru{сопутствующего}
\en{pseudovector}\ru{псевдовектора}~$\bm{a}$

\nopagebreak\vspace{-0.1em}\begin{equation*}
\begin{array}{c}
\bm{A} = - \, \permutationsparitytensor \dotp \bm{a} = - \hspace{.1ex} \permutationsparitysymbols{i\hspace{-0.1ex}j\hspace{-0.1ex}k} \hspace{.1ex} \bm{e}_i \bm{e}_j a_k ,
\\[.3em]
A_{i\hspace{-0.1ex}j} = - \hspace{.1ex} \permutationsparitysymbols{i\hspace{-0.1ex}j\hspace{-0.1ex}k} \hspace{.1ex} a_k \hspace{.1em} = \hspace{-0.1em}
\scalebox{0.9}[0.9]{$\left[ \begin{array}{ccc}
0 & -a_3 & a_2 \\
a_3 & 0 & -a_1 \\
-a_2 & a_1 & 0
\end{array} \hspace{.25ex}\right]$}
\end{array}
\end{equation*}

\vspace{-0.4em}
\noindent
\en{and vice versa}\ru{и~наоборот}

\nopagebreak\vspace{-0.5em}\begin{equation*}\begin{array}{c}
\bm{a} = \smalldisplaystyleonehalf \, \bm{A} \dotdotp \permutationsparitytensor =
\smalldisplaystyleonehalf \hspace{.25ex} A_{j\hspace{-0.1ex}k} \permutationsparitysymbols{kj\hspace{-0.06ex}i} \hspace{.2ex} \bm{e}_i , \\[0.64em]
a_{i} \hspace{-0.16ex} = \smalldisplaystyleonehalf \hspace{.32ex} \permutationsparitysymbols{ikj} \hspace{.1ex} A_{j\hspace{-0.1ex}k} = \hspace{.1em}
\displaystyle \onehalf \scalebox{0.9}[0.9]{$\left[\hspace{-0.25ex} \begin{array}{c}
\permutationsparitysymbols{123} \hspace{.1ex} A_{32} + \permutationsparitysymbols{132} \hspace{.1ex} A_{23} \\
\permutationsparitysymbols{213} \hspace{.1ex} A_{31} + \permutationsparitysymbols{231} \hspace{.1ex} A_{13} \\
\permutationsparitysymbols{312} \hspace{.1ex} A_{21} + \permutationsparitysymbols{321} \hspace{.1ex} A_{12}
\end{array} \right]$} \hspace{-0.2em} = \hspace{.1em}
\displaystyle \onehalf \scalebox{0.9}[0.9]{$\left[\hspace{-0.3ex} \begin{array}{c}
A_{32} - A_{23} \\
A_{13} - A_{31} \\
A_{21} - A_{12}
\end{array} \right]$} .
\vspace{.1em}\end{array}\end{equation*}

\en{The easy to memorize}\ru{Легко запоминающийся}
\inquotes{\en{pseudovector invariant}\ru{псевдовекторный инвариант}}
${\!\bm{A}_{\!\bm{\times}}}$
\en{comes from}\ru{происходит из}
\en{the original tensor}\ru{оригинального тензора}~${\hspace{-0.2ex}\bm{A}}$
\en{via replacing}\ru{заменой}
\en{the dyadic product}\ru{диадного произведения}
\en{by the cross product}\ru{на векторное произведение}

\nopagebreak\vspace{-0.15em}
\begin{equation}\label{pseudovectorinvariant}
\begin{array}{c}
\bm{A}_{\Xcompanion} \equiv A_{i\hspace{-0.1ex}j} \hspace{.25ex} \bm{e}_i \times \bm{e}_j = - \hspace{.1ex} \bm{A} \hspace{.1ex} \dotdotp \permutationsparitytensor
\hspace{.2ex}, \\[.3em]
%
\bm{A}_{\Xcompanion}
\hspace{-0.16ex} =
\left(^{\mathstrut} \hspace{-0.1ex} \bm{a} \times\hspace{-0.1ex} \UnitDyad \hspace{.2ex} \right)_{\hspace{-0.25ex}\Xcompanion}
\hspace{-0.25ex} = \hspace{-0.12ex}
- 2 \hspace{.16ex} \bm{a} \hspace{.2ex},\:\:
\bm{a}
=
- \hspace{.2ex} \smalldisplaystyleonehalf \hspace{.25ex} \bm{A}_{\Xcompanion}
=
- \hspace{.2ex} \smalldisplaystyleonehalf \left(^{\mathstrut} \hspace{-0.1ex} \bm{a} \times\hspace{-0.1ex} \UnitDyad \hspace{.2ex} \right)_{\hspace{-0.25ex}\Xcompanion}
\hspace{-0.2ex} .
\end{array}
\end{equation}

\textcolor{magenta}{\en{Explaination}\ru{Объяснение}}:

\nopagebreak\vspace{-0.1em}
\begin{gather*}
\bm{a} \times\hspace{-0.1ex} \UnitDyad
= - \hspace{.2ex} \smalldisplaystyleonehalf \hspace{.4ex} \bm{A}_{\Xcompanion} \times \UnitDyad
= - \hspace{.2ex} \smalldisplaystyleonehalf \, A_{i\hspace{-0.1ex}j} \hspace{-0.2ex}
\left( \hspace{.1ex} \tikzmark{beginFirstCrossProduct} {\bm{e}_i \times \bm{e}_j} \tikzmark{endFirstCrossProduct} \hspace{.16ex} \right)
\times
\bm{e}_k \bm{e}_k
%
\\[1.5em]
%
= - \hspace{.2ex} \smalldisplaystyleonehalf \hspace{.32ex} A_{i\hspace{-0.1ex}j} \hspace{.12ex}
\tikzmark{beginTwoPermutationParities} \permutationsparitysymbols{ni\hspace{-0.1ex}j} \permutationsparitysymbols{nkp} \tikzmark{endTwoPermutationParities}
\hspace{.32ex} \bm{e}_p \bm{e}_k = - \hspace{.2ex} \smalldisplaystyleonehalf \hspace{.32ex} A_{i\hspace{-0.1ex}j} \hspace{-0.1ex} \left( \bm{e}_j \bm{e}_i - \bm{e}_i \bm{e}_j \right)
%
\\[.8em]
%
\hspace{13.2em}= - \hspace{.2ex} \smalldisplaystyleonehalf \left({ \bm{A}^{\hspace{-0.1em}\T} \hspace{-0.25ex} - \hspace{-0.2ex} \bm{A} \hspace{.2ex}}\right) = \bm{A}^{\mathsf{\hspace{-0.1ex}A}} = \bm{A}.
\end{gather*}
\AddUnderBrace[line width=.75pt][0,-0.25ex]%
              {beginFirstCrossProduct}{endFirstCrossProduct}
{${ \scriptstyle \permutationsparitysymbols{i\hspace{-0.1ex}jn} \bm{e}_n }$}
\AddUnderBrace[line width=.75pt][.25ex,-0.25ex]%
              {beginTwoPermutationParities}{endTwoPermutationParities}%
{${ \scriptstyle \hspace{3.2em}
   \delta_{j\hspace{-0.1ex}p}
   \delta_{ik}
   \, - \;
   \delta_{ip}
   \delta_{j\hspace{-0.1ex}k}
}$}

%%the accompanying vector
%%сопутствующий вектор

\vspace{-0.6em}
\en{The accompanying vector}\ru{Сопутствующий вектор}
\en{can be introduced}\ru{может быть введён}
\en{for any}\ru{для любого}
\en{bivalent tensor}\ru{бивалентного тензора}.
\en{But only}\ru{Но только}
\en{the asymmetric part}\ru{антисимметричная часть}
\en{contributes here}\ru{даёт здесь вклад}:
${
   \bm{C}^{\hspace{.2ex}\mathsf{A}}
   \hspace{-0.1ex} = \hspace{-0.1ex}
   - \hspace{.1ex}
   \onehalf \hspace{.32ex} \bm{C}_{\hspace{-0.1ex}\Xcompanion}
   \hspace{-0.16ex} \times \hspace{-0.16ex}
   \UnitDyad
}$.

\en{For}\ru{Для}
\en{a~symmetric tensor}\ru{симметричного тензора}\en{,}
\en{the accompanying vector}\ru{сопутствующий вектор}
\en{is zero}\ru{это нуль}:

\noindent
\begin{equation*}
\bm{B}_{\Xcompanion} \hspace{-0.25ex}
= \bm{0} \hspace{.25ex} \Leftrightarrow
\bm{B} = \bm{B}^{\T} \hspace{-0.32ex} = \bm{B}^{\mathsf{\hspace{.1ex}}}
\hspace{-0.32ex} .
\end{equation*}

\en{With}\ru{С}~\eqref{pseudovectorinvariant}
\en{the decomposition of some tensor}\ru{разложение какого\hbox{-}либо тензора}~$\bm{C}$
\en{on}\ru{на}
\en{the symmetric}\ru{симметричную}
\en{and}\ru{и}
\en{the antisymmetric}\ru{антисимметричную}
\en{parts}\ru{части}
\en{looks like}\ru{выглядит как}

\nopagebreak\vspace{-0.1em}\begin{equation}\label{symmetricantisymmetricdecompositionofsometensor}
\bm{C} = \bm{C}^{\mathsf{\hspace{.1ex}S}} \hspace{-0.32ex} - \hspace{.1ex} \smalldisplaystyleonehalf \hspace{.32ex} \bm{C}_{\hspace{-0.1ex}\Xcompanion} \hspace{-0.16ex} \times \hspace{-0.16ex} \UnitDyad
\hspace{.1ex} .
\end{equation}




\vspace{-0.8em}\noindent
\en{For}\ru{Для}
\en{a~dyad}\ru{диады}

\nopagebreak
{\centering \eqref{vectorcrossvectorcrossidentity}~$\Rightarrow$~${\left( \bm{c} \hspace{-0.1ex} \times \hspace{-0.2ex} \bm{d} \hspace{.2ex} \right) \hspace{-0.12ex} \times \hspace{-0.25ex} \UnitDyad = \bm{d} \bm{c} - \hspace{-0.1ex} \bm{c} \bm{d} \hspace{.1ex} = \hspace{-0.1ex} - \hspace{.1ex} 2 \hspace{.15ex} \bm{c}\bm{d}^{\mathsf{\hspace{.3ex}A}}\hspace{-0.2ex}}$,\hspace{.32em}
${\left( \bm{c} \bm{d} \hspace{.2ex} \right)_{\hspace{-0.15ex}\Xcompanion} \hspace{-0.32ex} =
\hspace{.1ex} \bm{c} \hspace{-0.1ex} \times \hspace{-0.15ex} \bm{d} \hspace{.12ex}}$, \par}

\nopagebreak\vspace{-0.1em}\noindent
\en{and its decomposition}\ru{и её разложение}

\nopagebreak\vspace{-0.6em}\begin{equation}\label{symmetricantisymmetricdecompositionofdyad}
\hspace*{1em} \bm{c}\bm{d} \hspace{.1ex} = \hspace{.1ex}
%%\smalldisplaystyleonehalf \hspace{-0.1ex} \left( \bm{c}\bm{d} + \hspace{-0.1ex} \bm{d}\bm{c} \hspace{.1ex} \right) - \hspace{.16ex} \smalldisplaystyleonehalf \hspace{-0.1ex} \left( \bm{d} \bm{c} - \hspace{-0.1ex} \bm{c} \bm{d} \hspace{.2ex} \right) = \hspace{.1ex}
\smalldisplaystyleonehalf \hspace{-0.1ex} \left( \bm{c}\bm{d} + \hspace{-0.1ex} \bm{d}\bm{c} \hspace{.1ex} \right)
- \hspace{.16ex} \smalldisplaystyleonehalf \hspace{-0.1ex} \left( \bm{c} \hspace{-0.1ex} \times \hspace{-0.2ex} \bm{d} \hspace{.2ex} \right) \hspace{-0.1ex} \times \hspace{-0.25ex} \UnitDyad
\hspace{.1ex} .
\end{equation}

