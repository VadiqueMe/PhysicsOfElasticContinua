\en{\section{Measures (tensors) of deformation}}

\ru{\section{Меры (тензоры) деформации}}

\label{section:deformationtensors}

\en{And this}\ru{А это}\en{ is}\ru{\:---}
\en{where}\ru{где}
\ru{возникает }\en{the extra complexity}\ru{экстра сложность}\en{ arose}.
%
\en{Although}\ru{Хотя},
\en{multi\-variance}\ru{многовариантность}
\en{is often seen}\ru{нередко видится}
\en{as a~big gift}\ru{как большой дар}.

\en{The motion gradient}\ru{Градиент движения}~$\bm{F}$
\en{characterizes}\ru{характеризует}
\en{both the~deformation of a~body}\ru{и~деформацию тела}\ru{,}
\en{and the~rotation of a~body as a~whole}\ru{и~поворот тела как~целого}.
%
\en{The deformation-only tensors}\ru{Тензорами лишь-деформации}
\en{are}\ru{являются}
\en{the stretch tensors}\ru{тензоры искажений}~${\bm{U}\hspace{-0.25ex}}$
\en{and}\ru{и}~${\bm{V}\hspace{-0.1ex}}$
\en{from the polar decomposition}\ru{из полярного разложения}
${\bm{F} \hspace{-0.1ex} = \rotationtensor \dotp \hspace{.25ex} \bm{U} \hspace{-0.2ex} = \bm{V} \hspace{-0.3ex} \dotp \hspace{.15ex} \rotationtensor}$,
\en{as well as}\ru{так~же как и}~\en{another tensors}\ru{другие тензоры},
\en{originating}\ru{происходящие}
\en{from}\ru{от}~${\bm{U}\hspace{-0.25ex}}$
\ru{или\,(и)}\en{or\,(and)}~${\bm{V}\hspace{-0.3ex}}$.

\en{The widely used ones are}\ru{Широко используются}
\en{the~}\inquotes{\en{squares}\ru{квадраты}}
\en{of~}${\bm{U}\hspace{-0.25ex}}$
\en{and}\ru{и}~${\bm{V}\hspace{-0.1ex}}$

\nopagebreak\vspace{-0.1em}\begin{equation}\label{deformationtensors.nonlinear}
\begin{array}{c}
\hspace*{-2.33em} \bigl( \hspace{.1ex} \bm{U}^{\hspace{.1ex}2} \hspace{-0.3ex} = \hspace{.22ex} \bigr) \hspace{.8ex}
\bm{U} \hspace{-0.3ex} \dotp \hspace{.1ex} \bm{U} \hspace{-0.2ex}
= \bm{F}^{\hspace{.1ex}\T} \hspace{-0.4ex} \dotp \bm{F}
\equiv \bm{G}
\hspace{.1ex} ,
\\[.1em]
%
\hspace*{-2.33em} \bigl( \hspace{.1ex} \bm{V}^{\hspace{.04ex}2} \hspace{-0.3ex} = \hspace{.22ex} \bigr) \hspace{.8ex}
\bm{V} \hspace{-0.3ex} \dotp \bm{V} \hspace{-0.25ex}
= \bm{F} \dotp \bm{F}^{\hspace{.1ex}\T} \hspace{-0.36ex}
\equiv \mathboldPhi
\hspace{.1ex} .
\end{array}
\end{equation}

\vspace{-0.25em}\noindent
\en{These are}\ru{Это}
\en{the }\ru{тензор деформации }Green’\en{s}\ru{а}\en{ deformation tensor}
(\en{or}\ru{или}
\en{the right}\ru{правый}
\ru{тензор }Cauchy--Green\ru{’а}\en{ tensor})~$\bm{G}$
\en{and}\ru{и}
\en{the }\ru{тензор деформации }Finger’\en{s}\ru{а}\en{ deformation tensor}
(\en{or}\ru{или}
\en{the left}\ru{левый}
\ru{тензор }Cauchy--Green\ru{’а}\en{ tensor})~$\mathboldPhi$.
%
\en{They have}\ru{У~них есть}
\en{the~convenient link}\ru{удобная связь}
\en{with}\ru{с}~\en{the motion gradient}\ru{градиентом движения}~$\bm{F}$,
\en{without}\ru{без}
\en{calculating square roots}\ru{вычисления квадратных корней}
(\en{as}\ru{как}
\en{it’s needed}\ru{это нужно}
\en{for}\ru{для}~${\bm{U}\hspace{-0.25ex}}$
\en{and}\ru{и}~${\bm{V}\hspace{-0.25ex}}$%
).
%
\en{That’s}\ru{Таков\'{а}}
\en{the big reason}\ru{больш\'{а}я причина}\ru{,}
\en{why}\ru{почему}
\en{tensors}\ru{тензоры}~$\bm{G}$
\en{and}\ru{и}~$\mathboldPhi$
\en{are so widely used}\ru{так широк\'{о} используются}.

\en{Tensor}\ru{Тензор}~$\bm{G}$
\en{was first used}\ru{впервые использовал}
\en{by }George Green\hspace{-0.1ex}%
\footnote{%
\href{https://en.wikipedia.org/wiki/George_Green_(mathematician)}{\bookauthor{Green, George}}.
\href{https://hdl.handle.net/2027/mdp.39015027059651?urlappend=\%3Bseq=133}{(1839) On the~propagation of~light in crystallized media. \emph{Transactions of the~Cambridge Philosophical Society.} 1842, vol.\:7, part~II, pages 121\hbox{--}140.}
}\hspace{-0.5ex}.

\en{An~inversion}\ru{Обращение}
\en{of~}$\mathboldPhi$
\en{and}\ru{и}~$\bm{G}$
\en{gives}\ru{даёт} \en{the two more}\ru{ещё два} \en{deformation tensors}\ru{тензора деформации}

\nopagebreak\vspace{-0.2em}\begin{equation}\label{moredeformationtensors.nonlinear}
\begin{array}{c}
\bm{V}^{\expminustwo} \hspace{-0.25ex}
= \mathboldPhi^{\expminusone} \hspace{-0.2ex}
= \hspace{-0.1ex} \left( \bm{F} \dotp \bm{F}^{\hspace{.1ex}\T} \hspace{.1ex} \right)^{\hspace{-0.33ex}\expminusone} \hspace{-0.4ex}
= \bm{F}^{\expminusT} \hspace{-0.3ex} \dotp \bm{F}^{\expminusone} \hspace{-0.25ex}
\equiv {^2\hspace{-0.2ex}\bm{c}}
\hspace{.2ex} ,
\\
%
\bm{U}^{\expminustwo} \hspace{-0.25ex}
= \bm{G}^{\hspace{.12ex}\expminusone} \hspace{-0.2ex}
= \hspace{-0.1ex} \left( \bm{F}^{\hspace{.1ex}\T} \hspace{-0.4ex} \dotp \bm{F} \hspace{.2ex} \right)^{\hspace{-0.33ex}\expminusone} \hspace{-0.4ex}
= \bm{F}^{\expminusone} \hspace{-0.3ex} \dotp \bm{F}^{\expminusT} \hspace{-0.3ex}
\equiv {^2\hspace{-0.4ex}\bm{f}}
\hspace{-0.1ex} ,
\end{array}
\end{equation}

\vspace{-0.2em}\noindent
\en{each of~which}\ru{каждый из~которых}
\en{is sometimes called}\ru{иногда называется}
\ru{тензором }\en{the~}\href{https://en.wikipedia.org/wiki/Gabrio_Piola}{Piola}\en{ tensor}
\en{or}\ru{или}
\ru{тензором }\en{the~}\href{https://en.wikipedia.org/wiki/Josef_Finger}{Finger}\ru{’а}\en{ tensor}.
%
\en{The~inverse}\ru{Обратный}
\en{of the left}\ru{к~левому}
\ru{тензору }Cauchy--Green\ru{’а}\en{ tensor}~${\hspace{-0.1ex}\mathboldPhi\hspace{.1ex}}$
\en{is known as}\ru{известен как}
\ru{тензор деформации }\en{the }Cauchy\en{ deformation tensor}~${\hspace{-0.2ex}{^2}\hspace{-0.2ex}\bm{c}}$.

\en{The~components}\ru{Компоненты}
\en{of these tensors}\ru{этих тензоров}\en{ are}

\nopagebreak\vspace{-0.1em}
\begin{equation*}\label{componentsofdeformationtensors}
\begin{array}{r@{\hspace{.66em}}l}
\bm{G} = \initiallocationvector^i \currentlocationvector_\differentialindex{i} \hspace{-0.1ex} \dotp \currentlocationvector_\differentialindex{\hspace{-0.1ex}j} \initiallocationvector^j \hspace{-0.25ex}
= G_{\hspace{-0.15ex}i\hspace{-0.1ex}j} \hspace{.1ex} \initiallocationvector^i \initiallocationvector^j
\hspace{-0.3ex} , &
G_{\hspace{-0.15ex}i\hspace{-0.1ex}j} \hspace{-0.2ex} \equiv
\currentlocationvector_\differentialindex{i} \hspace{-0.1ex} \dotp \currentlocationvector_\differentialindex{\hspace{-0.1ex}j}
\hspace{.1ex} ,
\\[.25em]
%
{{^2}\hspace{-0.4ex}\bm{f}} \hspace{-0.2ex} = \initiallocationvector_\differentialindex{i} \currentlocationvector^{i} \hspace{-0.2ex} \dotp \currentlocationvector^j \initiallocationvector_\differentialindex{\hspace{-0.1ex}j} \hspace{-0.2ex}
= G^{\hspace{.1ex}i\hspace{-0.1ex}j} \initiallocationvector_\differentialindex{i} \initiallocationvector_\differentialindex{\hspace{-0.1ex}j}
, &
G^{\hspace{.1ex}i\hspace{-0.1ex}j} \hspace{-0.2ex} \equiv
\currentlocationvector^{i} \hspace{-0.2ex} \dotp \hspace{-0.15ex} \currentlocationvector^j
\hspace{-0.3ex} ,
\\[.25em]
%
{{^2}\hspace{-0.2ex}\bm{c}} = \hspace{-0.1ex} \currentlocationvector^{i} \initiallocationvector_\differentialindex{i} \hspace{-0.1ex} \dotp \hspace{.1ex} \initiallocationvector_\differentialindex{\hspace{-0.1ex}j} \currentlocationvector^{j} \hspace{-0.25ex}
= \textsl{g}_{i\hspace{-0.1ex}j} \hspace{.1ex} \currentlocationvector^{i} \currentlocationvector^{j}
\hspace{-0.3ex} , &
\textsl{g}_{i\hspace{-0.1ex}j} \hspace{-0.15ex} \equiv
\initiallocationvector_\differentialindex{i} \hspace{-0.1ex} \dotp \hspace{.1ex} \initiallocationvector_\differentialindex{\hspace{-0.1ex}j}
\hspace{.1ex} ,
\\[.25em]
%
\mathboldPhi = \hspace{-0.1ex} \currentlocationvector_\differentialindex{i} \hspace{.1ex} \initiallocationvector^i \hspace{-0.15ex} \dotp \hspace{.1ex} \initiallocationvector^j \currentlocationvector_\differentialindex{\hspace{-0.1ex}j} \hspace{-0.2ex}
= \textsl{g}^{\hspace{.2ex}i\hspace{-0.1ex}j} \currentlocationvector_\differentialindex{i} \currentlocationvector_\differentialindex{\hspace{-0.1ex}j}
\hspace{.1ex} , &
\textsl{g}^{\hspace{.2ex}i\hspace{-0.1ex}j} \hspace{-0.15ex} \equiv
\initiallocationvector^{\hspace{.1ex}i} \hspace{-0.15ex} \dotp \hspace{.1ex} \initiallocationvector^{\hspace{.1ex}j}
\hspace{-0.3ex} ,
\end{array}
\end{equation*}

\vspace{-0.2em}\noindent
\en{and }\ru{и~}\en{they}\ru{они}
\en{coincide}\ru{совпадают}
\en{with the components}\ru{с~компонентами}
\en{of the unit}\ru{единичного}~(\en{metric}\ru{метрического})
\en{tensor}\ru{тензора}

\nopagebreak\vspace{-0.4em}
\begin{multline*}
\shoveleft{
   \hspace{3em} \UnitDyad
   = \hspace{-0.2ex}
   \currentlocationvector_\differentialindex{i} \currentlocationvector^{i}
   \hspace{-0.2ex} =
   G_{\hspace{-0.15ex}i\hspace{-0.1ex}j} \currentlocationvector^{i} \currentlocationvector^j
   \hspace{-0.25ex} = \hspace{-0.2ex}
   \currentlocationvector^{i} \currentlocationvector_\differentialindex{i}
   \hspace{-0.2ex} =
   G^{\hspace{.1ex}i\hspace{-0.1ex}j} \hspace{-0.2ex} \currentlocationvector_\differentialindex{i} \currentlocationvector_\differentialindex{\hspace{-0.1ex}j}
\hfill }
\\[-0.2em]
%
= \initiallocationvector^i \initiallocationvector_\differentialindex{i} \hspace{-0.2ex}
= \textsl{g}^{\hspace{.2ex}i\hspace{-0.1ex}j} \initiallocationvector_\differentialindex{i} \initiallocationvector_\differentialindex{\hspace{-0.1ex}j} \hspace{-0.2ex}
= \initiallocationvector_\differentialindex{i} \initiallocationvector^i \hspace{-0.25ex}
= \textsl{g}_{i\hspace{-0.1ex}j} \hspace{.1ex} \initiallocationvector^i \initiallocationvector^j
\hspace{-0.2ex} ,
\end{multline*}

\vspace{-0.2em}\noindent
\en{but}\ru{но}
\en{the components’ bases}\ru{базисы компонент}
\en{are different}\ru{разные}.
%
\en{Using}\ru{Пользуясь}
\en{only}\ru{только}
\en{the~index notation}\ru{индексной записью},
\en{it’s easy to get confused}\ru{легко запутаться}
\en{due to the~similarity}\ru{из\hbox{-}за сходства}
\en{between}\ru{между}
\en{the~unit}\ru{единичным}
\en{tensor}\ru{тензором}~$\UnitDyad$
\en{and the~strain tensors}\ru{и~тензорами деформации}
$\bm{G}$,
$\mathboldPhi$,
${{^2}\hspace{-0.4ex}\bm{f}\hspace{-0.1ex}}$,
${{^2}\hspace{-0.2ex}\bm{c}}$.
%
\en{The direct indexless notation}\ru{Прямая безиндексная запись}
\en{has}\ru{имеет}
\ru{тут }\en{the obvious}\ru{явное}
\en{advantage}\ru{преимущество}\en{ here}.

\en{As was mentioned}\ru{Как упоминалось}
\en{in}\ru{в}~\chapterdotsectionref{chapter:mathapparatus}{section:polardecomposition},
\en{the~invariants}\ru{инварианты}
\en{of~the~stretch tensors}\ru{тензоров искажений}
$\bm{U}\hspace{-0.1em}$
\en{and}\ru{и}~$\bm{V}\hspace{-0.1em}$
\en{are the~same}\ru{одинаковые}.
%
\en{If}\ru{Если}
${w_{i}}$~\en{are}\ru{это}
\en{the~three eigenvalues}\ru{три собственных значения}
\en{of~}${\bm{U}\hspace{-0.1em}}$
\en{and}\ru{и}~${\bm{V}\hspace{-0.2em}}$,
\en{that is}\ru{то есть}
\en{the~roots}\ru{корни}
\en{of the characteristic equations}\ru{характеристических уравнений}
\en{for these tensors}\ru{для этих тензоров},
\en{then}\ru{то}
\en{here are}\ru{вот}
\en{their}\ru{их}
\en{invariants}\ru{инварианты}:

\nopagebreak\vspace{-0.3em}
\begin{gather*}
\anyfirstinvariantof{\bm{U}}
= \anyfirstinvariantof{\bm{V}}
= \trace{\bm{U}} \hspace{-0.25ex}
= \trace{\hspace{-0.2ex}\bm{V}} \hspace{-0.25ex}
= \textstyle\sum \hspace{-0.2ex} U_{\hspace{-0.2ex}j\hspace{-0.2ex}j}
= \textstyle\sum \hspace{-0.2ex} V_{\hspace{-0.1ex}j\hspace{-0.2ex}j}
= \textstyle\sum \hspace{-0.2ex} w_{i}
\hspace{.1ex} ,
\\
%
\anysecondinvariantof{\bm{U}} \hspace{-0.2ex}
= \anysecondinvariantof{\bm{V}} \hspace{-0.2ex}
= \vphantom{\textstyle\sum}
w_1 w_2 \hspace{-0.2ex}
+ w_1 w_3 \hspace{-0.2ex}
+ w_2 \hspace{.15ex} w_3
\hspace{.1ex} ,
\\
%
\anythirdinvariantof{\bm{U}} \hspace{-0.2ex}
= \anythirdinvariantof{\bm{V}} \hspace{-0.2ex}
= \vphantom{\textstyle\sum}
w_1 w_2 \hspace{.15ex} w_3
\hspace{.1ex} .
\end{gather*}

\en{The invariants}\ru{Инварианты}
\en{of~}$\bm{G}$
\en{and}\ru{и}~$\mathboldPhi$
\en{coincide too}\ru{тоже совпадают}:

\nopagebreak
\begin{equation*}
\anyfirstinvariantof{\bm{G}}
\hspace{-0.2ex}
= \anyfirstinvariantof{\mathboldPhi}
\hspace{.1ex} , \dots
\end{equation*}

\en{Without}\ru{Без}
\en{deformation}\ru{деформации}

\nopagebreak\vspace{-0.2em}\begin{equation*}
\bm{F} = \bm{U} \hspace{-0.3ex} = \bm{V} \hspace{-0.3ex} = \bm{G} = \mathboldPhi = {^2}\hspace{-0.4ex}\bm{f} = {^2}\hspace{-0.2ex}\bm{c} = \UnitDyad
\hspace{.1ex}
,
\end{equation*}

\vspace{-0.2em}\noindent
\en{thus}\ru{поэтому}
\en{as characteristics of~deformation}\ru{как характеристики деформации}
\en{it’s worth taking}\ru{ст\'{о}ит взять}
\en{the differences}\ru{разности}
\en{like}\ru{типа}
${\bm{U} \hspace{-0.2ex} - \UnitDyad}$,
${\bm{U} \hspace{-0.3ex} \dotp \bm{U} \hspace{-0.2ex} - \UnitDyad}$, \dots

...

\subsection*{The right Cauchy\hbox{--}Green deformation tensor}

George Green discovered a deformation tensor known as the right Cauchy\hbox{--}Green deformation tensor or Green’s deformation tensor

\nopagebreak\begin{equation*}
\bm{G}
= \bm{F}^{\hspace{.1ex}\T} \hspace{-0.5ex} \dotp \bm{F}
= \bm{U}^{2}
\hspace{1em} \text{\en{or}\ru{или}} \hspace{1em}
G_{i\hspace{-0.1ex}j} \hspace{-0.2ex}
= F_{k' i} \hspace{.25ex} F_{k' \hspace{-0.15ex}j} \hspace{-0.15ex}
= \frac{\partial \hspace{-0.1ex} x_{\hspace{-0.1ex}k'}}{\partial \mathcircabove{x}_{i}} \hspace{.15ex} \frac{\partial \hspace{-0.1ex} x_{\hspace{-0.1ex}k'}}{\partial \mathcircabove{x}_{\hspace{-0.2ex}j}}
\hspace{.1ex} .
\end{equation*}

This tensor \textcolor{magenta}{gives the~\inquotes{square} of local change in distances} due to deformation:
${\displaystyle d\currentlocationvector \dotp d\currentlocationvector = d\initiallocationvector \dotp \bm{G} \dotp d\initiallocationvector}$

The most popular invariants of~${\bm{G}}$ are
%%used in expressions for the potential energy of elastic deformations of an~isotropic body.
\[
\begin{array}{r@{\hspace{.25em}}c@{\hspace{.33em}}l}
\anyfirstinvariantof{\bm{G}} & \equiv &
\trace{\bm{G}}
= G_{ii} \hspace{-0.2ex} = \gamma_{1}^{2} + \gamma_{2}^{2} + \gamma_{3}^{2}
\\[.25em]
%
\anysecondinvariantof{\bm{G}} & \equiv &
\smalldisplaystyleonehalf \bigl( G_{\hspace{-0.2ex}j\hspace{-0.1ex}j}^{\hspace{.25ex}2} \hspace{-0.1ex} - G_{ik} G_{ki} \hspace{.1ex} \bigr) \hspace{-0.25ex}
= \gamma_{1}^{2}\gamma_{2}^{2} + \gamma_{2}^{2}\gamma_{3}^{2} + \gamma_{3}^{2}\gamma_{1}^{2}
\\[.4em]
%
\anythirdinvariantof{\bm{G}} & \equiv &
\determinant \hspace{.1ex} \bm{G}
= \gamma_{1}^{2}\gamma_{2}^{2}\gamma_{3}^{2}
\end{array}
\]
where ${\gamma_{i}\hspace{-0.2ex}}$ are stretch ratios for unit fibers that are initially oriented along directions of eigenvectors of the right stretch tensor~${\bm{U}\hspace{-0.2ex}}$.

\subsection*{The inverse of Green’s deformation tensor}

Sometimes called the Finger tensor or the Piola tensor, the~inverse of the right Cauchy\hbox{--}Green deformation tensor

\nopagebreak\vspace{-0.25em}\begin{equation*}
{^2\hspace{-0.4ex}\bm{f}}
= \bm{G}^{\expminusone} \hspace{-0.25ex}
= \bm{F}^{\expminusone} \hspace{-0.4ex} \dotp \bm{F}^{\expminusT}
\hspace{1em} \text{\en{or}\ru{или}} \hspace{1em}
f_{i\hspace{-0.1ex}j} \hspace{-0.2ex} = \frac{\partial \mathcircabove{x}_{i}}{\partial x_{\hspace{-0.1ex}k'}} \hspace{.15ex} \frac{\partial \mathcircabove{x}_{\hspace{-0.2ex}j}}{\partial x_{\hspace{-0.1ex}k'}}
\end{equation*}

\subsection*{The left Cauchy\hbox{--}Green or Finger deformation tensor}

Swapping multipliers in the formula for the right Green–Cauchy deformation tensor leads to the left Cauchy\hbox{--}Green deformation tensor, defined as

\nopagebreak\vspace{-0.2em}\begin{equation*}
\mathboldPhi
= \bm{F} \dotp \bm{F}^{\hspace{.1ex}\T} \hspace{-0.4ex}
= \bm{V}^{2}
\hspace{1em} \text{\en{or}\ru{или}} \hspace{1em}
\Phi_{i\hspace{-0.1ex}j} \hspace{-0.2ex}
= \frac{\partial x_{i}}{\partial \mathcircabove{x}_{k}} \hspace{.15ex} \frac{\partial x_{\hspace{-0.15ex}j}}{\partial \mathcircabove{x}_{k}}
\end{equation*}

The left Cauchy\hbox{--}Green deformation tensor is often called the Finger’s deformation tensor, named after Josef Finger (1894).

Invariants of ${\mathboldPhi}$ are also used in expressions for strain energy density functions.
The conventional invariants are defined as

\nopagebreak\begin{equation*}
\begin{aligned}
I_{1} & \equiv \Phi_{ii} = \lambda_{1}^{2} + \lambda_{2}^{2} + \lambda_{3}^{2}
\\
%
I_{2} & \equiv \tfrac{1}{2} \bigl( \Phi_{ii}^{2} - \Phi_{jk}\Phi_{kj} \bigr) = \lambda_{1}^{2}\lambda_{2}^{2} + \lambda_{2}^{2}\lambda_{3}^{2} + \lambda_{3}^{2}\lambda_{1}^{2}
\\
%
I_{3} & \equiv \det \mathboldPhi = \Jacobian^{\hspace{.1ex}2} \hspace{-0.4ex} = \lambda_{1}^{2}\lambda_{2}^{2}\lambda_{3}^{2}
\end{aligned}
\end{equation*}

\vspace{-0.2em}\noindent
(${\Jacobian \equiv \det{\bm{F}}}$\en{ is}\ru{\:---}
\en{the~Jacobian}\ru{якобиан},
\en{determinant of the motion gradient}\ru{определитель градиента движения})

\subsection*{The Cauchy deformation tensor}

The Cauchy deformation tensor is defined as the~inverse of the left Cauchy\hbox{--}Green deformation tensor

\nopagebreak\vspace{-0.4em}\begin{equation*}
{^2\hspace{-0.2ex}\bm{c}} = \mathboldPhi^{\expminusone} \hspace{-0.25ex}
= \bm{F}^{\expminusT} \hspace{-0.4ex} \dotp \bm{F}^{\expminusone}
\hspace{1em} \text{\en{or}\ru{или}} \hspace{1em}
c_{i\hspace{-0.1ex}j} \hspace{-0.2ex}
= \frac{\partial \mathcircabove{x}_{k}}{\partial \hspace{-0.1ex} x_{i}} \hspace{.15ex} \frac{\partial \mathcircabove{x}_{k}}{\partial \hspace{-0.1ex} x_{\hspace{-0.15ex}j}}
\end{equation*}

${\displaystyle d\initiallocationvector \dotp d\initiallocationvector = d\currentlocationvector \dotp {^2\hspace{-0.2ex}\bm{c}} \dotp d\currentlocationvector}$

This tensor is also called the Piola tensor or the Finger tensor in rheology and fluid dynamics literature.

\subsection*{Finite strain tensors}

The concept of \emph{strain} is used to evaluate how much a~given displacement differs locally from a~body displacement as a~whole (a~\inquotes{rigid body displacement}).
One of such strains for large (finite) deformations is the \emph{Green strain tensor}
(\emph{Green\hbox{--}Lagrangian strain tensor}, \emph{Green\hbox{--}Saint\hbox{-\hspace{-0.2ex}}Venant strain tensor}).
It measures how much $\bm{G}$ differs from~$\UnitDyad$

\nopagebreak\begin{equation}
\label{theGreenFiniteStrainTensor}
\displaystyle \bm{C}
= \hspace{.1ex} \smalldisplaystyleonehalf \bigl( \bm{G} - \hspace{-0.1ex} \UnitDyad \hspace{.2ex} \bigr) \hspace{-0.3ex}
= \hspace{.1ex} \smalldisplaystyleonehalf \bigl( \bm{F}^{\hspace{.1ex}\T} \hspace{-0.6ex} \dotp \hspace{-0.2ex} \bm{F} - \UnitDyad \hspace{.2ex} \bigr)
\end{equation}

\noindent
or as the function of the displacement gradient tensor

\nopagebreak\begin{equation*}
\displaystyle \bm{C} = \hspace{.1ex} \smalldisplaystyleonehalf \hspace{-0.1ex} \Bigl( \hspace{-0.1ex}
     \boldnablacircled\bm{u}
     + \hspace{-0.1ex} \boldnablacircled\bm{u}^{\hspace{-0.1ex}\T} \hspace{-0.3ex}
     + \hspace{-0.1ex} \boldnablacircled\bm{u} \dotp \hspace{-0.1ex} \boldnablacircled\bm{u}^{\hspace{-0.1ex}\T} \hspace{-0.1ex}
\Bigr)
\hspace{-0.1ex} ,
\end{equation*}

\noindent
in cartesian coordinates

\nopagebreak\begin{equation*}
\displaystyle C_{i\hspace{-0.1ex}j} \hspace{-0.2ex}
= \hspace{.1ex} \smalldisplaystyleonehalf \hspace{-0.1ex} \Bigl(
     \scalebox{.8}{$ \displaystyle \frac{\partial x_{k'}}{\partial \mathcircabove{x}_{i}} $} \hspace{.1ex}
     \scalebox{.8}{$ \displaystyle \frac{\partial x_{k'}}{\partial \mathcircabove{x}_{\hspace{-0.2ex}j}} $}
     - \delta_{i\hspace{-0.1ex}j} \hspace{-0.2ex}
\Bigr) \hspace{-0.3ex}
= \hspace{.1ex} \smalldisplaystyleonehalf \hspace{-0.1ex} \Bigl( \hspace{.1ex}
     \scalebox{.8}{$ \displaystyle \frac{\partial u_{\hspace{-0.1ex}j}}{\partial \mathcircabove{x}_{i}} $}
     + \scalebox{.8}{$ \displaystyle \frac{\partial u_{i}}{\partial \mathcircabove{x}_{\hspace{-0.2ex}j}} $}
     + \scalebox{.8}{$ \displaystyle \frac{\partial u_{k}}{\partial \mathcircabove{x}_{i}} \frac{\partial u_{k}}{\partial \mathcircabove{x}_{\hspace{-0.2ex}j}} $}
\Bigr)
\hspace{-0.1ex} .
\end{equation*}

The \emph{Almansi\hbox{--}Hamel strain tensor}, referenced to the deformed configuration (\inquotes{Eulerian description}), is defined as

\nopagebreak\vspace{-0.5em}\begin{equation*}
{^2\hspace{-0.2ex}\bm{a}} = \smalldisplaystyleonehalf \bigl( \UnitDyad - \hspace{-0.15ex} {^2\hspace{-0.2ex}\bm{c}} \hspace{.3ex} \bigr) \hspace{-0.3ex}
= \smalldisplaystyleonehalf \bigl( \UnitDyad - \mathboldPhi^{\expminusone} \hspace{.2ex} \bigr)
\hspace{1em} \text{\en{or}\ru{или}} \hspace{1em}
a_{i\hspace{-0.1ex}j} \hspace{-0.2ex}
= \onehalf \hspace{-0.25ex} \left( \hspace{-0.4ex} \delta _{i\hspace{-0.1ex}j} - \frac{\partial \mathcircabove{x}_{k}}{\partial \hspace{-0.1ex} x_{i}} \hspace{.15ex} \frac{\partial \mathcircabove{x}_{k}}{\partial \hspace{-0.1ex} x_{\hspace{-0.15ex}j}} \right)
\end{equation*}

\vspace{-0.4em}\noindent
or as function of the displacement gradient

\nopagebreak\begin{equation*}
{^2}\hspace{-0.2ex}\bm{a} = \smalldisplaystyleonehalf \bigl(
\boldnabla\bm{u}^{\hspace{-0.1ex}\T} \hspace{-0.3ex}
+ \hspace{-0.1ex} \boldnabla\bm{u}
- \hspace{-0.1ex} \boldnabla\bm{u} \dotp \hspace{-0.1ex} \boldnabla\bm{u}^{\hspace{-0.1ex}\T}
\bigr)
\end{equation*}

\nopagebreak\vspace{-0.2em}\begin{equation*}
\displaystyle a_{i\hspace{-0.1ex}j} \hspace{-0.2ex} = \onehalf \hspace{-0.25ex} \left(
\frac{\partial u_{i}}{\partial x_{\hspace{-0.15ex}j}}
+ \frac{\partial u_{\hspace{-0.1ex}j}}{\partial \hspace{-0.1ex} x_{i}}
- \frac{\partial u_{k}}{\partial x_{i}} \frac{\partial u_{k}}{\partial x_{\hspace{-0.15ex}j}}
\right)
\end{equation*}

\subsection*{Seth\hbox{--}Hill family of abstract strain tensors}

B. R. Seth was the first to show that the Green and Almansi strain tensors are special cases of a more abstract measure of deformation.
The idea was further expanded upon by Rodney Hill in~1968 \textcolor{red}{(publication??)}.
The Seth\hbox{--}Hill family of strain measures (also called Doyle\hbox{--}Ericksen tensors) is expressed as

\nopagebreak\vspace{-0.1em}\begin{equation*}
\displaystyle \bm{D}_{(m)} \hspace{-0.2ex}
= \frac{\raisebox{-0.2em}{1}}{2m} \left( \hspace{.1ex} \bm{U}^{2m} \hspace{-0.4ex} - \UnitDyad \hspace{.2ex} \right)
= \frac{\raisebox{-0.2em}{1}}{2m} \left( \bm{G}^{m} \hspace{-0.4ex} - \UnitDyad \hspace{.1ex} \right) \end{equation*}

\vspace{.1em} \noindent \en{For various}\ru{Для разных}~$m$
\en{it gives}\ru{это даёт}

\nopagebreak\begin{equation*}
\begin{array}{r@{\hspace{0.1em}}ll}
\bm{D}_{(1)} & = \smalldisplaystyleonehalf \hspace{-0.25ex} \left( \bm{U}^{2} \hspace{-0.25ex} - \UnitDyad \right) = \smalldisplaystyleonehalf (\bm{G} - \UnitDyad) & \text{\scalebox{.9}{Green strain tensor}}
\\[.4em]
\bm{D}_{(\nicefrac{1}{2})} & = \bm{U} \hspace{-0.15ex} - \UnitDyad = \bm{G}^{\hspace{.1ex}\nicefrac{1}{2}} \hspace{-0.25ex} - \UnitDyad & \text{\scalebox{.9}{Biot strain tensor}}
\\[.4em]
\bm{D}_{(0)} & = \ln \bm{U} = \smalldisplaystyleonehalf \ln \bm{G} & \text{\scalebox{.9}{logarithmic strain, Hencky strain}}
\\[.4em]
\bm{D}_{(-\hspace{-0.1ex}1)} & = \smalldisplaystyleonehalf \hspace{-0.25ex} \left( \hspace{-0.1ex} \UnitDyad - \bm{U}^{-2} \hspace{.1ex} \right) & \text{\scalebox{.9}{Almansi strain}}
\end{array}
\end{equation*}

The second\hbox{-}order approximation of these tensors is
\[ \bm{D}_{(m)} \hspace{-0.2ex} =
\infinitesimaldeformation
+ \smalldisplaystyleonehalf \hspace{.1ex} \boldnabla\bm{u} \dotp \hspace{-0.1ex} \boldnabla\bm{u}^{\hspace{-0.1ex}\T} \hspace{-0.3ex}
- (1 - m) \hspace{.2ex} \infinitesimaldeformation \dotp \infinitesimaldeformation \]

\vspace{-0.25em}\noindent
where ${\infinitesimaldeformation \equiv \hspace{-0.2ex} \boldnabla {\bm{u}}^{\hspace{.1ex}\mathsf{S}}}$ is the infinitesimal deformation tensor.

Many other different definitions of measures~$\bm{D}$ are possible, provided that they satisfy these conditions:

\begin{itemize}
\item $\bm{D}$ vanishes for any movement of a~body as a~rigid whole
\item dependence of~$\bm{D}$ on displacement gradient tensor~${\nabla \bm{u}}$ is continuous, continuously differentiable and monotonic
\item it’s desired that $\bm{D}$ reduces to the infinitesimal linear deformation tensor~${\infinitesimaldeformation}$ when ${\boldnabla \bm{u} \to 0}$
\end{itemize}

\noindent For example, tensors from the set
\[ \displaystyle \bm{D}^{(n)} \hspace{-0.32ex} = \left( {\bm{U}}^{n} \hspace{-0.4ex} - {\bm{U}}^{-n} \right) \hspace{-0.4ex} / \hspace{.25ex} 2n \]
aren’t from the Seth\hbox{--}Hill family, but for any~$n$ they have the same 2nd\hbox{-}order approximation as Seth\hbox{--}Hill measures with~${m=0}$.

\vspace{.4em} \noindent \hfill \textboldoblique{Wikipedia, the free encyclopedia}\:--- \href{https://en.wikipedia.org/wiki/Finite_strain_theory}{Finite strain theory}

...


\subsection*{\en{Logarithmic strain, Hencky’s strain}\ru{Логарифмическая деформация, деформация Hencky}}

\href{https://en.wikipedia.org/wiki/Heinrich_Hencky}{%
\bookauthor{Heinrich Hencky}%
}.
Über die Form des Elastizitätsgesetzes bei ideal elastischen Stoffen.
Zeitschrift für technische Physik, Vol.\:9~(1928),
Seiten~215\hbox{--}220.

....

