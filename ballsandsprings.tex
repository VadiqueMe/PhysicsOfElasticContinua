\documentclass[tikz]{standalone}

\usepackage{tikz}

\begin{document}

%argument is where to draw
\newcommand{\drawBlick}[1] {
   \tikz\draw[ #1, radius=.7*\ballradius ]
      arc[ line width=1pt, color=black, start angle=-80, end angle=-10 ] ;
}

%argument is where to draw
\newcommand{\drawBall}[1] {
   \tikz\draw[ #1, radius=\ballradius ]
      circle[ line width=1pt, color=white, fill=white, radius=\ballradius ]
      \node (#1)
      circle[ line width=1pt, color=black, radius=\ballradius ] ;

   \drawBlick{#1}
}

\newcommand{\tikzredcircle}[2][red,fill=red]{
   \tikz[ baseline=-0.5ex ]\draw[ #1,radius=#2 ] (0,0) circle ;}

\begin{tikzpicture}

  \def\ballradius{.5ex}

  \def\ballxstep{4em}
  \def\ballystep{4em}

  %\coordinate (ball1) at (-\ballxstep,\ballystep);
  %\coordinate (ball2) at (0,\ballystep);
  %\coordinate (ball3) at (\ballxstep,\ballystep);
  %\coordinate (ball4) at (-\ballxstep,0);
  %\coordinate (ball5) at (0,0);
  %\coordinate (ball6) at (\ballxstep,0);
  %\coordinate (ball7) at (-\ballxstep,-\ballystep);
  %\coordinate (ball8) at (0,-\ballystep);
  %\coordinate (ball9) at (\ballxstep,-\ballystep);

  \drawBall{(-\ballxstep,\ballystep)}
  \drawBall{(0,\ballystep)} ;
  \drawBall{\ballxstep,\ballystep)} ;
  %%\drawBall{(ball4)} ;
  %%\drawBall{(ball5)} ;
  %%\drawBall{(ball6)} ;
  %%\drawBall{(ball7)} ;
  %%\drawBall{(ball8)} ;
  %%\drawBall{(ball9)} ;

\end{tikzpicture}

\end{document}
