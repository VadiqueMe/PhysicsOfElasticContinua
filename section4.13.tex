\en{\section{Antiplane shear}} % Antiplane deformation (strain)

\ru{\section{Антиплоский сдвиг}}

\newcommand\antiplanedisplacement{\mathcolor{red}{\upupsilon}}

\en{This is}\ru{Это}
\en{such a~problem}\ru{такая проблема}
\en{of the linear theory of~elasticity}\ru{линейной теории упругости}\ru{,}
\en{where}\ru{где}
\en{non-trivial results}\ru{нетривиальные результаты}\footnote{%
   \en{Non-trivial}\ru{Нетривиальное}
   \en{in the theory of~elasticity}\ru{в~теории упругости}
   \en{is}\ru{это},
   \en{for example}\ru{для примера},
   \en{when}\ru{когда}
   \en{dividing}\ru{деление}
   \en{the~force by the~area}\ru{силы на~площадь}
   \en{gives}\ru{даёт}
   \en{an~infinitely large error}\ru{бесконечно больш\'{у}ю ошибку}
   \en{in calculating the~stress}\ru{в~вычислении напряжения}.%
}
\en{are obtained}\ru{получаются}
\en{by simple}\ru{простыми}
\en{conclusions}\ru{выводами}.

\en{This problem}\ru{Эта проблема}\en{ is}\ru{\:---}
\en{about}\ru{об}
\en{an isotropic elastic}\ru{изотропном упругом}
\en{continuum}\ru{\rucontinuum{}е}
\en{in the cartesian coordinates}\ru{в~декартовых координатах}

\nopagebreak\vspace{-0.2ex}
\begin{equation*}
x_\alpha
\hspace{.1ex} ,
\hspace{1.5em}
\alpha = 1,2
,
\hspace{1.5em}
x_1
\hspace{1.2ex} \text{\en{and}\ru{и}} \hspace{1.2ex}
x_2
\hspace{.1ex} .
\end{equation*}

\en{The plane}\ru{Плоскость}
${x_1 \hspace{.1ex} , \hspace{.8ex} x_2}$\en{ is}\ru{\:---}
\en{a~cross-section of a~rod}\ru{поперечное сечение стержня},
\en{the third coordinate}\ru{третья координата}~$\mathcolor{blue}{x_3}$
\en{is perpendicular to the section}\ru{перпендикулярна сечению}.
\en{The basis vectors}\ru{Базисные векторы}\en{ are}

\begin{equation*}
\bm{e}_i \hspace{-0.2ex} = \partial_i \hspace{.1ex} \locationvector
\hspace{-0.1ex} ,
\hspace{1em}
\locationvector = x_i \bm{e}_i
\hspace{.1ex} ,
\hspace{1em}
\bm{e}_i \bm{e}_i \hspace{-0.2ex} = \hspace{-0.15ex} \UnitDyad
\hspace{.2ex} \Leftrightarrow \hspace{.25ex}
\bm{e}_i \hspace{-0.1ex}
\dotp
\bm{e}_j \hspace{-0.2ex}
= \delta_{i\hspace{-0.1ex}j}
\hspace{.15ex} .
\end{equation*}

\en{In}\ru{В}~\en{case}\ru{случае}
\en{of anti\-plane strain}\ru{анти\-плоской деформации}
 (\en{anti\-plane shear}\ru{анти\-плоского сдвига}),
\en{the field}\ru{поле}
\en{of displacements}\ru{смещений}~$\bm{u}(\bm{r})$
\en{is parallel}\ru{параллелен}
\en{to the third coordinate}\ru{третьей координате}~$\mathcolor{blue}{x_3}$\::

\begin{equation*}
\bm{u} = \antiplanedisplacement \hspace{.1ex} \mathcolor{blue}{\bm{e}_3}
\hspace{.1ex} ,
\end{equation*}

\noindent
\en{and}\ru{и}~$\antiplanedisplacement$
\en{doesn’t depend on}\ru{не~зависит от}~$\mathcolor{blue}{x_3}$\::

\begin{equation*}
\antiplanedisplacement \narroweq \antiplanedisplacement(x_1, x_2),
\hspace{1em}
\partial_{\raisemath{-0.15ex}{\mathcolor{blue}{3}}} \antiplanedisplacement \hspace{-0.15ex} = 0
\hspace{.1ex} .
\end{equation*}

\en{The deformation}\ru{Деформация}

\nopagebreak\vspace{-0.25em}
\begin{equation}
\infinitesimaldeformation
\equiv \hspace{-0.1ex}
\boldnabla {\bm{u}}^{\hspace{.1ex}\mathsf{S}} \hspace{-0.2ex}
= \hspace{-0.2ex}
\boldnabla \hspace{.1ex} \bigl(
   \antiplanedisplacement
   \hspace{.1ex}
   \mathcolor{blue}{\bm{e}_3}
\bigr)^{\hspace{-0.2ex} \mathsf{S} } \hspace{-0.25ex}
   = \mathcolor{blue}{\bm{e}_3}
   \hspace{-0.15ex} \boldnabla \hspace{.1ex}
   \antiplanedisplacement^{\hspace{.1ex}\mathsf{S}}
   \hspace{-0.25ex} +
   \antiplanedisplacement \hspace{.1ex}
   \tikzmark{beginZero2}
   % beginZero2
      \boldnabla
      \mathcolor{blue}{\bm{e}_3}
   \tikzmark{endZero2}^{\mathsf{S}} \hspace{-0.2ex}
   = \hspace{.1ex}
   \smalldisplaystyleonehalf \hspace{.1ex}
   \bigl(
      \boldnabla \hspace{.1ex} \antiplanedisplacement \hspace{.12ex}
      \mathcolor{blue}{\bm{e}_3} \hspace{-0.1ex}
      + \mathcolor{blue}{\bm{e}_3} \hspace{-0.15ex}
      \boldnabla \hspace{.1ex} \antiplanedisplacement
   \bigr)
\end{equation}%
\AddUnderBrace[line width=.75pt][.2ex, -0.1ex]%
{beginZero2}{endZero2}{${ \scriptstyle \zerobivalent }$}%

\en{In the plane}\ru{В~плоскости}
${x_1, x_2}$
\en{of the cross-section}\ru{поперечного сечения}

\begin{equation*}
\mu \narroweq \mu(x_1, x_2) ,
\hspace{1.5em}
\partial_{
   \raisemath{-0.15ex}{ \mathcolor{blue}{3} }
} \hspace{.08ex} \mu \hspace{-0.1ex} = 0
\end{equation*}

\noindent
---
\en{a~possible}\ru{возможная}
\en{inhomogeneity}\ru{неоднородность}
\en{of the medium}\ru{среды}.
