\vspace*{2\baselineskip}

\newcommand\forewordname{\en{Foreword}\ru{Предисловие}}
\chapter*{\forewordname}
\addcontentsline{toc}{chapter}{\forewordname}

\noindent
\en{I~presented}\ru{Я~представил}
\en{the~following models}\ru{следующие модели}
\en{of an~elastic continuum}\ru{упругого \rucontinuum{}а}
\en{in this book}\ru{в~этой книге}:
\en{the~non\-linear and the~linear ones}\ru{нелинейные и~линейные},
\en{the~micropolar and the~classical moment\-less}\ru{микрополярные и~классические без\-момент\-ные};
\en{the~three-di\-men\-sion\-al}\ru{трёх\-мерные},
\en{the~two-di\-men\-sion\-al}\ru{дву\-мерные}~(\en{shells and~plates}\ru{оболочки и~пластины})
\en{and}\ru{и}
\en{the~one-di\-men\-sion\-al}\ru{одно\-мерные}~(\en{rods}\ru{стержни},
\en{including thin\hbox{-}walled ones}\ru{включая тонко\-стен\-ные}).
\en{I also explained}\ru{Также я объяснил}
\en{the~fundamentals of dynamics}\ru{основы динамики}\:---
\en{oscillations, waves and stability}\ru{колебания, волны и~устойчивость}.
\en{For the~thermo\-elasticity and the~magneto\-elasticity}\ru{Для термо\-упругости и~магнито\-упругости}\en{,}
\en{I gave}\ru{я дал}
\en{the~summary}\ru{сводку}
\en{of the~classical theories}\ru{классических теорий}
\en{of the~thermo\-dynamics and the~electro\-dynamics}\ru{термо\-динамики и~электро\-динамики}.
\en{The~dynamics of~destruction}\ru{Динамика разрушения}
\en{is described}\ru{описана}
\en{via the~theories of~defects and fractures}\ru{через теории дефектов и~трещин}.
\ru{Также показаны }\en{The~approaches to~modeling}\ru{подходы к~моделированию}
\en{of human\hbox{-}made}\ru{сделанных человеком}
\en{inhomogeneous materials}\ru{неоднородных материалов},
\en{\inquotesx{composites}[,]}\ru{\inquotesx{композитов}[.]}\en{ are also shown.}

\vspace{\baselineskip}

\noindent
\en{The~word}\ru{Слово}
\inquotes{\en{continua}\ru{\rucontinuum{}ов}}
\en{in the~title}\ru{в~названии}
%%\en{of~this book}\ru{этой книги}
\en{says that}\ru{говорит, что}
\en{a~body}\ru{тело}~(\en{a~medium}\ru{среда})
\en{is modeled}\ru{моделируется}
\en{here}\ru{здесь}
\en{not}\ru{не}
\en{as}\ru{как}
\en{a~discrete collection}\ru{дискретная коллекция}
\en{of~particles}\ru{частиц},
\en{but}\ru{но}
\en{as}\ru{как}
\en{a~continuous}\ru{континуальное~(непрер\'{ы}вное)}
\en{field}\ru{поле}
\en{of~location vectors}\ru{векторов положения},
\en{a~continuous}\ru{континуальная}
\en{matter}\ru{материя}.
%
\en{It}\ru{Это}
\en{gives}\ru{даёт}
\en{a~large}\ru{большое}
\en{convenience}\ru{удобство},
\en{because}\ru{потому что}
\en{the apparatus}\ru{аппарат}
\en{of~}calculus\ru{’а}\ru{~(исчисления)}
\en{of~infinitesimals}\ru{бесконечно м\'{а}лых}
\en{can be used}\ru{может быть использован}
\en{for}\ru{для}
\en{such models}\ru{таких моделей}.

\vspace{\baselineskip}

\noindent
\en{When}\ru{Когда}
\en{I just began}\ru{я только н\'{а}чал}
\en{writing}\ru{пис\'{а}ть}
\en{this book}\ru{эту книгу},
\ru{я думал о~читателе}\en{I thought of a~reader}\ru{,}
\en{who}\ru{который}
\en{is pretty acquainted}\ru{весьма знак\'{о}м}
\en{with }\ru{с~}\inquotes{\en{higher}\ru{высшей}}
\en{mathematics}\ru{математикой}.
%
\en{But}\ru{Но}
\en{later}\ru{позже}
\en{I decided}\ru{я решил}
\en{to conduct}\ru{провести}
\en{such an~acquaintance}\ru{такое знак\'{о}мство}
\en{by myself}\ru{сам},
\en{and yet}\ru{и~уж\'{е}},
\en{as a~side effect}\ru{как побочный эффект},
\en{every reader}\ru{каждый читатель}
\en{with any knowledge of~math}\ru{с~любым знанием математики}
\en{can}\ru{может}
\en{comprehend}\ru{постигнуть}
\en{the~content of the~book}\ru{содержимое книги}.

\vspace{\baselineskip}

\noindent
\en{The~book}\ru{Книга}
\en{is written}\ru{написана}
\en{using the compact and elegant}\ru{с~использованием компактной и~элегантной}
\en{direct indexless}\ru{прямой безиндексной}
\en{tensor notation}\ru{тензорной записи}.
\en{The mathematical apparatus}\ru{Математический аппарат}
\en{for interpreting}\ru{для интерпретирования}
\en{the direct tensor relations}\ru{прямых тензорных соотношений}
\en{is located in the first chapter}\ru{находится в~первой главе}.

\vspace{\baselineskip}

\noindent
\en{I am writing this book}\ru{Я пишу эту книгу}
\en{simultaneously in the two languages}\ru{одновременно на двух языках},
\ru{английском и~русском}\en{English and Russian}.
\en{The reader}\ru{Читатель}
\en{is free}\ru{свободен}
\en{to pick}\ru{выбрать}
\en{any language}\ru{любой язык}
\en{of the two}\ru{из двух}.

\vspace{3\baselineskip}

\noindent
\scalebox{.9}{ \href{https://github.com/VadiqueMe/PhysicsOfElasticContinua}{\textit{github.com/VadiqueMe/PhysicsOfElasticContinua}} }

\newpage
