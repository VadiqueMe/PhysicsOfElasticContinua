\en{\chapter{Periodic structures}}

\ru{\chapter{Периодические структуры}}

%% Periodic structure is made of an infinite or finite repetition of a unit cell

\thispagestyle{empty}

\label{chapter:periodicstructures}

\begin{otherlanguage}{russian}

\section{Одномерная задача}

\lettrine[lines=2, findent=2pt, nindent=0pt]{В}{\hspace{-0.25ex}} одномерной задаче статики имеем уравнение

...



\section{Трёхмерная среда}

Исходим из уравнений в~перемещениях

...



\section{Волокнистая структура}

Тензор~${\stiffnesstensor}$ в~этом случае постоянен вдоль оси

...



\section{Статика периодического стержня}

В~уравнениях линейной статики стержня

...




\end{otherlanguage}

\section*{\small \wordforbibliography}

\begin{changemargin}{\parindent}{0pt}
\fontsize{10}{12}\selectfont

\begin{otherlanguage}{russian}

Лежащий в~основе этой главы асимптотический метод представлен~(с~разной степенью математической стройности) в~книгах~\cite{bakhvalovpanasenko, asymptoticanalysisforperiodicstructures, kravchuk.mayboroda.urzhumtsev-polymericandcompositematerials, pobedrya-composites}.

\end{otherlanguage}

\end{changemargin}
