\en{\chapter{Shells and plates}}

\ru{\chapter{Оболочки и пластины}}

\thispagestyle{empty}

\label{chapter:shellsandplates}

\en{\section{Surface geometry}}

\ru{\section{Геометрия поверхности}}

\begin{otherlanguage}{russian}

\lettrine[lines=2, findent=2pt, nindent=0pt]{П}{оверхность} в~трёхмерном пространстве определяется функцией вектора\hbox{-}радиуса точек поверхности от~двух координат ${\bm{r} \!=\! \bm{r}(q^{\hspace{.1ex}\alpha})}$, ${\alpha = 1, 2}$. Непрерывное изменение одной из координат~${q^1\hspace{-0.25ex}}$ при постоянной другой~${q^2\hspace{-0.25ex}}$ даёт координатную линию~${q^1\hspace{-0.25ex}}$. Пере\-сече\-ние двух координатных линий однозначно определяет точку поверхности. Векторы

\nopagebreak\vspace{-0.5em}\begin{equation*}
\bm{r}_{\hspace{-0.2ex}\alpha} \hspace{-0.16ex} \equiv \partial_{\hspace{-0.1ex}\alpha} \bm{r}
, \;\;
\partial_{\hspace{-0.1ex}\alpha} \hspace{-0.16ex} \equiv \textstyle\frac{\partial \bm{r}}{\partial q^\alpha}
\end{equation*}

\vspace{-0.2em} \noindent лежат на~касательных к~координатным линиям. Они являются базисом для представления любого вектора~$\mathtwoabove{\bm{v}}$ в~касательной плоскости как линейной комбинации

\nopagebreak\vspace{-0.2em}\begin{equation*}
\mathtwoabove{\bm{v}} = v^\alpha \bm{r}_{\hspace{-0.2ex}\alpha} \hspace{-0.16ex} = v_\alpha \bm{r}^\alpha
\hspace{-0.4ex} , \;\;
v^\alpha \hspace{-0.25ex} = \mathtwoabove{\bm{v}} \dotp \bm{r}^\alpha
\hspace{-0.4ex} , \:\;
v_\alpha \hspace{-0.2ex} = \mathtwoabove{\bm{v}} \dotp \bm{r}_{\hspace{-0.2ex}\alpha}
\end{equation*}

\vspace{-0.25em} \noindent (введён взаимный базис в~касательной плоскости: ${\bm{r}^\alpha \hspace{-0.25ex} \dotp \bm{r}_{\hspace{-0.2ex}\beta} \hspace{-0.2ex} = \delta^\alpha_{\hspace{-0.2ex}\beta}}$).

Добавление в~каждой точке поверхности единичного вектора нормали~${\bm{n}(q^{\hspace{.1ex}\alpha})}$
% field of unit normal vectors

\nopagebreak\vspace{-1em}\begin{equation*}
\bm{n} = \scalebox{.92}{$
\displaystyle \frac{\bm{r}_1 \hspace{-0.15ex} \times \bm{r}_2}%
{\hspace{.4ex} | \hspace{.2ex} \bm{r}_1 \hspace{-0.15ex} \times \bm{r}_2 \hspace{.2ex} | \hspace{.4ex}}
$}
%%\hspace{.25ex} ,
\end{equation*}

\vspace{-0.1em} \noindent даёт разложение для любого вектора (и~вообще любого тензора) в~пространстве, например~${\bm{u} = u^\alpha \bm{r}_{\hspace{-0.2ex}\alpha} \hspace{-0.15ex} + u_n \bm{n}}$.

Единичные (\inquotes{метрические}) тензоры~$\bm{E}$ в~пространстве и~$\mathtwoabove{\bm{e}}$ в~касательной плоскости

\nopagebreak\vspace{-0.25em}\begin{equation*}
\bm{E} = \mathtwoabove{\bm{e}} + \bm{n} \bm{n}
\hspace{.1ex} , \:\:
\mathtwoabove{\bm{e}} \hspace{.1ex} \equiv \bm{E} - \bm{n} \bm{n} = \bm{r}_{\hspace{-0.2ex}\alpha} \bm{r}^\alpha \hspace{-0.25ex} = \bm{r}^\alpha \bm{r}_{\hspace{-0.2ex}\alpha}
\hspace{.1ex} .
\vspace{-0.2em}
\end{equation*}

Представление вектора\hbox{-}радиуса точки пространства на~рас\-стоя\-нии~$h$~(${\frac{\partial h}{\partial q^{\alpha}} \hspace{-0.2ex} = 0}$) от~поверхности

\nopagebreak\vspace{-0.2em}\begin{equation*}
\bm{R}(q^{\hspace{.1ex}\alpha} \hspace{-0.4ex}, h) = \hspace{.1ex} \bm{r}(q^{\hspace{.1ex}\alpha}) + \hspace{.1ex} h \hspace{.1ex} \bm{n}(q^{\hspace{.1ex}\alpha})
\end{equation*}

\vspace{-0.5em} \noindent определяет базис

\nopagebreak\vspace{-0.8em}\begin{equation*}
\begin{array}{c}
\bm{R}_n \hspace{-0.2ex} = \bm{n} = \bm{R}^n \hspace{-0.25ex} ,
\\[.1em]
\bm{R}_\alpha \hspace{-0.2ex} \equiv \partial_{\hspace{-0.1ex}\alpha} \bm{R}
= \partial_{\hspace{-0.1ex}\alpha} \bm{r} + h \hspace{.2ex} \partial_{\hspace{-0.1ex}\alpha} \bm{n}
= \bm{r}_{\hspace{-0.2ex}\alpha} + \hspace{.1ex} h \hspace{.2ex} \bm{r}_{\hspace{-0.2ex}\alpha} \hspace{-0.2ex} \dotp \bm{r}^\beta \hspace{-0.1ex} \partial_{\hspace{-0.1ex}\beta} \bm{n}
\hspace{.1ex} .
\end{array}
\end{equation*}

...

\[
\bm{R}_\alpha \hspace{-0.2ex}
= \bm{r}_{\hspace{-0.2ex}\alpha} \hspace{-0.15ex} \dotp \hspace{-0.1ex}
\left(
\bm{r}^\beta \bm{r}_{\hspace{-0.2ex}\beta}
+ h \hspace{.2ex} \bm{r}^\beta \hspace{-0.1ex} \partial_{\hspace{-0.1ex}\beta} \bm{n}
\right) \hspace{-0.25ex}
= \bm{r}_{\hspace{-0.2ex}\alpha} \hspace{-0.15ex} \dotp \hspace{-0.1ex}
\left(
\mathtwoabove{\bm{e}} + h \hspace{.1ex} \boldnablaflat \bm{n}
\right) \hspace{-0.25ex}
= \bm{r}_{\hspace{-0.2ex}\alpha} \hspace{-0.15ex} \dotp \hspace{-0.1ex}
\left(
\mathtwoabove{\bm{e}} - h \hspace{.1ex} \mathtwoabove{\bm{c}}
\hspace{.2ex} \right)
\]

${\mathtwoabove{\bm{c}} \equiv - \boldnablaflat \bm{n}}$

...

\noindent кобазис и~дифференциальный оператор \inquotes{набла}

${\boldnabla \equiv \bm{R}^{\hspace{.1ex}i} \hspace{-0.1ex} \partial_i}$

${\boldnablaflat \equiv \hspace{.1ex} \bm{r}^\alpha \hspace{-0.1ex} \partial_{\hspace{-0.1ex}\alpha}}$

...




\end{otherlanguage}

\en{\section{Model of a~shell}}

\ru{\section{Модель оболочки}}

\begin{otherlanguage}{russian}

Располагая моделями трёхмерного моментного континуума, \textcolor{magenta}{стержней и~пластин}, не~так~уж тяжело разобраться в~механике оболочек. Как~геометрический объект, оболочка определяется ...

...







\end{otherlanguage}

\en{\section{Balance of forces and moments for a~shell}}

\ru{\section{Баланс сил и моментов для оболочки}}

\begin{otherlanguage}{russian}

При~${\variation{\fieldofdisplacements} = \boldconstant}$ и~${\variation{\fieldofrotations} = \bm{0}}$~(трансляция) ...

...



\end{otherlanguage}

\en{\section{Shells: Relations of elasticity}}

\ru{\section{Оболочки: Отношения упругости}}

\begin{otherlanguage}{russian}

Локальное соотношение~\eqref{somereftwothreeeee} после вывода уравнений баланса ...

...



\end{otherlanguage}

\en{\section{Classical theory of shells}}

\ru{\section{Классическая теория оболочек}}

\begin{otherlanguage}{russian}

Вышеизложенная теория (напоминающая балку Тимошенко и~континуумы Коссера) рассматривает~$\fieldofrotations$ независимо от~$\fieldofdisplacements$. Но обыденный опыт подсказывает: материальный элемент, нормальный к~срединной поверхности до~деформации, остаётся таковым и~после (кинематическая гипотеза Кирхгофа). В~классической теории Кирхгофа, Арона и~Лява $\fieldofrotations$ выражается через~$\fieldofdisplacements$, что в~конце концов позволяет всё свести к~одному векторному уравнению для~$\fieldofdisplacements$.

Предположим, что в~основе классической теории лежит внутренняя связь

...



\end{otherlanguage}

\en{\section{Shells: A plate}}

\ru{\section{Оболочки: Пластина}}

\begin{otherlanguage}{russian}

Это простейший случай оболочки. Орт ${\bm{n} = \bm{k}}$ направлен по декартовой оси~$z$, в~качестве координат ...

...



\end{otherlanguage}

\en{\section{Shells: Approach with Lagrange multipliers}}

\ru{\section{Оболочки: Подход с множителями Лагранжа}}

\begin{otherlanguage}{russian}

Уязвимым местом этого изложения теории оболочек являются формулы

...



\end{otherlanguage}

\en{\section{Cylindrical shell}}

\ru{\section{Цилиндрическая оболочка}}

\begin{otherlanguage}{russian}

Существуют разные уравнения цилиндрической оболочки. Приводятся громоздкие выкладки с~отбрасыванием некоторых малых членов, и~не~всегда ясно, какие именно члены действительно можно отбросить.

Предлагаемая читателю теория оболочек иного свойства: лишних членов нет, все уравнения записаны в~компактной тензорной форме\:--- остаётся лишь грамотно действовать с~компонентами тензоров. В~качестве иллюстрации рассмотрим цилиндрическую оболочку.

В~декартовой системе

...



\end{otherlanguage}

\en{\section{Shells: Common theorems}}

\ru{\section{Оболочки: Общие теоремы}}

\begin{otherlanguage}{russian}

Пусть край закреплён

...



\end{otherlanguage}

\en{\section{Shells: Boundary conditions}}

\ru{\section{Оболочки: Краевые условия}}

\begin{otherlanguage}{russian}

В~рамках рассматриваемого прямого подхода к~оболочкам как материальным поверхностям наиболее надёжным способом вывода граничных условий представляется вариационный. Исходим из вариационного уравнения:

...



\end{otherlanguage}

\en{\section{Shells of revolution}}

\ru{\section{Оболочки вращения}}

\noindent \emph{Surface of~revolution (reference surface of shell of revolution) is created by rotating a~plane curve (the~meridian, the~generatrix) about a~straight line in the~plane of curve (an~axis of~rotation).}

\begin{otherlanguage}{russian}

Разберёмся в~геометрии поверхности вращения~( рис. ?? 29 ?? ). Меридиан можно задать зависимостью декартовых координат

...



\end{otherlanguage}

\en{\section{Momentless theory of shells}}

\ru{\section{Безмоментная теория оболочек}}

\begin{otherlanguage}{russian}

В~отличие от~пластины, оболочка способна выдерживать нормальную распределённую нагрузку без появления внутренних моментов. В~безмоментном состоянии напряжения равномерно распределены по~толщине оболочки, безмоментные оболочечные конструкции можно считать оптимально спроектированными.

Уравнения безмоментной теории

...



\end{otherlanguage}

\en{\section{Shells: Nonlinear momentless theory}}

\ru{\section{Оболочки: Нелинейная безмоментная теория}}

\begin{otherlanguage}{russian}

Вышеизложенную безмоментную теорию оболочек возможно просто и~корректно обобщить на нелинейную постановку. Материальная поверхность состоит из~частиц

...



\end{otherlanguage}

\en{\section{Shells: Other variant of classical theory}}

\ru{\section{Оболочки: Иной вариант классической теории}}

\begin{otherlanguage}{russian}

%% \noindent \emph{(добавлено в~готовую рукопись перед изданием 1999го года)}

\noindent Выше при изложении моментной теории оболочек частицы материальной поверхности считались твёрдыми телами с~шестью степенями свободы

...



\end{otherlanguage}

\en{\section{Plates: Overall concepts}}

\ru{\section{Пластины: Общие представления}}

\label{para:overviewofplates}

\begin{otherlanguage}{russian}

\lettrine[lines=2, findent=2pt, nindent=0pt]{П}{ластиной} называется тонкое трёхмерное тело, ограниченное двумя параллельными плоскостями и~боковой цилиндрической поверхностью (рисунок ?? 26 ???). В~декартовых координатах ${x_1, x_2, z}$ поперечная координата ...

...

В~теории пластин рассматриваются двумерные задачи. Переход от трёхмерной задачи наиболее достоверен на~пути асимптотики. Но логическая стройность и~эффективность присуща и~вариационному подходу, основанному на аппроксимации по~толщине решения трёхмерной вариационной задачи. Самое~же простое корректное изложение теории пластин характерно для прямого подхода к~ним как материальным плоскостям.

...



\end{otherlanguage}

\en{\section{Timoshenko-like model of a~plate (direct approach)}}

\ru{\section{Модель пластины типа Тимошенко (прямой подход)}}

\begin{otherlanguage}{russian}

Пластина рассматривается как материальная плоскость, частицы которой

...



\end{otherlanguage}

\en{\section{Kirchhoff’s classical theory of plates}}

\ru{\section{Классическая теория пластин Кирхгофа}}

\begin{otherlanguage}{russian}

Принимается внутренняя связь

...



\end{otherlanguage}

\en{\section{Plates: Asymptotic matching of two-dimensional models}}

\ru{\section{Пластины: Асимптотическое сопоставление двумерных моделей}}

\begin{otherlanguage}{russian}

При малой толщине из теории типа Тимошенко вытекает классическая. Толщина~$h$ определяется отношением жёсткостей. Перепишем

...



\end{otherlanguage}

\en{\section{Plates: Variational transition from three-dimensional model}}

\ru{\section{Пластины: Вариационный переход от трёхмерной модели}}

\begin{otherlanguage}{russian}

Используя вариационные принципы Лагранжа или Рейсснера--Хеллингера с~аппроксимацией решения по~толщине, можно получить двумерные вариационные постановки. Из~них вытекают и~соотношения внутри области, и~естественные краевые условия.

Для примера построим модель типа Тимошенко с~аппроксимацией перемещений

...


...

Рассмотренные вариационные переходы легко обобщаются на~случаи неоднородности и~анизотропии материала, температурных деформаций, динамики. Достоинство принципа Рейсснера--Хеллингера\:--- в~явном представлении напряжений. \hbox{Зато} принцип Лагранжа примен\'{и}м к~нелинейным задачам (\customref[глава~]{chapter:nonlinearcontinuum} содержит трёхмерную постановку).

\end{otherlanguage}

\en{\section{Plates: Splitting of three-dimensional bending problem}} % Asymptotic splitting

\ru{\section{Пластины: Ращепление трёхмерной задачи изгиба}} % Асимптотическое ращепление

\begin{otherlanguage}{russian}

Двумерная классическая теория изгиба пластин легко выводится из трёхмерной постановки с~малым параметром. Представив радиус\hbox{-}вектор в~объёме

...



\end{otherlanguage}

\en{\section{Circular plates}} % Round plates

\ru{\section{Круглые пластины}}

\begin{otherlanguage}{russian}

В~качестве иллюстрации рассмотрим широко представленный в~литературе вопрос об~уравнениях теории Кирхгофа в~полярных координатах. Имеем

...



\end{otherlanguage}

\en{\section{Plates: Plane stress}}

\ru{\section{Пластины: Плоское напряжение}}

\begin{otherlanguage}{russian}

Это вторая из~двух задач, о~которых говорилось в~\pararef{para:overviewofplates}. Силы

...




\end{otherlanguage}

\vspace{8mm}
\hfill\begin{minipage}[b]{0.95\linewidth}
\fontsize{10}{12}\selectfont

\section*{\wordforbibliography}

\begin{otherlanguage}{russian}

Теория оболочек изложена в~монографиях А.\,Л.\;Гольденвейзера~\cite{goldenveizer-thinshells}, В.\,В.\;Новожилова~\cite{novozhilov-theoryofthinshells}, А.\,И.\;Лурье~\cite{lurie-thinwalledshells}, В.\,С.\;Черниной~\cite{chernina-thinwalledshells} и~ряде других. Достоинства этих книг перекрывают неразвитость формального аппарата. Переход от~трёхмерной модели оболочки к~двумерной рассмотрен у~...

...

Техническая теория изгиба пластин изложена ...

...

\end{otherlanguage}

\end{minipage}
