\en{\chapter{Shells and plates}}

\ru{\chapter{Оболочки и пластины}}

\makeatletter

\newcommand{\accentoneinbracketsabove}{%
	\hbox{\fontfamily{lmr}\fontsize{.5\dimexpr(\f@size pt)}{0}\selectfont$\hspace{.25ex}(\hspace{-0.2ex}1\hspace{-0.2ex})$}}
\DeclareRobustCommand{\mathoneinbracketsabove}{\accentset{\accentoneinbracketsabove}}

\newcommand{\accenttwoinbracketsabove}{%
	\hbox{\fontfamily{lmr}\fontsize{.5\dimexpr(\f@size pt)}{0}\selectfont$\hspace{.25ex}(\hspace{-0.2ex}2\hspace{-0.2ex})$}}
\DeclareRobustCommand{\mathtwoinbracketsabove}{\accentset{\accenttwoinbracketsabove}}

\newcommand{\accentthreeinbracketsabove}{%
	\hbox{\fontfamily{lmr}\fontsize{.5\dimexpr(\f@size pt)}{0}\selectfont$\hspace{.25ex}(\hspace{-0.2ex}3\hspace{-0.2ex})$}}
\DeclareRobustCommand{\maththreeinbracketsabove}{\accentset{\accentthreeinbracketsabove}}

\makeatother

\newcommand\oneparametervector[1]{\smash{\mathoneinbracketsabove{#1}}}
\newcommand\twoparametervector[1]{\smash{\mathtwoinbracketsabove{#1}}}
\newcommand\threeparametervector[1]{\smash{\maththreeinbracketsabove{#1}}}

\newcommand\twoparameterbivalent[1]{\smash{{^2}\mathtwoinbracketsabove{#1}}}

\newcommand\twoparameterunitdyad{\bm{I}\hspace{-0.1ex}}

\newcommand{\boldnablaflat}{\mathtwoinbracketsabove{\bm{\nabla}}\hspace{-0.2ex}}

\newcommand\threedimensionlocationvector{\smash{\maththreeinbracketsabove{\mathboldrcursive}}}
\newcommand\threedimensionallocation{{\hspace{.15ex}\threedimensionlocationvector\hspace{.22ex}}}

\newcommand\threedimensionlocationvectorandupperindex[1]{\maththreeinbracketsabove{\mathboldrcursive}^{#1}}
\newcommand\threedimensionallocationwithupperindex[1]{{\hspace{.15ex}\threedimensionlocationvectorandupperindex{#1}\hspace{.22ex}}}

\thispagestyle{empty}

\label{chapter:shellsandplates}

\en{\section{Surface geometry}}

\ru{\section{Геометрия поверхности}}

\dropcap{\en{A\hspace*{-0.1ex}}\ru{П}}{\en{surface}\ru{оверхность}}
\en{is described}\ru{описывается}
\en{by a~function}\ru{функцией}~(\en{a~morphism}\ru{морфизмом})

\nopagebreak\vspace{-0.25em}
\begin{equation}
\locationvector \hspace{-0.4ex} = \hspace{-0.3ex} \locationvector(q^{\hspace{.1ex}\alpha})
\hspace{.1ex} , \hspace{.4em}
\alpha \hspace{-0.1ex} = \hspace{-0.12ex} 1, 2
\end{equation}

\nopagebreak\vspace{-0.25em}\noindent
\en{of~two}\ru{двух} \en{mutually independent}\ru{взаимно независимых} \en{variable}\ru{переменных} \en{parameters}\ru{параметров} (\en{coordinates}\ru{координат})~${q^{\hspace{.1ex}\alpha}\hspace{-0.33ex}}$,
\en{then}\ru{тогда}
$\locationvector$\en{~is}\ru{\:---}
\en{the~location vector}\ru{вектор положения}~(\en{the~radius vector}\ru{вектор-радиус})
\en{of~the~surface’s points}\ru{точек поверхности}.

\begin{tcolorbox}
\small\setlength{\abovedisplayskip}{2pt}\setlength{\belowdisplayskip}{2pt}

\emph{\en{Examples}\ru{Примеры}}

\begin{itemize}

\vspace{.2em}\item
\en{a~linear mapping}\ru{линейное отображение} \ru{есть}\en{is} \en{a~}\href{https://en.wikipedia.org/wiki/Plane_(geometry)}{\en{plane}\ru{плоскость}}
${\locationvector(a,\hspace{-0.15ex} b) \hspace{-0.25ex} = a \hspace{.1ex} \bm{e}_1 \hspace{-0.2ex} + b \hspace{.1ex} \bigl( \hspace{-0.1ex} \bm{e}_2 \hspace{-0.2ex} + \hspace{-0.1ex} \bm{e}_3 \bigr)}$

\vspace{.2em}\item
\en{a~}\href{https://en.wikipedia.org/wiki/Helicoid}{\en{helicoid}\ru{геликоид}}
${\locationvector(u,\hspace{-0.2ex} v) \hspace{-0.25ex} = u \sine v \hspace{.33ex} \bm{e}_1 \hspace{-0.15ex} + u \cosine v \hspace{.33ex} \bm{e}_2 \hspace{-0.15ex} + v \hspace{.1ex} \bm{e}_3}$

\vspace{.2em}\item
\en{a~}\en{cone}\ru{конус}
${\locationvector(u,\hspace{-0.2ex} v) \hspace{-0.25ex} = u \sine v \hspace{.33ex} \bm{e}_1 \hspace{-0.15ex} + u \cosine v \hspace{.33ex} \bm{e}_2 \hspace{-0.15ex} + u \hspace{.1ex} \bm{e}_3}$

\vspace{.2em}\item
\en{a~}\en{cylinder}\ru{цилиндр} \en{of~radius}\ru{радиуса}~${r \narroweq \constant}$

\nopagebreak\vspace{-0.2em}\begin{equation*}
\locationvector(u,\hspace{-0.2ex} v) \hspace{-0.25ex} = r \hspace{.1ex} \bigl( \cosine u \hspace{.33ex} \bm{e}_1 \hspace{-0.2ex} + \sine u \hspace{.33ex} \bm{e}_2 \bigr) \hspace{-0.3ex} + v \hspace{.1ex} \bm{e}_3
\end{equation*}

\item
\en{a~}\href{https://en.wikipedia.org/wiki/Torus}{\en{torus of~revolution}\ru{тор вращения}} \en{with radii}\ru{с~радиусами} $r$ \en{and}\ru{и}~$R$

\nopagebreak\vspace{-0.12em}\begin{equation*}
\hspace{10000pt minus 1fil}
\locationvector(\hspace{.1ex}p,\hspace{-0.1ex} q) \hspace{-0.25ex}
= \bm{e}_1 \bigl( r \cosine p + \hspace{-0.15ex} R \hspace{.1ex} \bigr) \hspace{-0.2ex} \cosine q
+ \bm{e}_2 \bigl( r \cosine p + \hspace{-0.15ex} R \hspace{.1ex} \bigr) \hspace{-0.2ex} \sine q
+ \bm{e}_3 \hspace{.2ex} r \sine p
\hfilneg
\end{equation*}

\item
\en{a~}\href{https://en.wikipedia.org/wiki/N-sphere}{2\hbox{-}\en{sphere}\ru{сфера}}\:--- \en{a~torus}\ru{тор} \en{with}\ru{с}~${\hspace{-0.1ex} R = 0}$

\nopagebreak\vspace{-0.12em}\begin{equation*}
\hspace{10000pt minus 1fil}
\locationvector(\hspace{.1ex}p,\hspace{-0.1ex} q) \hspace{-0.25ex}
= r \hspace{.1ex} \bigl( \cosine p \hspace{.1ex} \cosine q \hspace{.33ex} \bm{e}_1 \hspace{-0.15ex}
+ \cosine p \hspace{.1ex} \sine q \hspace{.33ex} \bm{e}_2 \hspace{-0.15ex}
+ \sine p \hspace{.33ex} \bm{e}_3 \bigr)
\hfilneg
\end{equation*}

\vspace{-0.1em}\item
\en{a~}\href{https://en.wikipedia.org/wiki/Paraboloid}{\en{paraboloid}\ru{параболоид}}
${
\locationvector(u,\hspace{-0.2ex} v) \hspace{-0.25ex} = u \hspace{.1ex} \bm{e}_1 \hspace{-0.2ex} + v \hspace{.1ex} \bm{e}_2 \hspace{-0.2ex} + \hspace{-0.2ex} \bigl( u^2 \hspace{-0.3ex} + \hspace{-0.1ex} v^2 \hspace{.1ex} \bigr) \bm{e}_3
}$
\en{or}\ru{или}
\en{via a~cylindrical parameterization}\ru{цилиндрической параметризацией}
${
\locationvector(\hspace{.1ex}\rho,\hspace{-0.15ex} \vartheta) \hspace{-0.25ex} = \hspace{-0.1ex} \rho \cosine \vartheta \hspace{.33ex} \bm{e}_1 \hspace{-0.2ex} + \hspace{-0.1ex} \rho \sine \vartheta \hspace{.33ex} \bm{e}_2 \hspace{-0.2ex} + \hspace{-0.1ex} \rho^2 \bm{e}_3
}$
\en{for}\ru{для}
\en{a~paraboloid}\ru{параболоида}
\en{of~revolution}\ru{вращения}.

\end{itemize}
\par\end{tcolorbox}
\vspace{-0.2em}

\en{The~continuous change}\ru{Непрерывное изменение}
\en{of~the~first coordinate}\ru{первой координаты}~${q^{1}\hspace{-0.3ex}}$,
\en{while}\ru{пока}
\en{the~second one}\ru{вторая}~${q^{2} \hspace{-0.1ex} \narroweq \hspace{.1ex} u^{\hspace{-0.1ex}*} \hspace{-0.1ex} \narroweq \hspace{.15ex} \constant}$
\en{is }\inquotesx{\en{frozen}\ru{заморожена}}[,]
\en{gives}\ru{даёт}
\en{the~coordinate line}\ru{координатную линию}~${\oneparametervector{\locationvector}(q^1) \hspace{-0.3ex} = \hspace{-0.12ex} \locationvector(q^1 \hspace{-0.4ex} , u^{\hspace{-0.1ex}*} \hspace{-0.1ex})}$.
\en{The~crossing}\ru{Пересечение}
\en{of~the~two coordinate lines}\ru{двух координатных линий}
${q^1 \hspace{-0.1ex} \narroweq \hspace{.1ex} v^{*}\hspace{-0.3ex}}$
\en{and}\ru{и}~${q^2 \hspace{-0.1ex} \narroweq \hspace{.1ex} w^{*}\hspace{-0.3ex}}$
\en{uniquely identifies}\ru{однозначно определяет}
\en{the~point}\ru{точку}~${\locationvector(v^{*} \hspace{-0.4ex}, w^{*}\hspace{-0.1ex})}$
\en{of~the~surface}\ru{поверхности}.

\en{Vectors}\ru{Векторы}

\nopagebreak\vspace{-1em}\begin{equation}
\locationvector_\differentialindex{\alpha} \hspace{-0.15ex} \equiv \partial_{\hspace{-0.1ex}\alpha} \locationvector
, \hspace{.5em}
\partial_{\hspace{-0.1ex}\alpha} \hspace{-0.15ex} \equiv \scalebox{0.8}{$\displaystyle\frac{\raisemath{-0.2em}{\partial}}{\partial q^{\hspace{.1ex}\alpha}}$}
\end{equation}

\vspace{-0.2em}\noindent
\en{are tangent}\ru{касательны}
\en{to coordinate lines}\ru{к~координатным линиям}.
\en{If}\ru{Если}
\en{they}\ru{они}
\en{are linearly independent}\ru{линейно независимы}
(\en{that is}\ru{то есть}
${\locationvector_\differentialindex{1} \hspace{-0.3ex} \times \hspace{-0.1ex} \locationvector_\differentialindex{2} \hspace{-0.1ex} \neq \zerovector}$)\footnote{%
\en{Sometimes somewhere}\ru{Иногда где\hbox{-}нибудь}\:---
\en{at the~so\hbox{-}called singular points}\ru{в~так называемых сингулярных точках}\:---
\en{it’s not so}\ru{это не так}.
\en{As example}\ru{Как пример},
\en{for}\ru{для}
\en{a~}2\hbox{-}\en{sphere}\ru{сферы}
\en{of~unit radius}\ru{единичного радиуса}

\nopagebreak\vspace{.44em}\scalebox{.92}{\begin{minipage}{\linewidth}
\begin{align*}
\locationvector(\hspace{.1ex}p, q) \hspace{-0.15ex} &= \hspace{.1ex} \cosine p \hspace{.1ex} \cosine q \hspace{.33ex} \bm{e}_1 \hspace{-0.15ex} + \cosine p \hspace{.1ex} \sine q \hspace{.33ex} \bm{e}_2 \hspace{-0.15ex} + \sine p \hspace{.33ex} \bm{e}_3
\hspace{.1ex} ,
\\
%
\locationvector_\differentialindex{p} \hspace{-0.2ex} = \partial_{p} \locationvector \hspace{-0.1ex} &= - \hspace{.1ex} \sine p \hspace{.1ex} \cosine q \hspace{.33ex} \bm{e}_1 \hspace{-0.2ex} - \sine p \hspace{.1ex} \sine q \hspace{.33ex} \bm{e}_2 \hspace{-0.15ex} + \cosine p \hspace{.33ex} \bm{e}_3
\hspace{.1ex} ,
\\[-0.1em]
%
\locationvector_\differentialindex{q} \hspace{-0.2ex} = \partial_{\hspace{-0.06ex}q} \locationvector \hspace{-0.1ex} &= - \hspace{.1ex} \cosine p \hspace{.1ex} \sine q \hspace{.33ex} \bm{e}_1 \hspace{-0.15ex} + \cosine p \hspace{.1ex} \cosine q \hspace{.33ex} \bm{e}_2 \hspace{-0.15ex} + 0 \hspace{.2ex} \bm{e}_3
\hspace{.1ex} ,
\end{align*}\end{minipage}}

\nopagebreak\vspace{.2em}\noindent
\en{a~pole}\ru{полюс}~%
${p \hspace{-0.2ex} = \hspace{-0.2ex} \pm \frac{\pi}{2}}$\ru{\:---}\en{ is}
\en{a~singular point}\ru{сингулярная точка}:
${\locationvector_\differentialindex{q} \bigr|_{\raisemath{.1em}{p \hspace{.2ex} = \hspace{.2ex} \raisemath{.2em}{\pm \hspace{.1ex} \pi} \hspace{-0.4ex} / \hspace{-0.2ex} \raisemath{-0.15em}{\scalebox{0.66}{$2$}}}} \hspace{-0.44ex}
= \zerovector \hspace{.1ex}}$.
}\hbox{\hspace{-0.3ex},}
\en{they compose}\ru{они составляют}
\en{the~local basis}\ru{локальный базис}
\en{for representing}\ru{для представления}
\en{any vector}\ru{любого вектора}~$\twoparametervector{\bm{v}}$
\en{in the~tangent plane}\ru{в~касательной плоскости}
\en{as linear combination}\ru{как линейной комбинации}

\nopagebreak\vspace{-0.2em}
\begin{equation}
\begin{array}{c}
\twoparametervector{\bm{v}} = v^\alpha \locationvector_\differentialindex{\alpha} \hspace{-0.22ex} = v_\alpha \locationvector^\alpha
\hspace{-0.4ex} ,
\\[.1em]
%
v^\alpha \hspace{-0.33ex} = \twoparametervector{\bm{v}} \dotp \locationvector^\alpha
\hspace{-0.4ex} , \hspace{.5em}
v_\alpha \hspace{-0.33ex} = \twoparametervector{\bm{v}} \dotp \locationvector_\differentialindex{\alpha}
\hspace{.1ex} , \hspace{.5em}
\locationvector^\alpha \hspace{-0.25ex} \dotp \locationvector_\differentialindex{\hspace{-0.1ex}\beta} \hspace{-0.25ex} = \hspace{-0.1ex} \delta^\alpha_{\hspace{-0.2ex}\beta}
\hspace{.1ex} .
\end{array}
\end{equation}

\vspace{-0.25em}\noindent
\en{Here}\ru{Здесь}
${\locationvector^{\alpha}\hspace{-0.15ex}}$\en{ is}\ru{\:---}
\en{the~local reciprocal basis}\ru{локальный взаимный базис}
\en{in the~(co)tangent plane}\ru{в~(ко)касательной плоскости}.

\en{The~dield}\ru{Поле}
\en{of~unit normal vectors}\ru{единичных нормальных векторов}~${\unitnormalvector(q^{\hspace{.1ex}\alpha})}$
\en{adds}\ru{добавляет}
\en{at~every point}\ru{в~каждой точке}
\en{of~the~surface}\ru{поверхности}
(${\hspace{.2ex} \forall \locationvector(q^{\hspace{.1ex}\alpha}) \hspace{.12ex}\Leftrightarrow\hspace{.2ex} \forall q^{\hspace{.1ex}\alpha}}$)
\en{the~unit}\ru{единичную}\footnote{%
${\| \bm{a} \| \hspace{-0.1ex} \equiv \hspace{-0.1ex} \smash{\sqrt{\vphantom{j} \bm{a} \hspace{-0.12ex} \dotp \hspace{-0.1ex} \bm{a}}}\hspace{.1ex}}$\en{ is}\ru{\:---}
\en{the~length}\ru{длина}
\en{of~vector}\ru{вектора}~${\bm{a}\hspace{.1ex}}$.
}\hspace{-0.3ex}
\textcolor{magenta}{\en{normal}\ru{нормаль}}

\nopagebreak\vspace{-0.2em}
\begin{equation}
\unitnormalvector = \scalebox{.92}{$
\displaystyle \frac{\raisemath{-0.1em}{\locationvector_\differentialindex{1} \hspace{-0.2ex} \times \locationvector_\differentialindex{2}}}%
{\hspace{.4ex} \| \hspace{.16ex} \locationvector_\differentialindex{1} \hspace{-0.2ex} \times \locationvector_\differentialindex{2} \hspace{.2ex} \| \hspace{.4ex}}
$}
\hspace{.2ex} .
\end{equation}

\vspace{-0.1em}\noindent
\en{At non-singular points}\ru{В~несингулярных точках},
\en{three vectors}\ru{три вектора}
${\locationvector_\differentialindex{1}}$,
${\locationvector_\differentialindex{2}}$
\en{and}\ru{и}~${\unitnormalvector}$
\en{can be taken as a~basis}\ru{могут быть взяты как базис}
\en{for the~entire three-dimensional space}\ru{для всего трёхмерного пространства},
\en{giving}\ru{давая}
\en{decomposition}\ru{разложение}
\en{for}\ru{для}
\en{any vector}\ru{любого вектора}
\en{and}\ru{и}~\en{any tensor}\ru{любого тензора},
\en{for example}\ru{например}
${\threeparametervector{\bm{u}} = u^\alpha \locationvector_\differentialindex{\alpha} \hspace{-0.2ex} + u^{\hspace{-0.1ex}n} \unitnormalvector}$.

${u^{\hspace{-0.1ex}n} \hspace{-0.25ex} = u_n}$

\begin{tcolorbox}
\small\setlength{\abovedisplayskip}{2pt}\setlength{\belowdisplayskip}{2pt}

\en{for}\ru{для}
\en{a~}2\hbox{-}\en{sphere}\ru{сферы}
\en{of~unit radius}\ru{единичного радиуса}

\begin{equation*}
\locationvector_\differentialindex{p} \hspace{-0.25ex} \times \locationvector_\differentialindex{q} \hspace{-0.2ex}
= - \determinant\hspace{-0.2ex}
\scalebox{.92}{$\left[\hspace{-0.2ex}\begin{array}{c@{\hspace{.6em}}c@{\hspace{.44em}}c}
- \hspace{.1ex} \sine p \hspace{.1ex} \cosine q & \bm{e}_1 & - \hspace{.1ex} \cosine p \hspace{.1ex} \sine q \\
- \hspace{.1ex} \sine p \hspace{.1ex} \sine q & \bm{e}_2 & \cosine p \hspace{.1ex} \cosine q \\
\cosine p & \bm{e}_3 & 0
\end{array}\hspace{-0.1ex}\right]$} \hspace{-0.5ex} = ...
\end{equation*}

\par\end{tcolorbox}
\vspace{-0.2em}

\en{The~bivalent}\ru{Бивалентные}
\en{unit}\ru{единичные}~(\inquotes{\en{metric}\ru{метрические}})
\en{tensors}\ru{тензоры},
$\UnitDyad$~\en{in space}\ru{в~пространстве}
\en{and}\ru{и}~%
$\twoparameterunitdyad$~\en{in the~tangent plane}\ru{в~касательной плоскости}

\nopagebreak\vspace{-0.25em}
\begin{equation*}
\UnitDyad = \twoparameterunitdyad + \unitnormalvector \unitnormalvector
\hspace{.1ex} , \:\:
\twoparameterunitdyad \hspace{.1ex} \equiv \UnitDyad - \unitnormalvector \unitnormalvector
= \locationvector_\differentialindex{\alpha} \locationvector^\alpha \hspace{-0.25ex}
= \locationvector^\alpha \locationvector_\differentialindex{\alpha}
\hspace{.1ex} .
\vspace{-0.2em}
\end{equation*}

\en{Representation}\ru{Представление}
\en{of~the~location vector}\ru{вектора положения}~$\threedimensionallocation$
\en{for}\ru{для}
\en{any point in space}\ru{любой точки пространства}
\en{at distance}\ru{на расстоянии}~$h$
\en{from the~surface}\ru{от~поверхности}
(${\partial_{\hspace{-0.1ex}\alpha} h = 0}$)

\nopagebreak\vspace{-0.2em}
\begin{equation}\label{locationvector.fromsurfacetospace}
\threedimensionallocation(q^{\hspace{.1ex}\alpha} \hspace{-0.4ex}, h)
= \hspace{.1ex} \locationvector(q^{\hspace{.1ex}\alpha}) + \hspace{.1ex} h \hspace{.1ex} \unitnormalvector(q^{\hspace{.1ex}\alpha})
\end{equation}

\nopagebreak\vspace{-0.25em}\noindent
\en{gives}\ru{даёт}
\en{the~local}\ru{локальный}
\en{basis}\ru{базис}
\en{in~the~tangent space}\ru{в~касательном пространстве}

\nopagebreak\vspace{-0.2em}
\begin{gather*}
\threedimensionallocation_\differentialindex{n} \hspace{-0.25ex} = \unitnormalvector = \threedimensionallocationwithupperindex{n} \hspace{-0.33ex} ,
\\
%
\threedimensionallocation_\differentialindex{\alpha} \hspace{-0.2ex} \equiv \partial_{\hspace{-0.1ex}\alpha} \threedimensionallocation \hspace{-0.15ex}
= \partial_{\hspace{-0.1ex}\alpha} \locationvector + h \hspace{.2ex} \partial_{\hspace{-0.1ex}\alpha} \unitnormalvector
= \locationvector_\differentialindex{\alpha} + \hspace{.1ex} h \hspace{.2ex} \locationvector_\differentialindex{\alpha} \hspace{-0.2ex} \dotp \locationvector^\beta \hspace{-0.1ex} \partial_{\hspace{-0.1ex}\beta} \unitnormalvector
\hspace{.1ex} .
\end{gather*}

...

\noindent
\en{differential operator}\ru{дифференциальный оператор}
\inquotes{\en{nabla}\ru{набла}}

\en{in space}\ru{в~пространстве}
${\boldnabla \equiv \threedimensionallocationwithupperindex{i} \hspace{-0.1ex} \partial_i}$

\en{in the~tangent plane}\ru{в~касательной плоскости}
${\boldnablaflat \equiv \hspace{.1ex} \locationvector^\alpha \hspace{-0.1ex} \partial_{\hspace{-0.1ex}\alpha}}$

...

\begin{equation*}
\threedimensionallocation_\differentialindex{\alpha} \hspace{-0.2ex}
= \locationvector_\differentialindex{\alpha} \hspace{-0.15ex} \dotp \hspace{-0.16ex}
\Bigl(
\locationvector^\beta \locationvector_\differentialindex{\beta}
+ h \hspace{.2ex} \locationvector^\beta \hspace{-0.1ex} \partial_{\hspace{-0.1ex}\beta} \unitnormalvector
\Bigr) \hspace{-0.3ex}
= \locationvector_\differentialindex{\alpha} \hspace{-0.15ex} \dotp \hspace{-0.16ex}
\Bigl(
\twoparameterunitdyad \hspace{.1ex}
+ h \hspace{.1ex} \boldnablaflat \unitnormalvector
\Bigr) \hspace{-0.3ex}
= \locationvector_\differentialindex{\alpha} \hspace{-0.15ex} \dotp \hspace{-0.12ex}
\Bigl(
\twoparameterunitdyad \hspace{.1ex}
- \hspace{-0.15ex} \twoparameterbivalent{\bm{c}} \hspace{.2ex} h
\Bigr)
\end{equation*}

\en{The two\hbox{-}coordinate}\ru{Двухкоординатный}
\en{bivalent}\ru{бивалентный}
\en{tensor}\ru{тензор}

\nopagebreak\vspace{-0.2em}
\begin{equation}
\twoparameterbivalent{\bm{c}} \hspace{.1ex} \equiv - \boldnablaflat \unitnormalvector = - \hspace{.2ex} \locationvector^\alpha \hspace{-0.1ex} \partial_{\hspace{-0.1ex}\alpha} \unitnormalvector
\end{equation}

\nopagebreak\vspace{-0.25em}\noindent
\en{characterizes}\ru{характеризует}
\en{the~surface’s curvature}\ru{кривизну поверхности}.

....

\en{cobasis}\ru{кобазис}
${\threedimensionallocationwithupperindex{\alpha} \hspace{-0.4ex} \dotp \hspace{-0.1ex} \threedimensionallocation_\differentialindex{\beta} \hspace{-0.2ex} = \delta^\alpha_{\hspace{-0.2ex}\beta}}$,
${\threedimensionallocationwithupperindex{i} \hspace{-0.4ex} \dotp \hspace{-0.1ex} \threedimensionallocation_\differentialindex{\hspace{-0.1ex}j} \hspace{-0.2ex} = \delta^{i}_{\hspace{-0.25ex}j}}$

\nopagebreak\vspace{-0.2em}
\begin{equation*}
\threedimensionallocationwithupperindex{\alpha} \hspace{-0.5ex} = \hspace{-0.2ex} \Bigl( \hspace{-0.1ex}
\twoparameterunitdyad \hspace{.1ex}
+ h \hspace{.1ex} \boldnablaflat \unitnormalvector
\Bigr)^{\hspace{-0.6ex}\expminusone} \hspace{-0.9ex} \dotp \hspace{.1ex} \locationvector^\alpha
\hspace{-0.25ex} , \hspace{.5em}
\threedimensionallocationwithupperindex{n} \hspace{-0.33ex} = \hspace{.1ex} \unitnormalvector
\end{equation*}

\en{relation between}\ru{связь между}
${\hspace{-0.2ex}\boldnabla}$
\en{and}\ru{и}~${\hspace{-0.2ex}\boldnablaflat}$

\nopagebreak\vspace{-0.2em}
\begin{equation*}
\boldnabla = \hspace{-0.16ex} \Bigl( \hspace{-0.1ex}
\twoparameterunitdyad \hspace{.1ex}
+ h \hspace{.1ex} \boldnablaflat \unitnormalvector
\Bigr)^{\hspace{-0.6ex}\expminusone} \hspace{-0.9ex} \dotp \hspace{-0.2ex} \boldnablaflat
\hspace{.2ex} + \hspace{.1ex}
\unitnormalvector \hspace{.12ex} \partial_n
\end{equation*}

...

\en{\section{The model of a~shell}}

\ru{\section{Модель оболочки}}

\en{Having}\ru{Имея}
\en{the~models}\ru{модели}
\en{of~three-dimensional}\ru{трёхмерного}
\en{micropolar}\ru{микрополярного}
\en{continuum}\ru{\rucontinuum{}а}
(\chapterref{chapter:cosseratcontinuum})
\en{and}\ru{и}~\en{one-dimensional rods}\ru{одномерных стержней}
(\chapterref{chapter:rods},
\chapterref{chapter:thinwalledrods}),
\en{the~mechanics}\ru{механику}
\en{of~two-dimensional}\ru{двумерных}
\en{shells}\ru{оболочек}
\en{is pretty easy}\ru{довольно легко}
\en{to~describe}\ru{опис\'{а}ть}.

\en{As}\ru{Как}
\en{a~geometrical object}\ru{геометрический объект},
\en{the shell}\ru{оболочка}
\en{is defined}\ru{определяется}
\en{by its}\ru{своей}
\en{middle surface}\ru{срединной поверхностью}
\en{and}\ru{и}~\en{thickness}\ru{толщиной}~${\hcursive\hspace{-0.22ex}}$,
\en{thus}\ru{так что}
\en{in}\ru{в}~\eqref{locationvector.fromsurfacetospace}

\nopagebreak\vspace{-0.2em}%
\begin{equation*}
\scalebox{.9}{$%
\raisemath{.3em}{- \hspace{.1ex} \hcursive}
\hspace{-0.5ex} / \hspace{-0.3ex}
\raisemath{-0.4em}{\scalebox{.9}{$ 2 $}}%
$}
\leq
\hspace{.1ex} h
\leq
\scalebox{.9}{$%
\raisemath{.3em}{\hcursive}
\hspace{-0.5ex} / \hspace{-0.3ex}
\raisemath{-0.4em}{\scalebox{.9}{$ 2 $}}%
$}
\hspace{.4ex} .
\end{equation*}

...


\en{\section{The balance of forces and moments for a~shell}}

\ru{\section{Баланс сил и моментов для оболочки}}

\en{When}\ru{Когда}
${\variation{\fieldofdisplacements} = \boldconstant}$
\en{and}\ru{и}~${\variation{\fieldofrotations} = \zerovector}$
(\en{a~translation}\ru{трансляция}) ...

....


\en{\section{Shells: The relations of elasticity}}

\ru{\section{Оболочки: Отношения упругости}}

\begin{otherlanguage}{russian}

Локальное соотношение~\eqref{somereftwothreeeee} после вывода уравнений баланса ...

...



\end{otherlanguage}

\en{\section{The classical theory of shells}}

\ru{\section{Классическая теория оболочек}}

\begin{otherlanguage}{russian}

Вышеизложенная теория
(напоминающая балку Тимошенко
и~\rucontinuum{}ы Cosserat)
рассматривает
\en{rotations}\ru{повороты}~$\fieldofrotations$
независимо от \en{displacements}\ru{смещений}~$\fieldofdisplacements$.
Но опыт подсказывает, что
материальный элемент,
нормальный к~срединной поверхности до~деформации,
остаётся таким
и~после деформации
(кинематическая гипотеза Kirchhoff’а).
В~классической теории Kirchhoff’а,
Арона и~Love’а
\en{field}\ru{поле}~$\fieldofrotations$
выражается через~$\fieldofdisplacements$,
что в~конце концов
даёт свести всё
к~одному векторному уравнению
для~$\fieldofdisplacements$.

Предположим,
что в~основе классической теории
лежит
внутренняя связь

...


\end{otherlanguage}

\en{\section{Shells: A plate}}

\ru{\section{Оболочки: Пластина}}

\begin{otherlanguage}{russian}

\en{Plate}\ru{Пластина} \en{is}\ru{есть} \en{the~simplest kind}\ru{простейший вид} \en{of~shell}\ru{оболочки}.
Единичный перпендикуляр ${\unitnormalvector = \bm{k}}$ направлен по декартовой оси~$z$, в~качестве координат ...

...



\end{otherlanguage}

\en{\section{Shells: Approach with Lagrange multipliers}}

\ru{\section{Оболочки: Подход с множителями Лагранжа}}

\begin{otherlanguage}{russian}

Уязвимым местом этого изложения теории оболочек являются формулы

...



\end{otherlanguage}

\en{\section{Cylindrical shell}}

\ru{\section{Цилиндрическая оболочка}}

\begin{otherlanguage}{russian}

Существуют разные уравнения цилиндрической оболочки.
Приводятся громоздкие выкладки с~отбрасыванием некоторых малых членов, и~не~всегда ясно, какие именно члены действительно можно отбросить.

Предлагаемая читателю теория оболочек иного свойства: лишних членов нет, все уравнения записаны в~компактной тензорной форме\:--- остаётся лишь \textcolor{magenta}{грамотно} действовать с~компонентами тензоров.
В~качестве иллюстрации рассмотрим цилиндрическую оболочку.

В~декартовой системе

...



\end{otherlanguage}

\en{\section{Shells: Common theorems}}

\ru{\section{Оболочки: Общие теоремы}}

\begin{otherlanguage}{russian}

Пусть край закреплён

...



\end{otherlanguage}

\en{\section{Shells: Boundary conditions}}

\ru{\section{Оболочки: Краевые условия}}

\begin{otherlanguage}{russian}

В~рамках рассматриваемого прямого подхода к~оболочкам как материальным поверхностям наиболее надёжным способом вывода граничных условий представляется вариационный.
Исходим из вариационного уравнения:

...



\end{otherlanguage}

\en{\section{Shells of revolution}}

\ru{\section{Оболочки вращения}}

\noindent
\emph{Surface of~revolution (reference surface of shell of revolution) is created by rotating a~plane curve (the~meridian, the~generatrix) about a~straight line in the~plane of curve (an~axis of~rotation).}

\begin{otherlanguage}{russian}

Разберёмся в~геометрии поверхности вращения~(~\textcolor{blue}{рисунок}~).
Меридиан можно задать зависимостью декартовых координат

...



\end{otherlanguage}

\en{\section{Momentless theory of shells}}

\ru{\section{Безмоментная теория оболочек}}

\begin{otherlanguage}{russian}

В~отличие от~пластины, оболочка способна выдерживать нормальную распределённую нагрузку без появления внутренних моментов.
В~безмоментном состоянии напряжения равномерно распределены по~толщине оболочки, безмоментные оболочечные конструкции можно считать оптимально спроектированными.

Уравнения безмоментной теории

...



\end{otherlanguage}

\en{\section{Shells: Nonlinear momentless theory}}

\ru{\section{Оболочки: Нелинейная безмоментная теория}}

\begin{otherlanguage}{russian}

Вышеизложенную безмоментную теорию оболочек возможно просто и~корректно обобщить на нелинейную постановку.
Материальная поверхность состоит из~частиц

...



\end{otherlanguage}

\en{\section{Shells: Other variant of classical theory}}

\ru{\section{Оболочки: Иной вариант классической теории}}

\begin{otherlanguage}{russian}

%% \noindent \emph{(добавлено в~готовую рукопись перед изданием 1999го года)}

\noindent Выше при изложении моментной теории оболочек частицы материальной поверхности считались твёрдыми телами с~шестью степенями свободы

...



\end{otherlanguage}

\en{\section{Plates: Overall concepts}}

\ru{\section{Пластины: Общие представления}}

\label{section:overviewofplates}

\begin{otherlanguage}{russian}

\dropcap{П}{ластиной} называется тонкое трёхмерное тело, ограниченное двумя параллельными плоскостями и~боковой цилиндрической поверхностью (?? \textcolor{blue}{рисунок} ??).
В~декартовых координатах ${x_1, x_2, z}$ поперечная координата ...

...

В~теории пластин рассматриваются двумерные задачи.
Переход от трёхмерной задачи наиболее достоверен на~пути асимптотики.
Но логическая стройность и~эффективность присуща и~вариационному подходу, основанному на аппроксимации по~толщине решения трёхмерной вариационной задачи.
Самое~же простое корректное изложение теории пластин характерно для прямого подхода к~ним как материальным плоскостям.

...



\end{otherlanguage}

\en{\section{Timoshenko-like model of a~plate (direct approach)}}

\ru{\section{Модель пластины типа Тимошенко (прямой подход)}}

\begin{otherlanguage}{russian}

Пластина рассматривается как материальная плоскость, частицы которой

...



\end{otherlanguage}

\en{\section{Kirchhoff’s classical theory of plates}}

\ru{\section{Классическая теория пластин Kirchhoff’а}}

\begin{otherlanguage}{russian}

Принимается внутренняя связь

...



\end{otherlanguage}

\en{\section{Plates: Asymptotic matching of two-dimensional models}}

\ru{\section{Пластины: Асимптотическое сопоставление двумерных моделей}}

\begin{otherlanguage}{russian}

При малой толщине из теории типа Тимошенко следует классическая теория.
Толщина~${\hcursive\hspace{-0.22ex}}$ определяется отношением жёсткостей.
Перепишем

...



\end{otherlanguage}

\en{\section{Plates: Variational transition from the three-dimensional model}}

\ru{\section{Пластины: Вариационный переход от трёхмерной модели}}

\begin{otherlanguage}{russian}

\en{Using}\ru{Используя}
\en{the variational principles}\ru{вариационные принципы}
\en{by }Lagrange\ru{’а}
\en{or}\ru{или}
\en{by }Hellinger\ru{’а}
\en{and}\ru{и}
Reissner\ru{’а}
с~аппроксимацией решения по~толщине,
можно получить
двумерные
вариационные
формулировки
проблем.
Из этих вариационных принципов
вытекают
и~соотношения внутри области,
\en{and the natural}\ru{и~натуральные}%естественные
\en{boundary conditions}\ru{краевые условия}.

\en{For example}\ru{Для примера}\en{,}
\en{here is}\ru{вот}
\en{the model}\ru{модель}
\en{of the Timoshenko type}\ru{типа Тимошенко}
\en{with the approximation}\ru{с~аппроксимацией}
\en{of~displacements}\ru{смещений}


...


...



\en{The variational transitions}\ru{Вариационные переходы}
\en{can be easily generalized}\ru{могут быть легко обобщены}
\en{for the cases}\ru{для случаев}
\en{of the temperature deformations}\ru{температурных деформаций},
\en{the inhomogeneity}\ru{неоднородности}
(\en{heterogeneity}\ru{гетерогенности})
\en{and}\ru{и}~\en{anisotropy}\ru{анизотропии}
\en{материала}\ru{of the material},
\en{the dynamics}\ru{динамики}.
\en{The advantage}\ru{Преимущество}
\en{of the}\ru{принципа} Hellinger\ru{’а}--Reissner\ru{’а}\en{ principle}
состоит
в~явном представлении
напряжений.
\hbox{Зато}
принцип Лагранжа
примен\'{и}м
\en{and}\ru{и}
к~нелинейным задачам
(\en{in}\ru{в}~\customref[\en{the chapter}\ru{главе}~]{chapter:nonlinearcontinuum}
описана трёхмерная постановка).

\end{otherlanguage}

\en{\section{Plates: Splitting of three-dimensional bending problem}} % Asymptotic splitting

\ru{\section{Пластины: Расщепление трёхмерной задачи изгиба}} % Асимптотическое расщепление

\begin{otherlanguage}{russian}

Двумерная классическая теория изгиба пластин легко выводится из трёхмерной постановки с~малым параметром.
Представив радиус\hbox{-}вектор в~объёме

...



\end{otherlanguage}

\en{\section{Circular plates}} % Round plates

\ru{\section{Круглые пластины}}

\begin{otherlanguage}{russian}

В~качестве иллюстрации рассмотрим широко представленный в~литературе вопрос об~уравнениях теории Kirchhoff’а в~полярных координатах.

...



\end{otherlanguage}

\en{\section{Plates: Plane stress}}

\ru{\section{Пластины: Плоское напряжение}}

\begin{otherlanguage}{russian}

Это вторая из~двух задач, о~которых говорилось в~\sectionref{section:overviewofplates}.
Силы

...




\end{otherlanguage}

\vspace{8mm}
\hfill\begin{minipage}[b]{0.95\linewidth}
\fontsize{10}{12}\selectfont

\section*{\wordforbibliography}

\begin{otherlanguage}{russian}

Теория оболочек изложена в~монографиях А.\,Л.\;Гольденвейзера~\cite{goldenveizer-thinshells}, В.\,В.\;Новожилова~\cite{novozhilov-theoryofthinshells}, А.\,И.\;Лурье~\cite{lurie-thinwalledshells}, В.\,С.\;Черниной~\cite{chernina-thinwalledshells} и~ряде других. Достоинства этих книг перекрывают неразвитость формального аппарата. Переход от~трёхмерной модели оболочки к~двумерной рассмотрен у~...

...

Техническая теория изгиба пластин изложена ...

...

\end{otherlanguage}

\end{minipage}
