\en{\chapter{Shells and plates}}

\ru{\chapter{Оболочки и пластины}}

\thispagestyle{empty}

\label{chapter:shellsandplates}

\begin{otherlanguage}{russian}

\section{Оболочки: Геометрия поверхностей}

\lettrine[lines=2, findent=2pt, nindent=0pt]{П}{оверхность} в~трёхмерном пространстве определяется заданием вектора\hbox{-}радиуса как функции двух координат ${\bm{r} \!=\! \bm{r}(q^{\hspace{.16ex}\alpha})}$, ${\alpha = 1, 2}$. Фиксировав одну координату~--- пусть ...




\section{Оболочки: Модель оболочки}

Располагая моделями трёхмерного моментного континуума, стержней и~пластин, не~так~уж сложно разобраться в~механике оболочек. Как~геометрический объект, оболочка определяется ...

...




\section{Оболочки: Баланс сил и моментов}

При~${\variation{\bm{u}} = \boldconst}$ и~${\variation{\bm{\theta}} = \bm{0}}$~(трансляция) ...

...




\en{\section{Shells: Relations of elasticity}}

\ru{\section{Оболочки: Отношения упругости}}

Локальное соотношение~\eqref{somereftwothree} после вывода уравнений баланса ...

...



\section{Оболочки: Классическая теория}

Вышеизложенная теория (напоминающая балку Тимошенко и~континуумы Коссера) рассматривает~$\bm{\theta}$ независимо от~$\bm{u}$. Но обыденный опыт подсказывает: материальный элемент, нормальный к~срединной поверхности до~деформации, остаётся таковым и~после (кинематическая гипотеза Кирхгофа). В~классической теории Кирхгофа, Арона и~Лява $\bm{\theta}$ выражается через~$\bm{u}$, что в~конце концов позволяет всё свести к~одному векторному уравнению для~$\bm{u}$.

Предположим, что в~основе классической теории лежит внутренняя связь

...



\section{Оболочки: Пластина}

Это простейший вырожденный случай оболочки. Орт ${\bm{n} = \bm{k}}$ направлен по декартовой оси~$z$, в~качестве координат ...

...



\section{Оболочки: Подход с множителями Лагранжа}

Уязвимым местом нашего изложения теории оболочек являются формулы

...



\section{Оболочки: Цилиндрическая оболочка}

Существуют разные уравнения цилиндрической оболочки. Приводятся громоздкие выкладки с~отбрасыванием некоторых малых членов, и~не~всегда ясно, какие именно члены действительно можно отбросить.

Предлагаемая читателю теория оболочек иного свойства: лишних членов нет, все уравнения записаны в~компактной тензорной форме~--- остаётся лишь грамотно действовать с~компонентами тензоров. В~качестве иллюстрации рассмотрим цилиндрическую оболочку.

В~декартовой системе

...



\section{Оболочки: Общие теоремы}

Пусть край закреплён

...



\section{Оболочки: Краевые условия}

В~рамках рассматриваемого прямого подхода к~оболочкам как материальным поверхностям наиболее надёжным способом вывода граничных условий представляется вариационный. Исходим из вариационного уравнения:

...



\section{Оболочки: Оболочки вращения}

Разберёмся в~геометрии поверхности вращения~( рис. ?? 29 ?? ). Меридиан можно задать зависимостью декартовых координат

...



\section{Оболочки: Безмоментная теория}

В~отличие от~пластины, оболочка способна выдерживать нормальную распределённую нагрузку без появления внутренних моментов. В~безмоментном состоянии напряжения равномерно распределены по~толщине оболочки, безмоментные оболочечные конструкции можно считать оптимально спроектированными.

Уравнения безмоментной теории

...



\section{Оболочки: Нелинейная безмоментная теория}

Вышеизложенная безмоментная теория допускает простое и~корректное нелинейное обобщение. Материальная поверхность состоит из~частиц

...



\section{Оболочки: Иной вариант классической теории}

%% \noindent \emph{(добавлено в~готовую рукопись перед изданием 1999го года)}

\noindent Выше при изложении моментной теории оболочек частицы материальной поверхности считались твёрдыми телами с~шестью степенями свободы

...




\section{Пластины: Общие представления}
\label{para:overviewofplates}

\lettrine[lines=2, findent=2pt, nindent=0pt]{П}{ластиной} называется тонкое трёхмерное тело, ограниченное двумя параллельными плоскостями и~боковой цилиндрической поверхностью (рисунок ?? 26 ???). В~декартовой системе $x_1$,\;$x_2$,\;$z$ поперечная координата ...

...



\section{Пластины: Модель типа Тимошенко (прямой подход)}

Пластина рассматривается как материальная плоскость, частицы которой

...



\section{Пластины: Классическая теория Кирхгофа}

Принимается внутренняя связь

...



\section{Пластины: Асимптотическое соотношение двумерных моделей}

При малой толщине из теории типа Тимошенко вытекает классическая. Толщина~$h$ определяется отношением жёсткостей. Перепишем

...



\section{Пластины: Вариационный переход от трёхмерной модели}

Используя вариационные принципы Лагранжа или Рейсснера--Хеллингера с~аппроксимацией решения по~толщине, можно получить двумерные вариационные постановки. Из~них вытекают и~соотношения внутри области, и~естественные краевые условия.

Для примера рассмотрим построение модели типа Тимошенко с~аппроксимацией перемещений

...


...

Рассмотренные вариационные переходы легко обобщаются на~случаи неоднородности и~анизотропии материала, температурных деформаций, динамики. Достоинство принципа Рейсснера--Хеллингера~--- в~явном представлении напряжений. \hbox{Зато} принцип Лагранжа примен\'{и}м к~нелинейным задачам (\customref[глава~]{chapter:nonlinearcontinuum} содержит трёхмерную постановку).

\en{\section{Plates: Splitting of three-dimensional bending problem}} % Asymptotic splitting

\ru{\section{Пластины: Ращепление трёхмерной задачи изгиба}} % Асимптотическое ращепление

Двумерная классическая теория изгиба пластин легко выводится из трёхмерной постановки с~малым параметром. Представив радиус\hbox{-}вектор в~объёме

...



\en{\section{Plates: Circular plates}} % Round plates

\ru{\section{Пластины: Круглые пластины}}

В~качестве иллюстрации рассмотрим широко представленный в~литературе вопрос об~уравнениях теории Кирхгофа в~полярных координатах. Имеем

...



\section{Пластины: Плоское напряжённое состояние}

Это вторая из~двух задач, о~которых говорилось в~\pararef{para:overviewofplates}. Силы

...




\end{otherlanguage}

\vspace{8mm}
\hfill\begin{minipage}[b]{0.95\linewidth}
\fontsize{10}{12}\selectfont

\section*{\wordforbibliography}

\begin{otherlanguage}{russian}

Теория оболочек изложена в~монографиях А.\,Л.\;Гольденвейзера~\cite{goldenveizer-thinshells}, В.\,В.\;Новожилова~\cite{novozhilov-theoryofthinshells}, А.\,И.\;Лурье~\cite{lurie-thinwalledshells}, В.\,С.\;Черниной~\cite{chernina-thinwalledshells} и~ряде других. Достоинства этих книг перекрывают неразвитость формального аппарата. Переход от~трёхмерной модели оболочки к~двумерной рассмотрен у~...

...

Техническая теория изгиба пластин изложена ...

...

\end{otherlanguage}

\end{minipage}
