\en{\section{Eigenvectors and eigenvalues}}

\ru{\section{Собственные векторы и собственные числа}}

\label{para:eigenvectorseigenvalues}

\en{If}\ru{Если}
\en{for}\ru{для}
\en{some tensor}\ru{какого-нибудь тензора}~${^2\!\bm{B}}$
\en{and}\ru{и}~\en{the~nonzero vector}\ru{ненулевого вектора}~${\bm{a}}$

\nopagebreak\vspace{-0.2em}\begin{equation}\label{eigenvalues equation}
^2\!\bm{B} \dotp \bm{a} = \eigenvalue \bm{a}
\hspace{.1ex} ,
\:\:
\bm{a} \neq \bm{0}
\end{equation}
\vspace{-1.33em}\begin{equation*}
\scalebox{.92}{$
{^2\!\bm{B}} \dotp \bm{a} = \eigenvalue \UnitDyad \dotp \bm{a}
\hspace{.1ex} ,
\;\;
%%B_{i\hspace{-0.1ex}j} a_j = \eigenvalue \delta_{i\hspace{-0.1ex}j} a_j
%%\hspace{.1ex} ,
%%\;\;
\left( \hspace{.1ex}
{^2\!\bm{B} - \eigenvalue \UnitDyad}
\hspace{.2ex} \right)
\hspace{-0.1ex} \dotp \hspace{.1ex}
\bm{a}
= \bm{0}
$}
\hspace{.1ex} ,
\end{equation*}

\vspace{-0.6em}\noindent
\en{then}\ru{то}
$\eigenvalue$
\en{is called}\ru{называется}
\en{the eigenvalue}\ru{собственным числом}
(\en{or}\ru{или}
\en{the characteristic value}\ru{характеристическим значением, eigenvalue})
\en{of~tensor}\ru{тензора}~${^2\!\bm{B}}$,
\en{and}\ru{а}~\en{the axis~(direction)}\ru{ось~(направление)}
\en{of~eigenvector}\ru{собственного вектора}~$\bm{a}$
\en{is called}\ru{называется}
\en{its}\ru{его}
\ru{характеристической}\en{characteristic}
\en{axis}\ru{осью}
(\en{direction}\ru{направлением}).

\en{In components}\ru{В~компонентах}\en{,}
\en{this is}\ru{это}
\en{an~eigenvalue problem}\ru{задача на~собственные значения}
\en{for a~matrix}\ru{для матрицы}.
\en{A~homogeneous}\ru{Однородная} \en{system}\ru{система}
\en{of linear equations}\ru{линейных уравнений}
${ ( B_{i\hspace{-0.1ex}j} \hspace{-0.15ex}
- \eigenvalue \delta_{i\hspace{-0.1ex}j} )
\hspace{.25ex} a_j \hspace{-0.1ex}
= 0 }$
\en{has}\ru{имеет}
\en{a~non-zero solution}\ru{ненулевое решение}\ru{,}
\en{if}\ru{если}
\en{the determinant}\ru{определитель}
\en{of a~matrix of~components}\ru{матрицы компонент}
\begin{equation*}
\smash{
   \underset{
      \raisebox{.15em}{
         \scalebox{.7}{ i, \hspace{.15ex}j }
      } %end of raisebox
   } %end of undeset
\operatorname{det}
} %end of smash
\hspace{.3ex}
( B_{i\hspace{-0.1ex}j} \hspace{-0.15ex} - \eigenvalue \delta_{ i \hspace{-0.1ex} j } )
\end{equation*}
\en{is equal to zero}\ru{равен нулю}:

\nopagebreak\vspace{-0.1em}
\begin{equation}\label{chardetequation}
\operatorname{det} \hspace{-0.4ex}
\scalebox{.9}[.92]{$
   \left[
      \begin{array}{ccc}
         B_{1\hspace{-0.1ex}1} \hspace{-0.16ex} - \eigenvalue & B_{12} & B_{13} \\
         B_{21} & B_{22} \hspace{-0.16ex} - \eigenvalue & B_{23} \\
         B_{31} & B_{32} & B_{33} \hspace{-0.16ex} - \eigenvalue
      \end{array}
   \hspace{.1ex}
   \right]
$} %end of scalebox
\hspace{-0.4ex} = - \hspace{.2ex}
\eigenvalue^3 \hspace{-0.25ex}
+ \firstcharacteristicinvariant \hspace{.25ex}
\eigenvalue^2 \hspace{-0.25ex}
- \secondcharacteristicinvariant \hspace{.25ex}
\eigenvalue
+ \thirdcharacteristicinvariant
= 0
\hspace{.2ex} ;
\end{equation}

\vspace{-0.25em}
\begin{equation}\label{invariants:2}
\begin{array}{r@{\hspace{.4em}}c@{\hspace{.4em}}l}
\firstcharacteristicinvariant
& = &
\trace{^2\!\bm{B}} = B_{kk}
= \scalebox{.92}[.96]{$ B_{1\hspace{-0.1ex}1}
\hspace{-0.2ex} + \hspace{-0.1ex}
B_{22}
\hspace{-0.2ex} + \hspace{-0.1ex}
B_{33} $}
\hspace{.1ex} ,
\\[.1em]
%
\secondcharacteristicinvariant
& = &
\scalebox{.92}[.96]{$ B_{1\hspace{-0.1ex}1}B_{22}
\hspace{-0.2ex} - \hspace{-0.1ex}
B_{12}B_{21}
\hspace{-0.2ex} + \hspace{-0.1ex}
B_{1\hspace{-0.1ex}1}B_{33}
\hspace{-0.2ex} - \hspace{-0.1ex}
B_{13}B_{31}
\hspace{-0.2ex} + \hspace{-0.1ex}
B_{22}B_{33}
\hspace{-0.2ex} - \hspace{-0.1ex}
B_{23}B_{32} $}
\hspace{.1ex} ,
\\[.1em]
%
\thirdcharacteristicinvariant
& = &
\operatorname{det} \hspace{.1ex} {^2\!\bm{B}}
= \hspace{.15ex}
\underset{\raisebox{.15em}{%
   \scalebox{.7}{ $i$,$\hspace{.15ex}j$ }%
}}%
{\operatorname{det}} \, B_{i\hspace{-0.1ex}j} \hspace{-0.15ex}
= e_{i\hspace{-0.1ex}j\hspace{-0.1ex}k} \hspace{.1ex} B_{1i} B_{2j} B_{3k} \hspace{-0.1ex}
= e_{i\hspace{-0.1ex}j\hspace{-0.1ex}k} \hspace{.1ex} B_{i1} B_{\hspace{-0.1ex}j2} B_{k3}
\hspace{.1ex} .
\end{array}\end{equation}

\vspace{-0.4em}
\en{The roots}\ru{Корни}
\en{of the characteristic equation}\ru{характеристического уравнения}~\eqref{chardetequation}\:--- \en{the eigenvalues}\ru{собственные числа}\:---
\en{don’t depend}\ru{не~зависят}
\en{on the basis}\ru{от~базиса}
\en{and therefore}\ru{и~потому}
\en{are invariants}\ru{инварианты}.

\en{The coefficients}\ru{Коэффициенты}~\en{of~}\eqref{invariants:2}
\en{also}\ru{тоже}
\en{don’t depend}\ru{не~зависят}
\en{on the basis}\ru{от~базиса};
\en{they}\ru{они}
\en{are called}\ru{называются}
\en{the first}\ru{первым},
\en{the second}\ru{вторым}
\en{and}\ru{и}
\en{the third}\ru{третьим}
\en{characteristic}\ru{характеристическими}
\en{invariants}\ru{инвариантами}
\en{of a~tensor}\ru{тензора}.
\en{The first invariant}\ru{Первый инвариант}~$\firstcharacteristicinvariant$
\en{is the trace}\ru{это след}.
\en{It was described}\ru{Он был описан}
\en{earlier}\ru{ранее}
\en{in}\ru{в}~\pararef{para:operationswithtensors}.
\en{The second invariant}\ru{Второй инвариант}~$\secondcharacteristicinvariant$
\en{is}\ru{это}
\en{the trace}\ru{след}
\en{of~the~adjugate}\ru{присоединённой~(взаимной, adjugate)}
\en{matrix}\ru{матрицы}\:---
\en{the transpose}\ru{транспонированной}
\en{of the cofactor matrix}
\en(of the matrix of algebraic complements)}\ru{матрицы алгебраических дополнений}
\begin{equation*}
\secondcharacteristicinvariant\hspace{.16ex}( \hspace{.1ex}{^2\!\bm{B}} )
\hspace{-0.12ex} \equiv \hspace{.1ex}
\trace{\hspace{-0.33ex} \left(
   \operatorname{adj}{ B_{i\hspace{-0.1ex}j} }
}
\right) }} %end of trace
\end{equation*}
(\en{it’s hard, yeah}\ru{тяжеловато, да}).
\en{Or}\ru{Или}

\nopagebreak\vspace{-0.2em}\begin{equation*}
\secondcharacteristicinvariant\hspace{.16ex}(\hspace{.1ex}{^2\!\bm{B}}) \hspace{-0.2ex}
\equiv \hspace{.12ex} \smalldisplaystyleonehalf \scalebox{.84}{$\left[
\left( {^2\!\bm{B}} \tracedot\hspace{.1ex} \right)^{\!2} \hspace{-0.5ex}
- {^2\!\bm{B}} \hspace{-0.2ex} \dotdotp \hspace{-0.25ex} {^2\!\bm{B}} \hspace{.16ex}
\right]$} \hspace{-0.5ex}
= \smalldisplaystyleonehalf \scalebox{.84}{$\left[
\left( B_{kk} \right)^{\hspace{-0.1ex}2} \hspace{-0.4ex}
- \hspace{-0.2ex} B_{i\hspace{-0.1ex}j} B_{\hspace{-0.1ex}j\hspace{-0.06ex}i} \hspace{.1ex}
\right]$}
\hspace{-0.25ex} .
\end{equation*}

\vspace{-0.2em}\noindent
\en{And}\ru{И}~\en{the third invariant}\ru{третий инвариант}~$\thirdcharacteristicinvariant$
\en{is}\ru{это}
\en{the determinant}\ru{определитель~(детерминант)}
\en{of a~matrix of tensor components}\ru{матрицы компонент тензора}:
${ \thirdcharacteristicinvariant\hspace{.16ex}( \hspace{.1ex}{^2\!\bm{B}} ) \hspace{-0.1ex}
\equiv \operatorname{det} \hspace{.1ex} {^2\!\bm{B}}
}$.

\en{This}\ru{Это} \en{applies to all}\ru{относится ко~всем} \en{second complexity tensors}\ru{тензорам второй сложности}.
\en{Besides that}\ru{Кроме того}, \en{in case}\ru{в~случае} \en{of a~symmetric tensor}\ru{симметричного тензора}\en{,} \en{the following is true}\ru{истинно следующее}:\\
\indent 1$^{\circ}$\hspace{-1ex}.\, \en{The eigenvalues}\ru{Собственные значения} \en{of a~symmetric bivalent tensor}\ru{симметричного бивалентного тензора}\ru{\:---}\en{ are} \en{real numbers}\ru{действительные числа}.\\
\indent 2$^{\circ}$\hspace{-1ex}.\, \en{The characteristic axes~(directions)}\ru{Характеристические оси~(направления)} \en{for different eigenvalues}\ru{для различных собственных значений} \en{are orthogonal}\ru{ортогональны} \en{to each other}\ru{друг другу}.

\noindent
${ \tikz[baseline=-1ex] \draw [line width=.5pt, color=black, fill=white] (0, 0) circle (.8ex);
\hspace{.6em} }$
\en{The first statement}\ru{Первое утверждение} \en{is proved}\ru{доказывается} \en{by contradiction}\ru{от~противного}.
\en{If}\ru{Если}~$\eigenvalue$\ru{\:---}\en{ is} \en{a~complex root}\ru{компл\'{е}ксный корень}\en{ of}~\eqref{chardetequation}\ru{,} \en{corresponding to eigenvector}\ru{соответствующий собственному вектору}~$\bm{a}$, \en{then}\ru{то} \en{conjugate number}\ru{сопряжённое число}~$\lineover{\eigenvalue}$ \en{will also be}\ru{тоже будет} \en{the root}\ru{корнем}\en{ of}~\eqref{chardetequation}.
\ru{Соответствует ему }\en{Eigenvector}\ru{собственный вектор}~$\lineover{\bm{a}}$ \en{with }\ru{с~}\en{the conjugate}\ru{сопряжёнными} \en{components}\ru{компонентами}\en{ corresponds to it}.
\en{And then}\ru{И~тогда}

\nopagebreak\vspace{-0.5em}\begin{multline*}
\eqref{eigenvalues equation}
\:\,\Rightarrow\:\,
\left(\hspace{.25ex} {\lineover{\bm{a}} \, \dotp} \hspace{.25ex}\right)
\hspace{.5ex}
{^2\!\bm{B}} \dotp \bm{a} \hspace{.2ex} = \eigenvalue \bm{a},
\;\;
\left(\hspace{.2ex}{\bm{a} \, \dotp}\hspace{.2ex}\right)
\hspace{.5ex}
{^2\!\bm{B}} \dotp \lineover{\bm{a}} \hspace{.2ex} = \hspace{.1ex} \lineover{\eigenvalue} \, \lineover{\bm{a}}
\:\;\Rightarrow
\\[-0.1em]
%
\Rightarrow\;\:
\lineover{\bm{a}} \dotp {^2\!\bm{B}} \dotp \bm{a} - \bm{a} \dotp {^2\!\bm{B}} \dotp \lineover{\bm{a}} \hspace{.2ex} = \left(\hspace{.1ex}{\eigenvalue - \lineover{\eigenvalue}}\hspace{.2em}\right)\hspace{-0.1em} \bm{a} \dotp \lineover{\bm{a}}
\hspace{.2ex} .
\end{multline*}

\vspace{-0.2em}\noindent
\en{Here}\ru{Здесь} \en{on the left}\ru{слева}\en{ is}\ru{\:---} \en{zero}\ru{нуль}, \en{because}\ru{поскольку} ${\bm{a} \dotp {^2\!\bm{B}} \dotp \bm{c} = \bm{c} \dotp {^2\!\bm{B}^{\T}\!} \dotp \bm{a}}$ \en{and}\ru{и}~${{^2\!\bm{B}} = {^2\!\bm{B}^{\T}\!}}$.
\en{Thence}\ru{Поэтому} ${\eigenvalue = \lineover{\eigenvalue}}$, \en{that is}\ru{то~есть} \en{a~real number}\ru{действительное число}.

\en{Just as simple}\ru{Так~же просто} \en{looks}\ru{выглядит} \en{the proof}\ru{доказательство} \en{of~}2$^{\circ}$:

\nopagebreak\vspace{-0.5em}\begin{multline*}
\tikzmark{BeginEqualsZeroBrace} {\bm{a}_2 \dotp {^2\!\bm{B}} \dotp \bm{a}_1 - \bm{a}_1 \dotp {^2\!\bm{B}} \dotp \bm{a}_2} \hspace{.1ex} \tikzmark{EndEqualsZeroBrace}
= \hspace{.12ex} \left(\hspace{.1ex}{\eigenvalue_1 \hspace{-0.25ex} - \eigenvalue_2}\right) \bm{a}_1 \hspace{-0.1ex} \dotp \hspace{.1ex} \bm{a}_2 \hspace{.1ex} , \;
\eigenvalue_1 \hspace{-0.1ex} \neq \hspace{.1ex} \eigenvalue_2
\;\Rightarrow
\\
%
\Rightarrow\;
\bm{a}_1 \hspace{-0.1ex} \dotp \hspace{.1ex} \bm{a}_2 = 0
\hspace{.1ex} .
\hspace{4em} \tikz[baseline=-0.6ex] \draw [color=black, fill=black] (0, 0) circle (.8ex);
\end{multline*}
\AddUnderBrace[line width=.75pt][0.1ex,-0.1ex]%
{BeginEqualsZeroBrace}{EndEqualsZeroBrace}{${\scriptstyle =\;0}$}

\vspace{-1.4em}
\en{If}\ru{Если} \en{the roots}\ru{корни} \en{of the characteristic equation}\ru{характеристического уравнения} (\en{the eigenvalues}\ru{собственные числа}) \en{are different}\ru{различны}, \en{then}\ru{то} \en{one unit long eigenvectors}\ru{собственные векторы единичной длины}~${\mathboldae_i}$ \en{compose}\ru{составляют} \en{an~orthonormal basis}\ru{ортонормальный базис}.
\en{What are}\ru{Каков\'{ы}~же} \en{tensor components}\ru{компоненты тензора} \en{in such a~basis}\ru{в~таком базисе}?

\nopagebreak\vspace{-0.25em}\begin{equation*}\begin{array}{c}
{^2\!\bm{B}} \dotp \mathboldae_k = \hspace{.2ex} \scalebox{0.82}{$\tikzcancel[blue]{$\displaystyle\sum_k^{~}$}$}\: \eigenvalue_{\hspace{-0.15ex}k} \hspace{.1ex} \mathboldae_k \hspace{.1ex}, \:\: k =\hspace{-0.1ex} 1, 2, 3
\\[.9em]
%
{^2\!\bm{B}} \dotp \tikzmark{BeginEqualsEBrace} {\mathboldae_k \mathboldae_k} \tikzmark{EndEqualsEBrace} = \hspace{-0.1ex} \scalebox{0.9}{$\displaystyle \sum_{\smash{k}}$} \hspace{.2ex} \eigenvalue_{\hspace{-0.15ex}k} \hspace{.1ex} \mathboldae_k \mathboldae_k
%%{^2\!\bm{B}} = \displaystyle \sum_{\smash{k}} \eigenvalue_k \mathboldae_k \mathboldae_k
\end{array}\end{equation*}
\AddUnderBrace[line width=.75pt]%
{BeginEqualsEBrace}{EndEqualsEBrace}{${\scriptstyle \UnitDyad}$}

\vspace{-0.25em}
\en{In a~common case}\ru{В~общем случае} ${B_{i\hspace{-0.1ex}j} \hspace{-0.2ex} = \bm{e}_i \dotp {^2\!\bm{B}} \dotp \bm{e}_{\hspace{-0.1ex}j}}$.
\en{In the basis}\ru{В~базисе} ${\mathboldae_1}$, ${\mathboldae_2}$, ${\mathboldae_3}$ \en{of mutually perpendicular}\ru{взаимно перпендикулярных} \en{one unit long}\ru{единичной длины} ${\mathboldae_i \dotp \mathboldae_{\hspace{-0.1ex}j} \hspace{-0.15ex} = \delta_{i\hspace{-0.1ex}j}}$ \en{eigenvectors}\ru{собственных векторов} \en{of a~symmetric tensor}\ru{симметричного тензора}

\nopagebreak\vspace{-0.2em}\begin{equation*}\begin{array}{c}
B_{1\hspace{-0.1ex}1} \hspace{-0.12ex} = \mathboldae_1 \dotp \left({\eigenvalue_1 \mathboldae_1 \mathboldae_1 + \eigenvalue_2 \mathboldae_2 \mathboldae_2 + \eigenvalue_3 \mathboldae_3 \mathboldae_3}\right) \dotp \mathboldae_1 = \eigenvalue_1
\hspace{.1ex} ,
\\[.1em]
B_{12} \hspace{-0.12ex} = \mathboldae_1 \dotp \left({\eigenvalue_1 \mathboldae_1 \mathboldae_1 + \eigenvalue_2 \mathboldae_2 \mathboldae_2 + \eigenvalue_3 \mathboldae_3 \mathboldae_3}\right) \dotp \mathboldae_2 = 0
\hspace{.1ex} ,
\\[-0.2em]
\ldots
\end{array}\end{equation*}

\vspace{-0.3em}\noindent
\en{The~matrix of components}\ru{Матрица компонент}
\en{is diagonal}\ru{диагональна}
\en{and}\ru{и}~${ {^2\!\bm{B}} = \sum \hspace{-0.2ex}
\eigenvalue_{\hspace{-0.15ex}i} \hspace{.1ex} \mathboldae_i \mathboldae_i }$.

\en{Here}\ru{Здесь}
\en{goes}\ru{идёт}
\en{a~summation}\ru{суммирование}
\en{over the three repeating indices}\ru{по~трём повторяющимся индексам},
\en{because}\ru{ведь}
\ru{используется }\en{the special basis}\ru{особенный базис}\en{ is used}.

\en{The case of multiplicity}\ru{Случай кратности}
\en{of the eigenvalues}\ru{собственных значений}
\en{is considered}\ru{рассматривается}
\en{in the limit}\ru{в~пределе}.

\en{If simpler}\ru{Если попроще}
${\eigenvalue_2 \hspace{-0.16ex} \to \eigenvalue_1}$,
\en{then}\ru{то}
\en{any linear combination}\ru{любая линейная комбинация}
\en{of vectors}\ru{векторов}
${\bm{a}_1}$ \en{and}\ru{и} ${\bm{a}_2}$
\en{in the limit}\ru{в~пределе}
\en{satisfies the equation}\ru{удовлетворяет уравнению}
\eqref{eigenvalues equation}.

\en{Then}\ru{Тогда}
\en{any}\ru{любая}
\en{axis}\ru{ось}
\en{in the plane}\ru{в~плоскости}
(${\bm{a}_1, \bm{a}_2}$)
\en{becomes characteristic}\ru{становится собственной}.

\en{When}\ru{Когда~же}
\en{the three eigenvalues}\ru{все три собственных числ\'{а}}
\en{coincide}\ru{совпадают},
\en{then}\ru{то}
\en{any axis}\ru{любая ось}
\en{in the space}\ru{в~пространстве}
\en{is characteristic}\ru{собственная}.

\en{Then}\ru{Тогда}
${{^2\!\bm{B}} = \eigenvalue \UnitDyad}$,
\en{such tensors}\ru{такие тензоры}
\en{are called}\ru{называются}
\en{isotropic}\ru{изотропными}
\en{or}\ru{или}
\inquotesx{\en{spherical}\ru{шаровыми}}.

