\en{\section{Equations in displacements}}

\ru{\section{Уравнения в смещениях}}

\label{para:equationsindisplacements.linearelasticity}
 
\en{The~complete set of equations}\ru{Полный набор уравнений}~\eqref{lineartheory:wholesetofequations}
\en{contains unknowns}\ru{содержит неизвестные}
$\linearstress$, $\infinitesimaldeformation$ \en{and}\ru{и}~$\bm{u}$.
\en{Excluding}\ru{Исключая}~$\linearstress$
\en{and}\ru{и}~$\infinitesimaldeformation$,
\en{we come}\ru{приходим}
\en{to the formulation}\ru{к~формулировке}
\en{in displacements}\ru{в~смещениях}
(\en{the symmetrization of}\ru{симметризация}~${\hspace{-0.2ex} \boldnabla \bm{u}}$
\en{is redundant}\ru{лишняя}
\en{due to the following symmetry}\ru{из-за следующей симметрии} 

\noindent
\begin{equation}\label{thesymmetryofthestiffnesstensor}
\stiffnesstensor_{
   \hspace{.12ex}
   3 \scalebox{.6}[.8]{$\rightleftarrows$} 4
}
\hspace{-0.25ex} =
\stiffnesstensor
\end{equation}
).

\nopagebreak\vspace{-0.1em}
\begin{equation}\label{lineartheory:equationsindisplacements}
\begin{array}{c}
\boldnabla \dotp \left( \stiffnesstensor \dotdotp \hspace{-0.12ex} \boldnabla \bm{u} \hspace{.1ex} \right) + \hspace{.1ex} \bm{f} = \hspace{.1ex} \bm{0}
\hspace{.16ex} , \\[.4em]
%
\bm{u} \hspace{.1ex} \bigr|_{o_1} \hspace{-0.64ex} = \hspace{.2ex} \bm{u}_{\raisemath{-0.1em}{0}}
\hspace{.16ex} , \:\:
\unitnormalvector \dotp %%\tikzmark{TauTensorBegin}
\stiffnesstensor \hspace{-0.08ex} \dotdotp \hspace{-0.24ex} \boldnabla \bm{u}
%%\tikzmark{TauTensorEnd}
\hspace{.25ex} \bigr|_{o_2} \hspace{-0.64ex} = \hspace{.2ex} \bm{p}
\hspace{.16ex} .
\end{array}
\end{equation}%
%%\AddOverBrace[line width=.75pt][-0.1ex,0.1em]%
%%{TauTensorBegin}{TauTensorEnd}{${\scriptstyle \linearstress}$}

%% \footnote{По\hbox{-}прежнему под~$\bm{f}$ подразумеваем сумму обычной силы и~даламберовой силы инерции~${(- \rho \mathdotdotabove{\bm{u}}\hspace{.25ex})}$.}

\en{In an isotropic medium}\ru{В~изотропной среде} \eqref{lineartheory:equationsindisplacements} \en{takes the form}\ru{принимает вид}

...

\begin{otherlanguage}{russian}

Общее решение однородного уравнения (...) нашёл \href{https://de.wikipedia.org/wiki/Heinz_Neuber}{Heinz Neuber}

\href{https://ru.wikipedia.org/wiki/%D0%9F%D0%B0%D0%BF%D0%BA%D0%BE%D0%B2%D0%B8%D1%87,_%D0%9F%D1%91%D1%82%D1%80_%D0%A4%D1%91%D0%B4%D0%BE%D1%80%D0%BE%D0%B2%D0%B8%D1%87}{П.\,Ф.\:Папкович}

\end{otherlanguage}

...
