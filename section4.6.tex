\en{\section{Equations in displacements}}

\ru{\section{Уравнения в смещениях}}

\label{section:equationsindisplacements.linearelasticity}
 
\en{The~complete set of equations}\ru{Полный набор уравнений}~\eqref{lineartheory:wholesetofequations}
\en{contains unknowns}\ru{содержит неизвестные}
$\linearstress$, $\infinitesimaldeformation$ \en{and}\ru{и}~$\bm{u}$.
\en{Excluding}\ru{Исключая}~$\linearstress$
\en{and}\ru{и}~$\infinitesimaldeformation$,
\en{we get}\ru{мы получаем}
\en{the~formulation}\ru{формулировку}
\en{in displacements}\ru{в~смещениях}
(\en{symmetrization of}\ru{симметризация}~${\hspace{-0.2ex} \boldnabla \bm{u}}$
\en{is redundant}\ru{лишняя}
\en{due to}\ru{из\hbox{-}за}
\en{the~symmetry}\ru{симметрии} ${%
\stiffnesstensor_{\hspace{.12ex} \indexjuggling{3}{4}} \hspace{-0.25ex} = \stiffnesstensor%
}$).

\nopagebreak\vspace{-0.1em}
\begin{equation}\label{lineartheory:equationsindisplacements}
\begin{array}{c}
\boldnabla \dotp \left( \stiffnesstensor \dotdotp \hspace{-0.12ex} \boldnabla \bm{u} \hspace{.1ex} \right) + \hspace{.1ex} \volumeloadvector = \hspace{.1ex} \zerovector
\hspace{.16ex} , \\[.4em]
%
\bm{u} \hspace{.1ex} \bigr|_{o_1} \hspace{-0.64ex} = \hspace{.2ex} \bm{u}_{\raisemath{-0.1em}{0}}
\hspace{.16ex} , \:\:
\unitnormalvector \dotp %%\tikzmark{TauTensorBegin}
\stiffnesstensor \hspace{-0.08ex} \dotdotp \hspace{-0.24ex} \boldnabla \bm{u}
%%\tikzmark{TauTensorEnd}
\hspace{.25ex} \bigr|_{o_2} \hspace{-0.64ex} = \hspace{.2ex} \bm{p}
\hspace{.16ex} .
\end{array}
\end{equation}%
%%\AddOverBrace[line width=.75pt][-0.1ex,0.1em]%
%%{TauTensorBegin}{TauTensorEnd}{${\scriptstyle \linearstress}$}

%% \footnote{По\hbox{-}прежнему под~$\volumeloadvector$ подразумеваем сумму активной силы и~даламберовой силы инерции~${(- \rho \mathdotdotabove{\bm{u}}\hspace{.25ex})}$.}

\en{For}\ru{Для}
\en{an~isotropic medium}\ru{изотропной среды}
\eqref{lineartheory:equationsindisplacements}
\en{becomes}\ru{становится}

....

\en{The~solution}\ru{Решение}
\en{for the~homogeneous part}\ru{для однородной части}\footnote{%
\en{A~homogeneous}\ru{Однородное}
\en{differential}\ru{дифференциальное}
\en{equation}\ru{уравнение}
\en{contains}\ru{содержит}
\en{a~differentiation}\ru{дифференцирование}
\en{and a~homogeneous function}\ru{и~однородную функцию}
\en{with a~set of variables}\ru{с~набором переменных}.%
}
\en{of~equation}\ru{уравнения}~(...)
\en{was found by}\ru{нашли}

\noindent
\href{https://ru.wikipedia.org/wiki/%D0%9F%D0%B0%D0%BF%D0%BA%D0%BE%D0%B2%D0%B8%D1%87,_%D0%9F%D1%91%D1%82%D1%80_%D0%A4%D1%91%D0%B4%D0%BE%D1%80%D0%BE%D0%B2%D0%B8%D1%87}{\russianlanguage{Пётр Ф.\:Папкович}
(\href{https://de.wikipedia.org/wiki/Pjotr_Fjodorowitsch_Papkowitsch}{Pjotr F.\:Papkowitsch})}
Papkovich

\bookauthor{\russianlanguage{Пётр Ф.\:Папкович}}.
\href{https://www.mathnet.ru/rus/im5172}{\russianlanguage{Выражение общего интеграла основных уравнений теории упругости через гармонические функции}~//~\russianlanguage{Известия Академии наук СССР}. \russianlanguage{Отделение математических и естественных наук}. 1932, \russianlanguage{выпуск} 10, \russianlanguage{страницы} 1425\hbox{--}1435}.

\noindent
\en{and}\ru{и}
\href{https://de.wikipedia.org/wiki/Heinz_Neuber}{Heinz Neuber}

\bookauthor{Heinz Neuber}.
Ein neuer Ansatz zur Lösung räumlicher Probleme der Elastizitätstheorie. Der Hohlkegel unter Einzellast als Beispiel~//~Zeitschrift für Angewandte Mathematik und Mechanik (ZAMM), 1934, Band~14, Nr.\:4, Seiten 203\hbox{--}212.

\russianlanguage{Пётр Ф.\:Папкович}\en{~(Pjotr F.\:Papkowitsch)}
\en{in}\ru{в}~1932
\en{and}\ru{и}
Heinz Neuber
\en{in}\ru{в}~1934
\en{proposed}\ru{предложили}
\en{to~represent}\ru{представлять}
\en{displacements}\ru{перемещения}
\en{as}\ru{как}
\en{harmonic functions}\ru{гармонические функции}
\en{and}\ru{и},
\en{therefore}\ru{следовательно},
\en{to make use of}\ru{воспользоваться}
\en{a~wide catalogue}\ru{обширным каталогом}
\en{of particular solutions}\ru{частных решений}
\en{to the}\ru{уравнения} Laplace’\en{s}\ru{а}\en{ equation}.
%
\en{And sometimes}\ru{А~иногда}
\en{the~problem of elasticity}\ru{задача упругости}
\en{can be reduced}\ru{может быть сведена},
\en{at~least}\ru{хотя~бы}
\en{partially}\ru{частично},
\en{to one of}\ru{к~одной из}
\en{the classical problems}\ru{классических задач}
\en{of the~theory of~harmonic functions}\ru{теории гармонических функций}
(\en{potential theory}\ru{теории потенциала}).

{
\begin{otherlanguage}{russian}
\fontsize{8}{9}\selectfont

Приведя
полную совокупность уравнений~\eqref{lineartheory:wholesetofequations}
к~дифференциальному уравнению второй степени
(в~компонентах\:--- к трём таким уравнениям)
\en{for}\ru{для}
\en{a~twice-differentiable}\ru{дважды дифференцируемой}
\en{function}\ru{функции}
(трёх функций в~компонентах),
описывающей перемещение точек тела,
Пётр Папкович и Hanz Neuber смогли записать общее решение,
однако,
с существенным ограничением класса объёмных сил\:---
рассматривались только потенциальные.
Классические силы механического характера
(силы тяжести, силы инерции при равномерном вращении тел, силы гравитационного взаимодействия)
потенциальны.

Для консервативных объёмных сил возможно аналитическое решение.
Подход Папковича\hbox{--}Neuber’а не~применим, если воздействия механической и иной физической природы не~потенциальны.

\end{otherlanguage}
}

.....

\href{https://www.sciencedirect.com/science/article/pii/S0020768314002340}{Three\hbox{-}dimensional elasticity based on quaternion\hbox{-}valued potentials}

....
