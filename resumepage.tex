\newcommand\resumename{\en{What is this book about?}\ru{О~чём эта книга?}}
\chapter*{\resumename}
%%\addcontentsline{toc}{chapter}{\resumename}

\en{ %%\begin{otherlanguage}{english}

\noindent
In this book,
all models of an~elastic continuum are presented: non\-linear and~\hbox{linear}, micropolar and classical momentless; three-di\-men\-sion\-al, two-di\-men\-sion\-al~(shells and~plates), one-di\-men\-sion\-al~(rods, including thin\hbox{-}walled ones).
Fundamentals of dynamics\:--- oscillations, waves and~stability\:--- are explained.
For thermo\-elasticity and~magneto\-elasticity, the~summary of~theories of~thermo\-dynamics and~electro\-dynamics is given.
Theories of~defects and fractures are described.
Approaches to~modeling inhomogeneous materials\:— composites\:— are shown.

\vspace{\baselineskip}

\noindent
The~book is written using compact and elegant direct indexless tensor notation, operating with invariant objects — tensors.
The mathematical apparatus for interpreting direct tensor relations is contained in the~first chapter.

}\ru{ %%\begin{otherlanguage}{russian}

\noindent
В~этой книге
представлены все модели упругого континуума: нелинейные и~линейные, микрополярные и~классические безмоментные; трёх\-мерные, дву\-мерные~(оболочки и~пластины), одно\-мерные~(стержни, в~том~числе тонко\-стен\-ные).
Изложены основы динамики\:--- колебания, волны и~устойчивость.
Для термо\-упругости и~магнито\-упругости даётся сводка теорий термо\-динамики и~электро\-динамики.
Описаны теории дефектов и~трещин.
Показаны подходы к~моделированию неоднородных материалов — композитов.

\vspace{\baselineskip}

\noindent
Книга написана с~использованием компактной и~элегантной прямой безиндексной тензорной записи, оперирующей с~инвариантными объектами — тензорами.
Математический аппарат для интерпретации прямых тензорных соотношений содержится в~первой главе.

}

\vspace{3\baselineskip}

\noindent
\scalebox{.9}{\href{https://github.com/VadiqueMe/PhysicsOfElasticContinua}{\textit{github.com/VadiqueMe/PhysicsOfElasticContinua}}}

\newpage

