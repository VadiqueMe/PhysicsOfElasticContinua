\section{\en{The integral theorems}\ru{Интегральные теоремы}}

\en{For vector fields}\ru{Для векторных полей}\en{,}
\ru{известны }\en{the~integral theorems}\ru{интегральные теоремы}\en{ are known}\:---
\en{the}\ru{теорема} Gauss’\ru{а}
\en{or}\ru{или} \en{Ostrogradsky’s (}\russianlanguage{Остроградского}\en{)}
\en{divergence theorem}\ru{о~дивергенции}
\en{and}\ru{и}
\ru{теорема }Stokes’\ru{а}
\en{circulation theorem}\ru{о~циркуляции}.

\noindent\leavevmode{\indent}{\small Gauss’ theorem (divergence theorem) enables an~integral taken over a~volume to be replaced by one taken over the closed surface bounding that volume, and vice versa.\par}

\noindent\leavevmode{\indent}{\small Stokes’ theorem enables an~integral taken around a closed curve to be replaced by one taken over \emph{any} surface bounded by that curve. Stokes’ theorem relates a~line integral around a closed path to a surface integral over what is called a~\emph{capping surface} of the path.\par}

\subsection*{\en{The Gauss’s divergence theorem}\ru{Теорема Gauss’а о~дивергенции}}

\href{https://en.wikipedia.org/wiki/Divergence_theorem}{(wikipedia) Divergence theorem}

\en{This theorem}\ru{Эта теорема}\en{ is}\ru{\:---}
\en{about how to replace}\ru{про то, как заменить}
\en{a~volume integral}\ru{объёмный интеграл}
\en{with a~surface one}\ru{поверхностным}
(\en{and vice versa}\ru{и~наоборот}).

\begin{otherlanguage}{russian}

В~этой
теореме
рассматривается
поток (ef)flux
вектора
через
ограничивающую
объём~$\mathcal{V}$
з\'{а}мкнутую поверхность~${\mathcal{O}(\boundary \mathcal{V})}$...

\nopagebreak\vspace{-0.2em}
\begin{equation}\label{thedivergencetheorem.Gauss}
\scalebox{.93}{$ \displaystyle \ointegral \displaylimits_{ \mathclap{\mathcal{O}(\boundary \mathcal{V})} } $} \hspace{.2ex}
\bm{n} \dotp \bm{a} \hspace{.4ex} d\mathcal{O}
\hspace{.1ex} = \hspace{-0.2ex}
\scalebox{.93}{$ \displaystyle \integral \displaylimits_{\mathcal{V}} $} \hspace{-0.2ex}
\boldnabla \hspace{-0.12ex} \dotp \bm{a} \hspace{.4ex} d \mathcal{V}
\hspace{-0.1ex} ,
\end{equation}

\vspace{-0.7em}\noindent
\en{the~}\dotproductinquotes\hbox{-}\en{product}\ru{произведение}
\en{always commutes}\ru{всегда коммутирует}

\nopagebreak\vspace{-0.3em}
\begin{equation*}
\scalebox{.93}{$ \displaystyle \ointegral \displaylimits_{ \mathclap{\mathcal{O}(\boundary \mathcal{V})} } $} \hspace{.2ex}
\bm{a} \dotp \bm{n} \hspace{.2ex} d\mathcal{O}
\hspace{.1ex} = \hspace{-0.2ex}
\scalebox{.93}{$ \displaystyle \integral \displaylimits_{\mathcal{V}} $} \hspace{-0.2ex}
\bm{a} \dotp \hspace{-0.2ex} \boldnabla \hspace{.1ex} d \mathcal{V}
\hspace{-0.1ex} ,
\end{equation*}

\vspace{-0.7em}\noindent
${\bm{n} \hspace{.2ex} d\mathcal{O} = d \bm{\mathcal{O}}}$

\nopagebreak\vspace{-1.5em}
\begin{equation*}
\scalebox{.93}{$ \displaystyle \ointegral \displaylimits_{ \mathclap{\mathcal{O}(\boundary \mathcal{V})} } $} \hspace{.2ex}
\bm{a} \dotp d \bm{\mathcal{O}}
\hspace{.1ex} = \hspace{-0.2ex}
\scalebox{.93}{$ \displaystyle \integral \displaylimits_{\mathcal{V}} $} \hspace{-0.2ex}
\boldnabla \hspace{-0.12ex} \dotp \bm{a} \hspace{.4ex} d \mathcal{V}
\hspace{-0.1ex} ,
\end{equation*}

\vspace{-0.8em}\noindent
$\bm{n}$\en{ is}\ru{\:---}
\en{the~unit vector}\ru{единичный вектор}
\en{of~outward normal}\ru{внешней нормали}
\en{to}\ru{к}~\en{surface}\ru{поверхности}~${\mathcal{O}(\boundary \mathcal{V})}$.

\en{Volume}\ru{Объём}~$\mathcal{V}$
нарезается тремя семействами координатных поверхностей на~множество бесконечно малых элементов. 
Поток через поверхность ${\mathcal{O}(\boundary \mathcal{V})}$ равен сумме потоков через края получившихся элементов.
В~бесконечной малости каждый такой элемент\:--- малюсенький локальный дифференциальный кубик~(параллелепипед).
....
\en{The~flux}\ru{Поток}
\en{of~vector}\ru{вектора}~$\bm{a}$
\en{through}\ru{через}
\en{the~faces}\ru{грани}
\en{of a~small cube}\ru{малого кубика} %%\en{of~volume}\ru{объёма}~${d \mathcal{V}}$
\en{is equal to}\ru{равен}
${\sum_{i = 1}^{6} \bm{n}_i \dotp \bm{a} \hspace{.2ex} \mathcal{O}_i}$,
а~поток
через объём~${d \mathcal{V}}$
этого малого кубика
равен
${\boldnabla \dotp \bm{a} \hspace{.32ex} d \mathcal{V}}$.

\en{A~similar interpretation}\ru{Похожая трактовка}
\en{of this theorem}\ru{этой теоремы}
\en{is given}\ru{даётся},
\en{for example}\ru{для примера},
\en{in}\ru{в~лекциях} Richard\ru{’а} Feynman’\en{s}\ru{а}\en{ lectures}~\cite{feynman-lecturesonphysics}.

\textcolor{magenta}{\emph{( рисунок с кубиками )}}

to dice\:--- нарез\'{а}ть кубиками

small cube, little cube

локально ортонормальные координаты
${\bm{\xi} = \xi_i \hspace{.2ex} \bm{n}_i \hspace{.1ex}}$,
${d\bm{\xi} = d \xi_i \hspace{.2ex} \bm{n}_i}$,
${\boldnabla = \bm{n}_i \partial_i}$

разложение вектора
${\bm{a} = a_i \bm{n}_i \hspace{.1ex}}$

\subsection*{\en{The Stokes’ circulation theorem}\ru{Теорема Stokes’а о~циркуляции}}

\href{https://en.wikipedia.org/wiki/Stokes%27_theorem}{(wikipedia) Stokes’ theorem}

\en{This theorem}\ru{Эта теорема}
\en{is formulated as}\ru{формулируется как}
\en{the~equality}\ru{равенство}

\nopagebreak\vspace{-0.2em}
\begin{equation}\label{thecirculationtheorem.Stokes}
\scalebox{.93}{$ \displaystyle \ointegral \displaylimits_{ \mathclap{\mathcal{C}(\boundary \mathcal{O})} } $} \hspace{.2ex}
\bm{a} \dotp d \bm{\mathcal{C}}
\hspace{.1ex} = \hspace{-0.2ex}
\scalebox{.93}{$ \displaystyle \integral \displaylimits_{\mathcal{O}} $} \hspace{-0.1ex}
\bm{n} \dotp \hspace{-0.15ex} \bigl( \boldnabla \hspace{-0.2ex} \times \hspace{-0.2ex} \bm{a} \bigr) \hspace{.2ex} d \mathcal{O}
\hspace{-0.1ex} .
\end{equation}

\nopagebreak\vspace{-1.2em}
\begin{equation*}
\scalebox{.93}{$ \displaystyle \ointegral \displaylimits_{ \mathclap{\mathcal{C}(\boundary \mathcal{O})} } $} \hspace{.2ex}
\bm{a} \dotp d \bm{\mathcal{C}}
\hspace{.1ex} = \hspace{-0.2ex}
\scalebox{.93}{$ \displaystyle \integral \displaylimits_{\mathcal{O}} $} \hspace{-0.3ex}
\bigl( \boldnabla \hspace{-0.2ex} \times \hspace{-0.2ex} \bm{a} \bigr) \hspace{-0.2ex} \dotp \hspace{.1ex} \bm{n} \hspace{.2ex} d \mathcal{O}
\hspace{-0.1ex} .
\end{equation*}

\nopagebreak\vspace{-1.6em}
\begin{equation*}
\scalebox{.93}{$ \displaystyle \ointegral \displaylimits_{ \mathclap{\mathcal{C}(\boundary \mathcal{O})} } $} \hspace{.2ex}
\bm{a} \dotp d \bm{\mathcal{C}}
\hspace{.1ex} = \hspace{-0.2ex}
\scalebox{.93}{$ \displaystyle \integral \displaylimits_{\mathcal{O}} $} \hspace{-0.3ex}
\bigl( \boldnabla \hspace{-0.2ex} \times \hspace{-0.2ex} \bm{a} \bigr) \hspace{-0.25ex} \dotp d \bm{\mathcal{O}}
\hspace{-0.1ex} .
\end{equation*}


...



\end{otherlanguage}

