\section{\en{The integral theorems}\ru{Интегральные теоремы}}

\en{For vector fields}\ru{Для векторных полей}\en{,}
\ru{известны }\en{the~integral theorems}\ru{интегральные теоремы}\en{ are known}\:---
\en{the}\ru{теорема} Gauss’\ru{а}
\en{or}\ru{или} \en{Ostrogradsky’s (}\russianlanguage{Остроградского}\en{)}
\en{divergence theorem}\ru{о~дивергенции}
\en{and}\ru{и}
\ru{теорема }Stokes’\ru{а}
\en{circulation theorem}\ru{о~циркуляции}.

\noindent\leavevmode{\indent}{\small Gauss’ theorem (divergence theorem) enables an~integral taken over a~volume to be replaced by one taken over the closed surface bounding that volume, and vice versa.\par}

\noindent\leavevmode{\indent}{\small Stokes’ theorem enables an~integral taken around a closed curve to be replaced by one taken over \emph{any} surface bounded by that curve. Stokes’ theorem relates a~line integral around a closed path to a surface integral over what is called a~\emph{capping surface} of the path.\par}

\subsection*{\en{The Gauss’s divergence theorem}\ru{Теорема Gauss’а о~дивергенции}}

\href{https://en.wikipedia.org/wiki/Divergence_theorem}{(wikipedia) Divergence theorem}

\en{This theorem}\ru{Эта теорема}\en{ is}\ru{\:---}
\en{about how to replace}\ru{про то, как заменить}
\en{a~volume integral}\ru{объёмный интеграл}
\en{with a~surface one}\ru{поверхностным}
(\en{and vice versa}\ru{и~наоборот}).

\begin{otherlanguage}{russian}

В~этой
теореме
рассматривается
поток (ef)flux
вектора
через
ограничивающую
объём~$V$
з\'{а}мкнутую поверхность~${\mathcal{O}(\boundary V)}$.
Единичный вектор
внешней нормали~$\bm{n}$
к~поверхности~${\mathcal{O}(\boundary V)}$

\nopagebreak\vspace{-0.1em}%
\begin{equation}
\ointegral \displaylimits_{ \mathclap{\mathcal{O}(\boundary V)} } \hspace{-0.1ex}
\bm{n} \dotp \bm{a} \hspace{.4ex} d\mathcal{O} \hspace{.12ex}
=
\integral \displaylimits_{V} \hspace{-0.3ex}
\boldnabla \hspace{-0.12ex} \dotp \bm{a} \hspace{.4ex} dV
\hspace{-0.25ex} .
\end{equation}

Объём~$V$ нарезается тремя семействами координатных поверхностей на~множество бесконечно малых элементов. 
Поток через поверхность ${\mathcal{O}(\boundary V)}$ равен сумме потоков через края получившихся элементов.
В~бесконечной малости каждый такой элемент\:--- маленький локальный дифференциальный кубик~(параллелепипед).
...
Поток вектора~$\bm{a}$
через грани малого кубика
с~объёмом~$dV$
равен
${\sum_{i = 1}^{6} \bm{n}_i \dotp \bm{a} \hspace{.2ex} \mathcal{O}_i}$,
а~поток
через сам этот объём
равен
${\boldnabla \dotp \bm{a} \hspace{.32ex} dV}$.

Похожая трактовка этой теоремы есть,
для примера,
\en{in}\ru{в~лекциях} Richard\ru{’а} Feynman’\en{s}\ru{а}\en{ lectures}~\cite{feynman-lecturesonphysics}.

\textcolor{magenta}{\emph{( рисунок с кубиками )}}

to dice\:--- нарез\'{а}ть кубиками

small cube, little cube

локально ортонормальные координаты
${\bm{\xi} = \xi_i \hspace{.2ex} \bm{n}_i \hspace{.1ex}}$,
${d\bm{\xi} = d \xi_i \hspace{.2ex} \bm{n}_i}$,
${\boldnabla = \bm{n}_i \partial_i}$

разложение вектора
${\bm{a} = a_i \bm{n}_i \hspace{.1ex}}$

\subsection*{Теорема Stokes’а о~циркуляции}

Эта теорема выражается равенством


...



\end{otherlanguage}

