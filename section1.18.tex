\en{\section{Spatial differentiation}}

\ru{\section{Пространственное дифференцирование}}

\label{section:spatialdifferentiationoftensorfields} <<<<<< rename: remove fields

\begin{changemargin}{\parindent}{\parindent}
\vspace{-0.1em}
\small
\flushright
\textit{\en{Tensor field}\ru{Тензорное поле}}\ru{\:---}\en{ is} \en{a~tensor}\ru{это тензор}\ru{,} \en{varying from~point to~point}\ru{меняющийся от~точки к~точке} (\en{variable in~space}\ru{переменный в~пространстве}, \en{coordinate dependent}\ru{зависящий от~координат}).

\par\vspace{.3em}
\end{changemargin}

\begin{otherlanguage}{russian}

\noindent
Пусть \en{at each point}\ru{в~каждой точке} \en{of some region}\ru{некоторой области} \en{of a~three-dimensional space}\ru{трёхмерного пространства} определена величина~$\varsigma$.
Тогда говорят, что есть тензорное поле~${\varsigma \!=\! \varsigma(\locationvector)}$, \en{where}\ru{где}~$\locationvector$\en{ is}\ru{\:---} \en{location vector}\ru{вектор положения}~(\en{radius vector}\ru{вектор\hbox{-}радиус}) \en{of~a~point}\ru{точки} \en{in~space}\ru{пространства}.

Величина~$\varsigma$ может~быть тензором любой сложности.
Пример скалярного поля\:--- поле температуры в~среде, векторного поля\:--- скорости частиц жидкости.

Концепт тензорного поля никак не~связан с~концептом поля с~операциями $+$ и~$*$ с~11~свойствами этих операций.

\end{otherlanguage}

% ~ ~ ~ ~ ~
\begin{wrapfigure}{R}{0.55\textwidth}
\makebox[0.55\textwidth][c]{%
\hspace{2em}
\begin{minipage}[t]{.55\textwidth}

\begin{tikzpicture}[scale=0.5]

%%\clip (-6, -6) rectangle + (12, 12) ; % crop it
\clip (0, 0) circle (6cm) ; % crop it

\tikzset{%
	tangent/.style={
		decoration={
			markings,% switch on markings
			mark=
			at position #1
			with
			{
				\def\numberoftangent{\pgfkeysvalueof{/pgf/decoration/mark info/sequence number}}
				\coordinate (tangent point-\numberoftangent) at (0, 0);
				\coordinate (tangent unit vector-\numberoftangent) at (1, 0);
				\coordinate (tangent orthogonal unit vector-\numberoftangent) at (0, 1);
			}
		},
		postaction=decorate
	},
	use tangent/.style={
		shift=(tangent point-#1),
		x=(tangent unit vector-#1),
		y=(tangent orthogonal unit vector-#1)
	},
	use tangent/.default=1
}

\tikzset{%
	show curve controls/.style={
		postaction={
			decoration={
				show path construction,
				curveto code={
					\fill [black, opacity=.5]
						(\tikzinputsegmentfirst) circle (.4ex)
						(\tikzinputsegmentlast) circle (.4ex) ;
					\draw [black, opacity=.5, line cap=round, dash pattern=on 0pt off 1.6\pgflinewidth]
						(\tikzinputsegmentfirst) -- (\tikzinputsegmentsupporta)
						(\tikzinputsegmentlast) -- (\tikzinputsegmentsupportb) ;
					\fill [magenta, opacity=.5, line cap=round, dash pattern=on 0pt off 1.6\pgflinewidth]
						(\tikzinputsegmentsupporta) circle [radius=.4ex]
						(\tikzinputsegmentsupportb) circle [radius=.4ex] ;
				}
			},
			decorate
}	}	}

%%\foreach \cycle in {0, 1, ..., 15}
%%	\draw [color=green]
%%		($ (0, 0) - (\cycle, 1.2*\cycle) $)
%%		parabola ($ (4, 3) + 0.5*(1.6*\cycle, \cycle) $);

\foreach \c in {-10, -9.5, ..., 10}
{
	\def\offset{0.2*\c, -0.1*\c}
	\pgfmathsetmacro\bottomoffsetx{-.24 * ( \c )}
	\pgfmathsetmacro\bottomoffsety{-.1 * abs( \c ) + .1 * ( \c )}
	\pgfmathsetmacro\bottomangle{12 - 1.2 * abs( \c )}
	\pgfmathsetmacro\bottomnudge{2}
	\pgfmathsetmacro\midoffsetx{-.1 * abs( \c )}
	\pgfmathsetmacro\midoffsety{.1 * abs( \c )}
	\pgfmathsetmacro\midangle{63 + 1.2 * abs( \c )}
	\pgfmathsetmacro\midnudge{4 + ( .1 * abs( \c ) )}
	\pgfmathsetmacro\topoffsetx{.32 * ( \c ) + 0 * abs( \c )}
	\pgfmathsetmacro\topoffsety{-.16 * ( \c ) + 0 * abs( \c )}
	\pgfmathsetmacro\topangle{166 + 1.6 * ( \c ) + 1.2 * abs( \c )}
	\pgfmathsetmacro\topnudge{5 + ( .25 * abs( \c ) )}
	\draw	[ line width=.4pt
		, color=blue!50
		%%, show curve controls
		]
		($ (-6, -4.5) + 5*(\offset) + (\bottomoffsetx, \bottomoffsety) $)
		.. controls ++(\bottomangle: \bottomnudge) and ++(\midangle: -\midnudge) ..
		($ 4*(\offset) + (\midoffsetx, \midoffsety) $)
		.. controls ++(\midangle: \midnudge) and ++(\topangle: \topnudge) ..
		($ (8, 6) + 2.5*(\offset) + (\topoffsetx, \topoffsety) $) ;
}

\foreach \c in {-10, -9.5, ..., 10}
{
	\def\offset{0.2*\c, 0.1*\c}
	\pgfmathsetmacro\leftoffsetx{- .1 * abs ( \c )}
	\pgfmathsetmacro\leftoffsety{.4 * ( \c )}
	\pgfmathsetmacro\leftangle{33 + .2 * abs( \c )}
	\pgfmathsetmacro\leftnudge{1.6 + .5 * abs( \c )}
	\pgfmathsetmacro\midoffsetx{-.2 * abs( \c )}
	\pgfmathsetmacro\midoffsety{.2 * abs( \c )}
	\pgfmathsetmacro\midangle{111 + 1.2 * abs( \c )}
	\pgfmathsetmacro\midnudge{5}
	\pgfmathsetmacro\rightoffsetx{.25 * abs( \c )}
	\pgfmathsetmacro\rightoffsety{.16 * ( \c )}
	\pgfmathsetmacro\rightangle{177 + 2 * ( \c )}
	\pgfmathsetmacro\rightnudge{abs( 2 - ( .5 * ( \c ) ) )}
	\draw	[ line width=.4pt
		, color=red!50
		%%, show curve controls
		]
		($ (-12, 5) + 2.5*(\offset) + (\leftoffsetx, \leftoffsety) $)
		.. controls ++(\leftangle: \leftnudge) and ++(\midangle: \midnudge) ..
		($ 5*(\offset) + (\midoffsetx, \midoffsety) $)
		.. controls ++(\midangle: -\midnudge) and ++(\rightangle: \rightnudge) ..
		($ (8, -5) + 4*(\offset) + (\rightoffsetx, \rightoffsety) $);
}

\foreach \c in {-10, -9.5, ..., 10}
{
	\def\offset{0*\c, 0.25*\c}
	\pgfmathsetmacro\midnudge{6 + .16 * ( \c )}
	\draw	[ line width=.4pt
		, color=green!50
		%%, show curve controls
		]
		($ (12, 10) + 4*(\offset) $)
		.. controls ++(88: -4) and ++(11: \midnudge) ..
		($ 4*(\offset) $)
		.. controls ++(11: -\midnudge) and ++(99: -4) ..
		($ (-12, 4) + 4*(\offset) $) ;
}

\draw	[ line width=.8pt
	, color=blue!50!black
	%%, show curve controls
	]
	(-6, -4.5)
	.. controls ++(12: 2) and ++(63: -4) ..
	(0, 0);

\draw	[ line width=.8pt
	, color=blue!50!black
	%%, show curve controls
	, tangent=0
	, tangent=0.4
	]
	(0, 0)
	.. controls ++(63: 4) and ++(166: 5) ..
	(8, 6) ;

\path [use tangent=1]
	(0, 0) -- (.4*4, 0)
	node [color=blue, pos=0.86, above left, shape=circle, fill=white, outer sep=4pt, inner sep=1pt]
		{$\bm{r}_3$} ;

\draw [line width=1.25pt, color=blue, use tangent=1, -{Latex[round, length=3.6mm, width=2.4mm]}]
	(0, 0) -- (.4*4, 0) ;

\path [use tangent=2]
	(0, 0) -- (0, -1)
	node [color=blue!50!black, pos=0.48, above, shape=circle, fill=white, outer sep=0pt, inner sep=0.25pt]
		{$q^{\hspace{.1ex}3}$} ;

%%\fill [fill=blue, use tangent=1] (0, 0) circle (1mm);

\draw	[ line width=.8pt
	, color=red!50!black
	%%, show curve controls
	]
	(-12, 5)
	.. controls ++(33: 1.6) and ++(111: 5) ..
	(0, 0);

\draw	[ line width=.8pt
	, color=red!50!black
	%%, show curve controls
	, tangent=0
	, tangent=0.5
	]
	(0, 0)
	.. controls ++(111: -5) and ++(177: 2) ..
	(8, -5) ;

\path [use tangent=1]
	(0, 0) -- (.4*5, 0)
	node [color=red, pos=0.86, below left, shape=circle, fill=white, outer sep=4pt, inner sep=1pt]
		{$\bm{r}_1$} ;

\draw [line width=1.25pt, color=red, use tangent=1, -{Latex[round, length=3.6mm, width=2.4mm]}]
	(0, 0) -- (.4*5, 0);

\path [use tangent=2]
	(0, 0) -- (0, 1)
	node [color=red!50!black, pos=0.16, above, shape=circle, fill=white, outer sep=0pt, inner sep=0.25pt]
		{$q^{1}$} ;

%%\fill [fill=red, use tangent=1] (0, 0) circle (1mm);

\draw	[ line width=.8pt
	, color=green!50!black
	%%, show curve controls
	]
	(12, 10)
	.. controls ++(88: -4) and ++(11: 6) ..
	(0, 0) ;

\draw	[ line width=.8pt
	, color=green!50!black
	%%, show curve controls
	, tangent=0
	, tangent=0.36
	]
	(0, 0)
	.. controls ++(11: -6) and ++(99: -4) ..
	(-12, 4) ;

\path [use tangent=1]
	(0, 0) -- (.4*6, 0)
	node [color=green, pos=0.92, below right, shape=circle, fill=white, outer sep=5pt, inner sep=1pt]
		{$\bm{r}_2$} ;

\draw [line width=1.25pt, color=green, use tangent=1, -{Latex[round, length=3.6mm, width=2.4mm]}]
	(0, 0) -- (.4*6, 0);

\path [use tangent=2]
	(0, 0) -- (0, -1)
	node [color=green!50!black, pos=0.12, above, shape=circle, fill=white, outer sep=0pt, inner sep=0.25pt]
		{$q^{\hspace{.1ex}2}$} ;

%%\fill [fill=green, use tangent=1] (0, 0) circle (1mm);

\coordinate (theOrigin) at (5, -2) ;
\path (0, 0) circle (1mm) node [shape=circle, inner sep=.5mm, outer sep=0] (theCircleOfO) {} ;

\draw [line width=1.5pt, black, -{Stealth[round,length=4mm,width=2.8mm]}] (theOrigin) -- (theCircleOfO)
		node [pos=0.64, above right, shape=circle, fill=white, outer sep=2pt, inner sep=1.2pt]
			{$\bm{r}$} ;

\draw [line width=1.2pt, color=black, fill=white] (0, 0) circle (1ex);

\draw [line width=1.2pt, color=black, fill=white] (theOrigin) circle (1ex);

\end{tikzpicture}

\vspace{0.1em}\caption{}\label{fig:curvilinearcoordinates}
\end{minipage}}
\end{wrapfigure}

% ~ ~ ~ ~ ~

\begin{otherlanguage}{russian}

Не~только для~решения прикладных задач, но нередко и в~\inquotes{чистой тео\-рии} вместо аргумента~$\locationvector$ ис\-поль\-зу\-ет\-ся набор (какая-либо трой\-ка) криво\-линей\-ных координат~${q^{\hspace{.1ex}i}\hspace{-0.2ex}}$.
Если непрерывно менять лишь одну координату из~трёх, получается координатная линия.
Каждая точка трёхмерного пространства лежит на~пересечении трёх координатных линий (\figureref{fig:curvilinearcoordinates}).
Вектор положения точки выражается через набор координат \en{as}\ru{как} \en{relation}\ru{отношение} ${\locationvector \hspace{-0.4ex} = \hspace{-0.4ex} \locationvector(q^{\hspace{.1ex}i}\hspace{.1ex})}$.

\end{otherlanguage}

Commonly used \en{sets of~coordinates}\ru{наборы координат}<<<<<<
\en{Rectangular}\ru{Прямоугольные} (\inquotes{\en{cartesian}\ru{декартовы}}), \en{spherical}\ru{сферические} \en{and }\ru{и~}\en{cylindrical}\ru{цилиндрические} \en{coordinates}\ru{координаты}\en{ are}\ru{\:---}

Curvilinear coordinates may be derived from a~set of~rectangular~(\inquotes{cartesian}) coordinates by using a~transformation that is locally invertible (a~one-to-one map) at~each point.
\en{Therefore}\ru{Поэтому} \en{rectangular coordinates}\ru{прямоугольные координаты} \en{of~any point}\ru{любой точки} \en{of~space}\ru{пространства} \en{can be converted}\ru{могут быть преобразованы} \en{to }\ru{в~}\en{some}\ru{какие-либо} \en{curvilinear coordinates}\ru{криволинейные координаты} \en{and}\ru{и}~\en{vice versa}\ru{обратно}.

...

The~differential of a~function presents a~change in the~linearization of this function.

...

\en{partial derivative}\ru{частная производная}

\nopagebreak\vspace{-0.4em}\begin{equation*}
\partial_i \equiv \scalebox{0.9}{$ \displaystyle\frac{\raisebox{-0.2em}{$\partial$}}{\raisebox{-0.1em}{$\partial q^i$}} $}
\end{equation*}

...

\en{differential}\ru{дифференциал} \en{of~}${\varsigma(q^i)}$

\nopagebreak\vspace{-0.4em}\begin{equation}
d\varsigma \hspace{-0.1ex}
= \scalebox{0.9}{$ \displaystyle\frac{\raisebox{-0.2em}{$\partial \hspace{.15ex} \varsigma$}}{\raisebox{-0.1em}{$\partial q^i$}} $} \hspace{.2ex} dq^i \hspace{-0.2ex}
= \partial_i \varsigma \hspace{.15ex} dq^i
\end{equation}

...

\en{Linearity}\ru{Линейность}

\nopagebreak\vspace{-0.4em}\begin{equation}\label{linearityordifferentiation}
\partial_i \bigl( \lambda \hspace{.1ex} p + \hspace{-0.2ex} \mu \hspace{.1ex} q \hspace{.1ex} \bigr) \hspace{-0.2ex}
= \lambda \bigl( \partial_i \hspace{.1ex} p \hspace{.1ex} \bigr) \hspace{-0.2ex} + \hspace{.1ex}
\mu \bigl( \partial_i \hspace{.1ex} q \hspace{.1ex} \bigr)
\end{equation}

\inquotes{Product rule}

\nopagebreak\vspace{-0.4em}\begin{equation}\label{productrulefordifferentiation}
\partial_i \bigl( \hspace{.15ex} p \circ q \hspace{.1ex} \bigr) \hspace{-0.2ex}
= \hspace{-0.2ex} \bigl( \partial_i \hspace{.1ex} p \hspace{.1ex} \bigr) \hspace{-0.25ex} \circ q \hspace{.12ex} +
\hspace{.1ex} p \circ \hspace{-0.25ex} \bigl( \partial_i \hspace{.1ex} q \hspace{.1ex} \bigr)
\end{equation}

...

Local basis ${\locationvector_\differentialindex{i}}$

\en{The~differential}\ru{Дифференциал} \en{of~location vector}\ru{вектора положения}~${\locationvector(q^{\hspace{.1ex}i}\hspace{.1ex})}$ \en{is}\ru{есть}

\nopagebreak\vspace{-0.2em}\begin{equation}\label{differentialoflocationvector}
d\locationvector
=
\scalebox{0.9}{$ \displaystyle\frac{\raisebox{-0.2em}{$ \partial \hspace{.15ex} \locationvector $}}{\partial q^{\hspace{.1ex}i}} $} \hspace{.2ex} dq^i \hspace{-0.1ex}
=
dq^i \locationvector_\differentialindex{i}
\hspace{.1ex} , \hspace{.5em}
\locationvector_\differentialindex{i} \hspace{-0.1ex} \equiv \scalebox{0.9}{$ \displaystyle\frac{\raisebox{-0.2em}{$ \partial \hspace{.15ex} \locationvector $}}{\partial q^{\hspace{.1ex}i}} $} \hspace{-0.1ex}
\equiv \partial_i \hspace{.1ex} \locationvector
\end{equation}

...

Local cobasis ${\locationvector^i}$, ${\locationvector^i \hspace{-0.32ex} \dotp \locationvector_\differentialindex{\hspace{-0.1ex}j} \hspace{-0.22ex} = \delta_{\hspace{-0.15ex}j}^{\hspace{.2ex}i}}$

...

\begin{equation*}
\displaystyle\frac{\raisebox{-0.2em}{$\partial \hspace{.15ex} \varsigma$}}{\partial \locationvector}
=
\displaystyle\frac{\raisebox{-0.2em}{$\partial \hspace{.15ex} \varsigma$}}{\raisebox{-0.1em}{$\partial q^i$}} \hspace{.1ex} \locationvector^i \hspace{-0.25ex}
=
\partial_i \varsigma \hspace{.2ex} \locationvector^i
\end{equation*}

\begin{equation}
d \varsigma \hspace{-0.1ex}
=
\scalebox{0.9}{$ \displaystyle\frac{\raisebox{-0.2em}{$\partial \hspace{.15ex} \varsigma$}}{\partial \locationvector} $} \dotp d\locationvector \hspace{-0.1ex}
=
\partial_i \varsigma \hspace{.2ex} \locationvector^i \hspace{-0.15ex} \dotp dq^{\hspace{.12ex}j} \hspace{-0.1ex} \locationvector_\differentialindex{\hspace{-0.1ex}j} \hspace{-0.2ex}
=
\partial_i \varsigma \hspace{.15ex} dq^i
\end{equation}

...

\en{The bivalent unit tensor}\ru{Бивалентный единичный тензор}~(\en{metric tensor}\ru{метрический тензор})~${\hspace{-0.1ex}\UnitDyad}$,
\en{which}\ru{который} \en{is neutral}\ru{нейтрален}~\eqref{definingpropertyofidentitytensor} \en{to the } \hbox{\hspace{-0.2ex}\inquotes{${\dotp\hspace{.22ex}}$}\hspace{-0.2ex}}-\en{product}\ru{произведению} (dot product\ru{’у}),
\en{can be represented as}\ru{может быть представлен как}

\nopagebreak\vspace{-0.1em}\begin{equation}
\UnitDyad
= \locationvector^i \locationvector_\differentialindex{i} \hspace{-0.15ex}
= \tikzmark{beginOriginOfNabla} \locationvector^i \partial_i \tikzmark{endOriginOfNabla} \hspace{.1ex} \locationvector = \hspace{-0.16ex} \boldnabla \locationvector ,
\end{equation}
\AddUnderBrace[line width=.75pt][0,-0.1ex]%
{beginOriginOfNabla}{endOriginOfNabla}%
{${\scriptstyle \boldnabla}$}

\vspace{-0.4em}\noindent
\en{where appears}\ru{где появляется} \en{the~}\en{differential}\ru{дифференциальный} \ru{оператор }\inquotes{\en{nabla}\ru{набла}}\en{ operator}

\nopagebreak\vspace{-0.2em}\begin{equation}
\boldnabla \equiv \locationvector^i \partial_i \hspace{.1ex} .
\end{equation}

...

\begin{equation}
d \varsigma \hspace{-0.1ex}
=
\scalebox{0.9}{$ \displaystyle\frac{\raisebox{-0.2em}{$\partial \hspace{.15ex} \varsigma$}}{\raisebox{-0.05em}{$\partial \locationvector$}} $} \dotp d\locationvector \hspace{-0.1ex}
=
d\locationvector \dotp \hspace{-0.11ex} \boldnabla \varsigma \hspace{-0.1ex}
=
\partial_i \varsigma \hspace{.15ex} dq^i
\end{equation}

\vspace{1.1em}${
d\locationvector = d\locationvector \dotp \hspace{-0.2ex} \tikzmark{beginItsUnitTensorE} \boldnabla \locationvector \tikzmark{endItsUnitTensorE}
}$%
\AddOverBrace[line width=.75pt][0,0.1ex]{beginItsUnitTensorE}{endItsUnitTensorE}{${\scriptstyle \UnitDyad}$}

...

\en{Divergence}\ru{Дивергенция} \en{of~the~dyadic product}\ru{диадного произведения} \en{of~two vectors}\ru{двух векторов}

\nopagebreak\vspace{-0.3em}\begin{multline}\label{divergenceofdyadicproducoftwovectors}
\boldnabla \hspace{-0.16ex} \dotp \hspace{-0.2ex} \bigl( \hspace{-0.1ex} \bm{a} \bm{b} \hspace{.05ex} \bigr) \hspace{-0.33ex}
= \locationvector^i \partial_i \hspace{-0.1ex} \dotp \hspace{-0.24ex} \bigl( \hspace{-0.1ex} \bm{a} \bm{b} \bigr) \hspace{-0.33ex}
= \locationvector^i \hspace{-0.3ex} \dotp \partial_i \bigl( \hspace{-0.1ex} \bm{a} \bm{b} \bigr) \hspace{-0.3ex}
= \locationvector^i \hspace{-0.3ex} \dotp \hspace{-0.15ex} \bigl( \partial_i \bm{a} \bigr) \bm{b} \hspace{.1ex} + \locationvector^i \hspace{-0.3ex} \dotp \bm{a} \hspace{.1ex} \bigl( \partial_i \bm{b} \bigr) \hspace{-0.33ex} =
\\[-0.1em]
%
= \hspace{-0.15ex} \bigl( \locationvector^i \hspace{-0.3ex} \dotp \partial_i \bm{a} \bigr) \bm{b} \hspace{.1ex} + \bm{a} \dotp \locationvector^i \hspace{-0.15ex} \bigl( \partial_i \bm{b} \bigr) \hspace{-0.33ex}
= \hspace{-0.15ex} \bigl( \locationvector^i \partial_i \hspace{-0.1ex} \dotp \bm{a} \bigr) \bm{b} \hspace{.1ex} + \bm{a} \dotp \hspace{-0.1ex} \bigl( \locationvector^i \partial_i \bm{b} \bigr) \hspace{-0.33ex} =
\\
%
= \hspace{-0.15ex} \bigl( \boldnabla \hspace{-0.15ex} \dotp \hspace{-0.1ex} \bm{a} \bigr) \bm{b} \hspace{.1ex} + \bm{a} \dotp \hspace{-0.12ex} \bigl( \boldnabla \hspace{.1ex} \bm{b} \bigr)
\end{multline}

\vspace{-0.2em}\noindent
--- \en{here’s no~need}\ru{тут нет нужды} \en{to~expand}\ru{разворачивать} \en{vectors}\ru{векторы}~$\bm{a}$ \en{and}\ru{и}~$\bm{b}$, \en{expanding just}\ru{развернув лишь} \en{differential operator}\ru{дифференциальный оператор}~${\hspace{-0.13ex}\boldnabla}$.

...

\en{Gradient of cross product of two vectors}\ru{Градиент векторного произведения двух векторов},
\en{applying}\ru{применяя} \inquotes{product rule}~\eqref{productrulefordifferentiation}
\en{and}\ru{и}~\en{relation}\ru{соотношение}~\eqref{crossproductoftwovectors} \en{for any two vectors}\ru{для любых двух векторов}
(\en{partial derivative}\ru{частная производная}~$\partial_i$ \en{of~some~vector by scalar coordinate}\ru{некоторого вектора по скалярной координате}~$q^i\hspace{-0.1ex}$ \en{is a~vector too}\ru{это тоже вектор})

\nopagebreak\vspace{-0.4em}\begin{multline}\label{gradientofcrossproductoftwovectors}
\boldnabla \hspace{-0.2ex} \left( \bm{a} \hspace{-0.1ex} \times \hspace{-0.1ex} \bm{b} \right) \hspace{-0.2ex}
= \hspace{.1ex} \locationvector^i \partial_i \hspace{-0.3ex} \left( \bm{a} \hspace{-0.2ex} \times \hspace{-0.2ex} \bm{b} \right) \hspace{-0.2ex}
= \locationvector^i \hspace{-0.4ex} \left( \partial_i \bm{a} \hspace{-0.2ex} \times \hspace{-0.2ex} \bm{b} \hspace{.1ex} +
\bm{a} \hspace{-0.2ex} \times \hspace{-0.2ex} \partial_i \bm{b} \right) \hspace{-0.2ex} =
\\[-0.1em]
%
= \locationvector^i \hspace{-0.4ex} \left( \partial_i \bm{a} \hspace{-0.2ex} \times \hspace{-0.2ex} \bm{b} \hspace{.1ex} -
\partial_i \bm{b} \hspace{-0.2ex} \times \hspace{-0.2ex} \bm{a} \right) \hspace{-0.2ex}
= \hspace{.1ex} \locationvector^i \partial_i \hspace{.1ex} \bm{a} \hspace{-0.2ex} \times \hspace{-0.2ex} \bm{b} \hspace{.1ex} - \hspace{.1ex}
\locationvector^i \partial_i \hspace{.1ex} \bm{b} \hspace{-0.2ex} \times \hspace{-0.2ex} \bm{a} =
\\[-0.1em]
%
= \hspace{-0.12ex} \boldnabla \bm{a} \hspace{-0.1ex} \times \hspace{-0.1ex} \bm{b} \hspace{.12ex} - \hspace{-0.12ex}
\boldnabla \hspace{.1ex} \bm{b} \hspace{-0.1ex} \times \hspace{-0.1ex} \bm{a}
\hspace{.2ex} .
\end{multline}

...

\en{Gradient}\ru{Градиент} \en{of }dot product\ru{’а} \en{of two vectors}\ru{двух векторов}

\nopagebreak\vspace{-0.4em}\begin{multline}\label{gradientofdotproductoftwovectors}
\boldnabla \hspace{.1ex} \bigl( \hspace{-0.05ex} \bm{a} \hspace{-0.1ex} \dotp \hspace{-0.1ex} \bm{b} \hspace{.05ex} \bigr) \hspace{-0.3ex}
= \hspace{.1ex} \locationvector^i \partial_i \bigl( \hspace{-0.05ex} \bm{a} \hspace{-0.1ex} \dotp \hspace{-0.1ex} \bm{b} \hspace{.05ex} \bigr) \hspace{-0.33ex}
= \hspace{.1ex} \locationvector^i \bigl( \partial_i \bm{a} \bigr) \hspace{-0.32ex} \dotp \bm{b} + \hspace{.1ex} \locationvector^i \bm{a} \hspace{-0.05ex} \dotp \hspace{-0.15ex} \bigl( \partial_i \bm{b} \bigr) \hspace{-0.33ex} =
\\[-0.1em]
%
= \hspace{-0.2ex} \bigl( \locationvector^i \partial_i \bm{a} \bigr) \hspace{-0.32ex} \dotp \bm{b} \hspace{.1ex} + \hspace{.1ex} \locationvector^i \bigl( \partial_i \bm{b} \bigr) \hspace{-0.33ex} \dotp \bm{a}
= \hspace{-0.16ex} \bigl( \boldnabla \hspace{-0.1ex} \bm{a} \bigr) \hspace{-0.3ex} \dotp \hspace{.1ex} \bm{b} \hspace{.1ex} + \hspace{-0.1ex} \bigl( \boldnabla \hspace{.1ex} \bm{b} \bigr) \hspace{-0.27ex} \dotp \hspace{.1ex} \bm{a}
\hspace{.2ex} .
\end{multline}

