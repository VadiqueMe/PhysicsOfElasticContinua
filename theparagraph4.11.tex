\en{\section{The principle of the minimum complementary energy}}

\ru{\section{Принцип минимума дополнительной энергии}}

\label{para:principleofminimumcomplementaryenergy}

\vspace{.2em}\begin{changemargin}{\parindent}{\parindent}
\small
\en{When}\ru{Когда}
\en{the~stress\hbox{--}strain relations}\ru{отношения напряжения\hbox{--}деформации}
(\ru{закон}\en{the} Hooke’\en{s}\ru{а}\en{ law})
assure the~existence of a~complementary energy function and the~geometrical boundary conditions are assumed constant during variation of~stresses, then the~principle of~minimum complementary energy emerges.
\par
\nopagebreak\vspace{.2em}
\end{changemargin}

\en{The complementary energy}\ru{Дополнительная энергия} \en{of a~linear-elastic body}\ru{линейно-упругого т\'{е}ла} \en{is}\ru{есть}
\en{the~following}\ru{следующий}
\en{functional}\ru{функционал}
\en{over the~field of~stresses}\ru{над~полем напряжений}:

\nopagebreak
\begin{equation}\label{thecomplementaryenergyfunctional}
\complementaryenergyfunctional (\hspace{-0.1ex}\linearstress\hspace{.1ex})
\equiv \hspace{-0.2ex}
\displaystyle \integral\displaylimits_{\mathcal{V}} \hspace{-0.5ex}
\widehat{\potential}(\hspace{-0.1ex}\linearstress\hspace{.1ex}) \hspace{.2ex} d\mathcal{V} \hspace{.1ex}
- \hspace{-0.2ex}
\displaystyle\integral\displaylimits_{o_1} \hspace{-0.5ex}
\unitnormalvector \hspace{.1ex} \dotp \linearstress \dotp \bm{u}_{\raisemath{-0.1em}{0}} \hspace{.25ex} do
\hspace{.2ex} ,
\hspace{.5em}
\bm{u}_{\raisemath{-0.1em}{0}} \hspace{-0.2ex} \equiv \bm{u} \hspace{.1ex} \bigr|_{o_1}
\hspace{-0.1ex} ,
\end{equation}
%
\nopagebreak\vspace{-0.4em}\begin{equation*}
\boldnabla \dotp \linearstress \hspace{.15ex} + \bm{f} = \hspace{.1ex} \bm{0} \hspace{.1ex} ,
\hspace{.6em}
\unitnormalvector \dotp \linearstress \hspace{.2ex} \bigr|_{o_2} \hspace{-0.66ex} = \hspace{.2ex} \bm{p}
\hspace{.2ex} .
\end{equation*}

...

\en{The variation}\ru{Вариация}
\en{of the balance of force equation}\ru{уравнения баланса сил}

\noindent
\begin{equation*}\label{the variation of the balance of force equation}
\variation{\hspace{.1ex} \bigl( \hspace{.1ex}
\boldnabla \dotp \linearstress
\hspace{.15ex} +
\bm{f}
\hspace{.15ex} \bigr)} \hspace{-0.2ex}
= \hspace{-0.1ex}
\boldnabla \dotp \variation{\linearstress} \hspace{.15ex}
= \hspace{.1ex} \bm{0}
\end{equation*}

...

\en{The~principle of the minimum complementary energy}\ru{Принцип минимума дополнительной энергии}
\en{is very useful}\ru{очень полезен}
\en{for estimating}\ru{для оценки} \en{inexact}\ru{неточных}~(\en{approximate}\ru{приближённых}) \en{solutions}\ru{решений}.
\en{But}\ru{Но}
\en{for computations}\ru{для вычислений}
\en{it isn’t so essential}\ru{он не~столь существенен,} \en{as}\ru{как}
\en{the}\ru{принцип}~(Lagrange\ru{’а})\en{ principle} \en{of~minimum potential energy}\ru{минимума потенциальной энергии}~\eqref{principleofminimumpotentialenergy.formulation}.

\en{To derive}\ru{Для вывода}
\en{the variational principles}\ru{вариационных принципов}
\en{it is natural}\ru{естественно}
\en{to use}\ru{использовать}
\en{the principle of the~virtual work}\ru{принцип виртуальной работы}
(\chapterdotsectionref{chapter:genericmechanics}{para:virtualworkprinciple.genericmechanics})
\en{as a~foundation}\ru{как фундамент}.
