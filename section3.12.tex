\en{\section{Variation of the present configuration}}

\ru{\section{Варьирование текущей конфигурации}}

\label{section:variationofconfiguration}

\en{Usually}\ru{Обыкновенно}
\ru{рассматриваются }\en{the two configurations}\ru{две конфигурации} \en{of a~nonlinear elastic medium}\ru{нелинейной упругой среды}\en{ are considered}\::
\en{the initial one}\ru{начальная}
\en{with location vectors}\ru{с~векторами положения}~$\initiallocationvector$
\en{and}\ru{и}
\en{the present (current) one}\ru{текущая (актуальная)}
\en{with}\ru{с}~$\currentlocationvector$.

\en{The following}\ru{Следующие}
\en{equations}\ru{уравнения}
\en{describe}\ru{описывают}
\en{a~small change}\ru{малое изменение}
\en{of the current configuration}\ru{текущей конфигурации}
\en{with infinitesimal changes}\ru{с~бесконечно-малыми изменениями}
\en{to the location vector}\ru{вектора положения}~$\variation{\currentlocationvector}$,
\en{to the vector of mass forces}\ru{вектора массовых сил}~${\variation{\hspace{-0.2ex}\massloadvector}\hspace{-0.2ex}}$,
\en{to the first}\ru{первого} \ru{тензора напряжения }Piola\hbox{--}Kirchhoff\ru{’а}\en{ stress tensor}~${\variation{\hspace{.1ex}\firstpiolakirchhoffstress}}$
\en{and}\ru{и}
\en{to the deformation tensor}\ru{тензора деформации}~${\variation{\hspace{.1ex}\bm{C}}}$.

\en{By varying}\ru{Варьируя}
\eqref{balanceoftranslationalmomentum.local.withfirstpiolakirchhoffstress}, (......)\footnote{%
${\boldnabla \hspace{-0.08ex}
= \hspace{-0.2ex} \boldnabla \dotp \hspace{-0.15ex} \boldnablacircled \initiallocationvector \hspace{-0.08ex}
= \currentlocationvector^{i} \hspace{-0.1ex} \partial_{i} \hspace{-0.15ex} \dotp \initiallocationvector^j \hspace{-0.1ex} \partial_{\hspace{-0.1ex}j} \initiallocationvector \hspace{-0.1ex}
\stackrel{?}{=} \currentlocationvector^{i} \hspace{-0.1ex} \partial_{i} \initiallocationvector \hspace{-0.1ex} \dotp \initiallocationvector^j \hspace{-0.1ex} \partial_{\hspace{-0.1ex}j} \hspace{-0.3ex}
= \hspace{-0.2ex} \boldnabla \initiallocationvector \dotp \hspace{-0.15ex} \boldnablacircled \hspace{-0.1ex}
= \hspace{-0.1ex} \bm{F}^{\hspace{.1ex}\expminusT} \hspace{-0.3ex} \dotp \hspace{-0.1ex} \boldnablacircled}$ \\
%
${\boldnablacircled \hspace{-0.1ex}
= \hspace{-0.2ex} \boldnablacircled \dotp \hspace{-0.2ex} \boldnabla \currentlocationvector
= \initiallocationvector^{i} \partial_{i} \hspace{-0.15ex} \dotp \hspace{-0.1ex} \currentlocationvector^{j} \hspace{-0.1ex} \partial_{\hspace{-0.1ex}j} \hspace{-0.1ex} \currentlocationvector
\stackrel{?}{=} \initiallocationvector^{i} \partial_{i} \currentlocationvector \dotp \hspace{-0.25ex} \currentlocationvector^{j} \hspace{-0.1ex} \partial_{\hspace{-0.1ex}j} \hspace{-0.3ex}
= \hspace{-0.3ex} \boldnablacircled \currentlocationvector \hspace{.1ex} \dotp \hspace{-0.15ex} \boldnabla \hspace{-0.25ex}
= \hspace{-0.1ex} \bm{F}^{\hspace{.1ex}\T} \hspace{-0.3ex} \dotp \hspace{-0.1ex} \boldnabla}$}
\en{and}\ru{и}~(......),
\en{we get}\ru{мы получаем}

\nopagebreak\vspace{-0.4em}
\begin{equation}\label{variationsforthecurrentconfiguration}
\begin{array}{c}
\boldnablacircled \hspace{-0.1ex} \dotp \variation{\hspace{.1ex}\firstpiolakirchhoffstress} \hspace{-0.1ex}
+ \mathcircabove{\rho} \hspace{.25ex} \variation{\hspace{-0.2ex}\massloadvector} \hspace{-0.1ex}
= \zerovector
\hspace{.1ex} , \:\,
%
\variation{\hspace{.1ex}\firstpiolakirchhoffstress} \hspace{-0.1ex}
= \hspace{-0.2ex} \left( \hspace{.1ex} \scalebox{.93}{$ \displaystyle\frac{\partial^2 \hspace{.1ex} \potentialenergydensity}{\raisemath{-0.1em}{\partial \hspace{.1ex} \bm{C} \hspace{.1ex} \partial \hspace{.1ex} \bm{C}}} $} \dotdotp \variation{\hspace{.1ex}\bm{C}} \hspace{-0.15ex} \right) \hspace{-0.3ex} \dotp \bm{F}^{\hspace{.1ex}\T} \hspace{-0.3ex}
+ \hspace{.1ex}
\displaystyle\frac{\partial \hspace{.1ex} \potentialenergydensity}{\raisemath{-0.1em}{\partial \hspace{.1ex} \bm{C}}} \dotp \variation{\bm{F}}^{\hspace{.1ex}\T}
\hspace{-0.4ex} ,
\\[1em]
%
\variation{\bm{F}}^{\hspace{.1ex}\T} \hspace{-0.5ex}
= \variation{\hspace{.1ex} \boldnablacircled \currentlocationvector}
= \hspace{-0.2ex} \boldnablacircled \hspace{.1ex} \variation{\currentlocationvector}
= \hspace{-0.1ex} \bm{F}^{\hspace{.1ex}\T} \hspace{-0.25ex} \dotp \boldnabla \hspace{.1ex} \variation{\currentlocationvector} \hspace{.15ex} ,
\:\,
\variation{\bm{F}} \hspace{-0.2ex} = \variation{\hspace{.1ex} \boldnablacircled \currentlocationvector^{\T}} \hspace{-0.4ex}
= \hspace{-0.2ex} \boldnabla \hspace{.1ex} \variation{\currentlocationvector}^{\hspace{.1ex}\T} \hspace{-0.25ex} \dotp \bm{F}
\hspace{-0.1ex} ,
\\[.5em]
%
\variation{\hspace{.1ex}\bm{C}}
= \smalldisplaystyleonehalf \hspace{.2ex} \variation{ \bigl( \bm{F}^{\hspace{.1ex}\T} \hspace{-0.4ex} \dotp \bm{F} \bigr) } \hspace{-0.25ex}
= \bm{F}^{\hspace{.1ex}\T} \hspace{-0.3ex} \dotp \hspace{.1ex} \infinimentpetitedeformationvariation \dotp \bm{F} ,
\:\,
\infinimentpetitedeformationvariation \equiv \insideinfinitesimalstrainvariation
\hspace{.1ex} .
\end{array}
\end{equation}

.....

\begin{equation*}\begin{array}{c}
\eqref{areachange:nansonformula}
\hspace{.4em} \Rightarrow \hspace{.4em}
%
\initialunitnormal \hspace{.1ex} do = J^{\hspace{.1ex}\expminusone} \currentunitnormal \hspace{.1ex} d\mathcal{O} \hspace{-0.1ex} \dotp \bm{F}
\hspace{.4em} \Rightarrow \hspace{.4em}
%
\initialunitnormal \dotp \variation{\firstpiolakirchhoffstress} \hspace{.1ex} do
= J^{\hspace{.12ex}\expminusone} \currentunitnormal \hspace{-0.1ex} \dotp \bm{F} \hspace{-0.1ex} \dotp \variation{\firstpiolakirchhoffstress} \hspace{.1ex} d\mathcal{O}
\\[.4em]
%
\text{\en{or}\ru{или}} \hspace{1.5ex}
\initialunitnormal \dotp \variation{\firstpiolakirchhoffstress} \hspace{.1ex} do
= \currentunitnormal \dotp \varbivalent{\hspace{-0.2ex}\cauchystress} \hspace{.25ex} d\mathcal{O} ,
\:\:
\varbivalent{\hspace{-0.2ex}\cauchystress} \equiv J^{\hspace{.12ex}\expminusone} \bm{F} \hspace{-0.1ex} \dotp \variation{\firstpiolakirchhoffstress}
\end{array}\end{equation*}

\begin{otherlanguage}{russian}

\vspace{-0.2em}\noindent
---
\ru{введённый здесь }\en{tensor}\ru{тензор}~${\varbivalent{\hspace{-0.2ex}\cauchystress}}$\en{ introduced here}
связан с~вариацией~$\variation{\hspace{.1ex}\firstpiolakirchhoffstress}$
так же, как
$\cauchystress$
связан с~$\firstpiolakirchhoffstress$
(${\cauchystress = \hspace{-0.1ex} J^{\hspace{.1ex}\expminusone} \bm{F} \dotp \hspace{.15ex} \firstpiolakirchhoffstress\hspace{.2ex}}$).
%
Из~\eqref{variationsforthecurrentconfiguration} и ...

.....

...
корректируя
коэффициенты
линейной функции~${\varbivalent{\hspace{-0.2ex}\cauchystress}\hspace{.15ex}( \infinimentpetitedeformationvariation )}$ ...

\end{otherlanguage}
