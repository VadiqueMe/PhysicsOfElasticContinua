\en{\section{The principle of virtual work}}

\ru{\section{Принцип виртуальной работы}}

\label{section:virtualworkprinciple.genericmechanics}

\emph{Mécanique analytique} (1788--89) is a two volume French treatise on analytical mechanics, written by Joseph Louis Lagrange, and published 101 years following Isaac Newton’s \emph{Philosophiæ Naturalis Principia Mathematica}.

\bookauthor{Joseph Louis Lagrange}.
\href{https://play.google.com/books/reader?id=Q8MKAAAAYAAJ&pg=GBS.PP7}{Mécanique analytique. Nouvelle édition, revue et augmentée par l’auteur. Tome premier. Mme Ve Courcier, Paris, 1811.} \howmanypages{490 pages.}

\bookauthor{Joseph Louis Lagrange}.
\href{https://play.google.com/books/reader?id=TmMSAAAAIAAJ&pg=GBS.PP9}{Mécanique analytique. Troisième édition, revue, corrigée et annotée par M.\:J.\:Bertrand. Tome second. Mallet-Bachelier, Paris, 1855.} \howmanypages{416 pages.}

The historical transition from geometrical methods, as presented in Newton’s Principia, to methods of mathematical analysis.

{\small
\setlength{\abovedisplayskip}{2pt}\setlength{\belowdisplayskip}{2pt}

Consider the~exact differential of any set of location vectors~${\locationvector_{\hspace{-0.1ex}i}}$, that are functions of other variable parameters (coordinates) ${q^{1} \hspace{-0.25ex}, q^{2} \hspace{-0.25ex}, ..., q^{n}}$ and time~$t$.

The actual displacement is the differential
\[
\displaystyle d\locationvector_{\hspace{-0.1ex}i} = \frac{\partial \locationvector_{\hspace{-0.1ex}i}}{\partial t} \hspace{.16ex} dt \hspace{.2ex} + \sum_{j=1}^{n} {\frac{\partial \locationvector_{\hspace{-0.1ex}i}}{\partial q^{\hspace{.1ex}j}}} \hspace{.2ex} dq^{\hspace{.1ex}j}
\]

Now, imagine an arbitrary path through the configuration space/manifold. This means it has to satisfy the constraints of the system but not the actual applied forces
\[
\delta \locationvector_{\hspace{-0.1ex}i} = \displaystyle\sum_{j=1}^{n} {\displaystyle\frac {\partial \locationvector_{\hspace{-0.1ex}i}}{\partial q^{\hspace{.1ex}j}}} \hspace{.2ex} \delta q^{\hspace{.1ex}j}
\]

\par}

A~virtual infinitesimal displacement of a~system of particles refers to a~change in the configuration of a~system as the result of any arbitrary infinitesimal change of location vectors (or coordinates) ${\variation{\hspace{.1ex}\locationvector_{\hspace{-0.1ex}k}}}$, consistent with the forces and constraints imposed on the system at the current/given instant~$t$.
This displacement is called \inquotes{virtual} to distinguish it from an~actual displacement of the system occurring in a~time interval~${dt}$, during which the forces and constraints may be changing.

Assume the system is in equilibrium, that is the full force on each particle vanishes, ${\bm{F}_i \hspace{-0.2ex} = \hspace{-0.1ex} \zerovector \hspace{.4em} \forall i}$.
Then clearly the term ${\bm{F}_i \hspace{-0.1ex} \dotp \variation{\hspace{.1ex}\locationvector_i}}$, which is the virtual work of force ${\bm{F}_i}$ in displacement ${\variation{\hspace{.1ex}\locationvector_i}}$, also vanishes for each particle, ${\bm{F}_i \hspace{-0.1ex} \dotp \variation{\hspace{.1ex}\locationvector_i} \hspace{-0.1ex} = 0 \hspace{.4em} \forall i}$.
The~sum of these vanishing products over all particles is likewise equal to zero:
\[
\displaystyle\sum_{\smash{i}} \bm{F}_i \hspace{-0.1ex} \dotp \variation{\hspace{.1ex}\locationvector_i} \hspace{-0.1ex} = 0
\hspace{.15ex} .
\]

Decompose
the full force ${\bm{F}_i}$
into the applied (active) force $\activeforcewithindex{i}$
and the force of constraint $\constraintforcewithindex{i}$,
\[
\bm{F}_i \hspace{-0.1ex} = \hspace{-0.1ex} \activeforcewithindex{i} \hspace{-0.2ex} + \hspace{-0.1ex} \constraintforcewithindex{i}
\]

We now restrict ourselves to systems for which the net virtual work of the force of every constraint is zero:
\[
\displaystyle\sum_{\smash{i}} \constraintforcewithindex{i} \hspace{-0.1ex} \dotp \variation{\hspace{.1ex}\locationvector_i} \hspace{-0.1ex} = 0
\hspace{.15ex} .
\]

We therefore have as the condition for equilibrium of a system that the virtual work of all applied forces vanishes:
\[
\displaystyle\sum_{\smash{i}} \activeforcewithindex{i} \hspace{-0.25ex} \dotp \variation{\hspace{.1ex}\locationvector_i} \hspace{-0.1ex} = 0
\hspace{.15ex} .
\]
--- the principle of virtual work.

Note that coefficients ${\activeforcewithindex{i}}$ can no longer be thought equal to zero: in common ${\activeforcewithindex{i} \hspace{-0.3ex} \neq \hspace{-0.1ex} 0}$, since ${\variation{\hspace{.1ex}\locationvector_i}}$ are not independent but are bound by constraints.

\en{A~virtual displacement}\ru{Виртуальным смещением}
\en{of~a~particle}\ru{частицы}
\en{with vector radius}\ru{с~вектором\hbox{-}радиусом}~${\locationvector_{\hspace{-0.1ex}k}}$
\en{is}\ru{это}
\en{variation}\ru{вариация}
${\variation{\hspace{.1ex}\locationvector_{\hspace{-0.1ex}k}}}$\:---
\en{any}\ru{любое}
\en{infinitesimal}\ru{бесконечно малое}
\en{change}\ru{изменение}
\en{of~vector}\ru{вектора}~${\locationvector_{\hspace{-0.1ex}k}}$,
\en{which is}\ru{которое}
\en{compatible}\ru{совместимо}
\en{with the }\ru{со связями (}constraints\ru{)}.
\en{If}\ru{Если}
\en{the system is free}\ru{система свободна},
\en{that is}\ru{то есть}
\en{there are no constraints}\ru{связей нет},
\en{then}\ru{тогда}
\en{virtual displacements}\ru{виртуальные смещения}
${\variation{\hspace{.1ex}\locationvector_{\hspace{-0.1ex}k}}}$
\en{are perfectly random}\ru{совершенно случайны}.

\begin{otherlanguage}{russian}

Связи бывают
голономные
(holonomic, или геометрические),
связывающие только положения~(смещения)\:---
\en{they are functions}\ru{это функции}
\en{of only}\ru{лишь}
\en{the coordinates}\ru{координат}
\en{and}\ru{и}\ru{,}
\en{probably}\ru{возможно}\ru{,}
\en{time}\ru{времени}

\nopagebreak\vspace{-0.1em}
\begin{equation}\label{holonomicconstraint}
c\hspace{.2ex}(\locationvector, t) = 0
\end{equation}

\vspace{-0.1em}\noindent
--- и~неголономные~(или дифференциальные),
содержащие производные координат по~времени:
${c\hspace{.2ex}(\locationvector, \mathdotabove{\locationvector}, t) = 0}$
\en{and not}\ru{и~не} интегрируемые
\en{till}\ru{до}
\en{the geometrical constraints}\ru{геометрических связей}.

\en{When}\ru{Когда}
\en{all}\ru{все}
\en{constraints}\ru{связи}
\en{are holonomic}\ru{голономные},
\en{then}\ru{тогда}
\en{the virtual displacements}\ru{виртуальные смещения}
\en{of~a~particle}\ru{частицы}
\inquotes{${\hspace{.05ex}k\hspace{.25ex}}$}
\en{satisfy}\ru{удовлетворяют}
\en{the equation}\ru{уравнению}

\nopagebreak\vspace{-0.1em}\begin{equation}\label{requirementforvirtualdisplacements}
\displaystyle \sum_{\smash{j=1}}^{m}
\scalebox{.92}{$ \displaystyle
   \frac{\raisemath{-0.12em}{\partial \hspace{.1ex} c_{j}}}{\partial \hspace{.1ex} \locationvector_{\hspace{-0.1ex}k}} $}
\hspace{-0.1ex} \dotp
\variation{\hspace{.1ex}\locationvector_{\hspace{-0.1ex}k}}
\hspace{-0.1ex} =
0
\hspace{.1ex} .
\vspace{-0.25em}\end{equation}

\en{In constrained}\ru{В~связанных}~%
(\en{non-free}\ru{несвободных})
\en{systems}\ru{системах},
\en{all forces}\ru{все силы}
\en{can be}\ru{могут быть}
\en{divided}\ru{поделен\'{ы}}
\en{into two groups}\ru{на две группы}\::
\en{the~active forces}\ru{активные силы}~${\activeforcewithindex{k}}$
\en{and }\ru{и~}%
\en{the~}\ru{силы }\en{constraint}\ru{связи}
(\en{or}\ru{или}
\en{reaction}\ru{реакции})\en{ forces}.

Реакция~$\mathboldPhi_k$
действует со~стороны всех материальных ограничителей
на~частицу \inquotes{${\hspace{.05ex}k\hspace{.25ex}}$}
и~меняется согласно уравнению~\eqref{holonomicconstraint}
для каждой связи.

\en{The~constraints}\ru{Связи}
\en{are assumed}\ru{предполагаются}
\en{to be ideal}\ru{идеальными},
\en{that is}\ru{то есть}

\nopagebreak
\begin{equation}\label{theidealityofconstraints}
\scalebox{.9}{$\displaystyle \sum_{\smash{k}}$} \hspace{.2ex} \mathboldPhi_k \hspace{-0.2ex} \dotp \variation{\hspace{.1ex}\locationvector_{\hspace{-0.1ex}k}} \hspace{-0.16ex} = 0
%%\quad \textrm{---}
\vspace{-0.25em}\end{equation}
\noindent
\:---
\en{the~work}\ru{работа}
\en{of~constraint~(reaction) forces}\ru{сил связи~(реакции)}
\en{is equal to zero}\ru{равна нулю}
\en{on~any}\ru{на~любых}
\en{virtual displacements}\ru{виртуальных смещениях}.

\en{The~principle of virtual work}\ru{Принцип виртуальной работы}
\en{is}\ru{так\'{о}в}

\nopagebreak\vspace{-0.2em}
\begin{equation}\label{discrete:principleofvirtualwork}
\displaystyle \sum_{\smash{k}} \hspace{-0.1ex} \Bigl( \hspace{-0.1ex}
\activeforcewithindex{k} \hspace{-0.2ex} - m_k \mathdotdotabove{\locationvector}_{\hspace{-0.1ex}k}
\Bigr) \hspace{-0.3ex} \dotp \variation{\hspace{.1ex}\locationvector_{\hspace{-0.1ex}k}} \hspace{-0.1ex} = 0
\hspace{.1ex} ,
\vspace{-0.3em}
\end{equation}

\vspace{-0.2em}\noindent
\en{where}\ru{где}~${\activeforcewithindex{k}}$\en{ are}\ru{\:---}
\en{only}\ru{лишь}
\en{active forces}\ru{активные силы},
\en{without}\ru{без}
\en{reactions}\ru{реакций}
\en{of~constraints}\ru{связей}.

\en{Differential}\ru{Дифференциальное}
\en{variational}\ru{вариационное}
\en{equation}\ru{уравнение}~\eqref{discrete:principleofvirtualwork}
\en{may seem like}\ru{может показаться}
\en{a~trivial consequence}\ru{тривиальным следствием}
\en{of~the~}\ru{закона }Newton’\en{s}\ru{а}\en{ law}~\eqref{law:ofnewton}
\en{and }\ru{и~}\en{the~ideality of~constraints}\ru{идеальности связей}~\eqref{theidealityofconstraints}.
Однако
содержание~\eqref{discrete:principleofvirtualwork}
несравненно обширнее.
Читатель вскоре увидит, что
принцип~\eqref{discrete:principleofvirtualwork}
может быть положен
в~основу механики~\cite{gantmacher-analyticalmechanics}.
\en{The~various models}\ru{Разные модели}
\en{of~elastic media}\ru{упругих сред},
\en{being described}\ru{описываемые}
\en{in this book}\ru{в~этой книге},
\en{are based}\ru{основаны}
\en{on this}\ru{на~этом}
\en{principle}\ru{принципе}.

Для~примера рассмотрим совершенно жёсткое~(недеформируемое) твёрдое тело.

.... \eqref{completelyrigidbody.locationvectorofanypointdecomposed} ${\Rightarrow}$
${\variation{\locationvector} = \variation{\positionofthepole} + \variation{\bm{x}}}$

(begin copied from \chapterdotsectionref{chapter:mathapparatus}{section:calculusofvariations})

Варьируя тождество~\eqref{orthogonalityofrotationtensor}, получим ${\variation{\rotationtensor} \hspace{-0.2ex} \dotp \rotationtensor^{\T} \hspace{-0.2ex} = - \hspace{.2ex} \rotationtensor \dotp \variation{\rotationtensor}^{\T}\!}$.
Этот тензор антисимметричен, и~потому выражается через свой сопутствующий вектор~${\varvector{o}}$ как~${\variation{\rotationtensor} \hspace{-0.1ex} \dotp \rotationtensor^{\T} \hspace{-0.3ex} = \varvector{o} \hspace{-0.2ex} \times \hspace{-0.2ex} \UnitDyad}$.
Приходим к~соотношениям

\nopagebreak\vspace{-0.5em}\begin{equation}
\variation{\rotationtensor} \hspace{-0.1ex} = \varvector{o} \hspace{-0.1ex} \times \hspace{-0.1ex} \rotationtensor , \:\:
\varvector{o} = - \hspace{.2ex} \scalebox{.93}{$ \displaystyle\onehalf $} \hspace{-0.1ex} \Bigl( \hspace{-0.1ex} \variation{\rotationtensor} \hspace{-0.1ex} \dotp \rotationtensor^{\T} \Bigr)_{\hspace{-0.25em}\Xcompanion}
\hspace{-0.1ex} ,
\end{equation}

(end of copied from \chapterdotsectionref{chapter:mathapparatus}{section:calculusofvariations})

...


Проявилась замечательная особенность~\eqref{discrete:principleofvirtualwork}\::
это уравнение эквивалентно системе такого порядка,
каково число степеней свободы системы,
то~есть сколько независимых вариаций~${\variation{\hspace{.1ex}\locationvector_{\hspace{-0.1ex}k}}}$ мы имеем.
Если система $N$~точек имеет $m$ связей,
то число степеней свободы
${n = 3N \hspace{-0.25ex} - m}$.

...

