\en{\chapter{Perturbation methods (asymptotic methods)}}

\ru{\chapter{Методы возмущений (асимптотические методы)}}

\thispagestyle{empty}

\label{chapter:asymptoticperturbationmethods}

\begin{changemargin}{\parindent}{\parindent}
\vspace{-2.5em}
{\noindent\small
\setlength{\parskip}{\spacebetweenparagraphs}

\en{Of approximate approaches}\ru{Из приближённых подходов}
\en{to the analysis}\ru{к~анализу}
\en{of~nonlinear systems}\ru{нелинейных систем},
\en{perturbation methods}\ru{методы возмущений}
\en{are applied}\ru{применяются}
\en{the~most often}\ru{чаще всего}.

\par}
\vspace{-1.4em}
\end{changemargin}

\en{\section{Asymptotic decomposition}}

\ru{\section{Асимптотическое разложение}}

%%\en{in space and time}\ru{в~пространстве и~времени}

\en{\dropcap{U}{ntil} now}\ru{\dropcap{Д}{о}~сих пор}\en{,}
\en{the~arguments of a~function}\ru{аргументами функций}
\en{were only the~coordinates and time}\ru{были только координаты и~время}.
\en{Dimensions of a~body}\ru{Размеры тела},
\en{elastic moduli}\ru{упругие модули},
\en{ranges of actions}\ru{диапазоны воздействий}\:---
\en{all these parameters}\ru{все эти параметры}
\en{were considered known}\ru{считались известными}.
\en{All asymptotic methods}\ru{Все асимптотические методы}
\en{are based}\ru{основаны}
\en{on studying}\ru{на изучении}
\en{how the solution}\ru{как решение}
\en{depends on parameters}\ru{зависит от параметров}.

\en{A~decomposition of~perturbation}\ru{Разложение возмущения} \en{usually consists}\ru{обычно состоит} \en{of the first two terms}\ru{из первых двух слагаемых}.
\en{If}\ru{Если} \en{a~problem}\ru{задача} \en{with unknown}\ru{с~неизвестной}~$u$ \en{contains parameter}\ru{содержит параметр}~$a$, \en{then}\ru{то}, \en{assuming}\ru{полагая}

\nopagebreak\vspace{-0.2em}\begin{equation*}
a = a_{0} \hspace{-0.1ex} + \smallparameter a_{1}
\hspace{.1ex} ,
\end{equation*}

\vspace{-0.2em}\noindent
\en{a~solution}\ru{решение} \en{is sought}\ru{ищется} \en{as series}\ru{в~виде ряда}

\nopagebreak\vspace{-0.2em}\begin{equation}\label{asymptoticsolution}
u = u_0 \hspace{-0.1ex} + \smallparameter u_1 \hspace{-0.1ex} + \smallparameter^2 u_2 \hspace{-0.1ex}
+ \ldots
\end{equation}

\vspace{-0.2em}\noindent
\en{Additional argument}\ru{Дополнительный аргумент}~$\smallparameter$ \en{is called a~formal small parameter}\ru{называется формальным малым параметром}.

\en{A~decomposition of~perturbation}\ru{Разложение возмущения}
\en{may diverge}\ru{может расходиться},
\en{but}\ru{но}
\en{at the same time}\ru{в~то~же время}
\en{it}\ru{оно}
\en{can be}\ru{может быть}
\en{more useful}\ru{более полезным}
\en{description}\ru{описанием}
\en{of the~solution}\ru{решения}\ru{,}
\en{than}\ru{чем}
\en{a~representation}\ru{представление}\ru{,}
\en{that converges}\ru{которое сходится}
\en{uniformly and absolutely}\ru{равномерно и~абсолютно}.

\en{Decomposition}\ru{Разложение}~\eqref{asymptoticsolution}
\en{looks like}\ru{выглядит как}
\en{a~common power series}\ru{обычный степенной ряд}.
\en{However}\ru{Однако},
\en{the~approach}\ru{подход}
\en{within the~perturbation methods}\ru{в~методах возмущений}
\en{is different}\ru{отличается}:
\en{series}\ru{ряды}
\en{are considered there}\ru{рассматриваются там}
\en{as asymptotic}\ru{как асимптотические}
\en{with convergence}\ru{со сходимостью}
${\smallparameter \to 0}$,
\en{and not}\ru{а~не}~${n \to \infty}$,
\en{where}\ru{где}~n
\en{defines}\ru{определяет}\ru{,}
\en{how many terms}\ru{как много членов}
\en{to hold}\ru{удерживать}.

\nopagebreak\vspace{-0.2em}
\begin{equation}\label{powerseries-asymptotic}
\phi = \displaystyle\sum_{k=0}^{\infty} \phi_{k}(\smallparameter)
\hspace{.1ex} ,
\hspace{.5em}
\underset{\scriptstyle \smallparameter \to 0}{\operatorname{lim}} \displaystyle\frac{\phi - \scalebox{.88}{$ \displaystyle\sum_{k=0}^{n} $} \hspace{.1ex} \phi_{k}}{\phi_{n}} = 0
\hspace{.1ex} .
\end{equation}

\vspace{-0.2em}\noindent
\en{In other words}\ru{Иначе говоря},
\en{the remainder of series}\ru{остаток ряда} \en{is}\ru{есть} \en{a~higher order infinitesimal}\ru{бесконечно-малая высшего порядка} \en{in comparison with the last withheld term}\ru{по~сравнению с~последним удержанным членом}.
\en{Obviously}\ru{Очевидно}, \en{a~power series}\ru{степенной ряд}~\eqref{asymptoticsolution} \en{also converges as an asymptotic series}\ru{сходится и~как асимптотический ряд}.

\en{But}\ru{Но} \en{the decomposable unknowns}\ru{разлагаемые неизвестные} \en{usually also depend on}\ru{обычно также зависят от} \en{the }\inquotes{\en{main}\ru{главных}} \en{arguments}\ru{аргументов}\:--- \en{on coordinates and time}\ru{от~координат и~времени}.
\en{The convergence}\ru{Сходимость} \en{for}\ru{для}~${\smallparameter \to 0}$ \en{must be uniform}\ru{должна быть равномерной} \en{across these}\ru{по этим} \inquotes{\en{main}\ru{главным}} \en{arguments}\ru{аргументам}\:--- \en{this is a~requirement}\ru{это требование} \en{for effective use}\ru{к~эффективному использованию} \en{of asymptotic decompositions}\ru{асимптотических разложений}.
\en{For example}\ru{Например}

\nopagebreak\vspace{-0.3em}\begin{equation*}
\sine (1 + \smallparameter) t
= \sine t + \smallparameter t \cosine \smallparameter t
- \smallerdisplaystyleonehalf \smallparameter^2 t^2 \sine \smallparameter t
+ \ldots
\end{equation*}

\vspace{-0.2em}\noindent
\en{doesn’t satisfy the requirement of uniformity}\ru{не удовлетворяет требованию равномерности}, \en{because}\ru{поскольку} \en{with }\ru{с~}${t \to \infty}$ \en{subsequent terms prevail over previous ones}\ru{последующие члены преобладают над предыдущими}.

\en{Why}\ru{Чем~же} \en{asymptotic methods}\ru{асимптотические методы} \en{are so attractive}\ru{так привлекательны}?
\en{As example}\ru{Как пример} \en{take a~solution of the equation}\ru{возьмём решение уравнения}

\nopagebreak\vspace{-0.2em}\begin{equation*}
f(u, \smallparameter) \hspace{-0.1ex} = 0
\hspace{.1ex} .
\end{equation*}

\vspace{-0.2em}\noindent
\en{Substituting decomposition}\ru{Подставив разложение}~\eqref{asymptoticsolution} \en{into}\ru{в}~\eqref{powerseries-asymptotic} \en{and }\ru{и~}\en{equating coefficients}\ru{приравняв коэффициенты} \en{at the same degrees of}\ru{при одинаковых степенях}~$\smallparameter$, \en{we get}\ru{получим}

\nopagebreak\vspace{-0.2em}\begin{gather*}
f(u_{0}, 0) = 0
\hspace{.1ex} ,
\\
\bigl( \partial_{u} f \bigr)_{0} \hspace{-0.2ex}
+ u_{1} \bigl( \partial_{\smallparameter} f \bigr)_{0} \hspace{-0.2ex}
= 0
\hspace{.1ex} ,
\\[.2em]
\bigl( \partial_{u} f \bigr)_{0} \hspace{-0.1ex} u_{2} \hspace{-0.1ex}
+ \smallerdisplaystyleonehalf \bigl( \partial_{u}^2 f \bigr)_{0} \hspace{-0.1ex} u_{1}^2 \hspace{-0.2ex}
+ \smallerdisplaystyleonehalf \bigl( \partial_{\smallparameter}^2 f \bigr)_{0}
+ \bigl( \partial_{u} \partial_{\smallparameter} f \bigr)_{0} \hspace{-0.1ex} u_{1} \hspace{-0.1ex}
= 0
\hspace{.1ex} ,
\\
\ldots
\end{gather*}

\en{If the problem}\ru{Если проблема} \en{for}\ru{для}~${u_{0}}$ \en{is uniquely solvable}\ru{однозначно разрешима} \en{at the first step}\ru{на первом шаге}, \en{then}\ru{то} \en{subsequent steps}\ru{последующие шаги} \en{will give corrections}\ru{дадут поправки} ${u_{1}, u_{2}, \ldots}$
\en{Small corrections}\ru{Малые поправки} \en{are barely important}\ru{едва~ли важны} \en{and }\ru{и~}${u_{0}}$ \en{is enough}\ru{достаточно}, \en{but then}\ru{но тогда} \ru{исчезает }\en{the asymptotic analysis}\ru{асимптотический анализ}\en{ disappears}, \en{since}\ru{ведь} \en{formally}\ru{формально} \en{small terms}\ru{малые члены} \en{in}\ru{в}~\eqref{powerseries-asymptotic} \en{are just discarded}\ru{просто отбрасываются}.
\en{However}\ru{Впрочем}, \en{it happens that}\ru{бывает так, что} \en{these corrections}\ru{эти поправки} \en{contain in self}\ru{содержат в~себе} \en{some important information}\ru{некую важную информацию}, \en{absent}\ru{отсутствующую} \en{in}\ru{в}~${u_{0}}$\:--- \en{then they}\ru{тогда они} \en{play}\ru{играют} \en{the main role}\ru{главную роль}.
\en{It’s worth mentioning that}\ru{Ст\'{о}ит упомянуть, что} \en{all corrections}\ru{все поправки} \en{follow}\ru{следуют} \en{from the linear problem}\ru{из линейной проблемы} \en{with the same operator}\ru{с~одним и тем~же оператором}
${\bigl( \partial_{u} f \bigr)_{0}}$.

\en{But}\ru{Но} \en{all non-trivial and effective solutions}\ru{все нетривиальные и~эффективные решения} \en{are obtained}\ru{получаются} \en{in other ways}\ru{иными способами}, \en{non-uniqueness of the solution}\ru{неединственность решения} \en{at the first step}\ru{на первом шаге} \en{characterizes them}\ru{характеризует их}.
\textcolor{magenta}{\en{About this}\ru{Об этом}\:--- \en{in the next sections}\ru{в~следующих секциях}}.

\en{And yes}\ru{И~да}, \en{asymptotic methods}\ru{асимптотические методы} \en{change}\ru{меняют} \en{the initially complex formulation of the problem}\ru{первоначальную сложную постановку задачи}~\eqref{powerseries-asymptotic} \en{to a~simpler one}\ru{на более простую}.
\en{Essential is that}\ru{Существенно то, что} \en{this does not happen}\ru{это не происходит} \en{by }\inquotes{\en{simple}\ru{простым}} \en{discarding of terms}\ru{отбрасыванием членов}, \en{but quite correctly}\ru{но вполне корректно}\:--- \en{equalities remain equalities}\ru{равенства остаются равенствами}.
\en{However}\ru{Однако}, \en{the convergence is not proven}\ru{сходимость не доказывается}, \en{so}\ru{так что} \en{there is no }\en{complete mathematical accuracy}\ru{полной математической точности}\ru{ нет}.

\en{In previous chapters,}\ru{В~предыдущих главах} \en{asymptotic problems}\ru{асимптотические проблемы} \en{have already arisen}\ru{уж\'{е} возникали}.
\en{The~linear theory}\ru{Линейная теория}
\en{follows}\ru{следует}
\en{from the~non\-linear theory}\ru{из~нелинейной теории}
\en{thru}\ru{через}
\en{an~asymptotic decomposition}\ru{асимптотическое разложение}
\en{by the~value of~load}\ru{по величине нагрузки}~(\chapterdotsectionref{chapter:linearclassicalelasticity}{section:wholesetofequations.lineartheory}).
\en{The momentless theory}\ru{Безмоментная теория}
\en{derives}\ru{происходит}
(\en{accurate to edge effects}\ru{с~точностью до~краевых эффектов})
\en{from the moment (micropolar) theory}\ru{из моментной (микрополярной) теории}~(\chapterref{chapter:cosseratcontinuum})\ru{,}
\en{when}\ru{когда}
\en{the }\inquotes{\en{moment}\ru{моментные}}
\en{stiffnesses}\ru{жёсткости}
\en{are approaching infinity}\ru{приближаются к~бесконечности}.
\en{In}\ru{В}~\en{the~thermoelasticity}\ru{термоупругости}~(\chapterref{chapter:thermoelasticity})\en{,}
\en{the~use}\ru{использование}
\en{of the~heat equation}\ru{уравнения теплопроводности}~\eqrefwithchapterdotsection{heatequation:mathematicalphysics}{chapter:thermoelasticity}{section:heatequation}
\en{instead}\ru{вместо}
\en{of the whole entire}\ru{всего целого}
\en{balance of~energy}\ru{баланса энергии}
\en{needs to be proved}\ru{нужно доказать}
\en{by asymptotic methods}\ru{асимптотическими методами}.

\en{To a~certain degree}\ru{В~какой\hbox{-}то степени} \en{the introduction of a~small parameter}\ru{введение малого параметра}~${\smallparameter \to 0}$ \en{is the weakness}\ru{является слабостью} \en{of all asymptotic approaches}\ru{всех асимптотических подходов}.
\en{A~protestation like}\ru{Возражение типа} \inquotes{\en{infinitesimal parameters do not exist}\ru{бесконечно-малых параметров не~бывает}, \en{all quantities are finite}\ru{все величины конечны}} \en{is barely constructive here}\ru{едва~ли конструктивно здесь}.
\en{The more relevant question is}\ru{Более актуален вопрос}: \en{what is a~small parameter}\ru{что такое малый параметр}?
\en{Usually}\ru{Обычно} \en{a~problem}\ru{проблема} \en{is reformulated}\ru{переформулируется} \en{in }\ru{в~}\inquotes{\en{dimensionless}\ru{безразмерных}} \en{quantities}\ru{величинах}, \en{then}\ru{тогда} \en{that}\ru{тот} \inquotes{\en{dimensionless}\ru{безразмерный}} \en{parameter}\ru{параметер} \en{is taken}\ru{берётся} \en{as}\ru{за} \en{a~small parameter}\ru{малый параметр}~${\smallparameter}$, \en{which turns out to be small}\ru{который оказывается малым}.
\en{But another way is possible too}\ru{Но возможен и~другой путь}: \en{if it’s known}\ru{если известно}\ru{,} \en{that some parameter}\ru{что некий параметр}~${\omega}$ \en{influences the~solution only a~little}\ru{лишь немного влияет на решение}, \en{then}\ru{то}, \en{redesignating it}\ru{переобозначив его} \en{as}\ru{как}~${\smallparameter \omega}$, \en{do asymptotic analysis}\ru{сделать асимптотический анализ} \en{for}\ru{для}~${\smallparameter \to 0}$.

\en{Surely}\ru{Разумеется}, \en{these are not \inquotes{laws} of~asymptotics}\ru{это не \inquotes{законы} асимптотики}, \en{not recommendations}\ru{не рекомендации}, \en{but only considerations}\ru{а~лишь соображения}.
\en{There’s no common theory of asymptotic methods}\ru{Нет общей теории асимптотических методов}, \en{their application}\ru{их применение} \en{is to some extent}\ru{есть в~какой-то степени} \en{an art}\ru{искусство}.

\en{A~deeper description}\ru{Более глубокое описание} \en{of asymptotic methods}\ru{асимптотических методов} \en{is written}\ru{написано} \en{in the books}\ru{в~книгах} \en{by }Ali Hasan\ru{’а} Nayfeh \cite{nayfeh-introtoperturbation, nayfeh-perturbation}.

\en{\section{Splitting in a linear algebraic system}}

\ru{\section{Расщепление в линейной алгебраической системе}}

\en{This simple case}\ru{Этот простой случай} \ru{хорошо }\en{illustrates}\ru{иллюстрирует} \en{asymptotic methods}\ru{асимптотические методы}\en{ well}.

\begin{otherlanguage}{russian}

\en{A~linear system is considered}\ru{Рассматривается линейная система}

\nopagebreak\begin{equation}
C_{i\hspace{-0.1ex}j} \hspace{.2ex} u_{j} \hspace{-0.16ex} = f_{i} \hspace{.1ex} , \:\:
C_{i\hspace{-0.1ex}j} \hspace{-0.2ex} = C_{i\hspace{-0.1ex}j}^{\hspace{0.2ex}\scalebox{0.66}[0.66]{(0)}} \hspace{-0.2ex} + \smallparameter \hspace{.2ex} C_{i\hspace{-0.1ex}j}^{\hspace{0.2ex}\scalebox{0.66}[0.66]{(1)}}
\end{equation}

\noindent
с~матрицей~$C_{i\hspace{-0.1ex}j}$ и~столбцами неизвестных~$u_j$ и~нагрузок~$f_i$.
Процесс построения асимптотического решения определяется тем, вырождена матрица~${C_{i\hspace{-0.1ex}j}^{\hspace{.2ex}\scalebox{0.66}{(0)}}}$ или~нет.
\en{Three cases are possible}\ru{Возможны три случая}.

\indent
\textboldextended{1$^{\hspace{-0.1ex}\circ}$\hspace{-1ex}.}\, ${\operatorname{det} \hspace{.16ex} C_{i\hspace{-0.1ex}j}^{\hspace{0.2ex}\scalebox{0.66}[0.66]{(0)}} \hspace{-0.25ex} \neq 0}$.
Однородная задача
\vspace{-0.2em}\begin{equation}
C_{i\hspace{-0.1ex}j}^{\hspace{.2ex}\scalebox{0.66}[0.66]{(0)}} \hspace{.1ex} u_{j} \hspace{-0.16ex} = 0
\end{equation}

\nopagebreak\vspace{-0.5em}\noindent
имеет лишь тривиальное~(нулевое) решение.
Матрица~${C_{i\hspace{-0.1ex}j}^{\hspace{0.2ex}\scalebox{0.66}[0.66]{(0)}}}$ обратима, неоднородная задача всегда однозначно реш\'{и}ма. Решение строится так:
\begin{equation}
u_{j} = \ldots
\end{equation}

...


\indent \textboldextended{2$^{\hspace{.1ex}\circ}$\hspace{-1.1ex}.}\, ${\operatorname{det} \hspace{.16ex} C_{i\hspace{-0.1ex}j}^{\hspace{0.2ex}\scalebox{0.66}[0.66]{(0)}} \hspace{-0.25ex} = 0}$

...


\indent \textboldextended{3$^{\hspace{.1ex}\circ}$\hspace{-1.1ex}.}\, ${\operatorname{det} \hspace{.16ex} C_{i\hspace{-0.1ex}j}^{\hspace{0.2ex}\scalebox{0.66}[0.66]{(0)}} \hspace{-0.25ex} = 0}$

...



\end{otherlanguage}

\en{\section{Poincar\'{e} method}} % Jules Henri Poincaré

\ru{\section{Метод Poincar\'{e}}}

\en{This method}\ru{Этот метод},
\en{associated}\ru{ассоциируемый}
\en{with the~name}\ru{с~именем}
\href{https://en.wikipedia.org/wiki/Henri_Poincar%C3%A9}{Jules Henri Poincar\'{e}},
\en{is widely known}\ru{широк\'{о} известен}
\en{in the~theory of nonlinear oscillations}\ru{в~теории нелинейных колебаний}.
%
\en{It is intended}\ru{Он предназначен},
\en{in particular}\ru{в~частности},
\en{to determine}\ru{для~определения}
\en{the~periodic solutions}\ru{периодических решений}
\en{of equation}\ru{уравнения}

\nopagebreak\vspace{-0.2em}\begin{equation}\label{nonlinearvibrations:famousequation}
\mathdotdotabove{u} + u = \smallparameter \hspace{.1ex} f(u, \mathdotabove{u})
\end{equation}

....

\en{\section{Van der~Pol averaging method}} % Balthasar van der Pol

\ru{\section{Метод осреднения Van der~Pol’я}}

% https://en.wikipedia.org/wiki/Balthasar_van_der_Pol

\en{Here}\ru{Здесь}
\en{again}\ru{опять}
\en{figures}\ru{фигурирует}
\en{equation}\ru{уравнение}~\eqref{nonlinearvibrations:famousequation},
\en{but}\ru{но}
\en{the~solutions being found}\ru{находимые решения}\en{ are}\ru{\:---}
\ru{уж\'{е} }\en{not only periodic}\ru{не~только периодические}\en{ anymore}.
%
\en{Introducing the~phase plane}\ru{Ввод\'{я} фазовую плоскость}

...

\begin{otherlanguage}{russian}

Процедура осреднения применяется во многих темах,
таких как тонкие тела и~композиты.
Осреднение вне асимптотических методов
ведёт обычно к не~з\'{а}мкнутому набору уравнений.
Для замыкания системы приходится добавлять некие гипотезы,
убавляющие убедительность теории.
Иная ситуация в~асимптотике:
условия разрешимости для поправочных членов
с~необходимостью
приводят к соответствующим интегральным соотношениям.

\end{otherlanguage}

\en{\section{Coalescence of asymptotic decompositions}}

\ru{\section{Сращивание асимптотических разложений}}

\begin{otherlanguage}{russian}

Основоположник метода сращивания внешних и~внутренних асимптотических разложений\:--- Ludwig Prandtl.
Рассматривая течение вязкой жидкости, он заметил, что влияние малой вязкости локализовано у~края\:--- в~тонком слое на краю.
Вдал\'{и} от~края жидкость ведёт себя как идеальная.
Одни и~те~же уравнения Navier--Stokes\ru{’а} по-разному упрощаются вдали от края и~около него [...]

Метод сращивания состоит из трёх процедур: построения внешнего разложения, построения внутренних разложений и сращивания внешнего разложения с~внутренними.
Метод предназначен для дифференциальных уравнений с~малым параметром при старших производных.
Вдали от края решение меняется плавно, формально малые члены можно отбросить, уравнение имеет пониженный порядок\:--- всё это характерно для внешнего разложения.
У~края наоборот: решение меняется быстро,   первостепенную роль играют старшие производные, хотя имеют малые коэффициенты.
Но внешнее и~внутреннее разложения\:--- это разные формы одного решения, они должны быть состыкованы процедурой сращивания. Рассмотрим пример.

Задача о~прогибе $u(x)$ натянутой струны с~закреплёнными концами под действием равномерно распределённой нагрузки может быть поставлена так:

...

\end{otherlanguage}

\en{\section{Multiple-scale analysis (method of multiple scales)}}

\ru{\section{Многоуровневый анализ (метод многих масштабов)}}

\begin{otherlanguage}{russian}

\href{https://en.wikipedia.org/wiki/Multiple-scale_analysis}{Этот метод} привлекателен, естественен и\:--- как написано у Ali Hasan Nayfeh

...



\end{otherlanguage}

\en{\section{Equations with slowly varying parameters}} % coefficients

\ru{\section{Уравнения с медленно меняющимися параметрами}} % коэффициентами

\begin{otherlanguage}{russian}

Рассмотрим гармонический осциллятор, собственная частота которого медленно меняется во~времени

...



\end{otherlanguage}

\en{\section{Thin bodies}}

\ru{\section{Тонкие тела}}

\begin{otherlanguage}{russian}

Задачи теории упругости часто ставятся для тонких тел\:--- стержней, пластин и~оболочек.
Таковы многие элементы конструкций, но и в~природе вне~человека тонкие тела встречаются довольно часто.

Решение задач упругости для~тонких тел многие десятилетия основывалось на~неких гипотезах о~распределении решения по~толщине и~о~порядках одних неизвестных относительно других.
Построенные так теории сыграли больш\'{у}ю роль в~практике инженерных расчётов.
Однако, им не~хватало логической стройности и~убедительности, их хотелось обосновать, уточнить\:--- а~в~последнее время и~уничтожить (в~связ\'{и} с~появлением великолепных компьютеров).
Но открытие асимптотического расщепления прояснило картину: в~тонком теле трёхмерная задача расщепляется на задачи меньшей размерности.
Классические теории тонких тел получили и~подтверждение, и~развитие.

Рассмотрим задачу о~кручении из~...

...




\end{otherlanguage}

\section*{\small \wordforbibliography}

\begin{changemargin}{\parindent}{0pt}
\fontsize{10}{12}\selectfont

\begin{otherlanguage}{russian}

Ali Hasan Nayfeh’s book~\cite{nayfeh-perturbation} is an~excellent introduction to perturbation methods (asymptotic methods).

Всё~разнообразие асимптотических методов представлено в~монографиях ...

\end{otherlanguage}

\end{changemargin}
