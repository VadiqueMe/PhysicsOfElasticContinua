\en{\chapter{Asymptotic methods}}

\ru{\chapter{Асимптотические методы}}

\thispagestyle{empty}

\label{chapter:asymptoticmethods}

\en{\section{Asymptotic decompositions}}

\ru{\section{Асимптотические разложения}}

\begin{otherlanguage}{russian}

\lettrine[lines=2, findent=2pt, nindent=0pt]{Д}{о}~сих~пор аргументами рассматриваемых функций были только координаты и~время. Размеры тела, упругие модули, характерные масштабы изменения ...

...


В~предыдущих главах уж\'{е} возникали асимптотические проблемы. Линейная теория получается из~нелинейной с~помощью разложений по~масштабу нагрузки~(\chapdotpararef{chapter:linearclassicalelasticity}{para:wholesetofequations.lineartheory}). Безмоментная теория вытекает (с~точностью до~краевых эффектов) из~моментной~(\chapref{chapter:cosseratcontinuum}) при~устремлении \inquotes{моментных} жёсткостей к~бесконечности. В~термо\-упругости~(\chapref{chapter:thermoelasticity}) использование уравнения теплопроводности вместо полного баланса энергии должно быть обосновано асимпто\-ти\-чес\-кими методами.

Уязвимым в~какой\hbox{-}то степени местом


...



\end{otherlanguage}

\en{\section{Splitting in a linear algebraic system}}

\ru{\section{Расщепление в линейной алгебраической системе}}

\begin{otherlanguage}{russian}

Этот простой случай очень познавателен, он хорошо поясняет особенности асимптотических явлений. Рассматривается линейная система
\begin{equation}
C_{ij} \hspace{.2ex} u_{j} \hspace{-0.16ex} = f_{i} \hspace{.1ex} , \:\:
C_{ij} \hspace{-0.2ex} = C_{ij}^{\hspace{0.2ex}\scalebox{0.66}[0.66]{(0)}} \hspace{-0.2ex} + \smallparameter \hspace{.1ex} C_{ij}^{\hspace{0.2ex}\scalebox{0.66}[0.66]{(1)}}
\end{equation}
\noindent с~матрицей~$C_{ij}$ и~столбцами неизвестных~$u_j$ и~нагрузок~$f_i$. Процесс построения асимптотического решения определяется тем, вырождена матрица~${C_{ij}^{\hspace{0.2ex}\scalebox{0.66}[0.66]{(0)}}}$ или~нет. Возможны три случая.

\indent \textbf{1$^{\circ}$\hspace{-1ex}.}\, ${\operatorname{det} \hspace{.16ex} C_{ij}^{\hspace{0.2ex}\scalebox{0.66}[0.66]{(0)}} \hspace{-0.25ex} \neq 0}$. Однородная задача
\vspace{-0.2em}\begin{equation}
C_{ij}^{\hspace{0.2ex}\scalebox{0.66}[0.66]{(0)}} \hspace{.1ex} u_{j} \hspace{-0.16ex} = 0
\end{equation}

\vspace{-0.5em} \noindent имеет лишь тривиальное~(нулевое) решение.
Матрица~${C_{ij}^{\hspace{0.2ex}\scalebox{0.66}[0.66]{(0)}}}$ обратима, неоднородная задача всегда однозначно реш\'{и}ма. Решение строится так:
\begin{equation}
u_{j} = \ldots
\end{equation}

...


\indent \textbf{2$^{\circ}$\hspace{-1ex}.}\, ${\operatorname{det} \hspace{.16ex} C_{ij}^{\hspace{0.2ex}\scalebox{0.66}[0.66]{(0)}} \hspace{-0.25ex} = 0}$

...


\indent \textbf{3$^{\circ}$\hspace{-1ex}.}\, ${\operatorname{det} \hspace{.16ex} C_{ij}^{\hspace{0.2ex}\scalebox{0.66}[0.66]{(0)}} \hspace{-0.25ex} = 0}$

...



\end{otherlanguage}

\en{\section{Poincar\'{e} method}} % Jules Henri Poincaré

\ru{\section{Метод Пуанкаре}}

% https://en.wikipedia.org/wiki/Henri_Poincar%C3%A9

\begin{otherlanguage}{russian}

Этот метод широко известен в~теории нелинейных колебаний. Он предназначен, в~частности, для~определения периодических решений уравнения
\begin{equation}\label{nonlinearvibrations:famousequation}
\mathdotdotabove{u} + u = \smallparameter \hspace{.1ex} f(u, \mathdotabove{u})
\end{equation}

...



\end{otherlanguage}

\en{\section{Van der Pol averaging method}} % Balthasar van der Pol

\ru{\section{Метод осреднения Ван дер Поля}}

% https://en.wikipedia.org/wiki/Balthasar_van_der_Pol

\begin{otherlanguage}{russian}

Опять рассматривается уравнение~\eqref{nonlinearvibrations:famousequation}, но~теперь ищутся не~только периодические решения. Вводится фазовая плоскость

...



%%\end{otherlanguage}

\section{Метод сращивания асимптотических разложений}

Основоположник этого метода~--- Ludwig Prandtl. Рассматривая течение вязкой жидкости, он заметил, что влияние малой вязкости локализовано у~границы~--- в~тонком приграничном слое. Вдали от~границы жидкость ведёт себя как идеальная. Одни и~те~же уравнения

...




\section{Метод многих масштабов}

Этот метод привлекателен, естественен и~--- как написано

...




\section{Уравнения с медленно меняющимися коэффициентами}

Рассмотрим гармонический осциллятор, собственная частота которого медленно меняется во~времени

...



\end{otherlanguage}

\en{\section{Thin bodies}}

\ru{\section{Тонкие тела}}

\begin{otherlanguage}{russian}

Задачи теории упругости часто ставятся для тонких тел~--- стержней, пластин и~оболочек. Таковы многие элементы конструкций, но и в~природе вне~человека тонкие тела встречаются довольно часто.

Решение задач упругости для~тонких тел многие десятилетия основывалось на~неких гипотезах о~распределении решения по~толщине и~о~порядках одних неизвестных относительно других. Построенные так теории сыграли больш\'{у}ю роль в~практике инженерных расчётов. Однако, им не~хватало логической стройности и~убедительности, их хотелось обосновать, уточнить~--- а~в~последнее время и~уничтожить (в~связ\'{и} с~появлением великолепных компьютеров). Но открытое \textcolor{magenta}{не~так давн\'{о}} явление асимптотического ращепления прояснило картину: в тонком теле трёхмерная задача ращепляется на задачи меньшей размерности. Классические теории тонких тел получили и~подтверждение, и~развитие.

Рассмотрим задачу о~кручении из~...

...





\vspace{8mm}
\hfill\begin{minipage}[b]{0.95\linewidth}
\fontsize{10}{12}\selectfont

\section*{\wordforbibliography}

Всё~разнообразие асимптотических методов представлено в~монографиях ...

\end{minipage}

\end{otherlanguage}
