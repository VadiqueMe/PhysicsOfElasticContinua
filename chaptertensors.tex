\en{\chapter{Elements of tensor calculus}}

\ru{\chapter{Элементы тензорного исчисления}}

\thispagestyle{empty}

\label{chapter:elementsoftensorcalculus}

\en{\section{Vectors}}

\ru{\section{Векторы}}

\label{para:vectors}

\en{\lettrine[lines=2, findent=2pt, nindent=0pt]{M}{ention} of~tensors may scare away the reader, commonly avoiding needless complications. Don’t be afraid: tensors are introduced just because of their wonderful property of being invariant\:--- independent of coordinate systems. I propose to~begin familiarizing with tensors via reminiscences about such a~phenomenon as a~vector.}

\ru{\lettrine[lines=2, findent=2pt, nindent=0pt]{У}{поминание} тензоров может отпугнуть читателя, обычно избегающего ненужных сложностей. Не бойся: тензоры вводятся лишь из\hbox{-}за своего чудесного свойства инвариантности\:--- независимости от систем координат. Предлагаю начать знакомство с~тензорами через воспоминания о~таком феномене как вектор.}

\en{Here’s}\ru{Вот} \en{vector}\ru{вектор}~$\bm{v}$, \en{it is pretty well characterized by its length~(modulus, magnitude) and direction in space}\ru{он вполне характеризуется своей длиной~(модулем) и~направлением в~пространстве}, \en{but}\ru{но} \en{does not depend on}\ru{не~зависит от} \en{methods}\ru{методов} \en{and}\ru{и}~\en{units}\ru{единиц} \en{of~measurements}\ru{измерений} \en{of neither lengths nor directions}\ru{ни~длин, ни~направлений}.

\begin{wrapfigure}[12]{o}{.47\textwidth}
\makebox[.42\textwidth][c]{\begin{minipage}[t]{.47\textwidth}
\vspace{-1em}
\hspace{-2em}\scalebox{0.96}[0.96]{
\tdplotsetmaincoords{45}{125} % orientation of camera

% three parameters for vector
\pgfmathsetmacro{\lengthofvector}{2.5}
\pgfmathsetmacro{\anglefromz}{44}
\pgfmathsetmacro{\anglefromx}{70}

\begin{tikzpicture}[scale=2.5, tdplot_main_coords] % use 3dplot

	\coordinate (O) at (0,0,0);

	% draw axes
	\draw [line width=0.4pt, blue] (O) -- (1.22,0,0);
	\draw [line width=1.25pt, blue, -{Latex[round, length=3.6mm, width=2.4mm]}]
		(O) -- (1,0,0)
		node[pos=0.9, above, xshift=-0.6em, yshift=0.1em] {${\bm{e}}_1$};

	\draw [line width=0.4pt, blue] (O) -- (0,1.88,0);
	\draw [line width=1.25pt, blue, -{Latex[round, length=3.6mm, width=2.4mm]}]
		(O) -- (0,1,0)
		node[pos=0.9, above, xshift=0.2em, yshift=0.1em] {${\bm{e}}_2$};

	\draw [line width=0.4pt, blue] (O) -- (0,0,1.22);
	\draw [line width=1.25pt, blue, -{Latex[round, length=3.6mm, width=2.4mm]}]
		(O) -- (0,0,1)
		node[pos=0.9, below left, xshift=-0.1em, yshift=0.5em] {${\bm{e}}_3$};

	% draw vector
	\tdplotsetcoord{V}{\lengthofvector}{\anglefromz}{\anglefromx} % {length}{angle from z}{angle from x}
		% it also defines projections of point
	\draw [line width=1.6pt, black, -{Stealth[round, length=5mm, width=2.8mm]}]
		(O) -- (V)
		node[pos=0.8, above, yshift=0.2em] {\scalebox{1.2}[1.2]{${\bm{v}}$}};

	% draw components of vector
	%%\draw [line width=0.4pt, dotted, color=black] (O) -- (Vxy);
	\draw [line width=0.4pt, dotted, color=black] (Vxy) -- (Vy);

	\draw [color=black, line width=1.6pt, line cap=round, dash pattern=on 0pt off 1.6\pgflinewidth,
		-{Stealth[round, length=4mm, width=2.4mm]}]
		(O) -- (Vx)
		node[pos=0.5, above, xshift=-0.9em] {${v_1 \hspace{-0.1ex} \bm{e}_1}$};

	\draw [color=black, line width=1.6pt, line cap=round, dash pattern=on 0pt off 1.6\pgflinewidth,
		-{Stealth[round, length=4mm, width=2.4mm]}]
		(Vx) -- (Vxy)
		node[pos=0.53, below left, xshift=0.3em] {${v_2 \bm{e}_2}$};

	\draw [color=black, line width=1.6pt, line cap=round, dash pattern=on 0pt off 1.6\pgflinewidth,
		-{Stealth[round, length=4mm, width=2.4mm]}]
		(Vxy) -- (V)
		node[pos=0.5, above right, yshift=-0.2em] {${v_3 \bm{e}_3}$};

\end{tikzpicture}}
\vspace{-1.72em}\caption{}\label{fig:componentsofvector}
\end{minipage}}
\end{wrapfigure}

\en{Introduce cartesian coordinates}\ru{Введём декартовы координаты} \en{with mutually perpendicular unit vectors}\ru{со~взаимно перпендикулярными ортами} ${\bm{e}_1}$, ${\bm{e}_2}$, ${\bm{e}_3}$ \en{as~a~basis}\ru{в~\hbox{роли} базиса}.
\en{With such \hbox{a system}}\ru{С~такой \hbox{системой}}, \inquotes{${\dotp\hspace{.25ex}}$}\hbox{\hspace{-0.2ex}-}\en{products}\ru{произведения} \en{of~\hbox{basis} vectors}\ru{базисных векторов} \en{are equal to the~Kronecker~delta}\ru{равны \hbox{дельте}~Kronecker’а}:

\nopagebreak\vspace{-0.3em}\begin{equation*}
\bm{e}_i \dotp \hspace{.1ex} \bm{e}_j \hspace{-0.12ex}
= \hspace{.12ex} \delta_{i\hspace{-0.1ex}j}
= \hspace{.12ex} \scalebox{0.9}[0.9]{$\begin{cases}
1, \;i = j \\
0, \;i \neq j
\end{cases}$}
\end{equation*}

\en{Decomposing vector~$\bm{v}$ in basis}\ru{Разлагая вектор~${\bm{v}}$ в~базисе} ${\bm{e}_i}$~(${i = 1, 2, 3}$), \en{we get components of}\ru{получаем компоненты}~$\bm{v}$ \en{in that basis}\ru{в~том базисе}~(\figref{fig:componentsofvector})

\nopagebreak\vspace{-0.2em}\begin{equation}\label{vectorcomponents}
\bm{v} = \scalebox{0.9}[0.9]{$\displaystyle \sum_{i=1}^3$} \hspace{.25ex} {v_i \bm{e}_i} \hspace{-0.1ex} \equiv \hspace{.1ex} v_i \bm{e}_i
\hspace{.1ex} , \;\;
v_i \hspace{-0.1ex} = \bm{v} \dotp \bm{e}_i \hspace{-0.12ex} = |\bm{v}| \operatorname{cos} \measuredangle ({\bm{v}, \bm{e}_i})
\hspace{.1ex} .
\end{equation}

{\small\setlength{\abovedisplayskip}{1pt}\setlength{\belowdisplayskip}{1pt}

\en{Here and hereinafter, Einstein’s summation convention is accepted: having an~index repeated twice in a~single term implies summation over this index. Such an~index can’t be repeated more than twice in~the~same term. And a~non-repeating index is called \inquotesx{free}[,] it’s identical in both parts of~equation. These are examples:}

\ru{Здесь и~далее принимается соглашение о~суммировании~Einstein’а: наличие дважды повторённого индекса в~одночлене подразумевает суммирование по~этому индексу. Такой индекс не~может повторяться более чем дважды в~одном и~том~же одночлене. А~неповторяющийся индекс называется \inquotesx{свободным}[,] он одинаков в~обеих частях равенства. Вот примеры:}

\nopagebreak\vspace{-0.24em}\begin{equation*}
\sigma = \mathtau_{ii} \hspace{.1ex}, \;\;
p_j \hspace{-0.16ex} = n_i \mathtau_{i\hspace{-0.1ex}j} \hspace{.1ex}, \;\;
m_i \hspace{-0.16ex} = e_{i\hspace{-0.1ex}j\hspace{-0.1ex}k} \hspace{.16ex} x_j f_k \hspace{.1ex}, \;\;
a_i \hspace{-0.16ex} = \lambda b_i + \mu c_i
\hspace{.1ex} .
\end{equation*}

\vspace{-0.2ex} \noindent \en{But next equations are incorrect}\ru{Следующие~же равенства некорректны}

\nopagebreak\vspace{-0.24em}\begin{equation*}
a = b_{k\hspace{-0.1ex}k\hspace{-0.1ex}k} \hspace{.1ex}, \;\;
c = f_i + g_k \hspace{.1ex}, \;\;
a_{i\hspace{-0.1ex}j} \hspace{-0.25ex} = k_i \hspace{.2ex} \gamma_{i\hspace{-0.1ex}j}
\hspace{.1ex} .
\end{equation*}
\par}
\vspace{.1em}

\en{\begin{wrapfigure}[16]{o}{.5\textwidth}}\ru{\begin{wrapfigure}[17]{o}{.5\textwidth}}
\makebox[.48\textwidth][c]{\begin{minipage}[t]{.5\textwidth}
\vspace{-1em}
\hspace{-1.5em}\scalebox{0.96}[0.96]{
\tdplotsetmaincoords{35}{75} % orientation of camera

\begin{tikzpicture}[scale=3.2, tdplot_main_coords]

	\coordinate (O) at (0, 0, 0) ;

	% three coordinates of vector
	\pgfmathsetmacro{\Vx}{0.88}
	\pgfmathsetmacro{\Vy}{1.33}
	\pgfmathsetmacro{\Vz}{2.2}
	\coordinate (Vx) at (\Vx, 0, 0) ;
	\coordinate (Vy) at (0, \Vy, 0) ;
	\coordinate (Vz) at (0, 0, \Vz) ;
	\coordinate (V) at ($ (Vx) + (Vy) + (Vz) $) ;

	% draw vector
	\draw [line width=1.6pt, black, -{Stealth[round, length=5mm, width=2.8mm]}]
		(O) -- (V)
		node[pos=0.93, below right, inner sep=0pt, outer sep=4.4pt] {\scalebox{1.2}[1.2]{${\bm{v}}$}};

	\tdplotsetrotatedcoords{21}{-15}{-43} % 3-2-3 rotation sequence
	\begin{scope}[tdplot_rotated_coords]

	% draw axes

	\draw [line width=1.25pt, blue, -{Latex[round, length=3.6mm, width=2.4mm]}]
		(O) -- (1,0,0)
		node[pos=0.86, above left, inner sep=0pt, outer sep=2.5pt] {${\bm{e}}_1$};

	\draw [line width=1.25pt, blue, -{Latex[round, length=3.6mm, width=2.4mm]}]
		(O) -- (0,1,0)
		node[pos=0.9, above, inner sep=0pt, outer sep=5pt] {${\bm{e}}_3$};

	\draw [line width=1.25pt, blue, -{Latex[round, length=3.6mm, width=2.4mm]}]
		(O) -- (0,0,1)
		node[pos=0.92, below left, inner sep=0pt, outer sep=3.3pt] {${\bm{e}}_2$};

	%%\draw [line width=0.4pt, dotted, color=blue] (1,0,0) -- (0,1,0);
	%%\draw [line width=0.4pt, dotted, color=blue] (0,1,0) -- (0,0,1);
	%%\draw [line width=0.4pt, dotted, color=blue] (0,0,1) -- (1,0,0);

	\end{scope}

	% get projections of vector
	\tdplottransformmainrot{\Vx}{\Vy}{\Vz}
	\pgfmathsetmacro{\Vrotx}{\tdplotresx}
	\pgfmathsetmacro{\Vroty}{\tdplotresy}
	\pgfmathsetmacro{\Vrotz}{\tdplotresz}
	%%\draw [tdplot_rotated_coords, line width=1.6pt, blue, -{Stealth[round, length=5mm, width=2.8mm]}]
		%%(O) -- (\Vrotx, \Vroty, \Vrotz)
		%%node[pos=0.8, above left, inner sep=0pt, outer sep=2.5pt] {\scalebox{1.2}[1.2]{${\bm{v}}$}};

	% draw components of vector

	\draw [tdplot_rotated_coords, color=blue!50!black, line width=1.6pt, line cap=round, dash pattern=on 0pt off 1.6\pgflinewidth,
		-{Stealth[round, length=4mm, width=2.4mm]}]
		(O) -- (\Vrotx, 0, 0)
		node[pos=0.48, below right, inner sep=0pt, outer sep=2.5pt] {${v_1 \hspace{-0.1ex} \bm{e}_1}$};

	\draw [tdplot_rotated_coords, color=blue!50!black, line width=1.6pt, line cap=round, dash pattern=on 0pt off 1.6\pgflinewidth,
		-{Stealth[round, length=4mm, width=2.4mm]}]
		(\Vrotx, 0, 0) -- (\Vrotx, 0, \Vrotz)
		node[pos=0.47, below left, inner sep=0pt, outer sep=3.3pt] {${v_2 \bm{e}_2}$};

	\draw [tdplot_rotated_coords, color=blue!50!black, line width=1.6pt, line cap=round, dash pattern=on 0pt off 1.6\pgflinewidth,
		-{Stealth[round, length=4mm, width=2.4mm]}]
		(\Vrotx, 0, \Vrotz) -- (\Vrotx, \Vroty, \Vrotz)
		node[pos=0.5, above, inner sep=0pt, outer sep=3.5pt] {${v_3 \bm{e}_3}$};

	\tdplotsetrotatedcoords{2}{-8}{28} % 3-2-3 rotation sequence

	% draw axes

	\draw [tdplot_rotated_coords, line width=1.25pt, red, -{Latex[round, length=3.6mm, width=2.4mm]}]
		(O) -- (1,0,0)
		node[pos=0.86, above right, inner sep=0pt, outer sep=1.3pt] {${\bm{e}'\hspace{-0.6ex}}_{\raisemath{-0.15ex}{1}}$};

	\draw [tdplot_rotated_coords, line width=1.25pt, red, -{Latex[round, length=3.6mm, width=2.4mm]}]
		(O) -- (0,1,0)
		node[pos=0.9, above left, inner sep=0pt, outer sep=2.7pt] {${\bm{e}'\hspace{-0.6ex}}_{\raisemath{-0.15ex}{3}}$};

	\draw [tdplot_rotated_coords, line width=1.25pt, red, -{Latex[round, length=3.6mm, width=2.4mm]}]
		(O) -- (0,0,1)
		node[pos=0.98, below right, inner sep=0pt, outer sep=4.4pt] {${\bm{e}'\hspace{-0.6ex}}_{\raisemath{-0.15ex}{2}}$};

	%%\draw [tdplot_rotated_coords, line width=0.4pt, dotted, color=red] (1,0,0) -- (0,1,0);
	%%\draw [tdplot_rotated_coords, line width=0.4pt, dotted, color=red] (0,1,0) -- (0,0,1);
	%%\draw [tdplot_rotated_coords, line width=0.4pt, dotted, color=red] (0,0,1) -- (1,0,0);

	% get projections of vector
	\tdplottransformmainrot{\Vx}{\Vy}{\Vz}
	\pgfmathsetmacro{\Vrotx}{\tdplotresx}
	\pgfmathsetmacro{\Vroty}{\tdplotresy}
	\pgfmathsetmacro{\Vrotz}{\tdplotresz}

	% draw components of vector

	\draw [tdplot_rotated_coords, color=red!50!black, line width=1.6pt, line cap=round, dash pattern=on 0pt off 1.6\pgflinewidth,
		-{Stealth[round, length=4mm, width=2.4mm]}]
		(O) -- (\Vrotx, 0, 0)
		node[pos=0.81, below left, inner sep=0pt, outer sep=2pt] {${v'_{\raisemath{-0.15ex}{\hspace{-0.1ex}1}} \hspace{-0.1ex} \bm{e}'_{\raisemath{-0.15ex}{1}}}$};

	\draw [tdplot_rotated_coords, color=red!50!black, line width=1.6pt, line cap=round, dash pattern=on 0pt off 1.6\pgflinewidth,
		-{Stealth[round, length=4mm, width=2.4mm]}]
		(\Vrotx, 0, 0) -- (\Vrotx, 0, \Vrotz)
		node[pos=0.4, above right, inner sep=0pt, outer sep=3.3pt] {${v'_{\raisemath{-0.15ex}{\hspace{-0.1ex}2}} \bm{e}'_{\raisemath{-0.15ex}{2}}}$};

	\draw [tdplot_rotated_coords, color=red!50!black, line width=1.6pt, line cap=round, dash pattern=on 0pt off 1.6\pgflinewidth,
		-{Stealth[round, length=4mm, width=2.4mm]}]
		(\Vrotx, 0, \Vrotz) -- (\Vrotx, \Vroty, \Vrotz)
		node[pos=0.26, below right, inner sep=0pt, outer sep=1pt] {${v'_{\raisemath{-0.15ex}{\hspace{-0.1ex}3}} \bm{e}'_{\raisemath{-0.15ex}{3}}}$};

\end{tikzpicture}%
}
\en{\vspace{-2.1em}}\ru{\vspace{-1.8em}}\caption{}\label{fig:vectorintwoorthonormalbases}
\end{minipage}}
\end{wrapfigure}

\en{Vector}\ru{Вектор}~${\bm{v}}$ \en{is invariant}\ru{инвариантен}, \en{it doesn’t depend on any coordinate \hbox{system}}\ru{он не~зависит ни~от~какой \hbox{системы} координат}. \en{Decomposition in two cartesian systems with basis unit vectors}\ru{Разложение в~двух декартовых системах с~базисными ортами} ${\bm{e}_{i}}$ \en{and}\ru{и}~${\bm{e}'_{i}}$~(\figref{fig:vectorintwoorthonormalbases}) \en{gives}\ru{даёт}

\nopagebreak\vspace{-0.1em}\ru{\vspace{-0.7em}}\begin{equation*}
\bm{v} = v_{i} \hspace{.1ex} \bm{e}_{i} \hspace{-0.16ex} = v'_{\hspace{-0.1ex}i} \hspace{.1ex} \bm{e}'_{i}
\hspace{.1ex} ,
\end{equation*}

\vspace{-0.8em} \noindent \en{where}\ru{где}

\nopagebreak\vspace{-0.1em}\begin{equation*}
\begin{gathered}
\bm{v} \dotp \bm{e}_k \hspace{-0.15ex} = \hspace{-0.2ex} \tikzmark{beginComponentOfVector} \,v_k\, \tikzmark{endComponentOfVector} \hspace{-0.5ex} = v'_{\hspace{-0.1ex}i} \hspace{.2ex} \bm{e}'_{i} \hspace{-0.12ex} \dotp \bm{e}_k
\hspace{.1ex} ,
\\[1.1em]
%
\bm{v} \dotp \bm{e}'_{k} \hspace{-0.15ex} = \hspace{.12ex} v'_{\hspace{-0.1ex}k} \hspace{-0.1ex} = v_i \hspace{.2ex} \bm{e}_i \hspace{-0.12ex} \dotp \bm{e}'_{k} \hspace{.1ex}.
\end{gathered}
\end{equation*}
\AddUnderBrace[line width=.75pt][0,0][yshift=.12em]%
{beginComponentOfVector}{endComponentOfVector}%
{${\scriptstyle v_i \bm{e}_i \hspace{.2ex}\dotp\hspace{.32ex} \bm{e}_k \hspace{.2ex}=\hspace{.4ex} v_i \hspace{.1ex} \delta_{ik}}$}

\vspace{-0.5em} \en{Orthonormal basis vectors}\ru{Ортонормальные базисные векторы} \en{of an~}\inquotes{\en{old}\ru{старых}} \en{and}\ru{и}\en{~a}~\inquotesx{\en{new}\ru{новых}}[---] \en{rotated}\ru{повёрнутых}\:--- \en{cartesian coordiantes}\ru{декартовых координат} \en{are connected via rotation matrix}\ru{связываются через матрицу поворота} (\en{matrix of~\inquotes{direction} cosines}\ru{матрицу \inquotes{направляющих} косинусов}) ${\cosinematrix{i'\hspace{-0.1ex}k}\hspace{.1ex}}$:

\nopagebreak\vspace{-0.25em}\begin{equation}
\bm{e}'_{i} \hspace{-0.16ex} = \cosinematrix{i'\hspace{-0.1ex}k} \hspace{.1ex} \bm{e}_k \hspace{.1ex}, \;\:
(\hspace{.32ex} \dotp \hspace{.33ex} \bm{e}_k \hspace{.15ex})
\, \Rightarrow \hspace{.33ex}
\cosinematrix{i'\hspace{-0.1ex}k} \hspace{-0.16ex} = \bm{e}'_{i} \hspace{-0.15ex} \dotp \bm{e}_k \hspace{-0.16ex} = \operatorname{cos} \measuredangle (\bm{e}'_{i} \hspace{.1ex}, \bm{e}_k).
\end{equation}

\vspace{-0.1em} \en{Matrix of cosines is orthogonal, meaning that it inverses on~transposing}\ru{Матрица косинусов ортогональна, то~есть она обращается при~транспонировании}:

\en{\vspace{-1.2em}}\ru{\vspace{-0.6em}}\begin{equation}\label{orthogonalityofcosinematricies}
\cosinematrix{i'\hspace{-0.1ex}k} \hspace{.1ex} \cosinematrix{j'\hspace{-0.1ex}k} \hspace{-0.16ex} = \hspace{.2ex} \cosinematrix{k'\hspace{-0.1ex}i} \hspace{.1ex} \cosinematrix{k'\hspace{-0.1ex}j} \hspace{-0.16ex} = \hspace{.2ex} \delta_{i\hspace{-0.1ex}j}
\end{equation}

\vspace{-0.2em} \noindent (\en{proof}\ru{доказательство}:
$\delta_{i\hspace{-0.1ex}j} \hspace{-0.2ex}
= \bm{e}'_i \hspace{-0.12ex} \dotp \bm{e}'_j \hspace{-0.16ex}
= \cosinematrix{i'\hspace{-0.1ex}k} \hspace{.1ex} \bm{e}_k \hspace{-0.15ex} \dotp \cosinematrix{j'\hspace{-0.1ex}n} \hspace{.1ex} \bm{e}_n \hspace{-0.16ex}
= \cosinematrix{i'\hspace{-0.1ex}k} \hspace{.1ex} \cosinematrix{j'\hspace{-0.1ex}n} \delta_{kn} \hspace{-0.12ex}
= \cosinematrix{i'\hspace{-0.1ex}k} \hspace{.1ex} \cosinematrix{j'\hspace{-0.1ex}k} \hspace{0.1ex}$).

\en{Multiplying}\ru{Умножение} \en{an~orthogonal matrix}\ru{ортогональной матрицы} \en{by components of any vector}\ru{на~компоненты любого вектора} \en{retains}\ru{сохраняет} \en{the~length~(modulus)}\ru{длину~(модуль)} \en{of this vector}\ru{этого вектора}:

\nopagebreak \vspace{-0.25em} \begin{equation*}
|\bm{v}|^2 \hspace{-0.1ex} = \bm{v} \dotp \bm{v}
= v'_i \hspace{.2ex} v'_i \hspace{-0.16ex}
= \cosinematrix{i'\hspace{-0.1ex}k} \hspace{.1ex} v_k \hspace{.2ex} \cosinematrix{i'\hspace{-0.1ex}n} \hspace{.1ex} v_n \hspace{-0.16ex}
= v_n v_n
\end{equation*}

\vspace{-0.25em} \noindent
\en{--- this conclusion leans on}\ru{--- этот вывод опирается на}~\eqref{orthogonalityofcosinematricies}.

\vspace{.2em} \en{Orthogonal transformation of vector components}\ru{Ортогональное преобразование компонент вектора}

\nopagebreak \vspace{-0.3em}
\begin{equation}\label{orthotransform:1}
v'_i = v_k \hspace{.2ex} \bm{e}_k \hspace{-0.15ex} \dotp \bm{e}'_i \hspace{-0.12ex} = \cosinematrix{i'\hspace{-0.1ex}k} \hspace{.2ex} v_k
\end{equation}

\vspace{-0.2em}\noindent
\en{is sometimes used to define a~vector itself}\ru{иногда используется для определения самог\'{о} вектора}.
\en{Suppose that}\ru{Пусть}
\en{in each orthonormal basis}\ru{в~каждом ортонормированном базисе}~${\bm{e}_i}$ \ru{известна }\en{a~triplet of~numbers}\ru{тройка чисел}~${v_i}$\en{ is known},
\en{and}\ru{и} \en{with a~transition}\ru{с~переходом} \en{to a~new orthonormal basis}\ru{к~новому ортонормированному базису} \en{it’s transformed}\ru{она преобразуется} \en{according to}\ru{согласно}~\eqref{orthotransform:1};
\en{then}\ru{тогда} \en{this triplet of~components}\ru{эта тройка компонент} \en{represents}\ru{представляет} \en{an~invariant object}\ru{инвариантный объект}\:--- \en{vector}\ru{вектор}~${\bm{v}}$.

\en{\section{Tensor and its components}}

\ru{\section{Тензор и его компоненты}}

\label{para:tensoranditscomponents}

%% Conversion of tensor’s components from one orthonormal basis to another is done through an orthogonal transformation.

\en{\lettrine[lines=2, findent=2pt, nindent=0pt]{W}{hen} in each orthonormal basis~${\bm{e}_i}$ we have a~set of~nine (${3^2 \hspace{-0.24ex}=\hspace{-0.12ex} 9}$) numbers ${B_{i\hspace{-0.1ex}j}}$~(${i, j = 1, 2, 3}$), which is transformed during transition to~a~new~(rotated) orthonormal basis as}

\ru{\lettrine[lines=2, findent=2pt, nindent=0pt]{К}{огда} в~каждом ортонормальном базисе~${\bm{e}_i}$ имеем набор девяти (${3^2 \hspace{-0.24ex}=\hspace{-0.12ex} 9}$) чисел ${B_{i\hspace{-0.1ex}j}}$~(${i, j = 1, 2, 3}$), преобразующийся при~переходе к~новому~(повёрнутому) ортонормальному базису как}

\nopagebreak\vspace{-0.25em}\begin{equation}\label{orthotransform:2}
B\hspace{.16ex}'_{\hspace{-0.32ex}i\hspace{-0.1ex}j} \hspace{-0.16ex} = \cosinematrix{i'\hspace{-0.1ex}m} \hspace{.1ex} \cosinematrix{j'\hspace{-0.1ex}n} \hspace{.16ex} B_{mn}
\hspace{.1ex} ,
\end{equation}

\vspace{-0.2em} \noindent
\en{then this set of~components represents an~invariant object\:--- tensor of~second complexity (of~second valence, bivalent)}\ru{тогда этот набор компонент представляет инвариантный объект\:--- тензор второй сложности (второй валентности, бивалентный)}
${\hspace{-0.1ex}^2\!\bm{B}}$.

\en{In other words}\ru{Иными словами}, \en{tensor}\ru{тензор}~${^2\!\bm{B}}$ \en{reveals in every basis as a~collection of its components}\ru{проявляется в~каждом базисе совокупностью своих компонент}~${B_{i\hspace{-0.1ex}j}}$, \en{changing along with a~basis according to}\ru{меняющейся вместе с~базисом согласно}~\eqref{orthotransform:2}.

% dyad

\en{The key example}\ru{Ключевой пример} \en{of second complexity tensor}\ru{тензора второй сложности}\en{ is}\ru{\:---} \en{a~dyad}\ru{диада}.
\en{Having two vectors}\ru{Имея два вектора} ${\bm{a} \hspace{-0.2ex} = \hspace{-0.2ex}  a_i \bm{e}_i}$ \en{and}\ru{и}~${\bm{b} \hspace{-0.2ex} = \hspace{-0.2ex} b_i \bm{e}_i}$,
\en{in each basis}\ru{в~каждом базисе}~${\bm{e}_i}$ \en{assume}\ru{пол\'{о}жим} ${d_{i\hspace{-0.1ex}j} \hspace{-0.2ex} \equiv a_i b_j}$.
\en{I\kern-0.12ext’s easy to~see}\ru{Легко увидеть} \en{how components}\ru{как компоненты}~${d_{i\hspace{-0.1ex}j}}$ \en{transform}\ru{преобразуются} \en{according to}\ru{согласно}~\eqref{orthotransform:2}:

\nopagebreak\vspace{-0.3em}\begin{equation*}
\scalebox{0.96}[1]{$
a'_i \hspace{-0.2ex} = \cosinematrix{i'\hspace{-0.1ex}m} a_m
, \;
b\kern+0.1ex'_{\hspace{-0.2ex}j} \hspace{-0.2ex} = \cosinematrix{j'\hspace{-0.1ex}n} b_n
\hspace{-0.2ex} \:\Rightarrow\: \hspace{-0.1ex}
d\kern+0.1ex'_{\hspace{-0.1ex}i\hspace{-0.1ex}j} \hspace{-0.2ex} = a'_i b\kern+0.1ex'_{\hspace{-0.2ex}j} \hspace{-0.2ex}
= \cosinematrix{i'\hspace{-0.1ex}m} a_m \cosinematrix{j'\hspace{-0.1ex}n} b_n \hspace{-0.2ex}
= \cosinematrix{i'\hspace{-0.1ex}m} \cosinematrix{j'\hspace{-0.1ex}n} d_{mn}
$} .
\end{equation*}

\vspace{-0.25em} \noindent \en{Resulting}\ru{Получающийся} \en{tensor}\ru{тензор}~${^2\hspace{-0.25ex}\bm{d}}$
\en{is called}\ru{называется} \ru{диадным произведением~(}\en{a~}dyadic product\ru{)} \en{or}\ru{или} \en{just}\ru{просто} \ru{диадой~(}dyad\ru{)}
\en{and}\ru{и} \en{is written as}\ru{пишется как}~${\bm{a} \hspace{-0.15ex} \otimes \hspace{-0.15ex} \bm{b}}$ \en{or}\ru{или}~${\bm{a} \bm{b}}$.
\en{I choose notation}\ru{Я выбираю запись} ${^2\hspace{-0.25ex}\bm{d} = \hspace{-0.1ex} \bm{a} \bm{b}}$ \en{without}\ru{без} \en{symbol}\ru{символа}~${\otimes}$.

% unit dyad

\en{Essential example}\ru{Существенный пример} \en{of a~bivalent tensor}\ru{двухвалентного тензора} \ru{есть}\en{is} \en{the~unit dyad}\ru{единичная диада}~(\en{other names}\ru{другие именования}\ru{\:---}\en{ are} \en{unit tensor}\ru{единичный тензор}, \en{identity tensor}\ru{тождественный тензор} \en{and}\ru{и} \en{metric tensor}\ru{метрический тензор}).
\en{Let}\ru{Пусть} \en{for any}\ru{для~любого} \en{orthonormal}\ru{отронормального}~(\ru{декартова, }cartesian) \en{basis}\ru{базиса}
${E_{i\hspace{-0.1ex}j} \hspace{-0.2ex} \equiv \hspace{.12ex} \bm{e}_i \dotp \bm{e}_{\hspace{-0.15ex}j} \hspace{-0.2ex} = \delta_{i\hspace{-0.1ex}j}}$.
\en{These are really components of~tensor}\ru{Это действительно компоненты тензора}, \eqref{orthotransform:2} \en{is actual}\ru{актуально}:
${E\kern+0.1ex'_{\hspace{-0.12ex}mn} \hspace{-0.25ex} = \cosinematrix{m'\hspace{-0.1ex}i} \cosinematrix{n'\hspace{-0.2ex}j} E_{i\hspace{-0.1ex}j} \hspace{-0.2ex} = \cosinematrix{m'\hspace{-0.1ex}i} \cosinematrix{n'\hspace{-0.1ex}i} \hspace{-0.2ex} = \delta_{mn}}$.
\en{I~write}\ru{Я~пишу} \en{this tensor}\ru{этот тензор} \en{as}\ru{как}~${\bm{E}}$
(\en{other popular choices}\ru{другие популярные варианты}\ru{\:---}\en{ are} $\bm{I}$ \en{and}\ru{и}~${\hspace{-0.1ex} ^2\hspace{-0.1ex}\bm{1}}$).

\en{Immutability of~components upon any rotation}\ru{Неизменяемость компонент при~любом повороте} \en{makes tensor}\ru{делает тензор}~${\bm{E}}$ \en{isotropic}\ru{изотропным}. \en{There are no non\hbox{-}null vectors with such property}\ru{Ненулевых векторов с~таким свойством нет} (\en{all components}\ru{все компоненты} \en{of null vector}\ru{нуль\hbox{-}вектора}~$\bm{0}$ \en{are zero}\ru{равны нулю} \en{in any basis}\ru{в~любом базисе}).

% linear mapping (linear transformation)

\en{The next example is related to a~linear transformation (linear mapping) of vectors}\ru{Следующий пример связан с~линей\-ным преобразованием (линей\-ным отображением) векторов}.
\en{If}\ru{Если} ${\bm{b} \hspace{-0.2ex} = \hspace{-0.2ex} b_i \bm{e}_i}$ \en{is}\ru{есть} \en{linear}\ru{линей\-ная} (\en{preserving}\ru{сохраняющая} \en{addition}\ru{сложение} \en{and}\ru{и} \en{multiplication by number}\ru{умножение на~число}) \en{function}\ru{функция} \en{of}\ru{от}~${\bm{a} \hspace{-0.2ex} = \hspace{-0.2ex} a_{\hspace{-0.15ex}j} \bm{e}_{\hspace{-0.2ex}j}}$, \en{then}\ru{то} ${b_i \hspace{-0.2ex} = c_{i\hspace{-0.15ex}j} a_{\hspace{-0.15ex}j}}$ \en{in every basis}\ru{в~каждом базисе}. \en{Transformation coefficients}\ru{Коэффициенты преобразования}~${c_{i\hspace{-0.1ex}j}}$ \en{alter when a~basis rotates}\ru{меняются, когда базис вращается}:

\nopagebreak\vspace{-0.3em}\begin{equation*}
\scalebox{0.96}[1]{$
b\hspace{.16ex}'_{\hspace{-0.16ex}i}
= c\hspace{.16ex}'_{\hspace{-0.16ex}i\hspace{-0.1ex}j} a'_{\hspace{-0.15ex}j}
= \cosinematrix{i'\hspace{-0.1ex}k} b_k
= \cosinematrix{i'\hspace{-0.1ex}k} c_{kn} a_n
, \;
a_n \hspace{-0.25ex} = \cosinematrix{j'\hspace{-0.1ex}n} a'_{\hspace{-0.15ex}j}
\:\Rightarrow\:
c\hspace{0.16ex}'_{\hspace{-0.16ex}i\hspace{-0.1ex}j} \hspace{-0.2ex}
= \cosinematrix{i'\hspace{-0.1ex}k} \cosinematrix{j'\hspace{-0.1ex}n} c_{kn}
$} .
\end{equation*}

\vspace{-0.25em} \noindent \en{I\kern-0.12ext turns out that}\ru{Оказывается,} \en{a~set of~matrices}\ru{множество матриц}~${c_{i\hspace{-0.1ex}j}}$, ${c\hspace{0.16ex}'_{\hspace{-0.16ex}i\hspace{-0.1ex}j}}$, \dots, \en{describing}\ru{описывающих} \en{the~same}\ru{одно и~то~же} \en{linear mapping}\ru{линейное отображение} ${\bm{a} \mapsto \bm{b}}$, \en{but in various bases}\ru{но в~разных базисах}, \en{represents}\ru{представляет} \en{a~single invariant object}\ru{один инвариантный объект}\;--- \en{tensor of second complexity}\ru{тензор второй сложности}~${\hspace{-0.1ex} ^2\hspace{-0.2ex}\bm{c}}$.
\en{And many}\ru{И~многие} \en{book authors}\ru{авторы книг} \en{introduce}\ru{вводят} \en{tensors}\ru{тензоры} \en{in that way}\ru{таким путём}, \en{by means of}\ru{посредством} \en{linear mappings}\ru{линейных отображений} (\en{linear transformations}\ru{линейных преобразований}).

% bilinear form

\en{And}\ru{И} \en{the~last example}\ru{последний пример}\en{ is}\ru{\:---} \en{about}\ru{о} \en{a~bilinear form}\ru{билинейной форме} ${\digamma \hspace{-0.2ex} = f_{\hspace{-0.1ex}i\hspace{-0.1ex}j} \hspace{.2ex} a_i b_{\hspace{-0.1ex}j}}$,
\en{where}\ru{где}
${f_{\hspace{-0.1ex}i\hspace{-0.1ex}j}}$\ru{\:---}\en{ are} \en{coefficients}\ru{коэффициенты},
${a_i}$ \en{and}\ru{и}~${b_{\hspace{-0.1ex}j}}$\ru{\:---}\en{ are} \en{components of vector arguments}\ru{компоненты векторных аргументов} ${\bm{a} \hspace{-0.2ex} = \hspace{-0.2ex} a_i \bm{e}_i}$ \en{and}\ru{и}~${\bm{b} \hspace{-0.2ex} = \hspace{-0.2ex} b_{\hspace{-0.1ex}j} \bm{e}_{\hspace{-0.1ex}j}}$.
\en{The~result}\ru{Результат}~${\digamma \hspace{-0.15ex}}$ \en{is invariant}\ru{инвариантен} (\en{independent from basis}\ru{независим от базиса}),
${\digamma' \hspace{-0.2ex} = f\hspace{.1ex}'_{\hspace{-0.5ex}i\hspace{-0.1ex}j} \hspace{.2ex} a'_i b\hspace{.1ex}'_{\hspace{-0.2ex}j} \hspace{-0.1ex} = \digamma \hspace{-0.2ex}}$,
\en{this gives transformation}\ru{это даёт преобразование}~\eqref{orthotransform:2} \en{for coefficients}\ru{для коэффициентов}~${f_{\hspace{-0.1ex}i\hspace{-0.1ex}j}}$:

\nopagebreak\vspace{-0.15em}\begin{equation*}
f\hspace{.1ex}'_{\hspace{-0.5ex}i\hspace{-0.1ex}j} \hspace{.2ex} a'_i b\hspace{.1ex}'_{\hspace{-0.2ex}j} \hspace{-0.1ex} = f_{\hspace{-0.1ex}mn} \hspace{.4ex} \tikzmark{beginComponentOfFirstVector} \hspace{-0.2ex} a_m \tikzmark{endComponentOfFirstVector} \hspace{.2ex} \tikzmark{beginComponentOfSecondVector} b_n \hspace{-0.2ex} \tikzmark{endComponentOfSecondVector}
\;\Rightarrow\;
f\hspace{.1ex}'_{\hspace{-0.5ex}i\hspace{-0.1ex}j} \hspace{-0.2ex} = \cosinematrix{i'\hspace{-0.1ex}m} \hspace{.1ex} \cosinematrix{j'\hspace{-0.1ex}n} \hspace{.16ex} f_{\hspace{-0.1ex}mn}
\hspace{.1ex} .
\end{equation*}
\AddUnderBrace[line width=.75pt][-0.1ex, -0.1ex][xshift=-0.44em, yshift=.1em]%
{beginComponentOfFirstVector}{endComponentOfFirstVector}{${\scriptstyle \cosinematrix{\hspace{-0.1ex}i'\hspace{-0.2ex}m} a'_i}$}
\AddUnderBrace[line width=.75pt][.1ex, -0.1ex][xshift=.44em, yshift=.1em]%
{beginComponentOfSecondVector}{endComponentOfSecondVector}{${\scriptstyle \cosinematrix{\hspace{-0.2ex}j'\hspace{-0.2ex}n} b\hspace{.1ex}'_{\hspace{-0.2ex}j}}$}

\vspace{-0.1em} \noindent \en{If}\ru{Если} ${f_{\hspace{-0.1ex}i\hspace{-0.1ex}j} \hspace{-0.2ex} = \delta_{i\hspace{-0.1ex}j}}$, \en{then}\ru{то} ${\digamma \hspace{-0.2ex} = \delta_{i\hspace{-0.1ex}j} \hspace{.1ex} a_i b_{\hspace{-0.1ex}j} \hspace{-0.2ex} = a_i b_i}$\:--- \en{scalar product}\ru{скалярное произведение} (\en{inner product}\ru{внутреннее произведение}, dot product) \en{of~two vectors}\ru{двух векторов}.
\en{When}\ru{Когда} \en{both arguments}\ru{оба агрумента} \en{are the~same}\ru{одинаковые}~(${\bm{b} = \bm{a}}$), \en{such}\ru{такой} \en{homogeneous polynomial}\ru{однородный многочлен~(полином)} \en{of~second degree}\ru{второй степени} (\en{quadratic}\ru{квадратный}) \en{of~vector components}\ru{от~компонент вектора} ${\digamma \hspace{-0.2ex} = f_{\hspace{-0.1ex}i\hspace{-0.1ex}j} \hspace{.2ex} a_i a_{\hspace{-0.15ex}j}}$ \en{is called}\ru{называется} \en{a~quadratic form}\ru{квадратичной формой}.

% more complex tensors

\en{Now for more complex tensors}\ru{Теперь для более сложных тензоров} (\en{of~valence larger than two}\ru{валентности больше двух}).
\en{Tensor of~third complexity}\ru{Тензор третьей сложности}\;${^3\hspace{-0.1ex}\bm{C}}$ \en{is represented by a~collection of}\ru{представляется совокупностью} ${3^3 \hspace{-0.25ex} = \hspace{-0.1ex} 27}$ \en{numbers}\ru{чисел} ${C_{i\hspace{-0.1ex}j\hspace{-0.1ex}k}}$, \en{transforming as}\ru{преобразующихся как}

\nopagebreak\en{\vspace{-0.5em}}\ru{\vspace{-0.1em}}\begin{equation}\label{orthotransform:3}
C\hspace{.16ex}'_{\hspace{-0.32ex}i\hspace{-0.1ex}j\hspace{-0.1ex}k} \hspace{-0.2ex}
= \cosinematrix{i'\hspace{-0.1ex}p} \hspace{.1ex} \cosinematrix{j'\hspace{-0.1ex}q} \hspace{.1ex} \cosinematrix{k'\hspace{-0.1ex}r} \hspace{.16ex} C_{pqr}
\hspace{.1ex} .
\end{equation}

% triad

\en{The primary example}\ru{Первичный пример}\ru{\:---}\en{ is} \en{a~triad}\ru{триада} \en{of~three vectors}\ru{от трёх векторов} ${\bm{a} \hspace{-0.2ex} = \hspace{-0.2ex} a_i \bm{e}_i}$, ${\bm{b} \hspace{-0.2ex} = \hspace{-0.2ex} b_{\hspace{-0.1ex}j} \bm{e}_{\hspace{-0.1ex}j}}$ \en{and}\ru{и}~${\bm{c} \hspace{-0.2ex} = \hspace{-0.2ex} c_k \bm{e}_k}$:

\nopagebreak\vspace{-0.8em}\begin{equation*}
t_{i\hspace{-0.1ex}j\hspace{-0.1ex}k} \hspace{-0.2ex} \equiv a_i b_{\hspace{-0.1ex}j} c_k
\:\Leftrightarrow\:
{^3\bm{t}} = \bm{a} \bm{b} \bm{c}
\hspace{.1ex} .
\end{equation*}

\en{I\kern-0.12ext is seen that}\ru{Видно, что} \en{orthogonal transformations}\ru{ортогональные преобразования}~\eqref{orthotransform:3} \en{and}\ru{и}~\eqref{orthotransform:2}\ru{\:---}\en{ are} \en{results of}\ru{результаты} \inquotes{\en{repeating}\ru{повторения}} \en{vector’s}\ru{векторного}~\eqref{orthotransform:1}.
\en{The~reader}\ru{Читатель} \en{will easily compose}\ru{легко сост\'{а}вит} \en{a~transformation of~components}\ru{преобразование компонент} \en{for}\ru{для} \en{tensor of any complexity}\ru{тензора любой сложности} \en{and}\ru{и} \en{will write}\ru{нап\'{и}шет} \en{a~corresponding polyad}\ru{соответствующую полиаду} \en{as~an~example}\ru{как~пример}.

% vectors

\en{Vectors}\ru{Векторы} \en{with transformation}\ru{с~пребразованием}~\eqref{orthotransform:1} \en{are}\ru{суть} \en{tensors}\ru{тензоры} \en{of first complexity}\ru{первой сложности}.

% scalars

\en{In~the~end}\ru{Напоследок} \en{consider}\ru{рассмотрим} \en{the~least complex objects}\ru{наименее сложные объекты}\:--- \en{scalars}\ru{скаляры}, \en{they are}\ru{они~же} \en{tensors}\ru{тензоры} \en{of~zeroth complexity}\ru{нулевой сложности}.
\en{A~scalar}\ru{Скаляр} \en{is}\ru{это} \en{a~single}\ru{одно} ${(3^0 \hspace{-0.24ex}=\hspace{-0.12ex} 1)}$ \en{number}\ru{число}, \en{which doesn’t depend on a~basis}\ru{которое не~зависит от~базиса}: \en{energy}\ru{энергия}, \en{mass}\ru{масса}, \en{temperature}\ru{температура} \en{et~al.}\ru{и~др.}
\en{But what are}\ru{Но что такое} \en{components}\ru{компоненты}, \en{for example}\ru{к~примеру}, \en{of~vector}\ru{вектора} ${\bm{v} = v_i \bm{e}_i}$, ${v_i = \bm{v} \dotp \bm{e}_i}$?
\en{If}\ru{Если} \en{not scalars}\ru{не~скаляры}, \en{then}\ru{то} \en{what}\ru{что}?
\en{Here could be no simple answer}\ru{Здесь не~может быть простого ответа}.
\en{In~each particular basis}\ru{В~каждом отдельном базисе}, ${\bm{e}_i}$\ru{\:---}\en{ are} \en{vectors}\ru{векторы} \en{and}\ru{и}~${v_i}$\ru{\:---}\en{ are} \en{scalars}\ru{скаляры}.

\en{\section{Operations with tensors}}

\ru{\section{Действия с тензорами}}

\label{para:operationswithtensors}

\en{There’re four of them}\ru{Этих действий четыре}.

% linear combination

\nopagebreak\en{The~first}\ru{Первое}\:--- \textbold{\en{linear combination}\ru{линейная комбинация}}\:--- \en{aggregates}\ru{объединяет} \en{addition}\ru{сложение} \en{and}\ru{и}~\en{multiplication by~number}\ru{умножение на~число}.
\en{Arguments of this operation and the~result}\ru{Аргументы этого действия и~результат}\en{ are}\ru{\:---} \en{of the~same complexity}\ru{одинаковой сложности}.
\en{For two tensors it looks like this}\ru{Для двух тензоров оно выглядит так}:

\nopagebreak\vspace{-0.2em}\begin{equation}\label{action:1}
\lambda a_{i\hspace{-0.1ex}j\ldots} \hspace{-0.2ex} + \hspace{.2ex} \mu b_{i\hspace{-0.1ex}j\ldots} \hspace{-0.2ex} = \hspace{.2ex} c_{i\hspace{-0.1ex}j\ldots} \hspace{-0.2ex}
\;\Leftrightarrow\;
\lambda \bm{a} + \mu \bm{b} = \bm{c}
\hspace{.2ex} .
\end{equation}

\vspace{-0.1em} \noindent \en{Here}\ru{Здесь}
$\lambda$ \en{and}\ru{и}~$\mu$~\en{are}\ru{---} \en{scalar coefficients}\ru{коэффициенты\hbox{-}скаляры};
$\bm{a}$, $\bm{b}$ \en{and}\ru{и}~$\bm{c}$~\en{are}\ru{---} \en{tensors}\ru{тензоры} \en{of the~same complexity}\ru{одной и~той~же сложности}.
\en{I\kern-0.12ext’s easy to show}\ru{Легко показать,} \en{that}\ru{что} \en{components}\ru{компоненты} \en{of~result}\ru{результата}~$\bm{c}$ \en{satisfy}\ru{удовлетворяют} \en{the~orthogonal transformation}\ru{ортогональному преобразованию} \en{like}\ru{вида}~\eqref{orthotransform:2}.

\inquotesx{\en{Decomposition of~a~vector in a~basis}\ru{Разложение вектора в~базисе}}[---] \en{representation}\ru{представление} \en{of~a~vector}\ru{вектора} \en{as the~sum}\ru{суммой}~${\bm{v} = v_i \bm{e}_i}$\:--- \en{is nothing else but}\ru{это не~что~иное как} \en{the~linear combination}\ru{линейная комбинация} \en{of~basis vectors}\ru{векторов базиса}~${\bm{e}_i}$ \en{with coefficients}\ru{с~коэффициентами}~${v_i}$.

% multiplication

\en{The~second operation}\ru{Второе действие} \en{is}\ru{это} \textbold{\en{multiplication~(tensor product, direct product)}\ru{умножение~(тензорное произведение, прямое произведение)}}.
\en{I\kern-0.12ext~takes arguments of~any complexities}\ru{Оно принимает аргументы любых сложностей}, \en{returning the~result of~cumulative complexity}\ru{возвращая результат суммарной сложности}.
\en{Examples}\ru{Примеры}:

\nopagebreak\vspace{-0.1em}\begin{equation}\label{action:2}
\begin{array}{rcl}
v_i a_{j\hspace{-0.1ex}k} \hspace{-0.16ex} = C_{i\hspace{-0.1ex}j\hspace{-0.1ex}k} & \Leftrightarrow & \bm{v} \, {^2\hspace{-0.2ex}\bm{a}} = {^3\hspace{-0.15ex}\bm{C}},
\\[.1em]
a_{i\hspace{-0.1ex}j} B_{abc} \hspace{-0.16ex} = D_{i\hspace{-0.1ex}jabc} & \Leftrightarrow & {^2\hspace{-0.2ex}\bm{a}} \hspace{.25ex} {^3\hspace{-0.33ex}\bm{B}} = {^5\hspace{-0.33ex}\bm{D}}.
\end{array}\hspace{1.5em}
\end{equation}

\vspace{-0.1em}
\en{Transformation of a~collection of result’s components}\ru{Преобразование совокупности компонент результата}, \en{such as}\ru{такой как} ${C_{i\hspace{-0.1ex}j\hspace{-0.1ex}k} \hspace{-0.16ex} = v_i a_{j\hspace{-0.1ex}k}}$, \en{during rotation of~basis}\ru{при повороте базиса}\en{ is}\ru{\:---} \en{orthogonal}\ru{ортогональное}, \en{similar to}\ru{подобное}~\eqref{orthotransform:3}, \en{thus}\ru{посему} \en{here’s no~doubt that this collection is a~set of tensor components}\ru{тут нет сомнений, что эта совокупность есть набор компонент тензора}.

\en{Primary}\ru{Первичный} \en{and}\ru{и} \en{already known}\ru{уж\'{е} знакомый} (\en{from}\ru{по}~\pararef{para:tensoranditscomponents}) \en{subtype of~multiplication}\ru{подвид умножения}\en{ is}\ru{\:---} \en{the~dyadic product of two vectors}\ru{диадное произведение двух векторов}
${^2\hspace{-0.25em}\bm{A} \hspace{-0.15ex} = \bm{b}\bm{c}}$.

% contraction

\en{The~third operation}\ru{Третье действие} \en{is called}\ru{называется} \textbold{\ru{свёрткой~(}contraction\ru{)}}.
\en{I\kern-0.12ext applies}\ru{Оно применяется} \en{to bivalent and more complex tensors}\ru{к~бивалентным и более сложным тензорам}.
\en{This operation acts upon a~single tensor}\ru{Это действие над одним тензором}, \en{without}\ru{без} \en{other}\ru{других} \inquotesx{\en{participants}\ru{участников}}[.]
\en{Roughly speaking}\ru{Грубо говоря}, \en{contracting a~tensor}\ru{свёртывание тензора} \en{is}\ru{есть} \en{summing of its components}\ru{суммирование его компонент} \en{over}\ru{по} \en{some}\ru{какой\hbox{-}либо} \en{pair of~indices}\ru{паре индексов}.
\en{As a~result,}\ru{В~результате} \en{tensor’s complexity}\ru{сложность тензора} \en{decreases by two}\ru{уменьшается на~два}.

\en{For trivalent tensor}\ru{Для трёхвалентного тензора}~${\hspace{-0.1ex} ^3\hspace{-0.33ex}\bm{D}}$\en{,} \ru{возможны }\en{three variants}\ru{три варианта} \en{of~contraction}\ru{свёртки}\en{ are possible}, \en{giving vectors}\ru{дающие векторы}~${\bm{a}}$, ${\bm{b}}$ \en{and}\ru{и}~${\bm{c}}$ \en{with components}\ru{с~компонентами}

\nopagebreak\vspace{-0.33em}\begin{equation}\label{action:3}
a_{i} = D_{kki} \hspace{.1ex} ,
\;\;
b_{i} = D_{kik} \hspace{.1ex} ,
\;\;
c_{i} = D_{ikk} \hspace{.16ex} .
\end{equation}

\vspace{-0.33em}\noindent \en{A~rotation of~basis}\ru{Поворот базиса}

\nopagebreak\vspace{-0.25em}\begin{equation*}
a'_{i} = D\hspace{.16ex}'_{\hspace{-0.32ex}kki} \hspace{-0.16ex}
= \tikzmark{BeginDeltaPQBrace} {\cosinematrix{k'\hspace{-0.1ex}p} \hspace{.1ex} \cosinematrix{k'\hspace{-0.1ex}q}} \tikzmark{EndDeltaPQBrace} \hspace{.1ex} \cosinematrix{i'\hspace{-0.1ex}r} \hspace{0.16ex} D_{pqr} \hspace{-0.16ex}
= \cosinematrix{i'\hspace{-0.1ex}r} \hspace{.16ex} D_{ppr} \hspace{-0.16ex}
= \cosinematrix{i'\hspace{-0.1ex}r} \hspace{.16ex} a_{r}
\end{equation*}
\AddUnderBrace[line width = .75pt][0, -0.22ex]{BeginDeltaPQBrace}{EndDeltaPQBrace}%
{${\scriptstyle \delta_{pq}}$}

\nopagebreak\vspace{-0.33em} \noindent \en{shows}\ru{проявляет} \inquotes{\en{the~tensorial nature}\ru{тензорную природу}} \en{of~the~result of~contraction}\ru{результата свёртки}.

\en{For a~tensor}\ru{Для~тензора} \en{of~second complexity}\ru{второй сложности}\en{,} \ru{возможен }\en{the~only one}\ru{лишь один} \en{variant}\ru{вариант} \en{of~contraction}\ru{свёртки}\en{ is possible}, \en{giving a~scalar}\ru{дающий скаляр}\ru{,} \en{called}\ru{называемый} \en{the~trace}\ru{следом~(trace)} \en{or}\ru{или} \en{the~first invariant}\ru{первым инвариантом}

\nopagebreak\en{\vspace{-0.25em}}\ru{\vspace{-0.8em}}\begin{equation*}
\bm{B}\tracedot \hspace{.25ex} \equiv \hspace{.3ex}
\operatorname{tr} \bm{B} \hspace{.15ex} \equiv \hspace{.4ex}
\mathrm{I}\hspace{.16ex}(\bm{B}) \hspace{-0.15ex}
= B_{kk}
\hspace{.1ex} .
\end{equation*}

\vspace{-0.16em} \en{The~trace}\ru{След} \en{of~the~unit tensor}\ru{единичного тензора} (\inquotes{\en{contraction of~the~Kronecker delta}\ru{свёртка дельты Kronecker’а}}) \en{is equal to}\ru{равен} \en{the~dimension of~space}\ru{размерности пространства}

\nopagebreak\vspace{-0.2em}\begin{equation*}
\operatorname{tr} \bm{E} = \hspace{-0.1ex} \bm{E}\tracedot = \delta_{kk} \hspace{-0.2ex} = \hspace{.1ex} \delta_{1\hspace{-0.12ex}1} \hspace{-0.2ex} + \delta_{22} \hspace{-0.2ex} + \delta_{33} \hspace{-0.1ex} = \hspace{.1ex} 3
\hspace{.1ex} .
\end{equation*}

\begin{otherlanguage}{russian}

% index juggling

\en{The~fourth operation}\ru{Четвёртое действие} \en{is also applicable}\ru{также примен\'{и}мо} \en{to a~single tensor}\ru{к~одному тензору} \en{of~second and larger complexities}\ru{второй и~б\'{о}льших сложностей}.
\en{It}\ru{Оно} \en{is named}\ru{именуется} \en{as}\ru{как} \textbold{\ru{перестановка индексов~(}index swap\ru{)}, \ru{жонглирование индексами~(}index juggling\ru{)}, \ru{транспонирование~(}transposing\ru{)}}.
Из компонент тензора образуется новая совокупность величин с~другой последовательностью индексов, результатом является тензор той~же сложности.
Из тензора~${^3\hspace{-0.16em}\bm{D}}$, например, могут быть получены тензоры ${^3\hspace{-0.28em}\bm{A}}$, ${^3\hspace{-0.15em}\bm{B}}$, ${^3\hspace{-0.05em}\bm{C}}$ с~компонентами

\nopagebreak\vspace{-0.16em}\begin{equation}\label{action:4}
\begin{array}{rcl}
{^3\hspace{-0.28em}\bm{A}} = {^3\hspace{-0.16em}\bm{D}}_{1 \scalebox{0.6}[0.8]{$\rightleftarrows$} 2}
& \!\Leftrightarrow\!\! &
A_{i\hspace{-0.1ex}j\hspace{-0.1ex}k} = D_{j\hspace{-0.06ex}ik} \hspace{.1ex},
\\
{^3\hspace{-0.15em}\bm{B}} = {^3\hspace{-0.16em}\bm{D}}_{1 \scalebox{0.6}[0.8]{$\rightleftarrows$} 3}
& \!\Leftrightarrow\!\! & B_{i\hspace{-0.1ex}j\hspace{-0.1ex}k} = D_{kj\hspace{-0.06ex}i} \hspace{.1ex},
\\
{^3\hspace{-0.05em}\bm{C}} = {^3\hspace{-0.16em}\bm{D}}_{2 \scalebox{0.6}[0.8]{$\rightleftarrows$} 3}
& \!\Leftrightarrow\!\! & C_{i\hspace{-0.1ex}j\hspace{-0.1ex}k} = D_{ikj} \hspace{.1ex}.
\end{array}
\end{equation}

\en{For a~bivalent tensor,}\ru{Для бивалентного тензора}
\en{the only one transposition is possible}\ru{возможно лишь одно транспонирование}:
${\bm{A}^{\hspace{-0.05em}\T} \hspace{-0.15ex} \equiv \bm{A}_{1 \scalebox{0.6}[0.8]{$\rightleftarrows$} 2} = \hspace{-0.1ex} \bm{B}
\hspace{.4ex}\Leftrightarrow\hspace{.25ex}
B_{i\hspace{-0.1ex}j} \hspace{-0.1ex} = A_{j\hspace{-0.06ex}i}}$.
\en{Obviously}\ru{Очевидно}
${\bigl( \hspace{-0.1ex} \bm{A}^{\hspace{-0.05em}\T} \hspace{.15ex} \bigr)^{\hspace{-0.25ex}\T} \hspace{-0.2ex} = \bm{A}}$.

В~диадном умножении векторов ${\bm{a} \bm{b} = \bm{b} \bm{a} ^{\hspace{-0.05em}\T}\hspace{-0.4ex}}$.

% combining operations

Представленные четыре действия комбинируются в~разных сочетаниях. Комбинация умножения и~свёртки\:--- dot product\:--- самая частая, в~инвариантной безындексной записи она обозначается точкой, показывающей свёртку по~соседним индексам:
\begin{equation}
a_i \hspace{-0.1ex} = B_{i\hspace{-0.1ex}j} c_j \,\Leftrightarrow\, \bm{a} = \bm{B} \dotp \bm{c} \hspace{.1ex}, \;\:
A_{i\hspace{-0.1ex}j} \hspace{-0.2ex} = B_{ik} C_{kj} \,\Leftrightarrow\, \bm{A} = \bm{B} \dotp \bm{C}.
\end{equation}

Определяющее свойство единичного тензора

\nopagebreak\vspace{-0.15em}\begin{equation}\label{identifyofidentitytensor}
{^\mathrm{n}\hspace{-0.2ex}\bm{a}} \dotp \bm{E} = \bm{E} \dotp \hspace{-0.15ex} {^\mathrm{n}\hspace{-0.2ex}\bm{a}} = {^\mathrm{n}\hspace{-0.2ex}\bm{a}} \;\:\:
\forall \, {^\mathrm{n}\hspace{-0.2ex}\bm{a}} \;\; \forall \hspace{.1ex} \mathrm{n \!>\! 0}
\hspace{.1ex}.
\end{equation}

В~коммутативном скалярном произведении двух векторов точка имеет тот~же смысл:

\nopagebreak\vspace{-0.2em}\begin{equation}
\bm{a} \dotp \bm{b}
= ( \bm{a} \bm{b} )\hspace{.1ex}\tracedot \hspace{-0.1ex}
= \hspace{.1ex} \operatorname{tr} \bm{a}\bm{b}
= a_i b_i \hspace{-0.2ex}
= b_i a_i \hspace{-0.2ex}
= \hspace{.1ex} \operatorname{tr} \bm{b}\bm{a}
= ( \bm{b} \bm{a} )\hspace{.1ex}\tracedot \hspace{-0.1ex}
= \bm{b} \dotp \bm{a}
\hspace{.1ex} .
\end{equation}

Для dot product’а тензоров второй сложности справедливо следующее

\nopagebreak\vspace{-0.8em}\begin{equation}\label{transposeofdotproduct}
\begin{array}{c}
\bm{B} \dotp \bm{Q} \hspace{.1ex} = \left({ \bm{Q}^{\hspace{-0.1ex}\T} \hspace{-0.2ex}\dotp \bm{B}^{\T} }\right)^{\hspace{-0.25ex}\T}
\\[.1em]
\left({ \bm{B} \dotp \bm{Q}}\right)^{\hspace{-0.1ex}\T} \hspace{-0.3ex} = \hspace{.2ex} \bm{Q}^{\hspace{-0.1ex}\T} \hspace{-0.2ex}\dotp \bm{B}^{\T}
\end{array}
\end{equation}

\vspace{-0.1em} \noindent (\hspace{.25ex}например, для~диад ${\bm{B} \hspace{-0.1ex} = \bm{b}\bm{d}}$ и~${\bm{Q} \hspace{-0.1ex} = \bm{p}\bm{q}}$

\nopagebreak\vspace{-0.2em}\begin{equation*}\begin{array}{r@{\hspace{.3em}}c@{\hspace{.3em}}l}
\left(\hspace{.1ex} \bm{b}\bm{d} \dotp \bm{p}\bm{q} \hspace{.1ex}\right)^{\hspace{-0.1ex}\T} & \hspace{-0.5ex} = & \hspace{-0.1ex} \bm{p}\bm{q}^{\T} \hspace{-0.25ex} \dotp \bm{b}\bm{d}^{\hspace{.25ex}\T}
\\[.2em]
d_i p_i \hspace{.25ex} \bm{b}\bm{q}^{\T} & \hspace{-0.5ex} = & \hspace{-0.1ex} \bm{q}\bm{p} \dotp \bm{d} \bm{b}
\\[.1em]
d_i p_i \hspace{.25ex} \bm{q}\bm{b} & = & p_i d_i \hspace{.25ex} \bm{q}\bm{b} \;\; ).
\end{array}\end{equation*}

\end{otherlanguage}

\en{For a~vector and a~bivalent tensor}\ru{Для вектора и~бивалентного тензора}

\nopagebreak\vspace{-0.2em}\begin{equation}\label{vectortensordotproduct}
\bm{c} \hspace{.1ex} \dotp \bm{B} = \bm{B}^{\T} \hspace{-0.4ex} \dotp \bm{c} , \;\;
\bm{B} \dotp \bm{c} = \bm{c} \hspace{.2ex} \dotp \bm{B}^{\T} \hspace{-0.3ex} .
\end{equation}

%%\en{Tensor of~second valence \inquotes{squared} is}\ru{Тензор второй валентности \inquotes{в~квадрате} это}

%%\nopagebreak\vspace{-0.2em}\begin{equation}\label{exponentiation:two}
%%\bm{B}^2 \equiv\hspace{.2ex} \bm{B} \hspace{-0.1ex} \dotp \hspace{-0.1ex} \bm{B} .
%%\end{equation}

\en{Contraction can be repeated}\ru{Свёртка может повторяться}:
${\bm{A} \dotdotp \hspace{-0.1ex} \bm{B} = A_{\hspace{.1ex}i\hspace{-0.1ex}j} B_{\hspace{-0.1ex}j\hspace{-0.06ex}i}}$,
\en{and here are useful equations}\ru{и~вот полезные равенства} \en{for second complexity tensors}\ru{для тензоров второй сложности}

\nopagebreak\vspace{-0.1em}\begin{equation}
\begin{array}{c}
\bm{A} \dotdotp \hspace{-0.1ex} \bm{B} = \bm{B} \dotdotp \hspace{-0.1ex} \bm{A}
\hspace{.12ex} ,
\;\;
\bm{d} \hspace{.1ex} \dotp \hspace{-0.15ex} \bm{A} \dotp \bm{b} = \hspace{-0.1ex} \bm{A} \dotdotp \hspace{.1ex} \bm{b}\bm{d} = \bm{b}\bm{d} \hspace{.1ex} \dotdotp \hspace{-0.1ex} \bm{A} = b_{\hspace{-0.1ex}j} d_{\hspace{.1ex}i} A_{\hspace{.1ex}i\hspace{-0.1ex}j} \hspace{.12ex} ,
\\[.2em]
%
\bm{A} \dotdotp \bm{B} = \bm{A}^{\hspace{-0.16ex}\T} \hspace{-0.2ex} \dotdotp \bm{B}^{\T} \hspace{-0.2ex} = A_{\hspace{.1ex}i\hspace{-0.1ex}j} B_{\hspace{-0.1ex}j\hspace{-0.06ex}i}
\hspace{.12ex} ,
\;\;
\bm{A} \dotdotp \bm{B}^{\T} \hspace{-0.2ex} = \bm{A}^{\hspace{-0.16ex}\T} \hspace{-0.2ex} \dotdotp \bm{B} = A_{\hspace{.1ex}i\hspace{-0.1ex}j} B_{i\hspace{-0.1ex}j}
\hspace{.12ex} ,
\\[.2em]
%
\bm{A} \hspace{-0.1ex} \dotdotp \hspace{-0.1ex} \bm{E} = \bm{E} \dotdotp \hspace{-0.1ex} \bm{A} = \bm{A}\hspace{.15ex}\tracedot = A_{j\hspace{-0.15ex}j} \hspace{.12ex} ,
\\[.2em]
%
\bm{A} \narrowdotp \bm{B} \narrowdotdotp \bm{E} = \hspace{-0.16ex} A_{\hspace{.1ex}i\hspace{-0.1ex}j} B_{\hspace{-0.1ex}j\hspace{-0.1ex}k} \hspace{.2ex} \delta_{ki} \hspace{-0.1ex} = \bm{A} \narrowdotdotp \bm{B} ,
\:\:
\bm{A} \narrowdotp \bm{A} \narrowdotdotp \bm{E} = \bm{A} \narrowdotdotp \bm{A}
\hspace{.12ex} ,
\\[.2em]
%
\bm{A} \narrowdotdotp \bm{B} \narrowdotp \bm{C} = \bm{A} \narrowdotp \bm{B} \narrowdotdotp \hspace{.1ex} \bm{C} = \hspace{.1ex} \bm{C} \narrowdotdotp \hspace{-0.1ex} \bm{A} \narrowdotp \bm{B} = A_{\hspace{.1ex}i\hspace{-0.1ex}j} B_{\hspace{-0.1ex}j\hspace{-0.1ex}k} C_{ki} \hspace{.1ex} ,
\\[.15em]
%
\bm{A} \hspace{-0.1ex}\narrowdotdotp \bm{B} \narrowdotp \bm{C} \narrowdotp \bm{D} = \bm{A} \narrowdotp \bm{B} \narrowdotdotp \bm{C} \narrowdotp \bm{D} = \bm{A} \narrowdotp \bm{B} \narrowdotp \bm{C} \narrowdotdotp \bm{D} = \hspace*{3em} \\
\hspace{11em} = \bm{D} \narrowdotdotp \bm{A} \narrowdotp \bm{B} \narrowdotp \bm{C} = A_{\hspace{.1ex}i\hspace{-0.1ex}j} B_{\hspace{-0.1ex}j\hspace{-0.1ex}k} C_{kh} D_{hi} \hspace{.1ex}.
\end{array}
\end{equation}

\en{\section{Polyadic representation (decomposition)}}

\ru{\section{Полиадное представление (разложение)}}

\label{para:polyadicrepresentation}

\en{In}\ru{В}~\pararef{para:tensoranditscomponents}\en{, a~tensor was presented as some invariant object, showing itself in every basis as a~collection of numbers~(components)}\ru{ тензор был представлен как некий инвариантный объект, проявляющий себя в~каждом базисе совокупностью чисел~(компонент)}.
\hbox{\en{Such}\ru{Такое}} \en{a~representation is typical for majority of~books about tensors}\ru{представление характерно для большинства книг о~тензорах}.
\en{Index notation}\ru{Индексная запись} \en{can be constructive}\ru{может быть конструктивна}, \en{especially when cartesian coordinates are sufficient}\ru{особенно когда достаточно декартовых координат}, \en{but quite often it is not}\ru{но весьма часто это не~так}.
\en{And}\ru{И} \en{the relevant case is}\ru{подходящий случай\:---} \en{physics of~elastic continua}\ru{физика упругих сред}: \en{it needs more elegant, more powerful and~perfect apparatus of the~direct tensor calculus, operating just with indexless invariant objects}\ru{ей нужен более изящн\hbox{ь\kern-0.066emi}й, более мощный и~совершенный аппарат прямого тензорного исчисления, оперирующий лишь с~безындексными инвариантными объектами}.

\en{Linear combination}\ru{Линейная комбинация} ${\bm{v} = v_i \bm{e}_i}$ \en{from}\ru{из}~\eqref{vectorcomponents} \en{connects}\ru{соединяет} \en{vector}\ru{вектор}~${\bm{v}}$ \en{with basis}\ru{с~базисом}~${\bm{e}_i}$ \en{and vector’s components}\ru{и~компонентами}~${v_i}$ \en{in that basis}\ru{вектора в~том базисе}.
\en{Soon we will get the~similar relation for a~tensor of any complexity}\ru{Вскоре мы получим похожее соотношение для~тензора любой сложности}.

\en{Any bivalent tensor}\ru{Любой бивалентный тензор} ${^2\!\bm{B}}$
\en{is described}\ru{описывается} \en{by nine components}\ru{девятью компонентами}~${B_{i\hspace{-0.1ex}j}}$ \en{in~each basis}\ru{в~каждом базисе}.
\en{The~number}\ru{Число} \en{of~various dyads}\ru{различных диад}~${\bm{e}_i \bm{e}_{\hspace{-0.15ex}j}}$ \en{in the~same basis}\ru{в~том~же с\'{а}мом базисе}\ru{\:---}\en{ is} \en{nine}\ru{тоже девять}~($3^2$)\en{ too}.
\en{Linear combining}\ru{Линейное комбинирование} \en{of these dyads}\ru{этих диад} \en{with coefficients}\ru{с~коэффициентами}~${B_{i\hspace{-0.1ex}j}}$
\en{gives the~sum}\ru{даёт сумму}~${B_{i\hspace{-0.1ex}j} \bm{e}_i \bm{e}_{\hspace{-0.15ex}j}}$.
\en{This is tensor}\ru{Это тензор},\,--- \en{but what are its components}\ru{но каков\'{ы} его компоненты}, \en{and}\ru{и} \en{how does this representation change (or doesn’t change)}\ru{как это представление меняется (или не~меняется)} \en{with a~rotation of~basis}\ru{с~поворотом базиса}?

\en{Components of the~constructed sum}\ru{Компоненты построенной суммы}

\nopagebreak\vspace{-0.2em}\begin{equation*}
\bigl( B_{i\hspace{-0.1ex}j} \hspace{.1ex} \bm{e}_i \bm{e}_{\hspace{-0.1ex}j} \bigr)_{\hspace{-0.2ex}p\hspace{.1ex}q} \hspace{-0.25ex} = \hspace{-0.1ex} B_{i\hspace{-0.1ex}j} \hspace{.1ex} \delta_{ip} \delta_{\hspace{-0.15ex}j\hspace{-0.1ex}q} \hspace{-0.1ex} = \hspace{-0.1ex} B_{pq}
\end{equation*}

\vspace{-0.2em} \noindent \en{are components}\ru{суть компоненты} \en{of~tensor}\ru{тензора}~${^2\!\bm{B}}$. \en{And with a~rotation}\ru{С~поворотом~же} \en{of~basis}\ru{базиса}

\nopagebreak\vspace{-0.2em}\begin{equation*}
\scalebox{0.98}[0.99]{$
B'_{i\hspace{-0.1ex}j} \bm{e}'_{i} \bm{e}'_{\hspace{-0.1ex}j} \hspace{-0.1ex}
= \cosinematrix{i'\hspace{-0.1ex}p} \cosinematrix{j'\hspace{-0.1ex}q} B_{pq} \cosinematrix{i'\hspace{-0.1ex}n} \bm{e}_n \cosinematrix{j'\hspace{-0.1ex}m} \bm{e}_m \hspace{-0.2ex}
= \delta_{\hspace{-0.1ex}pn} \delta_{\hspace{-0.1ex}qm} B_{pq} \bm{e}_n \bm{e}_m \hspace{-0.2ex}
= B_{pq} \bm{e}_p \bm{e}_q
$}
\hspace{.1ex} .
\end{equation*}

\begin{otherlanguage}{russian}

Сомнения отпали\:--- приходим к диадному представлению тензора второй сложности

\nopagebreak\vspace{-0.2em}\begin{equation}\label{polyada:2}
^2\!\bm{B} = B_{i\hspace{-0.1ex}j} \hspace{.1ex} \bm{e}_i \bm{e}_{\hspace{-0.1ex}j} \hspace{.1ex} .
\end{equation}

\vspace{-0.2em} Для единичной диады (единичного тензора) (that’s why it is called the~unit dyad)

\nopagebreak\vspace{-0.1em}\begin{equation*}
\bm{E} =
E_{i\hspace{-0.1ex}j} \hspace{.16ex} \bm{e}_i \bm{e}_{\hspace{-0.1ex}j} \hspace{-0.1ex} =
\delta_{i\hspace{-0.1ex}j} \hspace{.16ex} \bm{e}_i \bm{e}_{\hspace{-0.1ex}j} \hspace{-0.1ex} =
\bm{e}_i \bm{e}_i \hspace{-0.1ex} =
\bm{e}_1 \bm{e}_1 + \bm{e}_2 \bm{e}_2 + \bm{e}_3 \bm{e}_3
\hspace{.1ex} .
\end{equation*}

\en{Polyadic representations like}\ru{Полиадные представления типа}~\eqref{polyada:2} помогают проще и~с~б\'{о}льшим пониманием оперировать с~тензорами:

\nopagebreak\vspace{-0.1em}\begin{equation*}
\bm{v} \dotp {^2\!\bm{B}} =
v_i\bm{e}_i \hspace{-0.1ex} \dotp B_{\hspace{-0.2ex}j\hspace{-0.1ex}k}\bm{e}_{\hspace{-0.1ex}j} \bm{e}_k \hspace{-0.1ex} =
v_i B_{\hspace{-0.2ex}j\hspace{-0.1ex}k} \delta_{i\hspace{-0.1ex}j} \hspace{.1ex} \bm{e}_k \hspace{-0.1ex} =
v_i B_{ik} \bm{e}_k
\hspace{.1ex} ,
\end{equation*}\vspace{-1.25em}
\begin{equation}\label{tensorcomponents}
\bm{e}_i \hspace{-0.1ex} \dotp \hspace{-0.1ex} {^2\!\bm{B}} \dotp \bm{e}_{\hspace{-0.1ex}j} \hspace{-0.1ex} =
\bm{e}_i \hspace{-0.1ex} \dotp B_{pq}\bm{e}_p\bm{e}_q \hspace{-0.1ex} \dotp \bm{e}_{\hspace{-0.1ex}j} \hspace{-0.15ex} =
\hspace{-0.1ex} B_{pq} \hspace{.1ex} \delta_{\hspace{-0.1ex}ip} \delta_{\hspace{-0.1ex}qj} \hspace{-0.15ex} =
\hspace{-0.1ex} B_{i\hspace{-0.1ex}j} \hspace{-0.15ex} =
{^2\!\bm{B}} \hspace{-0.1ex} \dotdotp \bm{e}_{\hspace{-0.1ex}j} \bm{e}_i
\hspace{.1ex} .
\end{equation}

Последняя строчка весьма интересна: компоненты тензора выражены через сам тензор. Ортогональное преобразование компонент при~повороте базиса~\eqref{orthotransform:2} оказывается очевидным следствием~\eqref{tensorcomponents}.

Аналогичным образом по базисным полиадам разлагается тензор любой сложности. Например, для трёхвалентного тензора

\nopagebreak\begin{equation}\label{polyada:3}
\begin{array}{c}
{^3\!\,\bm{C}} = C_{i\hspace{-0.1ex}j\hspace{-0.1ex}k} \hspace{.2ex} \bm{e}_i \bm{e}_j \bm{e}_k \hspace{.1ex},
\\[.8ex]
C_{i\hspace{-0.1ex}j\hspace{-0.1ex}k} \hspace{-0.2ex} = {^3\!\,\bm{C}} \dotdotdotp \bm{e}_k \bm{e}_j \bm{e}_i = \bm{e}_i \dotp {^3\!\,\bm{C}} \dotdotp \bm{e}_k \bm{e}_j = \bm{e}_j \bm{e}_i \dotdotp {^3\!\,\bm{C}} \dotp \bm{e}_k \hspace{.1ex}.
\end{array}
\end{equation}

\en{Using}\ru{Используя} \en{decomposition into polyads}\ru{разложение по~полиадам}, легко увидеть справедливость свойства~\eqref{identifyofidentitytensor} \inquotes{единичности} тензора~$\bm{E}$:

\nopagebreak\vspace{-0.1em}\[\begin{array}{c}
{^\mathrm{n}\hspace{-0.2ex}\bm{a}} = a_{i\hspace{-0.1ex}j \ldots q} \hspace{.4ex} \bm{e}_i \hspace{.2ex} \bm{e}_j \ldots \bm{e}_q , \;\,  \bm{E} = \bm{e}_e \bm{e}_e
\\[.4em]
%
{^\mathrm{n}\hspace{-0.2ex}\bm{a}} \dotp \bm{E} = a_{i\hspace{-0.1ex}j \ldots q} \hspace{.4ex} \bm{e}_i \hspace{.2ex} \bm{e}_j \ldots \tikzmark{BeginDeltaEQBrace} {\bm{e}_q \hspace{-0.1ex} \dotp \hspace{.1ex} \bm{e}_e} \tikzmark{EndDeltaEQBrace} \bm{e}_e = a_{i\hspace{-0.1ex}j \ldots q} \hspace{.4ex} \bm{e}_i \hspace{.2ex} \bm{e}_j \ldots \bm{e}_q = {^\mathrm{n}\hspace{-0.2ex}\bm{a}} \hspace{.1ex},
\\[1.1em]
%
\bm{E} \dotp {^\mathrm{n}\hspace{-0.2ex}\bm{a}} = \bm{e}_e \bm{e}_e \hspace{-0.1ex} \dotp \hspace{.1ex}  a_{i\hspace{-0.1ex}j \ldots q} \hspace{.4ex} \bm{e}_i \hspace{.2ex} \bm{e}_j \ldots \bm{e}_q = a_{i\hspace{-0.1ex}j \ldots q} \hspace{.2ex} \delta_{ei} \hspace{.2ex} \bm{e}_e \hspace{.2ex} \bm{e}_j \ldots \bm{e}_q = {^\mathrm{n}\hspace{-0.2ex}\bm{a}}.
\end{array}\]
\AddUnderBrace[line width=.75pt]{BeginDeltaEQBrace}{EndDeltaEQBrace}%
{${\scriptstyle \delta_{eq}}$}

\vspace{-0.5em} \en{A~polyadic representation}\ru{Полиадное представление} \en{connects}\ru{соединяет} \en{direct and index notations}\ru{прямую и~индексную записи} \en{together}\ru{воедино}.
\en{I\kern-0.12ext’s not worth}\ru{Не~ст\'{о}ит} \en{contraposing}\ru{пр\'{о}тиво\-постав\-л\'{я}ть} \en{one another}\ru{одно другому}.
\en{The~direct notation}\ru{Прямая запись} \en{is compact, elegant}\ru{компактна, изящна}, \en{it}\ru{она} \en{much more than others}\ru{много больше других} \en{suits}\ru{подходит} \en{for final relations}\ru{для конечных соотношений}.
Но~и~индексная запись бывает удобна, например, при громоздких выкладках с~тензорами.

\end{otherlanguage}

\en{\section{Matrices, permutations and determinants}}

\ru{\section{Матрицы, перестановки и определители}}

\label{para:matrices+permutations+determinants}

\en{I\kern-0.12ext’s sometimes pretty convenient}\ru{Иногда весьма удобно} \en{to represent}\ru{представлять} \en{the~set of~components}\ru{набор компонент} \en{of~a~bivalent tensor}\ru{бивалентного тензора} \en{in some basis}\ru{в~некотором базисе} \en{as matrix}\ru{как матрицу}

\nopagebreak\begin{equation*}
B_{i\hspace{-0.1ex}j} \hspace{-0.1ex} = \hspace{-0.2ex} \scalebox{0.9}{$\left[
\begin{array}{ccc}
B_{1\hspace{-0.12ex}1} & B_{12} & B_{13} \\
B_{21} & B_{22} & B_{23} \\
B_{31} & B_{32} & B_{33}
\end{array}
\right]$}
\end{equation*}

...

\en{Widespread in literature}\ru{Широк\'{о} распространённая в~литературе} \en{substitution}\ru{подмена} \en{of~tensors}\ru{тензоров} \en{for matrices of~components}\ru{матрицами компонент} \en{leads to mistakes}\ru{ведёт к~ошибкам}\:--- \en{without tracking the~basis}\ru{без слежения за~базисом,} \en{to which these matrices correspond}\ru{которому эти матрицы соответствуют}.

...

\nopagebreak\begin{equation*}
\delta_{i\hspace{-0.1ex}j} \hspace{-0.1ex} = \hspace{-0.2ex} \scalebox{0.9}{$\left[
\begin{array}{ccc}
\delta_{1\hspace{-0.12ex}1} & \delta_{12} & \delta_{13} \\
\delta_{21} & \delta_{22} & \delta_{23} \\
\delta_{31} & \delta_{32} & \delta_{33}
\end{array}
\right]$} \hspace{-0.2ex} = \hspace{-0.2ex} \scalebox{0.9}{$\left[
\begin{array}{ccc}
1 & 0 & 0 \\
0 & 1 & 0 \\
0 & 0 & 1
\end{array}
\right]$}
\end{equation*}

...

\begin{otherlanguage}{russian}

... Вводится символ перестановки (permutation symbol) \hbox{O.\hspace{.12ex}Veblen’а}~${e_{i\hspace{-0.1ex}j\hspace{-0.1ex}k}}$\footnote{%
\en{The~number of~indices}\ru{Число индексов} \en{in}\ru{в}~$e$\hbox{\hspace{.16ex}-\hspace{.16ex}}\en{symbols}\ru{символах} \en{is equal to}\ru{равно} \en{the~dimension of~space}\ru{размерности пространства}.}

\nopagebreak\vspace{-0.3em}\begin{equation*}
e_{123} \hspace{-0.15ex}= e_{231} \hspace{-0.15ex}= e_{312} \hspace{-0.15ex}= 1
\hspace{.1ex} ,
\hspace{1.8ex}
e_{213} \hspace{-0.15ex}= e_{321} \hspace{-0.15ex}= e_{132} \hspace{-0.15ex}= -1
\hspace{.1ex} ,
\vspace{-0.25em}\end{equation*}

\nopagebreak \noindent остальные нули.
Символ~${e_{i\hspace{-0.1ex}j\hspace{-0.1ex}k}}$~(${\pm 1}$ или~$0$) меняет знак при~перестановке любых двух индексов, не~меняется при двойной~(\inquotes{круговой}) перестановке, а~при~совпадении какой\hbox{-}либо пары индексов обращается в~нуль.

%% \begin{comment} %%
\vspace{1mm}

\begin{tcolorbox}[enhanced, colback = green!16, before upper={\parindent2ex}, parbox = false]
\small%
\setlength{\abovedisplayskip}{2pt}\setlength{\belowdisplayskip}{2pt}%

\[
e_{i\hspace{-0.1ex}j\hspace{-0.1ex}k}
= \hspace{.2ex} \left\{ \hspace{-0.2em} \begin{array}{r@{\hspace{.6em}}l}
0 & \text{when there is a~repeated index in}~i\hspace{-0.1ex}j\hspace{-0.1ex}k \hspace{.1ex} , \\
+1 & \text{when an~even number of~permutations of}~i\hspace{-0.1ex}j\hspace{-0.1ex}k \text{ is present} \hspace{.1ex} , \\
-1 & \text{when an~odd number of~permutations of}~i\hspace{-0.1ex}j\hspace{-0.1ex}k \text{ is present} \hspace{.1ex} .
\end{array} \right.
\]
\end{tcolorbox}

\vspace{1mm}
%% \end{comment} %%

...

Используя $e$\hbox{\hspace{.16ex}-\hspace{.16ex}}\en{symbols}\ru{символы} Веблена, \en{the~determinant}\ru{определитель~(детерминант)} \en{of~a~matrix}\ru{матрицы} \en{is expressed as}\ru{выражается как}

\nopagebreak\vspace{-0.2em}\begin{equation}
e_{pqr} \hspace{.4ex} \underset{\raisebox{.12em}{\scalebox{0.7}{$m$,$n$}}}{\operatorname{det}} \, A_{mn} \hspace{-0.2ex}
= e_{i\hspace{-0.1ex}j\hspace{-0.1ex}k} \hspace{.1ex} A_{pi} A_{qj} A_{rk} \hspace{-0.2ex}
= e_{i\hspace{-0.1ex}j\hspace{-0.1ex}k} \hspace{.1ex} A_{ip} A_{j\hspace{-0.1ex}q} A_{kr}
\hspace{.1ex} .
\end{equation}

... for $123$

\nopagebreak\vspace{-0.2em}\begin{equation*}
\underset{\raisebox{.12em}{\scalebox{0.7}{$m$,$n$}}}{\operatorname{det}} \, A_{mn} \hspace{-0.2ex}
= e_{i\hspace{-0.1ex}j\hspace{-0.1ex}k} \hspace{.1ex} A_{1i} A_{2j} A_{3k} \hspace{-0.2ex}
= e_{i\hspace{-0.1ex}j\hspace{-0.1ex}k} \hspace{.1ex} A_{i1} A_{j2} A_{k3}
\end{equation*}

\vspace{-0.2em}\begin{multline*}
\underset{\raisebox{.12em}{\scalebox{0.7}{$m$,$n$}}}{\operatorname{det}} \, A_{mn} \hspace{-0.15ex}
= \hspace{.1ex} A_{1\hspace{-0.12ex}1} A_{22} A_{33} - A_{1\hspace{-0.12ex}1} A_{23} A_{32}
- A_{12} A_{21} A_{33} \: + \\[-0.6em]
+ A_{12} A_{23} A_{31}
+ A_{13} A_{21} A_{32} - A_{13} A_{22} A_{31}
\end{multline*}

...

\nopagebreak\vspace{-0.1em}\begin{equation*}
e_{pqr} \hspace{.4ex} \underset{\raisebox{.12em}{\scalebox{0.7}{$m$,$n$}}}{\operatorname{det}} \, A_{mn} \hspace{-0.2ex}
= \hspace{.1ex}
\operatorname{det}\hspace{-0.25ex} \scalebox{0.92}{$\left[ \hspace{-0.15ex} \begin{array}{ccc}
A_{1p} & A_{1q} & A_{1r} \\
A_{2p} & A_{2q} & A_{2r} \\
A_{3p} & A_{3q} & A_{3r}
\end{array} \hspace{-0.1ex} \right]$} \hspace{-0.2ex}
= \hspace{.1ex}
\operatorname{det}\hspace{-0.25ex} \scalebox{0.92}{$\left[ \hspace{-0.15ex} \begin{array}{ccc}
A_{p1} & A_{p2} & A_{p3} \\
A_{q1} & A_{q2} & A_{q3} \\
A_{r1} & A_{r2} & A_{r3}
\end{array} \hspace{-0.1ex} \right]$}
\end{equation*}

...

${
\underset{\raisebox{.12em}{\scalebox{0.7}{$i$,$\hspace{.15ex}j$}}}{\operatorname{det}} \hspace{.4ex} \delta_{i\hspace{-0.1ex}j} \hspace{-0.16ex} = 1
}$

...

\vspace{-0.2em} А~символ перестановки через детерминант\:--- как

\nopagebreak\vspace{-0.2em}\begin{equation*}
e_{pqr} \hspace{-0.2ex}
= e_{i\hspace{-0.1ex}j\hspace{-0.1ex}k} \hspace{.1ex} \delta_{pi} \delta_{\hspace{-0.1ex}qj} \delta_{rk} \hspace{-0.2ex}
= e_{i\hspace{-0.1ex}j\hspace{-0.1ex}k} \hspace{.1ex} \delta_{ip} \delta_{\hspace{-0.15ex}j\hspace{-0.1ex}q} \delta_{kr}
%%\hspace{.1ex} .
\end{equation*}

\nopagebreak\vspace{-0.1em}\begin{equation}
e_{pqr} \hspace{-0.1ex}
= \hspace{.1ex}
\operatorname{det}\hspace{-0.25ex} \scalebox{0.92}{$\left[ \begin{array}{ccc}
\delta_{1p} & \delta_{1q} & \delta_{1r} \\
\delta_{2p} & \delta_{2q} & \delta_{2r} \\
\delta_{3p} & \delta_{3q} & \delta_{3r}
\end{array} \hspace{-0.1ex} \right]$} \hspace{-0.2ex}
= \hspace{.1ex}
\operatorname{det}\hspace{-0.25ex} \scalebox{0.92}{$\left[ \begin{array}{ccc}
\delta_{p1} & \delta_{p2} & \delta_{p3} \\
\delta_{q1} & \delta_{q2} & \delta_{q3} \\
\delta_{r1} & \delta_{r2} & \delta_{r3}
\end{array} \hspace{-0.1ex} \right]$}
\hspace{-0.2ex} .
\end{equation}



...

Определитель не~чувствителен к~транспонированию:

\nopagebreak\vspace{-0.25em}\begin{equation*}
\underset{\raisebox{.15em}{\scalebox{0.7}{$i$,$\hspace{.15ex}j$}}}{\operatorname{det}} \, A_{i\hspace{-0.1ex}j} \hspace{-0.16ex}
= \hspace{.1ex} \underset{\raisebox{.15em}{\scalebox{0.7}{$i$,$\hspace{.15ex}j$}}}{\operatorname{det}} \, A_{j\hspace{-0.06ex}i} \hspace{-0.16ex}
= \hspace{.1ex} \underset{\raisebox{.15em}{\scalebox{0.7}{$j\hspace{-0.2ex}$,$\hspace{.1ex}i$}}}{\operatorname{det}} \, A_{i\hspace{-0.1ex}j}
\hspace{.1ex} .
\end{equation*}

...

\inquotes{Определитель произведения равен произведению определителей}

\nopagebreak\vspace{-0.2em}\begin{equation}
\underset{\raisebox{.15em}{\scalebox{0.7}{$i$,$k$}}}{\operatorname{det}} \, B_{ik} \hspace{.4ex} \underset{\raisebox{.15em}{\scalebox{0.7}{$k$,$\hspace{.15ex}j$}}}{\operatorname{det}} \, C_{kj} \hspace{-0.15ex}
= \hspace{.1ex} \underset{\raisebox{.15em}{\scalebox{0.7}{$i$,$\hspace{.15ex}j$}}}{\operatorname{det}} \, B_{ik} \hspace{.1ex} C_{kj}
\end{equation}

\[
e_{\hspace{-0.25ex}f\hspace{-0.2ex}gh} \hspace{.33ex} \underset{\raisebox{.15em}{\scalebox{0.7}{$m$,$n$}}}{\operatorname{det}} \, B_{ms} \hspace{.1ex} C_{sn} \hspace{-0.2ex}
= e_{pqr} \hspace{.1ex} B_{\hspace{-0.25ex}f\hspace{-0.1ex}\mathcolor{blue}{i}} C_{\mathcolor{blue}{i}p} \hspace{.1ex} B_{\hspace{-0.1ex}g\mathcolor{blue}{j}} C_{\hspace{-0.1ex}\mathcolor{blue}{j}\hspace{-0.1ex}q} \hspace{.1ex} B_{h\mathcolor{blue}{k}} C_{\mathcolor{blue}{k}r}
\hspace{-0.2ex}
\]

\[
e_{\hspace{-0.25ex}f\hspace{-0.2ex}gh} \hspace{.33ex} \underset{\raisebox{.15em}{\scalebox{0.7}{$m$,$s$}}}{\operatorname{det}} \, B_{ms} \hspace{-0.2ex}
= e_{i\hspace{-0.1ex}j\hspace{-0.1ex}k} \hspace{.1ex} B_{\hspace{-0.25ex}f\hspace{-0.1ex}i} B_{\hspace{-0.1ex}gj} B_{hk}
\hspace{-0.2ex}
\]

\[
e_{i\hspace{-0.1ex}j\hspace{-0.1ex}k} \hspace{.33ex} \underset{\raisebox{.15em}{\scalebox{0.7}{$s$,$n$}}}{\operatorname{det}} \, C_{sn} \hspace{-0.2ex}
= e_{pqr} \hspace{.1ex} C_{ip} C_{\hspace{-0.1ex}j\hspace{-0.1ex}q} C_{kr}
\hspace{-0.2ex}
\]

\[
e_{\hspace{-0.25ex}f\hspace{-0.2ex}gh} \hspace{.2ex} \mathcolor{black!66}{e_{i\hspace{-0.1ex}j\hspace{-0.1ex}k}} \hspace{.33ex} \underset{\raisebox{.15em}{\scalebox{0.7}{$m$,$s$}}}{\operatorname{det}} \, B_{ms} \hspace{.33ex} \underset{\raisebox{.15em}{\scalebox{0.7}{$s$,$n$}}}{\operatorname{det}} \, C_{sn} \hspace{-0.2ex}
= \mathcolor{black!66}{e_{i\hspace{-0.1ex}j\hspace{-0.1ex}k}} \hspace{.2ex} e_{pqr} \hspace{.1ex} B_{\hspace{-0.25ex}f\hspace{-0.1ex}i} B_{\hspace{-0.1ex}gj} B_{hk} \hspace{.1ex} C_{ip} C_{\hspace{-0.1ex}j\hspace{-0.1ex}q} C_{kr}
\hspace{-0.2ex}
\]

...

Определитель компонент \en{of~a~bivalent tensor}\ru{бивалентного тензора}\en{ is}\ru{\:---} \en{the~invariant}\ru{инвариант}, он не~меняется с~поворотом базиса

\nopagebreak\vspace{-0.2em}\begin{equation*}
A'_{i\hspace{-0.1ex}j} \hspace{-0.16ex} = \cosinematrix{i'\hspace{-0.1ex}m} \hspace{.1ex} \cosinematrix{j'\hspace{-0.1ex}n} \hspace{.16ex} A_{mn}
\end{equation*}

...

\end{otherlanguage}

\en{\section{Cross product and Levi\hbox{-}Civita tensor}}

\ru{\section{Векторное произведение и тензор Л\'{е}ви\hbox{-\hspace{-0.2ex}}Чив\'{и}ты}}

\label{para:crossproduct+levicivita}

\en{By habitual notions}\ru{По~привычным представлениям},
\en{a~\inquotes{cross product}}\ru{\inquotes{векторное произведение}}~(\inquotesx{\en{vector product}\ru{cross product}}[,] \en{sometimes}\ru{иногда} \inquotes{oriented area product})
\en{of two vectors}\ru{двух векторов}
\en{is a~vector}\ru{есть вектор},
\en{heading perpendicular to the~plane of~multipliers}\ru{направленный перпендикулярно плоскости сомножителей},
\en{whose length is equal to}\ru{длина которого равна} \en{the~area}\ru{пл\'{о}\-ща\-ди} \en{of~parallelogram}\ru{параллелограмма}, \en{spanned by}\ru{охватываемого} \en{multipliers}\ru{сомножителями}

\nopagebreak\en{\vspace{-0.1em}}\ru{\vspace{-0.88em}}\begin{equation*}
\left| \hspace{.33ex} \bm{a} \times \bm{b} \hspace{.4ex} \right|
= |\bm{a}| \hspace{.1em} |\bm{b}| \hspace{.1em} \operatorname{sin} \measuredangle (\bm{a}, \bm{b})
\hspace{.1ex} .
\end{equation*}

\begin{wrapfigure}[8]{R}{.32\textwidth}
\makebox[.36\textwidth][c]{\hspace{.5em}\begin{minipage}[t]{.36\textwidth}
\vspace{-1.2em}
\hspace{1.25em}\scalebox{0.9}[0.9]{%
\begin{tikzpicture}[scale=0.8]
	\draw [line width=2pt, black, -{Latex[length=5mm, width=2mm]}]
		(0,0) -- (2.4,-0.8)
		node[above=1.4mm, xshift=-1.8mm] {$\bm{a}$};
	\draw [line width=2pt, black, -{Latex[length=5mm, width=2mm]}]
		(0,0) -- (1.6,0.8)
		node[midway, above=0.8mm] {$\bm{b}$};
	\draw [line width=2pt,blue,-{Latex[length=5mm, width=2mm]}]
		(0,0) -- (0,-1.95);
	\draw [line width=2pt,blue,-{Latex[length=5mm, width=2mm]}]
		(0,0) -- (0,-2.2)
		node[pos=0.75, right, xshift=0.4mm] {$\bm{c}$};
	\draw [line width=2pt,blue,-{Latex[length=5mm, width=2mm]}]
		(0,0) -- (0,1.95);
	\draw [line width=2pt,blue,-{Latex[length=5mm, width=2mm]}]
		(0,0) -- (0,2.2)
		node[pos=0.8, right, xshift=0.4mm] {$\bm{c}$};
	\draw [line width=0.4pt, black!50] (2.4, -0.8) -- (4, 0);
	\draw [line width=0.4pt, black!50] (1.6, 0.8) -- (4, 0);
\end{tikzpicture}
}\vspace{-1.6em}\caption{}\label{fig:crossproduct}
\end{minipage}}
\end{wrapfigure}

\en{However}\ru{Однако}, \en{a~}cross product\ru{~(векторное произведение)}\en{ isn’t quite}\ru{\:--- не~вполн\'{е}} \en{a~vector}\ru{вектор}, \en{since}\ru{поскольку} \en{it is not completely invariant}\ru{оно не~полностью инвариантно}.

\en{To begin with}\ru{Для начала},
\en{a~cross product}\ru{векторное произведение}
\en{may be}\ru{может быть} \en{oriented}\ru{ориентировано}
\en{in~either of~two possible ways}\ru{любым из~двух возможных способов}~(\figref{fig:crossproduct}).

... \en{the~sequential triple of~vectors}\ru{последовательная тройка векторов} $\bm{a}$, $\bm{b}$ \en{and}\ru{и}~$\bm{c}$ \en{as from}\ru{как из} ${\bm{c} = \bm{a} \times \bm{b}}$ is what is going to be oriented

... when vectors are measured using some basis, \en{the~sequential triple of~basis vectors}\ru{последовательная тройка векторов базиса} $\bm{e}_1$, $\bm{e}_2$, $\bm{e}_3$ is what is going to be oriented

\en{one direction out of two}\ru{одно направление из двух} \en{is chosen as}\ru{выбирается как} \inquotes{\en{positive}\ru{положительное}}

%%\en{oriented}\ru{ориентирована}
%%\en{either}\ru{либо} \en{positively}\ru{положительно}, \en{either}\ru{либо} \en{negatively}\ru{отрицательно}

%%\inquotes{\en{right-handed or right-chiral}\ru{правая}}
%%\inquotes{\en{left-handed or left-chiral}\ru{левая}}

{\small%
The orientation of a~basis in the~space is the (kind of) asymmetry that makes a reflection impossible to replicate by means of a rotation in that space (without lifting to a~more\hbox{-}dimensional space).
It’s impossible to turn the left hand of a human figure into the right hand by applying a rotation, but it is possible to do so by reflecting the figure in a mirror.

An orientation of the~space is an arbitrary choice of which \textcolor{magenta}{ordered} bases are \inquotes{positively} oriented and which are \inquotes{negatively} oriented.

In the three-dimensional Euclidean space, the two possible basis orientations are commonly called right-handed (right-chiral) and left-handed (left-chiral).

... chiral (adjective) = asymmetric in such a way that the thing and its \en{mirror image}\ru{зеркальное отражение} are not \en{superimposable}\ru{совмещаются}, cannot be superposed on its mirror image by any combination of rotations and translations

%%хиральность chirality χειρομορφία (χειρ = hand = рука)

The sequence of vectors in a~basis is significant. Two bases with different sequences of the same vectors differ by some permutation.
\par}

\en{A~pseudo\-vector}\ru{Псевдо\-вектор}~(\en{axial}\ru{аксиальный} \en{vector}\ru{вектор})\en{ is}\ru{\:---} \en{a~vector\hbox{-}like}\ru{похожий на~вектор} \en{three\hbox{-}dimensional}\ru{трёх\-мерный} \en{object}\ru{объект,}
\en{that is invariant under any rotation}\ru{инвариантный при любом повороте},
\en{but}\ru{но} \en{under a~reflection}\ru{при отражении}
\en{it additionally flips its direction backwards}\ru{он дополнительно меняет своё направление на~обратное}.

{\small%
\begin{otherlanguage}{russian}
При отражении направление полностью инвариантного (полярного) вектора, как и все пространство, изменяется (в~общем случае).

Псевдовектор (аксиальный вектор), в~отличие от (полярного) вектора, не меняет при отражении компоненту, ортогональную плоскости отражения, и оказывается \inquotes{перевёрнутым} относительно полярных векторов и геометрии всего пространства. Это происходит потому, что при изменении знака каждого векторного произведения (что соответствует отражению) меняется знак/направление и каждого аксиального вектора.
\end{otherlanguage}%
\par}

\en{The~otherness}\ru{Инаковость} \en{of~pseudovectors}\ru{псевдовекторов} \en{narrows}\ru{сужает} \en{the~variety}\ru{разнообразие} \en{of~formulas}\ru{формул}:
\en{a~pseudovector is not additive with a~vector}\ru{псевдовектор не~складывается с~вектором}.
\en{Formula}\ru{Формула}
${\bm{v} = \bm{v}_{\raisemath{-0.1em}{0}} + \hspace{.15ex} \bm{\omega} \hspace{-0.15ex}\times\hspace{-0.15ex} \bm{r}}$
\en{for~an~absolutely rigid undeformable body’s kinematics}\ru{для кинематики абсолютно жёсткого недеформируемого тела}
\en{is correct}\ru{корректна}, \en{because}\ru{поскольку}
$\bm{\omega}$\en{ is pseudovector there}\ru{ там\:--- псевдовектор},
\en{and}\ru{и} \en{with the~cross product}\ru{с~векторным произведением} \en{two}\ru{два} \inquotes{\en{pseudo}\ru{псевдо}} \en{mutually compensate each other}\ru{взаимно компенсируют друг друга}.

\begin{otherlanguage}{russian}

{\small Поворот (вращение, rotation) не может поменять ориентацию тройки.}

\en{However}\ru{Впрочем}, \en{the~cross product}\ru{векторное произведение} \en{is not}\ru{не~есть} \en{another new}\ru{ещё одно новое}, \en{entirely distinct operation}\ru{полностью отдельное действие}.
Оно сводится к~четырём ранее описанным~(\pararef{para:operationswithtensors}) и~обобщается на~тензоры любой сложности. Покажем это.

{\small A nonsingular linear mapping is orientation-preserving if its determinant is positive.}

Для этого познакомимся с~волюметрическим тензором третьей сложности\:--- тензором Levi\hbox{-}Civita\hspace{-0.1ex}’ы

\nopagebreak\vspace{-0.15em}\begin{equation}\label{levicivitaintro}
\levicivitatensor = \levicivita_{i\hspace{-0.1ex}j\hspace{-0.1ex}k} \hspace{.2ex} \bm{e}_i \bm{e}_j \bm{e}_k \hspace{.1ex}, \:\:
\levicivita_{i\hspace{-0.1ex}j\hspace{-0.1ex}k} \hspace{-0.12ex} \equiv \hspace{.1ex} \bm{e}_i \hspace{-0.2ex} \times \hspace{-0.2ex} \bm{e}_j \hspace{-0.1ex} \dotp \hspace{.12ex} \bm{e}_k
\end{equation}

\nopagebreak \vspace{-0.1em} \noindent
\en{with components}\ru{с~компонентами}~${\levicivita_{i\hspace{-0.1ex}j\hspace{-0.1ex}k}}$\ru{,} \en{equal to}\ru{равными} \inquotes{\en{triple}\ru{трой\-ным}}~(\inquotesx{\en{mixed}\ru{смешан\-ным}}[,] \inquotes{\en{cross\hbox{-}dot}\ru{векторно\hbox{-}скалярным}})
\en{products}\ru{произведениям} \en{of~basis \hbox{vectors}}\ru{\hbox{векторов} базиса}.

\en{The~absolute value}\ru{Абсолютная величина}~(\en{modulus}\ru{модуль}) \en{of~each nonzero component}\ru{всякой ненулевой компоненты} \en{of~tensor}\ru{тензора}~$\levicivitatensor$
\en{is equal to}\ru{\hbox{равна}} \en{the~volume}\ru{объёму}~\!${\sqrt{\hspace{-0.36ex}\mathstrut{\textsl{g}}}}$ \en{of~a~parallelopiped}\ru{параллелепипеда}, \en{drew upon a~basis}\ru{натянутого на \hbox{базис}}.
\en{For basis}\ru{Для базиса}~${\bm{e}_i}$ \en{of mutually orthogonal one\hbox{-}unit long vectors}\ru{из взаимно ортогональных векторов единичной длины}
${\hspace{-0.25ex}\sqrt{\hspace{-0.36ex}\mathstrut{\textsl{g}\hspace{.12ex}}} = \hspace{-0.1ex} 1}$.

Тензор~$\levicivitatensor$ изотропен, его компоненты не~меняются при~\hbox{любом} повороте базиса.
Но отражение\:--- изменение ориентации тройки базисных векторов~(перемена \inquotes{направления винта})\:---
меняет знак~$\levicivitatensor$, и~поэтому он является псевдотензором.

Если ${\bm{e}_1 \hspace{-0.2ex} \times \bm{e}_2 \hspace{-0.15ex} = \bm{e}_3}$,
то ${\bm{e}_i}$\:--- произвольно выбираемая из двух вариантов (\figref{fig:crossproduct}) как ориентированная положительно (\inquotes{правая}) \hbox{тройка},
и~компоненты~$\levicivitatensor$ равны символу Веблена:
${\levicivita_{i\hspace{-0.1ex}j\hspace{-0.1ex}k} \hspace{-0.1ex} = \hspace{.1ex} e_{i\hspace{-0.1ex}j\hspace{-0.1ex}k}}$.
Когда~же ${\bm{e}_1 \hspace{-0.2ex} \times \bm{e}_2 \hspace{-0.15ex} = - \hspace{.25ex} \bm{e}_3}$, тогда тройка~${\bm{e}_i}$ ориентирована отрицательно (\inquotesx{левая}[,] \inquotes{зеркальная}),
для неё ${\levicivita_{i\hspace{-0.1ex}j\hspace{-0.1ex}k} \hspace{-0.1ex} = - \hspace{.2ex} e_{i\hspace{-0.1ex}j\hspace{-0.1ex}k}}$
(а~${e_{i\hspace{-0.1ex}j\hspace{-0.1ex}k} \hspace{-0.12ex} = - \hspace{.33ex} \bm{e}_i \hspace{-0.2ex} \times \hspace{-0.2ex} \bm{e}_j \hspace{-0.1ex} \dotp \hspace{.12ex} \bm{e}_k}$).

С~тензором Л\'{е}ви\hbox{-\!}Чив\'{и}ты~$\levicivitatensor$ возможно по\hbox{-}новому взглянуть на~векторное произведение:

\nopagebreak\vspace{-0.3em}\begin{equation*}
\levicivita_{i\hspace{-0.1ex}j\hspace{-0.1ex}k} \hspace{-0.15ex} = \hspace{.15ex}
\hspace{.1ex} \bm{e}_i \hspace{-0.2ex} \times \hspace{-0.2ex} \bm{e}_j \hspace{-0.1ex} \dotp \hspace{.12ex} \bm{e}_k
\:\Leftrightarrow\:
\bm{e}_i \hspace{-0.1ex} \times \bm{e}_j \hspace{-0.12ex}
= \levicivita_{i\hspace{-0.1ex}j\hspace{-0.1ex}k} \hspace{.2ex} \bm{e}_k
\hspace{.1ex} ,
\end{equation*}\vspace{-1.6em}
\begin{multline}
\bm{a} \mathcolor{blue}{\times} \bm{b} \hspace{.2ex}
= \hspace{.1ex} a_i \bm{e}_i \times b_j \bm{e}_j
= \hspace{.1ex} a_i \hspace{.1ex} b_j \bm{e}_i \hspace{-0.2ex} \times \hspace{-0.2ex} \bm{e}_j
= \hspace{.1ex} a_i \hspace{.1ex} b_j \hspace{.1ex} \levicivita_{i\hspace{-0.1ex}j\hspace{-0.1ex}k} \hspace{.2ex} \bm{e}_k =
\\[-0.12em]
\shoveright{= \hspace{.2ex} b_j \hspace{.1ex} a_i \hspace{.1ex} \bm{e}_j \bm{e}_i \dotdotp \levicivita_{mnk} \hspace{.2ex} \bm{e}_m \bm{e}_n \bm{e}_k
= \hspace{.2ex} \bm{b} \hspace{.1ex} \bm{a} \hspace{.1ex} \dotdotp \levicivitatensor \hspace{.1ex} ,}
\\[-0.12em]
= \hspace{.1ex} a_i \hspace{.1ex} \levicivita_{i\hspace{-0.1ex}j\hspace{-0.1ex}k} \hspace{.2ex} \bm{e}_k \hspace{.1ex} b_j
= - \hspace{.2ex} a_i \hspace{.1ex} \levicivita_{ikj} \hspace{.2ex} \bm{e}_k \hspace{.1ex} b_j
= \mathcolor{blue}{-} \hspace{.2ex} \bm{a} \hspace{.1ex} \mathcolor{blue}{\dotp \levicivitatensor \dotp} \hspace{.12ex} \bm{b}
\hspace{.1ex} .
\end{multline}

\vspace{-0.12em} \noindent \inquotes{Векторное произведение} это всего лишь dot product\:--- комбинация умножения и~свёртки, \pararef{para:operationswithtensors}\:--- с~участием тензора~$\levicivitatensor$.

Такие комбинации возможны с~любыми тензорами:

\nopagebreak\vspace{-0.1em}\begin{equation*}
\begin{array}{c}
\bm{a} \hspace{.1ex} \mathcolor{blue}{\times} {^2\hspace{-0.16em}\bm{B}} = a_i \bm{e}_i \times B_{\hspace{-0.1ex}j\hspace{-0.1ex}k} \bm{e}_j \bm{e}_k \hspace{-0.1ex} = \tikzmark{BeginVectorCrossTensor} a_i B_{\hspace{-0.1ex}j\hspace{-0.1ex}k} \hspace{.1ex} \levicivita_{i\hspace{-0.1ex}jn} \tikzmark{EndVectorCrossTensor} \bm{e}_n \bm{e}_k = \mathcolor{blue}{-} \hspace{.2ex} \bm{a} \hspace{.15ex} \mathcolor{blue}{\dotp \levicivitatensor \dotp} {^2\!\bm{B}} ,
\\[1.6em]
%
{^2\hspace{-0.05ex}\bm{C}} \mathcolor{blue}{\times} \hspace{-0.1ex} \bm{d} \bm{b} = C_{i\hspace{-0.1ex}j} \bm{e}_i \bm{e}_j \hspace{-0.1ex} \times d_p b_q \bm{e}_p \bm{e}_q \hspace{-0.1ex} = \bm{e}_i C_{i\hspace{-0.1ex}j} d_p \tikzmark{BeginTensorCrossTensor} \levicivita_{j\hspace{-0.1ex}pk} \tikzmark{EndTensorCrossTensor} \bm{e}_k b_q \bm{e}_q =
\\[1.6em]
\hspace{12.5em} =
- \hspace{.2ex} {^2\hspace{-0.05ex}\bm{C}} \hspace{-0.1ex} \bm{d} \hspace{.12ex} \dotdotp \levicivitatensor \hspace{.1ex} \bm{b} \hspace{.2ex} =
\mathcolor{blue}{-} \hspace{.2ex} {^2\hspace{-0.05ex}\bm{C}} \mathcolor{blue}{\dotp \levicivitatensor \dotp} \bm{d} \bm{b},\\
\end{array}
\end{equation*}%
\AddUnderBrace[line width=.75pt][0,-0.2ex]%
{BeginVectorCrossTensor}{EndVectorCrossTensor}{${\scriptstyle - a_i \levicivita_{in\hspace{-0.1ex}j} B_{\hspace{-0.1ex}j\hspace{-0.1ex}k}}$}%
\AddUnderBrace[line width=.75pt][.2ex,-0.2ex]%
{BeginTensorCrossTensor}{EndTensorCrossTensor}{${\scriptstyle \;- \levicivita_{pj\hspace{-0.1ex}k} \:=\: - \levicivita_{j\hspace{-0.1ex}kp}}$}%
%
\vspace{-0.32em}\begin{equation}\label{iso:twothree}
\bm{E} \times \hspace{-0.16ex} \bm{E} = \bm{e}_i \bm{e}_i \times \bm{e}_j \bm{e}_j = \hspace{-0.4ex} \tikzmark{BeginECrossE} - \hspace{-0.2ex} \levicivita_{i\hspace{-0.1ex}j\hspace{-0.1ex}k} \bm{e}_i \bm{e}_j \bm{e}_k \tikzmark{EndECrossE} = - \hspace{.2ex} \levicivitatensor \hspace{0.1ex}.
\end{equation}
%
\AddUnderBrace[line width=.75pt][.2ex,-0.2ex]%
{BeginECrossE}{EndECrossE}{${\scriptstyle \;\;+\levicivita_{i\hspace{-0.1ex}j\hspace{-0.1ex}k} \bm{e}_i \bm{e}_k \bm{e}_j}$}

\vspace{-0.5em} \noindent Последнее равенство представляет собой связь изотропных тензоров второй и~третьей сложности.

Обобщая на все тензоры ненулевой сложности

\nopagebreak\vspace{-0.2em}\begin{equation}\label{crossproductforanytwotensors}
{^\mathrm{n}\hspace{-0.12ex}\bm{\xi}} \times \hspace{-0.12ex} {^\mathrm{m}\hspace{-0.12ex}\bm{\zeta}}
=
- \hspace{.25ex} {^\mathrm{n}\hspace{-0.12ex}\bm{\xi}} \dotp \levicivitatensor \dotp \hspace{-0.1ex} {^\mathrm{m}\hspace{-0.12ex}\bm{\zeta}}
\;\;\:
%
\forall \hspace{.4ex} {^\mathrm{n}\hspace{-0.12ex}\bm{\xi}} ,
\hspace{-0.12ex} {^\mathrm{m}\hspace{-0.12ex}\bm{\zeta}}
\;\;
\forall \hspace{.25ex} n \hspace{-0.25ex} > \hspace{-0.25ex} 0 , \,
m \hspace{-0.25ex} > \hspace{-0.25ex} 0
\hspace{.1ex} .
\end{equation}

\vspace{-0.1em} \noindent Когда один из тензоров\:--- единичный~(\inquotes{метрический}), из~\eqref{crossproductforanytwotensors} и~\eqref{identifyofidentitytensor} ${\forall \, {^\mathrm{n}\hspace{-0.12ex}\bm{\Upsilon}}}$ ${\forall \,\mathrm{n \!>\! 0}}$

\nopagebreak\vspace{-0.2em}\begin{equation*}\begin{array}{c}
\bm{E} \hspace{.1ex} \times \hspace{-0.1ex} {^\mathrm{n}\hspace{-0.12ex}\bm{\Upsilon}}
= - \hspace{.2ex} \bm{E} \hspace{.1ex} \dotp \levicivitatensor \dotp {^\mathrm{n}\hspace{-0.12ex}\bm{\Upsilon}}
= - \hspace{.2ex} \levicivitatensor \dotp {^\mathrm{n}\hspace{-0.12ex}\bm{\Upsilon}} \hspace{-0.15ex},
\\[.2em]
%
{^\mathrm{n}\hspace{-0.12ex}\bm{\Upsilon}} \times \bm{E}
= - \hspace{.2ex} {^\mathrm{n}\hspace{-0.12ex}\bm{\Upsilon}} \hspace{-0.1ex} \dotp \levicivitatensor \dotp \hspace{-0.1ex} \bm{E}
= - \hspace{.2ex} {^\mathrm{n}\hspace{-0.12ex}\bm{\Upsilon}} \hspace{-0.1ex} \dotp \levicivitatensor
\hspace{.1ex} .
\end{array}\end{equation*}

\vspace{-0.1em}
\en{A~cross product}\ru{Векторное произведение} \en{is~not~commutative}\ru{не~коммутативно} but anticommutative.
\en{For any two vectors}\ru{Для любых двух векторов} $\bm{a}$ \en{and}\ru{и}~$\bm{b}$

\nopagebreak\begin{equation}\label{crossproductoftwovectors}
\begin{array}{c}
\bm{a} \times \bm{b}
= \bm{a} \dotp \hspace{-0.1ex} \left( \bm{b} \hspace{-0.1ex} \times \hspace{-0.25ex} \bm{E} \hspace{.1ex} \right)
= \left( \bm{a} \hspace{-0.1ex} \times \hspace{-0.25ex} \bm{E} \hspace{.1ex} \right) \hspace{-0.1ex} \dotp \bm{b}
= - \hspace{.25ex} \bm{a} \bm{b} \dotdotp \levicivitatensor
= - \hspace{.2ex} \levicivitatensor \dotdotp \bm{a} \bm{b}
\hspace{.15ex} ,
\\[.15em]
%
\bm{b} \times \bm{a}
= \bm{b} \dotp \hspace{-0.1ex} \left( \bm{a} \hspace{-0.1ex} \times \hspace{-0.25ex} \bm{E} \hspace{.1ex} \right)
= \left( \bm{b} \hspace{-0.1ex} \times \hspace{-0.25ex} \bm{E} \hspace{.1ex} \right)  \hspace{-0.1ex} \dotp \bm{a}
= - \hspace{.25ex} \bm{b} \bm{a} \dotdotp \levicivitatensor
= - \hspace{.2ex} \levicivitatensor \dotdotp \bm{b} \bm{a}
\hspace{.15ex} ,
\\[.2em]
%
\bm{a} \times \bm{b} \hspace{.16ex}
= - \hspace{.33ex} \bm{a} \bm{b} \dotdotp \levicivitatensor
= \bm{b} \bm{a} \dotdotp \levicivitatensor
\:\Rightarrow\:
\bm{a} \times \bm{b} \hspace{.16ex} = - \hspace{.32ex} \bm{b} \times \bm{a}
\hspace{.2ex} .
\end{array}
\end{equation}

\vspace{-0.2em} \noindent
\en{For}\ru{Для} \en{any}\ru{любого} \en{bivalent tensor}\ru{бивалентного тензора}~${\hspace{-0.2ex}^2\hspace{-0.16em}\bm{B}}$
\en{and}\ru{и}~\en{a~tensor of first complexity}\ru{тензора первой сложности}~(\en{vector}\ru{вектора})~$\bm{a}$

\nopagebreak\vspace{-0.3em}\begin{equation*}\label{crossproductisnotcommutative}
{^2\hspace{-0.16em}\bm{B}} \hspace{-0.1ex} \times \bm{a}
= \bm{e}_i B_{i\hspace{-0.1ex}j} \bm{e}_{\hspace{-0.1ex}j} \hspace{-0.25ex} \times \hspace{-0.1ex} a_k \bm{e}_k \hspace{-0.2ex}
= \bigl( \hspace{-0.1ex} - \hspace{.25ex} a_k \bm{e}_k \hspace{-0.25ex} \times \hspace{-0.1ex} \bm{e}_{\hspace{-0.1ex}j} B_{i\hspace{-0.1ex}j} \bm{e}_i \hspace{.16ex} \bigr)^{\hspace{-0.25ex}\T} \hspace{-0.4ex}
= - \hspace{.25ex} \bigl( \hspace{.12ex} \bm{a} \times \hspace{-0.2ex} {^2\hspace{-0.16em}\bm{B}^{\T}} \hspace{.1ex} \bigr)^{\hspace{-0.3ex}\T}
\hspace{-0.4ex} .
\end{equation*}

\vspace{-0.15em} \noindent
\en{However}\ru{Однако}, \en{in~the~particular case of~}\ru{в~частном случае }\en{the~unit tensor}\ru{единичного тензора}~${\hspace{-0.2ex}\bm{E}}$ \en{and}\ru{и}~\en{a~vector}\ru{вектора}

%%\bigl( \bm{E} \times \bm{a} \bigr)^{\hspace{-0.25ex}\T} \hspace{-0.4ex}
%%= - \hspace{.33ex} \bm{a} \times \bm{E}
%%= - \hspace{.25ex} \bm{E} \times \bm{a}
%%= \hspace{-0.1ex} \bigl( \bm{a} \times \bm{E} \hspace{.1ex} \bigr)^{\hspace{-0.25ex}\T}

\nopagebreak\vspace{-0.25em}\begin{equation}\begin{array}{c}
\bm{E} \times \bm{a}
= - \hspace{.25ex} \bigl( \hspace{.1ex} \bm{a} \times \hspace{-0.25ex} \bm{E}^{\hspace{.1ex}\T} \hspace{.1ex} \bigr)^{\hspace{-0.3ex}\T} \hspace{-0.4ex}
= - \hspace{.25ex} \bigl( \bm{a} \times \hspace{-0.25ex} \bm{E} \hspace{.15ex} \bigr)^{\hspace{-0.25ex}\T} \hspace{-0.4ex}
= \bm{a} \times \hspace{-0.25ex} \bm{E}
\hspace{.15ex} ,
\\[.25em]
%
\bm{E} \times \bm{a} = \hspace{.1ex} \bm{a} \times \hspace{-0.25ex} \bm{E}
= - \hspace{.25ex} \bm{a} \dotp \levicivitatensor
= - \hspace{.2ex} \levicivitatensor \dotp \bm{a}
\hspace{.2ex} .
\vspace{-0.2em}
\end{array}\end{equation}

Справедливо такое соотношение

\nopagebreak\vspace{-0.1em}\begin{equation}\label{doubleveblen}
e_{i\hspace{-0.1ex}j\hspace{-0.1ex}k} \hspace{.1ex} e_{pqr}
\hspace{-0.1ex} = \hspace{.1ex}
\operatorname{det} \hspace{-0.25ex} \left[
\begin{array}{ccc}
\delta_{ip} & \delta_{iq} & \delta_{ir} \\
\delta_{\hspace{-0.1ex}j\hspace{-0.1ex}p} & \delta_{\hspace{-0.1ex}j\hspace{-0.1ex}q} & \delta_{\hspace{-0.1ex}j\hspace{-0.1ex}r} \\
\delta_{kp} & \delta_{kq} & \delta_{kr}
\end{array}\hspace{-0.1ex}
\right]
\end{equation}

\noindent
${ \tikz[baseline=-1ex] \draw [line width=.5pt, color=black, fill=white] (0, 0) circle (.8ex);
\hspace{.6em} }$
Доказательство начнём с~представлений символов Веблена как определителей
(${ e_{i\hspace{-0.1ex}j\hspace{-0.1ex}k} \hspace{-0.2ex}
= \pm \hspace{.4ex}
\bm{e}_i \hspace{-0.36ex} \times \hspace{-0.2ex} \bm{e}_j \hspace{-0.15ex} \dotp \hspace{.1ex} \bm{e}_k }$ по строкам,
${ e_{pqr} \hspace{-0.2ex}
= \pm \hspace{.4ex}
\bm{e}_p \hspace{-0.36ex} \times \hspace{-0.2ex} \bm{e}_q \hspace{-0.15ex} \dotp \hspace{.1ex} \bm{e}_r }$ по~столбцам,
с~\inquotes{$-$} для~\inquotes{левой} тройки)

\nopagebreak\vspace{-0.25em}\begin{equation*}
e_{i\hspace{-0.1ex}j\hspace{-0.1ex}k} \hspace{-0.1ex} = \hspace{.1ex}
\operatorname{det}\hspace{-0.25ex} \scalebox{0.96}[0.96]{$\left[\begin{array}{ccc}
\delta_{i1} & \delta_{i2} & \delta_{i3} \\
\delta_{k1} & \delta_{k2} & \delta_{k3}
\end{array}\right]$} \hspace{-0.25ex}, \:\:
e_{pqr} \hspace{-0.1ex} = \hspace{.1ex}
\operatorname{det}\hspace{-0.25ex} \scalebox{0.96}[0.96]{$\left[\begin{array}{ccc}
\delta_{p1} & \delta_{q1} & \delta_{r1} \\
\delta_{p2} & \delta_{q2} & \delta_{r2} \\
\delta_{p3} & \delta_{q3} & \delta_{r3}
\end{array}\right]$}
\hspace{-0.2ex} .
\end{equation*}

\vspace{-0.1em} \noindent В~левой части~\eqref{doubleveblen} сто\'{и}т произведение ${e_{i\hspace{-0.1ex}j\hspace{-0.1ex}k} \hspace{.1ex} e_{pqr}}$ этих определителей. Но~${\operatorname{det} \hspace{.2ex} (\hspace{-0.1ex}AB\hspace{.1ex}) = (\operatorname{det} A)(\operatorname{det} B)}$\:--- определитель произведения матриц равен произведению определителей. В~матрице\hbox{-}произведении элемент~${\left[{\,\cdots\,}\right]}_{1\hspace{-0.12ex}1}$ равен~${\delta_{is} \delta_{ps} \hspace{-0.2ex} = \delta_{ip}}$, как~и~в~\eqref{doubleveblen}; легко проверить и~другие фрагменты~\eqref{doubleveblen}.
${ \hspace{.6em}
\tikz[baseline=-0.6ex] \draw [color=black, fill=black] (0, 0) circle (.8ex); }$

\en{Contraction of}\ru{Свёртка}~\eqref{doubleveblen} приводит к~полезным формулам

\nopagebreak\vspace{-0.2em}\begin{equation*}\begin{array}{c}
e_{i\hspace{-0.1ex}j\hspace{-0.1ex}k} \hspace{.1ex} e_{pqk} \hspace{-0.1ex} = \hspace{.2ex}
\operatorname{det} \hspace{-0.25ex} \left[
\begin{array}{ccc}
\delta_{ip} & \delta_{iq} & \delta_{ik} \\
\delta_{\hspace{-0.1ex}j\hspace{-0.1ex}p} & \delta_{\hspace{-0.1ex}j\hspace{-0.1ex}q} & \delta_{\hspace{-0.1ex}j\hspace{-0.1ex}k} \\
\delta_{kp} & \delta_{kq} & \delta_{kk}
\end{array}
\right] \hspace{-0.5ex} = \hspace{.2ex}
\operatorname{det} \hspace{-0.25ex} \left[
\begin{array}{ccc}
\delta_{ip} & \delta_{iq} & \delta_{ik} \\
\delta_{\hspace{-0.1ex}j\hspace{-0.1ex}p} & \delta_{\hspace{-0.1ex}j\hspace{-0.1ex}q} & \delta_{\hspace{-0.1ex}j\hspace{-0.1ex}k} \\
\delta_{kp} & \delta_{kq} & 3
\end{array}
\right] \hspace{-0.5ex} =
\\[1.5em]
%
= \hspace{.1ex} 3 \hspace{.2ex} \delta_{ip} \delta_{\hspace{-0.1ex}j\hspace{-0.1ex}q} \hspace{-0.2ex}
+ \delta_{iq} \delta_{\hspace{-0.1ex}j\hspace{-0.1ex}k} \delta_{kp} \hspace{-0.2ex}
+ \delta_{ik} \delta_{\hspace{-0.1ex}j\hspace{-0.1ex}p} \delta_{kq} \hspace{-0.2ex}
- \delta_{ik} \delta_{\hspace{-0.1ex}j\hspace{-0.1ex}q} \delta_{kp} \hspace{-0.2ex}
- \hspace{.1ex} 3 \hspace{.2ex} \delta_{iq} \delta_{\hspace{-0.1ex}j\hspace{-0.1ex}p} \hspace{-0.2ex}
- \delta_{ip} \delta_{\hspace{-0.1ex}j\hspace{-0.1ex}k} \delta_{kq} \hspace{-0.2ex} =
\\[.25em]
%
= \hspace{.1ex} 3 \hspace{.2ex} \delta_{ip} \delta_{\hspace{-0.1ex}j\hspace{-0.1ex}q} \hspace{-0.2ex}
+ \delta_{iq} \delta_{\hspace{-0.1ex}j\hspace{-0.1ex}p} \hspace{-0.2ex}
+ \delta_{iq} \delta_{\hspace{-0.1ex}j\hspace{-0.1ex}p} \hspace{-0.2ex}
- \delta_{ip} \delta_{\hspace{-0.1ex}j\hspace{-0.1ex}q} \hspace{-0.2ex}
- \hspace{.1ex} 3 \hspace{.2ex} \delta_{iq} \delta_{\hspace{-0.1ex}j\hspace{-0.1ex}p} \hspace{-0.2ex}
- \delta_{ip} \delta_{\hspace{-0.1ex}j\hspace{-0.1ex}q} \hspace{-0.2ex} =
\\[.25em]
%
= \delta_{ip} \delta_{\hspace{-0.1ex}j\hspace{-0.1ex}q} \hspace{-0.1ex}
- \hspace{.1ex} \delta_{iq} \delta_{\hspace{-0.1ex}j\hspace{-0.1ex}p}
\hspace{.2ex} ,
\end{array}\end{equation*}

\nopagebreak\vspace{-0.15em}\begin{equation*}
e_{i\hspace{-0.1ex}j\hspace{-0.1ex}k} \hspace{.1ex} e_{pj\hspace{-0.1ex}k} \hspace{-0.15ex}
= \delta_{ip} \delta_{\hspace{-0.1ex}j\hspace{-0.12ex}j} \hspace{-0.1ex} - \hspace{.1ex} \delta_{i\hspace{-0.1ex}j} \delta_{\hspace{-0.1ex}j\hspace{-0.1ex}p} \hspace{-0.15ex}
= \hspace{.1ex} 3 \hspace{.2ex} \delta_{ip} \hspace{-0.2ex} - \delta_{ip} \hspace{-0.15ex}
= \hspace{.1ex} 2 \hspace{.1ex} \delta_{ip}
\hspace{.2ex} ,
\end{equation*}

\nopagebreak\vspace{-0.15em}\begin{equation*}
e_{i\hspace{-0.1ex}j\hspace{-0.1ex}k} \hspace{.1ex} e_{i\hspace{-0.1ex}j\hspace{-0.1ex}k} \hspace{-0.15ex}
= \hspace{.1ex} 2 \hspace{.2ex} \delta_{ii} \hspace{-0.15ex}
= \hspace{.1ex} 6
\hspace{.2ex} .
\end{equation*}

\vspace{-0.33em} \noindent \en{Or}\ru{Или} \en{in short}\ru{к\'{о}ротко}

\nopagebreak\vspace{-0.3em}\begin{equation}\label{veblencontraction}
e_{i\hspace{-0.1ex}j\hspace{-0.1ex}k} \hspace{.1ex} e_{pqk} \hspace{-0.16ex} = \delta_{ip} \delta_{\hspace{-0.1ex}j\hspace{-0.1ex}q} \hspace{-0.1ex} - \hspace{.1ex} \delta_{iq} \delta_{\hspace{-0.1ex}j\hspace{-0.1ex}p}
\hspace{.2ex} ,
\:\,
%
e_{i\hspace{-0.1ex}j\hspace{-0.1ex}k} \hspace{.1ex} e_{pj\hspace{-0.1ex}k} \hspace{-0.15ex} = \hspace{.1ex} 2 \hspace{.2ex} \delta_{ip}
\hspace{.2ex} ,
\:\,
%
e_{i\hspace{-0.1ex}j\hspace{-0.1ex}k} \hspace{.1ex} e_{i\hspace{-0.1ex}j\hspace{-0.1ex}k} \hspace{-0.15ex} = \hspace{.1ex} 6
\hspace{.2ex} .
\end{equation}

Первая из~этих формул даёт представление двойного векторного произведения

\nopagebreak\vspace{-0.55em}\begin{multline}
\bm{a} \times \hspace{-0.15ex} \left(\hspace{.2ex}{\bm{b} \times \bm{c}}\hspace{.15ex}\right)
= a_i \bm{e}_i \times \levicivita_{pqj} \hspace{.2ex} b_p c_q \bm{e}_j \hspace{-0.2ex}
= \levicivita_{ki\hspace{-0.1ex}j} \levicivita_{pqj} \hspace{.2ex} a_i b_p c_q \bm{e}_k \hspace{-0.2ex} =
\\
= \left({\delta_{kp}\delta_{iq} \hspace{-0.2ex} - \delta_{kq}\delta_{ip}}\right) a_i b_p c_q \bm{e}_k \hspace{-0.2ex}
= a_i b_k c_i \bm{e}_k \hspace{-0.15ex} - a_i b_i c_k \bm{e}_k \hspace{-0.2ex} =
\\
= \bm{a} \dotp \bm{c} \bm{b} - \bm{a} \dotp \bm{b} \bm{c}
= \bm{a} \dotp \hspace{-0.1ex} \bigl( \bm{c} \bm{b} - \bm{b} \bm{c} \hspace{.15ex} \bigr) \hspace{-0.3ex}
= \bm{a} \dotp \bm{c} \bm{b} - \bm{c} \bm{b} \dotp \bm{a}
%%= \bm{b} \bm{a} \dotp \bm{c} - \bm{c} \bm{a} \dotp \bm{b}
\hspace{.1ex} .
\end{multline}

\end{otherlanguage}

\vspace{-0.16em} \noindent \en{By~another interpretation}\ru{По~другой интерпретации}, \en{the~}dot product \en{of~a~dyad}\ru{диады} \en{and}\ru{и}~\en{a~vector}\ru{вектора} \en{is not~commutative}\ru{не~коммутативен}:\ru{\hspace{.2ex}}
${\bm{b} \bm{d} \hspace{.1ex} \dotp \bm{c} \hspace{.25ex} \neq \hspace{.25ex} \bm{c} \hspace{.1ex} \dotp \bm{b} \bm{d}}$,
\en{and}\ru{и}~\en{this difference}\ru{эта разница} \en{can be expressed as}\ru{может быть выражена как}

\nopagebreak\vspace{-1.2em}\begin{equation}
\bm{b} \bm{d} \hspace{.1ex} \dotp \bm{c} \hspace{.25ex} - \hspace{.25ex} \bm{c} \hspace{.1ex} \dotp \bm{b} \bm{d}
\hspace{.25ex} = \hspace{.25ex}
\bm{c} \times \hspace{-0.2ex} \bigl( \hspace{.1ex} \bm{b} \times \bm{d} \hspace{.25ex} \bigr)
\hspace{-0.1ex} .
\end{equation}

\noindent ${
\bm{a} \dotp \bm{b} \bm{c} = \bm{c} \bm{b} \dotp \bm{a} = \bm{c} \bm{a} \dotp \bm{b} = \bm{b} \dotp \bm{a} \bm{c}
}$

\noindent ${
\left(\hspace{.1ex} \bm{a} \times \bm{b} \hspace{.2ex}\right) \hspace{-0.2ex} \times \bm{c}
= - \hspace{.33ex} \bm{c} \times \hspace{-0.15ex} \left(\hspace{.2ex}{\bm{a} \times \bm{b}}\hspace{.15ex}\right) \hspace{-0.1ex}
= \bm{c} \times \hspace{-0.2ex} \left(\hspace{.2ex}{\bm{b} \times \bm{a}}\hspace{.15ex}\right)
}$

\vspace{-0.15em} \noindent
\textcolor{magenta}{\en{The~same way}\ru{Тем~же путём} \en{it may~be derived that}\ru{выводится}}

\nopagebreak\vspace{-0.2em}\begin{equation}
\left(\hspace{.1ex} \bm{a} \times \bm{b} \hspace{.2ex}\right) \hspace{-0.2ex} \times \bm{c}
= \hspace{-0.15ex} \bigl( \hspace{.1ex} \bm{b} \bm{a} - \bm{a} \bm{b} \hspace{.15ex} \bigr) \hspace{-0.3ex} \dotp \bm{c}
= \bm{b} \bm{a} \dotp \bm{c} - \bm{a} \bm{b} \dotp \bm{c}
\hspace{.15ex} .
\end{equation}

\begin{otherlanguage}{russian}

\vspace{-0.15em} \noindent И~такие тождества для~любых двух векторов~$\bm{a}$ и~$\bm{b}$

\nopagebreak\vspace{-0.3em}
\begin{multline}\label{vectorcrossvectorcrossidentity}
\left(\hspace{.1ex} \bm{a} \times \bm{b} \hspace{.2ex}\right) \hspace{-0.2ex} \times \hspace{-0.1ex} \bm{E}
= \levicivita_{i\hspace{-0.1ex}j\hspace{-0.1ex}k} \hspace{.2ex} a_i \hspace{.1ex} b_j \hspace{.1ex} \bm{e}_k \hspace{-0.16ex} \times \hspace{-0.1ex} \bm{e}_n \bm{e}_n \hspace{-0.2ex}
= a_i \hspace{.1ex} b_j \hspace{.1ex} \levicivita_{i\hspace{-0.1ex}j\hspace{-0.1ex}k} \levicivita_{knq} \hspace{.2ex} \bm{e}_q \bm{e}_n \hspace{-0.2ex} =
\\
%
= a_i b_j \hspace{-0.2ex} \left( \delta_{in} \delta_{j\hspace{-0.1ex}q} \hspace{-0.2ex} - \delta_{iq} \delta_{jn} \right) \hspace{-0.2ex} \bm{e}_q \bm{e}_n \hspace{-0.2ex}
= a_i b_j \bm{e}_j \bm{e}_i \hspace{-0.12ex} - a_i b_j \bm{e}_i \bm{e}_j \hspace{-0.16ex} =
\\[-0.08em]
%
= \bm{b} \bm{a} - \bm{a} \bm{b}
\hspace{.2ex},
\end{multline}

\vspace{-1em}\begin{multline}\label{vectorcrossidentitydotvectorcrossidentity}
\left( \hspace{.1ex} \bm{a} \times\hspace{-0.1ex} \bm{E} \hspace{.16ex} \right) \hspace{-0.1ex} \dotp \left( \hspace{.12ex} \bm{b} \times\hspace{-0.1ex} \bm{E} \hspace{.16ex} \right) \hspace{-0.2ex}
= \left( \hspace{.1ex} \bm{a} \hspace{.2ex} \dotp \levicivitatensor \hspace{.16ex}\right) \hspace{-0.12ex}\dotp \left(\hspace{.1ex} \bm{b} \hspace{.2ex} \dotp \levicivitatensor \hspace{.16ex}\right) =
\\[-0.1em]
%
= a_i \levicivita_{ipn} \hspace{.1ex} \bm{e}_p \bm{e}_n \dotp \hspace{.2ex} b_j \levicivita_{jsk} \hspace{.1ex} \bm{e}_s \bm{e}_k
= a_i b_j \levicivita_{ipn} \levicivita_{nkj} \hspace{.05ex} \bm{e}_p \bm{e}_k =
\\
%
= a_i b_j \hspace{-0.2ex} \left( \delta_{ik} \delta_{pj} \hspace{-0.2ex} - \delta_{i\hspace{-0.1ex}j} \delta_{pk} \right) \hspace{-0.2ex} \bm{e}_p \bm{e}_k
= a_i b_j \bm{e}_j \bm{e}_i \hspace{-0.12ex} - a_i b_i \bm{e}_k \bm{e}_k \hspace{-0.16ex} =
\\[-0.08em]
%
= \hspace{.1ex} \bm{b} \bm{a} - \bm{a} \hspace{-0.1ex} \dotp \bm{b} \hspace{.2ex} \bm{E} \hspace{.1ex}.
\end{multline}

\vspace{-0.2em} Ещё одно соотношение между изотропными тензорами второй и~третьей сложности:

\nopagebreak\vspace{-0.1em}\begin{equation}
\hspace{4.4em} \levicivitatensor \hspace{.2ex} \dotdotp \levicivitatensor = \levicivita_{i\hspace{-0.1ex}j\hspace{-0.1ex}k} \bm{e}_i \hspace{.2ex} \levicivita_{kjn} \bm{e}_n = - 2 \hspace{.2ex} \delta_{in} \hspace{.12ex} \bm{e}_i \bm{e}_n = - 2 \hspace{.1ex} \bm{E}
\hspace{.1ex} .
\end{equation}

\end{otherlanguage}

\en{\section{Symmetric and antisymmetric tensors}}

\ru{\section{Симметричные и антисимметричные тензоры}}

\label{para:tensors.symmetric+skewsymmetric}

\begin{otherlanguage}{russian}

Тензор, не~меняющийся при перестановке какой\hbox{-}либо пары своих индексов, называется симметричным по~этой~паре индексов. Если~же при~перестановке пары индексов тензор меняет свой знак, то он называется антисимметричным~(кососимметричным) по~этой~паре индексов.

Тензор Л\'{е}ви\hbox{-\!}Чив\'{и}ты~$\levicivitatensor$ антисимметричен по~любой паре индексов, то~есть он полностью антисимметричен~(абсолютно кососимметричен).

Тензор второй сложности ${\bm{B}}$ симметричен, если ${\bm{B} = \bm{B}^\T}$. Когда транспонирование меняет знак тензора, то~есть ${\bm{A}^{\hspace{-0.1em}\T} = - \bm{A}}$, тогда он антисимметричен (кососимметричен).

Любой тензор второй сложности представ\'{и}м суммой симметричной и~антисимметричной частей

\nopagebreak\vspace{-0.1em}\begin{equation}\begin{array}{c}
\bm{C} = \bm{C}^{\mathsf{\hspace{.1ex}S}} \hspace{-0.1ex}+\hspace{.1ex} \bm{C}^{\mathsf{\hspace{.1ex}A}}\hspace{.1ex} , \;
\bm{C}^{\hspace{.1ex}\T} \hspace{-0.33ex} = \bm{C}^{\mathsf{\hspace{.1ex}S}} \hspace{-0.1ex}-\hspace{.1ex} \bm{C}^{\mathsf{\hspace{.1ex}A}}
\hspace{.2ex} ;
\\[0.2em]
\bm{C}^{\mathsf{\hspace{.1ex}S}} \hspace{-0.22ex} \equiv \hspace{.1ex} \displaystyle \onehalf \left( {\bm{C} + \bm{C}^{\hspace{.1ex}\T}}\hspace{.1ex} \right)\!, \;
\bm{C}^{\mathsf{\hspace{.1ex}A}} \hspace{-0.22ex} \equiv \hspace{.1ex} \displaystyle \onehalf \left( {\bm{C} - \bm{C}^{\hspace{.1ex}\T}}\hspace{.1ex} \right)\!.
\end{array}\end{equation}

\noindent Для диады ${\bm{c}\bm{d} \hspace{.1ex} = \bm{c}\bm{d}^{\mathsf{\hspace{.25ex}S}} \hspace{-0.2ex} + \hspace{.1ex} \bm{c}\bm{d}^{\mathsf{\hspace{.25ex}A}} \hspace{-0.16ex} = \hspace{.1ex}
\smalldisplaystyleonehalf \hspace{-0.1ex} \left( \bm{c}\bm{d} + \hspace{-0.1ex} \bm{d}\bm{c} \hspace{.1ex} \right)
+ \hspace{.16ex} \smalldisplaystyleonehalf \hspace{-0.1ex} \left( \bm{c} \bm{d} - \hspace{-0.1ex} \bm{d} \bm{c} \hspace{.1ex} \right)}$.

Произведение двух симметричных тензоров~${\bm{C}^{\mathsf{\hspace{.1ex}S}} \hspace{-0.1ex}\dotp \bm{D}^{\mathsf{\hspace{.1ex}S}}}$ симметрично далеко не~всегда, но~только когда ${\bm{D}^{\mathsf{\hspace{.1ex}S}} \hspace{-0.1ex}\dotp\hspace{.12ex} \bm{C}^{\mathsf{\hspace{.1ex}S}} \hspace{-0.2ex}=\hspace{.1ex} \bm{C}^{\mathsf{\hspace{.1ex}S}} \hspace{-0.1ex}\dotp \bm{D}^{\mathsf{\hspace{.1ex}S}}\hspace{-0.4ex}}$,
ведь по~\eqref{transposeofdotproduct} ${\left(\hspace{.08ex} \bm{C}^{\mathsf{\hspace{.1ex}S}} \hspace{-0.1ex}\dotp \bm{D}^{\mathsf{\hspace{.1ex}S}} \hspace{.1ex}\right)^{\hspace{-0.32ex}\T} \hspace{-0.16ex} = \hspace{.1ex} \bm{D}^{\mathsf{\hspace{.1ex}S}} \hspace{-0.1ex}\dotp\hspace{.12ex} \bm{C}^{\mathsf{\hspace{.1ex}S}} \hspace{-0.4ex}}$.

В~нечётномерных пространствах любой антисимметричный тензор второй сложности необрат\'{и}м, определитель матрицы компонент для~него\:--- нулевой.

Существует взаимно\hbox{-}однозначное соответствие между антисимметричными тензорами второй сложности и~(псевдо)векторами. Компоненты кососимметричного тензора полностью определяются тройкой чисел (диагональные элементы матрицы компонент\:--- нули, а~недиагональные\:--- попарно противоположны). Dot product кососимметричного~${\hspace{-0.2ex}\bm{A}}$ и~какого\hbox{-}либо тензора~${\hspace{-0.2ex}{^\mathrm{n}\hspace{-0.12ex}\bm{\xi}}}$ однозначно соответствует cross product’у псевдовектора~$\bm{a}$ и~того~же тензора~${\hspace{-0.2ex}{^\mathrm{n}\hspace{-0.12ex}\bm{\xi}}}$

\nopagebreak\vspace{-0.1em}\begin{equation}\begin{array}{c}
\hspace{1.2em} \bm{b} \hspace{.3ex}
= \bm{A} \hspace{.2ex}\dotp\hspace{.1ex} {^\mathrm{n}\hspace{-0.12ex}\bm{\xi}}
\;\,\Leftrightarrow\;
\bm{a} \hspace{.1ex} \times {^\mathrm{n}\hspace{-0.12ex}\bm{\xi}}
\hspace{.1ex} = \hspace{.15ex} \bm{b} \hspace{.1ex} \;\:\:
\forall \bm{A} \!=\! \bm{A}^{\mathsf{\!\,A}} \;\;
\forall \, {^\mathrm{n}\hspace{-0.12ex}\bm{\xi}} \;\; \forall \,\mathrm{n \!>\! 0}
\hspace{.15ex} ,
\\[.2em]
%
\hspace{1.2em} \bm{d} \hspace{.3ex}
= {^\mathrm{n}\hspace{-0.12ex}\bm{\xi}} \hspace{.2ex} \dotp \bm{A}
\;\,\Leftrightarrow\;
{^\mathrm{n}\hspace{-0.12ex}\bm{\xi}} \hspace{.2ex}\times \bm{a}
\hspace{.1ex} = \hspace{.15ex} \bm{d} \hspace{.1ex} \;\:\:
\forall \bm{A} \!=\! \bm{A}^{\mathsf{\!\,A}} \;\;
\forall \, {^\mathrm{n}\hspace{-0.12ex}\bm{\xi}} \;\; \forall \,\mathrm{n \!>\! 0}
\hspace{.15ex} .
\end{array}\end{equation}

Раскроем это соответствие~${\bm{A} \narroweq \bm{A}(\bm{a})}$:

\nopagebreak\vspace{-0.1em}\begin{equation*}
\begin{array}{r@{\hspace{1ex}}c@{\hspace{1ex}}l}
\bm{A} \hspace{.2ex}\dotp\hspace{.1ex} {^\mathrm{n}\hspace{-0.12ex}\bm{\xi}} & = & \bm{a} \hspace{.1ex} \times {^\mathrm{n}\hspace{-0.12ex}\bm{\xi}}
\\
%
A_{hi} \bm{e}_h \bm{e}_i \hspace{.1ex}\dotp\hspace{.25ex} \xi_{j\hspace{-0.1ex}k \ldots q} \hspace{.2ex} \bm{e}_j \bm{e}_k \ldots \bm{e}_q & = & a_{i} \bm{e}_i \times\hspace{.2ex} \xi_{j\hspace{-0.1ex}k \ldots q} \hspace{.2ex} \bm{e}_j \bm{e}_k \ldots \bm{e}_q
\\[.1em]
%
A_{hj} \hspace{.2ex} \xi_{j\hspace{-0.1ex}k \ldots q} \hspace{.2ex} \bm{e}_h \bm{e}_k \ldots \bm{e}_q & = & a_{i} \levicivita_{i\hspace{-0.1ex}jh} \hspace{.2ex} \xi_{j\hspace{-0.1ex}k \ldots q} \hspace{.2ex} \bm{e}_h \bm{e}_k \ldots \bm{e}_q
\\[.1em]
%
A_{hj} & = & a_{i} \levicivita_{i\hspace{-0.1ex}jh}
\\[.1em]
%
A_{hj} & = & \! - \, a_{i} \levicivita_{ihj}
\\[.2em]
%
\bm{A} & = & \! - \, \bm{a} \dotp \levicivitatensor
\end{array}
\end{equation*}

Так~же из ${{^\mathrm{n}\hspace{-0.12ex}\bm{\xi}} \hspace{.2ex} \dotp \bm{A} = {^\mathrm{n}\hspace{-0.12ex}\bm{\xi}} \hspace{.2ex}\times \bm{a}}$ получается ${\bm{A} = - \hspace{.2ex} \levicivitatensor \dotp \bm{a}}$.

Или проще, согласно~\eqref{crossproductforanytwotensors}

\nopagebreak\vspace{-0.1em}\begin{equation*}\begin{array}{c}
\bm{A} = \bm{A} \hspace{.1ex} \dotp \bm{E} = \hspace{.1ex} \bm{a} \times \hspace{-0.1ex} \bm{E} = - \hspace{.3ex} \bm{a} \dotp \levicivitatensor
\hspace{.1ex} ,
\\[.1em]
\bm{A} = \bm{E} \hspace{.1ex} \dotp \bm{A} = \bm{E} \times \bm{a} = - \hspace{.2ex} \levicivitatensor \dotp \bm{a}
\hspace{.2ex} .
\end{array}\end{equation*}

В~общем, для взаимно\hbox{-}однозначного соответствия между~${\hspace{-0.2ex}\bm{A}}$ и~$\bm{a}$ имеем

\nopagebreak\vspace{-0.25em}\begin{equation}\label{companionvector}
\begin{array}{c}
\bm{A} \hspace{.2ex} = \hspace{.1ex} - \hspace{.3ex} \bm{a} \dotp \levicivitatensor \hspace{.2ex} = \hspace{.2ex} \bm{a} \times\hspace{-0.1ex} \bm{E} \hspace{.2ex} = \hspace{.1ex} - \hspace{.2ex} \levicivitatensor \dotp \bm{a} \hspace{.2ex} = \hspace{.1ex} \bm{E} \times \bm{a}
\hspace{.1ex} ,
\\[.3em]
%
\bm{a} = \bm{a} \hspace{.1ex} \dotp \bm{E} = \bm{a} \dotp \left( \hspace{-0.3ex} - \displaystyle \hspace{.2ex} \smalldisplaystyleonehalf \hspace{.4ex} \levicivitatensor \dotdotp \hspace{-0.1ex} \levicivitatensor \right) \hspace{-0.4ex} = \hspace{.1ex} \smalldisplaystyleonehalf \hspace{.32ex} \bm{A} \dotdotp \levicivitatensor
\hspace{.1ex} .
\end{array}
\end{equation}

Компоненты кососимметричного~${\hspace{-0.2ex}\bm{A}}$ через компоненты сопутствующего ему псевдовектора~$\bm{a}$
\textcolor{magenta}{($\bm{a}$ называется сопутствующим для~${\hspace{-0.2ex}\bm{A}}$)}

\nopagebreak\vspace{-0.1em}\begin{equation*}
\begin{array}{c}
\bm{A} = - \, \levicivitatensor \dotp \bm{a} = - \hspace{.1ex} \levicivita_{i\hspace{-0.1ex}j\hspace{-0.1ex}k} \hspace{.1ex} \bm{e}_i \bm{e}_j a_k ,
\\[.3em]
A_{i\hspace{-0.1ex}j} = - \hspace{.1ex} \levicivita_{i\hspace{-0.1ex}j\hspace{-0.1ex}k} \hspace{.1ex} a_k \hspace{.1em} = \hspace{-0.1em}
\scalebox{0.9}[0.9]{$\left[ \begin{array}{ccc}
0 & -a_3 & a_2 \\
a_3 & 0 & -a_1 \\
-a_2 & a_1 & 0
\end{array} \hspace{.25ex}\right]$}
\end{array}
\end{equation*}

\vspace{-0.4em} \noindent и~наоборот

\nopagebreak\vspace{-0.5em}\begin{equation*}\begin{array}{c}
\bm{a} = \smalldisplaystyleonehalf \, \bm{A} \dotdotp \levicivitatensor =
\smalldisplaystyleonehalf \hspace{.25ex} A_{j\hspace{-0.1ex}k} \levicivita_{kj\hspace{-0.06ex}i} \hspace{.2ex} \bm{e}_i , \\[0.64em]
a_{i} \hspace{-0.16ex} = \smalldisplaystyleonehalf \hspace{.32ex} \levicivita_{ikj} \hspace{.1ex} A_{j\hspace{-0.1ex}k} = \hspace{.1em}
\displaystyle \onehalf \scalebox{0.9}[0.9]{$\left[\hspace{-0.25ex} \begin{array}{c}
\levicivita_{123} \hspace{.1ex} A_{32} + \levicivita_{132} \hspace{.1ex} A_{23} \\
\levicivita_{213} \hspace{.1ex} A_{31} + \levicivita_{231} \hspace{.1ex} A_{13} \\
\levicivita_{312} \hspace{.1ex} A_{21} + \levicivita_{321} \hspace{.1ex} A_{12}
\end{array} \right]$} \hspace{-0.2em} = \hspace{.1em}
\displaystyle \onehalf \scalebox{0.9}[0.9]{$\left[\hspace{-0.3ex} \begin{array}{c}
A_{32} - A_{23} \\
A_{13} - A_{31} \\
A_{21} - A_{12}
\end{array} \right]$} .
\vspace{.1em}\end{array}\end{equation*}

Легко запоминающийся вспомогательный \inquotes{псевдовекторный инвариант}~${\!\bm{A}_{\!\bm{\times}}}$ получается из тензора~${\hspace{-0.2ex}\bm{A}}$ заменой диадного произведения на~векторное

\nopagebreak\vspace{-0.15em}\begin{equation}\label{pseudovectorinvariant}
\begin{array}{c}
\bm{A}_{\Xcompanion} \equiv A_{i\hspace{-0.1ex}j} \hspace{.25ex} \bm{e}_i \times \bm{e}_j = - \hspace{.1ex} \bm{A} \hspace{.1ex} \dotdotp \levicivitatensor
\hspace{.2ex}, \\[.3em]
%
\bm{A}_{\Xcompanion} \hspace{-0.16ex} = \left(^{\mathstrut} \hspace{-0.1ex} \bm{a} \times\hspace{-0.1ex} \bm{E} \hspace{.2ex} \right)_{\hspace{-0.25ex}\Xcompanion} \hspace{-0.25ex} = \hspace{-0.12ex}
- 2 \hspace{.16ex} \bm{a} \hspace{.2ex},\:\:
\bm{a} = - \hspace{.2ex} \smalldisplaystyleonehalf \hspace{.25ex} \bm{A}_{\Xcompanion} = - \hspace{.2ex} \smalldisplaystyleonehalf \left(^{\mathstrut} \hspace{-0.1ex} \bm{a} \times\hspace{-0.1ex} \bm{E} \hspace{.2ex} \right)_{\hspace{-0.25ex}\Xcompanion} \hspace{-0.2ex}.
\end{array}
\end{equation}

Обоснование~\eqref{pseudovectorinvariant}:

\nopagebreak\vspace{-0.1em}\[\begin{array}{c}
\bm{a} \times\hspace{-0.1ex} \bm{E} = - \hspace{.2ex} \smalldisplaystyleonehalf \hspace{.4ex} \bm{A}_{\Xcompanion} \times \bm{E} = - \hspace{.2ex} \smalldisplaystyleonehalf \, A_{i\hspace{-0.1ex}j} \hspace{-0.2ex}
\left( \hspace{.08ex} \tikzmark{beginFirstCrossProduct} {\bm{e}_i \times \bm{e}_j} \tikzmark{endFirstCrossProduct} \hspace{.16ex} \right)
\times \bm{e}_k \bm{e}_k = \\[1.5em]
%
= - \hspace{.2ex} \smalldisplaystyleonehalf \hspace{.32ex} A_{i\hspace{-0.1ex}j} \hspace{.12ex}
\tikzmark{beginTwoLeviCivitas} \levicivita_{ni\hspace{-0.1ex}j} \levicivita_{nkp} \tikzmark{endTwoLeviCivitas}
\hspace{.32ex} \bm{e}_p \bm{e}_k = - \hspace{.2ex} \smalldisplaystyleonehalf \hspace{.32ex} A_{i\hspace{-0.1ex}j} \hspace{-0.1ex} \left( \bm{e}_j \bm{e}_i - \bm{e}_i \bm{e}_j \right) = \\[.8em]
%
\hspace{13.2em}= - \hspace{.2ex} \smalldisplaystyleonehalf \left({ \bm{A}^{\hspace{-0.1em}\T} \hspace{-0.25ex} - \hspace{-0.2ex} \bm{A} \hspace{.2ex}}\right) = \bm{A}^{\mathsf{\hspace{-0.1ex}A}} = \bm{A}.
\end{array}\]
\AddUnderBrace[line width=.75pt][0,-0.25ex]{beginFirstCrossProduct}{endFirstCrossProduct}%
{${\scriptstyle \levicivita_{i\hspace{-0.1ex}jn} \bm{e}_n}$}
\AddUnderBrace[line width=.75pt][.25ex,-0.25ex]{beginTwoLeviCivitas}{endTwoLeviCivitas}%
{${\scriptstyle \hspace{3.2em}\delta_{j\hspace{-0.1ex}p} \delta_{ik} \,-\; \delta_{ip} \delta_{j\hspace{-0.1ex}k}}$}

\vspace{-0.6em} Сопутствующий вектор \textcolor{magenta}{можно ввести} для любого бивалентного тензора, но лишь антисимметричная часть при~этом даёт вклад: ${\bm{C}^{\hspace{.2ex}\mathsf{A}} \hspace{-0.1ex} = \hspace{-0.1ex} - \hspace{.1ex}\onehalf\hspace{.32ex} \bm{C}_{\hspace{-0.1ex}\Xcompanion} \hspace{-0.16ex} \times \hspace{-0.16ex} \bm{E}}$. Для~симметричного тензора сопутствующий вектор\:--- нулевой: ${\bm{B}_{\Xcompanion} \hspace{-0.25ex} = \bm{0} \hspace{.25ex} \Leftrightarrow \bm{B} = \bm{B}^{\T} \hspace{-0.32ex} = \bm{B}^{\mathsf{\hspace{.1ex}S}}\hspace{-0.32ex}}$.

С~\eqref{pseudovectorinvariant} разложение какого\hbox{-}либо тензора~$\bm{C}$ на~симметричную и~антисимметричную части выглядит как

\nopagebreak\vspace{-0.1em}\begin{equation}\label{symmetricantisymmetricdecompositionofsometensor}
\bm{C} = \bm{C}^{\mathsf{\hspace{.1ex}S}} \hspace{-0.32ex} - \hspace{.1ex} \smalldisplaystyleonehalf \hspace{.32ex} \bm{C}_{\hspace{-0.1ex}\Xcompanion} \hspace{-0.16ex} \times \hspace{-0.16ex} \bm{E} \hspace{.1ex} .
\end{equation}

\vspace{-0.8em}\noindent Для диады~же

\nopagebreak{\centering \eqref{vectorcrossvectorcrossidentity}~$\Rightarrow$~${\left( \bm{c} \hspace{-0.1ex} \times \hspace{-0.2ex} \bm{d} \hspace{.2ex} \right) \hspace{-0.12ex} \times \hspace{-0.25ex} \bm{E} = \bm{d} \bm{c} - \hspace{-0.1ex} \bm{c} \bm{d} \hspace{.1ex} = \hspace{-0.1ex} - \hspace{.1ex} 2 \hspace{.15ex} \bm{c}\bm{d}^{\mathsf{\hspace{.3ex}A}}\hspace{-0.2ex}}$,\hspace{.32em}
${\left( \bm{c} \bm{d} \hspace{.2ex} \right)_{\hspace{-0.15ex}\Xcompanion} \hspace{-0.32ex} =
\hspace{.1ex} \bm{c} \hspace{-0.1ex} \times \hspace{-0.16ex} \bm{d} \hspace{.12ex}}$, \par}

\nopagebreak\vspace{-0.1em}\noindent и разложение~её

\nopagebreak\vspace{-0.6em}\begin{equation}\label{symmetricantisymmetricdecompositionofdyad}
\hspace*{1em} \bm{c}\bm{d} \hspace{.1ex} = \hspace{.1ex}
%%\smalldisplaystyleonehalf \hspace{-0.1ex} \left( \bm{c}\bm{d} + \hspace{-0.1ex} \bm{d}\bm{c} \hspace{.1ex} \right) - \hspace{.16ex} \smalldisplaystyleonehalf \hspace{-0.1ex} \left( \bm{d} \bm{c} - \hspace{-0.1ex} \bm{c} \bm{d} \hspace{.2ex} \right) = \hspace{.1ex}
\smalldisplaystyleonehalf \hspace{-0.1ex} \left( \bm{c}\bm{d} + \hspace{-0.1ex} \bm{d}\bm{c} \hspace{.1ex} \right)
- \hspace{.16ex} \smalldisplaystyleonehalf \hspace{-0.1ex} \left( \bm{c} \hspace{-0.1ex} \times \hspace{-0.2ex} \bm{d} \hspace{.2ex} \right) \hspace{-0.12ex} \times \hspace{-0.25ex} \bm{E} \hspace{.1ex} .
\end{equation}

\end{otherlanguage}

\en{\section{Eigenvectors and eigenvalues of tensor}}

\ru{\section{Собственные векторы и собственные числа тензора}}

\label{para:eigenvectorseigenvalues}

\begin{otherlanguage}{russian}

Если для тензора~${^2\!\bm{B}}$ и~ненулевого вектора~${\bm{a}}$

\nopagebreak\vspace{-0.24em}\begin{equation}\label{eigenvalues:eq}
^2\!\bm{B} \dotp \bm{a} = \eigenvalue \bm{a} \hspace{.1ex} ,
\:\:
\bm{a} \neq \bm{0}
\end{equation}
\vspace{-1.33em}\[
\scalebox{0.92}[0.92]{${^2\!\bm{B}} \dotp \bm{a} = \eigenvalue \bm{E} \dotp \bm{a} \hspace{.1ex},\;\;
%%B_{i\hspace{-0.1ex}j} a_j = \eigenvalue \delta_{i\hspace{-0.1ex}j} a_j \hspace{.1ex},\;\;
\left(\hspace{.1ex}
{^2\!\bm{B} - \eigenvalue \bm{E}}
\hspace{.2ex}\right) \hspace{-0.1ex} \dotp \hspace{.1ex} \bm{a} = \bm{0}$} \hspace{.1ex},
\]

\vspace{-0.64em} \noindent то $\eigenvalue$ называется собственным числом~(собственным значением, eigenvalue, proper value, главным значением)~${^2\!\bm{B}}$, а~определяемая собственным вектором~$\bm{a}$ ось~(направление)\:--- его собственной~(главной, principal) осью~(направлением).

В~компонентах это матричная задача на~собственные значения ${( B_{i\hspace{-0.1ex}j} \hspace{-0.12ex} - \eigenvalue \delta_{i\hspace{-0.1ex}j} ) \hspace{.25ex} a_j \hspace{-0.1ex} = 0}$\:--- однородная линейная алгебраическая система, имеющая ненулевые решения при~равенстве нулю определителя ${\smash{\underset{\raisebox{.15em}{\scalebox{0.7}{$i$,$\hspace{.15ex}j$}}}{\operatorname{det}}} \hspace{.32ex} ( B_{i\hspace{-0.1ex}j} \hspace{-0.12ex} - \eigenvalue \delta_{i\hspace{-0.1ex}j} )}$:

\nopagebreak\vspace{-0.1em}\begin{equation}\label{chardetequation}
\operatorname{det} \hspace{-0.4ex} \scalebox{0.9}[0.92]{$\left[
\begin{array}{ccc}
B_{1\hspace{-0.12ex}1} \hspace{-0.16ex} - \eigenvalue & B_{12} & B_{13} \\
B_{21} & B_{22} \hspace{-0.16ex} - \eigenvalue & B_{23} \\
B_{31} & B_{32} & B_{33} \hspace{-0.16ex} - \eigenvalue
\end{array}
\hspace{.1ex}\right]$} \hspace{-0.4ex} = - \hspace{.2ex} \eigenvalue^3 \hspace{-0.25ex} + \mathrm{I}\hspace{.25ex} \eigenvalue^2 \hspace{-0.25ex} - \mathrm{II}\hspace{.25ex} \eigenvalue + \mathrm{III} = 0 \hspace{.2ex} ;
\end{equation}

\vspace{-0.25em}\begin{equation}\label{invariants:2}
\begin{array}{r@{\hspace{.4em}}c@{\hspace{.4em}}l}
\mathrm{I} & = & \operatorname{tr}\hspace{.1ex} {^2\!\bm{B}} = B_{kk}
= \scalebox{0.92}[0.96]{$B_{1\hspace{-0.12ex}1} \hspace{-0.2ex} + \hspace{-0.1ex} B_{22} \hspace{-0.2ex} + \hspace{-0.1ex} B_{33}$}
\hspace{.1ex} ,
\\[.1em]
%
\mathrm{II} & = & \scalebox{0.92}[0.96]{$B_{1\hspace{-0.12ex}1}B_{22} \hspace{-0.2ex} - \hspace{-0.1ex} B_{12}B_{21} \hspace{-0.2ex} + \hspace{-0.1ex} B_{1\hspace{-0.12ex}1}B_{33} \hspace{-0.2ex} - \hspace{-0.1ex} B_{13}B_{31} \hspace{-0.2ex} + \hspace{-0.1ex} B_{22}B_{33} \hspace{-0.2ex} - \hspace{-0.1ex} B_{23}B_{32}$}
\hspace{.1ex} ,
\\[.1em]
%
\mathrm{III} & = & \operatorname{det} \hspace{.1ex} {^2\!\bm{B}}
= \hspace{.15ex} \underset{\raisebox{.15em}{\scalebox{0.7}{$i$,$\hspace{.15ex}j$}}}{\operatorname{det}} \, B_{i\hspace{-0.1ex}j} \hspace{-0.15ex}
= e_{i\hspace{-0.1ex}j\hspace{-0.1ex}k} \hspace{.1ex} B_{1i} B_{2j} B_{3k} \hspace{-0.1ex}
= e_{i\hspace{-0.1ex}j\hspace{-0.1ex}k} \hspace{.1ex} B_{i1} B_{\hspace{-0.1ex}j2} B_{k3}
\hspace{.1ex} .
\end{array}\end{equation}

\vspace{-0.4em} \en{Roots}\ru{Корни} \en{of characteristic equation}\ru{характеристического уравнения}~\eqref{chardetequation}\:--- собственные числа\:--- не~зависят от~базиса и~потому инвариантны.

Коэффициенты~\eqref{invariants:2} тоже не~зависят от~базиса; они называются первым, вторым и~третьим инвариантами тензора.
С~первым инвариантом~${\mathrm{I}}$\:--- следом тензора\:--- мы уже встречались в~\pararef{para:operationswithtensors}.
Второй инвариант~${\mathrm{II}}$ это след союзной~(взаимной, adjugate) матрицы\:--- транспонированной матрицы дополнений:
${\mathrm{II}\hspace{.16ex}(\hspace{.1ex}{^2\!\bm{B}}) \hspace{-0.12ex}
\equiv \hspace{.1ex} \operatorname{tr} \left({\operatorname{adj}{B_{i\hspace{-0.1ex}j}}}\right)}$.
Или он~же

\nopagebreak\vspace{-0.22em}\begin{equation*}
\mathrm{II}\hspace{.16ex}(\hspace{.1ex}{^2\!\bm{B}}) \hspace{-0.2ex}
\equiv \hspace{.12ex} \smalldisplaystyleonehalf \scalebox{0.8}[0.82]{$\left[
\left(\operatorname{tr}\hspace{.1ex} {^2\!\bm{B}}\right)^{\!2} \hspace{-0.5ex}
- \operatorname{tr} \hspace{-0.33ex} \left( \hspace{.12ex} {^2\!\bm{B}} \hspace{-0.2ex} \dotp \hspace{-0.2ex} {^2\!\bm{B}} \right) \right]$} \hspace{-0.5ex}
= \smalldisplaystyleonehalf \scalebox{0.8}[0.82]{$\left[
\left(\operatorname{tr}\hspace{.1ex} {^2\!\bm{B}}\right)^{\!2} \hspace{-0.5ex}
- {^2\!\bm{B}} \hspace{-0.2ex} \dotdotp \hspace{-0.2ex} {^2\!\bm{B}} \hspace{.16ex}
\right]$} \hspace{-0.5ex}
= \smalldisplaystyleonehalf \scalebox{0.8}[0.82]{$\left[
\left( B_{kk} \right)^{\hspace{-0.1ex}2} \hspace{-0.4ex}
- \hspace{-0.2ex} B_{i\hspace{-0.1ex}j} B_{\hspace{-0.1ex}j\hspace{-0.06ex}i} \hspace{.12ex}
\right]$}
\hspace{-0.25ex} .
\end{equation*}

\vspace{-0.3em} \noindent
И~третий инвариант~${\mathrm{III}}$ это определитель~(детерминант) компонент тензора:
${\mathrm{III}\hspace{.16ex}(\hspace{.1ex}{^2\!\bm{B}}) \hspace{-0.12ex}
\equiv \operatorname{det}\hspace{.1ex} {^2\!\bm{B}}}$.

Это относилось ко~всем тензорам второй сложности. Для~случая~же симметричного тензора справедливо следующее:\\
\indent 1$^{\circ}$\hspace{-1ex}.\, Собственные числа симметричного тензора вещественны.\\
\indent 2$^{\circ}$\hspace{-1ex}.\, Собственные оси для~разных собственных чисел ортогональны.

\noindent
${ \tikz[baseline=-1ex] \draw [line width=.5pt, color=black, fill=white] (0, 0) circle (.8ex);
\hspace{.6em} }$
Первое утверждение доказывается от~противного. Если~$\eigenvalue$\:--- компл\'{е}ксный корень~\eqref{chardetequation}, определяющий собственный вектор~$\bm{a}$, то сопряжённое число~$\lineover{\eigenvalue}$ также будет корнем. Ему соответствует собственный вектор~$\lineover{\bm{a}}$ с~сопряжёнными компонентами. При этом

\nopagebreak\vspace{-0.5em}\begin{multline*}
\eqref{eigenvalues:eq}
\:\,\Rightarrow\:\,
\left(\hspace{.25ex} {\lineover{\bm{a}} \, \dotp} \hspace{.25ex}\right) \hspace{.5ex} {^2\!\bm{B}} \dotp \bm{a} \hspace{.2ex} = \eigenvalue \bm{a}, \;\;
\left(\hspace{.2ex}{\bm{a} \, \dotp}\hspace{.2ex}\right) \hspace{.5ex} {^2\!\bm{B}} \dotp \lineover{\bm{a}} \hspace{.2ex} = \hspace{.1ex} \lineover{\eigenvalue} \, \lineover{\bm{a}}
\:\;\Rightarrow \\[-0.1em]
%
\Rightarrow\;\: \lineover{\bm{a}} \dotp {^2\!\bm{B}} \dotp \bm{a} - \bm{a} \dotp {^2\!\bm{B}} \dotp \lineover{\bm{a}} \hspace{.2ex} = \left(\hspace{.1ex}{\eigenvalue - \lineover{\eigenvalue}}\hspace{.16em}\right)\hspace{-0.1em} \bm{a} \dotp \lineover{\bm{a}} \hspace{.2ex} .
\end{multline*}

\vspace{-0.16em} \noindent Но слева здесь\:--- нуль, поскольку ${\bm{a} \dotp {^2\!\bm{B}} \dotp \bm{c} = \bm{c} \dotp {^2\!\bm{B}^{\T}\!} \dotp \bm{a}}$ и~${{^2\!\bm{B}} = {^2\!\bm{B}^{\T}\!}}$. Поэтому ${\eigenvalue = \lineover{\eigenvalue}}$, то~есть вещественно.

Столь~же просто обосновывается и~2$^{\circ}$:

\nopagebreak\vspace{-0.5em}\begin{multline*}
\tikzmark{BeginEqualsZeroBrace} {\bm{a}_2 \dotp {^2\!\bm{B}} \dotp \bm{a}_1 - \bm{a}_1 \dotp {^2\!\bm{B}} \dotp \bm{a}_2} \hspace{.1ex} \tikzmark{EndEqualsZeroBrace}
= \hspace{.12ex} \left(\hspace{.1ex}{\eigenvalue_1 \hspace{-0.25ex} - \eigenvalue_2}\right) \bm{a}_1 \hspace{-0.1ex} \dotp \hspace{.1ex} \bm{a}_2 \hspace{.1ex} , \;
\eigenvalue_1 \hspace{-0.1ex} \neq \hspace{.1ex} \eigenvalue_2
\;\Rightarrow
\\
%
\Rightarrow\;
\bm{a}_1 \hspace{-0.1ex} \dotp \hspace{.1ex} \bm{a}_2 = 0
\hspace{.1ex} .
\hspace{4em} \tikz[baseline=-0.6ex] \draw [color=black, fill=black] (0, 0) circle (.8ex);
\end{multline*}
\AddUnderBrace[line width=.75pt][0.1ex,-0.1ex]%
{BeginEqualsZeroBrace}{EndEqualsZeroBrace}{${\scriptstyle =\;0}$}

\vspace{-1.4em} При~различных собственных числах собственные векторы единичной длины~${\mathboldae_i}$ образуют ортонормальный базис. Каков\'{ы}~же в~нём компоненты тензора?

\nopagebreak\vspace{-0.25em}\begin{equation*}\begin{array}{c}
{^2\!\bm{B}} \dotp \mathboldae_k = \hspace{.2ex} \scalebox{0.82}{$\tikzcancel[blue]{$\displaystyle\sum_k^{~}$}$}\: \eigenvalue_{\hspace{-0.15ex}k} \hspace{.1ex} \mathboldae_k \hspace{.1ex}, \:\: k =\hspace{-0.1ex} 1, 2, 3
\\[.9em]
%
{^2\!\bm{B}} \dotp \tikzmark{BeginEqualsEBrace} {\mathboldae_k \mathboldae_k} \tikzmark{EndEqualsEBrace} = \hspace{-0.1ex} \scalebox{0.9}{$\displaystyle \sum_{\smash{k}}$} \hspace{.2ex} \eigenvalue_{\hspace{-0.15ex}k} \hspace{.1ex} \mathboldae_k \mathboldae_k
%%{^2\!\bm{B}} = \displaystyle \sum_{\smash{k}} \eigenvalue_k \mathboldae_k \mathboldae_k
\end{array}\end{equation*}
\AddUnderBrace[line width=.75pt]%
{BeginEqualsEBrace}{EndEqualsEBrace}{${\scriptstyle \bm{E}}$}

\vspace{-0.25em} В~общем случае ${B_{i\hspace{-0.1ex}j} \hspace{-0.2ex} = \bm{e}_i \dotp {^2\!\bm{B}} \dotp \bm{e}_{\hspace{-0.1ex}j}}$, в~базисе~же ${\mathboldae_1}$, ${\mathboldae_2}$, ${\mathboldae_3}$ единичных взаимно ортогональных ${\mathboldae_i \dotp \mathboldae_{\hspace{-0.1ex}j} \hspace{-0.15ex} = \delta_{i\hspace{-0.1ex}j}}$ собственных осей симметричного тензора:

\nopagebreak\vspace{-0.2em}\begin{equation*}\begin{array}{c}
B_{1\hspace{-0.12ex}1} \hspace{-0.12ex} = \mathboldae_1 \dotp \left({\eigenvalue_1 \mathboldae_1 \mathboldae_1 + \eigenvalue_2 \mathboldae_2 \mathboldae_2 + \eigenvalue_3 \mathboldae_3 \mathboldae_3}\right) \dotp \mathboldae_1 = \eigenvalue_1
\hspace{.1ex} ,
\\[.1em]
B_{12} \hspace{-0.12ex} = \mathboldae_1 \dotp \left({\eigenvalue_1 \mathboldae_1 \mathboldae_1 + \eigenvalue_2 \mathboldae_2 \mathboldae_2 + \eigenvalue_3 \mathboldae_3 \mathboldae_3}\right) \dotp \mathboldae_2 = 0
\hspace{.1ex} ,
\\[-0.2em]
\ldots
\end{array}\end{equation*}

\vspace{-0.3em} \noindent
Матрица компонент диагональна и~${{^2\!\bm{B}} = \sum \hspace{-0.2ex} \eigenvalue_{\hspace{-0.15ex}i} \hspace{.1ex} \mathboldae_i \mathboldae_i}$. Здесь идёт суммирование по~трём повторяющися индексам, ведь используется особенный базис.

Случай кратных главных значений можно рассмотреть с~помощью предельного перехода. При~${\eigenvalue_2 \hspace{-0.16ex} \to \eigenvalue_1}$ любая линейная комбинация ${\bm{a}_1}$ и~${\bm{a}_2}$ в~пределе удовлетворяет~\eqref{eigenvalues:eq}; это значит, что любая ось в~плоскости~${\bm{a}_1, \bm{a}_2}$ становится собственной. Если~же совпадают все три собственных числ\'{а}, то любая ось в~пространстве\:--- собственная. При~этом ${{^2\!\bm{B}} = \eigenvalue \bm{E}}$, такие тензоры называются изотропными или шаровыми.

\end{otherlanguage}

\en{\section{Rotation tensor}}

\ru{\section{Тензор поворота}}

\label{para:rotationtensor}

\begin{otherlanguage}{russian}

Соотношение между двумя \inquotes{правыми}~(или двумя \inquotes{левыми}) орто\-нормаль\-ными базисами ${\bm{e}_i}$ и~${\mathcircabove{\bm{e}}_i}$ вполне определено матрицей косинусов~(\pararef{para:vectors})
\nopagebreak\vspace{-0.5em}\[
\bm{e}_i \hspace{-0.2ex} = \bm{e}_i \dotp \hspace{.1ex} \tikzmark{beginEqualsE} \mathcircabove{\bm{e}}_j \mathcircabove{\bm{e}}_j \tikzmark{endEqualsE} \hspace{-0.1ex} = \hspace{.1ex} \cosinematrix{\hspace{-0.2ex}i\mathcircabove{j}} \, \mathcircabove{\bm{e}}_j , \:\:
\cosinematrix{\hspace{-0.2ex}i\mathcircabove{j}} \hspace{.1ex} \equiv \hspace{.1ex} \bm{e}_i \dotp \mathcircabove{\bm{e}}_j
\hspace{.1ex} .
\]
\AddUnderBrace[line width=.75pt][0,-0.2ex]%
{beginEqualsE}{endEqualsE}{${\scriptstyle \bm{E}}$}

\vspace{-0.32em} \noindent \en{But one may write like this:}\ru{Но~можно напис\'{а}ть и~так:}
\begin{equation}\label{introductionofrotationtensor}
\bm{e}_i = \bm{e}_j \hspace{.16ex} \tikzmark{beginEqualsKroneckerDelta} \mathcircabove{\bm{e}}_j \hspace{-0.16ex} \dotp \mathcircabove{\bm{e}}_i \tikzmark{endEqualsKroneckerDelta} = \bm{P} \hspace{-0.16ex} \dotp \mathcircabove{\bm{e}}_i \hspace{.1ex}, \:\,
\bm{P} \equiv \bm{e}_j \mathcircabove{\bm{e}}_j = \scalebox{0.96}[1]{$\bm{e}_1 \hspace{-0.1ex} \mathcircabove{\bm{e}}_1 + \bm{e}_2 \mathcircabove{\bm{e}}_2 + \bm{e}_3 \mathcircabove{\bm{e}}_3$}
\hspace{.1ex} .
\end{equation}
\AddUnderBrace[line width=.75pt][0.1ex,-0.2ex]%
{beginEqualsKroneckerDelta}{endEqualsKroneckerDelta}{${\scriptstyle \delta_{j\hspace{-0.06ex}i}}$}

\vspace{-0.32em} \noindent \en{${\bm{P}}$ is called rotation tensor.}\ru{${\bm{P}}$ называется тензором поворота.}

Компоненты ${\bm{P}}$ и~в~начальном~${\mathcircabove{\bm{e}}_i}$, и~в~повёрнутом~${\bm{e}_i}$ базисах образуют одну и~ту~же матрицу, равную транспонированной матрице косинусов~${\cosinematrix{\!j\mathcircabove{i}} = \hspace{.12ex} \mathcircabove{\bm{e}}_i \dotp \bm{e}_j}$\hspace{.1ex}:
\begin{equation}\label{componentsofrotationtensor}
\begin{array}{c}
\bm{e}_i \dotp \bm{P} \hspace{-0.12ex} \dotp \bm{e}_j =
\hspace{.2ex} \tikzmark{beginEqualsKroneckerDeltaPresent} \bm{e}_i \hspace{-0.1ex} \dotp \bm{e}_k \tikzmark{endEqualsKroneckerDeltaPresent} \hspace{.2ex} \mathcircabove{\bm{e}}_k \hspace{-0.12ex} \dotp \bm{e}_j =
\hspace{.2ex} \mathcircabove{\bm{e}}_i \hspace{-0.12ex} \dotp \bm{e}_j , \\[1.4em]
%
\mathcircabove{\bm{e}}_i \dotp \bm{P} \hspace{-0.12ex} \dotp \mathcircabove{\bm{e}}_j =
\hspace{.2ex} \mathcircabove{\bm{e}}_i \hspace{-0.1ex} \dotp \bm{e}_k \hspace{.2ex} \tikzmark{beginEqualsKroneckerDeltaPast} \mathcircabove{\bm{e}}_k \hspace{-0.12ex} \dotp \mathcircabove{\bm{e}}_j \tikzmark{endEqualsKroneckerDeltaPast} =
\hspace{.2ex} \mathcircabove{\bm{e}}_i \hspace{-0.12ex} \dotp \bm{e}_j , \\[1.5em]
%
\bm{P} = \cosinematrix{\!j\mathcircabove{i}} \hspace{.5ex} {\bm{e}}_i {\bm{e}}_j = \cosinematrix{\!j\mathcircabove{i}} \hspace{.5ex} \mathcircabove{\bm{e}}_i \mathcircabove{\bm{e}}_j
\hspace{.1ex} .
\end{array}
\end{equation}
\AddUnderBrace[line width=.75pt][0.1ex,-0.2ex]%
{beginEqualsKroneckerDeltaPresent}{endEqualsKroneckerDeltaPresent}{${\scriptstyle \delta_{ik}}$}
\AddUnderBrace[line width=.75pt][0,-0.2ex]%
{beginEqualsKroneckerDeltaPast}{endEqualsKroneckerDeltaPast}{${\scriptstyle \delta_{kj}}$}

\vspace{-0.4em} Тензор~$\bm{P}$ связывает два вектора\:--- \inquotes{до~поворота}~${\mathcircabove{\bm{r}} = \rho_i \mathcircabove{\bm{e}}_i}$ и~\inquotes{после~поворота}~${\bm{r} = \rho_i \bm{e}_i}$\:--- с~теми~же компонентами~$\rho_i$ у~$\bm{r}$ в~актуальном повёрнутом базисе~${\bm{e}_i}$, что~и~у~${\mathcircabove{\bm{r}}}$ в~неподвижном базисе~${\mathcircabove{\bm{e}}_i}$~(\inquotes{вектор вращается вместе с~базисом}): поскольку ${\bm{e}_i = \bm{e}_j \mathcircabove{\bm{e}}_j \dotp \mathcircabove{\bm{e}}_i \:\Leftrightarrow\, \rho_i \bm{e}_i = \bm{e}_j \mathcircabove{\bm{e}}_j \dotp \rho_i \mathcircabove{\bm{e}_i}}$, то

\nopagebreak\begin{equation}\label{rodriguesrotationformula}
\bm{r} = \bm{P} \dotp\hspace{.2ex} \mathcircabove{\bm{r}}
\end{equation}

\vspace{-0.4em} \noindent(эта связь\:--- обобщённая \href{https://fr.wikipedia.org/wiki/Rotation_vectorielle#Cas_g%C3%A9n%C3%A9ral}{формула поворота Rodrigues’а}).

\vspace{-0.1em}Поворот~же тензора второй сложности ${\mathcircabove{\bm{C}} = C_{i\hspace{-0.1ex}j} \mathcircabove{\bm{e}}_i \mathcircabove{\bm{e}}_j}$ в~текущее~(актуальное) положение~${\bm{C} = C_{i\hspace{-0.1ex}j} \bm{e}_i \bm{e}_j}$ происходит так:
\begin{equation}
C_{i\hspace{-0.1ex}j} \bm{e}_i \bm{e}_j = \bm{e}_i \mathcircabove{\bm{e}}_i \dotp C_{pq} \mathcircabove{\bm{e}}_p \mathcircabove{\bm{e}}_q \dotp \mathcircabove{\bm{e}}_j \bm{e}_j \;\Leftrightarrow\; \bm{C} = \bm{P} \dotp\hspace{.1ex} \mathcircabove{\bm{C}} \dotp \bm{P}^{\T} \hspace{-0.32ex}.
\end{equation}

\end{otherlanguage}

\en{Essential property}\ru{Существенное свойство} \en{of a~rotation tensor}\ru{тензора поворота}\:--- \en{orthogonality}\ru{ортогональность}\:--- \en{is expressed as}\ru{выражается как}

\nopagebreak\en{\vspace{-0.8em}}\ru{\vspace{-0.25em}}
\begin{equation}\label{orthogonalityofrotationtensor}
\aunderbrace[l1r]{\bm{P}}_{\bm{e}_i \mathcircabove{\bm{e}}_i} \hspace{.1em} \dotp \hspace{.1em} \aunderbrace[l1r]{\bm{P}^{\T}\hspace{-0.2em}}_{\mathcircabove{\bm{e}}_j \bm{e}_j}
\hspace{.2ex} = \hspace{.2ex}
\aunderbrace[l1r]{\bm{P}^{\T}\hspace{-0.2em}}_{\mathcircabove{\bm{e}}_i \bm{e}_i} \hspace{.2em} \dotp \aunderbrace[l1r]{\bm{P}}_{\bm{e}_j \mathcircabove{\bm{e}}_j}
\hspace{-0.2ex} = \hspace{.3ex}
\aunderbrace[l1r]{\hspace{.2ex}\bm{E}\hspace{.2ex}}_{\mathclap{\begin{subarray}{l} \mathcircabove{\bm{e}}_i \mathcircabove{\bm{e}}_i \\ \bm{e}_i \bm{e}_i \end{subarray}}}
\hspace{.1em} ,
\end{equation}

\vspace{-0.1em} \noindent \en{that~is}\ru{то~есть} \en{the~transposed tensor}\ru{транспонированный тензор} \en{coincides}\ru{совпадает} \en{with}\ru{с}~\en{the~reciprocal tensor}\ru{обратным тензором}: ${\bm{P}^{\T} \hspace{-0.32ex}= \bm{P}^{\expminusone} \Leftrightarrow \bm{P} = \bm{P}^{\expminusT}\hspace{-0.25ex}}$.

\begin{otherlanguage}{russian}

Ортогональный тензор не~меняет скалярное произведение векторов, сохраняя дл\'{и}ны и~углы (\inquotes{метрику})
%% ${\bm{e}_i \dotp \bm{e}_j \hspace{-0.1ex} = \hspace{.1ex} \mathcircabove{\bm{e}}_i \dotp \mathcircabove{\bm{e}}_j}$
\vspace{.1em}\begin{equation}\label{rotationtensorkeepsmetrics}
\left( \bm{P} \hspace{-0.2ex} \dotp \bm{a}\hspace{.2ex} \right) \dotp \left( \bm{P} \hspace{-0.2ex} \dotp \bm{b}\hspace{.2ex} \right) =
\bm{a} \dotp \bm{P}^{\T} \hspace{-0.4ex} \dotp \bm{P} \hspace{-0.1ex} \dotp \hspace{.1ex} \bm{b} = \bm{a} \dotp \bm{E} \dotp \bm{b} = \bm{a} \dotp \bm{b}
\hspace{.1ex} .
\end{equation}

Для всех ортогональных тензоров ${\left(\operatorname{det} \rotationtensor\hspace{.1ex}\right)^2 \hspace{-0.1ex} = 1}$:

\nopagebreak\vspace{-0.1em}\begin{equation*}
1 = \operatorname{det} \bm{E} = \operatorname{det} \left({\hspace{-0.1ex} \rotationtensor \dotp \rotationtensor^{\T} \hspace{.2ex}}\right) \hspace{-0.3ex}
= \left({\operatorname{det} \rotationtensor \hspace{.1ex}}\right) \left({\operatorname{det} \rotationtensor^{\T} \hspace{.2ex}}\right) \hspace{-0.3ex}
= \left(\operatorname{det} \rotationtensor \hspace{.1ex} \right)^2
\hspace{-0.1ex} \!.
\end{equation*}

Тензор поворота это ортогональный тензор \en{with}\ru{с}~${\operatorname{det} \bm{P} \hspace{-0.1ex} = 1}$.
Но не~только лишь тензоры поворота обладают свойством ортогональности.
\en{When}\ru{Когда} в~\eqref{introductionofrotationtensor} один из базисов \inquotesx{правый}[,] а~другой \inquotesx{левый}[,] \en{then it’s}\ru{тогда это} \en{combination of a~rotation and a~reflection}\ru{комбинация поворота и~отражения} (\inquotes{rotoreflection}) ${\rotationtensor = -\bm{E} \dotp \bm{P}}$ \en{with}\ru{с}~${\operatorname{det} \left( -\bm{E} \dotp \bm{P} \hspace{.16ex} \right) \hspace{-0.1ex} = -1}$.

У~любого бивалентного тензора в~трёхмерном пространстве как минимум одно собственное число\:--- корень~\eqref{chardetequation}\:--- действительное~(некомпл\'{е}ксное).
Для тензора поворота оно равно единице

\nopagebreak\vspace{-0.1em}\begin{equation*}\begin{array}{c}
\bm{P} \dotp \bm{a} = \eigenvalue \bm{a} \:\Rightarrow\:
\tikzmark{BeginPaBrace} \bm{a} \hspace{.16ex} \dotp \tikzmark{BeginEBrace} \bm{P}^{\T} \tikzmark{EndPaBrace} \hspace{-0.4ex} \dotp \bm{P} \hspace{-0.32ex}\tikzmark{EndEBrace}\hspace{.32ex} \dotp \hspace{.16ex} \bm{a} = \eigenvalue \bm{a} \dotp \eigenvalue \bm{a}
\:\Rightarrow\: \eigenvalue^{2} \hspace{-0.2ex} = 1 \hspace{.1ex} .
\end{array}\end{equation*}
\AddOverBrace[line width=.75pt][0.1ex,0.4ex]{BeginPaBrace}{EndPaBrace}{${\scriptstyle \bm{P} \:\dotp\; \bm{a}}$}
\AddUnderBrace[line width=.75pt][-0.1ex,0.1ex]{BeginEBrace}{EndEBrace}{${\scriptstyle \bm{E}}$}

\vspace{-0.5em} \noindent Соответствующая собственная ось называется осью поворота; теорема Euler’а о~конечном повороте в~том и~состоит, что такая ось существует. Если ${\bm{k}}$\:--- орт этой оси, а~${\vartheta}$\:--- величина угла поворота, то тензор поворота представ\'{и}м как

\nopagebreak\vspace{-0.1em}\begin{equation}\label{eulerfiniterotation}
\bm{P}\hspace{.1ex}(\bm{k},\vartheta) = \bm{E} \operatorname{cos} \vartheta + \bm{k} \times\hspace{-0.2ex} \bm{E} \operatorname{sin} \vartheta + \bm{k} \bm{k} \left({1 - \operatorname{cos} \vartheta}\right) \hspace{-0.2ex} .
\end{equation}

\vspace{-0.1em} Доказывается эта формула так. Направление~${\bm{k}}$ при~повороте не~меняется~(${\bm{P} \hspace{-0.2ex}\dotp \bm{k} = \bm{k}\hspace{.12ex}}$), поэтому на~оси поворота ${\mathcircabove{\bm{e}}_3 \hspace{-0.16ex} = \bm{e}_3 \hspace{-0.16ex} = \bm{k}}$. В~перпендикулярной плоскости~(\figref{fig:eulerfiniterotation}) ${\mathcircabove{\bm{e}}_1 \hspace{-0.16ex} = \bm{e}_1 \operatorname{cos} \vartheta - \bm{e}_2 \operatorname{sin} \vartheta}$, ${\mathcircabove{\bm{e}}_2 \hspace{-0.16ex} = \bm{e}_1 \operatorname{sin} \vartheta + \bm{e}_2 \operatorname{cos} \vartheta}$, ${\bm{P} = \bm{e}_i \mathcircabove{\bm{e}}_i \,\Rightarrow\hspace{.2ex}}$~\eqref{eulerfiniterotation}.

% ~ ~ ~ ~ ~
\begin{figure}[!htbp]

\vspace*{-0.5em}\[
\mathcircabove{\bm{e}}_i = \mathcircabove{\bm{e}}_i \dotp \bm{e}_j \bm{e}_j
\]

\vspace{-1.5em}\[
\left[ \begin{array}{c} \mathcircabove{\bm{e}}_1 \\ \mathcircabove{\bm{e}}_2 \\ \mathcircabove{\bm{e}}_3 \end{array} \right] =
\left[ \begin{array}{ccc}
\mathcircabove{\bm{e}}_1 \dotp \bm{e}_1 & \mathcircabove{\bm{e}}_1 \dotp \bm{e}_2 & \mathcircabove{\bm{e}}_1 \dotp \bm{e}_3 \\
\mathcircabove{\bm{e}}_2 \dotp \bm{e}_1 & \mathcircabove{\bm{e}}_2 \dotp \bm{e}_2 & \mathcircabove{\bm{e}}_2 \dotp \bm{e}_3 \\
\mathcircabove{\bm{e}}_3 \dotp \bm{e}_1 & \mathcircabove{\bm{e}}_3 \dotp \bm{e}_2 & \mathcircabove{\bm{e}}_3 \dotp \bm{e}_3
\end{array} \right] \hspace{-0.5ex}
\left[ \hspace{-0.12ex} \begin{array}{c} {\bm{e}}_1 \\ {\bm{e}}_2 \\ {\bm{e}_3} \end{array} \right]
\]

\vspace{-1.25em}

\begin{center}
\tdplotsetmaincoords{60}{120} % set orientation of axes
\pgfmathsetmacro{\angletheta}{42}
% three parameters for vector
\pgfmathsetmacro{\lengthofvector}{0.55}
\pgfmathsetmacro{\anglefromz}{40}
\pgfmathsetmacro{\anglefromx}{240}

\begin{tikzpicture}[scale=4, tdplot_main_coords] % tdplot_main_coords style to use 3dplot

	\coordinate (O) at (0,0,0);

	% draw initial axes
	\draw [line width=1.2pt, black, -{Stealth[round, length=4mm, width=2.4mm]}]
		(O) -- (1,0,0)
		node[pos=0.9, above, xshift=-0.8em] {$\mathcircabove{\bm{e}}_1$};

	\draw [line width=1.2pt, black, -{Stealth[round, length=4mm, width=2.4mm]}]
		(O) -- (0,1,0)
		node[pos=0.9, above, xshift=1em, yshift=-0.2em] {$\mathcircabove{\bm{e}}_2$};

	\draw [line width=1.2pt, red, -{Stealth[round,length=4mm,width=2.4mm]}]
		(O) -- (0,0,0.9)
		node[anchor=south] {$\mathcircabove{\bm{e}}_3 = \bm{e}_3 = \bm{k}$};

	% draw initial vector
	\tdplotsetcoord{point}{\lengthofvector}{\anglefromz}{\anglefromx} % {length}{angle from z}{angle from x}
		% it also defines (pointxy), (pointxz), and (pointyz) projections of point
	\draw [line width=1.2pt, black, -{Stealth[round, length=4mm, width=2.4mm]}]
		(O) -- (point)
		node[anchor=south] {$\mathcircabove{\bm{r}}$};
	% draw its projection on xy plane
	\draw [line width=0.4pt, dotted, color=black] (O) -- (pointxy);
	\draw [line width=0.4pt, dotted, color=black] (pointxy) -- (point);

	% draw the angle, and label it
	% syntax: \tdplotdrawarc[coordinate frame, draw options]{center point}{r}{angle}{end angle}{label options}{label}
	\tdplotdrawarc [line width=0.5pt, red, ->]
		{(O)}{0.4}{0}{\angletheta}{anchor=north}{$\vartheta$}
	\tdplotdrawarc [line width=0.5pt, red, ->]
		{(O)}{0.4}{90}{90+\angletheta}{anchor=west}{$\vartheta$}

	% rotate coordinates using Euler angles "z(\alpha)y(\beta)z(\gamma)"
	\tdplotsetrotatedcoords{\angletheta}{0}{0}

	% draw rotated axes
	\draw [line width=1.2pt, blue, tdplot_rotated_coords, -{Stealth[round, length=4mm, width=2.4mm]}]
		(O) -- (1,0,0)
		node[pos=0.9, left, xshift=-0.1em] {$\bm{e}_1$};

	\draw [line width=1.2pt, blue, tdplot_rotated_coords, -{Stealth[round, length=4mm, width=2.4mm]}]
		(O) -- (0,1,0)
		node[pos=0.9, above, xshift=0.2em, yshift=0.2em] {$\bm{e}_2$};

	%%\draw [line width=1.2pt, blue, tdplot_rotated_coords, -{Stealth[round, length=4mm, width=2.4mm]}]
		%%(O) -- (0,0,0.8) ;

	% draw rotated vector
	\tdplotsetcoord{rotatedpoint}%
		{\lengthofvector}{\anglefromz}{\anglefromx+\angletheta}
	\draw [line width=1.2pt, blue, tdplot_rotated_coords, -{Stealth[round, length=4mm, width=2.4mm]}]
		(O) -- (rotatedpoint)
		node[anchor=south] {$\bm{r}$};
	% draw its projection on xy plane
	\draw [line width=0.4pt, dotted, color=blue, tdplot_rotated_coords] (O) -- (rotatedpointxy);
	\draw [line width=0.4pt, dotted, color=blue, tdplot_rotated_coords] (rotatedpointxy) -- (rotatedpoint);

	\tdplotdrawarc [line width=0.5pt, red, ->]
		{(O)}{0.28}{\anglefromx}{\anglefromx+\angletheta}{anchor=south east, xshift=0.3em, yshift=-0.1em}{$\vartheta$}

\end{tikzpicture}
\end{center}

\vspace{-1em}\[
\scalebox{0.8}[0.85]{$\left[ \begin{array}{ccc}
\mathcircabove{\bm{e}}_1 \dotp \bm{e}_1 & \mathcircabove{\bm{e}}_1 \dotp \bm{e}_2 & \mathcircabove{\bm{e}}_1 \dotp \bm{e}_3 \\
\mathcircabove{\bm{e}}_2 \dotp \bm{e}_1 & \mathcircabove{\bm{e}}_2 \dotp \bm{e}_2 & \mathcircabove{\bm{e}}_2 \dotp \bm{e}_3 \\
\mathcircabove{\bm{e}}_3 \dotp \bm{e}_1 & \mathcircabove{\bm{e}}_3 \dotp \bm{e}_2 & \mathcircabove{\bm{e}}_3 \dotp \bm{e}_3
\end{array} \right]$} \hspace{-0.32ex} = \hspace{-0.2ex}
%
\scalebox{0.8}[0.85]{$\left[ \hspace{-0.2ex} \begin{array}{ccc}
\operatorname{cos} \vartheta & \hspace{-1ex} \operatorname{cos} \left( 90\degree \!+ \vartheta \right) & \operatorname{cos} 90\degree \\
\operatorname{cos} \left( 90\degree \!- \vartheta \right) & \operatorname{cos} \vartheta & \operatorname{cos} 90\degree \\
\operatorname{cos} 90\degree & \operatorname{cos} 90\degree & \operatorname{cos} 0\degree
\end{array} \right]$} \hspace{-0.32ex} = \hspace{-0.2ex}
%
\scalebox{0.8}[0.85]{$\left[ \hspace{-0.1ex} \begin{array}{ccc}
\operatorname{cos} \vartheta & - \operatorname{sin} \vartheta & 0 \\
\operatorname{sin} \vartheta & \operatorname{cos} \vartheta & 0 \\
0 & 0 & 1
\end{array} \right]$}
\]

\vspace{-0.8em}
\[\begin{array}{c}
\mathcircabove{\bm{e}}_1 \hspace{-0.16ex} = \bm{e}_1 \operatorname{cos} \vartheta \hspace{0.1ex} - \hspace{0.1ex} \bm{e}_2 \operatorname{sin} \vartheta \\[0.1em]
\mathcircabove{\bm{e}}_2 \hspace{-0.16ex} = \bm{e}_1 \operatorname{sin} \vartheta \hspace{0.1ex} + \hspace{0.1ex} \bm{e}_2 \operatorname{cos} \vartheta \\[0.1em]
\mathcircabove{\bm{e}}_3 \hspace{-0.16ex} = \bm{e}_3 = \bm{k}
\end{array}\]

\vspace{-1em}
\begin{multline*}
\shoveleft{ \bm{P} = \bm{e}_1 \hspace{-0.1ex} \mathcircabove{\bm{e}}_1 + \bm{e}_2 \mathcircabove{\bm{e}}_2 + \bm{e}_3 \mathcircabove{\bm{e}}_3 = \hfill }\\[1.5em]
%
= \hspace{0.2ex} \tikzmark{StartBraceE1E1} {\bm{e}_1 \bm{e}_1 \operatorname{cos} \vartheta - \bm{e}_1 \bm{e}_2 \operatorname{sin} \vartheta \hspace{0.2em}} \tikzmark{EndBraceE1E1} \hspace{-0.1ex} + \hspace{0.1ex} \tikzmark{StartBraceE2E2} {\bm{e}_2 \bm{e}_1 \operatorname{sin} \vartheta + \bm{e}_2 \bm{e}_2 \operatorname{cos} \vartheta \hspace{0.2em}} \tikzmark{EndBraceE2E2} \hspace{-0.1ex} + \tikzmark{StartBraceE3E3} {\hspace{0.25ex} \bm{k} \bm{k} \hspace{0.1ex}} \tikzmark{EndBraceE3E3} \hspace{0.1ex} =\\[0.32em]
%
= \hspace{0.1ex} \bm{E} \operatorname{cos} \vartheta - \hspace{-0.1ex} \tikzmark{StartBraceKk} {\hspace{0.1ex}\bm{e}_3 \bm{e}_3\hspace{0.1ex}} \tikzmark{EndBraceKk} \hspace{-0.25ex} \operatorname{cos} \vartheta \hspace{0.1ex} + \tikzmark{StartBraceLeviCivita} {\left( \bm{e}_2 \bm{e}_1 - \bm{e}_1 \bm{e}_2 \right)} \tikzmark{EndBraceLeviCivita} \operatorname{sin} \vartheta + \bm{k} \bm{k} \hspace{0.1ex} =\\[1.5em]
%
\shoveright{ \hfill = \bm{E} \operatorname{cos} \vartheta + \bm{k} \times\hspace{-0.2ex} \bm{E} \operatorname{sin} \vartheta + \bm{k} \bm{k} \left({1 - \operatorname{cos} \vartheta}\right) }
\end{multline*}

\AddOverBrace[line width=0.75pt]{StartBraceE1E1}{EndBraceE1E1}{${\scriptstyle \bm{e}_1 \mathcircabove{\bm{e}}_1}$}
\AddOverBrace[line width=0.75pt]{StartBraceE2E2}{EndBraceE2E2}{${\scriptstyle \bm{e}_2 \mathcircabove{\bm{e}}_2}$}
\AddOverBrace[line width=0.75pt]{StartBraceE3E3}{EndBraceE3E3}{${\scriptstyle \bm{e}_3 \mathcircabove{\bm{e}}_3}$}
\AddUnderBrace[line width=0.75pt][-0.1ex,-0.2ex]{StartBraceKk}{EndBraceKk}{${\scriptstyle \bm{k}\bm{k}}$}
\AddUnderBrace[line width=0.75pt][-0.1ex,-0.2ex][xshift=0.4ex]{StartBraceLeviCivita}{EndBraceLeviCivita}{${\scriptstyle \bm{e}_3 \times \bm{e}_i \bm{e}_i \:=\: \levicivita_{3ij} \bm{e}_j \bm{e}_i}$}

\vspace{-0.5em}
\caption{\inquotes{\en{Finite rotation}\ru{Конечный поворот}}}\label{fig:eulerfiniterotation}
\end{figure}

% ~ ~ ~ ~ ~

Из~\eqref{eulerfiniterotation} и~\eqref{rodriguesrotationformula} получаем формулу поворота Родрига в~параметрах~$\bm{k}$ и~$\vartheta$:

\nopagebreak\vspace{-0.3em}\begin{equation*}
\bm{r} \hspace{.3ex}=\hspace{.4ex} \mathcircabove{\bm{r}} \operatorname{cos} \vartheta \hspace{.3ex}+\hspace{.3ex} \bm{k} \times \mathcircabove{\bm{r}} \hspace{.2ex} \operatorname{sin} \vartheta \hspace{.3ex}+\hspace{.4ex} \bm{k} \bm{k} \dotp \hspace{.1ex} \mathcircabove{\bm{r}} \left({1 - \operatorname{cos} \vartheta}\right) \hspace{-0.25ex}.
\end{equation*}

\vspace{-0.16em} В~параметрах конечного поворота транспонирование, оно~же обращение, тензора~$\bm{P}$ эквивалентно перемене направления поворота\:--- знака угла~$\vartheta$
\[
\bm{P}^{\T} \hspace{-0.1ex}=\hspace{.1ex} \bm{P} \hspace{.1ex} \bigr|_{\vartheta \,=\hspace{.1ex} -\vartheta} \hspace{-0.1ex} = \bm{E} \operatorname{cos} \vartheta - \bm{k} \times\hspace{-0.2ex} \bm{E} \operatorname{sin} \vartheta + \bm{k} \bm{k} \left({1 - \operatorname{cos} \vartheta}\right) \hspace{-0.32ex}.
\]

Пусть теперь тензор поворота меняется со~временем: ${\bm{P} \!=\! \bm{P}(t)}$.
Псевдовектор угловой скорости~${\bm{\omega}}$ вводится через~$\bm{P}$ таким путём.
Дифференцируем тождество ортогональности~\eqref{orthogonalityofrotationtensor} по~времени\footnote{Various notations are used to denote the~full time derivative. In~addition to the Leibniz’s notation ${\frac{dx}{dt}}$, the~very popular short\hbox{-}hand one is the~\inquotes{over-dot} Newton’s notation ${\mathdotabove{x}}$.}

\nopagebreak\vspace{-0.1em}\begin{equation*}
\mathdotabove{\bm{P}} \dotp \bm{P}^{\T} \hspace{-0.1ex} + \hspace{.25ex} \bm{P} \dotp \mathdotabove{\bm{P}}^{\T} \hspace{-0.1ex} = \hspace{.1ex} {^2\bm{0}}
\hspace{.1ex} .
\end{equation*}

Тензор ${\mathdotabove{\bm{P}} \dotp \bm{P}^{\T\!}}$ (по~\eqref{transposeofdotproduct} ${\left({ \mathdotabove{\bm{P}} \dotp \bm{P}^{\T} }\right)^{\raisemath{-0.25em}{\!\T}} \hspace{-0.4ex} = \bm{P} \dotp \mathdotabove{\bm{P}}^{\T}}$) оказался анти\-сим\-метрич\-ным.
Поэтому согласно~\eqref{companionvector} он представ\'{и}м сопутствующим вектором как ${\mathdotabove{\bm{P}} \dotp \bm{P}^{\T\!} = \bm{\omega} \times \bm{E} = \bm{\omega} \times \bm{P} \dotp \bm{P}^{\T}}$\!.
То~есть

\nopagebreak\vspace{-0.1em}\begin{equation}\label{angularvelocityvector}
\mathdotabove{\bm{P}} = \bm{\omega} \times \bm{P}, \;\:\:
\bm{\omega} \equiv -\, \displaystyle \onehalf \left( \mathdotabove{\bm{P}} \dotp \bm{P}^{\T} \right)_{\hspace{-0.2em}\Xcompanion}
\vspace{-0.25em}\end{equation}

Помимо этого общего представления~вектора~${\bm{\omega}}$, для~него есть и~другие. Например, через параметры конечного поворота.

Производная~${\mathdotabove{\bm{P}}}$ в~параметрах конечного поворота в~общем случае (оба параметра\:--- и~единичный вектор~$\bm{k}$, и~угол~$\vartheta$\:--- переменны во~времени):
\vspace{.32em}%
\[\begin{array}{r@{\hspace{.33em}}c@{\hspace{.25em}}l}
\mathdotabove{\bm{P}} \hspace{.2em} & = & \hspace{.1em} \left(\bm{P}^{\mathsf{\hspace{.12ex}S}} \hspace{-0.2ex} +^{\mathstrut} \bm{P}^{\mathsf{\hspace{.12ex}A}}\right)^{\hspace{-0.2ex}\tikz[baseline=-0.5ex]\draw[black, fill=black] (0,0) circle (.266ex);} =
\hspace{.1em} \left(\hspace{.2ex} \tikzmark{StartBracePs} {\bm{E} \operatorname{cos} \vartheta + \bm{k} \bm{k} \left({1 \!-\! \operatorname{cos} \vartheta}\right)} \hspace{-0.2ex} \tikzmark{EndBracePs} \hspace{.32ex} +^{\mathstrut} \hspace{.2ex}
\tikzmark{StartBracePa} {\bm{k} \hspace{-0.24ex}\times\hspace{-0.4ex} \bm{E} \operatorname{sin} \vartheta \hspace{.2ex}} \tikzmark{EndBracePa} {} \hspace{.16ex}\right)^{\hspace{-0.2ex}\tikz[baseline=-0.5ex]\draw[black, fill=black] (0,0) circle (.266ex);} \hspace{-0.05em} = \\[0.4em]
%
& = & \hspace{.2em} \tikzmark{StartBraceDotPs} {\left( \hspace{.1ex} \bm{k} \bm{k} \hspace{-0.1ex} - \hspace{-0.2ex} \bm{E} \hspace{.1ex} \right) \hspace{-0.1ex} \mathdotabove{\vartheta} \operatorname{sin} \vartheta + \hspace{-0.2ex} ( \bm{k} \mathdotabove{\bm{k}} + \mathdotabove{\bm{k}} \bm{k} ) \hspace{-0.2ex} \left({1 \!-\! \operatorname{cos} \vartheta}\right)} \tikzmark{EndBraceDotPs} \hspace{.64ex} + \\[0.64em]
& & \hspace{13.2em} + \hspace{.72ex} \tikzmark{StartBraceDotPa} {\hspace{.12ex} \bm{k} \hspace{-0.24ex}\times\hspace{-0.4ex} \bm{E} \hspace{.4ex} \mathdotabove{\vartheta} \operatorname{cos} \vartheta + \mathdotabove{\bm{k}} \hspace{-0.24ex}\times\hspace{-0.4ex} \bm{E} \operatorname{sin} \vartheta} \tikzmark{EndBraceDotPa}
\hspace{.1em} .
\end{array}\]%
\vspace{-1.2em}

\AddOverBrace[line width=0.75pt][0.12ex,0]{StartBracePs}{EndBracePs}{${\scriptstyle \bm{P}^{\mathsf{\hspace{.12ex}S}}}$}
\AddOverBrace[line width=0.75pt][-0.12ex,0]{StartBracePa}{EndBracePa}{${\scriptstyle \bm{P}^{\mathsf{\hspace{.12ex}A}}}$}
\AddUnderBrace[line width=0.75pt]{StartBraceDotPs}{EndBraceDotPs}{${\scriptstyle \mathdotabove{\bm{P}}^{\mathsf{\hspace{.12ex}S}}}$}
\AddUnderBrace[line width=0.75pt][-0.1ex,0.1ex]{StartBraceDotPa}{EndBraceDotPa}{${\scriptstyle \mathdotabove{\bm{P}}^{\mathsf{\hspace{.12ex}A}}}$}

\vspace{-1.32em} \noindent Находим
\vspace{.2em}\[\begin{array}{r@{\hspace{.25em}}c@{\hspace{.4em}}l}
\mathdotabove{\bm{P}} \dotp \bm{P}^{\T} & = & (\hspace{.1em} \mathdotabove{\bm{P}}^{\mathsf{\hspace{.12ex}S}} \hspace{-0.16ex} + \mathdotabove{\bm{P}}^{\mathsf{\hspace{.12ex}A}} \hspace{.05em}) \hspace{-0.1ex} \dotp \hspace{-0.1ex} (\hspace{.1em} \bm{P}^{\mathsf{\hspace{.12ex}S}} \hspace{-0.16ex} - \bm{P}^{\mathsf{\hspace{.12ex}A}} \hspace{.1ex} \hspace{.05em}) =
\\[0.25em]
& = & \mathdotabove{\bm{P}}^{\mathsf{\hspace{.12ex}S}} \hspace{-0.2ex}\dotp \bm{P}^{\mathsf{\hspace{.12ex}S}}
+ \hspace{.2ex} \mathdotabove{\bm{P}}^{\mathsf{\hspace{.12ex}A}} \hspace{-0.2ex}\dotp \bm{P}^{\mathsf{\hspace{.12ex}S}}
- \hspace{.2ex} \mathdotabove{\bm{P}}^{\mathsf{\hspace{.12ex}S}} \hspace{-0.2ex}\dotp \bm{P}^{\mathsf{\hspace{.12ex}A}}
- \hspace{.2ex} \mathdotabove{\bm{P}}^{\mathsf{\hspace{.12ex}A}} \hspace{-0.2ex}\dotp \bm{P}^{\mathsf{\hspace{.12ex}A}} ,
\end{array}\]

\vspace{-0.5em} \noindent используя
\[\scalebox{0.95}[0.96]{$\begin{array}{c}
\bm{k} \dotp \bm{k} = 1 = \constant \,\Rightarrow\:
\bm{k} \dotp \mathdotabove{\bm{k}} + \mathdotabove{\bm{k}} \dotp \bm{k} = 0 \;\Leftrightarrow\; \mathdotabove{\bm{k}} \dotp \bm{k} = \bm{k} \dotp \mathdotabove{\bm{k}} = 0 \hspace{.1ex} ,
\\[.08em]
%
\bm{k} \bm{k} \hspace{-0.2ex}\dotp\hspace{-0.2ex} \bm{k} \bm{k} = \bm{k} \bm{k}, \:\:
\mathdotabove{\bm{k}} \bm{k} \hspace{-0.2ex}\dotp\hspace{-0.2ex} \bm{k} \bm{k} = \mathdotabove{\bm{k}} \bm{k} , \:\:
\bm{k} \mathdotabove{\bm{k}} \hspace{-0.2ex}\dotp\hspace{-0.2ex} \bm{k} \bm{k} = {\hspace{-0.2ex}^2\bm{0}} \hspace{.1ex},
\\[.16em]
%
\left( \hspace{.1ex} \bm{k} \bm{k} \hspace{-0.1ex} - \hspace{-0.2ex} \bm{E} \hspace{.1ex} \right) \hspace{-0.2ex} \dotp \bm{k} = \bm{k} - \bm{k} = {\bm{0}} \hspace{.1ex}, \,\,
\left( \hspace{.1ex} \bm{k} \bm{k} \hspace{-0.1ex} - \hspace{-0.2ex} \bm{E} \hspace{.1ex} \right) \hspace{-0.2ex} \dotp \bm{k} \bm{k} = \bm{k} \bm{k} - \bm{k} \bm{k} = {\hspace{-0.2ex}^2\bm{0}} \hspace{.1ex} ,
\\[.08em]
%
\bm{k} \dotp \hspace{-0.1ex} ( \bm{k} \hspace{-0.24ex}\times\hspace{-0.4ex} \bm{E} ) \hspace{-0.1ex}
= \hspace{-0.1ex} ( \bm{k} \hspace{-0.24ex}\times\hspace{-0.4ex} \bm{E} ) \hspace{-0.2ex} \dotp \bm{k}
= \bm{k} \hspace{-0.24ex}\times\hspace{-0.2ex} \bm{k} = \bm{0} \hspace{.1ex} , \,\,
\bm{k} \bm{k} \dotp \hspace{-0.1ex} ( \bm{k} \hspace{-0.24ex}\times\hspace{-0.4ex} \bm{E} ) \hspace{-0.1ex}
= \hspace{-0.1ex} ( \bm{k} \hspace{-0.24ex}\times\hspace{-0.4ex} \bm{E} ) \hspace{-0.2ex} \dotp \bm{k} \bm{k}
= {\hspace{-0.2ex}^2\bm{0}} \hspace{0.1ex},
\\[.08em]
%
\left( \hspace{0.1ex} \bm{k} \bm{k} \hspace{-0.1ex} - \hspace{-0.2ex} \bm{E} \hspace{0.1ex} \right) \hspace{-0.2ex} \dotp \hspace{-0.1ex} ( \bm{k} \hspace{-0.24ex}\times\hspace{-0.4ex} \bm{E} ) \hspace{-0.1ex} = \hspace{-0.1ex}
- \hspace{0.2ex} \bm{k} \hspace{-0.24ex}\times\hspace{-0.4ex} \bm{E} \hspace{0.1ex},
\\
%
( \bm{a} \hspace{-0.24ex}\times\hspace{-0.4ex} \bm{E} ) \hspace{-0.2ex} \dotp \bm{b} =
\bm{a} \hspace{-0.2ex}\times\hspace{-0.36ex} ( \bm{E} \hspace{-0.1ex} \dotp \bm{b} ) \hspace{-0.16ex} =
\bm{a} \hspace{-0.16ex}\times\hspace{-0.12ex} \bm{b} \:\,\Rightarrow\,
( \mathdotabove{\bm{k}} \hspace{-0.24ex}\times\hspace{-0.4ex} \bm{E} ) \hspace{-0.2ex} \dotp \bm{k} \bm{k} =
\mathdotabove{\bm{k}} \hspace{-0.16ex}\times\hspace{-0.2ex} \bm{k} \bm{k} \hspace{0.1ex},
\end{array}$}\]

\vspace{-0.4em} \noindent \eqref{vectorcrossidentitydotvectorcrossidentity} $\,\Rightarrow\,$
\scalebox{0.95}[0.96]{${( \bm{k} \hspace{-0.24ex}\times\hspace{-0.4ex} \bm{E} ) \hspace{-0.2ex} \dotp \hspace{-0.2ex} ( \bm{k} \hspace{-0.24ex}\times\hspace{-0.4ex} \bm{E} ) \hspace{-0.16ex} = \bm{k} \bm{k} \hspace{-0.1ex} - \hspace{-0.2ex} \bm{E}}$},\hspace{0.4ex}
%
\scalebox{0.95}[0.96]{${\displaystyle ( \mathdotabove{\bm{k}} \hspace{-0.24ex}\times\hspace{-0.4ex} \bm{E} ) \hspace{-0.2ex} \dotp \hspace{-0.2ex} ( \bm{k} \hspace{-0.24ex}\times\hspace{-0.4ex} \bm{E} ) \hspace{-0.16ex} = \bm{k} \mathdotabove{\bm{k}} \hspace{-0.1ex} - \tikzbackcancel[black!25]{$\mathdotabove{\bm{k}} \hspace{-0.2ex}\dotp\hspace{-0.2ex} \bm{k} \hspace{0.16ex} \bm{E}$}}$\hspace{0.16ex}},

\noindent \eqref{vectorcrossvectorcrossidentity} $\,\Rightarrow\,$
\scalebox{0.95}[0.96]{$\mathdotabove{\bm{k}} \bm{k} \hspace{-0.1ex} - \hspace{-0.1ex} \bm{k} \mathdotabove{\bm{k}} = \hspace{-0.16ex} ( \bm{k} \hspace{-0.2ex} \times \hspace{-0.24ex} \mathdotabove{\bm{k}} ) \hspace{-0.32ex} \times \hspace{-0.32ex} \bm{E}$},\hspace{.4ex}
%
\scalebox{0.95}[0.96]{${\displaystyle ( \mathdotabove{\bm{k}} \hspace{-0.2ex}\times\hspace{-0.24ex} \bm{k} ) \hspace{.2ex} \bm{k} \hspace{-0.1ex} - \hspace{-0.1ex} \bm{k} \hspace{.16ex} ( \mathdotabove{\bm{k}} \hspace{-0.2ex}\times\hspace{-0.24ex} \bm{k} ) \hspace{-0.16ex} = \bm{k} \hspace{-0.2ex} \times \hspace{-0.32ex} ( \mathdotabove{\bm{k}} \hspace{-0.2ex}\times\hspace{-0.24ex} \bm{k} ) \hspace{-0.32ex} \times \hspace{-0.32ex} \bm{E}}$\hspace{.16ex}}

\begin{fleqn}[0pt]
\begin{multline*}
\shoveleft{\scalebox{0.94}[0.96]{$\mathdotabove{\bm{P}}^{\mathsf{\hspace{0.12ex}S}} \hspace{-0.2ex}\dotp \bm{P}^{\mathsf{\hspace{.12ex}S}} \hspace{-0.25ex} = $} \hspace{2em} \hfill}
\\[-0.25em]
%
\shoveleft{\scalebox{0.8}[0.82]{$= \hspace{.2ex} \left( \hspace{.1ex} \bm{k} \bm{k} \hspace{-0.1ex} - \hspace{-0.2ex} \bm{E} \hspace{.1ex} \right) \hspace{-0.1ex} \mathdotabove{\vartheta} \operatorname{sin} \vartheta \dotp \bm{E} \operatorname{cos} \vartheta +
( \bm{k} \mathdotabove{\bm{k}} + \mathdotabove{\bm{k}} \bm{k} ) \hspace{-0.2ex} \left({1 \!-\! \operatorname{cos} \vartheta}\right) \dotp \bm{E} \operatorname{cos} \vartheta \hspace{.32em} +$} \hfill}
\\[-0.2em]
\shoveright{\hfill \scalebox{0.8}[0.82]{$+\; \tikzbackcancel[black!25]{$\left( \hspace{.1ex} \bm{k} \bm{k} \hspace{-0.1ex} - \hspace{-0.2ex} \bm{E} \hspace{.1ex} \right) \hspace{-0.1ex} \mathdotabove{\vartheta} \operatorname{sin} \vartheta \dotp \bm{k} \bm{k} \left({1 \!-\! \operatorname{cos} \vartheta}\right)$} \hspace{.2ex} +
( \bm{k} \mathdotabove{\bm{k}} + \mathdotabove{\bm{k}} \bm{k} ) \hspace{-0.2ex} \left({1 \!-\! \operatorname{cos} \vartheta}\right) \hspace{-0.2ex} \dotp \bm{k} \bm{k} \left({1 \!-\! \operatorname{cos} \vartheta}\right) =$}}
\\
%
\scalebox{0.8}[0.82]{$= \left( \hspace{.1ex} \bm{k} \bm{k} \hspace{-0.1ex} - \hspace{-0.2ex} \bm{E} \hspace{.1ex} \right) \hspace{-0.1ex} \mathdotabove{\vartheta} \operatorname{sin} \vartheta \operatorname{cos} \vartheta
+ ( \bm{k} \mathdotabove{\bm{k}} + \mathdotabove{\bm{k}} \bm{k} ) \hspace{-0.1ex} \operatorname{cos} \vartheta \left({1 \!-\! \operatorname{cos} \vartheta}\right) + ( \tikzbackcancel[black!25]{$\bm{k} \mathdotabove{\bm{k}} \hspace{-0.1ex}\dotp\hspace{-0.1ex} \bm{k} \bm{k}$} + \mathdotabove{\bm{k}} \bm{k} \hspace{-0.1ex}\dotp\hspace{-0.1ex} \bm{k} \bm{k} ) \left({1 \!-\! \operatorname{cos} \vartheta}\right)^{\hspace{-0.12ex}2} \hspace{-0.25ex} =$}
\\
%
\shoveleft{\scalebox{0.8}[0.82]{$= \left( \hspace{.1ex} \bm{k} \bm{k} \hspace{-0.1ex} - \hspace{-0.2ex} \bm{E} \hspace{.1ex} \right) \hspace{-0.1ex} \mathdotabove{\vartheta} \operatorname{sin} \vartheta \operatorname{cos} \vartheta + \bm{k} \mathdotabove{\bm{k}} \operatorname{cos} \vartheta \left({1 \!-\! \operatorname{cos} \vartheta}\right) +$} \hfill}
\\[-0.2em]
\shoveright{\hfill \scalebox{0.8}[0.82]{$+ \hspace{.24em} \mathdotabove{\bm{k}} \bm{k} \operatorname{cos} \vartheta - \mathdotabove{\bm{k}} \bm{k} \operatorname{cos}^{2\hspace{-0.4ex}} \vartheta + \mathdotabove{\bm{k}} \bm{k} - 2 \, \mathdotabove{\bm{k}} \bm{k} \operatorname{cos} \vartheta + \mathdotabove{\bm{k}} \bm{k} \operatorname{cos}^{2\hspace{-0.4ex}} \vartheta =$}}\\
%
%% \shoveright{\hfill \scalebox{0.8}[0.82]{$= \left( \hspace{.1ex} \bm{k} \bm{k} \hspace{-0.1ex} - \hspace{-0.2ex} \bm{E} \hspace{.1ex} \right) \hspace{-0.1ex} \mathdotabove{\vartheta} \operatorname{sin} \vartheta \operatorname{cos} \vartheta \hspace{.1ex}
%% + \hspace{.1ex} \bm{k} \mathdotabove{\bm{k}} \operatorname{cos} \vartheta \left({1 \!-\! \operatorname{cos} \vartheta}\right)
%% + \mathdotabove{\bm{k}} \bm{k}
%% - \mathdotabove{\bm{k}} \bm{k} \operatorname{cos} \vartheta =$}}\\
%
\shoveright{\hfill \hspace{4.8em} \scalebox{0.94}[0.96]{$= \left( \hspace{.1ex} \bm{k} \bm{k} \hspace{-0.1ex} - \hspace{-0.2ex} \bm{E} \hspace{.1ex} \right) \hspace{-0.1ex} \mathdotabove{\vartheta} \operatorname{sin} \vartheta \operatorname{cos} \vartheta \hspace{.1ex}
%% + \hspace{-0.1ex} ( \bm{k} \mathdotabove{\bm{k}} \operatorname{cos} \vartheta + \mathdotabove{\bm{k}} \bm{k} ) \hspace{-0.2ex} \left({1 \!-\! \operatorname{cos} \vartheta}\right)
+ \bm{k} \mathdotabove{\bm{k}} \operatorname{cos} \vartheta
- \bm{k} \mathdotabove{\bm{k}} \operatorname{cos}^{2\hspace{-0.4ex}} \vartheta
+ \mathdotabove{\bm{k}} \bm{k} \left({1 \!-\! \operatorname{cos} \vartheta}\right) \hspace{-0.16ex},$}}
\end{multline*}
\begin{multline*}
\shoveleft{\scalebox{0.94}[0.96]{$\mathdotabove{\bm{P}}^{\mathsf{\hspace{.12ex}A}} \hspace{-0.2ex}\dotp \bm{P}^{\mathsf{\hspace{.12ex}S}} \hspace{-0.25ex} = $} \hfill} \\[-0.25em]
%
\shoveleft{\scalebox{0.8}[0.82]{$= ( \hspace{.12ex} \bm{k} \hspace{-0.24ex}\times\hspace{-0.4ex} \bm{E} \hspace{.1ex} ) \hspace{-0.2ex} \dotp \hspace{-0.2ex} \bm{E} \hspace{.4ex} \mathdotabove{\vartheta} \operatorname{cos}^{2\hspace{-0.4ex}} \vartheta +
( \hspace{.12ex} \mathdotabove{\bm{k}} \hspace{-0.24ex}\times\hspace{-0.4ex} \bm{E} \hspace{.1ex} ) \hspace{-0.2ex} \dotp \hspace{-0.2ex} \bm{E} \hspace{.1ex} \operatorname{sin} \vartheta \operatorname{cos} \vartheta \:+$} \hfill} \\[-0.2em]
\shoveright{\hfill \scalebox{0.8}[0.82]{$+\; \tikzbackcancel[black!25]{$ ( \hspace{.12ex} \bm{k} \hspace{-0.24ex}\times\hspace{-0.4ex} \bm{E} \hspace{.1ex} ) \hspace{-0.2ex} \dotp \hspace{-0.1ex} \bm{k} \bm{k} \hspace{.5ex} \mathdotabove{\vartheta} \operatorname{cos} \vartheta \left({1 \!-\! \operatorname{cos} \vartheta}\right) $} \hspace{.2ex} +
( \hspace{.12ex} \mathdotabove{\bm{k}} \hspace{-0.24ex}\times\hspace{-0.4ex} \bm{E} \hspace{.1ex} ) \hspace{-0.25ex} \dotp \hspace{-0.1ex} \bm{k} \bm{k} \hspace{.2ex} \operatorname{sin} \vartheta \left({1 \!-\! \operatorname{cos} \vartheta}\right) =$}} \\
%
\hspace{3.85em} \scalebox{0.94}[0.96]{$= \hspace{.2ex} \bm{k} \hspace{-0.24ex}\times\hspace{-0.4ex} \bm{E} \hspace{.4ex} \mathdotabove{\vartheta} \operatorname{cos}^{2\hspace{-0.4ex}} \vartheta +
\mathdotabove{\bm{k}} \hspace{-0.24ex}\times\hspace{-0.4ex} \bm{E} \hspace{.1ex} \operatorname{sin} \vartheta \operatorname{cos} \vartheta +
%%\hspace{-0.12ex} ( \hspace{.12ex} \mathdotabove{\bm{k}} \hspace{-0.24ex}\times\hspace{-0.4ex} \bm{E} \hspace{.1ex} ) \hspace{-0.25ex} \dotp \hspace{-0.1ex} \bm{k} \bm{k} \hspace{.2ex}
\mathdotabove{\bm{k}} \hspace{-0.24ex}\times\hspace{-0.32ex} \bm{k} \bm{k} \hspace{.2ex}
\operatorname{sin} \vartheta \left({1 \!-\! \operatorname{cos} \vartheta}\right) \hspace{-0.16ex},$}
\end{multline*}
\begin{multline*}
\shoveleft{\scalebox{0.94}[0.96]{$\mathdotabove{\bm{P}}^{\mathsf{\hspace{.12ex}S}} \hspace{-0.2ex}\dotp \bm{P}^{\mathsf{\hspace{.12ex}A}} \hspace{-0.25ex} = $} \hfill} \\[-0.25em]
%
\scalebox{0.8}[0.82]{$= ( \hspace{.1ex} \bm{k} \bm{k} \hspace{-0.1ex} - \hspace{-0.2ex} \bm{E} \hspace{.1ex} ) \hspace{.25ex} \mathdotabove{\vartheta} \operatorname{sin} \vartheta \dotp ( \bm{k} \hspace{-0.24ex}\times\hspace{-0.4ex} \bm{E} ) \operatorname{sin} \vartheta +
( \bm{k} \mathdotabove{\bm{k}} + \mathdotabove{\bm{k}} \bm{k} ) \hspace{-0.2ex} \left({1 \!-\! \operatorname{cos} \vartheta}\right) \hspace{-0.1ex} \dotp ( \bm{k} \hspace{-0.24ex}\times\hspace{-0.4ex} \bm{E} ) \operatorname{sin} \vartheta =$} \\
%
\scalebox{0.78}[0.82]{$= \hspace{.2ex} \tikzbackcancel[black!25]{$\bm{k} \bm{k} \hspace{-0.12ex} \dotp \hspace{-0.12ex} ( \bm{k} \hspace{-0.24ex}\times\hspace{-0.4ex} \bm{E} \hspace{.1ex} ) \hspace{.32ex} \mathdotabove{\vartheta} \operatorname{sin}^{\hspace{-0.1ex}2\hspace{-0.4ex}} \vartheta$} \hspace{.12ex}
- \hspace{-0.1ex} \bm{E} \hspace{-0.16ex} \dotp \hspace{-0.12ex} ( \bm{k} \hspace{-0.24ex}\times\hspace{-0.4ex} \bm{E} \hspace{.1ex} ) \hspace{.25ex} \mathdotabove{\vartheta} \operatorname{sin}^{\hspace{-0.1ex}2\hspace{-0.4ex}} \vartheta
+ \hspace{-0.2ex} \left( \hspace{-0.1ex} \bm{k} \mathdotabove{\bm{k}} \dotp \hspace{-0.1ex} ( \bm{k} \hspace{-0.32ex}\times\hspace{-0.4ex} \bm{E} ) \hspace{-0.16ex} + \hspace{.1ex} \tikzbackcancel[black!25]{$\mathdotabove{\bm{k}} \bm{k} \dotp \hspace{-0.1ex} ( \bm{k} \hspace{-0.32ex}\times\hspace{-0.4ex} \bm{E}$} ) \hspace{-0.12ex} \right) \hspace{-0.2ex} \operatorname{sin} \vartheta \left({1 \!-\! \operatorname{cos} \vartheta} \right) =$} \\[-0.25em]
%
\hspace{13em} \scalebox{0.94}[0.96]{$= \hspace{-0.16ex} - \hspace{.2ex} \bm{k} \hspace{-0.24ex}\times\hspace{-0.4ex} \bm{E} \hspace{.4ex} \mathdotabove{\vartheta} \operatorname{sin}^{\hspace{-0.1ex}2\hspace{-0.4ex}} \vartheta
+ \bm{k} \mathdotabove{\bm{k}} \hspace{-0.24ex}\times\hspace{-0.32ex} \bm{k} \hspace{.2ex} \operatorname{sin} \vartheta \left({1 \!-\! \operatorname{cos} \vartheta}\right) \hspace{-0.16ex},$}
\end{multline*}
\begin{multline*}
\shoveleft{\scalebox{0.94}[0.96]{$\mathdotabove{\bm{P}}^{\mathsf{\hspace{.12ex}A}} \hspace{-0.2ex}\dotp \bm{P}^{\mathsf{\hspace{.12ex}A}} \hspace{-0.25ex} =
%
( \bm{k} \hspace{-0.24ex}\times\hspace{-0.4ex} \bm{E} ) \hspace{.32ex} \mathdotabove{\vartheta} \operatorname{cos} \vartheta \dotp ( \bm{k} \hspace{-0.24ex}\times\hspace{-0.4ex} \bm{E} ) \operatorname{sin} \vartheta + \hspace{-0.1ex}
( \mathdotabove{\bm{k}} \hspace{-0.24ex}\times\hspace{-0.4ex} \bm{E} ) \hspace{-0.16ex} \dotp \hspace{-0.16ex} ( \bm{k} \hspace{-0.24ex}\times\hspace{-0.4ex} \bm{E} ) \hspace{.1ex} \operatorname{sin}^{\hspace{-0.1ex}2\hspace{-0.4ex}} \vartheta = $} \hfill} \\
%
\shoveright{\hfill \hspace{16em} \scalebox{0.94}[0.96]{$= ( \bm{k} \bm{k} \hspace{-0.1ex} - \hspace{-0.2ex} \bm{E} ) \hspace{.32ex} \mathdotabove{\vartheta} \operatorname{sin} \vartheta \operatorname{cos} \vartheta
+ \bm{k} \mathdotabove{\bm{k}} \operatorname{sin}^{\hspace{-0.1ex}2\hspace{-0.4ex}} \vartheta \hspace{.2ex};$}}
\end{multline*}
\end{fleqn}

\begin{multline*}
\scalebox{0.94}[0.96]{$\mathdotabove{\bm{P}} \dotp \bm{P}^{\T}$} \hspace{.25ex}
\scalebox{0.92}[0.96]{$= \hspace{.1ex}
\mathdotabove{\bm{P}}^{\mathsf{\hspace{.12ex}S}} \hspace{-0.2ex}\dotp \bm{P}^{\mathsf{\hspace{.12ex}S}}
+ \hspace{.2ex} \mathdotabove{\bm{P}}^{\mathsf{\hspace{.12ex}A}} \hspace{-0.25ex}\dotp \bm{P}^{\mathsf{\hspace{.12ex}S}}
- \hspace{.2ex} \mathdotabove{\bm{P}}^{\mathsf{\hspace{.12ex}S}} \hspace{-0.25ex}\dotp \bm{P}^{\mathsf{\hspace{.12ex}A}}
- \hspace{.2ex} \mathdotabove{\bm{P}}^{\mathsf{\hspace{.12ex}A}} \hspace{-0.25ex}\dotp \bm{P}^{\mathsf{\hspace{.12ex}A}} \hspace{-0.25ex} =$} \\[-0.25em]
%
\scalebox{0.8}[0.82]{$= {\color{black!50}{\left( \hspace{.1ex} \bm{k} \bm{k} \hspace{-0.1ex} - \hspace{-0.2ex} \bm{E} \hspace{.1ex} \right) \hspace{-0.1ex} \mathdotabove{\vartheta} \operatorname{sin} \vartheta \operatorname{cos} \vartheta}}
+ \bm{k} \mathdotabove{\bm{k}} \operatorname{cos} \vartheta
- {\color{magenta!80!black}{\bm{k} \mathdotabove{\bm{k}}}} {\color{black!50}{\hspace{.4ex}\operatorname{cos}^{2\hspace{-0.4ex}} \vartheta}}
+ \mathdotabove{\bm{k}} \bm{k} \left({1 \!-\! \operatorname{cos} \vartheta}\right) +$} \\[-0.2em]
%
\scalebox{0.8}[0.82]{$+\; {\color{blue!80!black}{\bm{k} \hspace{-0.24ex}\times\hspace{-0.4ex} \bm{E} \hspace{.4ex} \mathdotabove{\vartheta}}} {\color{black!50}{\hspace{.4ex}\operatorname{cos}^{2\hspace{-0.4ex}} \vartheta}}
+ \mathdotabove{\bm{k}} \hspace{-0.24ex}\times\hspace{-0.4ex} \bm{E} \hspace{.1ex} \operatorname{sin} \vartheta \operatorname{cos} \vartheta
+ \mathdotabove{\bm{k}} \hspace{-0.24ex}\times\hspace{-0.32ex} \bm{k} \bm{k} \hspace{.2ex} {\color{green!50!black}{\hspace{.5ex}\operatorname{sin} \vartheta \left({1 \!-\! \operatorname{cos} \vartheta}\right)\hspace{.4ex}}} +$} \\[-0.25em]
%
\scalebox{0.8}[0.82]{$+\; {\color{blue!80!black}{\bm{k} \hspace{-0.24ex}\times\hspace{-0.4ex} \bm{E} \hspace{.4ex} \mathdotabove{\vartheta}}} {\color{black!50}{\hspace{.4ex}\operatorname{sin}^{\hspace{-0.1ex}2\hspace{-0.4ex}} \vartheta}}
- \bm{k} \mathdotabove{\bm{k}} \hspace{-0.24ex}\times\hspace{-0.32ex} \bm{k} {\color{green!50!black}{\hspace{.5ex}\operatorname{sin} \vartheta \left({1 \!-\! \operatorname{cos} \vartheta}\right)\hspace{.4ex}}}
{\color{black!50}{-\hspace{.5ex} ( \bm{k} \bm{k} \hspace{-0.1ex} - \hspace{-0.2ex} \bm{E} ) \hspace{.32ex} \mathdotabove{\vartheta} \operatorname{sin} \vartheta \operatorname{cos} \vartheta}}
- {\color{magenta!80!black}{\bm{k} \mathdotabove{\bm{k}}}} {\color{black!50}{\hspace{.4ex}\operatorname{sin}^{\hspace{-0.1ex}2\hspace{-0.4ex}} \vartheta}} =$} \\[-0.08em]
%
\scalebox{0.79}[0.82]{$= \bm{k} \hspace{-0.24ex}\times\hspace{-0.4ex} \bm{E} \hspace{.4ex} \mathdotabove{\vartheta} \hspace{-0.1ex}
+ \hspace{-0.1ex} \hspace{-0.16ex} ( \hspace{.2ex} \mathdotabove{\bm{k}} \bm{k} \hspace{-0.2ex} - \hspace{-0.2ex} \bm{k} \mathdotabove{\bm{k}} \hspace{.2ex} ) \hspace{.1ex} ({1 \!-\! \operatorname{cos} \vartheta}) \hspace{-0.2ex}
+ \mathdotabove{\bm{k}} \hspace{-0.24ex}\times\hspace{-0.4ex} \bm{E} \hspace{.1ex} \operatorname{sin} \vartheta \operatorname{cos} \vartheta \hspace{-0.1ex}
+ \hspace{-0.1ex} ( \mathdotabove{\bm{k}} \hspace{-0.28ex}\times\hspace{-0.32ex} \bm{k} \bm{k} \hspace{-0.12ex} - \hspace{-0.12ex} \bm{k} \mathdotabove{\bm{k}} \hspace{-0.28ex}\times\hspace{-0.32ex} \bm{k} ) \operatorname{sin} \vartheta \left({1 \!-\! \operatorname{cos} \vartheta}\right) = $} \\[-0.08em]
%
\scalebox{0.8}[0.82]{$= \bm{k} \hspace{-0.24ex}\times\hspace{-0.4ex} \bm{E} \hspace{.4ex} \mathdotabove{\vartheta} \hspace{-0.1ex}
+ \hspace{-0.1ex} \bm{k} \hspace{-0.2ex}\times\hspace{-0.2ex}  \mathdotabove{\bm{k}} \hspace{-0.2ex}\times\hspace{-0.4ex} \bm{E} \hspace{.25ex} ({1 \!-\! \operatorname{cos} \vartheta}) \hspace{-0.2ex}
+ \mathdotabove{\bm{k}} \hspace{-0.24ex}\times\hspace{-0.4ex} \bm{E} \hspace{.1ex} \operatorname{sin} \vartheta \operatorname{cos} \vartheta \hspace{-0.1ex}
+ \bm{k} \hspace{-0.25ex} \times \hspace{-0.32ex} ( \mathdotabove{\bm{k}} \hspace{-0.2ex}\times\hspace{-0.2ex} \bm{k} ) \hspace{-0.4ex}\times\hspace{-0.4ex} \bm{E} \hspace{.1ex} \operatorname{sin} \vartheta \left({1 \!-\! \operatorname{cos} \vartheta}\right) = $} \\[-0.08em]
%
\scalebox{0.78}[0.82]{$= \bm{k} \hspace{-0.24ex}\times\hspace{-0.4ex} \bm{E} \hspace{.4ex} \mathdotabove{\vartheta} \hspace{-0.1ex}
+ \mathdotabove{\bm{k}} \hspace{-0.24ex}\times\hspace{-0.4ex} \bm{E} \hspace{.1ex} \operatorname{sin} \vartheta \operatorname{cos} \vartheta \hspace{-0.1ex}
+ \hspace{-0.16ex} ( \mathdotabove{\bm{k}} \bm{k} \hspace{-0.1ex} \dotp \hspace{-0.16ex} \bm{k} \hspace{-0.1ex} - \tikzbackcancel[black!25]{$\bm{k} \mathdotabove{\bm{k}} \hspace{-0.16ex} \dotp \hspace{-0.1ex} \bm{k} \hspace{.1ex}$} ) \hspace{-0.4ex}\times\hspace{-0.4ex} \bm{E} \hspace{.1ex} \operatorname{sin} \vartheta \left({1 \!-\! \operatorname{cos} \vartheta}\right) \hspace{-0.1ex}
+ \hspace{-0.1ex} \bm{k} \hspace{-0.32ex}\times\hspace{-0.25ex}  \mathdotabove{\bm{k}} \hspace{-0.25ex}\times\hspace{-0.42ex} \bm{E} \hspace{.32ex} ({1 \!-\! \operatorname{cos} \vartheta}) = $} \\
%
\shoveright{\hfill \hspace{11.2em}\scalebox{.96}[.96]{$= \bm{k} \hspace{-0.24ex}\times\hspace{-0.4ex} \bm{E} \hspace{.4ex} \mathdotabove{\vartheta}
+ \mathdotabove{\bm{k}} \hspace{-0.24ex}\times\hspace{-0.4ex} \bm{E} \hspace{.1ex} \operatorname{sin} \vartheta
+ \hspace{-0.1ex} \bm{k} \hspace{-0.2ex}\times\hspace{-0.2ex}  \mathdotabove{\bm{k}} \hspace{-0.2ex}\times\hspace{-0.4ex} \bm{E} \hspace{.32ex} ({1 \hspace{-0.2ex} - \hspace{-0.2ex} \operatorname{cos} \vartheta}) \hspace{.1ex}.
$}}
\end{multline*}

Этот результат, подставленный в~определение~\eqref{angularvelocityvector} псевдо\-вектора~$\bm{\omega}$, даёт

\nopagebreak\vspace{-0.5em}\begin{equation}
\bm{\omega} = \bm{k} \hspace{.2ex} \mathdotabove{\vartheta}
+ \mathdotabove{\bm{k}} \operatorname{sin} \vartheta
+ \bm{k} \hspace{-0.1ex}\times\hspace{-0.1ex} \mathdotabove{\bm{k}} \left( 1 - \operatorname{cos} \vartheta \right) \hspace{-0.4ex}.
\end{equation}

\vspace{-0.32em} \noindent Вектор~$\bm{\omega}$ получился разложенным по~трём взаимно ортогональным направлениям\:--- $\bm{k}$, $\mathdotabove{\bm{k}}$ и~${\bm{k} \hspace{-0.1ex}\times\hspace{-0.1ex} \mathdotabove{\bm{k}}}$. При~неподвижной оси поворота ${\mathdotabove{\bm{k}} = \bm{0} \,\Rightarrow\, \bm{\omega} = \bm{k} \hspace{.1ex} \mathdotabove{\vartheta}}$.

Ещё одно представление~$\bm{\omega}$ связано с~компонентами тензора поворота~\eqref{componentsofrotationtensor}. Поскольку ${\bm{P} = \cosinematrix{\!j\mathcircabove{i}} \hspace{.4ex} \mathcircabove{\bm{e}}_i \mathcircabove{\bm{e}}_j}$, ${\bm{P}^{\T} \hspace{-0.32ex} = \cosinematrix{\hspace{-0.2ex}i\mathcircabove{j}} \hspace{.4ex} \mathcircabove{\bm{e}}_i \mathcircabove{\bm{e}}_j}$, а~векторы начального базиса~${\mathcircabove{\bm{e}}_i}$ неподвижны (со~временем не~меняются), то
\nopagebreak\vspace{.25em}\[ \mathdotabove{\bm{P}} = \cosinematrixdotted{\!j\mathcircabove{i}} \hspace{.4ex} \mathcircabove{\bm{e}}_i \mathcircabove{\bm{e}}_j
\hspace{.1ex} , \:\,
\mathdotabove{\bm{P}} \dotp \bm{P}^{\T} \hspace{-0.32ex} = \hspace{.1ex} \cosinematrixdotted{\hspace{-0.4ex}n\mathcircabove{i}} \hspace{.4ex}  \cosinematrix{\hspace{-0.4ex}n\mathcircabove{j}} \hspace{.4ex} \mathcircabove{\bm{e}}_i \mathcircabove{\bm{e}}_j
\hspace{.1ex}, \]
\nopagebreak\vspace{-0.64em}\begin{equation}
\bm{\omega} = - \hspace{.1ex} \smalldisplaystyleonehalf \hspace{.4ex} \cosinematrixdotted{\hspace{-0.4ex}n\mathcircabove{i}} \hspace{.4ex} \cosinematrix{\hspace{-0.4ex}n\mathcircabove{j}} \hspace{.5ex} \mathcircabove{\bm{e}}_i \hspace{-0.3ex}\times\hspace{-0.3ex} \mathcircabove{\bm{e}}_j \hspace{-0.1ex} =
\smalldisplaystyleonehalf \hspace{.2ex} \levicivita_{j\hspace{-0.06ex}ik} \hspace{.32ex} \cosinematrix{\hspace{-0.4ex}n\mathcircabove{j}} \hspace{.4ex} \cosinematrixdotted{\hspace{-0.4ex}n\mathcircabove{i}} \hspace{.4ex} \mathcircabove{\bm{e}}_k
\hspace{.2ex} .
\end{equation}

\vspace{-0.25em}
Отметим и~формулы
\nopagebreak\vspace{.16em}\begin{equation}\label{angularvelocityandbasisvectors}
\begin{array}{c}
\eqref{angularvelocityvector} \hspace{.32ex} \Rightarrow\:
\mathdotabove{\bm{e}}_i \mathcircabove{\bm{e}}_i \hspace{-0.1ex} = \bm{\omega} \times \hspace{-0.1ex} \bm{e}_i \mathcircabove{\bm{e}}_i \:\Rightarrow\:
\mathdotabove{\bm{e}}_i = \bm{\omega} \times \bm{e}_i \hspace{.12ex},
\\[.32em]
%
\eqref{angularvelocityvector} \hspace{.32ex} \Rightarrow\:
\bm{\omega} = \hspace{-0.1ex} - \hspace{.16ex} \smalldisplaystyleonehalf \hspace{-.2ex} \left( \mathdotabove{\bm{e}}_i \mathcircabove{\bm{e}}_i \hspace{-0.1ex} \dotp \mathcircabove{\bm{e}}_j \bm{e}_j \right)_{\hspace{-0.25ex}\Xcompanion} \hspace{-0.32ex}
= \hspace{-0.1ex} - \hspace{.16ex} \smalldisplaystyleonehalf \hspace{-.2ex} \left( \mathdotabove{\bm{e}}_i \bm{e}_i \hspace{.1ex} \right)_{\hspace{-0.1ex}\Xcompanion} \hspace{-0.25ex}
= \smalldisplaystyleonehalf \hspace{.4ex} \bm{e}_i \hspace{-0.16ex} \times \hspace{-0.1ex} \mathdotabove{\bm{e}}_i
\hspace{.12ex} .
\end{array}
\end{equation}



\textcolor{magenta}{Не всё то вектор, что имеет величину и направление. Поворот тела вокруг оси представляет, казалось~бы, вектор, имеющий численное значение, равное углу поворота, и~направление, совпадающее с~направлением оси вращения.} Однако, два таких поворота не~складываются как векторы, когда углы поворота не~бесконечно-м\'{а}лые. На~с\'{а}мом~же деле последовательные повороты не~складываются, а~умножаются.

Можно~ли складывать угловые скорости?\:--- Да, ведь угол поворота в~$\mathdotabove{\vartheta}$ бесконечномалый.\:--- Но только при вращении вокруг неподвижной оси?

...



\end{otherlanguage}

\en{\section{Variations}}

\ru{\section{Варьирование}}

\begin{otherlanguage}{russian}

Далее повсеместно будет использоваться сходная с~дифференцированием операция варьирования. Не~отсылая читателя к~\textcolor{magenta}{курсам} вариационного исчисления, ограничимся представлениями о~вариации~${\variation{x}}$ величины~$x$ как о~задаваемом нами бесконечно малом приращении, совместимом с~ограничениями\:--- связями~(constraints). Если ограничений для~$x$ нет, то ${\variation{x}}$ произвольна~(случайна). Но когда ${x \!=\! x(y\hspace{-0.1ex})}$\:--- функция независимого аргумента~$y$, тогда ${\variation{x} = x'\hspace{-0.25ex}(y\hspace{-0.1ex}) \hspace{.1ex} \variation{y}}$.



{\small
\setlength{\abovedisplayskip}{2pt}\setlength{\belowdisplayskip}{2pt}

Here we consider the exact differential of any set of position vectors~$\bm{r}_i$, that are functions of other variables ${\displaystyle \lbrace q_{1},q_{2},...,q_{m}\rbrace }$ and time t.

The actual displacement is the differential
\[\displaystyle d\bm{r}_{i} = \frac{\partial \bm{r}_{i}}{\partial t} \hspace{.16ex} dt \hspace{.2ex} + \sum_{j=1}^{m} {\frac{\partial \bm{r}_{i}}{\partial q^{\hspace{.1ex}j}}} \hspace{.2ex} dq^{\hspace{.1ex}j}\]

Now, imagine if we have an arbitrary path through the configuration space/manifold. This means it has to satisfy the constraints of the system but not the actual applied forces
\[\displaystyle \delta \bm{r}_{i}=\sum _{j=1}^{m} {\frac {\partial \bm{r}_{i}}{\partial q^{\hspace{.1ex}j}}} \hspace{.2ex} \delta q^{\hspace{.1ex}j}\]

\par}



З\'{а}писям с~вариациями характерны те~же особенности, что и с~дифференциалами.
Если, например, ${\variation{x}}$ и~${\variation{y}}$\:--- вариации~$x$ и~$y$, а~$u$ и~$v$\:--- кон\'{е}чные величины, то пишем ${u \variation{x} + v \variation{y} = \variation{w}}$, а~не~$w$\:--- даже когда ${\variation{w}}$ не~является вариацией величины~$w$; в~этом случае ${\variation{w}}$ это единое обозначение.
Разумеется, при~${u \narroweq u(x,y)}$, ${v \narroweq v(x,y)}$ и~${\partial_x v = \partial_y u}$ (${\hspace{.16ex}\frac{\partial}{\partial x} v = \frac{\partial}{\partial y} u\hspace{.16ex}}$) сумма~${\variation{w}}$ будет вариацией некой~$w$.

Варьируя тождество~\eqref{orthogonalityofrotationtensor}, получим ${\variation{\bm{P}} \hspace{-0.08ex} \dotp \bm{P}^{\T} \hspace{-0.2ex} = - \bm{P} \dotp \variation{\bm{P}}^{\T}\!}$.
Этот тензор антисимметричен, и~потому выражается через свой сопутствующий вектор~${\varvector{o}}$ как~${\variation{\bm{P}} \hspace{-0.08ex} \dotp \bm{P}^{\T} \hspace{-0.3ex} = \varvector{o} \hspace{-0.2ex} \times \hspace{-0.2ex} \bm{E}}$. Приходим к~соотношениям

\nopagebreak\vspace{-0.5em}\begin{equation}
\variation{\bm{P}} \hspace{-0.1ex} = \varvector{o} \hspace{-0.1ex} \times \hspace{-0.1ex} \bm{P} , \:\:
\varvector{o} = - \, \displaystyle \onehalf \left( { \variation{\bm{P}} \hspace{-0.08ex} \dotp^{\mathstrut} \bm{P}^{\T} } \right)_{\hspace{-0.25em}\Xcompanion} \hspace{-0.1ex} ,
\end{equation}

\vspace{-0.5em} \noindent аналогичным~\eqref{angularvelocityvector}. Вектор бесконечно малого поворота~${\varvector{o}}$ это не~\inquotesx{вариация $\bm{\mathrm{o}}$}[,] но единый символ (в~отличие от~${\variation{\bm{P}}}$).

Малый поворот определяется вектором~${\varvector{o}}$, но конечный поворот тоже \textcolor{magenta}{допускает(?)} векторное представление

...



\end{otherlanguage}

\en{\section{Polar decomposition}}

\ru{\section{Полярное разложение}}

\label{para:polardecomposition}

\begin{otherlanguage}{russian}

Любой тензор второй сложности~${\bm{F}}$ с~${\operatorname{det} F_{i\hspace{-0.1ex}j} \hspace{-0.16ex} \neq 0}$~(не~сингулярный) может быть представлен как

...

\begin{tcolorbox}
\small\setlength{\abovedisplayskip}{2pt}\setlength{\belowdisplayskip}{2pt}

\emph{Example.} Polar decompose tensor~${\bm{C} = C_{i\hspace{-0.1ex}j} \bm{e}_i \bm{e}_j}$, where $\bm{e}_k$ are mutually orthogonal unit vectors of basis, and $C_{i\hspace{-0.1ex}j}$ are tensor’s components

\begin{equation*}
C_{i\hspace{-0.1ex}j} =
\scalebox{0.92}[0.92]{$\left[\hspace{-0.2ex}\begin{array}{c@{\hspace{.6em}}c@{\hspace{.6em}}c}
-5 & 20 & 11 \\
10 & -15 & 23 \\
-3 & -5 & 10
\end{array}\hspace{-0.2ex}\right]$}
\end{equation*}

\begin{equation*}
\rotationtensor = O_{i\hspace{-0.1ex}j} \bm{e}_i \bm{e}_j \hspace{-0.2ex}
= \bm{O}_1 \hspace{-0.2ex} \dotp \bm{O}_2
\end{equation*}

\begin{equation*}
O_{i\hspace{-0.1ex}j} =
\scalebox{0.92}[0.92]{$\left[\hspace{-0.2ex}\begin{array}{c@{\hspace{.6em}}c@{\hspace{.6em}}c}
0 & \nicefrac{3}{5} & \nicefrac{4}{5} \\
0 & \nicefrac{4}{5} & - \hspace{.2ex} \nicefrac{3}{5} \\
-1 & 0 & 0
\end{array}\hspace{-0.4ex}\right]$}
\hspace{-0.25ex} = \hspace{-0.25ex}
\scalebox{0.92}[0.92]{$\left[\hspace{-0.2ex}\begin{array}{c@{\hspace{.6em}}c@{\hspace{.6em}}c}
0 & \mathcolor{red!33!white}{0} & 1 \\
\mathcolor{red!33!white}{0} & \mathcolor{red!77!black}{1} & \mathcolor{red!33!white}{0} \\
-1 & \mathcolor{red!33!white}{0} & 0
\end{array}\hspace{-0.2ex}\right]$}
\scalebox{0.92}[0.92]{$\left[\hspace{-0.2ex}\begin{array}{c@{\hspace{.6em}}c@{\hspace{.6em}}c}
\mathcolor{red!77!black}{1} & \mathcolor{red!33!white}{0} & \mathcolor{red!33!white}{0} \\
\mathcolor{red!33!white}{0} & \nicefrac{4}{5} & - \hspace{.2ex} \nicefrac{3}{5} \\
\mathcolor{red!33!white}{0} & \nicefrac{3}{5} & \nicefrac{4}{5}
\end{array}\hspace{-0.4ex}\right]$}
\end{equation*}

\begin{equation*}\begin{array}{c}
\bm{C} = \rotationtensor \hspace{-0.1ex} \dotp \bm{S_{\smash{\mathsf{R}}}}
\hspace{.1ex} , \:\:
\rotationtensor^{\T}\hspace{-0.5ex} \dotp \bm{C} = \bm{S_{\smash{\mathsf{R}}}}
\end{array}\end{equation*}

\begin{equation*}\begin{array}{c}
\bm{C} = \bm{S_{\smash{\mathsf{L}}}} \hspace{-0.25ex} \dotp \rotationtensor
\hspace{.1ex} , \:\:
\bm{C} \dotp \rotationtensor^{\T}\hspace{-0.5ex} = \bm{S_{\smash{\mathsf{L}}}}
\end{array}\end{equation*}

\begin{equation*}
S_{\smash{\mathsf{R}}\hspace{.15ex}i\hspace{-0.1ex}j} \hspace{-0.1ex} =
\scalebox{0.92}[0.92]{$\left[\hspace{-0.2ex}\begin{array}{c@{\hspace{.6em}}c@{\hspace{.6em}}c}
3 & 5 & -10 \\
5 & 0 & 25 \\
-10 & 25 & -5
\end{array}\hspace{-0.2ex}\right]$}
\end{equation*}

\begin{equation*}
S_{\smash{\mathsf{L}}\hspace{.15ex}i\hspace{-0.1ex}j} \hspace{-0.1ex} =
\scalebox{0.92}[0.92]{$\left[\hspace{-0.2ex}\begin{array}{c@{\hspace{.6em}}c@{\hspace{.6em}}c}
\nicefrac{104}{5} & \nicefrac{47}{5} & 5 \\
\nicefrac{47}{5} & - \hspace{.2ex} \nicefrac{129}{5} & -10 \\
5 & -10 & 3
\end{array}\hspace{-0.2ex}\right]$}
\end{equation*}

\par\end{tcolorbox}

...


\end{otherlanguage}


\newpage

\en{\section{Tensors in oblique basis}}

\ru{\section{Тензоры в косоугольном базисе}}

\en{Until now}\ru{До~сих~пор} \ru{использовался }\en{a~basis}\ru{базис} \en{of~three mutually orthogonal unit vectors}\ru{трёх взаимно ортогональных единичных векторов}~${\bm{e}_i}$\en{ was used}.
\en{Presently}\ru{Теперь} \en{take}\ru{возьмём} \en{a~basis}\ru{базис} \en{of~any three}\ru{из~любых трёх} \en{linearly independent}\ru{линейно независимых}~(\en{non\hbox{-}coplanar}\ru{некомпланарных}) \en{vectors}\ru{векторов}~${\bm{a}_i}$.

%%\begin{figure}[!htbp]
%%\begin{center}
\begin{wrapfigure}[20]{R}{.48\textwidth}
\makebox[.5\textwidth][c]{\begin{minipage}[t]{.5\textwidth}

% some vector to draw
\pgfmathsetmacro{\lengthofvector}{2.66}
	\pgfmathsetmacro{\vectoranglefromz}{33}
	\pgfmathsetmacro{\vectoranglefromx}{59}

\tdplotsetmaincoords{33}{109} % orientation of camera

% vectors of basis
\pgfmathsetmacro{\firstlength}{0.8}
	\pgfmathsetmacro{\firstanglefromz}{90} % first and second are xy plane
	\pgfmathsetmacro{\firstanglefromx}{77} % first is not orthogonal to second
\pgfmathsetmacro{\secondlength}{1.1}
	\pgfmathsetmacro{\secondanglefromz}{90} % first and second are xy plane
	\pgfmathsetmacro{\secondanglefromx}{0}
\pgfmathsetmacro{\thirdlength}{1}
	\pgfmathsetmacro{\thirdanglefromz}{-15}
	\pgfmathsetmacro{\thirdanglefromx}{50}

\vspace{-0.2em}
\hspace{1.32em}\scalebox{0.96}[0.96]{%
\begin{tikzpicture}[scale=2.5, tdplot_main_coords] % tdplot_main_coords style to use 3dplot

	\coordinate (O) at (0,0,0);

	% draw axes and vectors of basis
	\tdplotsetcoord{A1}{\firstlength}{\firstanglefromz}{\firstanglefromx}
	\tdplotsetcoord{A2}{\secondlength}{\secondanglefromz}{\secondanglefromx}
	\tdplotsetcoord{A3}{\thirdlength}{\thirdanglefromz}{\thirdanglefromx}

	\draw [line width=0.4pt, blue] (O) -- ($ 2.2*(A1) $);
	\draw [line width=1.25pt, blue, -{Latex[round, length=3.6mm, width=2.4mm]}]
		(O) -- (A1)
		node[pos=0.93, above, inner sep=0pt, outer sep=6pt] {${\bm{a}}_1$};

	\draw [line width=0.4pt, blue] (O) -- ($ 1.08*(A2) $);
	\draw [line width=1.25pt, blue, -{Latex[round, length=3.6mm, width=2.4mm]}]
		(O) -- (A2)
		node[pos=0.97, above left, inner sep=0pt, outer sep=3.3pt] {${\bm{a}}_2$};

	\draw [line width=0.4pt, blue] (O) -- ($ 1.12*(A3) $);
	\draw [line width=1.25pt, blue, -{Latex[round, length=3.6mm, width=2.4mm]}]
		(O) -- (A3)
		node[pos=1.02, below right, inner sep=0pt, outer sep=7pt] {${\bm{a}}_3$};

	% define vector by sperical coordinates {length}{angle from z}{angle from x}
	% (plus it defines its projections)
	\tdplotsetcoord{V}{\lengthofvector}{\vectoranglefromz}{\vectoranglefromx}

	% get components of vector
	\coordinate (ParallelToThird) at ($ (V) - (A3) $);
	\coordinate (VcomponentXY) at (intersection of V--ParallelToThird and O--Vxy);

	\coordinate (ParallelToSecond) at ($ (VcomponentXY) - (A2xy) $);
	\coordinate (ParallelToFirst) at ($ (VcomponentXY) - (A1xy) $);

	\coordinate (Vcomponent1) at (intersection of VcomponentXY--ParallelToSecond and O--A1);
	\coordinate (Vcomponent2) at (intersection of VcomponentXY--ParallelToFirst and O--A2);

	\draw [line width=0.4pt, dotted, color=black] (O) -- (VcomponentXY); % projection on first & second vectors’ plane
	\draw [line width=0.4pt, dotted, color=black] (V) -- (VcomponentXY);
	\draw [line width=0.4pt, dotted, color=black] (VcomponentXY) -- (Vcomponent1);
	\draw [line width=0.4pt, dotted, color=black] (VcomponentXY) -- (Vcomponent2);

	% draw parallelopiped
	\coordinate (onPlane23) at ($ (Vcomponent2) + (V) - (VcomponentXY) $);
	\draw [line width=0.4pt, dotted, color=black] (Vcomponent2) -- (onPlane23);
	\draw [line width=0.4pt, dotted, color=black] (V) -- (onPlane23);

	\coordinate (onPlane13) at ($ (Vcomponent1) + (V) - (VcomponentXY) $);
	\draw [line width=0.4pt, dotted, color=black] (Vcomponent1) -- (onPlane13);
	\draw [line width=0.4pt, dotted, color=black] (V) -- (onPlane13);

	\coordinate (onAxis3) at ($ (V) - (VcomponentXY) $);
	\draw [line width=0.4pt, dotted, color=black] (O) -- (onAxis3);
	\draw [line width=0.4pt, dotted, color=black] (onPlane13) -- (onAxis3);
	\draw [line width=0.4pt, dotted, color=black] (onPlane23) -- (onAxis3);

	\draw [line width=0.4pt, dotted, color=black] (O) -- (onPlane13);
	\draw [line width=0.4pt, dotted, color=black] (O) -- (onPlane23);

	% draw components of vector
	\draw [color=black, line width=1.6pt, line cap=round, dash pattern=on 0pt off 1.6\pgflinewidth,
		-{Stealth[round, length=4mm, width=2.4mm]}]
		(O) -- (Vcomponent1)
		node[pos=0.6, below, shape=circle, fill=white, inner sep=-2pt, outer sep=2pt] {${v^1 \hspace{-0.1ex} \bm{a}_1}$};

	\draw [color=black, line width=1.6pt, line cap=round, dash pattern=on 0pt off 1.6\pgflinewidth,
		-{Stealth[round, length=4mm, width=2.4mm]}]
		(Vcomponent1) -- (VcomponentXY)
		node[pos=0.4, below right, fill=white, shape=circle, inner sep=0pt, outer sep=4pt] {${v^2 \hspace{-0.1ex} \bm{a}_2}$};

	\draw [color=black, line width=1.6pt, line cap=round, dash pattern=on 0pt off 1.6\pgflinewidth,
		-{Stealth[round, length=4mm, width=2.4mm]}]
		(VcomponentXY) -- (V)
		node[pos=0.55, above right, shape=circle, fill=white, inner sep=0.3pt, outer sep=6.2pt] {${v^3 \hspace{-0.1ex} \bm{a}_3}$};

	% draw vector
	\draw [line width=1.6pt, black, -{Stealth[round, length=5mm, width=2.8mm]}]
		(O) -- (V)
		node[pos=0.68, above, fill=white, inner sep=1pt, outer sep=5pt] {\scalebox{1.2}[1.2]{${\bm{v}}$}};

\end{tikzpicture}}
\vspace{-0.32em}\caption{}\label{fig:ObliqueCoordinates}
\end{minipage}}
\end{wrapfigure}
%%\end{center}
%%\end{figure}\vspace{-1.5em}

\en{Decomposition}\ru{Декомпозиция~(разложение)} \en{of~vector}\ru{вектора}~$\bm{v}$ \en{in~basis}\ru{в~базисе}~${\bm{a}_i}$~(\figref{fig:ObliqueCoordinates}) \en{is linear combination}\ru{есть линейная комбинация}

\nopagebreak\vspace{-0.2em}\en{\vspace{-0.8em}}\begin{equation}\label{decompositionbyobliquebasis}
\bm{v} = v^{i} \hspace{-0.1ex} \bm{a}_i
\hspace{.1ex} .
\end{equation}

\begin{otherlanguage}{russian}

{\small\setlength{\abovedisplayskip}{2pt}\setlength{\belowdisplayskip}{2pt}
Соглашение о~суммировании обретает новые положения: повторяющиеся~(\inquotes{немые}) индексы суммирования находятся на~\hbox{разных} уровнях, а~свободные индексы в~обеих частях равенства\:--- на~одной высоте (${a_i \hspace{-0.16ex} = \hspace{.1ex} b_{i\hspace{-0.1ex}j} c^{\hspace{.2ex}j}}$\ru{\:---}\en{ is} \en{correct}\ru{корректно}, ${a_i \hspace{-0.16ex} = \hspace{.1ex} b_{kk}^{\hspace{.1ex}i}}$\ru{\:---}\en{ is} \en{wrong twice}\ru{дважды ошибочно}).
\par}

В~таком базисе уж\'{е} ${\bm{v} \dotp \bm{a}_i \hspace{-0.1ex} = \hspace{.1ex} v^{k} \bm{a}_k \hspace{-0.1ex} \dotp \bm{a}_i \neq\vspace{.2ex} v^{i}}$\hspace{-0.25ex}, ведь~тут ${\bm{a}_i \hspace{-0.1ex} \dotp \bm{a}_k \neq\vspace{.2ex} \delta_{ik}}$.

Дополним~же \hbox{базис}~${\bm{a}_i}$ ещё другой тройкой векторов~\hbox{${\bm{a}^{\hspace{-0.05ex}i}}$\hspace{-0.25ex},} \hbox{называемых} кобазисом или~взаимным базисом, чтобы

\nopagebreak\vspace{-0.3em}\begin{equation}\label{fundamentalpropertyofcobasis}
\begin{array}{c}
\bm{a}_i \dotp \bm{a}^{\hspace{.1ex}j} \hspace{-0.1ex} = \hspace{.1ex} \delta_i^{\hspace{.1ex}j} , \:\:
\bm{a}^{\hspace{-0.05ex}i} \hspace{-0.1ex} \dotp \bm{a}_j \hspace{-0.1ex} = \hspace{.1ex} \delta_{\hspace{-0.1ex}j}^{\hspace{.1ex}i} \hspace{.16ex}, \\[0.2em]
\bm{E} = \bm{a}^{\hspace{-0.1ex}i} \hspace{-0.1ex} \bm{a}_i \hspace{-0.1ex} = \bm{a}_i \hspace{.1ex} \bm{a}^{\hspace{-0.1ex}i}
\hspace{-0.2ex} .
\end{array}
\end{equation}

\vspace{-0.1em}\noindent Это\:--- основное свойство кобазиса. Орто\-нормирован\-ный~(орто\-нормаль\-ный) базис может быть определён как совпад\'{а}ющий со~своим кобазисом: ${\bm{e}^{\hspace{.05ex}i} \hspace{-0.2ex} = \bm{e}_{i}}$.

%%\inquotes{В~декартовых координатах}, когда базис\:--- отронормальный правый:
%%\begin{itemize}
%%\item компоненты единичного~(\inquotes{метрического}) тензора\:--- дельта Кронекера,
%%\item компоненты (псевдо)тензора Л\'{е}ви\hbox{-\!}Чив\'{и}ты\:--- символ Веблена c~\inquotes{+} для \inquotes{правой} и с~\inquotes{-} для \inquotes{левой} тройки базисных векторов.
%%\end{itemize}

\begin{comment} %%
\vspace{-0.5em}\[
\bm{a}_i \dotp \bm{a}^{\hspace{0.1ex}j} \hspace{-0.1ex} = \hspace{-0.2ex}
\scalebox{0.8}[0.8]{$\left[ \begin{array}{ccc}
\bm{a}_1 \hspace{-0.1ex} \dotp \bm{a}^{\hspace{-0.1ex}1} & \bm{a}_1 \hspace{-0.1ex} \dotp \bm{a}^2 & \bm{a}_1 \hspace{-0.1ex} \dotp \bm{a}^3 \\
\bm{a}_2 \hspace{-0.1ex} \dotp \bm{a}^{\hspace{-0.1ex}1} & \bm{a}_2 \hspace{-0.1ex} \dotp \bm{a}^2 & \bm{a}_2 \hspace{-0.1ex} \dotp \bm{a}^3 \\
\bm{a}_3 \hspace{-0.1ex} \dotp \bm{a}^{\hspace{-0.1ex}1} & \bm{a}_3 \hspace{-0.1ex} \dotp \bm{a}^2 & \bm{a}_3 \hspace{-0.1ex} \dotp \bm{a}^3
\end{array} \right]$} \!=\!
\scalebox{0.8}[0.8]{$\left[ \begin{array}{ccc}
1 & 0 & 0 \\
0 & 1 & 0 \\
0 & 0 & 1
\end{array} \right]$} \!=
\hspace{0.1ex} \delta_i^{\hspace{0.1ex}j}
\]
\end{comment} %%

Для, к~примеру, первого вектора кобазиса~${\bm{a}^{\hspace{-0.1ex}1}\hspace{-0.1ex}}$

\nopagebreak\vspace{-0.1em}\begin{equation*}
\scalebox{0.9}[0.9]{$\left\{\hspace{-0.16em}\begin{array}{l}
\bm{a}^{\hspace{-0.1ex}1} \hspace{-0.2ex} \dotp \bm{a}_{1} = 1 \\
\bm{a}^{\hspace{-0.1ex}1} \hspace{-0.2ex} \dotp \bm{a}_{2} = 0 \\
\bm{a}^{\hspace{-0.1ex}1} \hspace{-0.2ex} \dotp \bm{a}_{3} = 0 \\[.05em]
\end{array}\right.$}
\Rightarrow\hspace{.33em}
%
\scalebox{0.92}[0.92]{$\left\{\hspace{-0.12em}\begin{array}{l}
\bm{a}^{\hspace{-0.1ex}1} \hspace{-0.2ex} \dotp \hspace{.2ex} \bm{a}_{1} = 1 \\[.1em]
\gamma \hspace{.1ex} \bm{a}^{\hspace{-0.1ex}1} \hspace{-0.2ex} = \hspace{0.1ex} \bm{a}_2 \hspace{-0.1ex} \times \bm{a}_3 \\[.08em]
\end{array}\right.$}
\Rightarrow\hspace{.33em}
%
\scalebox{0.96}[0.96]{$\left\{\hspace{-0.11em}\begin{array}{l}
\bm{a}^{\hspace{-0.1ex}1} \hspace{-0.2ex} =
\displaystyle \nicefrac{\scalebox{0.95}{$1$}\hspace{.1ex}}{\scalebox{1.02}{$\gamma$}} \hspace{.5ex} \bm{a}_2 \hspace{-0.1ex} \times \bm{a}_3 \hspace{0.1ex} \\[.08em]
\gamma = \hspace{0.1ex} \bm{a}_2 \hspace{-0.1ex} \times \bm{a}_3 \hspace{.1ex} \dotp \hspace{.25ex} \bm{a}_{1} \\[.08em]
\end{array}\right.$}
\end{equation*}

\vspace{-0.1em} \noindent
Коэффициент~$\gamma$ получился равным (с~точностью до~знака для \inquotes{левой} тройки ${\bm{a}_i}$) объёму параллелепипеда, построенного на~векторах~$\bm{a}_i$.
В~\pararef{para:crossproduct+levicivita} тот~же объём был представлен как~\!${\sqrt{\hspace{-0.36ex}\mathstrut{\textsl{g}}}\hspace{.16ex}}$, и~это неспроста, ведь он совпад\'{а}ет с~квадратным корнем из~грамиана~\hbox{$\textsl{g} \hspace{.1ex} \hspace{.25ex} \equiv \hspace{.2ex} \operatorname{det} \textsl{g}_{i\hspace{-0.1ex}j}$\hspace{-0.12ex}}\:--- определителя симметричной матрицы~\hbox{J.\,P.\,Gram’а}~${\textsl{g}_{i\hspace{-0.1ex}j} \hspace{-0.1ex} \equiv \hspace{.2ex} \bm{a}_i \dotp \bm{a}_j}$.

\noindent
${ \tikz[baseline=-1ex] \draw [line width=.5pt, color=black, fill=white] (0, 0) circle (.8ex);
\hspace{.6em} }$
Доказательство напоминает вывод~\eqref{doubleveblen}. \inquotes{Тройное} произведение ${\bm{a}_i \hspace{-0.25ex} \times \hspace{-0.25ex} \bm{a}_j \dotp \hspace{.1ex} \bm{a}_k}$ в~каком\hbox{-}нибудь орто\-нормаль\-ном базисе~${\bm{e}_i}$ вычисл\'{и}мо как~детерминант (с~\inquotes{$-$} для \inquotes{левой} тройки ${\bm{a}_i}$) по~строкам

\nopagebreak\vspace{-0.2em}\begin{equation*}
\levicivita_{i\hspace{-0.1ex}j\hspace{-0.1ex}k} \hspace{-0.1ex}
\equiv \hspace{.16ex} \bm{a}_i \hspace{-0.25ex} \times \hspace{-0.25ex} \bm{a}_j \dotp \hspace{0.1ex} \bm{a}_k =
\hspace{.1ex} \pm \hspace{.1ex} \operatorname{det}\hspace{-0.1ex}
\scalebox{0.92}[0.92]{$\left[\hspace{-0.2ex}\begin{array}{c@{\hspace{.64em}}c@{\hspace{.64em}}c}
\bm{a}_i \narrowdotp\hspace{.12ex} \bm{e}_1 & \bm{a}_i \narrowdotp\hspace{.12ex} \bm{e}_2 & \bm{a}_i \narrowdotp\hspace{.12ex} \bm{e}_3 \\
\bm{a}_j \narrowdotp\hspace{.12ex} \bm{e}_1 & \bm{a}_j \narrowdotp\hspace{.12ex} \bm{e}_2 & \bm{a}_j \narrowdotp\hspace{.12ex} \bm{e}_3 \\
\bm{a}_k \narrowdotp\hspace{.12ex} \bm{e}_1 & \bm{a}_k \narrowdotp\hspace{.12ex} \bm{e}_2 & \bm{a}_k \narrowdotp\hspace{.12ex} \bm{e}_3
\end{array}\hspace{-0.12ex}\right]$}
\end{equation*}

\vspace{-0.2em} \noindent или по~столбцам

\nopagebreak\vspace{-0.4em}\begin{equation*}
\levicivita_{pqr} \hspace{-0.15ex}
\equiv \hspace{.16ex} \bm{a}_p \hspace{-0.25ex} \times \hspace{-0.25ex} \bm{a}_q \dotp \hspace{.1ex} \bm{a}_r =
\hspace{.1ex} \pm \hspace{.1ex} \operatorname{det}\hspace{-0.1ex}
\scalebox{0.92}[0.92]{$\left[\hspace{-0.2ex}\begin{array}{c@{\hspace{.64em}}c@{\hspace{.64em}}c}
\bm{a}_p \narrowdotp\hspace{.12ex} \bm{e}_1 & \bm{a}_q \narrowdotp\hspace{.12ex} \bm{e}_1 & \bm{a}_r \narrowdotp\hspace{.12ex} \bm{e}_1 \\
\bm{a}_p \narrowdotp\hspace{.12ex} \bm{e}_2 & \bm{a}_q \narrowdotp\hspace{.12ex} \bm{e}_2 & \bm{a}_r \narrowdotp\hspace{.12ex} \bm{e}_2 \\
\bm{a}_p \narrowdotp\hspace{.12ex} \bm{e}_3 & \bm{a}_q \narrowdotp\hspace{.12ex} \bm{e}_3 & \bm{a}_r \narrowdotp\hspace{.12ex} \bm{e}_3
\end{array}\hspace{-0.12ex}\right]$} .
\end{equation*}

\vspace{.1em} \noindent
Произведение определителей~${\levicivita_{i\hspace{-0.1ex}j\hspace{-0.1ex}k} \levicivita_{pqr}}$ равно определителю произведения матриц, \en{and}\ru{а}~\en{elements of~the~latter}\ru{элементы последнего}\ru{\:---}\en{ are} \en{sums}\ru{суммы} \en{like}\ru{вида}
${\bm{a}_i \hspace{-0.1ex} \dotp \bm{e}_s \bm{a}_p \hspace{-0.2ex} \dotp \bm{e}_s \hspace{-0.2ex}
= \bm{a}_i \hspace{-0.1ex} \dotp \bm{e}_s \bm{e}_s \hspace{-0.2ex} \dotp \bm{a}_p \hspace{-0.15ex}
= \bm{a}_i \hspace{-0.1ex} \dotp \hspace{-0.1ex} \bm{E} \dotp \bm{a}_p \hspace{-0.15ex}
= \bm{a}_i \hspace{-0.1ex} \dotp \bm{a}_p}$, в~результате

\nopagebreak\vspace{-0.2em}\begin{equation*}
\levicivita_{i\hspace{-0.1ex}j\hspace{-0.1ex}k} \levicivita_{pqr} = \hspace{0.25ex}
\operatorname{det}\hspace{-0.1ex}
\scalebox{0.92}[0.92]{$\left[\hspace{-0.16ex}\begin{array}{c@{\hspace{.64em}}c@{\hspace{.64em}}c}
\bm{a}_i \narrowdotp\hspace{.12ex} \bm{a}_p & \bm{a}_i \narrowdotp\hspace{.12ex} \bm{a}_q & \bm{a}_i \narrowdotp\hspace{.12ex} \bm{a}_r \\
\bm{a}_j \narrowdotp\hspace{.12ex} \bm{a}_p & \bm{a}_j \narrowdotp\hspace{.12ex} \bm{a}_q & \bm{a}_j \narrowdotp\hspace{.12ex} \bm{a}_r \\
\bm{a}_k \narrowdotp\hspace{.12ex} \bm{a}_p & \bm{a}_k \narrowdotp\hspace{.12ex} \bm{a}_q & \bm{a}_k \narrowdotp\hspace{.12ex} \bm{a}_r
\end{array}\hspace{-0.12ex}\right]$} ;
\end{equation*}

\vspace{-0.1em} \noindent
${i \narroweq p \narroweq 1}$, ${j \narroweq q \narroweq 2}$, ${k \narroweq r \narroweq 3}$ ${\,\Rightarrow}$ ${\levicivita_{123} \hspace{.2ex} \levicivita_{123} \hspace{-0.12ex} = \underset{\raisebox{.15em}{\scalebox{0.7}{$i$,$\hspace{.15ex}j$}}}{\operatorname{det}} \left( \bm{a}_i \dotp \bm{a}_j \right) \hspace{-0.1ex} = \underset{\raisebox{.15em}{\scalebox{0.7}{$i$,$\hspace{.15ex}j$}}}{\operatorname{det}} \, \textsl{g}_{i\hspace{-0.1ex}j}}$.
${ \hspace{.6em}
\tikz[baseline=-0.6ex] \draw [color=black, fill=black] (0, 0) circle (.8ex); }$

Представляя~${\bm{a}^{\hspace{-0.1ex}1}\hspace{-0.1ex}}$ и~остальные векторы кобазиза суммой

\nopagebreak\vspace{.8em}\begin{equation*}
\begin{array}{l@{\hspace{.3em}}c@{\hspace{.36em}}r}
\pm \hspace{.33ex} 2 \hspace{.1ex} \scalebox{0.95}[0.96]{$\sqrt{\hspace{-0.36ex}\mathstrut{\textsl{g}}}$} \hspace{.5ex} \bm{a}^{\hspace{-0.1ex}1} & = & \bm{a}_2 \times \bm{a}_3 \hspace{.2ex} \tikzmark{BeginPlusToMinus} - \hspace{.2ex} \bm{a}_3 \times \bm{a}_2 \tikzmark{EndPlusToMinus}
\hspace{.2ex} ,
\end{array}
\end{equation*}%
\AddOverBrace[line width=.75pt][0,-0.2ex]{BeginPlusToMinus}{EndPlusToMinus}%
{${\scriptstyle {+ \hspace{.4ex} \bm{a}_2 \hspace{.1ex} \times \hspace{.2ex} \bm{a}_3}}$}

\vspace{-1.3em} \noindent приходим к~общей формуле (с~\inquotes{$-$} для \inquotes{левой} тройки ${\bm{a}_i}$)

\nopagebreak\begin{equation}\label{basisvectorstocobasisvectors}
\bm{a}^{\hspace{-0.05ex}i} \hspace{-0.2ex}
= \displaystyle \pm \hspace{.2ex} \frac{\raisemath{-0.4ex}{1}}{2 \hspace{.1ex} \scalebox{0.95}[0.96]{$\sqrt{\hspace{-0.36ex}\mathstrut{\textsl{g}}}$}} \hspace{.5ex} e^{i\hspace{-0.1ex}j\hspace{-0.1ex}k} \hspace{.1ex} \bm{a}_j \hspace{-0.1ex} \times \bm{a}_k \hspace{.1ex},
\:\:
\sqrt{\hspace{-0.36ex}\mathstrut{\textsl{g}}} \hspace{.25ex}
\equiv \hspace{.15ex} \pm \hspace{.4ex} \bm{a}_1 \hspace{-0.2ex} \times \bm{a}_2 \hspace{.1ex} \dotp \hspace{.25ex} \bm{a}_3
> 0
\hspace{.2ex} .
\end{equation}

\vspace{-0.2em} \noindent
Здесь ${e^{i\hspace{-0.1ex}j\hspace{-0.1ex}k}}$ по\hbox{-}прежнему символ перестановки Veblen’а (${\pm 1}$ или~$0$):
${e^{i\hspace{-0.1ex}j\hspace{-0.1ex}k} \hspace{-0.1ex} \equiv e_{i\hspace{-0.1ex}j\hspace{-0.1ex}k}}$.
Произведение~${\bm{a}_j \hspace{-0.1ex} \times \bm{a}_k = \hspace{.1ex} \levicivita_{j\hspace{-0.1ex}kn} \hspace{.2ex} \bm{a}^n\hspace{-0.12ex}}$, компоненты тензора Л\'{е}ви\hbox{-\!}Чив\'{и}ты~${ \levicivita_{j\hspace{-0.1ex}kn} \hspace{-0.2ex} = \pm \hspace{.33ex} e_{j\hspace{-0.1ex}kn} \hspace{-0.1ex} \sqrt{\hspace{-0.36ex}\mathstrut{\textsl{g}}} }$,
\en{and}\ru{а}~\en{by}\ru{по}~\eqref{veblencontraction} ${e^{i\hspace{-0.1ex}j\hspace{-0.1ex}k} e_{j\hspace{-0.1ex}kn} \hspace{-0.25ex} = 2 \hspace{.1ex} \delta_n^{\hspace{.1ex}i}}$.
\en{Thus}\ru{Так что}

\nopagebreak\vspace{-0.1em}\begin{equation*}\scalebox{0.96}[0.96]{$%
\begin{array}{l@{\hspace{.25em}}c@{\hspace{.33em}}r}
\bm{a}^{\hspace{-0.1ex}1} & = & \pm \hspace{.25ex} \displaystyle \nicefrac{\scalebox{0.95}{$1$}}{\hspace{-0.25ex}\sqrt{\hspace{-0.2ex}\scalebox{0.96}{$\mathstrut{\textsl{g}}$}}} \hspace{.2ex} \left( \hspace{.1ex} \bm{a}_2 \hspace{-0.1ex} \times \hspace{-0.1ex} \bm{a}_3 \hspace{.1ex} \right) \hspace{-0.3ex},
\end{array}
\begin{array}{l@{\hspace{.25em}}c@{\hspace{.33em}}r}
\bm{a}^2 & = & \pm \hspace{.25ex} \displaystyle \nicefrac{\scalebox{0.95}{$1$}}{\hspace{-0.25ex}\sqrt{\hspace{-0.2ex}\scalebox{0.96}{$\mathstrut{\textsl{g}}$}}} \hspace{.2ex} \left( \hspace{.1ex} \bm{a}_3 \hspace{-0.1ex} \times \hspace{-0.1ex} \bm{a}_1 \hspace{.1ex} \right) \hspace{-0.3ex},
\end{array}
\begin{array}{l@{\hspace{.25em}}c@{\hspace{.33em}}r}
\bm{a}^3 & = & \pm \hspace{.25ex} \displaystyle \nicefrac{\scalebox{0.95}{$1$}}{\hspace{-0.25ex}\sqrt{\hspace{-0.2ex}\scalebox{0.96}{$\mathstrut{\textsl{g}}$}}} \hspace{.2ex} \left( \hspace{.1ex} \bm{a}_1 \hspace{-0.16ex} \times \hspace{-0.1ex} \bm{a}_2 \hspace{.1ex} \right) \hspace{-0.28ex}.
\end{array}%
$}\end{equation*}

\begin{tcolorbox}
\small\setlength{\abovedisplayskip}{2pt}\setlength{\belowdisplayskip}{2pt}

\emph{Example.} Get cobasis for basis~$\bm{a}_i$ when
\[ \begin{array}{l}
\bm{a}_1 = \bm{e}_1 \hspace{-0.2ex} + \bm{e}_2 \hspace{.1ex},\\
\bm{a}_2 = \bm{e}_1 \hspace{-0.2ex} + \bm{e}_3 \hspace{.1ex},\\
\bm{a}_3 = \bm{e}_2 \hspace{-0.2ex} + \bm{e}_3 \hspace{.1ex}.
\end{array} \]

\[
\sqrt{\hspace{-0.36ex}\mathstrut{\textsl{g}}} \hspace{.32ex} =
- \hspace{.4ex} \bm{a}_1 \hspace{-0.2ex} \times \bm{a}_2 \hspace{0.1ex} \dotp \hspace{0.25ex} \bm{a}_3 \hspace{.2ex} =
- \operatorname{det}\hspace{-0.1ex}
\scalebox{0.92}[0.92]{$\left[\hspace{-0.16ex}\begin{array}{c@{\hspace{.64em}}c@{\hspace{.64em}}c}
1 & 1 & 0 \\
1 & 0 & 1 \\
0 & 1 & 1
\end{array}\hspace{-0.12ex}\right]$} \hspace{-0.5ex} = 2 \hspace{.25ex};
\]
\[
- \hspace{.4ex} \bm{a}_2 \hspace{-0.2ex} \times \bm{a}_3 = \operatorname{det}\hspace{-0.1ex}
\scalebox{0.92}[0.92]{$\left[\hspace{-0.16ex}\begin{array}{c@{\hspace{.6em}}c@{\hspace{.5em}}c}
1 & \bm{e}_1 & 0 \\
0 & \bm{e}_2 & 1 \\
1 & \bm{e}_3 & 1
\end{array}\hspace{-0.2ex}\right]$} \hspace{-0.5ex} = \bm{e}_1 \hspace{-0.2ex} + \bm{e}_2 \hspace{-0.2ex} - \bm{e}_3 \hspace{.1ex},
\]
\[
- \hspace{.4ex} \bm{a}_3 \hspace{-0.2ex} \times \bm{a}_1 = \operatorname{det}\hspace{-0.1ex}
\scalebox{0.92}[0.92]{$\left[\hspace{-0.16ex}\begin{array}{c@{\hspace{.6em}}c@{\hspace{.5em}}c}
0 & \bm{e}_1 & 1 \\
1 & \bm{e}_2 & 1 \\
1 & \bm{e}_3 & 0
\end{array}\hspace{-0.2ex}\right]$} \hspace{-0.5ex} = \bm{e}_1 \hspace{-0.2ex} + \bm{e}_3 \hspace{-0.2ex} - \bm{e}_2 \hspace{.1ex},
\]
\[
- \hspace{.4ex} \bm{a}_1 \hspace{-0.2ex} \times \bm{a}_2 = \operatorname{det}\hspace{-0.1ex}
\scalebox{0.92}[0.92]{$\left[\hspace{-0.16ex}\begin{array}{c@{\hspace{.6em}}c@{\hspace{.5em}}c}
1 & \bm{e}_1 & 1 \\
1 & \bm{e}_2 & 0 \\
0 & \bm{e}_3 & 1
\end{array}\hspace{-0.2ex}\right]$} \hspace{-0.5ex} = \bm{e}_2 \hspace{-0.2ex} + \bm{e}_3 \hspace{-0.2ex} - \bm{e}_1 \hspace{.1ex}
\]

\vspace{-0.4em}and finally
\vspace{-0.4em}\[\begin{array}{l}
\bm{a}^1 \hspace{-0.2ex}=\hspace{.1ex} \smalldisplaystyleonehalf \hspace{-0.2ex} \left(^{\mathstrut} \bm{e}_1 \hspace{-0.2ex} + \bm{e}_2 \hspace{-0.2ex} - \bm{e}_3 \right)
\hspace{-0.5ex} ,
\\[.5em]
\bm{a}^2 \hspace{-0.2ex}=\hspace{.1ex} \smalldisplaystyleonehalf \hspace{-0.2ex} \left(^{\mathstrut} \bm{e}_1 \hspace{-0.2ex} - \bm{e}_2 \hspace{-0.2ex} + \bm{e}_3 \right)
\hspace{-0.5ex} ,
\\[.5em]
\bm{a}^3 \hspace{-0.2ex}=\hspace{.1ex} \smalldisplaystyleonehalf \hspace{-0.2ex} \left(^{\mathstrut} \hspace{-0.2ex} {- \bm{e}_1} \hspace{-0.2ex} + \bm{e}_2 \hspace{-0.2ex} + \bm{e}_3 \right)
\hspace{-0.5ex} .
\end{array}\]

\par\end{tcolorbox}

Имея кобазис, возможно не~только разложить по~нему любой вектор~(\figref{fig:DecompositionOfVector}), но~и найти коэффициенты разложения~\eqref{decompositionbyobliquebasis}:
\begin{equation}\begin{array}{c}
\bm{v} = v^{i} \hspace{-0.1ex} \bm{a}_i = v_{i} \hspace{0.1ex} \bm{a}^{\hspace{-0.05ex}i} \hspace{-0.25ex}, \\[0.16em]
\bm{v} \dotp \bm{a}^{\hspace{-0.05ex}i} = v^{k} \hspace{-0.1ex} \bm{a}_k \hspace{-0.1ex} \dotp \bm{a}^{\hspace{-0.05ex}i} = v^{i} \hspace{-0.25ex}, \:\;
v_{i} \hspace{-0.1ex} = \bm{v} \dotp \bm{a}_i \hspace{0.1ex}.
\end{array}\end{equation}
\noindent Коэффициенты~${v_i}$ называются ко\-вариант\-ными компонентами вектора~$\bm{v}$, а~${v^i \hspace{-0.25ex}}$\:--- его контра\-вариант\-ными%
\footnote{Потому что они меняются обратно~(contra) изменению длин базисных векторов~${\bm{a}_i}$.}\hspace{-0.2ex}
компонентами.

Есть литература о~тензорах, где различают ко\-вариант\-ные и~контра\-вариант\-ные... векторы~(\en{and}\ru{и}~\inquotes{\en{covectors}\ru{ковекторы}}, \inquotes{dual vectors}). Не~ст\'{о}ит вводить читателя в~заблуждение: вектор\hbox{-}то один и~тот~же, просто при~разложении по~двум разным базисам у~него два набора компонент.

% ~ ~ ~ ~ ~
% converts spherical coordinates to cartesian
\newcommand{\tdsphericaltocartesian}[6]{%
\def\thecostheta{cos(#2)}%
\def\thesintheta{sin(#2)}%
\def\thecosphi{cos(#3)}%
\def\thesinphi{sin(#3)}%
\pgfmathsetmacro{#4}{ #1 * \thesintheta * \thecosphi }%
\pgfmathsetmacro{#5}{ #1 * \thesintheta * \thesinphi }%
\pgfmathsetmacro{#6}{ #1 * \thecostheta }%
}

% takes two points as cartesian {x}{y}{z} and calculates cross product of their location vectors
% placing the result into last three arguments
\newcommand{\tdcrossproductcartesian}[9]{%
\def\crossz{ #1 * #5 - #2 * #4 }%
\def\crossx{ #2 * #6 - #3 * #5 }%
\def\crossy{ #3 * #4 - #1 * #6 }%
\pgfmathsetmacro{#7}{\crossx}%
\pgfmathsetmacro{#8}{\crossy}%
\pgfmathsetmacro{#9}{\crossz}%
}

% takes two points as spherical {length}{anglefromz}{anglefromx} and calculates cross product of their location vectors
% placing the result as cartesian {x}{y}{z} into last three arguments
\newcommand{\tdcrossproductspherical}[9]{%
%
\tdplotsinandcos{\firstsintheta}{\firstcostheta}{#2}%
\tdplotsinandcos{\firstsinphi}{\firstcosphi}{#3}%
\def\firstx{ #1 * \firstsintheta * \firstcosphi }%
\def\firsty{ #1 * \firstsintheta * \firstsinphi }%
\def\firstz{ #1 * \firstcostheta }%
%
\tdplotsinandcos{\secondsintheta}{\secondcostheta}{#5}%
\tdplotsinandcos{\secondsinphi}{\secondcosphi}{#6}%
\def\secondx{ #4 * \secondsintheta * \secondcosphi }%
\def\secondy{ #4 * \secondsintheta * \secondsinphi }%
\def\secondz{ #4 * \secondcostheta }%
%
\def\crossz{ \firstx * \secondy - \firsty * \secondx }%
\def\crossx{ \firsty * \secondz - \firstz * \secondy }%
\def\crossy{ \firstz * \secondx - \firstx * \secondz }%
\pgfmathsetmacro{#7}{\crossx}%
\pgfmathsetmacro{#8}{\crossy}%
\pgfmathsetmacro{#9}{\crossz}%
}

% calculates dot product of location vectors of two 3D points specified by cartesian coordinates
\newcommand{\tddotproductcartesian}[7]{%
\edef\tddotproductcartesianxint{ \xinttheexpr round( #1 * #4 + #2 * #5 + #3 * #6 , 10 ) \relax }%
\pgfmathsetmacro{#7}{\tddotproductcartesianxint}%
}

% calculates dot product of location vectors of two 3D points specified by spherical coordinates
\newcommand{\tddotproductspherical}[7]{%
%
\tdplotsinandcos{\firstsintheta}{\firstcostheta}{#2}%
\tdplotsinandcos{\firstsinphi}{\firstcosphi}{#3}%
\def\firstx{ ( #1 * \firstsintheta * \firstcosphi ) }%
\def\firsty{ ( #1 * \firstsintheta * \firstsinphi ) }%
\def\firstz{ ( #1 * \firstcostheta ) }%
%
\tdplotsinandcos{\secondsintheta}{\secondcostheta}{#5}%
\tdplotsinandcos{\secondsinphi}{\secondcosphi}{#6}%
\def\secondx{ ( #4 * \secondsintheta * \secondcosphi ) }%
\def\secondy{ ( #4 * \secondsintheta * \secondsinphi ) }%
\def\secondz{ ( #4 * \secondcostheta ) }%
%
\edef\tddotproductsphericalxint{ \xinttheexpr round( \firstx * \secondx + \firsty * \secondy + \firstz * \secondz , 10 ) \relax }%
\pgfmathsetmacro{#7}{\tddotproductsphericalxint}%
}

% takes three points as spherical {length}{anglefromz}{anglefromx}
% and calculates triple product r1 × r2 • r3 of their location vectors
% the result is placed into \LastThreeDTripleProduct
\newcommand{\tdtripleproductspherical}[9]{%
%
\tdsphericaltocartesian{#1}{#2}{#3}{\firstx}{\firsty}{\firstz}
\tdsphericaltocartesian{#4}{#5}{#6}{\secondx}{\secondy}{\secondz}
\tdsphericaltocartesian{#7}{#8}{#9}{\thirdx}{\thirdy}{\thirdz}
%
\def\crossz{ ( \firstx * \secondy - \firsty * \secondx ) }%
\def\crossx{ ( \firsty * \secondz - \firstz * \secondy ) }%
\def\crossy{ ( \firstz * \secondx - \firstx * \secondz ) }%
%
\edef\LastThreeDTripleProduct{ \xinttheexpr round( \crossx * \thirdx + \crossy * \thirdy + \crossz * \thirdz , 10 ) \relax }%
}

% orientation of camera
\def\cameraTheta{36} \def\cameraPhi{98}
%% \def\cameraTheta{89.99} \def\cameraPhi{120}
	% 90 gives “You asked me to calculate `1/0.0', but I cannot divide any number by zero.”
\tdplotsetmaincoords{\cameraTheta}{\cameraPhi}

% vectors of basis
\pgfmathsetmacro{\firstlength}{0.69}
	\pgfmathsetmacro{\firstanglefromz}{71}
	\pgfmathsetmacro{\firstanglefromx}{-16}
\pgfmathsetmacro{\secondlength}{0.88}
	\pgfmathsetmacro{\secondanglefromz}{86}
	\pgfmathsetmacro{\secondanglefromx}{77}
\pgfmathsetmacro{\thirdlength}{0.96}
	\pgfmathsetmacro{\thirdanglefromz}{-19}
	\pgfmathsetmacro{\thirdanglefromx}{45}

\tdsphericaltocartesian%
	{\firstlength}{\firstanglefromz}{\firstanglefromx}%
	{\firstcartesianx}{\firstcartesiany}{\firstcartesianz}
\tdsphericaltocartesian%
	{\secondlength}{\secondanglefromz}{\secondanglefromx}%
	{\secondcartesianx}{\secondcartesiany}{\secondcartesianz}
\tdsphericaltocartesian%
	{\thirdlength}{\thirdanglefromz}{\thirdanglefromx}%
	{\thirdcartesianx}{\thirdcartesiany}{\thirdcartesianz}

% some but very important vector
\pgfmathsetmacro{\lengthofvector}{3.33}
	\pgfmathsetmacro{\vectoranglefromz}{33}
	\pgfmathsetmacro{\vectoranglefromx}{44}

\tdsphericaltocartesian%
	{\lengthofvector}{\vectoranglefromz}{\vectoranglefromx}%
	{\vectorcartesianx}{\vectorcartesiany}{\vectorcartesianz}

%%\begin{comment} %%
\begin{minipage}{\textwidth}
\hfill\[\scalebox{0.9}[0.9]{$\begin{array}{l@{\hspace{0.2\textwidth}}l@{\hspace{1.2em}}l@{\hspace{0.8em}}l}
\theta = \pgfmathprintnumber{\cameraTheta}\degree \hspace{0.8em}
\phi = \pgfmathprintnumber{\cameraPhi}\degree
& \scalebox{1.05}{$\bm{v}^{\varrho} = \pgfmathprintnumber{\lengthofvector}$} &
	\scalebox{1.05}{$\bm{v}^{\theta} = \pgfmathprintnumber{\vectoranglefromz}\degree$} &
	\scalebox{1.05}{$\bm{v}^{\phi} = \pgfmathprintnumber{\vectoranglefromx}\degree$} \\[0.25em]
%
& {\bm{a}_{1}^{\varrho} = \pgfmathprintnumber{\firstlength}} &
	{\bm{a}_{1}^{\theta} = \pgfmathprintnumber{\firstanglefromz}\degree} &
		{\bm{a}_{1}^{\phi} = \pgfmathprintnumber{\firstanglefromx}\degree} \\[0.1em]
& {\bm{a}_{2}^{\varrho} = \pgfmathprintnumber{\secondlength}} &
	{\bm{a}_{2}^{\theta} = \pgfmathprintnumber{\secondanglefromz}\degree} &
		{\bm{a}_{2}^{\phi} = \pgfmathprintnumber{\secondanglefromx}\degree} \\[0.1em]
& {\bm{a}_{3}^{\varrho} = \pgfmathprintnumber{\thirdlength}} &
	{\bm{a}_{3}^{\theta} = \pgfmathprintnumber{\thirdanglefromz}\degree} &
		{\bm{a}_{3}^{\phi} = \pgfmathprintnumber{\thirdanglefromx}\degree}
\end{array}$}\]
\end{minipage}
%%\end{comment} %%

\begin{figure}[!htbp]
\begin{center}

\vspace{0.1em}
\begin{tikzpicture}[scale=3.2, tdplot_main_coords] % tdplot_main_coords style to use 3dplot

	\coordinate (O) at (0,0,0);

	% define axes
	\tdplotsetcoord{A1}{\firstlength}{\firstanglefromz}{\firstanglefromx}
	\tdplotsetcoord{A2}{\secondlength}{\secondanglefromz}{\secondanglefromx}
	\tdplotsetcoord{A3}{\thirdlength}{\thirdanglefromz}{\thirdanglefromx}

	% define vector
	\tdplotsetcoord{V}{\lengthofvector}{\vectoranglefromz}{\vectoranglefromx} % {length}{angle from z}{angle from x}

	% square root of Gram matrix’ determinant is a1 × a2 • a3
	\tdtripleproductspherical%
		{\firstlength}{\firstanglefromz}{\firstanglefromx}%
		{\secondlength}{\secondanglefromz}{\secondanglefromx}%
		{\thirdlength}{\thirdanglefromz}{\thirdanglefromx}
	\edef\sqrtGramian{\xinttheexpr round( \LastThreeDTripleProduct, 10 )\relax}
	\edef\inverseOfSqrtGramian{\xinttheexpr round( 1 / \sqrtGramian, 10 )\relax}

	\node[fill=white!50, inner sep=0pt, outer sep=2pt] at (1.2,0,-1.45)
		{$\scalebox{0.9}{$\begin{array}{r}\bm{a}_1 \hspace{-0.4ex} \times \hspace{-0.3ex} \bm{a}_2 \dotp \hspace{0.2ex} \bm{a}_3 \hspace{-0.2ex} = \hspace{-0.2ex} \sqrt{\hspace{-0.36ex}\mathstrut{\textsl{g}}} \hspace{0.1ex} = \hspace{-0.2ex} \pgfmathprintnumber[fixed, precision=5]{\sqrtGramian} \\[0.25em]
		\displaystyle \nicefrac{\scalebox{0.95}{$1$}}{\hspace{-0.25ex}\sqrt{\hspace{-0.2ex}\scalebox{0.96}{$\mathstrut{\textsl{g}}$}}} \hspace{0.1ex} = \hspace{-0.2ex} \pgfmathprintnumber[fixed, precision=5]{\inverseOfSqrtGramian}\end{array}$}$};

	% calculate vectors of cobasis
	\tdcrossproductspherical%
		{\firstlength}{\firstanglefromz}{\firstanglefromx}%
		{\secondlength}{\secondanglefromz}{\secondanglefromx}%
		{\firstsecondcrossx}{\firstsecondcrossy}{\firstsecondcrossz}
	\coordinate (cross12) at (\firstsecondcrossx, \firstsecondcrossy, \firstsecondcrossz);
	\draw [line width=1.25pt, orange, -{Latex[round, length=3.6mm, width=2.4mm]}]
		(O) -- (cross12)
		node[pos=0.64, above right, inner sep=0pt, outer sep=6pt]
		{$\scalebox{0.8}{$\bm{a}_1 \hspace{-0.4ex} \times \hspace{-0.3ex} \bm{a}_2$}$};

	\tdcrossproductspherical%
		{\thirdlength}{\thirdanglefromz}{\thirdanglefromx}%
		{\firstlength}{\firstanglefromz}{\firstanglefromx}%
		{\thirdfirstcrossx}{\thirdfirstcrossy}{\thirdfirstcrossz}
	\coordinate (cross31) at (\thirdfirstcrossx, \thirdfirstcrossy, \thirdfirstcrossz);
	\draw [line width=1.25pt, orange, -{Latex[round, length=3.6mm, width=2.4mm]}]
		(O) -- (cross31)
		node[pos=0.86, above, inner sep=0pt, outer sep=5pt]
		{$\scalebox{0.8}{$\bm{a}_3 \hspace{-0.4ex} \times \hspace{-0.3ex} \bm{a}_1$}$};

	\tdcrossproductspherical%
		{\secondlength}{\secondanglefromz}{\secondanglefromx}%
		{\thirdlength}{\thirdanglefromz}{\thirdanglefromx}%
		{\secondthirdcrossx}{\secondthirdcrossy}{\secondthirdcrossz}
	\coordinate (cross23) at (\secondthirdcrossx, \secondthirdcrossy, \secondthirdcrossz);
	\draw [line width=1.25pt, orange, -{Latex[round, length=3.6mm, width=2.4mm]}]
		(O) -- (cross23)
		node[pos=0.88, below right, inner sep=0pt, outer sep=2.5pt]
		{$\scalebox{0.8}{$\bm{a}_2 \hspace{-0.4ex} \times \hspace{-0.3ex} \bm{a}_3$}$};

	\coordinate (coA3) at ($ \inverseOfSqrtGramian*(cross12) $);
	\coordinate (coA2) at ($ \inverseOfSqrtGramian*(cross31) $);
	\coordinate (coA1) at ($ \inverseOfSqrtGramian*(cross23) $);

	% get vector’s projection on a1 & a2 plane (third co-vector a^3 is normal to that plane)
	% it’s as deep down parallel to a3 as v^3 = v • a^3 in units of a3
	\tddotproductcartesian%
		{\vectorcartesianx}{\vectorcartesiany}{\vectorcartesianz}%
		{\inverseOfSqrtGramian*\firstsecondcrossx}%
			{\inverseOfSqrtGramian*\firstsecondcrossy}%
				{\inverseOfSqrtGramian*\firstsecondcrossz}%
		{\vectorthirdcoco}
	% get third co-component and translate it to vector’s head
	\coordinate (Vcomponent3) at ($ \vectorthirdcoco*(A3) $);
	\coordinate (VcomponentXY) at ($ (V) - (Vcomponent3) $);

	% decompose vector via initial basis
	\coordinate (ParallelToSecond) at ($ (VcomponentXY) - (A2) $);
	\coordinate (ParallelToFirst) at ($ (VcomponentXY) - (A1) $);
	\coordinate (Vcomponent1) at (intersection of VcomponentXY--ParallelToSecond and O--A1);
	\coordinate (Vcomponent2) at (intersection of VcomponentXY--ParallelToFirst and O--A2);

	\draw [line width=0.4pt, dotted, color=blue] (O) -- (VcomponentXY); % projection on first & second vectors’ plane

	\draw [line width=0.4pt, dotted, color=blue] (V) -- (VcomponentXY);
	\draw [line width=0.4pt, dotted, color=blue] (VcomponentXY) -- (Vcomponent1);
	\draw [line width=0.4pt, dotted, color=blue] (VcomponentXY) -- (Vcomponent2);

	% check a^1 × a^2 direction to be the same as a3
	\tdcrossproductcartesian%
		{\inverseOfSqrtGramian*\secondthirdcrossx}%
			{\inverseOfSqrtGramian*\secondthirdcrossy}%
				{\inverseOfSqrtGramian*\secondthirdcrossz}%
		{\inverseOfSqrtGramian*\thirdfirstcrossx}%
			{\inverseOfSqrtGramian*\thirdfirstcrossy}%
				{\inverseOfSqrtGramian*\thirdfirstcrossz}%
		{\CofirstCosecondOrthoX}{\CofirstCosecondOrthoY}{\CofirstCosecondOrthoZ}
	\coordinate (co1co2ortho) at (\CofirstCosecondOrthoX, \CofirstCosecondOrthoY, \CofirstCosecondOrthoZ);
	\draw [line width=1.25pt, blue!50, -{Latex[round, length=3.6mm, width=2.4mm]}]
		(O) -- ($ \sqrtGramian*(co1co2ortho) $);

	% length of a^3
	\tddotproductcartesian%
		{\inverseOfSqrtGramian*\thirdfirstcrossx}%
			{\inverseOfSqrtGramian*\thirdfirstcrossy}%
				{\inverseOfSqrtGramian*\thirdfirstcrossz}%
		{\inverseOfSqrtGramian*\thirdfirstcrossx}%
			{\inverseOfSqrtGramian*\thirdfirstcrossy}%
				{\inverseOfSqrtGramian*\thirdfirstcrossz}%
		{\squaredlengthofthirdcovector}
	%%\node[fill=white!50, inner sep=0pt, outer sep=4pt] at (0,0,-2.25)
		%%{$\scalebox{0.9}{$ | \hspace{0.1ex} \bm{a}^{\hspace{-0.1ex}3} \hspace{0.06ex} | \hspace{0.1ex} =
			%%\sqrt{\pgfmathprintnumber[fixed, precision=5]{\squaredlengthofthirdcovector}} $}$};

	% get vector’s projection on a^1 & a^2 plane (third basis vector a3 is normal to that plane)
	% it’s as deep down parallel to a^3 as v3 = v • a3 in units of a^3
	\tddotproductspherical%
		{\lengthofvector}{\vectoranglefromz}{\vectoranglefromx}%
		{\thirdlength}{\thirdanglefromz}{\thirdanglefromx}%
		{\vectorthirdcomponent}
	% get third co-component and translate it to vector’s head
	\coordinate (Vcoco3) at ($ \vectorthirdcomponent*(coA3) $);
	\coordinate (VcocoXY) at ($ (V) - (Vcoco3) $);

	% decompose vector via cobasis
	%%\coordinate (ParallelToCothird) at ($ (V) - (coA3) $);
	\coordinate (ParallelToCosecond) at ($ (VcocoXY) - (coA2) $);
	\coordinate (ParallelToCofirst) at ($ (VcocoXY) - (coA1) $);
	\coordinate (Vcoco1) at (intersection of VcocoXY--ParallelToCosecond and O--coA1);
	\coordinate (Vcoco2) at (intersection of VcocoXY--ParallelToCofirst and O--coA2);

	\draw [line width=0.4pt, red] (O) -- (Vcoco2);

	\draw [line width=0.4pt, dotted, color=red] (O) -- (VcocoXY);

	\draw [line width=0.4pt, dotted, color=red] (V) -- (VcocoXY);
	\draw [line width=0.4pt, dotted, color=red] (VcocoXY) -- (Vcoco1);
	\draw [line width=0.4pt, dotted, color=red] (VcocoXY) -- (Vcoco2);

	% draw parallelepiped of decomposition
	\coordinate (onPlane23) at ($ (Vcomponent2) + (V) - (VcomponentXY) $);
	\draw [line width=0.4pt, dotted, color=blue] (Vcomponent2) -- (onPlane23);
	\draw [line width=0.4pt, dotted, color=blue] (V) -- (onPlane23);

	\coordinate (onPlane13) at ($ (Vcomponent1) + (V) - (VcomponentXY) $);
	\draw [line width=0.4pt, dotted, color=blue] (Vcomponent1) -- (onPlane13);
	\draw [line width=0.4pt, dotted, color=blue] (V) -- (onPlane13);

	\coordinate (onAxis3) at ($ (V) - (VcomponentXY) $);
	\draw [line width=0.4pt, dotted, color=blue] (O) -- (onAxis3);
	\draw [line width=0.4pt, dotted, color=blue] (onPlane13) -- (onAxis3);
	\draw [line width=0.4pt, dotted, color=blue] (onPlane23) -- (onAxis3);

	\draw [line width=0.4pt, dotted, color=blue] (O) -- (onPlane13);
	\draw [line width=0.4pt, dotted, color=blue] (O) -- (onPlane23);

	% draw co-parallelepiped of co-decomposition
	\coordinate (onCoplane23) at ($ (Vcoco2) + (V) - (VcocoXY) $);
	\draw [line width=0.4pt, dotted, color=red] (Vcoco2) -- (onCoplane23);
	\draw [line width=0.4pt, dotted, color=red] (V) -- (onCoplane23);

	\coordinate (onCoplane13) at ($ (Vcoco1) + (V) - (VcocoXY) $);
	\draw [line width=0.4pt, dotted, color=red] (Vcoco1) -- (onCoplane13);
	\draw [line width=0.4pt, dotted, color=red] (V) -- (onCoplane13);

	\coordinate (onCoAxis3) at ($ (V) - (VcocoXY) $);
	\draw [line width=0.4pt, dotted, color=red] (O) -- (onCoAxis3);
	\draw [line width=0.4pt, dotted, color=red] (onCoplane13) -- (onCoAxis3);
	\draw [line width=0.4pt, dotted, color=red] (onCoplane23) -- (onCoAxis3);

	\draw [line width=0.4pt, dotted, color=red] (O) -- (onCoplane13);
	\draw [line width=0.4pt, dotted, color=red] (O) -- (onCoplane23);

	% draw vectors of cobasis
	\draw [line width=0.4pt, red] (O) -- ($ 1.01*(coA3) $);
	\draw [line width=1.25pt, red, -{Latex[round, length=3.6mm, width=2.4mm]}]
		(O) -- (coA3)
		node[pos=0.8, above right, shape=circle, fill=white, inner sep=-1pt, outer sep=11pt]
		{${\bm{a}}^{\hspace{-0.1ex}3}$};

	\draw [line width=0.4pt, red] (O) -- ($ 1.01*(coA2) $);
	\draw [line width=1.25pt, red, -{Latex[round, length=3.6mm, width=3mm]}]
		(O) -- (coA2)
		node[pos=0.88, above, shape=circle, fill=white, inner sep=-1pt, outer sep=4pt]
		{${\bm{a}}^{\hspace{-0.1ex}2}$};

	\draw [line width=0.4pt, red] (O) -- ($ 1.01*(coA1) $);
	\draw [line width=1.25pt, red, -{Latex[round, length=3.6mm, width=2.4mm]}]
		(O) -- (coA1)
		node[pos=0.92, below right, shape=circle, fill=white, inner sep=-1pt, outer sep=4pt]
		{${\bm{a}}^{\hspace{-0.16ex}1}$};

	% draw vectors of basis
	\draw [line width=0.4pt, blue] (O) -- ($ 1.01*(A1) $);
	\draw [line width=1.25pt, blue, -{Latex[round, length=3.6mm, width=2.4mm]}]
		(O) -- (A1)
		node[pos=0.84, above left, shape=circle, fill=white, inner sep=-1pt, outer sep=6pt]
		{${\bm{a}}_{\hspace{-0.08ex}1}$};

	\draw [line width=0.4pt, blue] (O) -- ($ 1.01*(A2) $);
	\draw [line width=1.25pt, blue, -{Latex[round, length=3.6mm, width=2.4mm]}]
		(O) -- (A2)
		node[pos=0.88, below, shape=circle, fill=white, inner sep=-1pt, outer sep=6pt]
		{${\bm{a}}_2$};
	\draw [line width=0.4pt, blue] (O) -- (Vcomponent2);

	\draw [line width=0.4pt, blue] (O) -- ($ 1.01*(A3) $);
	\draw [line width=1.25pt, blue, -{Latex[round, length=3.6mm, width=2.4mm]}]
		(O) -- (A3)
		node[pos=0.71, above left, shape=circle, fill=white, inner sep=-1pt, outer sep=16pt]
		{${\bm{a}}_3$};

	% draw components of vector
	\draw [color=blue!50!black, line width=1.6pt, line cap=round, dash pattern=on 0pt off 1.6\pgflinewidth,
		-{Stealth[round, length=4mm, width=2.4mm]}]
		(O) -- (Vcomponent1)
		node[pos=0.67, above left, fill=white, shape=circle, inner sep=0pt, outer sep=4pt]
	{${v^{\hspace{-0.08ex}1} \hspace{-0.1ex} \bm{a}_{\hspace{-0.08ex}1}}$};

	\draw [color=blue!50!black, line width=1.6pt, line cap=round, dash pattern=on 0pt off 1.6\pgflinewidth,
		-{Stealth[round, length=4mm, width=2.4mm]}]
		(Vcomponent1) -- (VcomponentXY)
		node[pos=0.48, above, shape=circle, fill=white, inner sep=-2pt, outer sep=1pt]
	{${v^2 \hspace{-0.1ex} \bm{a}_2}$};

	\draw [color=blue!50!black, line width=1.6pt, line cap=round, dash pattern=on 0pt off 1.6\pgflinewidth,
		-{Stealth[round, length=4mm, width=2.4mm]}]
		(VcomponentXY) -- (V)
		node[pos=0.77, above right, shape=circle, fill=white, inner sep=-1pt, outer sep=7pt]
	{${v^3 \hspace{-0.1ex} \bm{a}_3}$};

	% draw co-components of vector
	\draw [color=red!50!black, line width=1.6pt, line cap=round, dash pattern=on 0pt off 1.6\pgflinewidth,
		-{Stealth[round, length=4mm, width=2.4mm]}]
		(O) -- (Vcoco1)
		node[pos=1.02, above left, fill=white, shape=circle, inner sep=-1pt, outer sep=5pt]
	{${v_{\raisemath{-0.2ex}{1}} \bm{a}^{\hspace{-0.16ex}1}}$};

	\draw [color=red!50!black, line width=1.6pt, line cap=round, dash pattern=on 0pt off 1.6\pgflinewidth,
		-{Stealth[round, length=4mm, width=2.4mm]}]
		(Vcoco1) -- (VcocoXY)
		node[pos=0.5, below right, shape=circle, fill=white, inner sep=-2pt, outer sep=7pt]
	{${v_{\raisemath{-0.2ex}{2}}  \hspace{0.1ex} \bm{a}^{\hspace{-0.1ex}2}}$};

	\draw [color=red!50!black, line width=1.6pt, line cap=round, dash pattern=on 0pt off 1.6\pgflinewidth,
		-{Stealth[round, length=4mm, width=2.4mm]}]
		(VcocoXY) -- (V)
		node[pos=0.37, above left, shape=circle, fill=white, inner sep=-1pt, outer sep=9pt]
	{${v_{\raisemath{-0.2ex}{3}} \hspace{0.1ex} \bm{a}^{\hspace{-0.1ex}3}}$};

	% draw vector
	\draw [line width=1.6pt, black, -{Stealth[round, length=5mm, width=2.8mm]}]
		(O) -- (V)
		node[pos=0.69, above, shape=circle, fill=white, inner sep=0pt, outer sep=3.33pt]
			{\scalebox{1.2}[1.2]{${\bm{v}}$}};

	% calculate a_i • a^j
	\tddotproductcartesian%
		{\firstcartesianx}{\firstcartesiany}{\firstcartesianz}%
		{\inverseOfSqrtGramian*\secondthirdcrossx}%
			{\inverseOfSqrtGramian*\secondthirdcrossy}%
				{\inverseOfSqrtGramian*\secondthirdcrossz}%
		{\FirstDotCofirst}
	\tddotproductcartesian%
		{\secondcartesianx}{\secondcartesiany}{\secondcartesianz}%
		{\inverseOfSqrtGramian*\thirdfirstcrossx}%
			{\inverseOfSqrtGramian*\thirdfirstcrossy}%
				{\inverseOfSqrtGramian*\thirdfirstcrossz}%
		{\SecondDotCosecond}
	\tddotproductcartesian%
		{\thirdcartesianx}{\thirdcartesiany}{\thirdcartesianz}%
		{\inverseOfSqrtGramian*\firstsecondcrossx}%
			{\inverseOfSqrtGramian*\firstsecondcrossy}%
				{\inverseOfSqrtGramian*\firstsecondcrossz}%
		{\ThirdDotCothird}
	%
	\tddotproductcartesian%
		{\secondcartesianx}{\secondcartesiany}{\secondcartesianz}%
		{\inverseOfSqrtGramian*\secondthirdcrossx}%
			{\inverseOfSqrtGramian*\secondthirdcrossy}%
				{\inverseOfSqrtGramian*\secondthirdcrossz}%
		{\SecondDotCofirst}
	\tddotproductcartesian%
		{\firstcartesianx}{\firstcartesiany}{\firstcartesianz}%
		{\inverseOfSqrtGramian*\thirdfirstcrossx}%
			{\inverseOfSqrtGramian*\thirdfirstcrossy}%
				{\inverseOfSqrtGramian*\thirdfirstcrossz}%
		{\FirstDotCosecond}
	\tddotproductcartesian%
		{\thirdcartesianx}{\thirdcartesiany}{\thirdcartesianz}%
		{\inverseOfSqrtGramian*\secondthirdcrossx}%
			{\inverseOfSqrtGramian*\secondthirdcrossy}%
				{\inverseOfSqrtGramian*\secondthirdcrossz}%
		{\ThirdDotCofirst}
	\tddotproductcartesian%
		{\firstcartesianx}{\firstcartesiany}{\firstcartesianz}%
		{\inverseOfSqrtGramian*\firstsecondcrossx}%
			{\inverseOfSqrtGramian*\firstsecondcrossy}%
				{\inverseOfSqrtGramian*\firstsecondcrossz}%
		{\FirstDotCothird}
	\tddotproductcartesian%
		{\secondcartesianx}{\secondcartesiany}{\secondcartesianz}%
		{\inverseOfSqrtGramian*\firstsecondcrossx}%
			{\inverseOfSqrtGramian*\firstsecondcrossy}%
				{\inverseOfSqrtGramian*\firstsecondcrossz}%
		{\SecondDotCothird}
	\tddotproductcartesian%
		{\thirdcartesianx}{\thirdcartesiany}{\thirdcartesianz}%
		{\inverseOfSqrtGramian*\thirdfirstcrossx}%
			{\inverseOfSqrtGramian*\thirdfirstcrossy}%
				{\inverseOfSqrtGramian*\thirdfirstcrossz}%
		{\ThirdDotCosecond}

	% show a_i • a^j as matrix
	\node[fill=white!50, inner sep=0pt, outer sep=2pt] at (0,0.45,-3.8)
		{$\scalebox{0.9}[0.9]{$
			\bm{a}_i \dotp \hspace{0.1ex} \bm{a}^{\hspace{0.1ex}j} \hspace{-0.1ex} =
			\hspace{-0.2ex}\scalebox{0.9}[0.9]{$\left[ \begin{array}{ccc}
				\bm{a}_1 \hspace{-0.1ex} \dotp \bm{a}^{\hspace{-0.1ex}1} &
					\bm{a}_1 \hspace{-0.1ex} \dotp \bm{a}^{\hspace{-0.06ex}2} &
						\bm{a}_1 \hspace{-0.1ex} \dotp \bm{a}^{\hspace{-0.06ex}3} \\
				\bm{a}_2 \hspace{-0.1ex} \dotp \bm{a}^{\hspace{-0.1ex}1} &
					\bm{a}_2 \hspace{-0.1ex} \dotp \bm{a}^{\hspace{-0.06ex}2} &
						\bm{a}_2 \hspace{-0.1ex} \dotp \bm{a}^{\hspace{-0.06ex}3} \\
				\bm{a}_3 \hspace{-0.1ex} \dotp \bm{a}^{\hspace{-0.1ex}1} &
					\bm{a}_3 \hspace{-0.1ex} \dotp \bm{a}^{\hspace{-0.06ex}2} &
						\bm{a}_3 \hspace{-0.1ex} \dotp \bm{a}^{\hspace{-0.06ex}3}
			\end{array} \right]$} \!=\!
			\scalebox{0.9}[0.9]{$\left[ \begin{array}{ccc}
				\pgfmathprintnumber[fixed, precision=3]{\FirstDotCofirst} &
					\pgfmathprintnumber[fixed, precision=3]{\FirstDotCosecond} &
						\pgfmathprintnumber[fixed, precision=3]{\FirstDotCothird} \\
				\pgfmathprintnumber[fixed, precision=3]{\SecondDotCofirst} &
					\pgfmathprintnumber[fixed, precision=3]{\SecondDotCosecond} &
						\pgfmathprintnumber[fixed, precision=3]{\SecondDotCothird} \\
				\pgfmathprintnumber[fixed, precision=3]{\ThirdDotCofirst} &
					\pgfmathprintnumber[fixed, precision=3]{\ThirdDotCosecond} &
						\pgfmathprintnumber[fixed, precision=3]{\ThirdDotCothird}
			\end{array} \right]$}
			\hspace{-0.1em} = %%\approx
			\hspace{0.1ex} \delta_i^{\hspace{0.1ex}j}
		$}$};

\end{tikzpicture}

\vspace{0.2em}\caption{\inquotes{Decomposition of vector}}\label{fig:DecompositionOfVector}

\end{center}
\end{figure}

% ~ ~ ~ ~ ~

От~векторов перейдём к~тензорам второй сложности. Имеем четыре комплекта диад:
${\bm{a}_i \hspace{0.1ex} \bm{a}_j}$,
\hbox{$\bm{a}^{\hspace{-0.05ex}i} \hspace{-0.1ex} \bm{a}^{\hspace{0.05ex}j}$\hspace{-0.25ex},}
\hbox{$\bm{a}_i \hspace{0.1ex} \bm{a}^{\hspace{0.05ex}j}$\hspace{-0.25ex},}
${\bm{a}^{\hspace{-0.05ex}i} \hspace{-0.1ex} \bm{a}_j}$.
Согласующиеся коэффициенты в~декомпозиции тензора называются его контра\-вариант\-ными, ко\-вариант\-ными и~смешан\-ными компонентами:
\vspace{0.1em}\begin{equation}\begin{array}{c}
{^2\!\bm{B}} \hspace{0.1ex} =
B^{\hspace{0.1ex}i\hspace{-0.1ex}j} \hspace{-0.1ex} \bm{a}_i \hspace{0.1ex} \bm{a}_j \hspace{-0.05ex} =
B_{i\hspace{-0.1ex}j} \hspace{0.1ex} \bm{a}^{\hspace{-0.05ex}i} \hspace{-0.1ex} \bm{a}^{\hspace{0.05ex}j} \hspace{-0.15ex} =
B_{\hspace{-0.2ex} \cdot j}^{\hspace{0.1ex}i} \hspace{0.1ex} \bm{a}_i \hspace{0.1ex} \bm{a}^{\hspace{0.05ex}j} \hspace{-0.15ex} =
B_{\hspace{-0.1ex}i}^{\hspace{-0.05ex} \cdot j} \hspace{-0.2ex} \bm{a}^{\hspace{-0.05ex}i} \hspace{-0.1ex} \bm{a}_j \hspace{0.1ex}, \\[0.4em]
%
B^{\hspace{0.1ex}i\hspace{-0.1ex}j} \hspace{-0.25ex} = \bm{a}^{\hspace{-0.05ex}i} \dotp {^2\!\bm{B}} \dotp \hspace{0.1ex} \bm{a}^{\hspace{0.05ex}j} \hspace{-0.2ex}, \:\,
B_{i\hspace{-0.1ex}j} = \bm{a}_i \dotp {^2\!\bm{B}} \dotp \hspace{0.1ex} \bm{a}_j \hspace{0.1ex}, \\[0.25em]
%
B_{\hspace{-0.2ex} \cdot j}^{\hspace{0.1ex}i} = \bm{a}^{\hspace{-0.05ex}i} \dotp {^2\!\bm{B}} \dotp \hspace{0.1ex} \bm{a}_j \hspace{0.1ex}, \:\,
B_{\hspace{-0.1ex}i}^{\hspace{-0.05ex} \cdot j} \hspace{-0.2ex} = \bm{a}_i \dotp {^2\!\bm{B}} \dotp \hspace{0.1ex} \bm{a}^{\hspace{0.1ex}j} \hspace{-0.1ex}.
\end{array}\end{equation}

\vspace{-0.1em}\noindent Для~двух видов смешанных компонент точка в~индексе это просто свободное место: у~${B_{\hspace{-0.2ex} \cdot j}^{\hspace{0.1ex}i}}$ верхний индекс~\inquotesx{$i\hspace{0.25ex}$}[---] первый, а~ниж\-ний~\inquotesx{$\hspace{-0.1ex}j\hspace{0.25ex}$}[---] второй.

Компоненты единичного~(\inquotes{метрического}) тензора %%~$\bm{E}$
\vspace{0.1em}\begin{equation}\begin{array}{c}
\bm{E} = \bm{a}^{k} \hspace{-0.1ex} \bm{a}_{k} \hspace{-0.1ex} = \bm{a}_{k} \hspace{0.1ex} \bm{a}^{k} \hspace{-0.2ex} = \textsl{g}_{j\hspace{-0.1ex}k} \hspace{0.1ex} \bm{a}^{\hspace{0.1ex}j} \hspace{-0.1ex} \bm{a}^{k} \hspace{-0.2ex} = \textsl{g}^{\hspace{0.25ex}j\hspace{-0.1ex}k} \hspace{-0.1ex} \bm{a}_j \bm{a}_k \hspace{-0.32ex}: \\[0.2em]
%
\bm{a}_i \dotp \bm{E} \dotp \hspace{0.1ex} \bm{a}^{\hspace{0.1ex}j} \hspace{-0.2ex} = \bm{a}_i \hspace{0.1ex} \dotp \hspace{0.1ex} \bm{a}^{\hspace{0.1ex}j} \hspace{-0.2ex} = \delta_i^{\hspace{0.1ex}j} , \hspace{0.32em}
\bm{a}^{\hspace{-0.05ex}i} \dotp \bm{E} \dotp \hspace{0.1ex} \bm{a}_j = \bm{a}^{\hspace{-0.05ex}i} \hspace{-0.2ex} \dotp \hspace{0.1ex} \bm{a}_j \hspace{-0.2ex} = \delta_{\hspace{-0.1ex}j}^{\hspace{0.1ex}i} \hspace{0.2ex},
\\[0.2em]
%
\bm{a}_i \dotp \bm{E} \dotp \hspace{0.1ex} \bm{a}_j = \bm{a}_i \dotp \hspace{0.1ex} \bm{a}_j \equiv \hspace{0.16ex} \textsl{g}_{i\hspace{-0.1ex}j} \hspace{0.1ex} , \hspace{0.32em}
\bm{a}^{\hspace{-0.05ex}i} \dotp \bm{E} \dotp \hspace{0.1ex} \bm{a}^{\hspace{0.1ex}j} \hspace{-0.15ex} = \bm{a}^{\hspace{-0.05ex}i} \dotp \hspace{0.1ex} \bm{a}^{\hspace{0.1ex}j} \hspace{-0.15ex} \hspace{0.1ex} \equiv \hspace{0.16ex} \textsl{g}^{\hspace{0.2ex}i\hspace{-0.1ex}j} \hspace{0.1ex} ; \\[0.2em]
%
\scalebox{0.96}[0.97]{$\bm{E} \dotp \hspace{-0.1ex} \bm{E} = \textsl{g}_{i\hspace{-0.1ex}j} \hspace{0.1ex} \bm{a}^{\hspace{-0.05ex}i} \hspace{-0.1ex} \bm{a}^{\hspace{0.05ex}j} \hspace{-0.1ex} \dotp \hspace{0.1ex} \textsl{g}^{\hspace{0.2ex}nk} \bm{a}_n \bm{a}_k \hspace{-0.1ex} = \textsl{g}_{i\hspace{-0.1ex}j} \hspace{0.1ex} \textsl{g}^{\hspace{0.2ex}j\hspace{-0.1ex}k} \bm{a}^{\hspace{-0.05ex}i} \bm{a}_k \hspace{-0.1ex} = \bm{E}$}
\:\Rightarrow\: \textsl{g}_{i\hspace{-0.1ex}j} \hspace{0.1ex} \textsl{g}^{\hspace{.2ex}j\hspace{-0.1ex}k} \hspace{-0.25ex} = \delta_i^{\hspace{0.05ex}k} \hspace{-0.1ex}.
\end{array}\end{equation}

\vspace{0.1em}\noindent Вдобавок к~\eqref{fundamentalpropertyofcobasis} и~\eqref{basisvectorstocobasisvectors} открылся ещё~один способ найти векторы кобазиса\:--- через матрицу~\hbox{$\textsl{g}^{\hspace{.2ex}i\hspace{-0.1ex}j}$\hspace{-0.3ex}}, обратную матрице Грама~${\textsl{g}_{i\hspace{-0.1ex}j}}$. И~наоборот:

\nopagebreak\vspace{-0.12em}
\begin{equation}\begin{array}{c}
\bm{a}^{\hspace{-0.05ex}i} \hspace{-0.1ex}
= \bm{E} \dotp \bm{a}^{\hspace{-0.05ex}i} \hspace{-0.15ex}
= \textsl{g}^{\hspace{.2ex}j\hspace{-0.1ex}k} \bm{a}_j \bm{a}_k \hspace{-0.1ex} \dotp \bm{a}^{\hspace{-0.05ex}i} \hspace{-0.15ex}
= \textsl{g}^{\hspace{.2ex}j\hspace{-0.1ex}k} \bm{a}_j \hspace{.16ex} \delta_{k}^{\hspace{.1ex}i}
= \textsl{g}^{\hspace{.2ex}j\hspace{-0.06ex}i} \bm{a}_j \hspace{.1ex} ,
\\[.25em]
%
\bm{a}_i
= \bm{E} \dotp \bm{a}_i \hspace{-0.1ex}
= \textsl{g}_{j\hspace{-0.1ex}k} \hspace{.1ex} \bm{a}^{\hspace{.1ex}j} \hspace{-0.1ex} \bm{a}^{k} \hspace{-0.2ex} \dotp \bm{a}_i \hspace{-0.1ex}
= \textsl{g}_{j\hspace{-0.1ex}k} \hspace{.1ex} \bm{a}^{\hspace{.1ex}j} \hspace{.1ex} \delta_{i}^{k}
= \textsl{g}_{j\hspace{-0.06ex}i} \hspace{.16ex} \bm{a}^{\hspace{.1ex}j} \hspace{-0.2ex} .
\end{array}\end{equation}

\begin{tcolorbox}
\small\setlength{\abovedisplayskip}{2pt}\setlength{\belowdisplayskip}{2pt}

\emph{Example.} Using reversed Gram matrix, get cobasis for basis~$\bm{a}_i$ when
\[ \begin{array}{l}
\bm{a}_1 = \bm{e}_1 \hspace{-0.2ex} + \bm{e}_2 \hspace{.1ex},\\
\bm{a}_2 = \bm{e}_1 \hspace{-0.2ex} + \bm{e}_3 \hspace{.1ex},\\
\bm{a}_3 = \bm{e}_2 \hspace{-0.2ex} + \bm{e}_3 \hspace{.1ex}.
\end{array} \]

\[\begin{array}{c}
\textsl{g}_{i\hspace{-0.1ex}j} \hspace{-0.32ex} = \bm{a}_i \dotp \bm{a}_j \hspace{-0.12ex} = \hspace{-0.16ex}
\scalebox{0.92}[0.92]{$\left[\hspace{-0.16ex}\begin{array}{c@{\hspace{.64em}}c@{\hspace{.64em}}c}
2 & 1 & 1 \\
1 & 2 & 1 \\
1 & 1 & 2
\end{array}\hspace{-0.12ex}\right]$} \hspace{-0.2ex}, \:\:
\operatorname{det} \hspace{.12ex} \textsl{g}_{i\hspace{-0.1ex}j} \hspace{-0.25ex} = 4 \hspace{.16ex}, \\
%
\operatorname{adj} \hspace{.12ex} \textsl{g}_{i\hspace{-0.1ex}j} \hspace{-0.25ex} = \hspace{-0.16ex}
\scalebox{0.92}[0.92]{$\left[\hspace{-0.4ex}\begin{array}{r@{\hspace{.5em}}r@{\hspace{.5em}}r}
3 & -1 & -1 \\
-1 & 3 & -1 \\
-1 & -1 & 3
\end{array}\hspace{-0.12ex}\right]^{\hspace{-0.4ex}\scalebox{1.02}{$\T$}}$} \hspace{-0.8ex}, \\
%
\textsl{g}^{\hspace{.32ex}i\hspace{-0.1ex}j} \hspace{-0.25ex} = \textsl{g}_{i\hspace{-0.1ex}j}^{\hspace{.4ex}\expminusone} \hspace{-0.12ex} = \displaystyle \frac{\operatorname{adj} \hspace{.12ex} \textsl{g}_{i\hspace{-0.1ex}j}}{\operatorname{det} \hspace{.12ex} \textsl{g}_{i\hspace{-0.1ex}j}} =
\displaystyle \frac{1}{4} \hspace{.12ex}
\scalebox{0.92}[0.92]{$\left[\hspace{-0.4ex}\begin{array}{r@{\hspace{.5em}}r@{\hspace{.5em}}r}
3 & -1 & -1 \\
-1 & 3 & -1 \\
-1 & -1 & 3
\end{array}\hspace{-0.12ex}\right]^{\mathstrut}$} \hspace{-0.25ex}.
\end{array}\]

\vspace{-0.5em}Using ${\bm{a}^i = \textsl{g}^{\hspace{.32ex}i\hspace{-0.1ex}j} \bm{a}_j}$
\[\begin{array}{l}
\bm{a}^1 \hspace{-0.2ex}=\hspace{.1ex} \textsl{g}^{\hspace{.2ex}1\hspace{-0.12ex}1} \bm{a}_1 \hspace{-0.2ex} + \textsl{g}^{\hspace{.2ex}12} \bm{a}_2 \hspace{-0.2ex} + \textsl{g}^{\hspace{.2ex}13} \bm{a}_3 \hspace{-0.2ex} = \smalldisplaystyleonehalf \bm{e}_1 \hspace{-0.2ex} + \smalldisplaystyleonehalf \bm{e}_2 \hspace{-0.2ex} - \smalldisplaystyleonehalf \bm{e}_3 \hspace{.1ex},\\[.5em]
%
\bm{a}^2 \hspace{-0.2ex}=\hspace{.1ex} \textsl{g}^{\hspace{.2ex}21} \bm{a}_1 \hspace{-0.2ex} + \textsl{g}^{\hspace{.2ex}22} \bm{a}_2 \hspace{-0.2ex} + \textsl{g}^{\hspace{.2ex}23} \bm{a}_3 \hspace{-0.2ex} = \smalldisplaystyleonehalf \bm{e}_1 \hspace{-0.2ex} - \smalldisplaystyleonehalf \bm{e}_2 \hspace{-0.2ex} + \smalldisplaystyleonehalf \bm{e}_3 \hspace{.1ex},\\[.5em]
%
\bm{a}^3 \hspace{-0.2ex}=\hspace{.1ex} \textsl{g}^{\hspace{.2ex}31} \bm{a}_1 \hspace{-0.2ex} + \textsl{g}^{\hspace{.2ex}32} \bm{a}_2 \hspace{-0.2ex} + \textsl{g}^{\hspace{.2ex}33} \bm{a}_3 \hspace{-0.2ex} = - \smalldisplaystyleonehalf \bm{e}_1 \hspace{-0.2ex} + \smalldisplaystyleonehalf \bm{e}_2 \hspace{-0.2ex} + \smalldisplaystyleonehalf \bm{e}_3 \hspace{.16ex}.
\end{array}\]

\par\end{tcolorbox}

...


Единичный тензор~(unit tensor, identity tensor,  metric tensor)

$\bm{E} \dotp \hspace{0.1ex} \bm{\xi} = \bm{\xi} \hspace{0.1ex} \dotp \bm{E} = \bm{\xi} \quad \forall \bm{\xi}$

$\bm{E} \dotdotp \bm{a} \bm{b} = \bm{a} \bm{b} \dotdotp \bm{E} = \bm{a} \dotp \bm{E} \dotp \bm{b} = \bm{a} \dotp \bm{b}$

$\bm{E} \dotdotp \bm{A} = \bm{A} \dotdotp \bm{E} = \operatorname{tr} \bm{A}$


$\bm{E} \dotdotp \hspace{-0.1ex} \bm{A} = \bm{A} \dotdotp \hspace{-0.1ex} \bm{E} = \operatorname{tr} \bm{A} = \operatorname{not anymore} A_{j\hspace{-0.12ex}j}$

Thus for, say, trace of some tensor ${\bm{A} = A_{i\hspace{-0.1ex}j} \bm{r}^i \bm{r}^j}$: $\bm{A} \dotdotp \bm{E} = \operatorname{tr} \bm{A}$, you have

$\bm{A} \dotdotp \bm{E} = A_{i\hspace{-0.1ex}j} \bm{r}^i \bm{r}^j \dotdotp \bm{r}_k \bm{r}^k = A_{i\hspace{-0.1ex}j} \bm{r}^i \dotp \bm{r}^j = A_{i\hspace{-0.1ex}j} \textsl{g}^{\hspace{.32ex}i\hspace{-0.1ex}j}$


...

Тензор поворота~(rotation tensor)

$\bm{P} \hspace{-0.2ex} = \hspace{-0.1ex} \bm{a}_i \hspace{.1ex} \mathcircabove{\bm{a}}^i \hspace{-0.2ex} = \hspace{-0.1ex} \bm{a}^{\hspace{-0.16ex}i} \mathcircabove{\bm{a}}_i \hspace{-0.2ex} = \hspace{-0.1ex} \bm{P}^{\expminusT}$

$\bm{P}^{\expminusone} \hspace{-0.4ex} = \hspace{-0.1ex} \mathcircabove{\bm{a}}_i \hspace{.1ex} \bm{a}^{\hspace{-0.16ex}i} \hspace{-0.2ex} = \hspace{-0.1ex} \mathcircabove{\bm{a}}^i \bm{a}_i \hspace{-0.2ex} = \hspace{-0.1ex} \bm{P}^{\T}$

$\bm{P}^{\T} \hspace{-0.4ex} = \hspace{-0.1ex} \mathcircabove{\bm{a}}^i \bm{a}_i \hspace{-0.2ex} = \hspace{-0.1ex} \mathcircabove{\bm{a}}_i \hspace{.1ex} \bm{a}^{\hspace{-0.16ex}i} \hspace{-0.2ex} = \hspace{-0.1ex} \bm{P}^{\expminusone}$

...



... Характеристическое уравнение~\eqref{chardetequation} быстро приводит к~тождеству К\kern-0.04em\'{э}ли\hbox{--}Г\kern-0.08em\'{а}мильтона~(Cayley\hbox{--}Hamilton)

\nopagebreak\vspace{-0.2em}\begin{equation}\label{cayley-hamilton:eq}
\begin{array}{c}
-\bm{B} \hspace{-0.2ex} \dotp \hspace{-0.2ex} \bm{B} \hspace{-0.2ex} \dotp \hspace{-0.2ex} \bm{B}
+ \hspace{.1ex} \mathrm{I} \hspace{.2ex} \bm{B} \hspace{-0.2ex} \dotp \hspace{-0.2ex} \bm{B}
- \hspace{.1ex} \mathrm{II} \hspace{.2ex} \bm{B}
+ \hspace{.1ex} \mathrm{III} \hspace{.2ex} \bm{E}
= {^2\bm{0}}
\hspace{.1ex} ,
\\[.16em]
-\bm{B}^{3} \hspace{-0.2ex}
+ \hspace{.1ex} \mathrm{I} \hspace{.2ex} \bm{B}^{2} \hspace{-0.2ex}
- \hspace{.1ex} \mathrm{II} \hspace{.2ex} \bm{B}
+ \hspace{.1ex} \mathrm{III} \hspace{.2ex} \bm{E}
= {^2\bm{0}}
\hspace{.1ex} .
\end{array}
\end{equation}

\end{otherlanguage}

\en{\section{Tensor functions}}

\ru{\section{Тензорные функции}}

\label{para:tensorfunctions}

\noindent \en{In~the~conception of~function}\ru{В~представлении о~функции}~${y \narroweq \hspace{-0.15ex} f(x)}$
\en{as of mapping (morphism)}\ru{как отображении (морфизме)} ${\smash{f \hspace{-0.2ex}\colon x \mapsto \hspace{-0.16ex} y}}$,
\en{an~input~(argument)}\ru{прообраз~(аргумент)}~$x$ \en{and an~output~(result)}\ru{и~образ~(результат)}~$y$ \en{may be tensors of any complexities}\ru{могут быть тензорами любых сложностей}.

\en{Consider}\ru{Рассмотрим} \en{at~least}\ru{хотя~бы} \en{a~scalar function}\ru{скалярную функцию} \en{of~a~bivalent tensor}\ru{двухвалентного тензора}~${\varphi \narroweq \varphi(\bm{B}\hspace{.1ex})}$.
\en{Examples}\ru{Примеры}\ru{\:---}\en{ are} ${\bm{B} \hspace{-0.2ex} \dotdotp \hspace{-0.25ex} \scalebox{1.1}[1]{$\bm{\mathit{\Phi}}$}}$ (\en{or}\ru{или}~${\bm{p} \dotp \hspace{-0.2ex} \bm{B} \hspace{-0.16ex} \dotp \hspace{-0.1ex} \bm{q}}$) \en{and}\ru{и}~${\bm{B} \hspace{-0.2ex} \dotdotp \hspace{-0.2ex} \bm{B}}$.
\en{Then}\ru{Тогда} \en{in~each basis}\ru{в~каждом базисе}~${\bm{a}_i}$ \en{paired with}\ru{в~паре с}~\en{cobasis}\ru{кобазисом}~${\bm{a}^{\hspace{-0.1ex}i}}$ \en{we have}\ru{имеем} \en{function}\ru{функцию}~${\varphi(B_{i\hspace{-0.1ex}j})}$ \en{of~nine numeric arguments}\ru{девяти числовых аргументов}\:--- \en{components}\ru{компонент}~$B_{i\hspace{-0.1ex}j}$ \en{of~tensor}\ru{тензора}~$\bm{B}$.
\en{For example}\ru{Для примера}

\nopagebreak\vspace{-0.2em}\begin{equation*}
\varphi(\bm{B}\hspace{.1ex}) \hspace{-0.2ex}
= \bm{B} \hspace{-0.2ex} \dotdotp \hspace{-0.28ex} \scalebox{1.1}[1]{$\bm{\mathit{\Phi}}$} \hspace{-0.1ex}
= \hspace{-0.1ex} B_{i\hspace{-0.1ex}j} \hspace{.16ex} \bm{a}^{\hspace{-0.1ex}i} \hspace{-0.16ex} \bm{a}^j \hspace{-0.3ex} \dotdotp \bm{a}_m \bm{a}_n \hspace{-0.2ex} \scalebox{1.2}[1]{$\mathit{\Phi}$}^{mn} \hspace{-0.25ex}
= \hspace{-0.1ex} B_{i\hspace{-0.1ex}j} \hspace{-0.15ex} \scalebox{1.2}[1]{$\mathit{\Phi}$}^{\hspace{.1ex}j\hspace{-0.06ex}i} \hspace{-0.25ex}
= \varphi(B_{i\hspace{-0.1ex}j})
\hspace{.1ex} .
\end{equation*}

\vspace{-0.25em} \noindent
\en{With any transition}\ru{С~любым переходом} \en{to~a~new basis,}\ru{к~новому базису} \en{the~result}\ru{результат} \en{doesn’t change}\ru{не~меняется}:
${\varphi(B_{i\hspace{-0.1ex}j}) \hspace{-0.2ex} = \varphi(B\hspace{.16ex}'_{\hspace{-0.32ex}i\hspace{-0.1ex}j}) \hspace{-0.2ex} = \varphi(\bm{B}\hspace{.1ex})}$.

\en{Differentiation of}\ru{Дифференцирование}~${\varphi(\bm{B}\hspace{.1ex})}$ \en{looks like}\ru{выглядит как}

\nopagebreak\vspace{-0.2em}\begin{equation}
d \hspace{.1ex} \varphi
= \displaystyle \frac{\partial \hspace{.1ex} \varphi}{\partial \hspace{-0.2ex} B_{i\hspace{-0.1ex}j}} \hspace{.2ex} d B_{i\hspace{-0.1ex}j} \hspace{-0.2ex}
= \displaystyle \frac{\partial \hspace{.1ex} \varphi}{\partial \hspace{-0.1ex} \bm{B}} \dotdotp d \bm{B}^{\T}
\hspace{-0.25ex} .
\end{equation}

\en{\vspace{-0.15em}}\ru{\vspace{-0.25em}}\noindent
\en{Tensor}\ru{Тензор}~${\scalebox{0.98}[1]{$\raisemath{.16em}{\scalebox{0.92}[0.92]{$\partial \hspace{.15ex} \varphi$}} \hspace{-0.1ex} / \hspace{-0.1ex} \raisemath{-0.32em}{\scalebox{0.92}[0.92]{$\partial \bm{B}$}}\hspace{.1ex}$}}$
\en{is called}\ru{называется} \en{the~derivative}\ru{производной} \en{of~function}\ru{функции}~$\varphi$ \en{by~argument}\ru{по~аргументу}~${\hspace{-0.15ex}\bm{B}\hspace{.1ex}}$;
${d \bm{B}}$\en{~is}\ru{\:---} \en{the~differential}\ru{дифференциал} \en{of~tensor}\ru{тензора}~$\bm{B}$,
${d \bm{B} \hspace{-0.1ex} = d B_{i\hspace{-0.1ex}j} \hspace{.16ex} \bm{a}^{\hspace{-0.1ex}i} \hspace{-0.16ex} \bm{a}^j}$;
${\smash{\scalebox{0.98}[1]{$\raisemath{.16em}{\scalebox{0.92}[0.92]{$\partial \hspace{.15ex} \varphi$}} \hspace{-0.1ex} / \hspace{-0.1ex} \raisemath{-0.32em}{\scalebox{0.92}[0.92]{$\partial \hspace{-0.1ex} B_{i\hspace{-0.1ex}j}$}}$}}}$\ru{\:---}\en{~are} \en{components}\ru{компоненты}~(\en{contra\-variant ones}\ru{контра\-вариант\-ные}) \en{of~\,}${\smash{\scalebox{0.98}[1]{$\raisemath{.16em}{\scalebox{0.92}[0.92]{$\partial \hspace{.15ex} \varphi$}} \hspace{-0.1ex} / \hspace{-0.1ex} \raisemath{-0.32em}{\scalebox{0.92}[0.92]{$\partial \bm{B}$}}\hspace{.2ex}$}}}$

\nopagebreak\vspace{-0.12em}\begin{equation*}
\bm{a}^{\hspace{-0.1ex}i} \hspace{-0.15ex} \dotp \scalebox{0.92}[0.92]{$\displaystyle \frac{\partial \hspace{.1ex} \varphi}{\partial \hspace{-0.1ex} \bm{B}}$} \dotp \bm{a}^j \hspace{-0.2ex}
=
\scalebox{0.92}[0.92]{$\displaystyle \frac{\partial \hspace{.1ex} \varphi}{\partial \hspace{-0.1ex} \bm{B}}$} \dotdotp \bm{a}^j \hspace{-0.2ex} \bm{a}^{\hspace{-0.1ex}i} \hspace{-0.2ex}
=
\scalebox{0.92}[0.92]{$\displaystyle \frac{\partial \hspace{.1ex} \varphi}{\partial \hspace{-0.15ex} B_{i\hspace{-0.1ex}j}}$}
\;\Leftrightarrow\;
\scalebox{0.92}[0.92]{$\displaystyle \frac{\partial \hspace{.1ex} \varphi}{\partial \hspace{-0.1ex} \bm{B}}$}
=
\scalebox{0.92}[0.92]{$\displaystyle \frac{\partial \hspace{.1ex} \varphi}{\partial \hspace{-0.15ex} B_{i\hspace{-0.1ex}j}}$} \hspace{.25ex} \bm{a}_i \bm{a}_{\hspace{-0.1ex}j}
\hspace{.1ex} .
\end{equation*}


...


\nopagebreak\begin{equation*}\begin{array}{c}
\varphi(\bm{B}\hspace{.1ex}) \hspace{-0.2ex}
= \bm{B} \hspace{-0.2ex} \dotdotp \hspace{-0.25ex} \scalebox{1.1}[1]{$\bm{\mathit{\Phi}}$}
\\[.2em]
%
d \hspace{.1ex} \varphi
= d \hspace{.2ex} ( \bm{B} \hspace{-0.2ex} \dotdotp \hspace{-0.25ex} \scalebox{1.1}[1]{$\bm{\mathit{\Phi}}$} \hspace{.1ex} ) \hspace{-0.2ex}
= d \bm{B} \hspace{-0.2ex} \dotdotp \hspace{-0.25ex} \scalebox{1.1}[1]{$\bm{\mathit{\Phi}}$}
= \hspace{-0.1ex} \scalebox{1.1}[1]{$\bm{\mathit{\Phi}}$} \hspace{-0.12ex} \dotdotp d \bm{B}
= \hspace{-0.1ex} \scalebox{1.1}[1]{$\bm{\mathit{\Phi}}$}^{\T} \hspace{-0.4ex} \dotdotp d \bm{B}^{\T}
\\[.2em]
%
d \hspace{.1ex} \varphi
= \scalebox{0.92}[0.92]{$\displaystyle \frac{\partial \hspace{.1ex} \varphi}{\partial \hspace{-0.1ex} \bm{B}} \dotdotp d \bm{B}^{\T}$} \hspace{-0.3ex} ,
\:\:
\scalebox{0.92}[0.92]{$\displaystyle \frac{\partial \hspace{-0.1ex} \left( \bm{B} \hspace{-0.2ex} \dotdotp \hspace{-0.25ex} \scalebox{1.1}[1]{$\bm{\mathit{\Phi}}$} \right)}{\partial \hspace{-0.1ex} \bm{B}}$}
= \hspace{-0.1ex} \scalebox{1.1}[1]{$\bm{\mathit{\Phi}}$}^{\T}
\end{array}\end{equation*}

${\bm{p} \dotp \hspace{-0.2ex} \bm{B} \hspace{-0.16ex} \dotp \hspace{-0.1ex} \bm{q} \hspace{.1ex} =
\hspace{-0.1ex} \bm{B} \hspace{-0.16ex} \dotdotp \hspace{-0.1ex} \bm{q} \bm{p}}$

\nopagebreak\begin{equation*}
\scalebox{0.92}[0.92]{$\displaystyle \frac{\partial \hspace{-0.1ex} \left( \hspace{.1ex} \bm{p} \dotp \hspace{-0.2ex} \bm{B} \hspace{-0.16ex} \dotp \hspace{-0.1ex} \bm{q} \right)}{\partial \hspace{-0.1ex} \bm{B}}$}
= \bm{p} \bm{q}
\end{equation*}

...

\nopagebreak\begin{equation*}\begin{array}{c}
\varphi(\bm{B}\hspace{.1ex}) \hspace{-0.2ex}
= \bm{B} \hspace{-0.2ex} \dotdotp \hspace{-0.2ex} \bm{B}
\\[.2em]
%
d \hspace{.1ex} \varphi
= d \hspace{.2ex} ( \bm{B} \hspace{-0.2ex} \dotdotp \hspace{-0.2ex} \bm{B} \hspace{.1ex} ) \hspace{-0.2ex}
= d ...
\end{array}\end{equation*}


...


\begin{otherlanguage}{russian}

Но согласно опять\hbox{-}таки~\eqref{cayley-hamilton:eq}
${\hspace{-0.1ex} -\bm{B}^{2} \hspace{-0.2ex} + \mathrm{I}\hspace{0.16ex} \bm{B} \hspace{-0.1ex} - \mathrm{II}\hspace{0.16ex} \bm{E} + \mathrm{III}\hspace{0.16ex} \bm{B}^{\expminusone} \hspace{-0.25ex} = {^2\bm{0}}}$, поэтому


...


Скалярная функция~${\varphi(\bm{B})}$ называется изотропной, если она не~чувствительна к~повороту аргумента:
\nopagebreak\vspace{.1em}\begin{equation*}
\varphi(\bm{B}) \hspace{-0.12ex} = \varphi ( \rotationtensor \narrowdotp \smash{\mathcircabove{\bm{B}}} \narrowdotp \hspace{.15ex} \rotationtensor^{\hspace{-0.1ex}\T} ) \hspace{-0.2ex} = \varphi(\smash{\mathcircabove{\bm{B}}}) \;\;\,
\forall \rotationtensor \hspace{-0.2ex} = \hspace{-0.1ex} \bm{a}_i \hspace{.1ex} \mathcircabove{\bm{a}}^i \hspace{-0.25ex} = \hspace{-0.1ex} \bm{a}^{\hspace{-0.2ex}i} \mathcircabove{\bm{a}}_i \hspace{-0.16ex} = \hspace{-0.1ex} \rotationtensor^{\hspace{-0.1ex}\expminusT}
\end{equation*}
\par\vspace{-0.25em}\noindent для~любого ортогонального тензора~$\rotationtensor$ (тензора поворота, \pararef{para:rotationtensor}).

Симметричный тензор~${\bm{B}^{\mathsf{\hspace{0.1ex}S}}}$ полностью определяется тройкой инвариантов и~угловой ориентацией собственных осей (они~же взаимно ортогональны, \pararef{para:eigenvectorseigenvalues}). Ясно, что изотропная функция~${\varphi(\bm{B}^{\mathsf{\hspace{0.1ex}S}})}$ симметричного аргумента является функцией лишь инвариантов ${\mathrm{I}\hspace{0.16ex}({\bm{B}^{\mathsf{\hspace{0.1ex}S}}})}$, ${\mathrm{II}\hspace{0.16ex}({\bm{B}^{\mathsf{\hspace{0.1ex}S}}})}$, ${\mathrm{III}\hspace{0.16ex}({\bm{B}^{\mathsf{\hspace{0.1ex}S}}})}$; она дифференцируется согласно~\eqref{fonvccbnmxghjsxmnxjsdjhga}, где транспонирование излишне.

\end{otherlanguage}

\en{\section{Tensor fields. Differentiation}}

\ru{\section{Тензорные поля. Дифференцирование}}

\label{para:differentiationoftensorfields}

\begin{otherlanguage}{russian}

\begin{changemargin}{\parindent}{\parindent}
\vspace{-0.1em}
\small \flushright \textit{\en{Tensor field}\ru{Тензорное поле}}\ru{\:---}\en{ is} \en{a~tensor}\ru{это тензор}, \en{varying from~point to~point}\ru{меняющийся от~точки к~точке} (\en{variable in~space}\ru{переменный в~пространстве}, \en{coordinate dependent}\ru{зависящий от~координат})
\par\vspace{.25em}
\end{changemargin}

\noindent Путь в~каждой точке некоторой области трёхмерного пространства
% известно значение величины (value of value)
известна величина~$\varsigma$. Тогда имеем поле~${\varsigma \!=\! \varsigma(\bm{r})}$, где~$\bm{r}$\:--- радиус\hbox{-}вектор точки. Например, поле температуры в~среде, поле давления в~идеальной жидкости. Величина~$\varsigma$ может~быть тензором любой сложности. Пример векторного поля\:--- скорости частиц жидкости.

\end{otherlanguage}

% ~ ~ ~ ~ ~
\begin{wrapfigure}{R}{0.55\textwidth}
\makebox[0.55\textwidth][c]{%
\hspace{2em}
\begin{minipage}[t]{.55\textwidth}

\begin{tikzpicture}[scale=0.5]

%%\clip (-6, -6) rectangle + (12, 12) ; % crop it
\clip (0, 0) circle (6cm) ; % crop it

\tikzset{%
	tangent/.style={
		decoration={
			markings,% switch on markings
			mark=
			at position #1
			with
			{
				\def\numberoftangent{\pgfkeysvalueof{/pgf/decoration/mark info/sequence number}}
				\coordinate (tangent point-\numberoftangent) at (0, 0);
				\coordinate (tangent unit vector-\numberoftangent) at (1, 0);
				\coordinate (tangent orthogonal unit vector-\numberoftangent) at (0, 1);
			}
		},
		postaction=decorate
	},
	use tangent/.style={
		shift=(tangent point-#1),
		x=(tangent unit vector-#1),
		y=(tangent orthogonal unit vector-#1)
	},
	use tangent/.default=1
}

\tikzset{%
	show curve controls/.style={
		postaction={
			decoration={
				show path construction,
				curveto code={
					\fill [black, opacity=.5]
						(\tikzinputsegmentfirst) circle (.4ex)
						(\tikzinputsegmentlast) circle (.4ex) ;
					\draw [black, opacity=.5, line cap=round, dash pattern=on 0pt off 1.6\pgflinewidth]
						(\tikzinputsegmentfirst) -- (\tikzinputsegmentsupporta)
						(\tikzinputsegmentlast) -- (\tikzinputsegmentsupportb) ;
					\fill [magenta, opacity=.5, line cap=round, dash pattern=on 0pt off 1.6\pgflinewidth]
						(\tikzinputsegmentsupporta) circle [radius=.4ex]
						(\tikzinputsegmentsupportb) circle [radius=.4ex] ;
				}
			},
			decorate
}	}	}

%%\foreach \cycle in {0, 1, ..., 15}
%%	\draw [color=green]
%%		($ (0, 0) - (\cycle, 1.2*\cycle) $)
%%		parabola ($ (4, 3) + 0.5*(1.6*\cycle, \cycle) $);

\foreach \c in {-10, -9.5, ..., 10}
{
	\def\offset{0.2*\c, -0.1*\c}
	\pgfmathsetmacro\bottomoffsetx{-.24 * ( \c )}
	\pgfmathsetmacro\bottomoffsety{-.1 * abs( \c ) + .1 * ( \c )}
	\pgfmathsetmacro\bottomangle{12 - 1.2 * abs( \c )}
	\pgfmathsetmacro\bottomnudge{2}
	\pgfmathsetmacro\midoffsetx{-.1 * abs( \c )}
	\pgfmathsetmacro\midoffsety{.1 * abs( \c )}
	\pgfmathsetmacro\midangle{63 + 1.2 * abs( \c )}
	\pgfmathsetmacro\midnudge{4 + ( .1 * abs( \c ) )}
	\pgfmathsetmacro\topoffsetx{.32 * ( \c ) + 0 * abs( \c )}
	\pgfmathsetmacro\topoffsety{-.16 * ( \c ) + 0 * abs( \c )}
	\pgfmathsetmacro\topangle{166 + 1.6 * ( \c ) + 1.2 * abs( \c )}
	\pgfmathsetmacro\topnudge{5 + ( .25 * abs( \c ) )}
	\draw	[ line width=.4pt
		, color=blue!50
		%%, show curve controls
		]
		($ (-6, -4.5) + 5*(\offset) + (\bottomoffsetx, \bottomoffsety) $)
		.. controls ++(\bottomangle: \bottomnudge) and ++(\midangle: -\midnudge) ..
		($ 4*(\offset) + (\midoffsetx, \midoffsety) $)
		.. controls ++(\midangle: \midnudge) and ++(\topangle: \topnudge) ..
		($ (8, 6) + 2.5*(\offset) + (\topoffsetx, \topoffsety) $) ;
}

\foreach \c in {-10, -9.5, ..., 10}
{
	\def\offset{0.2*\c, 0.1*\c}
	\pgfmathsetmacro\leftoffsetx{- .1 * abs ( \c )}
	\pgfmathsetmacro\leftoffsety{.4 * ( \c )}
	\pgfmathsetmacro\leftangle{33 + .2 * abs( \c )}
	\pgfmathsetmacro\leftnudge{1.6 + .5 * abs( \c )}
	\pgfmathsetmacro\midoffsetx{-.2 * abs( \c )}
	\pgfmathsetmacro\midoffsety{.2 * abs( \c )}
	\pgfmathsetmacro\midangle{111 + 1.2 * abs( \c )}
	\pgfmathsetmacro\midnudge{5}
	\pgfmathsetmacro\rightoffsetx{.25 * abs( \c )}
	\pgfmathsetmacro\rightoffsety{.16 * ( \c )}
	\pgfmathsetmacro\rightangle{177 + 2 * ( \c )}
	\pgfmathsetmacro\rightnudge{abs( 2 - ( .5 * ( \c ) ) )}
	\draw	[ line width=.4pt
		, color=red!50
		%%, show curve controls
		]
		($ (-12, 5) + 2.5*(\offset) + (\leftoffsetx, \leftoffsety) $)
		.. controls ++(\leftangle: \leftnudge) and ++(\midangle: \midnudge) ..
		($ 5*(\offset) + (\midoffsetx, \midoffsety) $)
		.. controls ++(\midangle: -\midnudge) and ++(\rightangle: \rightnudge) ..
		($ (8, -5) + 4*(\offset) + (\rightoffsetx, \rightoffsety) $);
}

\foreach \c in {-10, -9.5, ..., 10}
{
	\def\offset{0*\c, 0.25*\c}
	\pgfmathsetmacro\midnudge{6 + .16 * ( \c )}
	\draw	[ line width=.4pt
		, color=green!50
		%%, show curve controls
		]
		($ (12, 10) + 4*(\offset) $)
		.. controls ++(88: -4) and ++(11: \midnudge) ..
		($ 4*(\offset) $)
		.. controls ++(11: -\midnudge) and ++(99: -4) ..
		($ (-12, 4) + 4*(\offset) $) ;
}

\draw	[ line width=.8pt
	, color=blue!50!black
	%%, show curve controls
	]
	(-6, -4.5)
	.. controls ++(12: 2) and ++(63: -4) ..
	(0, 0);

\draw	[ line width=.8pt
	, color=blue!50!black
	%%, show curve controls
	, tangent=0
	, tangent=0.4
	]
	(0, 0)
	.. controls ++(63: 4) and ++(166: 5) ..
	(8, 6) ;

\path [use tangent=1]
	(0, 0) -- (.4*4, 0)
	node [color=blue, pos=0.86, above left, shape=circle, fill=white, outer sep=4pt, inner sep=1pt]
		{$\bm{r}_3$} ;

\draw [line width=1.25pt, color=blue, use tangent=1, -{Latex[round, length=3.6mm, width=2.4mm]}]
	(0, 0) -- (.4*4, 0) ;

\path [use tangent=2]
	(0, 0) -- (0, -1)
	node [color=blue!50!black, pos=0.48, above, shape=circle, fill=white, outer sep=0pt, inner sep=0.25pt]
		{$q^{\hspace{.1ex}3}$} ;

%%\fill [fill=blue, use tangent=1] (0, 0) circle (1mm);

\draw	[ line width=.8pt
	, color=red!50!black
	%%, show curve controls
	]
	(-12, 5)
	.. controls ++(33: 1.6) and ++(111: 5) ..
	(0, 0);

\draw	[ line width=.8pt
	, color=red!50!black
	%%, show curve controls
	, tangent=0
	, tangent=0.5
	]
	(0, 0)
	.. controls ++(111: -5) and ++(177: 2) ..
	(8, -5) ;

\path [use tangent=1]
	(0, 0) -- (.4*5, 0)
	node [color=red, pos=0.86, below left, shape=circle, fill=white, outer sep=4pt, inner sep=1pt]
		{$\bm{r}_1$} ;

\draw [line width=1.25pt, color=red, use tangent=1, -{Latex[round, length=3.6mm, width=2.4mm]}]
	(0, 0) -- (.4*5, 0);

\path [use tangent=2]
	(0, 0) -- (0, 1)
	node [color=red!50!black, pos=0.16, above, shape=circle, fill=white, outer sep=0pt, inner sep=0.25pt]
		{$q^{1}$} ;

%%\fill [fill=red, use tangent=1] (0, 0) circle (1mm);

\draw	[ line width=.8pt
	, color=green!50!black
	%%, show curve controls
	]
	(12, 10)
	.. controls ++(88: -4) and ++(11: 6) ..
	(0, 0) ;

\draw	[ line width=.8pt
	, color=green!50!black
	%%, show curve controls
	, tangent=0
	, tangent=0.36
	]
	(0, 0)
	.. controls ++(11: -6) and ++(99: -4) ..
	(-12, 4) ;

\path [use tangent=1]
	(0, 0) -- (.4*6, 0)
	node [color=green, pos=0.92, below right, shape=circle, fill=white, outer sep=5pt, inner sep=1pt]
		{$\bm{r}_2$} ;

\draw [line width=1.25pt, color=green, use tangent=1, -{Latex[round, length=3.6mm, width=2.4mm]}]
	(0, 0) -- (.4*6, 0);

\path [use tangent=2]
	(0, 0) -- (0, -1)
	node [color=green!50!black, pos=0.12, above, shape=circle, fill=white, outer sep=0pt, inner sep=0.25pt]
		{$q^{\hspace{.1ex}2}$} ;

%%\fill [fill=green, use tangent=1] (0, 0) circle (1mm);

\coordinate (theOrigin) at (5, -2) ;
\path (0, 0) circle (1mm) node [shape=circle, inner sep=.5mm, outer sep=0] (theCircleOfO) {} ;

\draw [line width=1.5pt, black, -{Stealth[round,length=4mm,width=2.8mm]}] (theOrigin) -- (theCircleOfO)
		node [pos=0.64, above right, shape=circle, fill=white, outer sep=2pt, inner sep=1.2pt]
			{$\bm{r}$} ;

\draw [line width=1.2pt, color=black, fill=white] (0, 0) circle (1ex);

\draw [line width=1.2pt, color=black, fill=white] (theOrigin) circle (1ex);

\end{tikzpicture}

\vspace{0.1em}\caption{}\label{fig:curvilinearcoordinates}
\end{minipage}}
\end{wrapfigure}

% ~ ~ ~ ~ ~

\begin{otherlanguage}{russian}

Не~только при~решении прикладных задач, но нередко и для~\inquotes{чистой тео\-рии} вместо аргумента~$\bm{r}$ ис\-поль\-зу\-ет\-ся какая-либо трой\-ка криво\-линей\-ных координат~${q^{\hspace{.1ex}i}\hspace{-0.2ex}}$. При~этом ${\bm{r} \!=\! \bm{r}(q^{\hspace{.1ex}i}\hspace{.1ex})}$. Если непрерывно менять лишь одну координату из~трёх, получается координатная линия. Каждая точка трёхмерного пространства лежит на~пересечении трёх координатных линий (\figref{fig:curvilinearcoordinates}).

Commonly used curvilinear coordinate systems include: rectangular~(\inquotes{cartesian}), spherical, and cylindrical coordinate systems. These coordinates may be derived from a~set of cartesian coordinates by using a~transformation that is locally invertible (a~one-to-one map) at~each point. This means that one can convert a~point given in a~cartesian coordinate system to its curvilinear coordinates and~back.

...

The differential represents a~change in the linearization of a~function.

...

\nopagebreak\vspace{-0.1em}\begin{equation*}
\partial_i \equiv \frac{\raisebox{-0.2em}{$\partial$}}{\raisebox{-0.1em}{$\partial q^i$}}
\end{equation*}

...

\en{Linearity}\ru{Линейность}

\nopagebreak\vspace{-0.4em}\begin{equation}\label{linearityordifferentiation}
\partial_i \hspace{-0.2ex} \left( \lambda \hspace{.1ex} p + \hspace{-0.1ex} \mu \hspace{.1ex} q \right)
= \lambda \hspace{-0.1ex} \left( \partial_i \hspace{.1ex} p \right) + \hspace{.1ex}
\mu \hspace{-0.1ex} \left( \partial_i \hspace{.1ex} q \right)
\end{equation}

\inquotes{Product rule}

\nopagebreak\vspace{-0.4em}\begin{equation}\label{productrulefordifferentiation}
\partial_i \hspace{-0.2ex} \left( \hspace{.1ex} p \circ q \right)
= \left( \partial_i \hspace{.1ex} p \right) \hspace{-0.12ex} \circ q \hspace{.16ex} +
\hspace{.16ex} p \circ \hspace{-0.12ex} \left( \partial_i \hspace{.1ex} q \right)
\end{equation}

...

Bivalent unit tensor~(identity tensor,  metric tensor), the one which is neutral~\eqref{identifyofidentitytensor} to dot product operation,
\en{can be represented as}\ru{может быть представлен как}

\nopagebreak\vspace{-0.1em}\begin{equation}
\bm{E} = \bm{r}^i \bm{r}_i = \tikzmark{beginOriginOfNabla} \bm{r}^i \partial_i \tikzmark{endOriginOfNabla} \hspace{.1ex} \bm{r} = \hspace{-0.16ex} \boldnabla \bm{r} ,
\end{equation}
\AddUnderBrace[line width=.75pt][0,-0.1ex]%
{beginOriginOfNabla}{endOriginOfNabla}%
{${\scriptstyle \boldnabla}$}

\vspace{-0.4em} \noindent \en{where appears differential \inquotes{nabla} operator} \ru{где появляется дифференциальный оператор \inquotes{набла}}

\nopagebreak\vspace{-0.1em}\begin{equation}
\boldnabla \equiv \bm{r}^i \partial_i \hspace{.1ex} .
\end{equation}

...

\en{Divergence}\ru{Дивергенция} \en{of~the~dyadic product}\ru{диадного произведения} \en{of~two vectors}\ru{двух векторов}

\nopagebreak\vspace{-0.3em}\begin{multline}\label{divergenceofdyadicproducoftwovectors}
\boldnabla \hspace{-0.16ex} \dotp \hspace{-0.2ex} \bigl( \hspace{-0.1ex} \bm{a} \bm{b} \hspace{.05ex} \bigr) \hspace{-0.33ex}
= \bm{r}^i \partial_i \hspace{-0.1ex} \dotp \hspace{-0.24ex} \bigl( \hspace{-0.1ex} \bm{a} \bm{b} \bigr) \hspace{-0.33ex}
= \bm{r}^i \hspace{-0.3ex} \dotp \partial_i \bigl( \hspace{-0.1ex} \bm{a} \bm{b} \bigr) \hspace{-0.3ex}
= \bm{r}^i \hspace{-0.3ex} \dotp \hspace{-0.15ex} \bigl( \partial_i \bm{a} \bigr) \bm{b} \hspace{.1ex} + \bm{r}^i \hspace{-0.3ex} \dotp \bm{a} \hspace{.1ex} \bigl( \partial_i \bm{b} \bigr) \hspace{-0.33ex} =
\\[-0.1em]
%
= \hspace{-0.15ex} \bigl( \bm{r}^i \hspace{-0.3ex} \dotp \partial_i \bm{a} \bigr) \bm{b}  \hspace{.1ex} + \bm{a} \dotp \bm{r}^i \hspace{-0.15ex} \bigl( \partial_i \bm{b} \bigr) \hspace{-0.33ex}
= \hspace{-0.15ex} \bigl( \bm{r}^i \partial_i \hspace{-0.1ex} \dotp \bm{a} \bigr) \bm{b}  \hspace{.1ex} + \bm{a} \dotp \hspace{-0.1ex} \bigl( \bm{r}^i \partial_i \bm{b} \bigr) \hspace{-0.33ex} =
\\
%
= \hspace{-0.15ex} \bigl( \boldnabla \hspace{-0.16ex} \dotp \hspace{-0.1ex} \bm{a} \bigr) \bm{b} \hspace{.1ex} + \bm{a} \dotp \hspace{-0.12ex} \bigl( \boldnabla \hspace{.1ex} \bm{b} \bigr)
\end{multline}

\vspace{-0.2em} \noindent --- \en{here’s no~need}\ru{здесь не~нужно} \en{to~expand}\ru{разворачивать} \en{vectors}\ru{векторы}~$\bm{a}$ \en{and}\ru{и}~$\bm{b}$, \en{expanding just}\ru{развернув лишь} \en{differential operator nabla}\ru{дифференциальный оператор набла}~${\hspace{-0.13ex}\boldnabla}$.

...

\en{Gradient of cross product of two vectors}\ru{Градиент векторного произведения двух векторов},
\en{applying}\ru{применяя} \inquotes{product rule}~\eqref{productrulefordifferentiation}
\en{and}\ru{и}~\en{relation}\ru{соотношение}~\eqref{crossproductoftwovectors} \en{for any two vectors}\ru{для любых двух векторов}
(\en{partial derivative}\ru{частная производная}~$\partial_i$ \en{of~some~vector by scalar coordinate}\ru{некоторого вектора по скалярной координате}~$q^i\hspace{-0.1ex}$ \en{is a~vector too}\ru{это тоже вектор})

\nopagebreak\vspace{-0.4em}\begin{multline}\label{gradientofcrossproductoftwovectors}
\boldnabla \hspace{-0.2ex} \left( \bm{a} \hspace{-0.1ex} \times \hspace{-0.1ex} \bm{b} \right) \hspace{-0.2ex}
= \hspace{.1ex} \bm{r}^i \partial_i \hspace{-0.3ex} \left( \bm{a} \hspace{-0.2ex} \times \hspace{-0.2ex} \bm{b} \right) \hspace{-0.2ex}
= \bm{r}^i \hspace{-0.4ex} \left( \partial_i \bm{a} \hspace{-0.2ex} \times \hspace{-0.2ex} \bm{b} \hspace{.1ex} +
\bm{a} \hspace{-0.2ex} \times \hspace{-0.2ex} \partial_i \bm{b} \right) \hspace{-0.2ex} =
\\[-0.1em]
%
= \bm{r}^i \hspace{-0.4ex} \left( \partial_i \bm{a} \hspace{-0.2ex} \times \hspace{-0.2ex} \bm{b} \hspace{.1ex} -
\partial_i \bm{b} \hspace{-0.2ex} \times \hspace{-0.2ex} \bm{a} \right) \hspace{-0.2ex}
= \hspace{.1ex} \bm{r}^i \partial_i \hspace{.1ex} \bm{a} \hspace{-0.2ex} \times \hspace{-0.2ex} \bm{b} \hspace{.1ex} - \hspace{.1ex}
\bm{r}^i \partial_i \hspace{.1ex} \bm{b} \hspace{-0.2ex} \times \hspace{-0.2ex} \bm{a} =
\\[-0.1em]
%
= \hspace{-0.12ex} \boldnabla \bm{a} \hspace{-0.1ex} \times \hspace{-0.1ex} \bm{b} \hspace{.12ex} - \hspace{-0.12ex}
\boldnabla \hspace{.1ex} \bm{b} \hspace{-0.1ex} \times \hspace{-0.1ex} \bm{a}
\hspace{.2ex} .
\end{multline}

...

\en{Gradient}\ru{Градиент} \en{of }dot product\ru{’а} \en{of two vectors}\ru{двух векторов}

\nopagebreak\vspace{-0.4em}\begin{multline}\label{gradientofdotproductoftwovectors}
\boldnabla \hspace{.1ex} \bigl( \hspace{-0.05ex} \bm{a} \hspace{-0.1ex} \dotp \hspace{-0.1ex} \bm{b} \hspace{.05ex} \bigr) \hspace{-0.3ex}
= \hspace{.1ex} \bm{r}^i \partial_i \bigl( \hspace{-0.05ex} \bm{a} \hspace{-0.1ex} \dotp \hspace{-0.1ex} \bm{b} \hspace{.05ex} \bigr) \hspace{-0.33ex}
= \hspace{.1ex} \bm{r}^i \bigl( \partial_i \bm{a} \bigr) \hspace{-0.32ex} \dotp \bm{b} + \hspace{.1ex} \bm{r}^i \bm{a} \hspace{-0.05ex} \dotp \hspace{-0.15ex} \bigl( \partial_i \bm{b} \bigr)  \hspace{-0.33ex} =
\\[-0.1em]
%
= \hspace{-0.2ex} \bigl( \bm{r}^i \partial_i \bm{a} \bigr) \hspace{-0.32ex} \dotp \bm{b} \hspace{.1ex} + \hspace{.1ex} \bm{r}^i \bigl( \partial_i \bm{b} \bigr) \hspace{-0.33ex} \dotp \bm{a}
= \hspace{-0.16ex} \bigl( \boldnabla \hspace{-0.1ex} \bm{a} \bigr) \hspace{-0.3ex} \dotp \hspace{.1ex} \bm{b} \hspace{.1ex} + \hspace{-0.1ex} \bigl( \boldnabla \hspace{.1ex} \bm{b} \bigr) \hspace{-0.27ex} \dotp \hspace{.1ex} \bm{a}
\hspace{.2ex} .
\end{multline}

\newpage ...

\newpage ...



\end{otherlanguage}

\en{\section{Integral theorems}}

\ru{\section{Интегральные теоремы}}

\begin{otherlanguage}{russian}

Для векторных полей известны интегральные теоремы Gauss’а и~Stokes’а.

\noindent\leavevmode{\indent}{\small Gauss’ theorem (divergence theorem) enables an~integral taken over a~volume to be replaced by one taken over the closed surface bounding that volume, and vice versa.\par}

\noindent\leavevmode{\indent}{\small Stokes’ theorem enables an~integral taken around a closed curve to be replaced by one taken over \emph{any} surface bounded by that curve. Stokes’ theorem relates a~line integral around a closed path to a surface integral over what is called a~\emph{capping surface} of the path.\par}

Теорема Гаусса о~дивергенции\:--- про~то, как заменить объёмный интеграл поверхностным~(\en{and vice versa}\ru{и~наоборот}). В~этой теореме рассматривается поток (ef)flux вектора через ограничивающую объём~$V$ з\'{а}мкнутую поверхность ${\mathcal{O}(\boundary V)}$ с~единичным вектором внешней нормали~$\bm{n}$

\nopagebreak\vspace{-0.1em}\begin{equation}
\ointegral\displaylimits_{\mathclap{\mathcal{O}(\boundary V)}} \hspace{-0.1ex} \bm{n} \dotp \bm{a} \hspace{.4ex} d\mathcal{O} \hspace{.12ex} = \integral\displaylimits_{V} \hspace{-0.3ex} \boldnabla \hspace{-0.12ex} \dotp \bm{a} \hspace{.4ex} dV \hspace{-0.25ex}.
\end{equation}

Объём~$V$ нарезается тремя семействами координатных поверхностей на~множество бесконечно малых элементов. Поток через поверхность ${\mathcal{O}(\boundary V)}$ равен сумме потоков через края получившихся элементов. В~бесконечной малости каждый такой элемент\:--- маленький локальный дифференциальный кубик~(параллелепипед). ... Поток вектора~$\bm{a}$ через грани малого кубика объёма~$dV$ есть ${\sum_{i = 1}^{6} \bm{n}_i \dotp \bm{a} \hspace{.2ex} \mathcal{O}_i}$, а~через сам этот объём поток равен ${\boldnabla \dotp \bm{a} \hspace{.32ex} dV}$.

Похожая трактовка этой теоремы есть, к примеру, в~курсе Richard’а Feynman’а~\cite{feynman-lecturesonphysics}.

\emph{( рисунок с кубиками )}

to dice\:--- нарез\'{а}ть кубиками

small cube, little cube

локально ортонормальные координаты ${\bm{\xi} = \xi_i \hspace{.2ex} \bm{n}_i \hspace{.1ex}}$, ${d\bm{\xi} = d \xi_i \hspace{.2ex} \bm{n}_i}$, ${\boldnabla = \bm{n}_i \partial_i}$

разложение вектора ${\bm{a} = a_i \bm{n}_i \hspace{.1ex}}$

Теорема Стокса о~циркуляции выражается равенством

...

\newpage ...



\end{otherlanguage}



\newpage

\en{\section{Curvature tensors}}

\ru{\section{Тензоры кривизны}}

\label{para:curvaturetensors}

\begin{changemargin}{2\parindent}{\parindent}
\bgroup % to change \parindent locally
\setlength{\parindent}{\negparindent}
\setlength{\parskip}{0.8mm minus0.2mm}
\small

\leavevmode{\indent}The \emph{Riemann curvature tensor} or \emph{Riemann\hbox{--}Christoffel tensor} (after \href{https://en.wikipedia.org/wiki/Bernhard_Riemann}{\textbold{Bernhard Riemann}} and \href{https://en.wikipedia.org/wiki/Elwin_Bruno_Christoffel}{\textbold{Elwin Bruno Christoffel}}) is the most common method used to express the curvature of Riemannian manifolds. I\kern-0.12ext’s a~tensor field, it assigns a~tensor to each point of a~Riemannian manifold, that measures the extent to which the~metric tensor is not locally isometric to that of \inquotes{flat} space. The curvature tensor measures noncommutativity of the covariant derivative, and as such is the integrability obstruction for the existence of an isometry with \inquotes{flat} space. The curvature tensor can also be defined for any pseudo-Riemannian manifold, or any manifold equipped with an~\inquotes{affine connection} (a~choice of such connection makes a~manifold look infinitesimally like affine \inquotes{flat} space).

The \emph{Ricci curvature tensor}, named after \href{https://en.wikipedia.org/wiki/Gregorio_Ricci-Curbastro}{\textbold{Gregorio Ricci\hbox{-}Curbastro}}, represents the amount by which the~volume of a~narrow conical piece of~a~small geodesic ball in a~curved Riemannian manifold deviates from that of the standard ball in \inquotes{flat} space.

\vspace{.2em}
\hfill $\sim$\:\emph{from Wikipedia, the free encyclopedia}
\par
\egroup
\nopagebreak\vspace{.12em}
\end{changemargin}

\begin{otherlanguage}{russian}

\noindent Рассматривая операции тензорного анализа в~криволинейных координатах, мы исходили из~представления вектора\hbox{-}радиуса функцией этих координат: ${\bm{r} \!=\! \bm{r}(q^{\hspace{.1ex}i})}$. Этой зависимостью порождаются выражения
векторов локального касательного \hbox{базиса} ${\bm{r}_i \hspace{-0.16ex} \equiv \partial_i \hspace{.12ex} \bm{r}}$\:${( \hspace{.12ex} \partial_i \hspace{-0.1ex} \equiv \smash{\raisemath{0.16em}{\scalebox{0.88}{$\partial$}} \hspace{-0.24ex} / \hspace{-0.32ex} \raisemath{-0.32em}{\scalebox{0.88}{$\partial$} q^{\hspace{.1ex}i}}} \hspace{.16ex} )}$,
компонент ${\textsl{g}_{i\hspace{-0.1ex}j} \hspace{-0.24ex} \equiv \bm{r}_i \hspace{-0.16ex} \dotp \bm{r}_{\hspace{-0.2ex}j}}$ и~${\textsl{g}^{\hspace{.25ex}i\hspace{-0.1ex}j} \hspace{-0.32ex} \equiv \bm{r}^i \hspace{-0.32ex} \dotp \bm{r}^j \hspace{-0.32ex} = \smash{\textsl{g}_{i\hspace{-0.1ex}j}^{\hspace{.4ex}\expminusone}}}$ единичного \inquotes{метрического} тензора~${\bm{E} = \bm{r}_i \bm{r}^i \hspace{-0.2ex} = \bm{r}^i \bm{r}_i}$,
векторов локального взаимного кокасательного \hbox{базиса} ${\bm{r}^i \hspace{-0.32ex} \dotp \bm{r}_{\hspace{-0.2ex}j} \hspace{-0.16ex} = \delta_j^{\hspace{0.1ex}i}}$, ${\bm{r}^i \hspace{-0.25ex} = \textsl{g}^{\hspace{.25ex}i\hspace{-0.1ex}j} \bm{r}_{\hspace{-0.2ex}j}}$,
диф\-ферен\-циаль\-ного набла\hbox{-}оператора Hamilton’а ${\smash{\boldnabla \equiv \bm{r}^i \partial_i}}$,
${\bm{E} = \hspace{-0.25ex} \smash{\boldnabla \bm{r}}}$,
полного дифференциала ${d \bm{\xi} = d \bm{r} \dotp \hspace{-0.2ex} \boldnabla \hspace{-0.05ex} \bm{\xi}}$,
частных производных касательного \hbox{базиса} (вторых частных производных~$\bm{r}$) ${\bm{r}_{i\hspace{-0.1ex}j} \hspace{-0.16ex} \equiv \partial_i \partial_j \bm{r} \hspace{-0.1ex} = \partial_i \hspace{.12ex} \bm{r}_{\hspace{-0.2ex}j}}$,
символов \inquotes{связности} \hbox{Христоффеля}~(\hbox{Christoffel} symbols) ${\Gamma_{\hspace{-0.25ex}i\hspace{-0.1ex}j}^{\hspace{.25ex}k} \hspace{-0.1ex} \equiv \bm{r}_{i\hspace{-0.1ex}j} \hspace{-0.2ex} \dotp \bm{r}^k
%%\hspace{-0.32ex} = \Gamma_{\hspace{-0.25ex}i\hspace{-0.1ex}j\mathdotbelow{n}} \hspace{.25ex} \textsl{g}^{\hspace{.25ex}nk}\hspace{-0.25ex}
}$ и~${\Gamma_{\hspace{-0.25ex}i\hspace{-0.1ex}j\mathdotbelow{k}} \hspace{-0.16ex} \equiv \bm{r}_{i\hspace{-0.1ex}j} \hspace{-0.2ex} \dotp \bm{r}_k
%%\hspace{-0.2ex} = \Gamma_{\hspace{-0.25ex}i\hspace{-0.1ex}j}^{\hspace{.25ex}n} \hspace{.16ex} \textsl{g}_{nk}
}$.

Представим теперь, что функция~${\bm{r}(q^{\hspace{.1ex}k})}$ не~известна, но~\hbox{зат\'{о}} в~каждой точке пространства определены шесть независимых элементов положительной симметричной метрической матрицы Грама~${\textsl{g}_{i\hspace{-0.1ex}j}(q^{\hspace{.1ex}k})}$.

Билинейная форма

\nopagebreak ...

Поскольку шесть функций~${\textsl{g}_{i\hspace{-0.1ex}j}(q^{\hspace{.1ex}k})}$ происходят от векторной функции~${\bm{r}(q^{\hspace{.1ex}k})}$, то между элементами~$\textsl{g}_{i\hspace{-0.1ex}j}$ существуют некие соотношения.

\en{Expression}\ru{Выражение} ${d\bm{r} = d\bm{r} \dotp \hspace{-0.2ex} \tikzmark{beginItsUnitTensorE} \boldnabla \bm{r} \tikzmark{endItsUnitTensorE} = \bm{r}_k \hspace{0.2ex} dq^{\hspace{.1ex}k}}$\en{ is}\ru{\:---} \en{exact differential}\ru{полный дифференциал}. Следовательно, вторые частные производные коммутируют: ${\partial_i \hspace{.12ex} \bm{r}_{\hspace{-0.2ex}j} \hspace{-0.2ex} = \partial_j \bm{r}_i}$~(${\bm{r}_{i\hspace{-0.1ex}j} \hspace{-0.2ex} = \bm{r}_{\hspace{-0.2ex}j\hspace{-0.06ex}i}}$).
Но это необходимое условие уж\'{е} обеспечено симметрией~${\textsl{g}_{i\hspace{-0.1ex}j}}$ (\inquotes{\en{connection}\ru{связностью}}~$\nabla_{\hspace{-0.32ex}i\hspace{.1ex}}$, её~же часто называют \inquotes{\en{covariant derivative}\ru{ковариантная производная}}\:--- а~символы Христоффеля это \inquotes{\en{components of~connection}\ru{компоненты связности}} \en{in local coordinates}\ru{в~локальных координатах}).
\AddOverBrace[line width=.75pt][0,0.1ex]{beginItsUnitTensorE}{endItsUnitTensorE}{${\scriptstyle \bm{E}}$}

\vspace{1.2em}\begin{equation*}
\bm{r}_{i\hspace{-0.1ex}j} \hspace{-0.1ex} = \hspace{.1ex}
\tikzmark{beginChristoffelSymbolOne} \bm{r}_{i\hspace{-0.1ex}j} \hspace{-0.15ex} \dotp \tikzmark{beginEtensorUpDown} \bm{r}^{k} \hspace{-0.4ex} \tikzmark{endChristoffelSymbolOne} \hspace{.4ex} \bm{r}_{\hspace{-0.1ex}k} \tikzmark{endEtensorUpDown} \hspace{-0.2ex}
= \tikzmark{beginChristoffelSymbolOther} \bm{r}_{i\hspace{-0.1ex}j} \hspace{-0.15ex} \dotp \tikzmark{beginEtensorDownUp} \bm{r}_{\hspace{-0.1ex}k} \tikzmark{endChristoffelSymbolOther} \bm{r}^k \tikzmark{endEtensorDownUp}
\end{equation*}%
\AddOverBrace[line width=.75pt][0,0.2ex]{beginEtensorUpDown}{endEtensorUpDown}{${\scriptstyle \bm{E}}$}%
\AddOverBrace[line width=.75pt][0,0.2ex]{beginEtensorDownUp}{endEtensorDownUp}{${\scriptstyle \bm{E}}$}%
\AddUnderBrace[line width=.75pt][0,-0.1ex]{beginChristoffelSymbolOne}{endChristoffelSymbolOne}{${\scriptstyle \Gamma_{\hspace{-0.25ex}i\hspace{-0.1ex}j}^{\hspace{.25ex}k}}$}%
\AddUnderBrace[line width=.75pt][0,-0.1ex]{beginChristoffelSymbolOther}{endChristoffelSymbolOther}{${\scriptstyle \Gamma_{\hspace{-0.25ex}i\hspace{-0.1ex}j\mathdotbelow{k}}}$}

${\Gamma_{\hspace{-0.25ex}i\hspace{-0.1ex}j}^{\hspace{.25ex}k} \hspace{.2ex} \bm{r}_k \hspace{-0.2ex} = \bm{r}_{i\hspace{-0.1ex}j} \hspace{-0.16ex} \dotp \hspace{.1ex} \bm{r}^k \bm{r}_k \hspace{-0.2ex} = \bm{r}_{i\hspace{-0.1ex}j}}$

$\boldnabla \bm{v} \hspace{-0.16ex}
= \bm{r}^{i} \partial_i \hspace{-0.32ex} \left( v^{\hspace{.12ex}j} \bm{r}_{\hspace{-0.2ex}j} \right) \hspace{-0.25ex}
= \bm{r}^{i} \hspace{-0.32ex} \left( \partial_i v^{\hspace{.12ex}j} \bm{r}_{\hspace{-0.2ex}j} \hspace{-0.12ex} + v^{\hspace{.12ex}j} \bm{r}_{i\hspace{-0.1ex}j} \right)$

$\boldnabla \bm{v} \hspace{-0.16ex}
= \bm{r}^{i} \bm{r}_{\hspace{-0.2ex}j} \nabla_{\hspace{-0.32ex}i\hspace{.1ex}} v^{\hspace{.12ex}j} \hspace{-0.25ex} , \:\:
\nabla_{\hspace{-0.32ex}i\hspace{.1ex}} v^{\hspace{.12ex}j} \hspace{-0.2ex} \equiv
\partial_i v^{\hspace{.12ex}j} \hspace{-0.25ex} + \Gamma_{\hspace{-0.25ex}in}^{\hspace{.25ex}j} v^{\hspace{.1ex}n}$

$\boldnabla \bm{r}_i \hspace{-0.2ex}
= \bm{r}^k \partial_k \bm{r}_i \hspace{-0.2ex}
= \bm{r}^k \bm{r}_{ki} \hspace{-0.2ex}
%%= \bm{r}^k \hspace{.2ex} \Gamma_{\hspace{-0.25ex}ki}^{\hspace{.25ex}n} \hspace{.2ex} \bm{r}_n \hspace{-0.2ex}
= \bm{r}^k \bm{r}_n \hspace{.1ex} \Gamma_{\hspace{-0.25ex}ki}^{\hspace{.25ex}n}
\hspace{.2ex} , \:\:
\nabla_{\hspace{-0.32ex}i\hspace{.1ex}} \bm{r}_{n} \hspace{-0.25ex}
= \Gamma_{\hspace{-0.25ex}in}^{\hspace{.25ex}k} \hspace{.16ex} \bm{r}_k$

\vspace{.2em} Christoffel symbols describe a~metric connection, that is how the~basis changes from point to~point.

\vspace{.2em} \noindent \textcolor{cyan}{добавить:} симметрия ${ \Gamma_{\hspace{-0.25ex}i\hspace{-0.1ex}j\mathdotbelow{k}} = \Gamma_{\hspace{-0.33ex}j\hspace{-0.06ex}i\mathdotbelow{k}} }$, поэтому ${3^3 \hspace{-0.2ex} - 3 \hspace{-0.2ex}\cdot\hspace{-0.2ex} 3 = 18}$ разных~(независимых) ${\Gamma_{\hspace{-0.25ex}i\hspace{-0.1ex}j\mathdotbelow{k}}}$

\begin{multline}
\Gamma_{\hspace{-0.25ex}i\hspace{-0.1ex}j}^{\hspace{.25ex}n} \hspace{.16ex} \textsl{g}_{nk} \hspace{-0.24ex} = \Gamma_{\hspace{-0.25ex}i\hspace{-0.1ex}j\mathdotbelow{k}} \hspace{-0.2ex} = \bm{r}_{i\hspace{-0.1ex}j} \hspace{-0.2ex} \dotp \bm{r}_k \hspace{-0.1ex} = \\[-0.1em]
%
= \smalldisplaystyleonehalf \hspace{-0.2ex} \left( \bm{r}_{i\hspace{-0.1ex}j} \hspace{-0.16ex} + \bm{r}_{\hspace{-0.2ex}j\hspace{-0.06ex}i} \right) \hspace{-0.1ex} \dotp \bm{r}_k \hspace{-0.1ex}
+ \smalldisplaystyleonehalf \hspace{-0.2ex} \left( \bm{r}_{\hspace{-0.2ex}j\hspace{-0.1ex}k} \hspace{-0.16ex} - \bm{r}_{kj} \right) \hspace{-0.1ex} \dotp \bm{r}_i \hspace{-0.1ex}
+ \smalldisplaystyleonehalf \hspace{-0.2ex} \left( \bm{r}_{ik} \hspace{-0.16ex} - \bm{r}_{ki} \right) \hspace{-0.1ex} \dotp \bm{r}_{\hspace{-0.2ex}j} \hspace{-0.1ex} = \\[-0.12em]
%
= \smalldisplaystyleonehalf \hspace{-0.2ex} \left( \scalebox{0.96}[1]{$\bm{r}_{i\hspace{-0.1ex}j} \hspace{-0.2ex} \dotp \bm{r}_k \hspace{-0.16ex} + \bm{r}_{ik} \hspace{-0.2ex} \dotp \bm{r}_{\hspace{-0.2ex}j}$} \right) \hspace{-0.16ex}
+ \smalldisplaystyleonehalf \hspace{-0.2ex} \left( \scalebox{0.96}[1]{$\bm{r}_{\hspace{-0.2ex}j\hspace{-0.06ex}i} \hspace{-0.2ex} \dotp \bm{r}_k \hspace{-0.16ex} + \bm{r}_{\hspace{-0.2ex}j\hspace{-0.1ex}k} \hspace{-0.2ex} \dotp \bm{r}_i$} \right) \hspace{-0.16ex}
- \smalldisplaystyleonehalf \hspace{-0.2ex} \left( \scalebox{0.96}[1]{$\bm{r}_{ki} \hspace{-0.2ex} \dotp \bm{r}_{\hspace{-0.2ex}j} \hspace{-0.16ex} + \bm{r}_{kj} \hspace{-0.2ex} \dotp \bm{r}_i$} \right) \hspace{-0.1ex} = \\[-0.25em]
%
= \smalldisplaystyleonehalf \hspace{-0.2ex} \left(^{\mathstrut} \hspace{-0.2ex}
\partial_i ( \bm{r}_{\hspace{-0.2ex}j} \hspace{-0.2ex} \dotp \bm{r}_k ) \hspace{-0.16ex}
+ \partial_j ( \bm{r}_{i} \hspace{-0.2ex} \dotp \bm{r}_k ) \hspace{-0.16ex}
- \partial_k ( \bm{r}_{i} \hspace{-0.2ex} \dotp \bm{r}_{\hspace{-0.2ex}j} )
\hspace{-0.12ex} \right) \hspace{-0.4ex} = \\[-0.25em]
%
= \smalldisplaystyleonehalf \hspace{-0.16ex} \left(
\partial_i \hspace{.12ex} \textsl{g}_{j\hspace{-0.1ex}k} \hspace{-0.2ex}
+ \partial_j \hspace{.1ex} \textsl{g}_{ik} \hspace{-0.2ex}
- \partial_k \hspace{.12ex} \textsl{g}_{i\hspace{-0.1ex}j}
\right) \hspace{-0.4ex} .
\end{multline}

Все символы Христоффеля тождественно равны нулю лишь в~ортонормальной~(декартовой) системе. (А~какие они для косоугольной?)

Пойдём дальше: ${d\bm{r}_i \hspace{-0.2ex} = d\bm{r} \dotp \hspace{-0.2ex} \boldnabla \bm{r}_i \hspace{-0.2ex} = dq^{\hspace{.1ex}k} \partial_k \bm{r}_i \hspace{-0.2ex} = \bm{r}_{ik} \hspace{0.2ex} dq^{\hspace{.1ex}k}\hspace{-0.25ex}}$\:--- тоже полные дифференциалы.
Поэтому ${\partial_i \partial_j \bm{r}_k \hspace{-0.2ex} = \partial_j \partial_i \bm{r}_k}$~(${\partial_i \bm{r}_{\hspace{-0.2ex}j\hspace{-0.1ex}k} \hspace{-0.2ex} = \partial_j \bm{r}_{ik}}$),
и~трёхиндексный объект из~векторов третьих частных производных

\nopagebreak\vspace{-0.25em}
\begin{equation}
\bm{r}_{i\hspace{-0.1ex}j\hspace{-0.1ex}k} \hspace{-0.1ex} \equiv \hspace{.1ex} \partial_i \partial_j \partial_k \bm{r}
= \partial_i \hspace{.12ex} \bm{r}_{\hspace{-0.2ex}j\hspace{-0.1ex}k} %%\hspace{-0.4ex} = \partial_k \bm{r}_{i\hspace{-0.1ex}j}
\end{equation}

\vspace{-0.24em} \noindent симметричен по~первому и~второму индексам (а~не~только по~второму и~третьему). И~тогда равен нулю~${\hspace{-0.16ex}^4\bm{0}}$ следующий тензор четвёртой сложности\:--- тензор кривизны Римана\hbox{--}Христоффеля

\nopagebreak\vspace{-0.1em}\begin{equation}\label{riemanncurvaturetensor}
{^4\bm{\mathfrak{R}}} = \hspace{.12ex} \mathfrak{R}_{\hspace{.1ex}hi\hspace{-0.1ex}j\hspace{-0.1ex}k} \hspace{.12ex} \bm{r}^h \bm{r}^i \bm{r}^j \bm{r}^k \hspace{-0.25ex},
\:\:
\mathfrak{R}_{\hspace{.1ex}hi\hspace{-0.1ex}j\hspace{-0.1ex}k} \hspace{-0.12ex} \equiv
\bm{r}_h \hspace{-0.15ex} \dotp \left( \hspace{.12ex} \bm{r}_{\hspace{-0.2ex}j\hspace{-0.06ex}ik} \hspace{-0.2ex} - \bm{r}_{i\hspace{-0.1ex}j\hspace{-0.1ex}k} \hspace{.12ex} \right)
\hspace{-0.3ex} .
\end{equation}

Выразим компоненты~${\mathfrak{R}_{\hspace{.1ex}i\hspace{-0.1ex}j\hspace{-0.1ex}kn}}$ через метрическую матрицу~${\textsl{g}_{i\hspace{-0.1ex}j}}$. Начнём с~дифференцирования локального кобазиса:
\[ \bm{r}^i \hspace{-0.32ex} \dotp \bm{r}_k \hspace{-0.16ex} = \delta_k^{\hspace{0.1ex}i} \;\Rightarrow\:
\partial_j \bm{r}^i \hspace{-0.32ex} \dotp \bm{r}_k + \hspace{.1ex} \bm{r}^i \hspace{-0.32ex} \dotp \bm{r}_{\hspace{-0.2ex}j\hspace{-0.1ex}k} = 0 \;\Rightarrow\:
\partial_j \bm{r}^i \hspace{-0.12ex} = - \hspace{.2ex} \Gamma_{\hspace{-0.25ex}j\hspace{-0.1ex}k}^{\hspace{.25ex}i} \hspace{.2ex} \bm{r}^k
\hspace{-0.4ex}. \]

...

Шесть независимых компонент:
${\mathfrak{R}_{\hspace{.1ex}1212}}$, ${\mathfrak{R}_{\hspace{.1ex}1213}}$, ${\mathfrak{R}_{\hspace{.1ex}1223}}$, ${\mathfrak{R}_{\hspace{.1ex}1313}}$, ${\mathfrak{R}_{\hspace{.1ex}1323}}$, ${\mathfrak{R}_{\hspace{.1ex}2323}}$.

...

\en{Symmetric}\ru{Симметричный} \en{bivalent}\ru{бивалентный} \ru{тензор кривизны }Ricci\en{ curvature tensor}

\begin{equation*}
\hspace{.1ex}\pmb{\scalebox{1.2}[1]{$\mathscr{R}$}} \equiv
\smalldisplaystyleonefourth \hspace{.4ex} \mathfrak{R}_{\hspace{.1ex}abi\hspace{-0.1ex}j} \hspace{.2ex} \bm{r}^a \hspace{-0.3ex} \times \hspace{-0.15ex} \bm{r}^b \bm{r}^i \hspace{-0.3ex} \times \hspace{-0.15ex} \bm{r}^j \hspace{-0.25ex}
= \smalldisplaystyleonefourth \hspace{.15ex} \levicivita^{abp} \levicivita^{i\hspace{-0.1ex}j\hspace{-0.1ex}q} \hspace{.25ex} \mathfrak{R}_{\hspace{.1ex}abi\hspace{-0.1ex}j} \hspace{.2ex} \bm{r}_{\hspace{-0.2ex}p} \bm{r}_{\hspace{-0.2ex}q} \hspace{-0.2ex}
= \mathscr{R}^{\hspace{.1ex}pq} \hspace{.1ex} \bm{r}_{\hspace{-0.2ex}p} \bm{r}_{\hspace{-0.2ex}q}
\end{equation*}

\vspace{-0.2em} \noindent (\en{coefficient}\ru{коэффициент}~$\onefourth$ \en{is used here for convenience}\ru{используется тут для удобства}) \en{with components}\ru{с~компонентами}

\vspace{.1em}\begin{equation*}
\begin{array}{ccc}
\mathscr{R}^{\hspace{.1ex}1\hspace{-0.12ex}1} \hspace{-0.3ex} =
\scalebox{0.8}{$ \displaystyle \frac{\raisemath{-0.2em}{1}}{\raisemath{.15em}{\smash{\textsl{g}}}} $} \hspace{.4ex} \mathfrak{R}_{\hspace{.1ex}2323}
\hspace{.2ex} ,
&
&
\\[.6em]
%
\mathscr{R}^{\hspace{.1ex}21} \hspace{-0.3ex} =
\scalebox{0.8}{$ \displaystyle \frac{\raisemath{-0.2em}{1}}{\raisemath{.15em}{\smash{\textsl{g}}}} $} \hspace{.4ex} \mathfrak{R}_{\hspace{.1ex}1323}
\hspace{.2ex} ,
&
\mathscr{R}^{\hspace{.1ex}22} \hspace{-0.3ex} =
\scalebox{0.8}{$ \displaystyle \frac{\raisemath{-0.2em}{1}}{\raisemath{.15em}{\smash{\textsl{g}}}} $} \hspace{.4ex} \mathfrak{R}_{\hspace{.1ex}1313}
\hspace{.2ex} ,
&
\\[.6em]
%
\mathscr{R}^{\hspace{.1ex}31} \hspace{-0.3ex} =
\scalebox{0.8}{$ \displaystyle \frac{\raisemath{-0.2em}{1}}{\raisemath{.15em}{\smash{\textsl{g}}}} $} \hspace{.4ex} \mathfrak{R}_{\hspace{.1ex}1223}
\hspace{.2ex} ,
&
\mathscr{R}^{\hspace{.1ex}32} \hspace{-0.3ex} =
\scalebox{0.8}{$ \displaystyle \frac{\raisemath{-0.2em}{1}}{\raisemath{.15em}{\smash{\textsl{g}}}} $} \hspace{.4ex} \mathfrak{R}_{\hspace{.1ex}1213}
\hspace{.2ex} ,
&
\mathscr{R}^{\hspace{.1ex}33} \hspace{-0.3ex} =
\scalebox{0.8}{$ \displaystyle \frac{\raisemath{-0.2em}{1}}{\raisemath{.15em}{\smash{\textsl{g}}}} $} \hspace{.4ex} \mathfrak{R}_{\hspace{.1ex}1212}
\hspace{.2ex} .
\end{array}
\end{equation*}

Равенство тензора Риччи нулю
${\hspace{.1ex}\pmb{\scalebox{1.2}[1]{$\mathscr{R}$}} \hspace{-0.16ex} = \hspace{-0.2ex} {^2\bm{0}}}$ (в~компонентах это шесть уравнений ${\hspace{.1ex}\mathscr{R}^{\hspace{.1ex}i\hspace{-0.1ex}j} \hspace{-0.3ex} = \mathscr{R}^{\hspace{.1ex}j\hspace{-0.06ex}i} \hspace{-0.3ex} = 0}$) \en{is}\ru{есть} \en{the~necessary condition}\ru{необходимое условие} \en{of~integrability}\ru{интегрируемости}~(\ru{\inquotes{совместности}, }\inquotes{compatibility}) для нахождения вектора-радиуса~${\bm{r}(q^{\hspace{.1ex}k})}$ по~полю~${\textsl{g}_{i\hspace{-0.1ex}j}(q^{\hspace{.1ex}k})}$.

\end{otherlanguage}

\section*{\small \wordforbibliography}

\begin{changemargin}{\parindent}{0pt}
\fontsize{10}{12}\selectfont

\begin{otherlanguage}{russian}

\en{There’re many books}\ru{Существует много книг}, \en{where only the~apparatus of~tensor calculus is described}\ru{где описывается только аппарат тензорного исчисления}~\cite{mcconnell-tensoranalysis, schouten-tensoranalysis, sokolnikoff-tensoranalysis, borisenko.tarapov, rashevsky-riemanniangeometry}. \en{However}\ru{Однако}, \inquotes{\en{the~index approach}\ru{индексный подход}} \en{prevails}\ru{преобладает}\:--- \en{tensors are treated as matrices of~components}\ru{тензоры трактуются как матрицы компонент}, \en{transforming in a~known way}\ru{преобразующиеся известным путём}.
\inquotes{\en{The direct approach}\ru{Прямой подход}} \en{to~tensors}\ru{к~тензорам} \en{is explicated}\ru{излагается}, \en{for instance}\ru{например}, в~приложениях книг А.\,И.\;Лурье~\cite{lurie-nonlinearelasticity, lurie-theoryofelasticity}.
Яркое описание теории векторных полей можно найти у~R.\:Feynman’а~\cite{feynman-lecturesonphysics}.
Сведения о~тензорном исчислении содержатся и~в~своеобразной и~глубокой книге C.\:Truesdell’а~\cite{truesdell-firstcourse}.

\end{otherlanguage}

\end{changemargin}
