\en{\chapter{Elements of tensor calculus}}

\ru{\chapter{Элементы тензорного исчисления}}

\thispagestyle{empty}

\label{chapter:elementsoftensorcalculus}

\en{\section{Vectors}}

\ru{\section{Векторы}}

\label{para:vectors}

\en{\lettrine[lines=2, findent=2pt, nindent=0pt]{M}{ention} of~tensors may scare away the reader, commonly avoiding needless complications. But tensors are introduced just because of their wonderful property of being invariant~--- independent of coordinate systems. I propose to~begin familiarizing with tensors from reminiscences of~vectors.}

\ru{\lettrine[lines=2, findent=2pt, nindent=0pt]{У}{поминание} о~тензорах может отпугнуть читателя, обычно избегающего ненужных сложностей. Но тензоры вводятся лишь из\hbox{-}за своего чудесного свойства инвариантности~--- независимости от систем координат. Знакомство с~тензорами предлагаю начать с~воспоминаний о~векторах.}

\en{Meet ${\bm{v}}$~--- a~vector; it is fully characterized by its length~(modulus, magnitude) and direction in space, but does not depend on units and methods of measurement neither lengths nor directions.}

\ru{Пусть ${\bm{v}}$~--- вектор; он вполне характеризуется своей длиной~(модулем) и~направлением в~пространстве, но не~зависит от~единиц и~методов измерений ни~длин, ни~направлений.}

%%\begin{center}
\begin{wrapfigure}[12]{o}{0.47\textwidth}
\makebox[0.47\textwidth][c]{\begin{minipage}[t]{0.47\textwidth}
\vspace{-1em}
\hspace{-2em}\scalebox{0.96}[0.96]{
\tdplotsetmaincoords{45}{125} % set orientation of axes
% three parameters for vector
\pgfmathsetmacro{\lengthofvector}{2.5}
\pgfmathsetmacro{\anglefromz}{44}
\pgfmathsetmacro{\anglefromx}{70}

\begin{tikzpicture}[scale=2.5, tdplot_main_coords] % tdplot_main_coords style to use 3dplot

	\coordinate (O) at (0,0,0);

	% draw axes
	\draw [line width=0.4pt, blue] (O) -- (1.22,0,0);
	\draw [line width=1.25pt, blue, -{Latex[round, length=3.6mm, width=2.4mm]}]
		(O) -- (1,0,0)
		node[pos=0.9, above, xshift=-0.6em, yshift=0.1em] {${\bm{e}}_1$};

	\draw [line width=0.4pt, blue] (O) -- (0,1.88,0);
	\draw [line width=1.25pt, blue, -{Latex[round, length=3.6mm, width=2.4mm]}]
		(O) -- (0,1,0)
		node[pos=0.9, above, xshift=0.2em, yshift=0.1em] {${\bm{e}}_2$};

	\draw [line width=0.4pt, blue] (O) -- (0,0,1.22);
	\draw [line width=1.25pt, blue, -{Latex[round, length=3.6mm, width=2.4mm]}]
		(O) -- (0,0,1)
		node[pos=0.9, below left, xshift=-0.1em, yshift=0.5em] {${\bm{e}}_3$};

	% draw vector
	\tdplotsetcoord{V}{\lengthofvector}{\anglefromz}{\anglefromx} % {length}{angle from z}{angle from x}
		% it also defines projections of point
	\draw [line width=1.6pt, black, -{Stealth[round, length=5mm, width=2.8mm]}]
		(O) -- (V)
		node[pos=0.8, above, yshift=0.2em] {\scalebox{1.2}[1.2]{${\bm{v}}$}};

	% draw components of vector
	\draw [line width=0.4pt, dotted, color=black] (O) -- (Vxy);
	%%\draw [line width=0.4pt, dotted, color=black] (Vxy) -- (V);
	%%\draw [line width=0.4pt, dotted, color=black] (Vxy) -- (Vx);
	\draw [line width=0.4pt, dotted, color=black] (Vxy) -- (Vy);

	\draw [color=black, line width=1.6pt, line cap=round, dash pattern=on 0pt off 1.6\pgflinewidth,
		-{Stealth[round, length=4mm, width=2.4mm]}]
		(O) -- (Vx)
		node[pos=0.5, above, xshift=-0.9em] {${v_1 \hspace{-0.1ex} \bm{e}_1}$};

	\draw [color=black, line width=1.6pt, line cap=round, dash pattern=on 0pt off 1.6\pgflinewidth,
		-{Stealth[round, length=4mm, width=2.4mm]}]
		(Vx) -- (Vxy)
		node[pos=0.53, below left, xshift=0.3em] {${v_2 \bm{e}_2}$};

	\draw [color=black, line width=1.6pt, line cap=round, dash pattern=on 0pt off 1.6\pgflinewidth,
		-{Stealth[round, length=4mm, width=2.4mm]}]
		(Vxy) -- (V)
		node[pos=0.5, above right, yshift=-0.2em] {${v_3 \bm{e}_3}$};

\end{tikzpicture}}
\vspace{-1.72em}\caption{}\label{fig:componentsofvector}
\end{minipage}}
\end{wrapfigure}
%%\end{center}

\en{Introduce some cartesian coordinate system with mutually perpendicular unit vectors of basis ${\bm{e}_1}$,\;${\bm{e}_2}$,\;${\bm{e}_3}$.
In such \hbox{a system}, dot products of basis vectors are equal to the~Kronecker~delta:}

\ru{Введём некую декартову систему координат со~взаимно ортогональными ортами базиса ${\bm{e}_1}$,\;${\bm{e}_2}$,\;${\bm{e}_3}$.
В~такой \hbox{системе} произведения базисных векторов равны \hbox{дельте}~Kronecker’а:}

\nopagebreak\[ \bm{e}_i \dotp \bm{e}_j = \hspace{.12ex} \delta_{ij} = \hspace{.12ex} \scalebox{0.92}[0.92]{$\begin{cases}
1, \;i = j \\
0, \;i \neq j
\end{cases}$} \]

\en{Decomposing vector~${\bm{v}}$ by basis ${\bm{e}_i}$~(${i = 1, 2, 3}$), we have components of~vector~(\figref{fig:componentsofvector})}

\ru{Разлагая вектор~${\bm{v}}$ по базису ${\bm{e}_i}$~(${i = 1, 2, 3}$), имеем компоненты вектора~(\figref{fig:componentsofvector})}

\nopagebreak\vspace{-0.2em}\begin{equation}\label{vectorcomponents}
\bm{v} = \scalebox{0.9}[0.9]{$\displaystyle \sum_{i=1}^3$} {v_i \bm{e}_i} \hspace{-0.1ex} \equiv \hspace{.1ex} v_i \bm{e}_i \hspace{0.1ex}, \;\;
v_i \hspace{-0.1ex} = \bm{v} \dotp \bm{e}_i \hspace{-0.16ex} = |\bm{v}| \operatorname{cos} \measuredangle ({\bm{v}, \bm{e}_i}).
\end{equation}

{\small\setlength{\abovedisplayskip}{1pt}\setlength{\belowdisplayskip}{1pt}

\en{Here and hereinafter, Einstein’s summation convention is accepted: having an~index repeated twice in a~single term implies summation over this index. More than two times in~the~same term index can’t be repeated. And a~non-repeating index is called free, it’s identical in both parts of~equation. There are examples:}

\ru{Здесь и~далее принимается соглашение о~суммировании~Einstein’а: наличие дважды повторённого индекса в~одночлене подразумевает суммирование по~этому индексу. Более двух раз в~том~же одночлене индекс повторяться не~может. А~неповторяющийся индекс называется свободным, он одинаков в~обеих частях равенства. Вот примеры:}

\nopagebreak\vspace{-0.2em}\[
\sigma = \tau_{ii} \hspace{0.1ex}, \;\;
p_j \hspace{-0.16ex} = n_i \tau_{ij} \hspace{0.1ex}, \;\;
m_i \hspace{-0.16ex} = e_{ijk} \hspace{.16ex} x_j f_k \hspace{0.1ex}, \;\;
a_i \hspace{-0.16ex} = \lambda b_i + \mu c_i \hspace{0.1ex}.
\]

\vspace{-0.2ex} \noindent \en{But following equations are incorrect}\ru{Следующие~же равенства некорректны}

\nopagebreak\vspace{-0.2em}\[
a = b_{kkk} \hspace{0.1ex}, \;\;
c = f_i + g_k \hspace{0.1ex}, \;\;
a_{ij} \hspace{-0.25ex} = k_i \gamma_{ij} \hspace{0.1ex}.
\]
\par}
\vspace{.1em}

\en{Vector~${\bm{v}}$ is invariant, it doesn’t depend on any coordinate system. Its decomposition in two cartesian systems with basis unit vectors ${\bm{e}_{i}}$ and~${\bm{e}'_{i}}$ gives}

\ru{Вектор~${\bm{v}}$ инвариантен, он не~зависит ни~от~какой системы координат. Его разложение в~двух декартовых системах с~базисными ортами ${\bm{e}_{i}}$ и~${\bm{e}'_{i}}$ даёт}

\nopagebreak\vspace{-0.8em}\begin{equation*}
\begin{array}{c}
\bm{v} = v_{i} \hspace{.1ex} \bm{e}_{i} \hspace{-0.16ex} = v'_{i} \hspace{.1ex} \bm{e}'_{i} \hspace{.1ex}, \\[.2em]
%
\bm{v} \dotp \bm{e}_k = \hspace{-0.2ex} \tikzmark{beginComponentOfVector} \,v_k\, \tikzmark{endComponentOfVector} \hspace{-0.4ex} = v'_{i} \hspace{.2ex} \bm{e}'_{i} \dotp \bm{e}_k \hspace{.1ex}, \;\;
\bm{v} \dotp \bm{e}'_{k} = \hspace{.12ex} v'_{k} = v_i \hspace{.2ex} \bm{e}_i \dotp \bm{e}'_{k} \hspace{.1ex}.
\end{array}
\vspace{.5em}\end{equation*}
\AddUnderBrace[line width=.75pt][0,0][yshift=.12em]%
{beginComponentOfVector}{endComponentOfVector}%
{${\scriptstyle v_i \bm{e}_i \hspace{0.2ex}\dotp\hspace{0.32ex} \bm{e}_k \hspace{.2ex}=\hspace{.4ex} v_i \hspace{.1ex} \delta_{ik}}$}

\en{Connection between basis vectors of an~\inquotes{old} and~a~\inquotesx{new}[---] \inquotesx{rotated}[---] orthonormal systems is represented by a~rotation matrix (a~matrix of \inquotes{direction} cosines)~${\cosinematrix{i'k}\hspace{.1ex}}$:}

\ru{Связь между базисными векторами \inquotes{старой} и~\inquotesx{новой}[---] \inquotesx{повёрнутой}[---] ортонормальных систем представима матрицей поворота (матрицей \inquotes{направляющих} косинусов)~${\cosinematrix{i'k}\hspace{.1ex}}$:}

\nopagebreak\vspace{-0.25em}\begin{equation}
\bm{e}'_{i} \hspace{-0.16ex} = \cosinematrix{i'k} \hspace{.1ex} \bm{e}_k \hspace{0.1ex}, \;\:
(\hspace{0.32ex} \dotp \hspace{.25ex} \bm{e}_k \hspace{0.25ex}) \,\Rightarrow\: \cosinematrix{i'k} \hspace{-0.16ex} = \bm{e}'_{i} \dotp \bm{e}_k \hspace{-0.16ex} = \operatorname{cos} \measuredangle (\bm{e}'_{i} \hspace{.1ex}, \bm{e}_k).
\end{equation}

\en{\vspace{-0.1em} Matrix of cosines is orthogonal, meaning that it inverses when transposing:}

\ru{\vspace{-0.1em} Матрица косинусов ортогональна, то~есть при~транспонировании она обращается:}

\vspace{-0.64em}\begin{equation}\label{orthogonalityofcosinematricies}
\cosinematrix{i'k} \hspace{.1ex} \cosinematrix{j'k} \hspace{-0.16ex} = \hspace{0.2ex} \cosinematrix{k'i} \hspace{.1ex} \cosinematrix{k'j} \hspace{-0.16ex} = \hspace{0.2ex} \delta_{ij}
\end{equation}

\vspace{-0.2em} \noindent (\en{proof}\ru{доказательство}: $\delta_{ij} \hspace{-0.16ex} = \bm{e}'_i \hspace{-0.12ex} \dotp \bm{e}'_j \hspace{-0.16ex} = \cosinematrix{i'k} \hspace{.1ex} \bm{e}_k \hspace{-0.12ex} \dotp \cosinematrix{j'n} \hspace{.1ex} \bm{e}_n \hspace{-0.16ex} = \cosinematrix{i'k} \hspace{.1ex} \cosinematrix{j'n} \delta_{kn} \hspace{-0.16ex} = \cosinematrix{i'k} \hspace{.1ex} \cosinematrix{j'k} \hspace{0.1ex}$).

\en{\vspace{.2em} Orthogonal transformation of vector components}

\ru{\vspace{.2em} Ортогональное преобразование компонент вектора}

\nopagebreak\vspace{-0.25em}\begin{equation}\label{orthotransform:1}
v'_i = v_k \hspace{.2ex} \bm{e}_k \hspace{-0.12ex}\dotp \bm{e}'_i \hspace{-0.12ex} = \cosinematrix{i'k} \hspace{.2ex} v_k
\end{equation}

\en{\vspace{-0.1em}\noindent can be used to define a~vector itself. Suppose that in each orthonormal basis~${\bm{e}_i}$ a~triplet of~numbers~${v_i}$ is given, and on~transition to a~new orthonormal basis it is transformed according to~\eqref{orthotransform:1}; then this triplet of components represents an~invariant object~--- vector~${\bm{v}}$.}

\ru{\vspace{-0.2em}\noindent может быть использовано для определения самог\'{о} вектора. Пусть в~каждом ортонормированном базисе~${\bm{e}_i}$ даётся тройка чисел~${v_i}$, и~при~переходе к~новому ортонормированному базису она преобразуется согласно~\eqref{orthotransform:1}; тогда эта тройка компонент представляет инвариантный объект~--- вектор~${\bm{v}}$.}

\en{Multiplying an~orthogonal matrix by components of any vector retains length~(modulus) of this vector:}

\ru{Умножение ортогональной матрицы на~компоненты любого вектора сохраняет длину~(модуль) этого вектора:}

\nopagebreak \vspace{-0.2em} ${|\bm{v}|^2 \hspace{-0.1ex} = \bm{v} \dotp \bm{v} = v'_i \hspace{.2ex} v'_i \hspace{-0.16ex} = \cosinematrix{i'k} \hspace{.1ex} v_k \hspace{.2ex} \cosinematrix{i'n} \hspace{.1ex} v_n \hspace{-0.16ex} = v_n v_n \hspace{-0.2ex}}$~\en{--- this conclusion leans on}\ru{--- этот вывод опирается на}~\eqref{orthogonalityofcosinematricies}.

\en{\section{Tensor and its components}}

\ru{\section{Тензор и его компоненты}}

\label{para:tensoranditscomponents}

\en{When in each orthonormal basis~${\bm{e}_i}$ we have a~set of~nine (${3^2 \hspace{-0.24ex}=\hspace{-0.12ex} 9}$) numbers ${B_{ij}}$~(${i, j = 1, 2, 3}$), which is transformed during transition to~a~new~(rotated) orthonormal basis as}

\ru{Когда в~каждом ортонормальном базисе~${\bm{e}_i}$ имеем совокупность девяти (${3^2 \hspace{-0.24ex}=\hspace{-0.12ex} 9}$) чисел ${B_{ij}}$~(${i, j = 1, 2, 3}$), преобразующуюся при~переходе к~новому~(повёрнутому) ортонормальному базису как}

\nopagebreak\vspace{-0.2em}\begin{equation}\label{orthotransform:2}
B'_{ij} \hspace{-0.16ex} = \cosinematrix{i'k} \hspace{.1ex} \cosinematrix{j'n} \hspace{.16ex} B_{kn} \hspace{0.1ex},
\end{equation}

\vspace{-0.2em} \noindent
\en{then this set of~components represents an~invariant object~--- a~tensor of~second complexity (of~second valence, bivalent)}\ru{тогда эта совокупность компонент представляет инвариантный объект~--- тензор второй сложности (второй валентности, бивалентный)}
${^2\!\bm{B}}$.

%% Using some orthonormal basis to represent a tensor as set of components. Conversion of tensor’s components from one cartesian basis to another is done through an orthogonal transformation.

\en{In other words, tensor~${^2\!\bm{B}}$ reveals in each basis as a~set~(matrix) of its components~${B_{ij}}$, changing along with a~basis according to~\eqref{orthotransform:2}. Attempts, widespread in literature, to~replace tensors with matrices lead to mistakes if you don’t track a~basis to which these matrices correspond.}

\ru{Иными словами, тензор~${^2\!\bm{B}}$ проявляется в~каждом базисе набором~(матрицей) своих компонент~${B_{ij}}$, меняющейся вместе с~базисом согласно~\eqref{orthotransform:2}. Распространённые в~литературе попытки заменить тензоры матрицами ведут к~ошибкам, если не~следить за~базисом, которому эти матрицы соответствуют.}

\begin{otherlanguage}{russian}

Ключевой пример тензора второй сложности~--- диада. Пусть ${\bm{a}}$ и~${\bm{b}}$~--- векторы. В~каждом базисе положим ${D_{ij} \hspace{-0.2ex} \equiv a_i b_j}$. Легко убедиться, что преобразование~\eqref{orthotransform:2} для~${D_{ij}}$ справедливо. Получившийся тензор ${^2\!\!\,\bm{D}}$ называется диадным произведением~(dyadic product) или просто диадой~(dyad); его пишут как~${\bm{a} \otimes \bm{b}}$ или~${\bm{a} \bm{b}}$. Предпочтём второе.

Ещё~один существенный пример двухвалентного тензора~--- единичный тензор~(он~же~\inquotes{метрический}). Для~любого декартова~(cartesian) базиса положим ${E_{ij} \hspace{-0.2ex} \equiv \hspace{0.12ex} \delta_{ij} \hspace{-0.2ex} = \bm{e}_i \dotp \bm{e}_j}$. Это действительно компоненты тензора, \eqref{orthotransform:2} действует. Назовём этот тензор ${\bm{E}}$.

Неизменность компонент при~повороте базиса делает тензор~${\bm{E}}$ изотропным. Ненулевых векторов с~таким свойством нет (все компоненты нуль\hbox{-}вектора~$\bm{0}$ равны нулю в~любом базисе).

Третий пример связан с~линейным преобразованием векторов: ${\bm{b}}$ есть линейная функция от ${\bm{a}}$. В~каждом базисе имеем ${b_i = c_{ij} a_j}$. Коэффициенты преобразования меняются при~перемене базиса:
\[
\scalebox{0.96}[1]{$b\hspace{0.16ex}'_{\hspace{-0.16ex}i} = c\hspace{0.16ex}'_{\hspace{-0.16ex}ij} a'_j = \cosinematrix{i'k} b_k = \cosinematrix{i'k} c_{kn} a_n ,\;
a_n \hspace{-0.25ex} = \cosinematrix{j'n} a'_j \;\Rightarrow\; c\hspace{0.16ex}'_{\hspace{-0.16ex}ij} \hspace{-0.2ex} = \cosinematrix{i'k} \cosinematrix{j'n} c_{kn}$}.
\]
\noindent Видим, что множество матриц ${c_{ij}}$, ${c\hspace{0.16ex}'_{\hspace{-0.16ex}ij}}$, \dots, определяющих одно и~то~же линейное преобразование ${\bm{a}}$ в~${\bm{b}}$, но в~разных базисах, сводится к~одному инвариантному объекту~--- тензору второй сложности ${\hspace{-0.1ex} ^2\hspace{-0.2ex}\bm{c}}$. Многие авторы так и~определяют тензор через линейное преобразование.

Четвёртый пример~--- коэффициенты инвариантной квадратичной формы ${\digamma(\bm{a}) = f_{\hspace{-0.2ex}ij} \hspace{.2ex} a_i a_j}$, где ${a_i}$ и~${a_j}$~--- компоненты векторного аргумента~${\bm{a}}$, и~результат~${\digamma}$ не~зависит от базиса. Из равенств ${\digamma' = f'_{\hspace{-0.2ex}ij} \hspace{.2ex} a'_i a'_j = \digamma}$ следует~\eqref{orthotransform:2} для~коэффициентов ${f_{\hspace{-0.2ex}ij}}$.

Обратимся теперь к тензорам б\'{о}льших сложностей. Тензор третьей валентности ${^3\!\!\;\bm{C}}$ представляется через совокупность ${3^3 \hspace{-0.24ex}=\hspace{-0.12ex} 27}$ чисел~${C_{ijk}}$, преобразующихся как

\nopagebreak\vspace{-0.1em}\begin{equation}\label{orthotransform:3}
C\hspace{0.16ex}'_{\hspace{-0.32ex}ijk}
= \cosinematrix{i'p} \hspace{.1ex} \cosinematrix{j'q} \hspace{.1ex} \cosinematrix{k'r} \hspace{.16ex} C_{pqr}
\hspace{.1ex} .
\end{equation}

% triad

Простейший пример~--- триада из векторов $\bm{a}$, $\bm{b}$ и~$\bm{c}$:
\[ C_{ijk} \equiv a_i b_j c_k \:\Leftrightarrow\: {^3\!\!\;\bm{C}} = \bm{a} \bm{b} \bm{c} \hspace{0.1ex}. \]

Отметим, что ортогональные преобразования~\eqref{orthotransform:3} и~\eqref{orthotransform:2}~--- результаты своеобразного \inquotes{повторения} векторного~\eqref{orthotransform:1}. Читатель без труда напишет преобразование компонент тензора любой сложности и~для~примера приведёт соответствующую полиаду.

Векторы с~пребразованием компонент~\eqref{orthotransform:1} суть тензоры первой сложности.

В~конце этого параграфа обратимся к~самым простым объектам~--- скалярам, то~есть тензорам нулевой сложности. Скаляр это одно ${(3^0 \hspace{-0.24ex}=\hspace{-0.12ex} 1)}$ число, не~зависящее от~базиса: масса, энергия, температура и~др. Но что такое, например, компоненты~\eqref{vectorcomponents} вектора ${\bm{v} = v_i \bm{e}_i}$ и~${v_i = \bm{v} \dotp \bm{e}_i}$? Если не~скаляры, то~что? Односложно ответить невозможно. В~каждом фиксированном базисе ${\bm{e}_i}$, конечно, векторы, а~${v_i}$~--- скаляры.

\end{otherlanguage}

\en{\section{Operations with tensors}}

\ru{\section{Действия с тензорами}}

\label{para:operationswithtensors}

\en{There’re four of them.}

\ru{Таких действий четыре.}

\en{The first~--- linear combination~--- aggregates addition and~multiplication by~number. Arguments of this operation and its result are of the same complexity. Combining two tensors looks like this:}

\ru{Первое~--- линейная комбинация~--- объединяет сложение и~умножение на~число. Аргументы этого действия и~его результат~--- одинаковой сложности. Комбинирование двух тензоров выглядит так:}

\nopagebreak\vspace{-0.2em}\begin{equation}\label{action:1}
\lambda a_{ij\ldots} \hspace{-0.2ex} + \hspace{0.2ex} \mu b_{ij\ldots} \hspace{-0.2ex} = \hspace{0.2ex} c_{ij\ldots} \hspace{-0.2ex} \;\Leftrightarrow\; \lambda \bm{a} + \mu \bm{b} = \bm{c}.
\end{equation}

\vspace{-0.1em} \noindent \en{Here}\ru{Здесь}
$\lambda$ \en{and}\ru{и}~$\mu$~\en{are}\ru{---} \en{scalar coefficients}\ru{коэффициенты\hbox{-}скаляры};
$\bm{a}$, $\bm{b}$ \en{and}\ru{и}~$\bm{c}$~\en{are}\ru{---} \en{tensors of same complexity}\ru{тензоры одной и~той~же сложности}.
\en{It’s easy to show}\ru{Легко показать,} \en{that}\ru{что} \en{components}\ru{компоненты} \en{of~the~result}\ru{результата}~$\bm{c}$ \en{satisfy}\ru{удовлетворяют} \en{orthogonal transformation}\ru{ортогональному преобразованию} \en{like}\ru{вида}~\eqref{orthotransform:2}.

\begin{otherlanguage}{russian}

\inquotesx{Разложение вектора по~базису}[---] представление вектора суммой~${\bm{v} = v_i \bm{e}_i}$~--- это не~что~иное, как линейная комбинация векторов базиса~${\bm{e}_i}$ с~коэффициентами~${v_i}$.

Второе действие это умножение. Сложности сомножителей~--- любые, сложность произведения~--- суммарная. Примеры:
\begin{equation}\label{action:2}
\begin{array}{rcl}
v_i a_{jk} \hspace{-0.16ex} = C_{ijk} & \Leftrightarrow & \bm{v} \, {^2\hspace{-0.2ex}\bm{a}} = {^3\!\!\;\bm{C}},\\[0.1em]
a_{ij} B_{kem} \hspace{-0.16ex} = D_{ijkem} & \Leftrightarrow & {^2\hspace{-0.2ex}\bm{a}} \, {^3\!\!\,\bm{B}} = {^5\!\!\,\bm{D}}.
\end{array}\hspace{1.5em}
\end{equation}

\vspace{-0.1em} Рассмотрев преобразование совокупностей ${C_{ijk}}$ и~${D_{ijkem}}$ при~повороте базиса, видим, что это действительно компоненты тензоров. Простейший пример умножения~--- диадное произведение векторов ${^2\hspace{-0.2em}\bm{A} \hspace{-0.15ex} = \bm{b}\bm{c}}$.

Третье действие называется свёрткой~(contraction). Это действие над одним тензором, других \inquotes{участников} нет. Грубо говоря, свёртка состоит в~суммировании компонент по какой\hbox{-}либо паре индексов. В~результате свёртки сложность тензора уменьшается на~два. Для~тензора, например, третьей сложности~${^3\!\!\,\bm{D}}$ возможны следующие варианты свёртки, приводящие к~векторам~${\bm{a}}$, ${\bm{b}}$ и~${\bm{c}}$:
\begin{equation}\label{action:3}
a_{i} = D_{kki},\;\; b_{i} = D_{kik},\;\; c_{i} = D_{ikk} \hspace{0.1ex}.
\end{equation}

\vspace{-0.16em} При повороте базиса
\[
a'_{i} = D'_{kki} = \tikzmark{BeginDeltaPQBrace} {\cosinematrix{k'p} \hspace{.1ex} \cosinematrix{k'q}} \tikzmark{EndDeltaPQBrace} \hspace{.1ex} \cosinematrix{i'r} \hspace{0.16ex} D_{pqr} = \cosinematrix{i'r} \hspace{.16ex} D_{ppr} = \cosinematrix{i'r} \hspace{0.16ex} a_{r},
\]
\AddUnderBrace[line width=.75pt][0,-0.2ex]{BeginDeltaPQBrace}{EndDeltaPQBrace}%
{${\scriptstyle \delta_{pq}}$}

\vspace{-0.25em} \noindent что доказывает тензорный характер результата свёртки.

Для тензора второй сложности возможен лишь один вариант свёртки, приводящий к~скаляру, называемому первым инвариантом или~следом тензора
\nopagebreak\vspace{-0.16em}\[
B_{kk} \hspace{-0.2ex} = \mathrm{I}\hspace{0.16ex}(\bm{B}) \equiv \operatorname{tr} \bm{B}.
\]

\vspace{-0.16em} След единичного тензора (\inquotes{свёртка дельты Кронекера}) есть размерность пространства:

\nopagebreak\vspace{-0.2em}\begin{equation*}
\operatorname{tr} \bm{E} = \delta_{kk} \hspace{-0.2ex} = \delta_{11} \hspace{-0.2ex} + \delta_{22} \hspace{-0.2ex} + \delta_{33} \hspace{-0.1ex} = 3.
\end{equation*}

Четвёртое действие именуется по\hbox{-}разному: перестановка индексов, жонглирование индексами, index swap и~др. Из компонент тензора образуется новая совокупность величин с~другой последовательностью индексов, результатом является тензор той~же сложности. Из тензора~${^3\!\!\,\bm{D}}$, например, можно получить тензоры ${^3\!\!\bm{A}}$, ${^3\!\!\,\bm{B}}$, ${^3\!\!\;\bm{C}}$ с~компонентами
\vspace{0.08em}\begin{equation}\label{action:4}
\begin{array}{rcl}
{^3\hspace{-0.25em}\bm{A}} = {^3\hspace{-0.16em}\bm{D}}_{1 \scalebox{0.6}[0.8]{$\rightleftarrows$} 2} & \!\Leftrightarrow\!\! &
A_{ijk} = D_{jik} \hspace{.1ex},
\\
{^3\hspace{-0.16em}\bm{B}} = {^3\hspace{-0.16em}\bm{D}}_{1 \scalebox{0.6}[0.8]{$\rightleftarrows$} 3} & \!\Leftrightarrow\!\! & B_{ijk} = D_{kji} \hspace{.1ex},
\\
{^3\hspace{-0.04em}\bm{C}} = {^3\hspace{-0.16em}\bm{D}}_{2 \scalebox{0.6}[0.8]{$\rightleftarrows$} 3} & \!\Leftrightarrow\!\! & C_{ijk} = D_{ikj} \hspace{.1ex}.
\end{array}
\end{equation}

Для тензора второй сложности возможна лишь одна перестановка, называемая транспонированием: ${B_{ij} = A_{ji} \,\Leftrightarrow\, \bm{B} = \bm{A}^{\hspace{-0.05em}\T}}$\!. Очевидно, ${\left( \bm{A}^{\hspace{-0.05em}\T} \right)^{\hspace{-0.4ex}\T} = \bm{A}}$.

При диадном умножении векторов ${\bm{a} \bm{b} = \bm{b} \bm{a} ^{\hspace{-0.05em}\T}\!}$.

Представленные четыре действия можно комбинировать в~разных сочетаниях. Чаще всего встречается комбинация умножения и~свёртки~--- dot product; при этом в~инвариантной безындексной записи ставится точка, указывающая на~свёртку по~соответствующим соседним индексам:
\begin{equation}
a_i = B_{ij} c_j \,\Leftrightarrow\, \bm{a} = \bm{B} \dotp \bm{c}, \;\;
A_{ij} = B_{ik} C_{kj} \,\Leftrightarrow\, \bm{A} = \bm{B} \dotp \bm{C}.
\end{equation}

Определяющее свойство единичного тензора
\vspace{0.16em}\begin{equation}\label{identifyofidentitytensor}
{^\mathrm{n}\hspace{-0.2ex}\bm{a}} \dotp \bm{E} = \bm{E} \dotp {^\mathrm{n}\hspace{-0.2ex}\bm{a}} = {^\mathrm{n}\hspace{-0.2ex}\bm{a}} \;\:\:
\forall \, {^\mathrm{n}\hspace{-0.2ex}\bm{a}} \;\; \forall \,\mathrm{n \!>\! 0}\,.
\end{equation}

В~коммутативном скалярном произведении двух векторов точка имеет тот~же смысл:
\begin{equation}
\bm{a} \dotp \bm{b} = a_i b_i \hspace{-0.1ex} = b_i a_i \hspace{-0.1ex} = \bm{b} \dotp \bm{a} = \hspace{0.1ex} \operatorname{tr} \bm{a}\bm{b} = \operatorname{tr} \bm{b}\bm{a} \hspace{0.1ex}.
\end{equation}

Для dot product’а тензоров второй сложности справедливо следующее
\vspace{-0.5em}\begin{equation}\label{transposeofdotproduct}
\begin{array}{c}
\bm{B} \dotp \bm{Q} \hspace{0.1ex} = \left({ \bm{Q}^{\hspace{-0.1ex}\T} \hspace{-0.2ex}\dotp \bm{B}^{\T} }\right)^{\hspace{-0.2ex}\T} \\[0.1em]
\left({ \bm{B} \dotp \bm{Q}}\right)^{\T} =\hspace{0.2ex} \bm{Q}^{\hspace{-0.1ex}\T} \hspace{-0.2ex}\dotp \bm{B}^{\T}
\end{array}
\end{equation}

\vspace{-0.36em} \noindent (\,например, для~диад ${\bm{B} = \bm{b}\bm{d}}$ и~${\bm{Q} = \bm{p}\bm{q}}$
\vspace{0.1em}\[\begin{array}{rcl}
\left(\hspace{0.1ex} \bm{b}\bm{d} \dotp \bm{p}\bm{q} \hspace{0.1ex}\right)^{\T} & \!=\!\! & \hspace{0.2ex} \bm{p}\bm{q}^{\T} \dotp\hspace{0.4ex} \bm{b}\bm{d}^{\hspace{0.25ex}\T} \\[0.2em]
d_i p_i \hspace{0.32ex} \bm{b}\bm{q}^{\T} & \!=\!\! & \hspace{0.2ex} \bm{q}\bm{p} \dotp \bm{d} \bm{b} \\[0.1em]
d_i p_i \hspace{0.32ex} \bm{q}\bm{b} & \!=\!\! & p_i d_i \hspace{0.32ex} \bm{q}\bm{b} \;\; ).
\end{array}\]

Для вектора и~тензора второй сложности
\begin{equation}\label{vectortensordotproduct}
\bm{c} \dotp \bm{B} = \bm{B}^{\T} \hspace{-0.4ex}\dotp \bm{c} , \;\;
\bm{B} \dotp \bm{c} = \bm{c} \hspace{0.2ex}\dotp \bm{B}^{\T} \hspace{-0.25ex}.
\end{equation}

\end{otherlanguage}

\en{Tensor of~second valence \inquotes{squared} is defined as}

\ru{Тензор второй валентности \inquotes{в~квадрате} определяется как}

\nopagebreak\vspace{-0.2em}\begin{equation}\label{exponentiation:two}
\bm{B}^2 \equiv\hspace{0.2ex} \bm{B} \dotp \bm{B}.
\end{equation}

\en{Contraction can be repeated}\ru{Свёртка может повторяться}:
${\bm{A} \dotdotp \hspace{-0.1ex} \bm{B} \equiv A_{ij} B_{ji}}$,
\en{and here are poco useful equations for tensors of~second complexity}\ru{и~вот немного полезных равенств для~тензоров второй сложности}

\nopagebreak\vspace{-0.1em}\begin{equation}
\begin{array}{c}
\bm{A} \hspace{-0.1ex} \dotdotp \hspace{-0.1ex} \bm{E} = \bm{E} \dotdotp \hspace{-0.1ex} \bm{A} = \operatorname{tr} \bm{A} = A_{jj} \hspace{0.1ex} ,
\\[.2em]
%
\bm{A} \dotdotp \hspace{-0.1ex} \bm{B} = \bm{B} \dotdotp \hspace{-0.1ex} \bm{A} \hspace{0.1ex}, \:\,
\bm{A} \dotdotp \bm{B}^{\T} \hspace{-0.2ex} = \bm{A}^{\hspace{-0.16ex}\T} \hspace{-0.2ex} \dotdotp \bm{B} ,
\\[.2em]
%
\bm{A} \narrowdotdotp \bm{B} \narrowdotp \bm{C} = \bm{A} \narrowdotp \bm{B} \narrowdotdotp \hspace{0.1ex} \bm{C} = \bm{C} \narrowdotdotp \hspace{-0.1ex} \bm{A} \narrowdotp \bm{B} = A_{ij} B_{jk} C_{ki} \hspace{0.1ex} ,
\\[.15em]
%
\bm{A} \hspace{-0.1ex}\narrowdotdotp \bm{B} \narrowdotp \bm{C} \narrowdotp \bm{D} = \bm{A} \narrowdotp \bm{B} \narrowdotdotp \bm{C} \narrowdotp \bm{D} = \bm{A} \narrowdotp \bm{B} \narrowdotp \bm{C} \narrowdotdotp \bm{D} = \hspace*{2.5em} \\
\hspace{10.5em} = \bm{D} \narrowdotdotp \bm{A} \narrowdotp \bm{B} \narrowdotp \bm{C} = A_{ij} B_{jk} C_{kn} D_{ni} \hspace{0.1ex}.
\end{array}
\end{equation}

\en{\section{Polyadic representation (decomposition)}}

\ru{\section{Полиадное представление (разложение)}}

\label{para:polyadicrepresentation}

\en{In}\ru{В}~\pararef{para:tensoranditscomponents}\en{, a~tensor was presented as some invariant object, showing itself in every basis as a~collection of numbers~(components)}\ru{ тензор был представлен как некий инвариантный объект, проявляющий себя в~каждом базисе совокупностью чисел~(компонент)}.
\hbox{\en{Such}\ru{Такое}} \en{representation is typical for majority of~books about tensors}\ru{представление характерно для большинства книг о~тензорах}.
\en{Index notation}\ru{Индексная запись} \en{can be constructive}\ru{может быть конструктивна}, \en{especially when cartesian coordinates are sufficient}\ru{особенно когда достаточно декартовых координат}, \en{but quite often it is not}\ru{но весьма часто это не~так}.
\en{The relevant case is---}\ru{Подходящий случай~---} \en{physics of~elastic continua}\ru{физика упругих сред}: \en{it needs more elegant, more powerful and~perfect apparatus of direct tensor calculus, operating just with indexless invariant objects}\ru{ей нужен более изящн\hbox{ь\kern-0.066emi}й, более мощный и~совершенный аппарат прямого тензорного исчисления, оперирующий лишь с~безындексными инвариантными объектами}.

\en{Relation}\ru{Отношение} ${\bm{v} = v_i \bm{e}_i}$ \en{from}\ru{из}~\eqref{vectorcomponents} \en{links}\ru{связывает} \en{vector}\ru{вектор}~${\bm{v}}$ \en{with basis}\ru{с~базисом}~${\bm{e}_i}$ \en{and vector’s components}\ru{и~компонентами}~${v_i}$ \en{in that basis}\ru{вектора в~том базисе}.
\en{Soon we will get similar relation for a~tensor of any complexity}\ru{Вскоре мы получим похожее соотношение для~тензора любой сложности}.

\begin{otherlanguage}{russian}

\en{Consider bivalent tensor}\ru{Рассмотрим бивалентный тензор} ${^2\!\bm{B}}$.
В~каждом базисе имеем девять чисел~${B_{ij}}$ и~столько~же диад~${\bm{e}_i \bm{e}_j}$. Линейным комбинированием возможно образовать сумму~${B_{ij} \bm{e}_i \bm{e}_j}$. Это тензор, но каковы его компоненты и~сохранится~ли подобное его представление при повороте базиса?

Компоненты построенной суммы
\[
\left( B_{ij} \bm{e}_i \bm{e}_j \right)_{pq} \hspace{-0.2ex} = B_{ij} \delta_{ip} \delta_{jq} \hspace{-0.1ex} = B_{pq}
\]

\vspace{-0.1em} \noindent это компоненты тензора ${^2\!\bm{B}}$. При повороте же базиса
\[
B'_{ij} \bm{e}'_{i} \bm{e}'_{j} \hspace{-0.1ex} = \cosinematrix{i'p} \cosinematrix{j'q} B_{pq} \cosinematrix{i'n} \bm{e}_n \cosinematrix{j'm} \bm{e}_m \hspace{-0.2ex} = \delta_{pn} \delta_{qm} B_{pq} \bm{e}_n \bm{e}_m \hspace{-0.2ex} = B_{pq} \bm{e}_p \bm{e}_q \hspace{.1ex} .
\]

Сомнения отпали~--- приходим к диадному представлению тензора второй сложности
\begin{equation}\label{polyada:2}
^2\!\bm{B} = B_{ij} \bm{e}_i \bm{e}_j \hspace{.1ex} .
\end{equation}

\vspace{-0.2em} Для единичного тензора
\[ \bm{E} =
E_{ij} \hspace{.16ex} \bm{e}_i \bm{e}_j \hspace{-0.1ex} =
\delta_{ij} \hspace{.16ex} \bm{e}_i \bm{e}_j \hspace{-0.1ex} =
\bm{e}_i \bm{e}_i \hspace{-0.1ex} =
\bm{e}_1 \bm{e}_1 + \bm{e}_2 \bm{e}_2 + \bm{e}_3 \bm{e}_3
\hspace{.1ex} . \]

\en{Polyadic representations like}\ru{Полиадные представления типа}~\eqref{polyada:2} помогают проще и~с~б\'{о}льшим пониманием оперировать с~тензорами:

\nopagebreak\vspace{-0.1em}\begin{equation*}
\bm{v} \dotp {^2\!\bm{B}} =
v_i\bm{e}_i \dotp B_{jk}\bm{e}_j\bm{e}_k \hspace{-0.1ex} =
v_i B_{jk} \delta_{ij} \bm{e}_k \hspace{-0.1ex} =
v_i B_{ik} \bm{e}_k \hspace{.1ex} ,
\end{equation*}\vspace{-1.25em}
\begin{equation}\label{tensorcomponents}
\bm{e}_i \dotp {^2\!\bm{B}} \dotp \bm{e}_j \hspace{-0.1ex} =
\bm{e}_i \dotp B_{pq}\bm{e}_p\bm{e}_q \dotp \bm{e}_j \hspace{-0.1ex} =
B_{pq} \delta_{ip} \delta_{qj} =
B_{ij} =
{^2\!\bm{B}} \dotdotp \bm{e}_j \bm{e}_i \hspace{.1ex} .
\end{equation}

Последняя строчка весьма интересна: компоненты тензора выражены через сам тензор. Ортогональное преобразование компонент при~повороте базиса~\eqref{orthotransform:2} оказывается очевидным следствием~\eqref{tensorcomponents}.

Аналогичным образом по базисным полиадам разлагается тензор любой сложности. Например, для трёхвалентного тензора

\nopagebreak\begin{equation}\label{polyada:3}
\begin{array}{c}
{^3\!\,\bm{C}} = C_{ijk} \hspace{0.2ex} \bm{e}_i \bm{e}_j \bm{e}_k \hspace{0.1ex},\\[0.8ex]
C_{ijk} = {^3\!\,\bm{C}} \dotdotdotp \bm{e}_k \bm{e}_j \bm{e}_i = \bm{e}_i \dotp {^3\!\,\bm{C}} \dotdotp \bm{e}_k \bm{e}_j = \bm{e}_j \bm{e}_i \dotdotp {^3\!\,\bm{C}} \dotp \bm{e}_k \hspace{0.1ex}.
\end{array}
\end{equation}

Используя разложение по~полиадам, легко увидеть справедливость свойства~\eqref{identifyofidentitytensor} \inquotes{единичности} тензора~$\bm{E}$:
\vspace{0.1em}\[\begin{array}{c}
{^\mathrm{n}\hspace{-0.2ex}\bm{a}} = a_{ij \ldots q} \hspace{0.4ex} \bm{e}_i \hspace{0.2ex} \bm{e}_j \ldots \bm{e}_q , \;\,  \bm{E} = \bm{e}_e \bm{e}_e
\\[.4em]
%
{^\mathrm{n}\hspace{-0.2ex}\bm{a}} \dotp \bm{E} = a_{ij \ldots q} \hspace{0.4ex} \bm{e}_i \hspace{0.2ex} \bm{e}_j \ldots \tikzmark{BeginDeltaEQBrace} {\bm{e}_q \dotp \bm{e}_e} \tikzmark{EndDeltaEQBrace} \bm{e}_e = a_{ij \ldots q} \hspace{0.4ex} \bm{e}_i \hspace{0.2ex} \bm{e}_j \ldots \bm{e}_q = {^\mathrm{n}\hspace{-0.2ex}\bm{a}} \hspace{0.1ex},
\\[1.1em]
%
\bm{E} \dotp {^\mathrm{n}\hspace{-0.2ex}\bm{a}} = \bm{e}_e \bm{e}_e \dotp a_{ij \ldots q} \hspace{0.4ex} \bm{e}_i \hspace{0.2ex} \bm{e}_j \ldots \bm{e}_q = a_{ij \ldots q} \hspace{0.2ex} \delta_{ei} \hspace{0.2ex} \bm{e}_e \hspace{0.2ex} \bm{e}_j \ldots \bm{e}_q = {^\mathrm{n}\hspace{-0.2ex}\bm{a}}.
\end{array}\]
\AddUnderBrace[line width=.75pt]{BeginDeltaEQBrace}{EndDeltaEQBrace}%
{${\scriptstyle \delta_{eq}}$}

\vspace{-0.5em} Полиадное представление соединяет вместе прямую и~индексную записи. Не~ст\'{о}ит пр\'{о}тиво\-постав\-л\'{я}ть их. Прямая запись компактна, изящна, и~лишь она должна быть в~фундаментальных соотношениях. Но~и~индексная запись полезна: при громоздких выкладках с~тензорами мы не~встретим трудностей.

\end{otherlanguage}

\en{\section{Cross product and Levi\hbox{-}Civita tensor}}

\ru{\section{Векторное произведение и тензор Л\'{е}ви\hbox{-\!}Чив\'{и}ты}}

\label{para:crossproduct+levicivita}

\begin{otherlanguage}{russian}

\en{By habitual notions}\ru{По~привычным представлениям},
\en{a~cross product of two vectors}\ru{векторное произведение двух векторов}
\en{is a~vector}\ru{есть вектор},
направленный перпендикулярно плоскости сомножителей,
длина которого равна площади образуемого сомножителями параллелограмма

\nopagebreak\vspace{-0.2em}\begin{equation*}
\left|\, \bm{a} \times \bm{b} \,\right| = |\bm{a}| \hspace{.1em} |\bm{b}| \hspace{.1em} \operatorname{sin} \measuredangle (\bm{a}, \bm{b})
\hspace{.1ex} .
\end{equation*}

\begin{wrapfigure}[8]{R}{0.32\textwidth}
\makebox[0.36\textwidth][c]{\hspace{0.5em}\begin{minipage}[t]{0.36\textwidth}
\vspace{-1.2em}
\hspace{1.25em}\scalebox{0.9}[0.9]{\begin{tikzpicture}[scale=0.8]
	\draw [line width=2pt, black, -{Latex[length=5mm, width=2mm]}]
		(0,0) -- (2.4,-0.8)
		node[above=1.2mm, xshift=-2mm] {$\bm{a}$};
	\draw [line width=2pt, black, -{Latex[length=5mm, width=2mm]}]
		(0,0) -- (1.6,0.8)
		node[midway, above=0.8mm] {$\bm{b}$};
	\draw [line width=2pt,blue,-{Latex[length=5mm, width=2mm]}]
		(0,0) -- (0,-2.2)
		node[pos=0.75, right, xshift=0.3mm] {$\bm{c}$};
 	\draw [line width=2pt,blue,-{Latex[length=5mm, width=2mm]}]
 		(0,0) -- (0,2.2)
		node[pos=0.8, right, xshift=0.3mm] {$\bm{c}$};
	\draw [line width=0.4pt, black!50] (2.4, -0.8) -- (4, 0);
	\draw [line width=0.4pt, black!50] (1.6, 0.8) -- (4, 0);
\end{tikzpicture}}
\vspace{-1.6em}\caption{}\label{fig:crossproduct}
\end{minipage}}
\end{wrapfigure}

\vspace{-0.1em} Cross product~(векторное произведение) ${\bm{c} = \bm{a} \times \bm{b}}$ считается псевдовектором, поскольку тройка векторов ${\bm{a}}$, ${\bm{b}}$ и~${\bm{c}}$ может быть либо \inquotesx{правой}[,] либо \inquotes{левой}~(\figref{fig:crossproduct}).

Различие векторов и~псевдовекторов \textcolor{magenta}{ограничивает} вид формул: векторы с~псевдовекторами не~складываются. Формула кинематики абсолютно жёсткого недеформируемого тела ${\bm{v} = \bm{v}_{\raisemath{-0.1em}{0}} + \hspace{.1ex} \bm{\omega} \hspace{-0.15ex}\times\hspace{-0.15ex} \bm{r}}$ корректна, поскольку во~втором слагаемом \inaltquotes{два \inquotes{псевдо} взаимно компенсируются}.

\en{However}\ru{Впрочем}, \en{the cross product is not entirely separate operation}\ru{векторное произведение это не~совсем отдельная операция}.
Оно сводится к~четырём ранее описанным~(\pararef{para:operationswithtensors}) и~обобщается на~тензоры любой сложности. Покажем это.

Сначала введём символ перестановки (permutation symbol) \hbox{O.\hspace{0.1ex}Veblen’а}~${e_{ijk}}$: ${e_{123} \hspace{-0.15ex}= e_{231} \hspace{-0.15ex}= e_{312} \hspace{-0.15ex}= 1}$, ${e_{213} \hspace{-0.15ex}= e_{321} \hspace{-0.15ex}= e_{132} \hspace{-0.15ex}= -1}$, остальные нули. Символ~${e_{ijk}}$~(${\pm 1}$ или~$0$) меняет знак при~перестановке любых двух индексов, не~меняется при двойной~(\inquotes{круговой}) перестановке индексов и~обращается в~нуль при~совпадении какой\hbox{-}либо пары.

Через символ Веблена определитель~(детерминант) матрицы~${A_{ij}}$ выражается так:

\nopagebreak\vspace{-0.25em}\begin{equation*}
\operatorname{det} A_{ij} \hspace{-0.2ex} = e_{ijk} A_{1i} A_{2j} A_{3k} \hspace{-0.2ex} = e_{ijk} A_{i1} A_{j2} A_{k3} \hspace{.1ex} .
\end{equation*}

Далее познакомимся с~волюметрическим тензором третьей сложности~--- тензором Levi\hbox{-}Civita\hspace{-0.1ex}’ы

\nopagebreak\vspace{-0.15em}\begin{equation}\label{levicivitaintro}
\levicivitatensor = \levicivita_{ijk} \bm{e}_i \bm{e}_j \bm{e}_k \hspace{0.1ex}, \:\:
\levicivita_{ijk} \equiv \bm{e}_i \hspace{-0.2ex} \times \hspace{-0.2ex} \bm{e}_j \hspace{-0.1ex} \dotp \bm{e}_k \hspace{.1ex} .
\end{equation}

Абсолютная величина каждой ненулевой компоненты ${\levicivita_{ijk}}$
равна объёму~\!${\sqrt{\hspace{-0.36ex}\mathstrut{\textsl{g}}}}$ параллелепипеда, натянутого на~базис~---
\inquotes{смешанному}~(\inquotes{тройному}, \inquotes{векторно\hbox{-}скалярному}) произведению базисных векторов.

Тензор~$\levicivitatensor$ изотропный, его компоненты при~повороте базиса не~меняются.
Но при~изменении ориентации тройки базисных векторов~(перемене \inquotes{направления винта})
$\levicivitatensor$ меняет знак и~поэтому является псевдотензором.

С~тензором Л\'{е}ви\hbox{-\!}Чив\'{и}ты~$\levicivitatensor$ возможно по\hbox{-}новому взглянуть на~векторное произведение:

\nopagebreak\vspace{-0.32em}\begin{equation*}
\levicivita_{ijk} = \bm{e}_i \hspace{-0.2ex} \times \hspace{-0.2ex} \bm{e}_j \hspace{-0.1ex} \dotp \bm{e}_k \:\Leftrightarrow\: \bm{e}_i \times \bm{e}_j
= \levicivita_{ijk} \hspace{.2ex} \bm{e}_k \hspace{.1ex} ,
\end{equation*}\vspace{-1.4em}
\begin{equation}
\bm{a} \mathcolor{blue}{\times} \bm{b} \hspace{.2ex}
= \hspace{.2ex} a_i \bm{e}_i \times b_j \bm{e}_j \hspace{.1ex}
= \hspace{.1ex} \levicivita_{ijk} \hspace{.2ex} a_i \hspace{.1ex} b_j \hspace{.1ex} \bm{e}_k \hspace{.1ex}
= \hspace{.2ex} \bm{b}\bm{a} \hspace{.1ex} \dotdotp \levicivitatensor
= \mathcolor{blue}{-} \hspace{.2ex} \bm{a} \mathcolor{blue}{\dotp \levicivitatensor \dotp} \bm{b}
\hspace{.1ex} .
\end{equation}

\vspace{-0.1em} \noindent Это всего лишь dot product~--- комбинация умножения и~свёртки~--- с~участием тензора~$\levicivitatensor$.

Такие комбинации возможны с~любыми тензорами:

\nopagebreak\vspace{-0.1em}\begin{equation*}
\begin{array}{c}
\bm{a} \hspace{.1ex} \mathcolor{blue}{\times} {^2\hspace{-0.16em}\bm{B}} = a_i \bm{e}_i \times B_{jk} \bm{e}_j \bm{e}_k \hspace{-0.1ex} = \tikzmark{BeginVectorCrossTensor} a_i B_{jk} \hspace{.1ex} \levicivita_{ijn} \tikzmark{EndVectorCrossTensor} \bm{e}_n \bm{e}_k = \mathcolor{blue}{-} \hspace{.2ex} \bm{a} \mathcolor{blue}{\dotp \levicivitatensor \dotp} {^2\!\bm{B}} ,
\\[1.6em]
%
{^2\hspace{-0.05ex}\bm{C}} \mathcolor{blue}{\times} \hspace{-0.1ex} \bm{d} \bm{b} = C_{ij} \bm{e}_i \bm{e}_j \hspace{-0.1ex} \times d_p b_q \bm{e}_p \bm{e}_q \hspace{-0.1ex} = \bm{e}_i C_{ij} d_p \tikzmark{BeginTensorCrossTensor} \levicivita_{jpk} \tikzmark{EndTensorCrossTensor} \bm{e}_k b_q \bm{e}_q =
\\[1.6em]
\hspace{12.5em} =
- \hspace{.2ex} {^2\hspace{-0.05ex}\bm{C}} \hspace{-0.1ex} \bm{d} \hspace{0.12ex} \dotdotp \levicivitatensor \bm{b} \hspace{0.2ex} =
\mathcolor{blue}{-} \hspace{.2ex} {^2\hspace{-0.05ex}\bm{C}} \mathcolor{blue}{\dotp \levicivitatensor \dotp} \bm{d} \bm{b},\\
\end{array}
\end{equation*}%
\AddUnderBrace[line width=.75pt][0,-0.2ex]%
{BeginVectorCrossTensor}{EndVectorCrossTensor}{${\scriptstyle - a_i \levicivita_{inj} B_{jk}}$}%
\AddUnderBrace[line width=.75pt][.2ex,-0.2ex]%
{BeginTensorCrossTensor}{EndTensorCrossTensor}{${\scriptstyle \;- \levicivita_{pjk} \:=\: - \levicivita_{jkp}}$}%
%
\vspace{-0.32em}\begin{equation}\label{iso:twothree}
\bm{E} \times \hspace{-0.16ex} \bm{E} = \bm{e}_i \bm{e}_i \times \bm{e}_j \bm{e}_j = \hspace{-0.4ex} \tikzmark{BeginECrossE} - \hspace{-0.2ex} \levicivita_{ijk} \bm{e}_i \bm{e}_j \bm{e}_k \tikzmark{EndECrossE} = - \hspace{.2ex} \levicivitatensor \hspace{0.1ex}.
\end{equation}
%
\AddUnderBrace[line width=.75pt][.2ex,-0.2ex]%
{BeginECrossE}{EndECrossE}{${\scriptstyle \;\;+\levicivita_{ijk} \bm{e}_i \bm{e}_k \bm{e}_j}$}

\vspace{-0.5em} \noindent Последнее равенство представляет собой связь изотропных тензоров второй и~третьей сложности.

Обобщая на любые тензоры ненулевой сложности

\nopagebreak\vspace{-0.25em}\begin{equation}\label{crossproductforanytwotensors}
{^\mathrm{n}\hspace{-0.12ex}\bm{\xi}} \times \hspace{-0.12ex} {^\mathrm{m}\hspace{-0.12ex}\bm{\zeta}}
=
- \hspace{.25ex} {^\mathrm{n}\hspace{-0.12ex}\bm{\xi}} \dotp \levicivitatensor \dotp \hspace{-0.1ex} {^\mathrm{m}\hspace{-0.12ex}\bm{\zeta}}
\;\;\:
%
\forall \hspace{.4ex} {^\mathrm{n}\hspace{-0.12ex}\bm{\xi}} ,
\hspace{-0.12ex} {^\mathrm{m}\hspace{-0.12ex}\bm{\zeta}}
\;\;
\forall \hspace{.25ex} n \hspace{-0.25ex} > \hspace{-0.25ex} 0 , \,
m \hspace{-0.25ex} > \hspace{-0.25ex} 0
\hspace{.1ex} .
\end{equation}

\vspace{-0.1em} \noindent Когда один из тензоров~--- единичный~(\inquotes{метрический}), из~\eqref{crossproductforanytwotensors} и~\eqref{identifyofidentitytensor} ${\forall \, {^\mathrm{n}\hspace{-0.12ex}\bm{\Upsilon}}}$ ${\forall \,\mathrm{n \!>\! 0}}$

\nopagebreak\vspace{-0.2em}\begin{equation*}\begin{array}{c}
\bm{E} \hspace{.1ex} \times \hspace{-0.1ex} {^\mathrm{n}\hspace{-0.12ex}\bm{\Upsilon}}
= - \hspace{.2ex} \bm{E} \hspace{.1ex} \dotp \levicivitatensor \dotp {^\mathrm{n}\hspace{-0.12ex}\bm{\Upsilon}}
= - \hspace{.2ex} \levicivitatensor \dotp {^\mathrm{n}\hspace{-0.12ex}\bm{\Upsilon}} \hspace{-0.15ex},
\\[.2em]
%
{^\mathrm{n}\hspace{-0.12ex}\bm{\Upsilon}} \times \bm{E}
= - \hspace{.2ex} {^\mathrm{n}\hspace{-0.12ex}\bm{\Upsilon}} \hspace{-0.1ex} \dotp \levicivitatensor \dotp \hspace{-0.1ex} \bm{E}
= - \hspace{.2ex} {^\mathrm{n}\hspace{-0.12ex}\bm{\Upsilon}} \hspace{-0.1ex} \dotp \levicivitatensor
\hspace{.1ex} .
\end{array}\end{equation*}

\vspace{-0.1em} \noindent \en{For any tensor of first complexity~(a~vector)}\ru{Для любого тензора первой сложности~(вектора)}

\nopagebreak\vspace{-0.2em}\begin{equation}
\bm{E} \times \bm{a} = \bm{a} \times \hspace{-0.1ex} \bm{E} = - \hspace{.25ex} \bm{a} \dotp \hspace{-0.1ex} \levicivitatensor = - \hspace{.2ex} \levicivitatensor \dotp \bm{a} \hspace{.1ex} .
\end{equation}

\vspace{-0.1em} \noindent \en{For any two vectors}\ru{Для любых двух векторов} $\bm{a}$ \en{and}\ru{и}~$\bm{b}$

\nopagebreak\vspace{-0.1em}\begin{equation}\label{crossproductoftwovectors}
\begin{array}{c}
\bm{a} \times \bm{b} = \bm{a} \dotp \hspace{-0.1ex} \left( \bm{b} \hspace{-0.1ex} \times \hspace{-0.25ex} \bm{E} \right) = \left( \bm{a} \hspace{-0.1ex} \times \hspace{-0.25ex} \bm{E} \right) \hspace{-0.1ex} \dotp \bm{b} = - \hspace{.25ex} \bm{a} \bm{b} \dotdotp \levicivitatensor = - \hspace{.2ex} \levicivitatensor \dotdotp \bm{a} \bm{b} \hspace{.1ex},
\\[.1em]
%
\bm{b} \times \bm{a} = \bm{b} \dotp \hspace{-0.1ex} \left( \bm{a} \hspace{-0.1ex} \times \hspace{-0.25ex} \bm{E} \right) = \left( \bm{b} \hspace{-0.1ex} \times \hspace{-0.25ex} \bm{E} \right)  \hspace{-0.1ex} \dotp \bm{a} = - \hspace{.25ex} \bm{b} \bm{a} \dotdotp \levicivitatensor = - \hspace{.2ex} \levicivitatensor \dotdotp \bm{b} \bm{a} \hspace{.1ex},
\\[.15em]
%
\bm{a} \times \bm{b} \hspace{.16ex} = - \hspace{.25ex} \bm{a} \bm{b} \dotdotp \levicivitatensor = \bm{b} \bm{a} \dotdotp \levicivitatensor
\:\Rightarrow\:
\bm{a} \times \bm{b} \hspace{.16ex} = - \hspace{.25ex} \bm{b} \times \bm{a} \hspace{.2ex}.
\end{array}
\end{equation}

Когда базис~${\bm{e}_i}$ это \inquotes{правая} тройка взаимно ортогональных векторов единичной длины, тогда~${\sqrt{\hspace{-0.36ex}\mathstrut{\textsl{g}}} = \hspace{-0.1ex} 1}$, и~компоненты~$\levicivitatensor$ равны символу Веблена: ${\levicivita_{ijk} \hspace{-0.2ex} = e_{ijk}}$. Если~же тройка~${\bm{e}_i}$ \inquotesx{левая}[,] то~${\levicivita_{ijk} \hspace{-0.2ex} = \hspace{-0.1ex} - \hspace{.1ex} e_{ijk}}$.

Справедливо такое соотношение

\nopagebreak\vspace{-0.1em}
\begin{equation}\label{doubleveblen}
e_{ijk} e_{pqr} \hspace{-0.1ex} = \hspace{.2ex}
\operatorname{det}\hspace{-0.25ex} \left[\begin{array}{ccc}
\delta_{ip} & \delta_{iq} & \delta_{ir} \\
\delta_{jp} & \delta_{jq} & \delta_{jr} \\
\delta_{kp} & \delta_{kq} & \delta_{kr}
\end{array}\right]
\end{equation}

\noindent Доказательство начнём с~представлений символов Веблена как определителей:
\vspace{-0.5em}\[
e_{ijk} \hspace{-0.1ex} = \hspace{.2ex}
\operatorname{det}\hspace{-0.25ex} \scalebox{0.96}[0.96]{$\left[\begin{array}{ccc}
\delta_{i1} & \delta_{i2} & \delta_{i3} \\
\delta_{j1} & \delta_{j2} & \delta_{j3} \\
\delta_{k1} & \delta_{k2} & \delta_{k3}
\end{array}\right]$} \hspace{-0.25ex}, \:\:
e_{pqr} \hspace{-0.1ex} = \hspace{.2ex}
\operatorname{det}\hspace{-0.25ex} \scalebox{0.96}[0.96]{$\left[\begin{array}{ccc}
\delta_{p1} & \delta_{q1} & \delta_{r1} \\
\delta_{p2} & \delta_{q2} & \delta_{r2} \\
\delta_{p3} & \delta_{q3} & \delta_{r3}
\end{array}\right]$} \hspace{-0.25ex}.
\]

\vspace{-0.1em} \noindent В~левой части~\eqref{doubleveblen} имеем произведение этих определителей. Но~${\operatorname{det} (AB) = (\operatorname{det} A)(\operatorname{det} B)}$~--- определитель произведения матриц равен произведению определителей. В~матрице\hbox{-}произведении элемент~${\left[{\,\cdots\,}\right]}_{11}$ равен~${\delta_{is} \delta_{ps} \hspace{-0.2ex} = \delta_{ip}}$, как~и~в~\eqref{doubleveblen}; легко проверить и~другие фрагменты~\eqref{doubleveblen}.

Свёртка~\eqref{doubleveblen} приводит к~полезным формулам
\begin{equation}\label{veblencontraction}
e_{ijk} e_{pqk} \hspace{-0.16ex} = \delta_{ip} \delta_{jq} \hspace{-0.2ex} - \delta_{iq} \delta_{jp} \hspace{.2ex}, \:\,
e_{ijk} e_{pjk} \hspace{-0.16ex} = 2 \hspace{.16ex} \delta_{ip} \hspace{.2ex}, \:\,
e_{ijk} e_{ijk} \hspace{-0.16ex} = 6 \hspace{.2ex}.
\end{equation}

Первая из~формул даёт знакомое представление двойного векторного произведения
\[\begin{array}{c}
\bm{a} \times \left(\hspace{0.2ex}{\bm{b} \times \bm{c}}\hspace{0.2ex}\right) = a_i \bm{e}_i \times \levicivita_{pqj} \hspace{.2ex} b_p c_q \bm{e}_j  = \levicivita_{kij} \levicivita_{pqj} \hspace{.2ex} a_i b_p c_q \bm{e}_k = \\[0.2em]
= \left({\delta_{kp}\delta_{iq} - \delta_{kq}\delta_{ip}}\right) a_i b_p c_q \bm{e}_k = a_i b_k c_i \bm{e}_k \hspace{-0.12ex} - a_i b_i c_k \bm{e}_k \hspace{-0.16ex} = \bm{b}\bm{a} \dotp \bm{c} - \bm{c}\bm{a} \dotp \bm{b} \hspace{.16ex}.
\end{array}\]\vspace{-0.6em}

\noindent Тем~же путём выводится ${\left(\hspace{0.1ex} \bm{a} \times \bm{b} \hspace{0.2ex}\right) \hspace{-0.1ex} \times \bm{c} = \bm{b}\bm{a} \dotp \bm{c} - \bm{a}\bm{b} \dotp \bm{c}}$\hspace{.16ex}.

И~такие тождества для~любых двух векторов~$\bm{a}$ и~$\bm{b}$
\nopagebreak\vspace{-0.2em}
\begin{multline}\label{vectorcrossvectorcrossidentity}
\left(\hspace{0.1ex} \bm{a} \times \bm{b} \hspace{0.2ex}\right) \hspace{-0.1ex} \times \hspace{-0.1ex} \bm{E}
= \levicivita_{ijk} \hspace{0.2ex} a_i \hspace{0.1ex} b_j \hspace{0.1ex} \bm{e}_k \hspace{-0.16ex} \times \hspace{-0.1ex} \bm{e}_n \bm{e}_n \hspace{-0.2ex}
= a_i \hspace{0.1ex} b_j \hspace{.1ex} \levicivita_{ijk} \levicivita_{knq} \hspace{.2ex} \bm{e}_q \bm{e}_n \hspace{-0.2ex} = \\[-0.1em]
%
= a_i b_j \hspace{-0.2ex} \left( \delta_{in} \delta_{jq} \hspace{-0.2ex} - \delta_{iq} \delta_{jn} \right) \hspace{-0.2ex} \bm{e}_q \bm{e}_n \hspace{-0.2ex}
= a_i b_j \bm{e}_j \bm{e}_i \hspace{-0.12ex} - a_i b_j \bm{e}_i \bm{e}_j \hspace{-0.16ex}
= \bm{b} \bm{a} - \bm{a} \bm{b} \hspace{0.2ex},
\end{multline}

\vspace{-0.45em}\begin{multline}\label{vectorcrossidentitydotvectorcrossidentity}
\left(\hspace{0.1ex} \bm{a} \times\hspace{-0.1ex} \bm{E} \hspace{0.16ex}\right) \dotp \left(\hspace{0.12ex} \bm{b} \times\hspace{-0.1ex} \bm{E} \hspace{0.16ex}\right) =
\left(\hspace{0.1ex} \bm{a} \hspace{0.2ex} \dotp \levicivitatensor \hspace{0.16ex}\right) \hspace{-0.12ex}\dotp \left(\hspace{0.1ex} \bm{b} \hspace{0.2ex} \dotp \levicivitatensor \hspace{0.16ex}\right) = \\[-0.2em]
%
= a_i \levicivita_{ipn} \hspace{0.1ex} \bm{e}_p \bm{e}_n \dotp \hspace{0.2ex} b_j \levicivita_{jsk} \hspace{0.1ex} \bm{e}_s \bm{e}_k
= a_i b_j \levicivita_{ipn} \levicivita_{nkj} \hspace{0.05ex} \bm{e}_p \bm{e}_k = \\[-0.08em]
%
= a_i b_j \hspace{-0.2ex} \left( \delta_{ik} \delta_{pj} \hspace{-0.2ex} - \delta_{ij} \delta_{pk} \right) \hspace{-0.2ex} \bm{e}_p \bm{e}_k
= a_i b_j \bm{e}_j \bm{e}_i \hspace{-0.12ex} - a_i b_i \bm{e}_k \bm{e}_k \hspace{-0.16ex} = \\[-0.18em]
%
= \hspace{0.1ex} \bm{b} \bm{a} - \bm{a} \hspace{-0.1ex} \dotp \bm{b} \hspace{0.2ex} \bm{E} \hspace{0.1ex}.
\end{multline}

\vspace{-0.2em} Ещё одно соотношение между изотропными тензорами второй и~третьей сложности:
\begin{equation}
\hspace{4.4em} \levicivitatensor \hspace{0.2ex} \dotdotp \levicivitatensor = \levicivita_{ijk} \bm{e}_i \hspace{0.2ex} \levicivita_{kjn} \bm{e}_n = - 2 \hspace{.2ex} \delta_{in} \hspace{.12ex} \bm{e}_i \bm{e}_n = - 2 \hspace{0.1ex} \bm{E} \hspace{0.1ex}.
\end{equation}

\end{otherlanguage}

\en{\section{Symmetric and antisymmetric tensors}}

\ru{\section{Симметричные и антисимметричные тензоры}}

\label{para:tensors.symmetric+skewsymmetric}

\begin{otherlanguage}{russian}

Тензор, не~меняющийся при перестановке какой\hbox{-}либо пары своих индексов, называется симметричным по~этой~паре индексов. Если~же при~перестановке пары индексов тензор меняет свой знак, то он называется антисимметричным~(кососимметричным) по~этой~паре индексов.

Тензор Л\'{е}ви\hbox{-\!}Чив\'{и}ты~$\levicivitatensor$ антисимметричен по~любой паре индексов, то~есть он полностью антисимметричен~(абсолютно кососимметричен).

Тензор второй сложности ${\bm{B}}$ симметричен, если ${\bm{B} = \bm{B}^\T}$. Когда транспонирование меняет знак тензора, то~есть ${\bm{A}^{\hspace{-0.1em}\T} = - \bm{A}}$, тогда он антисимметричен (кососимметричен).

Любой тензор второй сложности представ\'{и}м суммой симметричной и~антисимметричной частей
\vspace{0.2em}\begin{equation}\begin{array}{c}
\bm{C} = \bm{C}^{\mathsf{\hspace{0.1ex}S}} \hspace{-0.1ex}+\hspace{0.1ex} \bm{C}^{\mathsf{\hspace{0.1ex}A}}\hspace{0.1ex} , \;
\bm{C}^{\hspace{0.1ex}\T} \hspace{-0.32ex} = \bm{C}^{\mathsf{\hspace{0.1ex}S}} \hspace{-0.1ex}-\hspace{0.1ex} \bm{C}^{\mathsf{\hspace{0.1ex}A}}\hspace{0.2ex} ; \\[0.2em]
\bm{C}^{\mathsf{\hspace{0.1ex}S}} \equiv \displaystyle \onehalf \left({\bm{C} + \bm{C}^{\hspace{0.1ex}\T}}\hspace{0.1ex}\right)\!, \;
\bm{C}^{\mathsf{\hspace{0.1ex}A}} \equiv \displaystyle \onehalf \left({\bm{C} - \bm{C}^{\hspace{0.1ex}\T}}\hspace{0.1ex}\right)\!.
\end{array}\end{equation}

\noindent Для диады ${\bm{c}\bm{d} \hspace{.1ex} = \bm{c}\bm{d}^{\mathsf{\hspace{.25ex}S}} \hspace{-0.2ex} + \hspace{.1ex} \bm{c}\bm{d}^{\mathsf{\hspace{.25ex}A}} \hspace{-0.16ex} = \hspace{.1ex}
\smalldisplaystyleonehalf \hspace{-0.1ex} \left( \bm{c}\bm{d} + \hspace{-0.1ex} \bm{d}\bm{c} \hspace{0.1ex} \right)
+ \hspace{.16ex} \smalldisplaystyleonehalf \hspace{-0.1ex} \left( \bm{c} \bm{d} - \hspace{-0.1ex} \bm{d} \bm{c} \hspace{.1ex} \right)}$.

Произведение двух симметричных тензоров~${\bm{C}^{\mathsf{\hspace{0.1ex}S}} \hspace{-0.1ex}\dotp \bm{D}^{\mathsf{\hspace{0.1ex}S}}}$ симметрично далеко не~всегда, но~только когда ${\bm{D}^{\mathsf{\hspace{0.1ex}S}} \hspace{-0.1ex}\dotp\hspace{0.12ex} \bm{C}^{\mathsf{\hspace{0.1ex}S}} \hspace{-0.2ex}=\hspace{0.1ex} \bm{C}^{\mathsf{\hspace{0.1ex}S}} \hspace{-0.1ex}\dotp \bm{D}^{\mathsf{\hspace{0.1ex}S}}\hspace{-0.4ex}}$,
ведь по~\eqref{transposeofdotproduct} ${\left(\hspace{0.08ex} \bm{C}^{\mathsf{\hspace{0.1ex}S}} \hspace{-0.1ex}\dotp \bm{D}^{\mathsf{\hspace{0.1ex}S}} \hspace{0.1ex}\right)^{\hspace{-0.32ex}\T} \hspace{-0.16ex} = \hspace{0.1ex} \bm{D}^{\mathsf{\hspace{0.1ex}S}} \hspace{-0.1ex}\dotp\hspace{0.12ex} \bm{C}^{\mathsf{\hspace{0.1ex}S}} \hspace{-0.4ex}}$.

В~нечётномерных пространствах любой антисимметричный тензор второй сложности необрат\'{и}м, его определитель~${\operatorname{det} \bm{A} = 0}$.

Существует взаимно\hbox{-}однозначное соответствие между антисимметричными тензорами второй сложности и~(псевдо)векторами. Матрица компонент кососимметричного тензора определяется тройкой чисел (диагональные компоненты~--- нули, а~недиагональные~--- попарно противоположны). Dot product кососимметричного~${\hspace{-0.2ex}\bm{A}}$ и~какого\hbox{-}либо тензора~${\hspace{-0.2ex}{^\mathrm{n}\hspace{-0.12ex}\bm{\xi}}}$ однозначно соответствует cross product’у псевдовектора~$\bm{a}$ и~того~же тензора~${\hspace{-0.2ex}{^\mathrm{n}\hspace{-0.12ex}\bm{\xi}}}$
\nopagebreak\vspace{.2em}\begin{equation}
\hspace{3.2em} \bm{c} \:= \bm{A} \hspace{0.2ex}\dotp\hspace{0.1ex} {^\mathrm{n}\hspace{-0.12ex}\bm{\xi}} \;\,\Leftrightarrow\; \bm{a} \hspace{0.2ex}\times {^\mathrm{n}\hspace{-0.12ex}\bm{\xi}} \:=\, \bm{c} \hspace{0.1ex} \;\:\:
\forall \bm{A} \!=\! \bm{A}^{\mathsf{\!\,A}} \;\; \forall \, {^\mathrm{n}\hspace{-0.12ex}\bm{\xi}} \;\; \forall \,\mathrm{n \!>\! 0} \hspace{0.16ex}.
\end{equation}

Раскроем это соответствие~${\bm{A} \!=\! \bm{A}(\bm{a})}$:
\nopagebreak\[\begin{array}{r@{\hspace{1ex}}c@{\hspace{1ex}}l}
\bm{A} \hspace{0.2ex}\dotp\hspace{0.1ex} {^\mathrm{n}\hspace{-0.12ex}\bm{\xi}} & = & \bm{a} \hspace{0.2ex}\times {^\mathrm{n}\hspace{-0.12ex}\bm{\xi}} \\
%
A_{hi} \bm{e}_h \bm{e}_i \hspace{0.1ex}\dotp\hspace{0.25ex} \xi_{jk \ldots q} \hspace{0.2ex} \bm{e}_j \bm{e}_k \ldots \bm{e}_q & = & a_{i} \bm{e}_i \times\hspace{0.2ex} \xi_{jk \ldots q} \hspace{0.2ex} \bm{e}_j \bm{e}_k \ldots \bm{e}_q \\[0.1em]
%
A_{hj} \hspace{0.2ex} \xi_{jk \ldots q} \hspace{0.2ex} \bm{e}_h \bm{e}_k \ldots \bm{e}_q & = & a_{i} \levicivita_{ijh} \hspace{0.2ex} \xi_{jk \ldots q} \hspace{0.2ex} \bm{e}_h \bm{e}_k \ldots \bm{e}_q \\[0.1em]
%
%%A_{hj} & = & a_{i} \levicivita_{ijh} \\[0.1em]
A_{hj} & = & \! - \, a_{i} \levicivita_{ihj} \\[0.2em]
\bm{A} & = & \! - \, \bm{a} \dotp \levicivitatensor
\end{array}\]

Так~же из ${{^\mathrm{n}\hspace{-0.12ex}\bm{\xi}} \hspace{0.2ex}\dotp\hspace{0.1ex} \bm{A} = {^\mathrm{n}\hspace{-0.12ex}\bm{\xi}} \hspace{0.2ex}\times \bm{a}}$ получается ${\bm{A} = - \hspace{0.2ex} \levicivitatensor \dotp \bm{a}}$.

В~общем, для взаимно\hbox{-}однозначного соответствия между~${\hspace{-0.2ex}\bm{A}}$ и~$\bm{a}$ имеем

\nopagebreak\vspace{-0.1em}\begin{equation}\label{companionvector}
\begin{array}{c}
\bm{A} \:=\, - \, \bm{a} \dotp \levicivitatensor \:=\: \bm{a} \times\hspace{-0.1ex} \bm{E} \:=\, - \, \levicivitatensor \dotp \bm{a} \:=\: \bm{E} \times \bm{a} \hspace{0.1ex},\\[.3em]
%
\bm{a} = \bm{a} \hspace{.1ex} \dotp \bm{E} = \bm{a} \dotp \left( \! - \displaystyle \hspace{0.2ex} \smalldisplaystyleonehalf \hspace{.4ex} \levicivitatensor \hspace{0.2ex} \dotdotp \levicivitatensor \right) \hspace{-0.4ex} = \displaystyle \smalldisplaystyleonehalf \hspace{.32ex} \bm{A} \dotdotp \levicivitatensor \hspace{0.1ex}.
\end{array}
\end{equation}

\vspace{-0.1em} \noindent Вектор~$\bm{a}$ называется сопутствующим тензору~${\hspace{-0.2ex}\bm{A}}$.

Компоненты кососимметричного~${\hspace{-0.2ex}\bm{A}}$ через компоненты сопутствующего ему псевдовектора~$\bm{a}$
\vspace{0.2em}\[\begin{array}{c}
\bm{A} = - \, \levicivitatensor \dotp \bm{a} = - \hspace{0.1ex} \levicivita_{ijk} \hspace{0.1ex} \bm{e}_i \bm{e}_j a_k ,\\[0.3em]
A_{ij} = - \hspace{0.1ex} \levicivita_{ijk} \hspace{0.1ex} a_k \hspace{0.1em} = \hspace{-0.1em}
\scalebox{0.9}[0.9]{$\left[ \begin{array}{ccc}
0 & -a_3 & a_2 \\
a_3 & 0 & -a_1 \\
-a_2 & a_1 & 0
\end{array} \hspace{0.25ex}\right]$}
\end{array}\]

\vspace{-0.5em} \noindent и~обратно
\vspace{-0.25em}\[\begin{array}{c}
\bm{a} = \smalldisplaystyleonehalf \, \bm{A} \dotdotp \levicivitatensor =
\smalldisplaystyleonehalf \hspace{.25ex} A_{jk} \levicivita_{kji} \hspace{.2ex} \bm{e}_i , \\[0.64em]
a_{i} \hspace{-0.16ex} = \smalldisplaystyleonehalf \hspace{.32ex} \levicivita_{ikj} \hspace{0.1ex} A_{jk} = \hspace{0.1em}
\displaystyle \onehalf \scalebox{0.9}[0.9]{$\left[\hspace{-0.25ex} \begin{array}{c}
\levicivita_{123} \hspace{0.1ex} A_{32} + \levicivita_{132} \hspace{0.1ex} A_{23} \\
\levicivita_{213} \hspace{0.1ex} A_{31} + \levicivita_{231} \hspace{0.1ex} A_{13} \\
\levicivita_{312} \hspace{0.1ex} A_{21} + \levicivita_{321} \hspace{0.1ex} A_{12}
\end{array} \right]$} \hspace{-0.2em} = \hspace{0.1em}
\displaystyle \onehalf \scalebox{0.9}[0.9]{$\left[\hspace{-0.3ex} \begin{array}{c}
A_{32} - A_{23} \\
A_{13} - A_{31} \\
A_{21} - A_{12}
\end{array} \right]$} .
\vspace{.1em}\end{array}\]

Легко запоминающийся вспомогательный \inquotes{псевдовекторный инвариант}~${\!\bm{A}_{\!\bm{\times}}}$ получается из тензора~${\hspace{-0.2ex}\bm{A}}$ заменой диадных произведений на~векторные

\nopagebreak\vspace{-0.15em}\begin{equation}\label{pseudovectorinvariant}
\begin{array}{c}
\bm{A}_{\Xcompanion} \equiv A_{ij} \hspace{.25ex} \bm{e}_i \times \bm{e}_j = - \hspace{0.1ex} \bm{A} \hspace{0.1ex} \dotdotp \levicivitatensor
\hspace{.2ex}, \\[.3em]
%
\bm{A}_{\Xcompanion} \hspace{-0.16ex} = \left(^{\mathstrut} \hspace{-0.1ex} \bm{a} \times\hspace{-0.1ex} \bm{E} \hspace{.2ex} \right)_{\hspace{-0.25ex}\Xcompanion} \hspace{-0.25ex} = \hspace{-0.12ex}
- 2 \hspace{0.16ex} \bm{a} \hspace{.2ex},\:\:
\bm{a} = - \hspace{.2ex} \smalldisplaystyleonehalf \hspace{.25ex} \bm{A}_{\Xcompanion} = - \hspace{.2ex} \smalldisplaystyleonehalf \left(^{\mathstrut} \hspace{-0.1ex} \bm{a} \times\hspace{-0.1ex} \bm{E} \hspace{.2ex} \right)_{\hspace{-0.25ex}\Xcompanion} \hspace{-0.2ex}.
\end{array}
\end{equation}

Обоснование~\eqref{pseudovectorinvariant}:

\nopagebreak\vspace{-0.1em}\[\begin{array}{c}
\bm{a} \times\hspace{-0.1ex} \bm{E} = - \hspace{.2ex} \smalldisplaystyleonehalf \hspace{.4ex} \bm{A}_{\Xcompanion} \times \bm{E} = - \hspace{.2ex} \smalldisplaystyleonehalf \, A_{ij} \hspace{-0.2ex}
\left( \hspace{.08ex} \tikzmark{beginFirstCrossProduct} {\bm{e}_i \times \bm{e}_j} \tikzmark{endFirstCrossProduct} \hspace{.16ex} \right)
\times \bm{e}_k \bm{e}_k = \\[1.5em]
%
= - \hspace{.2ex} \smalldisplaystyleonehalf \hspace{.32ex} A_{ij} \hspace{.12ex}
\tikzmark{beginTwoLeviCivitas} \levicivita_{nij} \levicivita_{nkp} \tikzmark{endTwoLeviCivitas}
\hspace{.32ex} \bm{e}_p \bm{e}_k = - \hspace{.2ex} \smalldisplaystyleonehalf \hspace{.32ex} A_{ij} \hspace{-0.1ex} \left( \bm{e}_j \bm{e}_i - \bm{e}_i \bm{e}_j \right) = \\[.8em]
%
\hspace{13.2em}= - \hspace{.2ex} \smalldisplaystyleonehalf \left({ \bm{A}^{\hspace{-0.1em}\T} \hspace{-0.25ex} - \hspace{-0.2ex} \bm{A} \hspace{0.2ex}}\right) = \bm{A}^{\mathsf{\hspace{-0.1ex}A}} = \bm{A}.
\end{array}\]
\AddUnderBrace[line width=.75pt][0,-0.25ex]{beginFirstCrossProduct}{endFirstCrossProduct}%
{${\scriptstyle \levicivita_{ijn} \bm{e}_n}$}
\AddUnderBrace[line width=.75pt][.25ex,-0.25ex]{beginTwoLeviCivitas}{endTwoLeviCivitas}%
{${\scriptstyle \hspace{3.2em}\delta_{jp} \delta_{ik} \,-\; \delta_{ip} \delta_{jk}}$}

\vspace{-0.6em} Сопутствующий вектор \textcolor{magenta}{можно ввести} для любого бивалентного тензора, но лишь антисимметричная часть при~этом даёт вклад: ${\bm{C}^{\hspace{.2ex}\mathsf{A}} \hspace{-0.1ex} = \hspace{-0.1ex} - \hspace{.1ex}\onehalf\hspace{.32ex} \bm{C}_{\hspace{-0.1ex}\Xcompanion} \hspace{-0.16ex} \times \hspace{-0.16ex} \bm{E}}$. Для~симметричного тензора сопутствующий вектор~--- нулевой: ${\bm{B}_{\Xcompanion} \hspace{-0.25ex} = \bm{0} \hspace{.25ex} \Leftrightarrow \bm{B} = \bm{B}^{\T} \hspace{-0.32ex} = \bm{B}^{\mathsf{\hspace{.1ex}S}}\hspace{-0.32ex}}$.

С~\eqref{pseudovectorinvariant} разложение какого\hbox{-}либо тензора~$\bm{C}$ на~симметричную и~антисимметричную части выглядит как

\nopagebreak\vspace{-0.1em}\begin{equation}\label{symmetricantisymmetricdecompositionofsometensor}
\bm{C} = \bm{C}^{\mathsf{\hspace{.1ex}S}} \hspace{-0.32ex} - \hspace{.1ex} \smalldisplaystyleonehalf \hspace{.32ex} \bm{C}_{\hspace{-0.1ex}\Xcompanion} \hspace{-0.16ex} \times \hspace{-0.16ex} \bm{E} \hspace{.1ex} .
\end{equation}

\vspace{-0.8em}\noindent Для диады~же

\nopagebreak{\centering \eqref{vectorcrossvectorcrossidentity}~$\Rightarrow$~${\left( \bm{c} \hspace{-0.1ex} \times \hspace{-0.2ex} \bm{d} \hspace{.2ex} \right) \hspace{-0.12ex} \times \hspace{-0.25ex} \bm{E} = \bm{d} \bm{c} - \hspace{-0.1ex} \bm{c} \bm{d} \hspace{.1ex} = \hspace{-0.1ex} - \hspace{.1ex} 2 \hspace{.15ex} \bm{c}\bm{d}^{\mathsf{\hspace{.3ex}A}}\hspace{-0.2ex}}$,\hspace{.32em}
${\left( \bm{c} \bm{d} \hspace{.2ex} \right)_{\hspace{-0.15ex}\Xcompanion} \hspace{-0.32ex} =
\hspace{0.1ex} \bm{c} \hspace{-0.1ex} \times \hspace{-0.16ex} \bm{d} \hspace{.12ex}}$, \par}

\nopagebreak\vspace{-0.1em}\noindent и разложение~её

\nopagebreak\vspace{-0.6em}\begin{equation}\label{symmetricantisymmetricdecompositionofdyad}
\hspace*{1em} \bm{c}\bm{d} \hspace{.1ex} = \hspace{.1ex}
%%\smalldisplaystyleonehalf \hspace{-0.1ex} \left( \bm{c}\bm{d} + \hspace{-0.1ex} \bm{d}\bm{c} \hspace{0.1ex} \right) - \hspace{.16ex} \smalldisplaystyleonehalf \hspace{-0.1ex} \left( \bm{d} \bm{c} - \hspace{-0.1ex} \bm{c} \bm{d} \hspace{.2ex} \right) = \hspace{.1ex}
\smalldisplaystyleonehalf \hspace{-0.1ex} \left( \bm{c}\bm{d} + \hspace{-0.1ex} \bm{d}\bm{c} \hspace{0.1ex} \right)
- \hspace{.16ex} \smalldisplaystyleonehalf \hspace{-0.1ex} \left( \bm{c} \hspace{-0.1ex} \times \hspace{-0.2ex} \bm{d} \hspace{.2ex} \right) \hspace{-0.12ex} \times \hspace{-0.25ex} \bm{E} \hspace{.1ex} .
\end{equation}

\end{otherlanguage}

\en{\section{Eigenvectors and eigenvalues of tensor}}

\ru{\section{Собственные векторы и собственные числа тензора}}

\label{para:eigenvectorseigenvalues}

\begin{otherlanguage}{russian}

Если для тензора~${^2\!\bm{B}}$ и~ненулевого вектора~${\bm{a}}$

\nopagebreak\vspace{-0.2em}\begin{equation}\label{eigenvalues:eq}
^2\!\bm{B} \dotp \bm{a} = \eigenvalue \bm{a} \hspace{.1ex} ,
\:\:
\bm{a} \neq \bm{0}
\end{equation}
\vspace{-1.2em}\[
\scalebox{0.92}[0.92]{${^2\!\bm{B}} \dotp \bm{a} = \eigenvalue \bm{E} \dotp \bm{a} \hspace{0.1ex},\;\;
%%B_{ij} a_j = \eigenvalue \delta_{ij} a_j \hspace{0.1ex},\;\;
\left(\hspace{0.1ex}
{^2\!\bm{B} - \eigenvalue \bm{E}}
\hspace{0.2ex}\right) \hspace{-0.1ex} \dotp \hspace{.1ex} \bm{a} = \bm{0}$} \hspace{.1ex},
\]

\vspace{-0.64em} \noindent то $\eigenvalue$ называется собственным числом~(собственным значением, eigenvalue, главным значением)~${^2\!\bm{B}}$, а~определяемая собственным вектором~$\bm{a}$ ось~(направление)~--- его собственной~(главной, principal) осью~(направлением).

В~компонентах это матричная задача на~собственные значения ${( B_{ij} \hspace{-0.12ex} - \eigenvalue \delta_{ij} ) \hspace{.25ex} a_j \hspace{-0.1ex} = 0}$~--- однородная линейная алгебраическая система, имеющая ненулевые решения при~равенстве нулю определителя ${\operatorname{det}\hspace{.32ex} ( B_{ij} \hspace{-0.12ex} - \eigenvalue \delta_{ij} )}$:

\nopagebreak\vspace{-0.1em}\begin{equation}\label{chardetequation}
\operatorname{det}\! \scalebox{0.9}[0.92]{$\left[
\begin{array}{ccc}
B_{11} \hspace{-0.16ex} - \eigenvalue & B_{12} & B_{13} \\
B_{21} & B_{22} \hspace{-0.16ex} - \eigenvalue & B_{23} \\
B_{31} & B_{32} & B_{33} \hspace{-0.16ex} - \eigenvalue
\end{array}
\right]$} \!= - \eigenvalue^3 \hspace{-0.25ex} + \mathrm{I}\hspace{.25ex} \eigenvalue^2 \hspace{-0.25ex} - \mathrm{II}\hspace{.25ex} \eigenvalue + \mathrm{III} = 0 \hspace{.2ex} ;
\end{equation}

\vspace{-0.25em}\begin{equation}\label{invariants:2}
\begin{array}{r@{\hspace{.4em}}c@{\hspace{.4em}}l}
\mathrm{I} & = & \operatorname{tr}\hspace{.1ex} {^2\!\bm{B}} = B_{kk} = \scalebox{0.92}[0.96]{$B_{11} + B_{22} + B_{33}$} \hspace{.1ex} ,\\[0.1em]
\mathrm{II} & = & \scalebox{0.92}[0.96]{$B_{11}B_{22} - B_{12}B_{21} + B_{11}B_{33} - B_{13}B_{31} + B_{22}B_{33} - B_{23}B_{32}$} \hspace{.1ex} ,\\[0.1em]
\mathrm{III} & = & \operatorname{det} \hspace{.1ex} {^2\!\bm{B}} = \operatorname{det} B_{ij} = e_{ijk} \hspace{.1ex} B_{1i} B_{2j} B_{3k} = e_{ijk} \hspace{.1ex} B_{i1} B_{j2} B_{k3} \hspace{.1ex} .
\end{array}\end{equation}

Корни характеристического уравнения~\eqref{chardetequation}~--- собственные числа~--- не~зависят от~базиса и~потому инвариантны.

Коэффициенты~\eqref{invariants:2} тоже не~зависят от~базиса; они называются первым, вторым и~третьим инвариантами тензора.
С~первым инвариантом~${\mathrm{I}}$~--- следом тензора~--- мы уже встречались в~\pararef{para:operationswithtensors}.
Второй инвариант~${\mathrm{II}}$ это след союзной~(взаимной, adjugate) матрицы~--- транспонированной матрицы дополнений: ${\mathrm{II}\hspace{.16ex}(\hspace{.1ex}{^2\!\bm{B}}) \equiv \hspace{.1ex} \operatorname{tr} \left({\operatorname{adj}{B_{ij}}}\right)}$.
Или он~же \hbox{${\mathrm{II}\hspace{.16ex}(\hspace{.1ex}{^2\!\bm{B}}) \equiv \hspace{.2ex} \textstyle \onehalf \hspace{-0.08ex} \scalebox{0.96}[0.96]{$\left[ \left(\operatorname{tr}\hspace{.1ex} {^2\!\bm{B}}\right)^{\!2} \hspace{-0.5ex} - \operatorname{tr} \left( {^2\!\bm{B}} \dotp \hspace{-0.1ex} {^2\!\bm{B}} \right) \right]$} \hspace{-0.4ex} = \hspace{.1ex} \textstyle \onehalf \hspace{-0.08ex} \scalebox{0.96}[0.96]{$\left[ \left( B_{kk} \right)^{\hspace{-0.1ex}2} \hspace{-0.4ex} - \hspace{-0.2ex} B_{ij} B_{ji} \hspace{.2ex}\right]$}}$\hspace{-0.25ex}}.
И~третий инвариант~${\mathrm{III}}$ это определитель~(детерминант) компонент тензора: %%в~любом \inquotes{правом} ортонормальном или~смешанном базисе:
${\mathrm{III}\hspace{.16ex}(\hspace{.1ex}{^2\!\bm{B}}) \equiv \operatorname{det}\hspace{.1ex} {^2\!\bm{B}}}$.

Это относилось ко~всем тензорам второй сложности. Для~случая~же симметричного тензора справедливо следующее:\\
\indent 1$^{\circ}$\hspace{-1ex}.\, Собственные числа симметричного тензора вещественны.\\
\indent 2$^{\circ}$\hspace{-1ex}.\, Собственные оси для~разных собственных чисел ортогональны.

Первое утверждение доказывается от~противного. Если~$\eigenvalue$~--- компл\'{е}ксный корень~\eqref{chardetequation}, определяющий собственный вектор~$\bm{a}$, то сопряжённое число~$\overline{\eigenvalue}$ также будет корнем. Ему соответствует собственный вектор~$\overline{\bm{a}}$ с~сопряжёнными компонентами. При этом

\nopagebreak\vspace{-0.4em}\begin{multline*}
\eqref{eigenvalues:eq}
\:\,\Rightarrow\:\,
\left(\hspace{.25ex} {\overline{\bm{a}} \, \dotp} \hspace{.25ex}\right) \hspace{.5ex} {^2\!\bm{B}} \dotp \bm{a} \hspace{.2ex} = \eigenvalue \bm{a}, \;\;
\left(\hspace{.2ex}{\bm{a} \, \dotp}\hspace{.2ex}\right) \hspace{.5ex} {^2\!\bm{B}} \dotp \overline{\bm{a}} \hspace{.2ex} = \hspace{.1ex} \overline{\eigenvalue} \, \overline{\bm{a}}
\:\;\Rightarrow \\[-0.1em]
%
\Rightarrow\;\: \overline{\bm{a}} \dotp {^2\!\bm{B}} \dotp \bm{a} - \bm{a} \dotp {^2\!\bm{B}} \dotp \overline{\bm{a}} \hspace{.2ex} = \left(\hspace{.1ex}{\eigenvalue - \overline{\eigenvalue}}\hspace{.16em}\right)\hspace{-0.1em} \bm{a} \dotp \overline{\bm{a}} \hspace{.2ex} .
\end{multline*}

\vspace{-0.16em} \noindent Но слева здесь~--- нуль, поскольку ${\bm{a} \dotp {^2\!\bm{B}} \dotp \bm{c} = \bm{c} \dotp {^2\!\bm{B}^{\T}\!} \dotp \bm{a}}$ и~${{^2\!\bm{B}} = {^2\!\bm{B}^{\T}\!}}$. Поэтому ${\eigenvalue = \overline{\eigenvalue}}$, то~есть вещественно.

Столь~же просто обосновывается и~2$^{\circ}$:

\nopagebreak\vspace{-0.1em}\begin{equation*}\begin{array}{c}
\tikzmark{BeginEqualsZeroBrace} {\bm{a}_2 \dotp {^2\!\bm{B}} \dotp \bm{a}_1 - \bm{a}_1 \dotp {^2\!\bm{B}} \dotp \bm{a}_2} \hspace{.1ex} \tikzmark{EndEqualsZeroBrace} = \hspace{.2ex} \left(\hspace{.1ex}{\eigenvalue_1 \!-\! \eigenvalue_2}\right) \bm{a}_1 \dotp \bm{a}_2 \hspace{.1ex} , \;
\eigenvalue_1 \neq \eigenvalue_2 \;\Rightarrow \\[0.2em]
\hspace{16em} \Rightarrow\; \bm{a}_1 \dotp \bm{a}_2 = 0 \hspace{.1ex}.
\end{array}\end{equation*}
\AddUnderBrace[line width=.75pt][0.1ex,-0.1ex]%
{BeginEqualsZeroBrace}{EndEqualsZeroBrace}{${\scriptstyle =\;0}$}

\vspace{-0.64em} При~различных собственных числах собственные векторы единичной длины~${\mathboldae_i}$ образуют ортонормальный базис; каковы~же в~нём компоненты тензора?

\vspace{-0.1em}\[\begin{array}{c}
{^2\!\bm{B}} \dotp \mathboldae_k = \hspace{0.2ex} \scalebox{0.8}[0.8]{$\tikzcancel[blue]{$\displaystyle\sum_k^{~}$}$}\: \eigenvalue_k \mathboldae_k \hspace{0.1ex}, \:\: k =\hspace{-0.1ex} 1, 2, 3 \\[.9em]
%
{^2\!\bm{B}} \dotp \tikzmark{BeginEqualsEBrace} {\mathboldae_k \mathboldae_k} \tikzmark{EndEqualsEBrace} = \scalebox{0.9}[0.9]{$\displaystyle \sum_{\smash{k}}$} \hspace{0.1ex} \eigenvalue_k \mathboldae_k \mathboldae_k
%%{^2\!\bm{B}} = \displaystyle \sum_{\smash{k}} \eigenvalue_k \mathboldae_k \mathboldae_k
\end{array}\]
\AddUnderBrace[line width=.75pt]%
{BeginEqualsEBrace}{EndEqualsEBrace}{${\scriptstyle \bm{E}}$}

\vspace{-0.25em} В~общем случае ${B_{ij} = \bm{e}_i \dotp {^2\!\bm{B}} \dotp \bm{e}_j}$, в~базисе~же ${\mathboldae_1}$, ${\mathboldae_2}$, ${\mathboldae_3}$ единичных взаимно ортогональных ${\mathboldae_i \dotp \mathboldae_j = \delta_{ij}}$ собственных осей симметричного тензора:
\[\begin{array}{c}
B_{11} \hspace{-0.12ex} = \mathboldae_1 \dotp \left({\eigenvalue_1 \mathboldae_1 \mathboldae_1 + \eigenvalue_2 \mathboldae_2 \mathboldae_2 + \eigenvalue_3 \mathboldae_3 \mathboldae_3}\right) \dotp \mathboldae_1 = \eigenvalue_1, \\
B_{12} \hspace{-0.12ex} = \mathboldae_1 \dotp \left({\eigenvalue_1 \mathboldae_1 \mathboldae_1 + \eigenvalue_2 \mathboldae_2 \mathboldae_2 + \eigenvalue_3 \mathboldae_3 \mathboldae_3}\right) \dotp \mathboldae_2 = 0 \hspace{0.1ex}, \\[-0.2em]
\ldots
\end{array}\]
\noindent Матрица компонент диагональна и~${{^2\!\bm{B}} = \sum\! \eigenvalue_i \mathboldae_i \mathboldae_i}$. Здесь идёт суммирование по~трём повторяющися индексам, %%и~это естественно,
ведь используется особенный базис.

Случай кратных главных значений можно рассмотреть с~помощью предельного перехода. При~${\eigenvalue_2 \hspace{-0.16ex} \to \eigenvalue_1}$ любая линейная комбинация ${\bm{a}_1}$ и~${\bm{a}_2}$ в~пределе удовлетворяет~\eqref{eigenvalues:eq}; это значит, что любая ось в~плоскости~${\bm{a}_1, \bm{a}_2}$ становится собственной. Если~же совпадают все три собственных числ\'{а}, то любая ось в~пространстве~--- собственная. При~этом ${{^2\!\bm{B}} = \eigenvalue \bm{E}}$, такие тензоры называются изотропными или шаровыми.

\end{otherlanguage}

\en{\section{Rotation tensor}}

\ru{\section{Тензор поворота}}

\label{para:rotationtensor}

\begin{otherlanguage}{russian}

Соотношение между двумя \inquotes{правыми}~(или двумя \inquotes{левыми}) орто\-нормаль\-ными базисами ${\bm{e}_i}$ и~${\mathcircabove{\bm{e}}_i}$ вполне определено матрицей косинусов~(\pararef{para:vectors})
\nopagebreak\vspace{-0.5em}\[
\bm{e}_i \hspace{-0.2ex} = \bm{e}_i \dotp \hspace{0.1ex} \tikzmark{beginEqualsE} \mathcircabove{\bm{e}}_j \mathcircabove{\bm{e}}_j \tikzmark{endEqualsE} \hspace{-0.1ex} = \hspace{0.1ex} \cosinematrix{\hspace{-0.2ex}i\mathcircabove{j}} \, \mathcircabove{\bm{e}}_j , \:\:
\cosinematrix{\hspace{-0.2ex}i\mathcircabove{j}} \hspace{.1ex} \equiv \hspace{.1ex} \bm{e}_i \dotp \mathcircabove{\bm{e}}_j \hspace{0.1ex}.
\]
\AddUnderBrace[line width=.75pt][0,-0.2ex]%
{beginEqualsE}{endEqualsE}{${\scriptstyle \bm{E}}$}

\vspace{-0.32em} \noindent \en{But one may write like this:}\ru{Но~можно напис\'{а}ть и~так:}
\begin{equation}\label{introductionofrotationtensor}
\bm{e}_i = \bm{e}_j \hspace{.16ex} \tikzmark{beginEqualsKroneckerDelta} \mathcircabove{\bm{e}}_j \hspace{-0.16ex} \dotp \mathcircabove{\bm{e}}_i \tikzmark{endEqualsKroneckerDelta} = \bm{P} \hspace{-0.16ex} \dotp \mathcircabove{\bm{e}}_i \hspace{0.1ex}, \:\,
\bm{P} \equiv \bm{e}_j \mathcircabove{\bm{e}}_j = \scalebox{0.96}[1]{$\bm{e}_1 \hspace{-0.1ex} \mathcircabove{\bm{e}}_1 + \bm{e}_2 \mathcircabove{\bm{e}}_2 + \bm{e}_3 \mathcircabove{\bm{e}}_3$} \hspace{0.1ex}.
\end{equation}
\AddUnderBrace[line width=.75pt][0.1ex,-0.2ex]%
{beginEqualsKroneckerDelta}{endEqualsKroneckerDelta}{${\scriptstyle \delta_{ji}}$}

\vspace{-0.32em} \noindent \en{${\bm{P}}$ is called rotation tensor.}\ru{${\bm{P}}$ называется тензором поворота.}

Компоненты ${\bm{P}}$ и~в~начальном~${\mathcircabove{\bm{e}}_i}$, и~в~повёрнутом~${\bm{e}_i}$ базисах образуют одну и~ту~же матрицу, равную транспонированной матрице косинусов~${\cosinematrix{\!j\mathcircabove{i}} = \hspace{.12ex} \mathcircabove{\bm{e}}_i \dotp \bm{e}_j}$\hspace{0.1ex}:
\begin{equation}\label{componentsofrotationtensor}
\begin{array}{c}
\bm{e}_i \dotp \bm{P} \hspace{-0.12ex} \dotp \bm{e}_j =
\hspace{0.2ex} \tikzmark{beginEqualsKroneckerDeltaPresent} \bm{e}_i \hspace{-0.1ex} \dotp \bm{e}_k \tikzmark{endEqualsKroneckerDeltaPresent} \hspace{0.2ex} \mathcircabove{\bm{e}}_k \hspace{-0.12ex} \dotp \bm{e}_j =
\hspace{0.2ex} \mathcircabove{\bm{e}}_i \hspace{-0.12ex} \dotp \bm{e}_j , \\[1.4em]
%
\mathcircabove{\bm{e}}_i \dotp \bm{P} \hspace{-0.12ex} \dotp \mathcircabove{\bm{e}}_j =
\hspace{0.2ex} \mathcircabove{\bm{e}}_i \hspace{-0.1ex} \dotp \bm{e}_k \hspace{0.2ex} \tikzmark{beginEqualsKroneckerDeltaPast} \mathcircabove{\bm{e}}_k \hspace{-0.12ex} \dotp \mathcircabove{\bm{e}}_j \tikzmark{endEqualsKroneckerDeltaPast} =
\hspace{0.2ex} \mathcircabove{\bm{e}}_i \hspace{-0.12ex} \dotp \bm{e}_j , \\[1.5em]
%
\bm{P} = \cosinematrix{\!j\mathcircabove{i}} \hspace{0.5ex} {\bm{e}}_i {\bm{e}}_j = \cosinematrix{\!j\mathcircabove{i}} \hspace{0.5ex} \mathcircabove{\bm{e}}_i \mathcircabove{\bm{e}}_j \hspace{0.1ex}.
\end{array}
\end{equation}
\AddUnderBrace[line width=.75pt][0.1ex,-0.2ex]%
{beginEqualsKroneckerDeltaPresent}{endEqualsKroneckerDeltaPresent}{${\scriptstyle \delta_{ik}}$}
\AddUnderBrace[line width=.75pt][0,-0.2ex]%
{beginEqualsKroneckerDeltaPast}{endEqualsKroneckerDeltaPast}{${\scriptstyle \delta_{kj}}$}

\vspace{-0.4em} Тензор~$\bm{P}$ связывает два вектора~--- \inquotes{до~поворота}~${\mathcircabove{\bm{r}} = \rho_i \mathcircabove{\bm{e}}_i}$ и~\inquotes{после~поворота}~${\bm{r} = \rho_i \bm{e}_i}$~--- с~теми~же компонентами~$\rho_i$ у~$\bm{r}$ в~актуальном повёрнутом базисе~${\bm{e}_i}$, что~и~у~${\mathcircabove{\bm{r}}}$ в~неподвижном базисе~${\mathcircabove{\bm{e}}_i}$~(\inquotes{вектор вращается вместе с~базисом}): поскольку ${\bm{e}_i = \bm{e}_j \mathcircabove{\bm{e}}_j \dotp \mathcircabove{\bm{e}}_i \:\Leftrightarrow\, \rho_i \bm{e}_i = \bm{e}_j \mathcircabove{\bm{e}}_j \dotp \rho_i \mathcircabove{\bm{e}_i}}$, то
\begin{equation}\label{rodriguesrotationformula}
\bm{r} = \bm{P} \dotp\hspace{0.2ex} \mathcircabove{\bm{r}}
\end{equation}

\vspace{-0.4em} \noindent(эта связь~--- обобщённая \href{https://fr.wikipedia.org/wiki/Rotation_vectorielle#Cas_g%C3%A9n%C3%A9ral}{формула поворота Rodrigues’а}).

\vspace{-0.1em}Поворот~же тензора второй сложности ${\mathcircabove{\bm{C}} = C_{ij} \mathcircabove{\bm{e}}_i \mathcircabove{\bm{e}}_j}$ в~текущее~(актуальное) положение~${\bm{C} = C_{ij} \bm{e}_i \bm{e}_j}$ происходит так:
\begin{equation}
C_{ij} \bm{e}_i \bm{e}_j = \bm{e}_i \mathcircabove{\bm{e}}_i \dotp C_{pq} \mathcircabove{\bm{e}}_p \mathcircabove{\bm{e}}_q \dotp \mathcircabove{\bm{e}}_j \bm{e}_j \;\Leftrightarrow\; \bm{C} = \bm{P} \dotp\hspace{0.1ex} \mathcircabove{\bm{C}} \dotp \bm{P}^{\T} \hspace{-0.32ex}.
\end{equation}

Существенное свойство тензора поворота~--- ортогональность~--- выражается равенством
%%\vspace{-1.4em}\begin{flalign*} && \scalebox{0.8}[0.8]{$\bm{P} \dotp \bm{P}^{\T} \!= \bm{e}_i \mathcircabove{\bm{e}}_i \dotp \mathcircabove{\bm{e}}_i \bm{e}_i = \bm{e}_i \bm{e}_i$} \end{flalign*}\vspace{-1.4em}
\begin{equation}\label{orthogonalityofrotationtensor}
\aunderbrace[l1r]{\bm{P}}_{\bm{e}_i \mathcircabove{\bm{e}}_i} \dotp \hspace{0.12em} \aunderbrace[l1r]{\bm{P}^{\T}\hspace{-0.2em}}_{\mathcircabove{\bm{e}}_j \bm{e}_j}
\hspace{0.2ex} = \hspace{0.2ex}
\aunderbrace[l1r]{\bm{P}^{\T}\hspace{-0.2em}}_{\mathcircabove{\bm{e}}_i \bm{e}_i} \hspace{0.2em} \dotp \aunderbrace[l1r]{\bm{P}}_{\bm{e}_j \mathcircabove{\bm{e}}_j}
\hspace{-0.2ex} = \hspace{0.4ex}
\aunderbrace[l1r]{\hspace{0.2ex}\bm{E}\hspace{0.2ex}}_{\mathclap{\begin{subarray}{l} \mathcircabove{\bm{e}}_i \mathcircabove{\bm{e}}_i \\ \bm{e}_i \bm{e}_i \end{subarray}}} \hspace{0.1em},
\end{equation}\vspace{-0.6em}

\noindent то~есть ${\bm{P}^{\T} \hspace{-0.32ex}= \bm{P}^{\expminusone} \Leftrightarrow \bm{P} = \bm{P}^{\expminusT}\hspace{-0.25ex}}$~--- транспонированный тензор совпадает с~обратным.

Ортогональный тензор не~меняет скалярное произведение векторов, сохраняя дл\'{и}ны и~углы (\inquotes{метрику})
%% ${\bm{e}_i \dotp \bm{e}_j \hspace{-0.1ex} = \hspace{.1ex} \mathcircabove{\bm{e}}_i \dotp \mathcircabove{\bm{e}}_j}$
\vspace{0.1em}\begin{equation}\label{rotationtensorkeepsmetrics}
\left( \bm{P} \hspace{-0.2ex} \dotp \bm{a}\hspace{0.2ex} \right) \dotp \left( \bm{P} \hspace{-0.2ex} \dotp \bm{b}\hspace{0.2ex} \right) =
\bm{a} \dotp \bm{P}^{\T} \hspace{-0.4ex} \dotp \bm{P} \hspace{-0.1ex} \dotp \hspace{0.1ex} \bm{b} = \bm{a} \dotp \bm{E} \dotp \bm{b} = \bm{a} \dotp \bm{b}
\hspace{0.1ex}.
\end{equation}

Для всех ортогональных тензоров ${\left(\operatorname{det} \bm{P}\right)^2 \hspace{-0.1ex} = 1}$:
\vspace{0.1em}\[
1 = \operatorname{det} \bm{E} = \operatorname{det} \left({\hspace{-0.1ex} \bm{P} \dotp \bm{P}^{\T} \hspace{0.1ex}}\right) =
\left({\operatorname{det} \bm{P} \hspace{0.1ex}}\right) \left({\operatorname{det} \bm{P}^{\T} \hspace{0.2ex}}\right) =
\left(\operatorname{det} \bm{P} \hspace{0.1ex} \right)^2 \hspace{-0.1ex} \!.
\]

Тензор поворота это ортогональный тензор с~${\operatorname{det} \bm{P} \hspace{-0.1ex} = 1}$. Но не~только лишь тензоры поворота обладают свойством ортогональности. Когда в~\eqref{introductionofrotationtensor} один из базисов \inquotesx{правый}[,] а~другой \inquotesx{левый}[,] имеем комбинацию поворота с~отражением, и~${\operatorname{det} \left( -\bm{E} \dotp \bm{P} \hspace{0.16ex} \right) \hspace{-0.1ex} = -1}$.

У~любого тензора второй сложности в~трёхмерном пространстве как минимум одно собственное число~--- корень~\eqref{chardetequation}~--- действительное~(некомпл\'{е}ксное).
Для тензора поворота оно равно единице

\nopagebreak\vspace{-0.1em}\begin{equation*}\begin{array}{c}
\bm{P} \dotp \bm{a} = \eigenvalue \bm{a} \:\Rightarrow\:
\tikzmark{BeginPaBrace} \bm{a} \hspace{0.16ex} \dotp \tikzmark{BeginEBrace} \bm{P}^{\T} \tikzmark{EndPaBrace} \hspace{-0.4ex} \dotp \bm{P} \hspace{-0.32ex}\tikzmark{EndEBrace}\hspace{0.32ex} \dotp \hspace{0.16ex} \bm{a} = \eigenvalue \bm{a} \dotp \eigenvalue \bm{a}
\:\Rightarrow\: \eigenvalue^{2} \hspace{-0.2ex} = 1 \hspace{.1ex} .
\end{array}\end{equation*}
\AddOverBrace[line width=.75pt][0.1ex,0.4ex]{BeginPaBrace}{EndPaBrace}{${\scriptstyle \bm{P} \:\dotp\; \bm{a}}$}
\AddUnderBrace[line width=.75pt][-0.1ex,0.1ex]{BeginEBrace}{EndEBrace}{${\scriptstyle \bm{E}}$}

\vspace{-0.5em} \noindent Соответствующая собственная ось называется осью поворота; теорема Euler’а о~конечном повороте в~том и~состоит, что такая ось существует. Если ${\bm{k}}$~--- орт этой оси, а~${\vartheta}$~--- величина угла поворота, то тензор поворота представ\'{и}м как

\nopagebreak\vspace{-0.1em}\begin{equation}\label{eulerfiniterotation}
\bm{P}\hspace{0.1ex}(\bm{k},\vartheta) = \bm{E} \operatorname{cos} \vartheta + \bm{k} \times\hspace{-0.2ex} \bm{E} \operatorname{sin} \vartheta + \bm{k} \bm{k} \left({1 - \operatorname{cos} \vartheta}\right) \hspace{-0.2ex} .
\end{equation}

\vspace{-0.1em} Доказывается эта формула так. Направление~${\bm{k}}$ при~повороте не~меняется~(${\bm{P} \hspace{-0.2ex}\dotp \bm{k} = \bm{k}\hspace{0.12ex}}$), поэтому на~оси поворота ${\mathcircabove{\bm{e}}_3 \hspace{-0.16ex} = \bm{e}_3 \hspace{-0.16ex} = \bm{k}}$. В~перпендикулярной плоскости~(\figref{fig:eulerfiniterotation}) ${\mathcircabove{\bm{e}}_1 \hspace{-0.16ex} = \bm{e}_1 \operatorname{cos} \vartheta - \bm{e}_2 \operatorname{sin} \vartheta}$, ${\mathcircabove{\bm{e}}_2 \hspace{-0.16ex} = \bm{e}_1 \operatorname{sin} \vartheta + \bm{e}_2 \operatorname{cos} \vartheta}$, ${\bm{P} = \bm{e}_i \mathcircabove{\bm{e}}_i \,\Rightarrow\hspace{.2ex}}$~\eqref{eulerfiniterotation}.

% ~ ~ ~ ~ ~
\begin{figure}[!htbp]

\vspace*{-0.5em}\[
\mathcircabove{\bm{e}}_i = \mathcircabove{\bm{e}}_i \dotp \bm{e}_j \bm{e}_j
\]

\vspace{-1.5em}\[
\left[ \begin{array}{c} \mathcircabove{\bm{e}}_1 \\ \mathcircabove{\bm{e}}_2 \\ \mathcircabove{\bm{e}}_3 \end{array} \right] =
\left[ \begin{array}{ccc}
\mathcircabove{\bm{e}}_1 \dotp \bm{e}_1 & \mathcircabove{\bm{e}}_1 \dotp \bm{e}_2 & \mathcircabove{\bm{e}}_1 \dotp \bm{e}_3 \\
\mathcircabove{\bm{e}}_2 \dotp \bm{e}_1 & \mathcircabove{\bm{e}}_2 \dotp \bm{e}_2 & \mathcircabove{\bm{e}}_2 \dotp \bm{e}_3 \\
\mathcircabove{\bm{e}}_3 \dotp \bm{e}_1 & \mathcircabove{\bm{e}}_3 \dotp \bm{e}_2 & \mathcircabove{\bm{e}}_3 \dotp \bm{e}_3
\end{array} \right] \hspace{-0.5ex}
\left[ \hspace{-0.12ex} \begin{array}{c} {\bm{e}}_1 \\ {\bm{e}}_2 \\ {\bm{e}_3} \end{array} \right]
\]

\vspace{-1.25em}

\begin{center}
\tdplotsetmaincoords{60}{120} % set orientation of axes
\pgfmathsetmacro{\angletheta}{42}
% three parameters for vector
\pgfmathsetmacro{\lengthofvector}{0.55}
\pgfmathsetmacro{\anglefromz}{40}
\pgfmathsetmacro{\anglefromx}{240}

\begin{tikzpicture}[scale=4, tdplot_main_coords] % tdplot_main_coords style to use 3dplot

	\coordinate (O) at (0,0,0);

	% draw initial axes
	\draw [line width=1.2pt, black, -{Stealth[round, length=4mm, width=2.4mm]}]
		(O) -- (1,0,0)
		node[pos=0.9, above, xshift=-0.8em] {$\mathcircabove{\bm{e}}_1$};

	\draw [line width=1.2pt, black, -{Stealth[round, length=4mm, width=2.4mm]}]
		(O) -- (0,1,0)
		node[pos=0.9, above, xshift=1em, yshift=-0.2em] {$\mathcircabove{\bm{e}}_2$};

	\draw [line width=1.2pt, red, -{Stealth[round,length=4mm,width=2.4mm]}]
		(O) -- (0,0,0.9)
		node[anchor=south] {$\mathcircabove{\bm{e}}_3 = \bm{e}_3 = \bm{k}$};

	% draw initial vector
	\tdplotsetcoord{point}{\lengthofvector}{\anglefromz}{\anglefromx} % {length}{angle from z}{angle from x}
		% it also defines (pointxy), (pointxz), and (pointyz) projections of point
	\draw [line width=1.2pt, black, -{Stealth[round, length=4mm, width=2.4mm]}]
		(O) -- (point)
		node[anchor=south] {$\mathcircabove{\bm{r}}$};
	% draw its projection on xy plane
	\draw [line width=0.4pt, dotted, color=black] (O) -- (pointxy);
	\draw [line width=0.4pt, dotted, color=black] (pointxy) -- (point);

	% draw the angle, and label it
	% syntax: \tdplotdrawarc[coordinate frame, draw options]{center point}{r}{angle}{end angle}{label options}{label}
	\tdplotdrawarc [line width=0.5pt, red, ->]
		{(O)}{0.4}{0}{\angletheta}{anchor=north}{$\vartheta$}
	\tdplotdrawarc [line width=0.5pt, red, ->]
		{(O)}{0.4}{90}{90+\angletheta}{anchor=west}{$\vartheta$}

	% rotate coordinates using Euler angles "z(\alpha)y(\beta)z(\gamma)"
	\tdplotsetrotatedcoords{\angletheta}{0}{0}

	% draw rotated axes
	\draw [line width=1.2pt, blue, tdplot_rotated_coords, -{Stealth[round, length=4mm, width=2.4mm]}]
		(O) -- (1,0,0)
		node[pos=0.9, left, xshift=-0.1em] {$\bm{e}_1$};

	\draw [line width=1.2pt, blue, tdplot_rotated_coords, -{Stealth[round, length=4mm, width=2.4mm]}]
		(O) -- (0,1,0)
		node[pos=0.9, above, xshift=0.2em, yshift=0.2em] {$\bm{e}_2$};

	%%\draw [line width=1.2pt, blue, tdplot_rotated_coords, -{Stealth[round, length=4mm, width=2.4mm]}]
		%%(O) -- (0,0,0.8) ;

	% draw rotated vector
	\tdplotsetcoord{rotatedpoint}%
		{\lengthofvector}{\anglefromz}{\anglefromx+\angletheta}
	\draw [line width=1.2pt, blue, tdplot_rotated_coords, -{Stealth[round, length=4mm, width=2.4mm]}]
		(O) -- (rotatedpoint)
		node[anchor=south] {$\bm{r}$};
	% draw its projection on xy plane
	\draw [line width=0.4pt, dotted, color=blue, tdplot_rotated_coords] (O) -- (rotatedpointxy);
	\draw [line width=0.4pt, dotted, color=blue, tdplot_rotated_coords] (rotatedpointxy) -- (rotatedpoint);

	\tdplotdrawarc [line width=0.5pt, red, ->]
		{(O)}{0.28}{\anglefromx}{\anglefromx+\angletheta}{anchor=south east, xshift=0.3em, yshift=-0.1em}{$\vartheta$}

\end{tikzpicture}
\end{center}

\vspace{-1em}\[
\scalebox{0.8}[0.85]{$\left[ \begin{array}{ccc}
\mathcircabove{\bm{e}}_1 \dotp \bm{e}_1 & \mathcircabove{\bm{e}}_1 \dotp \bm{e}_2 & \mathcircabove{\bm{e}}_1 \dotp \bm{e}_3 \\
\mathcircabove{\bm{e}}_2 \dotp \bm{e}_1 & \mathcircabove{\bm{e}}_2 \dotp \bm{e}_2 & \mathcircabove{\bm{e}}_2 \dotp \bm{e}_3 \\
\mathcircabove{\bm{e}}_3 \dotp \bm{e}_1 & \mathcircabove{\bm{e}}_3 \dotp \bm{e}_2 & \mathcircabove{\bm{e}}_3 \dotp \bm{e}_3
\end{array} \right]$} \hspace{-0.32ex} = \hspace{-0.2ex}
%
\scalebox{0.8}[0.85]{$\left[ \hspace{-0.2ex} \begin{array}{ccc}
\operatorname{cos} \vartheta & \hspace{-1ex} \operatorname{cos} \left( 90\degree \!+ \vartheta \right) & \operatorname{cos} 90\degree \\
\operatorname{cos} \left( 90\degree \!- \vartheta \right) & \operatorname{cos} \vartheta & \operatorname{cos} 90\degree \\
\operatorname{cos} 90\degree & \operatorname{cos} 90\degree & \operatorname{cos} 0\degree
\end{array} \right]$} \hspace{-0.32ex} = \hspace{-0.2ex}
%
\scalebox{0.8}[0.85]{$\left[ \hspace{-0.1ex} \begin{array}{ccc}
\operatorname{cos} \vartheta & - \operatorname{sin} \vartheta & 0 \\
\operatorname{sin} \vartheta & \operatorname{cos} \vartheta & 0 \\
0 & 0 & 1
\end{array} \right]$}
\]

\vspace{-0.8em}
\[\begin{array}{c}
\mathcircabove{\bm{e}}_1 \hspace{-0.16ex} = \bm{e}_1 \operatorname{cos} \vartheta \hspace{0.1ex} - \hspace{0.1ex} \bm{e}_2 \operatorname{sin} \vartheta \\[0.1em]
\mathcircabove{\bm{e}}_2 \hspace{-0.16ex} = \bm{e}_1 \operatorname{sin} \vartheta \hspace{0.1ex} + \hspace{0.1ex} \bm{e}_2 \operatorname{cos} \vartheta \\[0.1em]
\mathcircabove{\bm{e}}_3 \hspace{-0.16ex} = \bm{e}_3 = \bm{k}
\end{array}\]

\vspace{-1em}
\begin{multline*}
\shoveleft{ \bm{P} = \bm{e}_1 \hspace{-0.1ex} \mathcircabove{\bm{e}}_1 + \bm{e}_2 \mathcircabove{\bm{e}}_2 + \bm{e}_3 \mathcircabove{\bm{e}}_3 = \hfill }\\[1.5em]
%
= \hspace{0.2ex} \tikzmark{StartBraceE1E1} {\bm{e}_1 \bm{e}_1 \operatorname{cos} \vartheta - \bm{e}_1 \bm{e}_2 \operatorname{sin} \vartheta \hspace{0.2em}} \tikzmark{EndBraceE1E1} \hspace{-0.1ex} + \hspace{0.1ex} \tikzmark{StartBraceE2E2} {\bm{e}_2 \bm{e}_1 \operatorname{sin} \vartheta + \bm{e}_2 \bm{e}_2 \operatorname{cos} \vartheta \hspace{0.2em}} \tikzmark{EndBraceE2E2} \hspace{-0.1ex} + \tikzmark{StartBraceE3E3} {\hspace{0.25ex} \bm{k} \bm{k} \hspace{0.1ex}} \tikzmark{EndBraceE3E3} \hspace{0.1ex} =\\[0.32em]
%
= \hspace{0.1ex} \bm{E} \operatorname{cos} \vartheta - \hspace{-0.1ex} \tikzmark{StartBraceKk} {\hspace{0.1ex}\bm{e}_3 \bm{e}_3\hspace{0.1ex}} \tikzmark{EndBraceKk} \hspace{-0.25ex} \operatorname{cos} \vartheta \hspace{0.1ex} + \tikzmark{StartBraceLeviCivita} {\left( \bm{e}_2 \bm{e}_1 - \bm{e}_1 \bm{e}_2 \right)} \tikzmark{EndBraceLeviCivita} \operatorname{sin} \vartheta + \bm{k} \bm{k} \hspace{0.1ex} =\\[1.5em]
%
\shoveright{ \hfill = \bm{E} \operatorname{cos} \vartheta + \bm{k} \times\hspace{-0.2ex} \bm{E} \operatorname{sin} \vartheta + \bm{k} \bm{k} \left({1 - \operatorname{cos} \vartheta}\right) }
\end{multline*}

\AddOverBrace[line width=0.75pt]{StartBraceE1E1}{EndBraceE1E1}{${\scriptstyle \bm{e}_1 \mathcircabove{\bm{e}}_1}$}
\AddOverBrace[line width=0.75pt]{StartBraceE2E2}{EndBraceE2E2}{${\scriptstyle \bm{e}_2 \mathcircabove{\bm{e}}_2}$}
\AddOverBrace[line width=0.75pt]{StartBraceE3E3}{EndBraceE3E3}{${\scriptstyle \bm{e}_3 \mathcircabove{\bm{e}}_3}$}
\AddUnderBrace[line width=0.75pt][-0.1ex,-0.2ex]{StartBraceKk}{EndBraceKk}{${\scriptstyle \bm{k}\bm{k}}$}
\AddUnderBrace[line width=0.75pt][-0.1ex,-0.2ex][xshift=0.4ex]{StartBraceLeviCivita}{EndBraceLeviCivita}{${\scriptstyle \bm{e}_3 \times \bm{e}_i \bm{e}_i \:=\: \levicivita_{3ij} \bm{e}_j \bm{e}_i}$}

\vspace{-0.5em}
\caption{\inquotes{\en{Finite rotation}\ru{Конечный поворот}}}\label{fig:eulerfiniterotation}
\end{figure}

% ~ ~ ~ ~ ~

Из~\eqref{eulerfiniterotation} и~\eqref{rodriguesrotationformula} получаем формулу поворота Родрига в~параметрах~$\bm{k}$ и~$\vartheta$:
\nopagebreak\vspace{-0.2em}\begin{equation*}
\bm{r} \hspace{0.4ex}=\hspace{0.5ex} \mathcircabove{\bm{r}} \operatorname{cos} \vartheta \hspace{0.4ex}+\hspace{0.4ex} \bm{k} \times \mathcircabove{\bm{r}} \hspace{0.2ex} \operatorname{sin} \vartheta \hspace{0.4ex}+\hspace{0.5ex} \bm{k} \bm{k} \dotp\hspace{0.1ex} \mathcircabove{\bm{r}} \left({1 - \operatorname{cos} \vartheta}\right) \hspace{-0.25ex}.
\end{equation*}

\vspace{-0.16em} В~параметрах конечного поворота транспонирование, оно~же обращение, тензора~$\bm{P}$ эквивалентно перемене направления поворота~--- знака угла~$\vartheta$
\[
\bm{P}^{\T} \hspace{-0.1ex}=\hspace{0.1ex} \bm{P} \hspace{0.1ex} \bigr|_{\vartheta \,=\hspace{0.1ex} -\vartheta} \hspace{-0.1ex} = \bm{E} \operatorname{cos} \vartheta - \bm{k} \times\hspace{-0.2ex} \bm{E} \operatorname{sin} \vartheta + \bm{k} \bm{k} \left({1 - \operatorname{cos} \vartheta}\right) \hspace{-0.32ex}.
\]

Пусть теперь тензор поворота меняется со~временем: ${\bm{P} \!=\! \bm{P}(t)}$.
Псевдовектор угловой скорости~${\bm{\omega}}$ вводится через~$\bm{P}$ таким путём.
Дифференцируем тождество ортогональности~\eqref{orthogonalityofrotationtensor} по~времени\footnote{A~variety of notations are used to denote the full time derivative. In~addition to the Leibniz’s notation ${\frac{dx}{dt}}$, very popular short-hand notation is the \inquotes{over-dot} Newton’s notation ${\mathdotabove{x}}$.}

\nopagebreak\vspace{-0.1em}\begin{equation*}
\mathdotabove{\bm{P}} \dotp \bm{P}^{\T} \hspace{-0.1ex} + \hspace{0.25ex} \bm{P} \dotp \mathdotabove{\bm{P}}^{\T} \hspace{-0.1ex} = \hspace{0.1ex} {^2\bm{0}}
\hspace{.1ex} .
\end{equation*}

Тензор ${\mathdotabove{\bm{P}} \dotp \bm{P}^{\T\!}}$ (по~\eqref{transposeofdotproduct} ${\left({ \mathdotabove{\bm{P}} \dotp \bm{P}^{\T} }\right)^{\raisemath{-0.25em}{\!\T}} \hspace{-0.36ex} = \bm{P} \dotp \mathdotabove{\bm{P}}^{\T}}$) оказался анти\-сим\-метрич\-ным, и~согласно~\eqref{companionvector} он представ\'{и}м сопутствующим вектором как ${\mathdotabove{\bm{P}} \dotp \bm{P}^{\T\!} = \bm{\omega} \times \bm{E} = \bm{\omega} \times \bm{P} \dotp \bm{P}^{\T}}$\!. То~есть
\vspace{0.1em}\begin{equation}\label{angularvelocityvector}
\mathdotabove{\bm{P}} = \bm{\omega} \times \bm{P}, \;\:\:
\bm{\omega} \equiv -\, \displaystyle \onehalf \left( \mathdotabove{\bm{P}} \dotp \bm{P}^{\T} \right)_{\hspace{-0.2em}\Xcompanion}
\vspace{-0.25em}\end{equation}

Помимо этого общего представления~вектора~${\bm{\omega}}$, для~него есть и~другие. Например, через параметры конечного поворота.

Производная~${\mathdotabove{\bm{P}}}$ в~параметрах конечного поворота в~общем случае (оба параметра~--- и~единичный вектор~$\bm{k}$, и~угол~$\vartheta$~--- переменны во~времени):
\vspace{0.32em}%
\[\begin{array}{r@{\hspace{0.33em}}c@{\hspace{0.25em}}l}
\mathdotabove{\bm{P}} \hspace{0.2em} & = & \hspace{0.1em} \left(\bm{P}^{\mathsf{\hspace{0.12ex}S}} \hspace{-0.2ex} +^{\mathstrut} \bm{P}^{\mathsf{\hspace{0.12ex}A}}\right)^{\hspace{-0.2ex}\tikz[baseline=-0.5ex]\draw[black, fill=black] (0,0) circle (.266ex);} =
\hspace{0.1em} \left(\hspace{0.2ex} \tikzmark{StartBracePs} {\bm{E} \operatorname{cos} \vartheta + \bm{k} \bm{k} \left({1 \!-\! \operatorname{cos} \vartheta}\right)} \hspace{-0.2ex} \tikzmark{EndBracePs} \hspace{0.32ex} +^{\mathstrut} \hspace{0.2ex}
\tikzmark{StartBracePa} {\bm{k} \hspace{-0.24ex}\times\hspace{-0.4ex} \bm{E} \operatorname{sin} \vartheta \hspace{0.2ex}} \tikzmark{EndBracePa} {} \hspace{0.16ex}\right)^{\hspace{-0.2ex}\tikz[baseline=-0.5ex]\draw[black, fill=black] (0,0) circle (.266ex);} \hspace{-0.05em} = \\[0.4em]
%
& = & \hspace{0.2em} \tikzmark{StartBraceDotPs} {\left( \hspace{0.1ex} \bm{k} \bm{k} \hspace{-0.1ex} - \hspace{-0.2ex} \bm{E} \hspace{0.1ex} \right) \hspace{-0.1ex} \mathdotabove{\vartheta} \operatorname{sin} \vartheta + \hspace{-0.2ex} ( \bm{k} \mathdotabove{\bm{k}} + \mathdotabove{\bm{k}} \bm{k} ) \hspace{-0.2ex} \left({1 \!-\! \operatorname{cos} \vartheta}\right)} \tikzmark{EndBraceDotPs} \hspace{0.64ex} + \\[0.64em]
& & \hspace{13.2em} + \hspace{0.72ex} \tikzmark{StartBraceDotPa} {\hspace{0.12ex} \bm{k} \hspace{-0.24ex}\times\hspace{-0.4ex} \bm{E} \hspace{0.4ex} \mathdotabove{\vartheta} \operatorname{cos} \vartheta + \mathdotabove{\bm{k}} \hspace{-0.24ex}\times\hspace{-0.4ex} \bm{E} \operatorname{sin} \vartheta} \tikzmark{EndBraceDotPa} \hspace{0.1em}.
\end{array}\]%
\vspace{-1.2em}

\AddOverBrace[line width=0.75pt][0.12ex,0]{StartBracePs}{EndBracePs}{${\scriptstyle \bm{P}^{\mathsf{\hspace{0.12ex}S}}}$}
\AddOverBrace[line width=0.75pt][-0.12ex,0]{StartBracePa}{EndBracePa}{${\scriptstyle \bm{P}^{\mathsf{\hspace{0.12ex}A}}}$}
\AddUnderBrace[line width=0.75pt]{StartBraceDotPs}{EndBraceDotPs}{${\scriptstyle \mathdotabove{\bm{P}}^{\mathsf{\hspace{0.12ex}S}}}$}
\AddUnderBrace[line width=0.75pt][-0.1ex,0.1ex]{StartBraceDotPa}{EndBraceDotPa}{${\scriptstyle \mathdotabove{\bm{P}}^{\mathsf{\hspace{0.12ex}A}}}$}

\vspace{-1.32em} \noindent Находим
\vspace{0.2em}\[\begin{array}{r@{\hspace{0.25em}}c@{\hspace{0.4em}}l}
\mathdotabove{\bm{P}} \dotp \bm{P}^{\T} & = & (\hspace{0.1em} \mathdotabove{\bm{P}}^{\mathsf{\hspace{0.12ex}S}} \hspace{-0.16ex} + \mathdotabove{\bm{P}}^{\mathsf{\hspace{0.12ex}A}} \hspace{0.05em}) \hspace{-0.1ex} \dotp \hspace{-0.1ex} (\hspace{0.1em} \bm{P}^{\mathsf{\hspace{0.12ex}S}} \hspace{-0.16ex} - \bm{P}^{\mathsf{\hspace{0.12ex}A}} \hspace{0.1ex} \hspace{0.05em}) = \\[0.25em]
& = & \mathdotabove{\bm{P}}^{\mathsf{\hspace{0.12ex}S}} \hspace{-0.2ex}\dotp \bm{P}^{\mathsf{\hspace{0.12ex}S}}
+ \hspace{0.2ex} \mathdotabove{\bm{P}}^{\mathsf{\hspace{0.12ex}A}} \hspace{-0.2ex}\dotp \bm{P}^{\mathsf{\hspace{0.12ex}S}}
- \hspace{0.2ex} \mathdotabove{\bm{P}}^{\mathsf{\hspace{0.12ex}S}} \hspace{-0.2ex}\dotp \bm{P}^{\mathsf{\hspace{0.12ex}A}}
- \hspace{0.2ex} \mathdotabove{\bm{P}}^{\mathsf{\hspace{0.12ex}A}} \hspace{-0.2ex}\dotp \bm{P}^{\mathsf{\hspace{0.12ex}A}} ,
\end{array}\]

\vspace{-0.5em} \noindent используя
\[\scalebox{0.95}[0.96]{$\begin{array}{c}
\bm{k} \dotp \bm{k} = 1 = \const \,\Rightarrow\:
\bm{k} \dotp \mathdotabove{\bm{k}} + \mathdotabove{\bm{k}} \dotp \bm{k} = 0 \;\Leftrightarrow\; \mathdotabove{\bm{k}} \dotp \bm{k} = \bm{k} \dotp \mathdotabove{\bm{k}} = 0 \hspace{0.1ex} ,
\\[.08em]
%
\bm{k} \bm{k} \hspace{-0.2ex}\dotp\hspace{-0.2ex} \bm{k} \bm{k} = \bm{k} \bm{k}, \:\:
\mathdotabove{\bm{k}} \bm{k} \hspace{-0.2ex}\dotp\hspace{-0.2ex} \bm{k} \bm{k} = \mathdotabove{\bm{k}} \bm{k} , \:\:
\bm{k} \mathdotabove{\bm{k}} \hspace{-0.2ex}\dotp\hspace{-0.2ex} \bm{k} \bm{k} = {\hspace{-0.2ex}^2\bm{0}} \hspace{0.1ex},
\\[0.16em]
%
\left( \hspace{0.1ex} \bm{k} \bm{k} \hspace{-0.1ex} - \hspace{-0.2ex} \bm{E} \hspace{0.1ex} \right) \hspace{-0.2ex} \dotp \bm{k} = \bm{k} - \bm{k} = {\bm{0}} \hspace{0.1ex}, \,\,
\left( \hspace{0.1ex} \bm{k} \bm{k} \hspace{-0.1ex} - \hspace{-0.2ex} \bm{E} \hspace{0.1ex} \right) \hspace{-0.2ex} \dotp \bm{k} \bm{k} = \bm{k} \bm{k} - \bm{k} \bm{k} = {\hspace{-0.2ex}^2\bm{0}} \hspace{0.1ex},
\\[0.08em]
%
\bm{k} \dotp \hspace{-0.1ex} ( \bm{k} \hspace{-0.24ex}\times\hspace{-0.4ex} \bm{E} ) \hspace{-0.1ex}
= \hspace{-0.1ex} ( \bm{k} \hspace{-0.24ex}\times\hspace{-0.4ex} \bm{E} ) \hspace{-0.2ex} \dotp \bm{k}
= \bm{k} \hspace{-0.24ex}\times\hspace{-0.2ex} \bm{k} = \bm{0} \hspace{0.1ex}, \,\,
\bm{k} \bm{k} \dotp \hspace{-0.1ex} ( \bm{k} \hspace{-0.24ex}\times\hspace{-0.4ex} \bm{E} ) \hspace{-0.1ex}
= \hspace{-0.1ex} ( \bm{k} \hspace{-0.24ex}\times\hspace{-0.4ex} \bm{E} ) \hspace{-0.2ex} \dotp \bm{k} \bm{k}
= {\hspace{-0.2ex}^2\bm{0}} \hspace{0.1ex},
\\[0.08em]
%
\left( \hspace{0.1ex} \bm{k} \bm{k} \hspace{-0.1ex} - \hspace{-0.2ex} \bm{E} \hspace{0.1ex} \right) \hspace{-0.2ex} \dotp \hspace{-0.1ex} ( \bm{k} \hspace{-0.24ex}\times\hspace{-0.4ex} \bm{E} ) \hspace{-0.1ex} = \hspace{-0.1ex}
- \hspace{0.2ex} \bm{k} \hspace{-0.24ex}\times\hspace{-0.4ex} \bm{E} \hspace{0.1ex},
\\
%
( \bm{a} \hspace{-0.24ex}\times\hspace{-0.4ex} \bm{E} ) \hspace{-0.2ex} \dotp \bm{b} = \bm{a} \hspace{-0.16ex}\times\hspace{-0.12ex} \bm{b} \:\,\Rightarrow\,
( \mathdotabove{\bm{k}} \hspace{-0.24ex}\times\hspace{-0.4ex} \bm{E} ) \hspace{-0.2ex} \dotp \bm{k} \bm{k} = \mathdotabove{\bm{k}} \hspace{-0.16ex}\times\hspace{-0.2ex} \bm{k} \bm{k} \hspace{0.1ex},
\end{array}$}\]

\vspace{-0.4em} \noindent \eqref{vectorcrossidentitydotvectorcrossidentity} $\,\Rightarrow\,$
\scalebox{0.95}[0.96]{${( \bm{k} \hspace{-0.24ex}\times\hspace{-0.4ex} \bm{E} ) \hspace{-0.2ex} \dotp \hspace{-0.2ex} ( \bm{k} \hspace{-0.24ex}\times\hspace{-0.4ex} \bm{E} ) \hspace{-0.12ex} = \bm{k} \bm{k} \hspace{-0.1ex} - \hspace{-0.2ex} \bm{E}}$},\hspace{0.4ex}
%
\scalebox{0.95}[0.96]{${\displaystyle ( \mathdotabove{\bm{k}} \hspace{-0.24ex}\times\hspace{-0.4ex} \bm{E} ) \hspace{-0.2ex} \dotp \hspace{-0.2ex} ( \bm{k} \hspace{-0.24ex}\times\hspace{-0.4ex} \bm{E} ) \hspace{-0.12ex} = \bm{k} \mathdotabove{\bm{k}} \hspace{-0.1ex} - \tikzbackcancel[black!25]{$\mathdotabove{\bm{k}} \hspace{-0.2ex}\dotp\hspace{-0.2ex} \bm{k} \hspace{0.16ex} \bm{E}$}}$\hspace{0.16ex}},

\noindent \eqref{vectorcrossvectorcrossidentity} $\,\Rightarrow\,$
\scalebox{0.95}[0.96]{$\mathdotabove{\bm{k}} \bm{k} \hspace{-0.1ex} - \hspace{-0.1ex} \bm{k} \mathdotabove{\bm{k}} = \hspace{-0.16ex} ( \bm{k} \hspace{-0.2ex} \times \hspace{-0.24ex} \mathdotabove{\bm{k}} ) \hspace{-0.32ex} \times \hspace{-0.32ex} \bm{E}$},\hspace{0.4ex}
%
\scalebox{0.95}[0.96]{${\displaystyle ( \mathdotabove{\bm{k}} \hspace{-0.2ex}\times\hspace{-0.24ex} \bm{k} ) \hspace{.2ex} \bm{k} \hspace{-0.1ex} - \hspace{-0.1ex} \bm{k} \hspace{.16ex} ( \mathdotabove{\bm{k}} \hspace{-0.2ex}\times\hspace{-0.24ex} \bm{k} ) \hspace{-0.12ex} = \bm{k} \hspace{-0.2ex} \times \hspace{-0.32ex} ( \mathdotabove{\bm{k}} \hspace{-0.2ex}\times\hspace{-0.24ex} \bm{k} ) \hspace{-0.32ex} \times \hspace{-0.32ex} \bm{E}}$\hspace{0.16ex}}

\begin{fleqn}[0pt]
\begin{multline*}
\shoveleft{\scalebox{0.94}[0.96]{$\mathdotabove{\bm{P}}^{\mathsf{\hspace{0.12ex}S}} \hspace{-0.2ex}\dotp \bm{P}^{\mathsf{\hspace{0.12ex}S}} \hspace{-0.25ex} = $} \hspace{2em} \hfill} \\[-0.25em]
%
\shoveleft{\scalebox{0.8}[0.82]{$= \hspace{0.2ex} \left( \hspace{0.1ex} \bm{k} \bm{k} \hspace{-0.1ex} - \hspace{-0.2ex} \bm{E} \hspace{0.1ex} \right) \hspace{-0.1ex} \mathdotabove{\vartheta} \operatorname{sin} \vartheta \dotp \bm{E} \operatorname{cos} \vartheta +
( \bm{k} \mathdotabove{\bm{k}} + \mathdotabove{\bm{k}} \bm{k} ) \hspace{-0.2ex} \left({1 \!-\! \operatorname{cos} \vartheta}\right) \dotp \bm{E} \operatorname{cos} \vartheta \hspace{0.32em} +$} \hfill} \\[-0.2em]
\shoveright{\hfill \scalebox{0.8}[0.82]{$+\; \tikzbackcancel[black!25]{$\left( \hspace{0.1ex} \bm{k} \bm{k} \hspace{-0.1ex} - \hspace{-0.2ex} \bm{E} \hspace{0.1ex} \right) \hspace{-0.1ex} \mathdotabove{\vartheta} \operatorname{sin} \vartheta \dotp \bm{k} \bm{k} \left({1 \!-\! \operatorname{cos} \vartheta}\right)$} \hspace{0.2ex} +
( \bm{k} \mathdotabove{\bm{k}} + \mathdotabove{\bm{k}} \bm{k} ) \hspace{-0.2ex} \left({1 \!-\! \operatorname{cos} \vartheta}\right) \hspace{-0.2ex} \dotp \bm{k} \bm{k} \left({1 \!-\! \operatorname{cos} \vartheta}\right) =$}} \\
%
\scalebox{0.8}[0.82]{$= \left( \hspace{0.1ex} \bm{k} \bm{k} \hspace{-0.1ex} - \hspace{-0.2ex} \bm{E} \hspace{0.1ex} \right) \hspace{-0.1ex} \mathdotabove{\vartheta} \operatorname{sin} \vartheta \operatorname{cos} \vartheta
+ ( \bm{k} \mathdotabove{\bm{k}} + \mathdotabove{\bm{k}} \bm{k} ) \hspace{-0.1ex} \operatorname{cos} \vartheta \left({1 \!-\! \operatorname{cos} \vartheta}\right) + ( \tikzbackcancel[black!25]{$\bm{k} \mathdotabove{\bm{k}} \hspace{-0.1ex}\dotp\hspace{-0.1ex} \bm{k} \bm{k}$} + \mathdotabove{\bm{k}} \bm{k} \hspace{-0.1ex}\dotp\hspace{-0.1ex} \bm{k} \bm{k} ) \left({1 \!-\! \operatorname{cos} \vartheta}\right)^{\hspace{-0.12ex}2} \hspace{-0.25ex} =$} \\
%
\shoveleft{\scalebox{0.8}[0.82]{$= \left( \hspace{0.1ex} \bm{k} \bm{k} \hspace{-0.1ex} - \hspace{-0.2ex} \bm{E} \hspace{0.1ex} \right) \hspace{-0.1ex} \mathdotabove{\vartheta} \operatorname{sin} \vartheta \operatorname{cos} \vartheta + \bm{k} \mathdotabove{\bm{k}} \operatorname{cos} \vartheta \left({1 \!-\! \operatorname{cos} \vartheta}\right) +$} \hfill} \\[-0.2em]
\shoveright{\hfill \scalebox{0.8}[0.82]{$+ \hspace{0.24em} \mathdotabove{\bm{k}} \bm{k} \operatorname{cos} \vartheta - \mathdotabove{\bm{k}} \bm{k} \operatorname{cos}^{2\hspace{-0.4ex}} \vartheta + \mathdotabove{\bm{k}} \bm{k} - 2 \, \mathdotabove{\bm{k}} \bm{k} \operatorname{cos} \vartheta + \mathdotabove{\bm{k}} \bm{k} \operatorname{cos}^{2\hspace{-0.4ex}} \vartheta =$}}\\
%
%% \shoveright{\hfill \scalebox{0.8}[0.82]{$= \left( \hspace{0.1ex} \bm{k} \bm{k} \hspace{-0.1ex} - \hspace{-0.2ex} \bm{E} \hspace{0.1ex} \right) \hspace{-0.1ex} \mathdotabove{\vartheta} \operatorname{sin} \vartheta \operatorname{cos} \vartheta \hspace{0.1ex}
%% + \hspace{0.1ex} \bm{k} \mathdotabove{\bm{k}} \operatorname{cos} \vartheta \left({1 \!-\! \operatorname{cos} \vartheta}\right)
%% + \mathdotabove{\bm{k}} \bm{k}
%% - \mathdotabove{\bm{k}} \bm{k} \operatorname{cos} \vartheta =$}}\\
%
\shoveright{\hfill \hspace{4.8em} \scalebox{0.94}[0.96]{$= \left( \hspace{0.1ex} \bm{k} \bm{k} \hspace{-0.1ex} - \hspace{-0.2ex} \bm{E} \hspace{0.1ex} \right) \hspace{-0.1ex} \mathdotabove{\vartheta} \operatorname{sin} \vartheta \operatorname{cos} \vartheta \hspace{0.1ex}
%% + \hspace{-0.1ex} ( \bm{k} \mathdotabove{\bm{k}} \operatorname{cos} \vartheta + \mathdotabove{\bm{k}} \bm{k} ) \hspace{-0.2ex} \left({1 \!-\! \operatorname{cos} \vartheta}\right)
+ \bm{k} \mathdotabove{\bm{k}} \operatorname{cos} \vartheta
- \bm{k} \mathdotabove{\bm{k}} \operatorname{cos}^{2\hspace{-0.4ex}} \vartheta
+ \mathdotabove{\bm{k}} \bm{k} \left({1 \!-\! \operatorname{cos} \vartheta}\right) \hspace{-0.16ex},$}}
\end{multline*}
\begin{multline*}
\shoveleft{\scalebox{0.94}[0.96]{$\mathdotabove{\bm{P}}^{\mathsf{\hspace{0.12ex}A}} \hspace{-0.2ex}\dotp \bm{P}^{\mathsf{\hspace{0.12ex}S}} \hspace{-0.25ex} = $} \hfill} \\[-0.25em]
%
\shoveleft{\scalebox{0.8}[0.82]{$= ( \hspace{0.12ex} \bm{k} \hspace{-0.24ex}\times\hspace{-0.4ex} \bm{E} \hspace{0.1ex} ) \hspace{-0.2ex} \dotp \hspace{-0.2ex} \bm{E} \hspace{0.4ex} \mathdotabove{\vartheta} \operatorname{cos}^{2\hspace{-0.4ex}} \vartheta +
( \hspace{0.12ex} \mathdotabove{\bm{k}} \hspace{-0.24ex}\times\hspace{-0.4ex} \bm{E} \hspace{0.1ex} ) \hspace{-0.2ex} \dotp \hspace{-0.2ex} \bm{E} \hspace{0.1ex} \operatorname{sin} \vartheta \operatorname{cos} \vartheta \:+$} \hfill} \\[-0.2em]
\shoveright{\hfill \scalebox{0.8}[0.82]{$+\; \tikzbackcancel[black!25]{$ ( \hspace{0.12ex} \bm{k} \hspace{-0.24ex}\times\hspace{-0.4ex} \bm{E} \hspace{0.1ex} ) \hspace{-0.2ex} \dotp \hspace{-0.1ex} \bm{k} \bm{k} \hspace{0.5ex} \mathdotabove{\vartheta} \operatorname{cos} \vartheta \left({1 \!-\! \operatorname{cos} \vartheta}\right) $} \hspace{0.2ex} +
( \hspace{0.12ex} \mathdotabove{\bm{k}} \hspace{-0.24ex}\times\hspace{-0.4ex} \bm{E} \hspace{0.1ex} ) \hspace{-0.25ex} \dotp \hspace{-0.1ex} \bm{k} \bm{k} \hspace{0.2ex} \operatorname{sin} \vartheta \left({1 \!-\! \operatorname{cos} \vartheta}\right) =$}} \\
%
\hspace{3.85em} \scalebox{0.94}[0.96]{$= \hspace{.2ex} \bm{k} \hspace{-0.24ex}\times\hspace{-0.4ex} \bm{E} \hspace{0.4ex} \mathdotabove{\vartheta} \operatorname{cos}^{2\hspace{-0.4ex}} \vartheta +
\mathdotabove{\bm{k}} \hspace{-0.24ex}\times\hspace{-0.4ex} \bm{E} \hspace{0.1ex} \operatorname{sin} \vartheta \operatorname{cos} \vartheta +
%%\hspace{-0.12ex} ( \hspace{0.12ex} \mathdotabove{\bm{k}} \hspace{-0.24ex}\times\hspace{-0.4ex} \bm{E} \hspace{0.1ex} ) \hspace{-0.25ex} \dotp \hspace{-0.1ex} \bm{k} \bm{k} \hspace{0.2ex}
\mathdotabove{\bm{k}} \hspace{-0.24ex}\times\hspace{-0.32ex} \bm{k} \bm{k} \hspace{0.2ex}
\operatorname{sin} \vartheta \left({1 \!-\! \operatorname{cos} \vartheta}\right) \hspace{-0.16ex},$}
\end{multline*}
\begin{multline*}
\shoveleft{\scalebox{0.94}[0.96]{$\mathdotabove{\bm{P}}^{\mathsf{\hspace{0.12ex}S}} \hspace{-0.2ex}\dotp \bm{P}^{\mathsf{\hspace{0.12ex}A}} \hspace{-0.25ex} = $} \hfill} \\[-0.25em]
%
\scalebox{0.8}[0.82]{$= ( \hspace{0.1ex} \bm{k} \bm{k} \hspace{-0.1ex} - \hspace{-0.2ex} \bm{E} \hspace{0.1ex} ) \hspace{0.25ex} \mathdotabove{\vartheta} \operatorname{sin} \vartheta \dotp ( \bm{k} \hspace{-0.24ex}\times\hspace{-0.4ex} \bm{E} ) \operatorname{sin} \vartheta +
( \bm{k} \mathdotabove{\bm{k}} + \mathdotabove{\bm{k}} \bm{k} ) \hspace{-0.2ex} \left({1 \!-\! \operatorname{cos} \vartheta}\right) \hspace{-0.1ex} \dotp ( \bm{k} \hspace{-0.24ex}\times\hspace{-0.4ex} \bm{E} ) \operatorname{sin} \vartheta =$} \\
%
\scalebox{0.78}[0.82]{$= \hspace{.2ex} \tikzbackcancel[black!25]{$\bm{k} \bm{k} \hspace{-0.12ex} \dotp \hspace{-0.12ex} ( \bm{k} \hspace{-0.24ex}\times\hspace{-0.4ex} \bm{E} \hspace{0.1ex} ) \hspace{0.32ex} \mathdotabove{\vartheta} \operatorname{sin}^{\hspace{-0.1ex}2\hspace{-0.4ex}} \vartheta$} \hspace{0.12ex}
- \hspace{-0.1ex} \bm{E} \hspace{-0.16ex} \dotp \hspace{-0.12ex} ( \bm{k} \hspace{-0.24ex}\times\hspace{-0.4ex} \bm{E} \hspace{0.1ex} ) \hspace{0.25ex} \mathdotabove{\vartheta} \operatorname{sin}^{\hspace{-0.1ex}2\hspace{-0.4ex}} \vartheta
+ \hspace{-0.2ex} \left( \hspace{-0.1ex} \bm{k} \mathdotabove{\bm{k}} \dotp \hspace{-0.1ex} ( \bm{k} \hspace{-0.32ex}\times\hspace{-0.4ex} \bm{E} ) \hspace{-0.16ex} + \hspace{.1ex} \tikzbackcancel[black!25]{$\mathdotabove{\bm{k}} \bm{k} \dotp \hspace{-0.1ex} ( \bm{k} \hspace{-0.32ex}\times\hspace{-0.4ex} \bm{E}$} ) \hspace{-0.12ex} \right) \hspace{-0.2ex} \operatorname{sin} \vartheta \left({1 \!-\! \operatorname{cos} \vartheta} \right) =$} \\[-0.25em]
%
\hspace{13em} \scalebox{0.94}[0.96]{$= \hspace{-0.16ex} - \hspace{0.2ex} \bm{k} \hspace{-0.24ex}\times\hspace{-0.4ex} \bm{E} \hspace{0.4ex} \mathdotabove{\vartheta} \operatorname{sin}^{\hspace{-0.1ex}2\hspace{-0.4ex}} \vartheta
+ \bm{k} \mathdotabove{\bm{k}} \hspace{-0.24ex}\times\hspace{-0.32ex} \bm{k} \hspace{0.2ex} \operatorname{sin} \vartheta \left({1 \!-\! \operatorname{cos} \vartheta}\right) \hspace{-0.16ex},$}
\end{multline*}
\begin{multline*}
\shoveleft{\scalebox{0.94}[0.96]{$\mathdotabove{\bm{P}}^{\mathsf{\hspace{0.12ex}A}} \hspace{-0.2ex}\dotp \bm{P}^{\mathsf{\hspace{0.12ex}A}} \hspace{-0.25ex} =
%
( \bm{k} \hspace{-0.24ex}\times\hspace{-0.4ex} \bm{E} ) \hspace{0.32ex} \mathdotabove{\vartheta} \operatorname{cos} \vartheta \dotp ( \bm{k} \hspace{-0.24ex}\times\hspace{-0.4ex} \bm{E} ) \operatorname{sin} \vartheta + \hspace{-0.1ex}
( \mathdotabove{\bm{k}} \hspace{-0.24ex}\times\hspace{-0.4ex} \bm{E} ) \hspace{-0.16ex} \dotp \hspace{-0.16ex} ( \bm{k} \hspace{-0.24ex}\times\hspace{-0.4ex} \bm{E} ) \hspace{0.1ex} \operatorname{sin}^{\hspace{-0.1ex}2\hspace{-0.4ex}} \vartheta = $} \hfill} \\
%
\shoveright{\hfill \hspace{16em} \scalebox{0.94}[0.96]{$= ( \bm{k} \bm{k} \hspace{-0.1ex} - \hspace{-0.2ex} \bm{E} ) \hspace{0.32ex} \mathdotabove{\vartheta} \operatorname{sin} \vartheta \operatorname{cos} \vartheta
+ \bm{k} \mathdotabove{\bm{k}} \operatorname{sin}^{\hspace{-0.1ex}2\hspace{-0.4ex}} \vartheta \hspace{0.2ex};$}}
\end{multline*}
\end{fleqn}

\begin{multline*}
\scalebox{0.94}[0.96]{$\mathdotabove{\bm{P}} \dotp \bm{P}^{\T}$} \hspace{0.25ex}
\scalebox{0.92}[0.96]{$= \hspace{0.1ex}
\mathdotabove{\bm{P}}^{\mathsf{\hspace{0.12ex}S}} \hspace{-0.2ex}\dotp \bm{P}^{\mathsf{\hspace{0.12ex}S}}
+ \hspace{0.2ex} \mathdotabove{\bm{P}}^{\mathsf{\hspace{0.12ex}A}} \hspace{-0.25ex}\dotp \bm{P}^{\mathsf{\hspace{0.12ex}S}}
- \hspace{0.2ex} \mathdotabove{\bm{P}}^{\mathsf{\hspace{0.12ex}S}} \hspace{-0.25ex}\dotp \bm{P}^{\mathsf{\hspace{0.12ex}A}}
- \hspace{0.2ex} \mathdotabove{\bm{P}}^{\mathsf{\hspace{0.12ex}A}} \hspace{-0.25ex}\dotp \bm{P}^{\mathsf{\hspace{0.12ex}A}} \hspace{-0.25ex} =$} \\[-0.25em]
%
\scalebox{0.8}[0.82]{$= {\color{black!50}{\left( \hspace{0.1ex} \bm{k} \bm{k} \hspace{-0.1ex} - \hspace{-0.2ex} \bm{E} \hspace{0.1ex} \right) \hspace{-0.1ex} \mathdotabove{\vartheta} \operatorname{sin} \vartheta \operatorname{cos} \vartheta}}
+ \bm{k} \mathdotabove{\bm{k}} \operatorname{cos} \vartheta
- {\color{magenta!80!black}{\bm{k} \mathdotabove{\bm{k}}}} {\color{black!50}{\hspace{0.4ex}\operatorname{cos}^{2\hspace{-0.4ex}} \vartheta}}
+ \mathdotabove{\bm{k}} \bm{k} \left({1 \!-\! \operatorname{cos} \vartheta}\right) +$} \\[-0.2em]
%
\scalebox{0.8}[0.82]{$+\; {\color{blue!80!black}{\bm{k} \hspace{-0.24ex}\times\hspace{-0.4ex} \bm{E} \hspace{0.4ex} \mathdotabove{\vartheta}}} {\color{black!50}{\hspace{0.4ex}\operatorname{cos}^{2\hspace{-0.4ex}} \vartheta}}
+ \mathdotabove{\bm{k}} \hspace{-0.24ex}\times\hspace{-0.4ex} \bm{E} \hspace{0.1ex} \operatorname{sin} \vartheta \operatorname{cos} \vartheta
+ \mathdotabove{\bm{k}} \hspace{-0.24ex}\times\hspace{-0.32ex} \bm{k} \bm{k} \hspace{0.2ex} {\color{green!50!black}{\hspace{0.5ex}\operatorname{sin} \vartheta \left({1 \!-\! \operatorname{cos} \vartheta}\right)\hspace{0.4ex}}} +$} \\[-0.25em]
%
\scalebox{0.8}[0.82]{$+\; {\color{blue!80!black}{\bm{k} \hspace{-0.24ex}\times\hspace{-0.4ex} \bm{E} \hspace{0.4ex} \mathdotabove{\vartheta}}} {\color{black!50}{\hspace{0.4ex}\operatorname{sin}^{\hspace{-0.1ex}2\hspace{-0.4ex}} \vartheta}}
- \bm{k} \mathdotabove{\bm{k}} \hspace{-0.24ex}\times\hspace{-0.32ex} \bm{k} {\color{green!50!black}{\hspace{0.5ex}\operatorname{sin} \vartheta \left({1 \!-\! \operatorname{cos} \vartheta}\right)\hspace{0.4ex}}}
{\color{black!50}{-\hspace{0.5ex} ( \bm{k} \bm{k} \hspace{-0.1ex} - \hspace{-0.2ex} \bm{E} ) \hspace{0.32ex} \mathdotabove{\vartheta} \operatorname{sin} \vartheta \operatorname{cos} \vartheta}}
- {\color{magenta!80!black}{\bm{k} \mathdotabove{\bm{k}}}} {\color{black!50}{\hspace{0.4ex}\operatorname{sin}^{\hspace{-0.1ex}2\hspace{-0.4ex}} \vartheta}} =$} \\[-0.08em]
%
\scalebox{0.79}[0.82]{$= \bm{k} \hspace{-0.24ex}\times\hspace{-0.4ex} \bm{E} \hspace{0.4ex} \mathdotabove{\vartheta} \hspace{-0.1ex}
+ \hspace{-0.1ex} \hspace{-0.16ex} ( \hspace{0.2ex} \mathdotabove{\bm{k}} \bm{k} \hspace{-0.2ex} - \hspace{-0.2ex} \bm{k} \mathdotabove{\bm{k}} \hspace{0.2ex} ) \hspace{0.1ex} ({1 \!-\! \operatorname{cos} \vartheta}) \hspace{-0.2ex}
+ \mathdotabove{\bm{k}} \hspace{-0.24ex}\times\hspace{-0.4ex} \bm{E} \hspace{.1ex} \operatorname{sin} \vartheta \operatorname{cos} \vartheta \hspace{-0.1ex}
+ \hspace{-0.1ex} ( \mathdotabove{\bm{k}} \hspace{-0.28ex}\times\hspace{-0.32ex} \bm{k} \bm{k} \hspace{-0.12ex} - \hspace{-0.12ex} \bm{k} \mathdotabove{\bm{k}} \hspace{-0.28ex}\times\hspace{-0.32ex} \bm{k} ) \operatorname{sin} \vartheta \left({1 \!-\! \operatorname{cos} \vartheta}\right) = $} \\[-0.08em]
%
\scalebox{0.8}[0.82]{$= \bm{k} \hspace{-0.24ex}\times\hspace{-0.4ex} \bm{E} \hspace{0.4ex} \mathdotabove{\vartheta} \hspace{-0.1ex}
+ \hspace{-0.1ex} \bm{k} \hspace{-0.2ex}\times\hspace{-0.2ex}  \mathdotabove{\bm{k}} \hspace{-0.2ex}\times\hspace{-0.4ex} \bm{E} \hspace{.25ex} ({1 \!-\! \operatorname{cos} \vartheta}) \hspace{-0.2ex}
+ \mathdotabove{\bm{k}} \hspace{-0.24ex}\times\hspace{-0.4ex} \bm{E} \hspace{.1ex} \operatorname{sin} \vartheta \operatorname{cos} \vartheta \hspace{-0.1ex}
+ \bm{k} \hspace{-0.25ex} \times \hspace{-0.32ex} ( \mathdotabove{\bm{k}} \hspace{-0.2ex}\times\hspace{-0.2ex} \bm{k} ) \hspace{-0.4ex}\times\hspace{-0.4ex} \bm{E} \hspace{.1ex} \operatorname{sin} \vartheta \left({1 \!-\! \operatorname{cos} \vartheta}\right) = $} \\[-0.08em]
%
\scalebox{0.78}[0.82]{$= \bm{k} \hspace{-0.24ex}\times\hspace{-0.4ex} \bm{E} \hspace{0.4ex} \mathdotabove{\vartheta} \hspace{-0.1ex}
+ \mathdotabove{\bm{k}} \hspace{-0.24ex}\times\hspace{-0.4ex} \bm{E} \hspace{.1ex} \operatorname{sin} \vartheta \operatorname{cos} \vartheta \hspace{-0.1ex}
+ \hspace{-0.16ex} ( \mathdotabove{\bm{k}} \bm{k} \hspace{-0.1ex} \dotp \hspace{-0.16ex} \bm{k} \hspace{-0.1ex} - \tikzbackcancel[black!25]{$\bm{k} \mathdotabove{\bm{k}} \hspace{-0.16ex} \dotp \hspace{-0.1ex} \bm{k} \hspace{.1ex}$} ) \hspace{-0.4ex}\times\hspace{-0.4ex} \bm{E} \hspace{.1ex} \operatorname{sin} \vartheta \left({1 \!-\! \operatorname{cos} \vartheta}\right) \hspace{-0.1ex}
+ \hspace{-0.1ex} \bm{k} \hspace{-0.32ex}\times\hspace{-0.25ex}  \mathdotabove{\bm{k}} \hspace{-0.25ex}\times\hspace{-0.42ex} \bm{E} \hspace{.32ex} ({1 \!-\! \operatorname{cos} \vartheta}) = $} \\
%
\shoveright{\hfill \hspace{11.2em}\scalebox{.96}[.96]{$= \bm{k} \hspace{-0.24ex}\times\hspace{-0.4ex} \bm{E} \hspace{0.4ex} \mathdotabove{\vartheta}
+ \mathdotabove{\bm{k}} \hspace{-0.24ex}\times\hspace{-0.4ex} \bm{E} \hspace{.1ex} \operatorname{sin} \vartheta
+ \hspace{-0.1ex} \bm{k} \hspace{-0.2ex}\times\hspace{-0.2ex}  \mathdotabove{\bm{k}} \hspace{-0.2ex}\times\hspace{-0.4ex} \bm{E} \hspace{.32ex} ({1 \hspace{-0.2ex} - \hspace{-0.2ex} \operatorname{cos} \vartheta}) \hspace{.1ex}.
$}}
\end{multline*}

Этот результат, подставленный в~определение~\eqref{angularvelocityvector} псевдо\-вектора~$\bm{\omega}$, даёт

\nopagebreak\vspace{-0.5em}\begin{equation}
\bm{\omega} = \bm{k} \hspace{0.2ex} \mathdotabove{\vartheta}
+ \mathdotabove{\bm{k}} \operatorname{sin} \vartheta
+ \bm{k} \hspace{-0.1ex}\times\hspace{-0.1ex} \mathdotabove{\bm{k}} \left( 1 - \operatorname{cos} \vartheta \right) \hspace{-0.4ex}.
\end{equation}

\vspace{-0.32em} \noindent Вектор~$\bm{\omega}$ получился разложенным по~трём взаимно ортогональным направлениям~--- $\bm{k}$, $\mathdotabove{\bm{k}}$ и~${\bm{k} \hspace{-0.1ex}\times\hspace{-0.1ex} \mathdotabove{\bm{k}}}$. При~неподвижной оси поворота ${\mathdotabove{\bm{k}} = \bm{0} \,\Rightarrow\, \bm{\omega} = \bm{k} \hspace{0.1ex} \mathdotabove{\vartheta}}$.

Ещё одно представление~$\bm{\omega}$ связано с~компонентами тензора поворота~\eqref{componentsofrotationtensor}. Поскольку ${\bm{P} = \cosinematrix{\!j\mathcircabove{i}} \hspace{0.4ex} \mathcircabove{\bm{e}}_i \mathcircabove{\bm{e}}_j}$, ${\bm{P}^{\T} \hspace{-0.32ex} = \cosinematrix{\hspace{-0.2ex}i\mathcircabove{j}} \hspace{0.4ex} \mathcircabove{\bm{e}}_i \mathcircabove{\bm{e}}_j}$, а~векторы начального базиса~${\mathcircabove{\bm{e}}_i}$ неподвижны (со~временем не~меняются), то
\nopagebreak\vspace{0.25em}\[ \mathdotabove{\bm{P}} = \cosinematrixdotted{\!j\mathcircabove{i}} \hspace{0.4ex} \mathcircabove{\bm{e}}_i \mathcircabove{\bm{e}}_j \hspace{0.1ex}, \:\,
\mathdotabove{\bm{P}} \dotp \bm{P}^{\T} \hspace{-0.32ex} = \hspace{0.1ex} \cosinematrixdotted{\hspace{-0.4ex}n\mathcircabove{i}} \hspace{0.4ex}  \cosinematrix{\hspace{-0.4ex}n\mathcircabove{j}} \hspace{0.4ex} \mathcircabove{\bm{e}}_i \mathcircabove{\bm{e}}_j \hspace{0.1ex}, \]
\nopagebreak\vspace{-0.64em}\begin{equation}
\bm{\omega} = - \hspace{0.1ex} \smalldisplaystyleonehalf \hspace{0.4ex} \cosinematrixdotted{\hspace{-0.4ex}n\mathcircabove{i}} \hspace{0.4ex} \cosinematrix{\hspace{-0.4ex}n\mathcircabove{j}} \hspace{0.5ex} \mathcircabove{\bm{e}}_i \hspace{-0.3ex}\times\hspace{-0.3ex} \mathcircabove{\bm{e}}_j \hspace{-0.1ex} =
\smalldisplaystyleonehalf \hspace{0.2ex} \levicivita_{jik} \hspace{0.32ex} \cosinematrix{\hspace{-0.4ex}n\mathcircabove{j}} \hspace{0.4ex} \cosinematrixdotted{\hspace{-0.4ex}n\mathcircabove{i}} \hspace{0.4ex} \mathcircabove{\bm{e}}_k \hspace{0.2ex}.
\end{equation}

\vspace{-0.25em}
Отметим и~формулы
\nopagebreak\vspace{.16em}\begin{equation}\label{angularvelocityandbasisvectors}
\begin{array}{c}
\eqref{angularvelocityvector} \hspace{.32ex} \Rightarrow\:
\mathdotabove{\bm{e}}_i \mathcircabove{\bm{e}}_i \hspace{-0.1ex} = \bm{\omega} \times \hspace{-0.1ex} \bm{e}_i \mathcircabove{\bm{e}}_i \:\Rightarrow\:
\mathdotabove{\bm{e}}_i = \bm{\omega} \times \bm{e}_i \hspace{0.12ex}, \\[.32em]
%
\eqref{angularvelocityvector} \hspace{.32ex} \Rightarrow\:
\bm{\omega} = \hspace{-0.1ex} - \hspace{0.16ex} \smalldisplaystyleonehalf \hspace{-.2ex} \left( \mathdotabove{\bm{e}}_i \mathcircabove{\bm{e}}_i \hspace{-0.1ex} \dotp \mathcircabove{\bm{e}}_j \bm{e}_j \right)_{\hspace{-0.25ex}\Xcompanion} \hspace{-0.32ex}
= \hspace{-0.1ex} - \hspace{0.16ex} \smalldisplaystyleonehalf \hspace{-.2ex} \left( \mathdotabove{\bm{e}}_i \bm{e}_i \hspace{.1ex} \right)_{\hspace{-0.1ex}\Xcompanion} \hspace{-0.25ex}
= \smalldisplaystyleonehalf \hspace{0.4ex} \bm{e}_i \hspace{-0.16ex} \times \hspace{-0.1ex} \mathdotabove{\bm{e}}_i \hspace{0.12ex}.
\end{array}
\end{equation}



Comment\footnote{Поворот тела вокруг какой\hbox{-}то оси представляет, казалось~бы, вектор, ведь он имеет численное значение, равное углу поворота, и~направление, совпадающее с~направлением оси вращения, котоpое определяется по~\inquotesx{правилу буравчика}[.] Однако два таких поворота не~складываются как векторы, если только углы поворота не~являются бесконечно малыми.}

На~самом~деле последовательные повороты не~складываются, а~умножаются.

Можно~ли складывать угловые скорости?~--- Да, потому что угол поворота в~$\mathdotabove{\vartheta}$ бесконечно малый.~--- Но только при вращении вокруг неподвижной оси?

...



\end{otherlanguage}

\en{\section{Variations}}

\ru{\section{Варьирование}}

\begin{otherlanguage}{russian}

Далее повсеместно будет использоваться сходная с~дифференцированием операция варьирования. Не~отсылая читателя к~курсам вариационного исчисления, ограничимся представлениями о~вариации~${\variation{x}}$ величины~$x$ как о~задаваемом нами бесконечно малом приращении, совместимом с~ограничениями~--- связями~(constraints). Если ограничений для~$x$ нет, то ${\variation{x}}$ произвольна~(случайна). Но когда ${x \!=\! x(y\hspace{-0.1ex})}$~--- функция независимого аргумента~$y$, следует считать ${\variation{x} = x'\hspace{-0.25ex}(y\hspace{-0.1ex}) \hspace{0.1ex} \variation{y}}$.



{\small
\setlength{\abovedisplayskip}{2pt}\setlength{\belowdisplayskip}{2pt}

Here we consider the exact differential of any set of position vectors~$\bm{r}_i$, that are functions of other variables ${\displaystyle \lbrace q_{1},q_{2},...,q_{m}\rbrace }$ and time t.

The actual displacement is the differential
\[\displaystyle d\bm{r}_{i} = \frac{\partial \bm{r}_{i}}{\partial t} \hspace{.16ex} dt \hspace{.2ex} + \sum_{j=1}^{m} {\frac{\partial \bm{r}_{i}}{\partial q^{\hspace{.1ex}j}}} \hspace{.2ex} dq^{\hspace{.1ex}j}\]

Now, imagine if we have an arbitrary path through the configuration space/manifold. This means it has to satisfy the constraints of the system but not the actual applied forces
\[\displaystyle \delta \bm{r}_{i}=\sum _{j=1}^{m} {\frac {\partial \bm{r}_{i}}{\partial q^{\hspace{.1ex}j}}} \hspace{.2ex} \delta q^{\hspace{.1ex}j}\]

\par}



В~записях с~вариациями действуют те~же правила, что и с~дифференциалами. Если, например, ${\variation{x}}$ и~${\variation{y}}$~--- вариации~$x$ и~$y$, а~$u$ и~$v$~--- кон\'{е}чные величины, то следует пис\'{а}ть ${u \variation{x} + v \variation{y} = \variation{w}}$, а~не~$w$~--- даже когда ${\variation{w}}$ не~является вариацией величины~$w$; в~этом случае ${\variation{w}}$ это единое обозначение. Разумеется, при~${u \!=\! u(x,y)}$, ${v \!=\! v(x,y)}$ и~${\partial_x v = \partial_y u}$ (${\hspace{0.16ex}\frac{\partial}{\partial x} v = \frac{\partial}{\partial y} u\hspace{0.16ex}}$) сумма~${\variation{w}}$ будет вариацией некой~$w$.

Варьируя тождество~\eqref{orthogonalityofrotationtensor}, получим ${\variation{\bm{P}} \hspace{-0.08ex} \dotp \bm{P}^{\T} \hspace{-0.2ex} = - \bm{P} \dotp \variation{\bm{P}}^{\T}}$\!. Этот тензор антисимметричен, и~потому выражается через свой сопутствующий вектор~${\varvector{o}}$ как~${\variation{\bm{P}} \hspace{-0.08ex} \dotp \bm{P}^{\T} \hspace{-0.25ex} = \hspace{0.1ex} \varvector{o} \times \bm{E}}$. Приходим к~соотношениям
\vspace{-0.4em}\begin{equation}
\variation{\bm{P}} \hspace{-0.1ex} = \varvector{o} \times \bm{P} , \:\:
\varvector{o} = - \, \displaystyle \onehalf \left( { \variation{\bm{P}} \hspace{-0.08ex} \dotp^{\mathstrut} \bm{P}^{\T} } \right)_{\hspace{-0.25em}\Xcompanion} \hspace{-0.1ex} ,
\end{equation}

\vspace{-0.5em} \noindent аналогичным~\eqref{angularvelocityvector}. Вектор бесконечно малого поворота~${\varvector{o}}$ это не~\inquotesx{вариация $\bm{\mathrm{o}}$}[,] но единый символ (в~отличие от~${\variation{\bm{P}}}$).

Малый поворот определяется вектором~${\varvector{o}}$, но конечный поворот тоже допускает(?) векторное представление

...



\end{otherlanguage}

\en{\section{Polar decomposition}}

\ru{\section{Полярное разложение}}

\label{para:polardecomposition}

\begin{otherlanguage}{russian}

Любой тензор второй сложности~${\bm{F}}$ с~${\operatorname{det} F_{ij} \hspace{-0.16ex} \neq 0}$~(не~сингулярный) может быть представлен как

...

\begin{tcolorbox}
\small\setlength{\abovedisplayskip}{2pt}\setlength{\belowdisplayskip}{2pt}

\emph{Example.} Polar decompose tensor~${\bm{C} = C_{ij} \bm{e}_i \bm{e}_j}$, where $\bm{e}_k$ are mutually orthogonal unit vectors of basis, and $C_{ij}$ are tensor’s components

\begin{equation*}
C_{ij} =
\scalebox{0.92}[0.92]{$\left[\hspace{-0.2ex}\begin{array}{c@{\hspace{.6em}}c@{\hspace{.6em}}c}
-5 & 20 & 11 \\
10 & -15 & 23 \\
-3 & -5 & 10
\end{array}\hspace{-0.2ex}\right]$}
\end{equation*}

\begin{equation*}
\rotationtensor = O_{ij} \bm{e}_i \bm{e}_j \hspace{-0.2ex}
= \bm{O}_1 \hspace{-0.2ex} \dotp \bm{O}_2
\end{equation*}

\begin{equation*}
O_{ij} =
\scalebox{0.92}[0.92]{$\left[\hspace{-0.2ex}\begin{array}{c@{\hspace{.6em}}c@{\hspace{.6em}}c}
0 & \nicefrac{3}{5} & \nicefrac{4}{5} \\
0 & \nicefrac{4}{5} & - \hspace{.2ex} \nicefrac{3}{5} \\
-1 & 0 & 0
\end{array}\hspace{-0.4ex}\right]$}
\hspace{-0.25ex} = \hspace{-0.25ex}
\scalebox{0.92}[0.92]{$\left[\hspace{-0.2ex}\begin{array}{c@{\hspace{.6em}}c@{\hspace{.6em}}c}
0 & \mathcolor{red!33!white}{0} & 1 \\
\mathcolor{red!33!white}{0} & \mathcolor{red!77!black}{1} & \mathcolor{red!33!white}{0} \\
-1 & \mathcolor{red!33!white}{0} & 0
\end{array}\hspace{-0.2ex}\right]$}
\scalebox{0.92}[0.92]{$\left[\hspace{-0.2ex}\begin{array}{c@{\hspace{.6em}}c@{\hspace{.6em}}c}
\mathcolor{red!77!black}{1} & \mathcolor{red!33!white}{0} & \mathcolor{red!33!white}{0} \\
\mathcolor{red!33!white}{0} & \nicefrac{4}{5} & - \hspace{.2ex} \nicefrac{3}{5} \\
\mathcolor{red!33!white}{0} & \nicefrac{3}{5} & \nicefrac{4}{5}
\end{array}\hspace{-0.4ex}\right]$}
\end{equation*}

\begin{equation*}\begin{array}{c}
\bm{C} = \rotationtensor \hspace{-0.1ex} \dotp \bm{S_{\smash{\mathsf{R}}}}
\hspace{.1ex} , \:\:
\rotationtensor^{\T}\hspace{-0.5ex} \dotp \bm{C} = \bm{S_{\smash{\mathsf{R}}}}
\end{array}\end{equation*}

\begin{equation*}\begin{array}{c}
\bm{C} = \bm{S_{\smash{\mathsf{L}}}} \hspace{-0.25ex} \dotp \rotationtensor
\hspace{.1ex} , \:\:
\bm{C} \dotp \rotationtensor^{\T}\hspace{-0.5ex} = \bm{S_{\smash{\mathsf{L}}}}
\end{array}\end{equation*}

\begin{equation*}
S_{\smash{\mathsf{R}}\hspace{.15ex}ij} \hspace{-0.1ex} =
\scalebox{0.92}[0.92]{$\left[\hspace{-0.2ex}\begin{array}{c@{\hspace{.6em}}c@{\hspace{.6em}}c}
3 & 5 & -10 \\
5 & 0 & 25 \\
-10 & 25 & -5
\end{array}\hspace{-0.2ex}\right]$}
\end{equation*}

\begin{equation*}
S_{\smash{\mathsf{L}}\hspace{.15ex}ij} \hspace{-0.1ex} =
\scalebox{0.92}[0.92]{$\left[\hspace{-0.2ex}\begin{array}{c@{\hspace{.6em}}c@{\hspace{.6em}}c}
\nicefrac{104}{5} & \nicefrac{47}{5} & 5 \\
\nicefrac{47}{5} & - \hspace{.2ex} \nicefrac{129}{5} & -10 \\
5 & -10 & 3
\end{array}\hspace{-0.2ex}\right]$}
\end{equation*}

\par\end{tcolorbox}

...


\end{otherlanguage}


\newpage

\en{\section{Tensors in oblique basis}}

\ru{\section{Тензоры в косоугольном базисе}}

\begin{otherlanguage}{russian}

До~сих~пор использовался базис из~ортогональной тройки единичных векторов~${\bm{e}_i}$.
Теперь рассмотрим базис из~трёх любых линейно независимых~(некомпланарных) векторов~${\bm{a}_i}$.

%%\begin{figure}[!htbp]
%%\begin{center}
\begin{wrapfigure}[20]{R}{0.48\textwidth}
\makebox[0.5\textwidth][c]{\begin{minipage}[t]{0.5\textwidth}

\tdplotsetmaincoords{33}{109} % orientation of camera
% vectors of basis
\pgfmathsetmacro{\firstlength}{1}
	\pgfmathsetmacro{\firstanglefromz}{90} % first and second are xy plane
	\pgfmathsetmacro{\firstanglefromx}{0} % first is just x
\pgfmathsetmacro{\secondlength}{1}
	\pgfmathsetmacro{\secondanglefromz}{90} % first and second are xy plane
	\pgfmathsetmacro{\secondanglefromx}{77} % but second is not orthogonal to first
\pgfmathsetmacro{\thirdlength}{1}
	\pgfmathsetmacro{\thirdanglefromz}{-15}
	\pgfmathsetmacro{\thirdanglefromx}{50}
% some vector
\pgfmathsetmacro{\lengthofvector}{2.66}
	\pgfmathsetmacro{\vectoranglefromz}{33}
	\pgfmathsetmacro{\vectoranglefromx}{59}

\vspace{-0.2em}
\hspace{1.32em}\scalebox{0.96}[0.96]{%
\begin{tikzpicture}[scale=2.5, tdplot_main_coords] % tdplot_main_coords style to use 3dplot

	\coordinate (O) at (0,0,0);

	% draw axes and vectors of basis
	\tdplotsetcoord{A1}{\firstlength}{\firstanglefromz}{\firstanglefromx}
	\tdplotsetcoord{A2}{\secondlength}{\secondanglefromz}{\secondanglefromx}
	\tdplotsetcoord{A3}{\thirdlength}{\thirdanglefromz}{\thirdanglefromx}

	\draw [line width=0.4pt, blue] (O) -- ($ 1.08*(A1) $);
	\draw [line width=1.25pt, blue, -{Latex[round, length=3.6mm, width=2.4mm]}]
		(O) -- (A1)
		node[pos=0.97, above left, inner sep=0pt, outer sep=3pt] {${\bm{a}}_1$};

	\draw [line width=0.4pt, blue] (O) -- ($ 1.88*(A2) $);
	\draw [line width=1.25pt, blue, -{Latex[round, length=3.6mm, width=2.4mm]}]
		(O) -- (A2)
		node[pos=0.93, above, inner sep=0pt, outer sep=6pt] {${\bm{a}}_2$};

	\draw [line width=0.4pt, blue] (O) -- ($ 1.1*(A3) $);
	\draw [line width=1.25pt, blue, -{Latex[round, length=3.6mm, width=2.4mm]}]
		(O) -- (A3)
		node[pos=1.02, below right, inner sep=0pt, outer sep=7pt] {${\bm{a}}_3$};

	% define vector by sperical coordinates {length}{angle from z}{angle from x}
	% (plusi it defines its projections)
	\tdplotsetcoord{V}{\lengthofvector}{\vectoranglefromz}{\vectoranglefromx}

	% get components of vector
	\coordinate (ParallelToThird) at ($ (V) - (A3) $);
	\coordinate (VcomponentXY) at (intersection of V--ParallelToThird and O--Vxy);

	\coordinate (ParallelToSecond) at ($ (VcomponentXY) - (A2xy) $);
	\coordinate (ParallelToFirst) at ($ (VcomponentXY) - (A1xy) $);

	\coordinate (Vcomponent1) at (intersection of VcomponentXY--ParallelToSecond and O--A1);
	\coordinate (Vcomponent2) at (intersection of VcomponentXY--ParallelToFirst and O--A2);

	\draw [line width=0.4pt, dotted, color=black] (O) -- (VcomponentXY); % projection on first & second vectors’ plane
	\draw [line width=0.4pt, dotted, color=black] (V) -- (VcomponentXY);
	\draw [line width=0.4pt, dotted, color=black] (VcomponentXY) -- (Vcomponent1);
	\draw [line width=0.4pt, dotted, color=black] (VcomponentXY) -- (Vcomponent2);

	% draw parallelepiped
	\coordinate (onPlane23) at ($ (Vcomponent2) + (V) - (VcomponentXY) $);
	\draw [line width=0.4pt, dotted, color=black] (Vcomponent2) -- (onPlane23);
	\draw [line width=0.4pt, dotted, color=black] (V) -- (onPlane23);

	\coordinate (onPlane13) at ($ (Vcomponent1) + (V) - (VcomponentXY) $);
	\draw [line width=0.4pt, dotted, color=black] (Vcomponent1) -- (onPlane13);
	\draw [line width=0.4pt, dotted, color=black] (V) -- (onPlane13);

	\coordinate (onAxis3) at ($ (V) - (VcomponentXY) $);
	\draw [line width=0.4pt, dotted, color=black] (O) -- (onAxis3);
	\draw [line width=0.4pt, dotted, color=black] (onPlane13) -- (onAxis3);
	\draw [line width=0.4pt, dotted, color=black] (onPlane23) -- (onAxis3);

	\draw [line width=0.4pt, dotted, color=black] (O) -- (onPlane13);
	\draw [line width=0.4pt, dotted, color=black] (O) -- (onPlane23);

	% draw components of vector
	\draw [color=black, line width=1.6pt, line cap=round, dash pattern=on 0pt off 1.6\pgflinewidth,
		-{Stealth[round, length=4mm, width=2.4mm]}]
		(O) -- (Vcomponent1)
		node[pos=0.52, above left, fill=white, shape=circle, inner sep=0pt, outer sep=4pt] {${v^1 \hspace{-0.1ex} \bm{a}_1}$};

	\draw [color=black, line width=1.6pt, line cap=round, dash pattern=on 0pt off 1.6\pgflinewidth,
		-{Stealth[round, length=4mm, width=2.4mm]}]
		(Vcomponent1) -- (VcomponentXY)
		node[pos=0.48, below, shape=circle, fill=white, inner sep=-2pt, outer sep=2pt] {${v^2 \hspace{-0.1ex} \bm{a}_2}$};

	\draw [color=black, line width=1.6pt, line cap=round, dash pattern=on 0pt off 1.6\pgflinewidth,
		-{Stealth[round, length=4mm, width=2.4mm]}]
		(VcomponentXY) -- (V)
		node[pos=0.52, above right, shape=circle, fill=white, inner sep=0pt, outer sep=7pt] {${v^3 \hspace{-0.1ex} \bm{a}_3}$};

	% draw vector
	\draw [line width=1.6pt, black, -{Stealth[round, length=5mm, width=2.8mm]}]
		(O) -- (V)
		node[pos=0.68, above, fill=white, inner sep=1pt, outer sep=5pt] {\scalebox{1.2}[1.2]{${\bm{v}}$}};

\end{tikzpicture}}
\vspace{-0.32em}\caption{}\label{fig:ObliqueCoordinates}
\end{minipage}}
\end{wrapfigure}
%%\end{center}
%%\end{figure}\vspace{-1.5em}

Декомпозиция~(разложение) вектора~$\bm{v}$ по базису~${\bm{a}_i}$~(\figref{fig:ObliqueCoordinates})~--- линейная комбинация

\nopagebreak\vspace{-0.2em}\begin{equation}\label{decompositionbyobliquebasis}
\bm{v} = v^{i} \hspace{-0.1ex} \bm{a}_i \hspace{0.1ex}.
\end{equation}

{\small\setlength{\abovedisplayskip}{2pt}\setlength{\belowdisplayskip}{2pt}
Соглашение о~суммировании обретает новые положения: повторяющиеся~(\inquotes{немые}) индексы суммирования расположены на~\hbox{разных} уровнях, а~свободные индексы в~обеих частях равенств~--- на~одной высоте (${a_i = \hspace{0.1ex} b_{ij} c^{\hspace{0.2ex}j}}$~--- корректно, ${a_i \hspace{-0.1ex} = \hspace{0.1ex} b_{kk}^{\hspace{0.1ex}i}}$~--- дважды ошибочно).
\par}

В~таком базисе уж\'{е} ${\bm{v} \dotp \bm{a}_i \hspace{-0.1ex} = \hspace{0.1ex} v^{k} \bm{a}_k \hspace{-0.1ex} \dotp \bm{a}_i \neq\vspace{0.2ex} v^{i}}$\hspace{-0.25ex}, ведь~тут ${\bm{a}_i \hspace{-0.1ex} \dotp \bm{a}_k \neq\vspace{0.2ex} \delta_{ik}}$.

Дополним~же \hbox{базис}~${\bm{a}_i}$ ещё другой тройкой векторов~\hbox{${\bm{a}^{\hspace{-0.05ex}i}}$\hspace{-0.25ex},} \hbox{называемых} кобазисом или~взаимным базисом, чтобы

\nopagebreak\vspace{-0.2em}\begin{equation}\label{fundamentalpropertyofcobasis}
\begin{array}{c}
\bm{a}_i \dotp \bm{a}^{\hspace{0.1ex}j} \hspace{-0.1ex} = \hspace{0.1ex} \delta_i^{\hspace{0.1ex}j} , \:\:
\bm{a}^{\hspace{-0.05ex}i} \hspace{-0.1ex} \dotp \bm{a}_j \hspace{-0.1ex} = \hspace{0.1ex} \delta_{\hspace{-0.1ex}j}^{\hspace{0.1ex}i} \hspace{0.16ex}, \\[0.2em]
\bm{E} = \bm{a}^{\hspace{-0.1ex}i} \hspace{-0.1ex} \bm{a}_i \hspace{-0.1ex} = \bm{a}_i \hspace{0.1ex} \bm{a}^{\hspace{-0.1ex}i} \hspace{-0.2ex}.
\end{array}
\end{equation}

\vspace{-0.1em}\noindent Это~--- основное свойство кобазиса. Орто\-нормирован\-ный~(орто\-нормаль\-ный) базис может быть определён как совпад\'{а}ющий со~своим кобазисом: ${\bm{e}^{\hspace{0.05ex}i} \hspace{-0.2ex} = \bm{e}_{i}}$.

%%\inquotes{В~декартовых координатах}, когда базис~--- отронормальный правый:
%%\begin{itemize}
%%\item компоненты единичного~(\inquotes{метрического}) тензора~--- дельта Кронекера,
%%\item компоненты (псевдо)тензора Л\'{е}ви\hbox{-\!}Чив\'{и}ты~--- символ Веблена.
%%\end{itemize}

\begin{comment} %%
\vspace{-0.5em}\[
\bm{a}_i \dotp \bm{a}^{\hspace{0.1ex}j} \hspace{-0.1ex} = \hspace{-0.2ex}
\scalebox{0.8}[0.8]{$\left[ \begin{array}{ccc}
\bm{a}_1 \hspace{-0.1ex} \dotp \bm{a}^{\hspace{-0.1ex}1} & \bm{a}_1 \hspace{-0.1ex} \dotp \bm{a}^2 & \bm{a}_1 \hspace{-0.1ex} \dotp \bm{a}^3 \\
\bm{a}_2 \hspace{-0.1ex} \dotp \bm{a}^{\hspace{-0.1ex}1} & \bm{a}_2 \hspace{-0.1ex} \dotp \bm{a}^2 & \bm{a}_2 \hspace{-0.1ex} \dotp \bm{a}^3 \\
\bm{a}_3 \hspace{-0.1ex} \dotp \bm{a}^{\hspace{-0.1ex}1} & \bm{a}_3 \hspace{-0.1ex} \dotp \bm{a}^2 & \bm{a}_3 \hspace{-0.1ex} \dotp \bm{a}^3
\end{array} \right]$} \!=\!
\scalebox{0.8}[0.8]{$\left[ \begin{array}{ccc}
1 & 0 & 0 \\
0 & 1 & 0 \\
0 & 0 & 1
\end{array} \right]$} \!=
\hspace{0.1ex} \delta_i^{\hspace{0.1ex}j}
\]
\end{comment} %%

Для, к~примеру, первого вектора кобазиса~${\bm{a}^{\hspace{-0.1ex}1}\hspace{-0.1ex}}$

\nopagebreak\vspace{-0.1em}\begin{equation*}
\scalebox{0.9}[0.9]{$\left\{\hspace{-0.16em}\begin{array}{l}
\bm{a}^{\hspace{-0.1ex}1} \hspace{-0.1ex} \dotp \bm{a}_{1} = 1 \\
\bm{a}^{\hspace{-0.1ex}1} \hspace{-0.1ex} \dotp \bm{a}_{2} = 0 \\
\bm{a}^{\hspace{-0.1ex}1} \hspace{-0.1ex} \dotp \bm{a}_{3} = 0 \\[0.05em]
\end{array}\right.$} \,\Rightarrow\hspace{0.32em}
%
\scalebox{0.92}[0.92]{$\left\{\hspace{-0.12em}\begin{array}{l}
\bm{a}^{\hspace{-0.1ex}1} \hspace{-0.1ex} \dotp \hspace{0.2ex} \bm{a}_{1} = 1 \\[0.1em]
\gamma \hspace{0.1ex} \bm{a}^{\hspace{-0.1ex}1} \hspace{-0.1ex} = \hspace{0.1ex} \bm{a}_2 \hspace{-0.1ex} \times \bm{a}_3 \\[0.08em]
\end{array}\right.$} \Rightarrow\hspace{0.32em}
%
\scalebox{0.96}[0.96]{$\left\{\hspace{-0.11em}\begin{array}{l}
\bm{a}^{\hspace{-0.1ex}1} \hspace{-0.2ex} =
\displaystyle \nicefrac{\scalebox{0.95}{$1$}\hspace{0.1ex}}{\scalebox{1.02}{$\gamma$}} \hspace{0.5ex} \bm{a}_2 \hspace{-0.1ex} \times \bm{a}_3 \hspace{0.1ex} \\[0.08em]
\gamma = \hspace{0.1ex} \bm{a}_2 \hspace{-0.1ex} \times \bm{a}_3 \hspace{0.1ex} \dotp \hspace{0.25ex} \bm{a}_{1} \\[0.08em]
\end{array}\right.$}
\end{equation*}

\vspace{-0.1em} \noindent Коэффициент~$\gamma$ получился равным объёму параллелепипеда, построенного на~векторах~$\bm{a}_i$; в~\pararef{para:crossproduct+levicivita} тот~же объём был представлен как~\!${\sqrt{\hspace{-0.36ex}\mathstrut{\textsl{g}}}\hspace{0.16ex}}$, и~это неспроста, поскольку он совпадает с~квадратным корнем из~грамиана~\hbox{$\textsl{g} \hspace{0.1ex} \hspace{0.25ex} \equiv \hspace{0.2ex} \operatorname{det} \textsl{g}_{ij}$\hspace{-0.12ex}}~--- определителя симметричной матрицы~\hbox{J.\,P.\,Gram’а}~${\textsl{g}_{ij} \hspace{-0.1ex} \hspace{0.16ex} \equiv \hspace{0.2ex} \bm{a}_i \dotp \bm{a}_j}$.

%%Матрица Грама~${\textsl{g}_{ik} \hspace{-0.1ex} \equiv \bm{e}_i \dotp \bm{e}_k}$, грамиан~${\textsl{g} \hspace{0.1ex} \equiv \operatorname{det} \textsl{g}_{ik}}$.

Доказательство напоминает вывод~\eqref{doubleveblen}. \inquotes{Тройное} произведение ${\bm{a}_i \hspace{-0.25ex} \times \hspace{-0.25ex} \bm{a}_j \dotp \hspace{0.1ex} \bm{a}_k}$ в~каком\hbox{-}нибудь орто\-нормаль\-ном базисе~${\bm{e}_i}$ вычисл\'{и}мо как~детерминант~(с~\inquotes{$-$} для~\inquotes{левой} базисной тройки) по~строкам
\vspace{-0.4em}\[
\levicivita_{ijk} =\hspace{0.16ex} \bm{a}_i \hspace{-0.25ex} \times \hspace{-0.25ex} \bm{a}_j \dotp \hspace{0.1ex} \bm{a}_k =
\hspace{0.1ex} \pm \operatorname{det}\hspace{-0.1ex}
\scalebox{0.92}[0.92]{$\left[\hspace{-0.2ex}\begin{array}{c@{\hspace{0.64em}}c@{\hspace{0.64em}}c}
\bm{a}_i \narrowdotp\hspace{0.12ex} \bm{e}_1 & \bm{a}_i \narrowdotp\hspace{0.12ex} \bm{e}_2 & \bm{a}_i \narrowdotp\hspace{0.12ex} \bm{e}_3 \\
\bm{a}_j \narrowdotp\hspace{0.12ex} \bm{e}_1 & \bm{a}_j \narrowdotp\hspace{0.12ex} \bm{e}_2 & \bm{a}_j \narrowdotp\hspace{0.12ex} \bm{e}_3 \\
\bm{a}_k \narrowdotp\hspace{0.12ex} \bm{e}_1 & \bm{a}_k \narrowdotp\hspace{0.12ex} \bm{e}_2 & \bm{a}_k \narrowdotp\hspace{0.12ex} \bm{e}_3
\end{array}\hspace{-0.12ex}\right]$}
\]
или по~столбцам
\vspace{-0.4em}\[
\levicivita_{pqr} =\hspace{0.16ex} \bm{a}_p \hspace{-0.25ex} \times \hspace{-0.25ex} \bm{a}_q \dotp \hspace{0.1ex} \bm{a}_r =
\hspace{0.1ex} \pm \operatorname{det}\hspace{-0.1ex}
\scalebox{0.92}[0.92]{$\left[\hspace{-0.2ex}\begin{array}{c@{\hspace{0.64em}}c@{\hspace{0.64em}}c}
\bm{a}_p \narrowdotp\hspace{0.12ex} \bm{e}_1 & \bm{a}_q \narrowdotp\hspace{0.12ex} \bm{e}_1 & \bm{a}_r \narrowdotp\hspace{0.12ex} \bm{e}_1 \\
\bm{a}_p \narrowdotp\hspace{0.12ex} \bm{e}_2 & \bm{a}_q \narrowdotp\hspace{0.12ex} \bm{e}_2 & \bm{a}_r \narrowdotp\hspace{0.12ex} \bm{e}_2 \\
\bm{a}_p \narrowdotp\hspace{0.12ex} \bm{e}_3 & \bm{a}_q \narrowdotp\hspace{0.12ex} \bm{e}_3 & \bm{a}_r \narrowdotp\hspace{0.12ex} \bm{e}_3
\end{array}\hspace{-0.12ex}\right]$} .
\]

\vspace{0.2em} \noindent Произведение определителей~${\levicivita_{ijk} \levicivita_{pqr}}$ равно определителю произведения матриц, элементы последнего~--- суммы вида ${\bm{a}_i \dotp \bm{e}_s \bm{e}_s \hspace{-0.2ex} \dotp \bm{a}_p \hspace{-0.12ex}= \bm{a}_i \dotp \bm{E} \dotp \bm{a}_p \hspace{-0.12ex}= \bm{a}_i \dotp \bm{a}_p}$ %%\hspace{0.1ex}:
\[
\levicivita_{ijk} \levicivita_{pqr} = \hspace{0.25ex}
\operatorname{det}\hspace{-0.1ex}
\scalebox{0.92}[0.92]{$\left[\hspace{-0.16ex}\begin{array}{c@{\hspace{0.64em}}c@{\hspace{0.64em}}c}
\bm{a}_i \narrowdotp\hspace{0.12ex} \bm{a}_p & \bm{a}_i \narrowdotp\hspace{0.12ex} \bm{a}_q & \bm{a}_i \narrowdotp\hspace{0.12ex} \bm{a}_r \\
\bm{a}_j \narrowdotp\hspace{0.12ex} \bm{a}_p & \bm{a}_j \narrowdotp\hspace{0.12ex} \bm{a}_q & \bm{a}_j \narrowdotp\hspace{0.12ex} \bm{a}_r \\
\bm{a}_k \narrowdotp\hspace{0.12ex} \bm{a}_p & \bm{a}_k \narrowdotp\hspace{0.12ex} \bm{a}_q & \bm{a}_k \narrowdotp\hspace{0.12ex} \bm{a}_r
\end{array}\hspace{-0.12ex}\right]$} ;
\]

\vspace{0.12em}\noindent ${i \narroweq p \narroweq 1}$, ${j \narroweq q \narroweq 2}$, ${k \narroweq r \narroweq 3}$ ${\,\Rightarrow}$ ${\levicivita_{123} \hspace{0.2ex} \levicivita_{123} \hspace{-0.12ex}= \underset{i,\hspace{0.1ex}j}{\operatorname{det}} \left( \bm{a}_i \dotp \bm{a}_j \right) \hspace{-0.1ex}= \underset{i,\hspace{0.1ex}j}{\operatorname{det}} \, \textsl{g}_{ij}}$.

\begin{comment} %%
\[\scalebox{0.92}[0.92]{$\begin{array}{l@{\hspace{0.3em}}c@{\hspace{0.36em}}r}
\gamma \hspace{0.1ex} \bm{a}^{\hspace{-0.1ex}1} & = & \hspace{0.1ex} \bm{a}_2 \hspace{-0.1ex} \times \bm{a}_3 \hspace{0.1ex} \\[0.2em]
\gamma \hspace{0.1ex} \bm{a}^2 & = & \hspace{0.1ex} \bm{a}_3 \hspace{-0.1ex} \times \bm{a}_1 \hspace{0.1ex} \\[0.2em]
\gamma \hspace{0.1ex} \bm{a}^3 & = & \hspace{0.1ex} \bm{a}_1 \hspace{-0.16ex} \times \bm{a}_2 \hspace{0.1ex}
\end{array}$}\]
\end{comment} %%

Представляя~${\bm{a}^{\hspace{-0.1ex}1}\hspace{-0.1ex}}$ и~остальные векторы кобазиза суммой
\vspace{1.1em}\[\begin{array}{l@{\hspace{0.3em}}c@{\hspace{0.36em}}r}
2 \hspace{0.1ex} \scalebox{0.95}[0.96]{$\sqrt{\hspace{-0.36ex}\mathstrut{\textsl{g}}}$} \hspace{0.5ex} \bm{a}^{\hspace{-0.1ex}1} & = & \bm{a}_2 \times \bm{a}_3 \hspace{0.2ex} \tikzmark{BeginPlusToMinus} - \hspace{0.2ex} \bm{a}_3 \times \bm{a}_2 \tikzmark{EndPlusToMinus} \hspace{0.2ex},
\end{array}\]%
\AddOverBrace[line width=.75pt][0,-0.2ex]{BeginPlusToMinus}{EndPlusToMinus}%
{${\scriptstyle {+ \hspace{0.4ex} \bm{a}_2 \hspace{0.1ex} \times \hspace{0.2ex} \bm{a}_3}}$}

\vspace{-1.2em} \noindent приходим к~общей формуле
\vspace{0.25em}\begin{equation}\label{basisvectorstocobasisvectors}
\bm{a}^{\hspace{-0.05ex}i} = \displaystyle \frac{\raisemath{-0.4ex}{1}}{2 \hspace{0.1ex} \scalebox{0.95}[0.96]{$\sqrt{\hspace{-0.36ex}\mathstrut{\textsl{g}}}$}} \hspace{0.5ex} e^{ijk} \hspace{0.1ex} \bm{a}_j \hspace{-0.1ex} \times \bm{a}_k \hspace{0.1ex}, \:\:
\sqrt{\hspace{-0.36ex}\mathstrut{\textsl{g}}} \hspace{0.32ex} \equiv \hspace{0.2ex} \bm{a}_1 \hspace{-0.2ex} \times \bm{a}_2 \hspace{0.1ex} \dotp \hspace{0.25ex} \bm{a}_3 \hspace{0.2ex}.
\end{equation}
\noindent Здесь ${e^{ijk}}$~\hspace{-0.2ex}--- по\hbox{-}прежнему символ Веблена~(${\pm 1}$ или~$0$), и~по~\eqref{veblencontraction} ${e^{ijk} e_{jkn} \hspace{-0.2ex} = 2 \hspace{0.1ex} \delta_n^{\hspace{0.1ex}i}}$. Произведение~${\bm{a}_j \hspace{-0.1ex} \times \bm{a}_k = \hspace{0.1ex} \levicivita_{jkn} \hspace{0.2ex} \bm{a}^n}$\hspace{-0.16ex}, а~компоненты тензора Л\'{е}ви\hbox{-\!}Чив\'{и}ты~${\levicivita_{jkn} = \sqrt{\hspace{-0.36ex}\mathstrut{\textsl{g}}} \hspace{0.4ex} e_{jkn}}$. Так что
\vspace{0.2em}\[\scalebox{0.96}[0.96]{$%
\begin{array}{l@{\hspace{0.3em}}c@{\hspace{0.36em}}r}
\bm{a}^{\hspace{-0.1ex}1} & = & \displaystyle \nicefrac{\scalebox{0.95}{$1$}}{\hspace{-0.25ex}\sqrt{\hspace{-0.2ex}\scalebox{0.96}{$\mathstrut{\textsl{g}}$}}} \hspace{0.2ex} \left( \hspace{0.1ex} \bm{a}_2 \hspace{-0.1ex} \times \bm{a}_3 \hspace{0.1ex} \right) \hspace{-0.3ex},
\end{array}
\begin{array}{l@{\hspace{0.3em}}c@{\hspace{0.36em}}r}
\bm{a}^2 & = & \displaystyle \nicefrac{\scalebox{0.95}{$1$}}{\hspace{-0.25ex}\sqrt{\hspace{-0.2ex}\scalebox{0.96}{$\mathstrut{\textsl{g}}$}}} \hspace{0.2ex} \left( \hspace{0.1ex} \bm{a}_3 \hspace{-0.1ex} \times \bm{a}_1 \hspace{0.1ex} \right) \hspace{-0.3ex},
\end{array}
\begin{array}{l@{\hspace{0.3em}}c@{\hspace{0.36em}}r}
\bm{a}^3 & = & \displaystyle \nicefrac{\scalebox{0.95}{$1$}}{\hspace{-0.25ex}\sqrt{\hspace{-0.2ex}\scalebox{0.96}{$\mathstrut{\textsl{g}}$}}} \hspace{0.2ex} \left( \hspace{0.1ex} \bm{a}_1 \hspace{-0.16ex} \times \bm{a}_2 \hspace{0.1ex} \right) \hspace{-0.28ex}.
\end{array}%
$}\]

\begin{tcolorbox}
\small\setlength{\abovedisplayskip}{2pt}\setlength{\belowdisplayskip}{2pt}

\emph{Example.} Get cobasis for basis~$\bm{a}_i$ when
\[ \begin{array}{l}
\bm{a}_1 = \bm{e}_1 \hspace{-0.2ex} + \bm{e}_2 \hspace{.1ex},\\
\bm{a}_2 = \bm{e}_1 \hspace{-0.2ex} + \bm{e}_3 \hspace{.1ex},\\
\bm{a}_3 = \bm{e}_2 \hspace{-0.2ex} + \bm{e}_3 \hspace{.1ex}.
\end{array} \]

\[
\sqrt{\hspace{-0.36ex}\mathstrut{\textsl{g}}} \hspace{.32ex} =
\hspace{0.2ex} \bm{a}_1 \hspace{-0.2ex} \times \bm{a}_2 \hspace{0.1ex} \dotp \hspace{0.25ex} \bm{a}_3 \hspace{0.2ex} =
- \operatorname{det}\hspace{-0.1ex}
\scalebox{0.92}[0.92]{$\left[\hspace{-0.16ex}\begin{array}{c@{\hspace{.64em}}c@{\hspace{.64em}}c}
1 & 1 & 0 \\
1 & 0 & 1 \\
0 & 1 & 1
\end{array}\hspace{-0.12ex}\right]$} \hspace{-0.5ex} = 2 \hspace{.25ex};
\]
\[
\bm{a}_2 \hspace{-0.2ex} \times \bm{a}_3 = - \operatorname{det}\hspace{-0.1ex}
\scalebox{0.92}[0.92]{$\left[\hspace{-0.16ex}\begin{array}{c@{\hspace{.6em}}c@{\hspace{.5em}}c}
1 & 0 & \bm{e}_1 \\
0 & 1 & \bm{e}_2 \\
1 & 1 & \bm{e}_3
\end{array}\hspace{-0.2ex}\right]$} \hspace{-0.5ex} = \bm{e}_1 \hspace{-0.2ex} + \bm{e}_2 \hspace{-0.2ex} - \bm{e}_3 \hspace{.1ex},
\]
\[
\bm{a}_3 \hspace{-0.2ex} \times \bm{a}_1 = - \operatorname{det}\hspace{-0.1ex}
\scalebox{0.92}[0.92]{$\left[\hspace{-0.16ex}\begin{array}{c@{\hspace{.6em}}c@{\hspace{.5em}}c}
0 & 1 & \bm{e}_1 \\
1 & 1 & \bm{e}_2 \\
1 & 0 & \bm{e}_3
\end{array}\hspace{-0.2ex}\right]$} \hspace{-0.5ex} = \bm{e}_1 \hspace{-0.2ex} + \bm{e}_3 \hspace{-0.2ex} - \bm{e}_2 \hspace{.1ex},
\]
\[
\bm{a}_1 \hspace{-0.2ex} \times \bm{a}_2 = - \operatorname{det}\hspace{-0.1ex}
\scalebox{0.92}[0.92]{$\left[\hspace{-0.16ex}\begin{array}{c@{\hspace{.6em}}c@{\hspace{.5em}}c}
1 & 1 & \bm{e}_1 \\
1 & 0 & \bm{e}_2 \\
0 & 1 & \bm{e}_3
\end{array}\hspace{-0.2ex}\right]$} \hspace{-0.5ex} = \bm{e}_2 \hspace{-0.2ex} + \bm{e}_3 \hspace{-0.2ex} - \bm{e}_1 \hspace{.1ex}
\]

\vspace{-0.4em}and finally
\vspace{-0.4em}\[\begin{array}{l}
\bm{a}^1 \hspace{-0.2ex}=\hspace{.1ex} \smalldisplaystyleonehalf \hspace{-0.12ex} \left(^{\mathstrut} \bm{e}_1 \hspace{-0.2ex} + \bm{e}_2 \hspace{-0.2ex} - \bm{e}_3 \right) \hspace{-0.5ex},\\[.5em]
\bm{a}^2 \hspace{-0.2ex}=\hspace{.1ex} \smalldisplaystyleonehalf \hspace{-0.12ex} \left(^{\mathstrut} \bm{e}_1 \hspace{-0.2ex} - \bm{e}_2 \hspace{-0.2ex} + \bm{e}_3 \right) \hspace{-0.5ex},\\[.5em]
\bm{a}^3 \hspace{-0.2ex}=\hspace{.1ex} \smalldisplaystyleonehalf \hspace{-0.12ex} \left(^{\mathstrut} \hspace{-0.2ex} {- \bm{e}_1} \hspace{-0.2ex} + \bm{e}_2 \hspace{-0.2ex} + \bm{e}_3 \right) \hspace{-0.5ex}.
\end{array}\]

\par\end{tcolorbox}

Имея кобазис, возможно не~только разложить по~нему любой вектор~(\figref{fig:DecompositionOfVector}), но~и найти коэффициенты разложения~\eqref{decompositionbyobliquebasis}:
\begin{equation}\begin{array}{c}
\bm{v} = v^{i} \hspace{-0.1ex} \bm{a}_i = v_{i} \hspace{0.1ex} \bm{a}^{\hspace{-0.05ex}i} \hspace{-0.25ex}, \\[0.16em]
\bm{v} \dotp \bm{a}^{\hspace{-0.05ex}i} = v^{k} \hspace{-0.1ex} \bm{a}_k \hspace{-0.1ex} \dotp \bm{a}^{\hspace{-0.05ex}i} = v^{i} \hspace{-0.25ex}, \:\;
v_{i} \hspace{-0.1ex} = \bm{v} \dotp \bm{a}_i \hspace{0.1ex}.
\end{array}\end{equation}
\noindent Коэффициенты~${v_i}$ называются ко\-вариант\-ными компонентами вектора~$\bm{v}$, а~${v^i \hspace{-0.25ex}}$~--- его контра\-вариант\-ными%
\footnote{Потому что они меняются обратно~(contra) изменению длин базисных векторов~${\bm{a}_i}$.}\hspace{-0.2ex}
компонентами.

Есть литература о~тензорах, где различают ко\-вариант\-ные и~контра\-вариант\-ные... векторы~(и~\inquotes{ковекторы}). Не~ст\'{о}ит вводить читателя в~заблуждение: вектор\hbox{-}то один и~тот~же, просто при~разложении по~двум разным базисам у~него два набора компонент.

% ~ ~ ~ ~ ~
% converts spherical coordinates to cartesian
\newcommand{\tdsphericaltocartesian}[6]{%
\def\thecostheta{cos(#2)}%
\def\thesintheta{sin(#2)}%
\def\thecosphi{cos(#3)}%
\def\thesinphi{sin(#3)}%
\pgfmathsetmacro{#4}{ #1 * \thesintheta * \thecosphi }%
\pgfmathsetmacro{#5}{ #1 * \thesintheta * \thesinphi }%
\pgfmathsetmacro{#6}{ #1 * \thecostheta }%
}

% takes two points as cartesian {x}{y}{z} and calculates cross product of their location vectors
% placing the result into last three arguments
\newcommand{\tdcrossproductcartesian}[9]{%
\def\crossz{ #1 * #5 - #2 * #4 }%
\def\crossx{ #2 * #6 - #3 * #5 }%
\def\crossy{ #3 * #4 - #1 * #6 }%
\pgfmathsetmacro{#7}{\crossx}%
\pgfmathsetmacro{#8}{\crossy}%
\pgfmathsetmacro{#9}{\crossz}%
}

% takes two points as spherical {length}{anglefromz}{anglefromx} and calculates cross product of their location vectors
% placing the result as cartesian {x}{y}{z} into last three arguments
\newcommand{\tdcrossproductspherical}[9]{%
%
\tdplotsinandcos{\firstsintheta}{\firstcostheta}{#2}%
\tdplotsinandcos{\firstsinphi}{\firstcosphi}{#3}%
\def\firstx{ #1 * \firstsintheta * \firstcosphi }%
\def\firsty{ #1 * \firstsintheta * \firstsinphi }%
\def\firstz{ #1 * \firstcostheta }%
%
\tdplotsinandcos{\secondsintheta}{\secondcostheta}{#5}%
\tdplotsinandcos{\secondsinphi}{\secondcosphi}{#6}%
\def\secondx{ #4 * \secondsintheta * \secondcosphi }%
\def\secondy{ #4 * \secondsintheta * \secondsinphi }%
\def\secondz{ #4 * \secondcostheta }%
%
\def\crossz{ \firstx * \secondy - \firsty * \secondx }%
\def\crossx{ \firsty * \secondz - \firstz * \secondy }%
\def\crossy{ \firstz * \secondx - \firstx * \secondz }%
\pgfmathsetmacro{#7}{\crossx}%
\pgfmathsetmacro{#8}{\crossy}%
\pgfmathsetmacro{#9}{\crossz}%
}

% calculates dot product of location vectors of two 3D points specified by cartesian coordinates
\newcommand{\tddotproductcartesian}[7]{%
\edef\tddotproductcartesianxint{ \xinttheexpr round( #1 * #4 + #2 * #5 + #3 * #6 , 10 ) \relax }%
\pgfmathsetmacro{#7}{\tddotproductcartesianxint}%
}

% calculates dot product of location vectors of two 3D points specified by spherical coordinates
\newcommand{\tddotproductspherical}[7]{%
%
\tdplotsinandcos{\firstsintheta}{\firstcostheta}{#2}%
\tdplotsinandcos{\firstsinphi}{\firstcosphi}{#3}%
\def\firstx{ ( #1 * \firstsintheta * \firstcosphi ) }%
\def\firsty{ ( #1 * \firstsintheta * \firstsinphi ) }%
\def\firstz{ ( #1 * \firstcostheta ) }%
%
\tdplotsinandcos{\secondsintheta}{\secondcostheta}{#5}%
\tdplotsinandcos{\secondsinphi}{\secondcosphi}{#6}%
\def\secondx{ ( #4 * \secondsintheta * \secondcosphi ) }%
\def\secondy{ ( #4 * \secondsintheta * \secondsinphi ) }%
\def\secondz{ ( #4 * \secondcostheta ) }%
%
\edef\tddotproductsphericalxint{ \xinttheexpr round( \firstx * \secondx + \firsty * \secondy + \firstz * \secondz , 10 ) \relax }%
\pgfmathsetmacro{#7}{\tddotproductsphericalxint}%
}

% takes three points as spherical {length}{anglefromz}{anglefromx}
% and calculates triple product r1 × r2 • r3 of their location vectors
% the result is placed into \LastThreeDTripleProduct
\newcommand{\tdtripleproductspherical}[9]{%
%
\tdsphericaltocartesian{#1}{#2}{#3}{\firstx}{\firsty}{\firstz}
\tdsphericaltocartesian{#4}{#5}{#6}{\secondx}{\secondy}{\secondz}
\tdsphericaltocartesian{#7}{#8}{#9}{\thirdx}{\thirdy}{\thirdz}
%
\def\crossz{ ( \firstx * \secondy - \firsty * \secondx ) }%
\def\crossx{ ( \firsty * \secondz - \firstz * \secondy ) }%
\def\crossy{ ( \firstz * \secondx - \firstx * \secondz ) }%
%
\edef\LastThreeDTripleProduct{ \xinttheexpr round( \crossx * \thirdx + \crossy * \thirdy + \crossz * \thirdz , 10 ) \relax }%
}

% orientation of camera
\def\cameraTheta{36} \def\cameraPhi{98}
%% \def\cameraTheta{89.99} \def\cameraPhi{120}
	% 90 gives “You asked me to calculate `1/0.0', but I cannot divide any number by zero.”
\tdplotsetmaincoords{\cameraTheta}{\cameraPhi}

% vectors of basis
\pgfmathsetmacro{\firstlength}{0.69}
	\pgfmathsetmacro{\firstanglefromz}{71}
	\pgfmathsetmacro{\firstanglefromx}{-16}
\pgfmathsetmacro{\secondlength}{0.88}
	\pgfmathsetmacro{\secondanglefromz}{86}
	\pgfmathsetmacro{\secondanglefromx}{77}
\pgfmathsetmacro{\thirdlength}{0.96}
	\pgfmathsetmacro{\thirdanglefromz}{-19}
	\pgfmathsetmacro{\thirdanglefromx}{45}

\tdsphericaltocartesian%
	{\firstlength}{\firstanglefromz}{\firstanglefromx}%
	{\firstcartesianx}{\firstcartesiany}{\firstcartesianz}
\tdsphericaltocartesian%
	{\secondlength}{\secondanglefromz}{\secondanglefromx}%
	{\secondcartesianx}{\secondcartesiany}{\secondcartesianz}
\tdsphericaltocartesian%
	{\thirdlength}{\thirdanglefromz}{\thirdanglefromx}%
	{\thirdcartesianx}{\thirdcartesiany}{\thirdcartesianz}

% some but very important vector
\pgfmathsetmacro{\lengthofvector}{3.33}
	\pgfmathsetmacro{\vectoranglefromz}{33}
	\pgfmathsetmacro{\vectoranglefromx}{44}

\tdsphericaltocartesian%
	{\lengthofvector}{\vectoranglefromz}{\vectoranglefromx}%
	{\vectorcartesianx}{\vectorcartesiany}{\vectorcartesianz}

%%\begin{comment} %%
\begin{minipage}{\textwidth}
\hfill\[\scalebox{0.9}[0.9]{$\begin{array}{l@{\hspace{0.2\textwidth}}l@{\hspace{1.2em}}l@{\hspace{0.8em}}l}
\theta = \pgfmathprintnumber{\cameraTheta}\degree \hspace{0.8em}
\phi = \pgfmathprintnumber{\cameraPhi}\degree
& \scalebox{1.05}{$\bm{v}^{\varrho} = \pgfmathprintnumber{\lengthofvector}$} &
	\scalebox{1.05}{$\bm{v}^{\theta} = \pgfmathprintnumber{\vectoranglefromz}\degree$} &
	\scalebox{1.05}{$\bm{v}^{\phi} = \pgfmathprintnumber{\vectoranglefromx}\degree$} \\[0.25em]
%
& {\bm{a}_{1}^{\varrho} = \pgfmathprintnumber{\firstlength}} &
	{\bm{a}_{1}^{\theta} = \pgfmathprintnumber{\firstanglefromz}\degree} &
		{\bm{a}_{1}^{\phi} = \pgfmathprintnumber{\firstanglefromx}\degree} \\[0.1em]
& {\bm{a}_{2}^{\varrho} = \pgfmathprintnumber{\secondlength}} &
	{\bm{a}_{2}^{\theta} = \pgfmathprintnumber{\secondanglefromz}\degree} &
		{\bm{a}_{2}^{\phi} = \pgfmathprintnumber{\secondanglefromx}\degree} \\[0.1em]
& {\bm{a}_{3}^{\varrho} = \pgfmathprintnumber{\thirdlength}} &
	{\bm{a}_{3}^{\theta} = \pgfmathprintnumber{\thirdanglefromz}\degree} &
		{\bm{a}_{3}^{\phi} = \pgfmathprintnumber{\thirdanglefromx}\degree}
\end{array}$}\]
\end{minipage}
%%\end{comment} %%

\begin{figure}[!htbp]
\begin{center}

\vspace{0.1em}
\begin{tikzpicture}[scale=3.2, tdplot_main_coords] % tdplot_main_coords style to use 3dplot

	\coordinate (O) at (0,0,0);

	% define axes
	\tdplotsetcoord{A1}{\firstlength}{\firstanglefromz}{\firstanglefromx}
	\tdplotsetcoord{A2}{\secondlength}{\secondanglefromz}{\secondanglefromx}
	\tdplotsetcoord{A3}{\thirdlength}{\thirdanglefromz}{\thirdanglefromx}

	% define vector
	\tdplotsetcoord{V}{\lengthofvector}{\vectoranglefromz}{\vectoranglefromx} % {length}{angle from z}{angle from x}

	% square root of Gram matrix’ determinant is a1 × a2 • a3
	\tdtripleproductspherical%
		{\firstlength}{\firstanglefromz}{\firstanglefromx}%
		{\secondlength}{\secondanglefromz}{\secondanglefromx}%
		{\thirdlength}{\thirdanglefromz}{\thirdanglefromx}
	\edef\sqrtGramian{\xinttheexpr round( \LastThreeDTripleProduct, 10 )\relax}
	\edef\inverseOfSqrtGramian{\xinttheexpr round( 1 / \sqrtGramian, 10 )\relax}

	\node[fill=white!50, inner sep=0pt, outer sep=2pt] at (1.2,0,-1.45)
		{$\scalebox{0.9}{$\begin{array}{r}\bm{a}_1 \hspace{-0.4ex} \times \hspace{-0.3ex} \bm{a}_2 \dotp \hspace{0.2ex} \bm{a}_3 \hspace{-0.2ex} = \hspace{-0.2ex} \sqrt{\hspace{-0.36ex}\mathstrut{\textsl{g}}} \hspace{0.1ex} = \hspace{-0.2ex} \pgfmathprintnumber[fixed, precision=5]{\sqrtGramian} \\[0.25em]
		\displaystyle \nicefrac{\scalebox{0.95}{$1$}}{\hspace{-0.25ex}\sqrt{\hspace{-0.2ex}\scalebox{0.96}{$\mathstrut{\textsl{g}}$}}} \hspace{0.1ex} = \hspace{-0.2ex} \pgfmathprintnumber[fixed, precision=5]{\inverseOfSqrtGramian}\end{array}$}$};

	% calculate vectors of cobasis
	\tdcrossproductspherical%
		{\firstlength}{\firstanglefromz}{\firstanglefromx}%
		{\secondlength}{\secondanglefromz}{\secondanglefromx}%
		{\firstsecondcrossx}{\firstsecondcrossy}{\firstsecondcrossz}
	\coordinate (cross12) at (\firstsecondcrossx, \firstsecondcrossy, \firstsecondcrossz);
	\draw [line width=1.25pt, orange, -{Latex[round, length=3.6mm, width=2.4mm]}]
		(O) -- (cross12)
		node[pos=0.64, above right, inner sep=0pt, outer sep=6pt]
		{$\scalebox{0.8}{$\bm{a}_1 \hspace{-0.4ex} \times \hspace{-0.3ex} \bm{a}_2$}$};

	\tdcrossproductspherical%
		{\thirdlength}{\thirdanglefromz}{\thirdanglefromx}%
		{\firstlength}{\firstanglefromz}{\firstanglefromx}%
		{\thirdfirstcrossx}{\thirdfirstcrossy}{\thirdfirstcrossz}
	\coordinate (cross31) at (\thirdfirstcrossx, \thirdfirstcrossy, \thirdfirstcrossz);
	\draw [line width=1.25pt, orange, -{Latex[round, length=3.6mm, width=2.4mm]}]
		(O) -- (cross31)
		node[pos=0.86, above, inner sep=0pt, outer sep=5pt]
		{$\scalebox{0.8}{$\bm{a}_3 \hspace{-0.4ex} \times \hspace{-0.3ex} \bm{a}_1$}$};

	\tdcrossproductspherical%
		{\secondlength}{\secondanglefromz}{\secondanglefromx}%
		{\thirdlength}{\thirdanglefromz}{\thirdanglefromx}%
		{\secondthirdcrossx}{\secondthirdcrossy}{\secondthirdcrossz}
	\coordinate (cross23) at (\secondthirdcrossx, \secondthirdcrossy, \secondthirdcrossz);
	\draw [line width=1.25pt, orange, -{Latex[round, length=3.6mm, width=2.4mm]}]
		(O) -- (cross23)
		node[pos=0.88, below right, inner sep=0pt, outer sep=2.5pt]
		{$\scalebox{0.8}{$\bm{a}_2 \hspace{-0.4ex} \times \hspace{-0.3ex} \bm{a}_3$}$};

	\coordinate (coA3) at ($ \inverseOfSqrtGramian*(cross12) $);
	\coordinate (coA2) at ($ \inverseOfSqrtGramian*(cross31) $);
	\coordinate (coA1) at ($ \inverseOfSqrtGramian*(cross23) $);

	% get vector’s projection on a1 & a2 plane (third co-vector a^3 is normal to that plane)
	% it’s as deep down parallel to a3 as v^3 = v • a^3 in units of a3
	\tddotproductcartesian%
		{\vectorcartesianx}{\vectorcartesiany}{\vectorcartesianz}%
		{\inverseOfSqrtGramian*\firstsecondcrossx}%
			{\inverseOfSqrtGramian*\firstsecondcrossy}%
				{\inverseOfSqrtGramian*\firstsecondcrossz}%
		{\vectorthirdcoco}
	% get third co-component and translate it to vector’s head
	\coordinate (Vcomponent3) at ($ \vectorthirdcoco*(A3) $);
	\coordinate (VcomponentXY) at ($ (V) - (Vcomponent3) $);

	% decompose vector via initial basis
	\coordinate (ParallelToSecond) at ($ (VcomponentXY) - (A2) $);
	\coordinate (ParallelToFirst) at ($ (VcomponentXY) - (A1) $);
	\coordinate (Vcomponent1) at (intersection of VcomponentXY--ParallelToSecond and O--A1);
	\coordinate (Vcomponent2) at (intersection of VcomponentXY--ParallelToFirst and O--A2);

	\draw [line width=0.4pt, dotted, color=blue] (O) -- (VcomponentXY); % projection on first & second vectors’ plane

	\draw [line width=0.4pt, dotted, color=blue] (V) -- (VcomponentXY);
	\draw [line width=0.4pt, dotted, color=blue] (VcomponentXY) -- (Vcomponent1);
	\draw [line width=0.4pt, dotted, color=blue] (VcomponentXY) -- (Vcomponent2);

	% check a^1 × a^2 direction to be the same as a3
	\tdcrossproductcartesian%
		{\inverseOfSqrtGramian*\secondthirdcrossx}%
			{\inverseOfSqrtGramian*\secondthirdcrossy}%
				{\inverseOfSqrtGramian*\secondthirdcrossz}%
		{\inverseOfSqrtGramian*\thirdfirstcrossx}%
			{\inverseOfSqrtGramian*\thirdfirstcrossy}%
				{\inverseOfSqrtGramian*\thirdfirstcrossz}%
		{\CofirstCosecondOrthoX}{\CofirstCosecondOrthoY}{\CofirstCosecondOrthoZ}
	\coordinate (co1co2ortho) at (\CofirstCosecondOrthoX, \CofirstCosecondOrthoY, \CofirstCosecondOrthoZ);
	\draw [line width=1.25pt, blue!50, -{Latex[round, length=3.6mm, width=2.4mm]}]
		(O) -- ($ \sqrtGramian*(co1co2ortho) $);

	% length of a^3
	\tddotproductcartesian%
		{\inverseOfSqrtGramian*\thirdfirstcrossx}%
			{\inverseOfSqrtGramian*\thirdfirstcrossy}%
				{\inverseOfSqrtGramian*\thirdfirstcrossz}%
		{\inverseOfSqrtGramian*\thirdfirstcrossx}%
			{\inverseOfSqrtGramian*\thirdfirstcrossy}%
				{\inverseOfSqrtGramian*\thirdfirstcrossz}%
		{\squaredlengthofthirdcovector}
	%%\node[fill=white!50, inner sep=0pt, outer sep=4pt] at (0,0,-2.25)
		%%{$\scalebox{0.9}{$ | \hspace{0.1ex} \bm{a}^{\hspace{-0.1ex}3} \hspace{0.06ex} | \hspace{0.1ex} =
			%%\sqrt{\pgfmathprintnumber[fixed, precision=5]{\squaredlengthofthirdcovector}} $}$};

	% get vector’s projection on a^1 & a^2 plane (third basis vector a3 is normal to that plane)
	% it’s as deep down parallel to a^3 as v3 = v • a3 in units of a^3
	\tddotproductspherical%
		{\lengthofvector}{\vectoranglefromz}{\vectoranglefromx}%
		{\thirdlength}{\thirdanglefromz}{\thirdanglefromx}%
		{\vectorthirdcomponent}
	% get third co-component and translate it to vector’s head
	\coordinate (Vcoco3) at ($ \vectorthirdcomponent*(coA3) $);
	\coordinate (VcocoXY) at ($ (V) - (Vcoco3) $);

	% decompose vector via cobasis
	%%\coordinate (ParallelToCothird) at ($ (V) - (coA3) $);
	\coordinate (ParallelToCosecond) at ($ (VcocoXY) - (coA2) $);
	\coordinate (ParallelToCofirst) at ($ (VcocoXY) - (coA1) $);
	\coordinate (Vcoco1) at (intersection of VcocoXY--ParallelToCosecond and O--coA1);
	\coordinate (Vcoco2) at (intersection of VcocoXY--ParallelToCofirst and O--coA2);

	\draw [line width=0.4pt, red] (O) -- (Vcoco2);

	\draw [line width=0.4pt, dotted, color=red] (O) -- (VcocoXY);

	\draw [line width=0.4pt, dotted, color=red] (V) -- (VcocoXY);
	\draw [line width=0.4pt, dotted, color=red] (VcocoXY) -- (Vcoco1);
	\draw [line width=0.4pt, dotted, color=red] (VcocoXY) -- (Vcoco2);

	% draw parallelepiped of decomposition
	\coordinate (onPlane23) at ($ (Vcomponent2) + (V) - (VcomponentXY) $);
	\draw [line width=0.4pt, dotted, color=blue] (Vcomponent2) -- (onPlane23);
	\draw [line width=0.4pt, dotted, color=blue] (V) -- (onPlane23);

	\coordinate (onPlane13) at ($ (Vcomponent1) + (V) - (VcomponentXY) $);
	\draw [line width=0.4pt, dotted, color=blue] (Vcomponent1) -- (onPlane13);
	\draw [line width=0.4pt, dotted, color=blue] (V) -- (onPlane13);

	\coordinate (onAxis3) at ($ (V) - (VcomponentXY) $);
	\draw [line width=0.4pt, dotted, color=blue] (O) -- (onAxis3);
	\draw [line width=0.4pt, dotted, color=blue] (onPlane13) -- (onAxis3);
	\draw [line width=0.4pt, dotted, color=blue] (onPlane23) -- (onAxis3);

	\draw [line width=0.4pt, dotted, color=blue] (O) -- (onPlane13);
	\draw [line width=0.4pt, dotted, color=blue] (O) -- (onPlane23);

	% draw co-parallelepiped of co-decomposition
	\coordinate (onCoplane23) at ($ (Vcoco2) + (V) - (VcocoXY) $);
	\draw [line width=0.4pt, dotted, color=red] (Vcoco2) -- (onCoplane23);
	\draw [line width=0.4pt, dotted, color=red] (V) -- (onCoplane23);

	\coordinate (onCoplane13) at ($ (Vcoco1) + (V) - (VcocoXY) $);
	\draw [line width=0.4pt, dotted, color=red] (Vcoco1) -- (onCoplane13);
	\draw [line width=0.4pt, dotted, color=red] (V) -- (onCoplane13);

	\coordinate (onCoAxis3) at ($ (V) - (VcocoXY) $);
	\draw [line width=0.4pt, dotted, color=red] (O) -- (onCoAxis3);
	\draw [line width=0.4pt, dotted, color=red] (onCoplane13) -- (onCoAxis3);
	\draw [line width=0.4pt, dotted, color=red] (onCoplane23) -- (onCoAxis3);

	\draw [line width=0.4pt, dotted, color=red] (O) -- (onCoplane13);
	\draw [line width=0.4pt, dotted, color=red] (O) -- (onCoplane23);

	% draw vectors of cobasis
	\draw [line width=0.4pt, red] (O) -- ($ 1.01*(coA3) $);
	\draw [line width=1.25pt, red, -{Latex[round, length=3.6mm, width=2.4mm]}]
		(O) -- (coA3)
		node[pos=0.8, above right, shape=circle, fill=white, inner sep=-1pt, outer sep=11pt]
		{${\bm{a}}^{\hspace{-0.1ex}3}$};

	\draw [line width=0.4pt, red] (O) -- ($ 1.01*(coA2) $);
	\draw [line width=1.25pt, red, -{Latex[round, length=3.6mm, width=3mm]}]
		(O) -- (coA2)
		node[pos=0.88, above, shape=circle, fill=white, inner sep=-1pt, outer sep=4pt]
		{${\bm{a}}^{\hspace{-0.1ex}2}$};

	\draw [line width=0.4pt, red] (O) -- ($ 1.01*(coA1) $);
	\draw [line width=1.25pt, red, -{Latex[round, length=3.6mm, width=2.4mm]}]
		(O) -- (coA1)
		node[pos=0.92, below right, shape=circle, fill=white, inner sep=-1pt, outer sep=4pt]
		{${\bm{a}}^{\hspace{-0.16ex}1}$};

	% draw vectors of basis
	\draw [line width=0.4pt, blue] (O) -- ($ 1.01*(A1) $);
	\draw [line width=1.25pt, blue, -{Latex[round, length=3.6mm, width=2.4mm]}]
		(O) -- (A1)
		node[pos=0.84, above left, shape=circle, fill=white, inner sep=-1pt, outer sep=6pt]
		{${\bm{a}}_{\hspace{-0.08ex}1}$};

	\draw [line width=0.4pt, blue] (O) -- ($ 1.01*(A2) $);
	\draw [line width=1.25pt, blue, -{Latex[round, length=3.6mm, width=2.4mm]}]
		(O) -- (A2)
		node[pos=0.88, below, shape=circle, fill=white, inner sep=-1pt, outer sep=6pt]
		{${\bm{a}}_2$};
	\draw [line width=0.4pt, blue] (O) -- (Vcomponent2);

	\draw [line width=0.4pt, blue] (O) -- ($ 1.01*(A3) $);
	\draw [line width=1.25pt, blue, -{Latex[round, length=3.6mm, width=2.4mm]}]
		(O) -- (A3)
		node[pos=0.71, above left, shape=circle, fill=white, inner sep=-1pt, outer sep=16pt]
		{${\bm{a}}_3$};

	% draw components of vector
	\draw [color=blue!50!black, line width=1.6pt, line cap=round, dash pattern=on 0pt off 1.6\pgflinewidth,
		-{Stealth[round, length=4mm, width=2.4mm]}]
		(O) -- (Vcomponent1)
		node[pos=0.67, above left, fill=white, shape=circle, inner sep=0pt, outer sep=4pt]
	{${v^{\hspace{-0.08ex}1} \hspace{-0.1ex} \bm{a}_{\hspace{-0.08ex}1}}$};

	\draw [color=blue!50!black, line width=1.6pt, line cap=round, dash pattern=on 0pt off 1.6\pgflinewidth,
		-{Stealth[round, length=4mm, width=2.4mm]}]
		(Vcomponent1) -- (VcomponentXY)
		node[pos=0.48, above, shape=circle, fill=white, inner sep=-2pt, outer sep=1pt]
	{${v^2 \hspace{-0.1ex} \bm{a}_2}$};

	\draw [color=blue!50!black, line width=1.6pt, line cap=round, dash pattern=on 0pt off 1.6\pgflinewidth,
		-{Stealth[round, length=4mm, width=2.4mm]}]
		(VcomponentXY) -- (V)
		node[pos=0.77, above right, shape=circle, fill=white, inner sep=-1pt, outer sep=7pt]
	{${v^3 \hspace{-0.1ex} \bm{a}_3}$};

	% draw co-components of vector
	\draw [color=red!50!black, line width=1.6pt, line cap=round, dash pattern=on 0pt off 1.6\pgflinewidth,
		-{Stealth[round, length=4mm, width=2.4mm]}]
		(O) -- (Vcoco1)
		node[pos=1.02, above left, fill=white, shape=circle, inner sep=-1pt, outer sep=5pt]
	{${v_{\raisemath{-0.2ex}{1}} \bm{a}^{\hspace{-0.16ex}1}}$};

	\draw [color=red!50!black, line width=1.6pt, line cap=round, dash pattern=on 0pt off 1.6\pgflinewidth,
		-{Stealth[round, length=4mm, width=2.4mm]}]
		(Vcoco1) -- (VcocoXY)
		node[pos=0.5, below right, shape=circle, fill=white, inner sep=-2pt, outer sep=7pt]
	{${v_{\raisemath{-0.2ex}{2}}  \hspace{0.1ex} \bm{a}^{\hspace{-0.1ex}2}}$};

	\draw [color=red!50!black, line width=1.6pt, line cap=round, dash pattern=on 0pt off 1.6\pgflinewidth,
		-{Stealth[round, length=4mm, width=2.4mm]}]
		(VcocoXY) -- (V)
		node[pos=0.37, above left, shape=circle, fill=white, inner sep=-1pt, outer sep=9pt]
	{${v_{\raisemath{-0.2ex}{3}} \hspace{0.1ex} \bm{a}^{\hspace{-0.1ex}3}}$};

	% draw vector
	\draw [line width=1.6pt, black, -{Stealth[round, length=5mm, width=2.8mm]}]
		(O) -- (V)
		node[pos=0.69, above, shape=circle, fill=white, inner sep=0pt, outer sep=3.33pt]
			{\scalebox{1.2}[1.2]{${\bm{v}}$}};

	% calculate a_i • a^j
	\tddotproductcartesian%
		{\firstcartesianx}{\firstcartesiany}{\firstcartesianz}%
		{\inverseOfSqrtGramian*\secondthirdcrossx}%
			{\inverseOfSqrtGramian*\secondthirdcrossy}%
				{\inverseOfSqrtGramian*\secondthirdcrossz}%
		{\FirstDotCofirst}
	\tddotproductcartesian%
		{\secondcartesianx}{\secondcartesiany}{\secondcartesianz}%
		{\inverseOfSqrtGramian*\thirdfirstcrossx}%
			{\inverseOfSqrtGramian*\thirdfirstcrossy}%
				{\inverseOfSqrtGramian*\thirdfirstcrossz}%
		{\SecondDotCosecond}
	\tddotproductcartesian%
		{\thirdcartesianx}{\thirdcartesiany}{\thirdcartesianz}%
		{\inverseOfSqrtGramian*\firstsecondcrossx}%
			{\inverseOfSqrtGramian*\firstsecondcrossy}%
				{\inverseOfSqrtGramian*\firstsecondcrossz}%
		{\ThirdDotCothird}
	%
	\tddotproductcartesian%
		{\secondcartesianx}{\secondcartesiany}{\secondcartesianz}%
		{\inverseOfSqrtGramian*\secondthirdcrossx}%
			{\inverseOfSqrtGramian*\secondthirdcrossy}%
				{\inverseOfSqrtGramian*\secondthirdcrossz}%
		{\SecondDotCofirst}
	\tddotproductcartesian%
		{\firstcartesianx}{\firstcartesiany}{\firstcartesianz}%
		{\inverseOfSqrtGramian*\thirdfirstcrossx}%
			{\inverseOfSqrtGramian*\thirdfirstcrossy}%
				{\inverseOfSqrtGramian*\thirdfirstcrossz}%
		{\FirstDotCosecond}
	\tddotproductcartesian%
		{\thirdcartesianx}{\thirdcartesiany}{\thirdcartesianz}%
		{\inverseOfSqrtGramian*\secondthirdcrossx}%
			{\inverseOfSqrtGramian*\secondthirdcrossy}%
				{\inverseOfSqrtGramian*\secondthirdcrossz}%
		{\ThirdDotCofirst}
	\tddotproductcartesian%
		{\firstcartesianx}{\firstcartesiany}{\firstcartesianz}%
		{\inverseOfSqrtGramian*\firstsecondcrossx}%
			{\inverseOfSqrtGramian*\firstsecondcrossy}%
				{\inverseOfSqrtGramian*\firstsecondcrossz}%
		{\FirstDotCothird}
	\tddotproductcartesian%
		{\secondcartesianx}{\secondcartesiany}{\secondcartesianz}%
		{\inverseOfSqrtGramian*\firstsecondcrossx}%
			{\inverseOfSqrtGramian*\firstsecondcrossy}%
				{\inverseOfSqrtGramian*\firstsecondcrossz}%
		{\SecondDotCothird}
	\tddotproductcartesian%
		{\thirdcartesianx}{\thirdcartesiany}{\thirdcartesianz}%
		{\inverseOfSqrtGramian*\thirdfirstcrossx}%
			{\inverseOfSqrtGramian*\thirdfirstcrossy}%
				{\inverseOfSqrtGramian*\thirdfirstcrossz}%
		{\ThirdDotCosecond}

	% show a_i • a^j as matrix
	\node[fill=white!50, inner sep=0pt, outer sep=2pt] at (0,0.45,-3.8)
		{$\scalebox{0.9}[0.9]{$
			\bm{a}_i \dotp \hspace{0.1ex} \bm{a}^{\hspace{0.1ex}j} \hspace{-0.1ex} =
			\hspace{-0.2ex}\scalebox{0.9}[0.9]{$\left[ \begin{array}{ccc}
				\bm{a}_1 \hspace{-0.1ex} \dotp \bm{a}^{\hspace{-0.1ex}1} &
					\bm{a}_1 \hspace{-0.1ex} \dotp \bm{a}^{\hspace{-0.06ex}2} &
						\bm{a}_1 \hspace{-0.1ex} \dotp \bm{a}^{\hspace{-0.06ex}3} \\
				\bm{a}_2 \hspace{-0.1ex} \dotp \bm{a}^{\hspace{-0.1ex}1} &
					\bm{a}_2 \hspace{-0.1ex} \dotp \bm{a}^{\hspace{-0.06ex}2} &
						\bm{a}_2 \hspace{-0.1ex} \dotp \bm{a}^{\hspace{-0.06ex}3} \\
				\bm{a}_3 \hspace{-0.1ex} \dotp \bm{a}^{\hspace{-0.1ex}1} &
					\bm{a}_3 \hspace{-0.1ex} \dotp \bm{a}^{\hspace{-0.06ex}2} &
						\bm{a}_3 \hspace{-0.1ex} \dotp \bm{a}^{\hspace{-0.06ex}3}
			\end{array} \right]$} \!=\!
			\scalebox{0.9}[0.9]{$\left[ \begin{array}{ccc}
				\pgfmathprintnumber[fixed, precision=3]{\FirstDotCofirst} &
					\pgfmathprintnumber[fixed, precision=3]{\FirstDotCosecond} &
						\pgfmathprintnumber[fixed, precision=3]{\FirstDotCothird} \\
				\pgfmathprintnumber[fixed, precision=3]{\SecondDotCofirst} &
					\pgfmathprintnumber[fixed, precision=3]{\SecondDotCosecond} &
						\pgfmathprintnumber[fixed, precision=3]{\SecondDotCothird} \\
				\pgfmathprintnumber[fixed, precision=3]{\ThirdDotCofirst} &
					\pgfmathprintnumber[fixed, precision=3]{\ThirdDotCosecond} &
						\pgfmathprintnumber[fixed, precision=3]{\ThirdDotCothird}
			\end{array} \right]$}
			\hspace{-0.1em} = %%\approx
			\hspace{0.1ex} \delta_i^{\hspace{0.1ex}j}
		$}$};

\end{tikzpicture}

\vspace{0.2em}\caption{\inquotes{Decomposition of vector}}\label{fig:DecompositionOfVector}

\end{center}
\end{figure}

% ~ ~ ~ ~ ~

От~векторов перейдём к~тензорам второй сложности. Имеем четыре комплекта диад:
${\bm{a}_i \hspace{0.1ex} \bm{a}_j}$,
\hbox{$\bm{a}^{\hspace{-0.05ex}i} \hspace{-0.1ex} \bm{a}^{\hspace{0.05ex}j}$\hspace{-0.25ex},}
\hbox{$\bm{a}_i \hspace{0.1ex} \bm{a}^{\hspace{0.05ex}j}$\hspace{-0.25ex},}
${\bm{a}^{\hspace{-0.05ex}i} \hspace{-0.1ex} \bm{a}_j}$.
Согласующиеся коэффициенты в~декомпозиции тензора называются его контра\-вариант\-ными, ко\-вариант\-ными и~смешан\-ными компонентами:
\vspace{0.1em}\begin{equation}\begin{array}{c}
{^2\!\bm{B}} \hspace{0.1ex} =
B^{\hspace{0.1ex}ij} \hspace{-0.1ex} \bm{a}_i \hspace{0.1ex} \bm{a}_j \hspace{-0.05ex} =
B_{ij} \hspace{0.1ex} \bm{a}^{\hspace{-0.05ex}i} \hspace{-0.1ex} \bm{a}^{\hspace{0.05ex}j} \hspace{-0.15ex} =
B_{\hspace{-0.2ex} \cdot j}^{\hspace{0.1ex}i} \hspace{0.1ex} \bm{a}_i \hspace{0.1ex} \bm{a}^{\hspace{0.05ex}j} \hspace{-0.15ex} =
B_{\hspace{-0.1ex}i}^{\hspace{-0.05ex} \cdot j} \hspace{-0.2ex} \bm{a}^{\hspace{-0.05ex}i} \hspace{-0.1ex} \bm{a}_j \hspace{0.1ex}, \\[0.4em]
%
B^{\hspace{0.1ex}ij} \hspace{-0.25ex} = \bm{a}^{\hspace{-0.05ex}i} \dotp {^2\!\bm{B}} \dotp \hspace{0.1ex} \bm{a}^{\hspace{0.05ex}j} \hspace{-0.2ex}, \:\,
B_{ij} = \bm{a}_i \dotp {^2\!\bm{B}} \dotp \hspace{0.1ex} \bm{a}_j \hspace{0.1ex}, \\[0.25em]
%
B_{\hspace{-0.2ex} \cdot j}^{\hspace{0.1ex}i} = \bm{a}^{\hspace{-0.05ex}i} \dotp {^2\!\bm{B}} \dotp \hspace{0.1ex} \bm{a}_j \hspace{0.1ex}, \:\,
B_{\hspace{-0.1ex}i}^{\hspace{-0.05ex} \cdot j} \hspace{-0.2ex} = \bm{a}_i \dotp {^2\!\bm{B}} \dotp \hspace{0.1ex} \bm{a}^{\hspace{0.1ex}j} \hspace{-0.1ex}.
\end{array}\end{equation}

\vspace{-0.1em}\noindent Для~двух видов смешанных компонент точка в~индексе это просто свободное место: у~${B_{\hspace{-0.2ex} \cdot j}^{\hspace{0.1ex}i}}$ верхний индекс~\inquotesx{$i\hspace{0.25ex}$}[---] первый, а~ниж\-ний~\inquotesx{$\hspace{-0.1ex}j\hspace{0.25ex}$}[---] второй.

Компоненты единичного~(\inquotes{метрического}) тензора %%~$\bm{E}$
\vspace{0.1em}\begin{equation}\begin{array}{c}
\bm{E} = \bm{a}^{k} \hspace{-0.1ex} \bm{a}_{k} \hspace{-0.1ex} = \bm{a}_{k} \hspace{0.1ex} \bm{a}^{k} \hspace{-0.2ex} = \textsl{g}_{jk} \hspace{0.1ex} \bm{a}^{\hspace{0.1ex}j} \hspace{-0.1ex} \bm{a}^{k} \hspace{-0.2ex} = \textsl{g}^{\hspace{0.25ex}jk} \hspace{-0.1ex} \bm{a}_j \bm{a}_k \hspace{-0.32ex}: \\[0.2em]
%
\bm{a}_i \dotp \bm{E} \dotp \hspace{0.1ex} \bm{a}^{\hspace{0.1ex}j} \hspace{-0.2ex} = \bm{a}_i \hspace{0.1ex} \dotp \hspace{0.1ex} \bm{a}^{\hspace{0.1ex}j} \hspace{-0.2ex} = \delta_i^{\hspace{0.1ex}j} , \hspace{0.32em}
\bm{a}^{\hspace{-0.05ex}i} \dotp \bm{E} \dotp \hspace{0.1ex} \bm{a}_j = \bm{a}^{\hspace{-0.05ex}i} \hspace{-0.2ex} \dotp \hspace{0.1ex} \bm{a}_j \hspace{-0.2ex} = \delta_{\hspace{-0.1ex}j}^{\hspace{0.1ex}i} \hspace{0.2ex},
\\[0.2em]
%
\bm{a}_i \dotp \bm{E} \dotp \hspace{0.1ex} \bm{a}_j = \bm{a}_i \dotp \hspace{0.1ex} \bm{a}_j \equiv \hspace{0.16ex} \textsl{g}_{ij} \hspace{0.1ex} , \hspace{0.32em}
\bm{a}^{\hspace{-0.05ex}i} \dotp \bm{E} \dotp \hspace{0.1ex} \bm{a}^{\hspace{0.1ex}j} \hspace{-0.15ex} = \bm{a}^{\hspace{-0.05ex}i} \dotp \hspace{0.1ex} \bm{a}^{\hspace{0.1ex}j} \hspace{-0.15ex} \hspace{0.1ex} \equiv \hspace{0.16ex} \textsl{g}^{\hspace{0.2ex}ij} \hspace{0.1ex} ; \\[0.2em]
%
\scalebox{0.96}[0.97]{$\bm{E} \dotp \hspace{-0.1ex} \bm{E} = \textsl{g}_{ij} \hspace{0.1ex} \bm{a}^{\hspace{-0.05ex}i} \hspace{-0.1ex} \bm{a}^{\hspace{0.05ex}j} \hspace{-0.1ex} \dotp \hspace{0.1ex} \textsl{g}^{\hspace{0.2ex}nk} \bm{a}_n \bm{a}_k \hspace{-0.1ex} = \textsl{g}_{ij} \hspace{0.1ex} \textsl{g}^{\hspace{0.2ex}jk} \bm{a}^{\hspace{-0.05ex}i} \bm{a}_k \hspace{-0.1ex} = \bm{E}$}
\:\Rightarrow\: \textsl{g}_{ij} \hspace{0.1ex} \textsl{g}^{\hspace{0.2ex}jk} \hspace{-0.25ex} = \delta_i^{\hspace{0.05ex}k} \hspace{-0.1ex}.
\end{array}\end{equation}

\vspace{0.1em}\noindent Вдобавок к~\eqref{fundamentalpropertyofcobasis} и~\eqref{basisvectorstocobasisvectors} открылся ещё~один способ найти векторы кобазиса~--- через матрицу~\hbox{$\textsl{g}^{\hspace{.2ex}ij}$\hspace{-0.3ex}}, обратную матрице Грама~${\textsl{g}_{ij}}$. И~наоборот:

\nopagebreak\vspace{-0.12em}
\begin{equation}\begin{array}{c}
\bm{a}^{\hspace{-0.05ex}i} \hspace{-0.1ex}
= \bm{E} \dotp \bm{a}^{\hspace{-0.05ex}i} \hspace{-0.15ex}
= \textsl{g}^{\hspace{0.25ex}jk} \bm{a}_j \bm{a}_k \hspace{-0.1ex} \dotp \bm{a}^{\hspace{-0.05ex}i} \hspace{-0.15ex}
= \textsl{g}^{\hspace{0.25ex}jk} \bm{a}_j \hspace{.16ex} \delta_{k}^{\hspace{.1ex}i}
= \textsl{g}^{\hspace{0.25ex}ji} \bm{a}_j \hspace{.1ex} ,
\\[.25em]
%
\bm{a}_i
= \bm{E} \dotp \bm{a}_i \hspace{-0.1ex}
= \textsl{g}_{jk} \hspace{.1ex} \bm{a}^{\hspace{.1ex}j} \hspace{-0.1ex} \bm{a}^{k} \hspace{-0.2ex} \dotp \bm{a}_i \hspace{-0.1ex}
= \textsl{g}_{jk} \hspace{.1ex} \bm{a}^{\hspace{.1ex}j} \hspace{.1ex} \delta_{i}^{k}
= \textsl{g}_{ji} \hspace{.16ex} \bm{a}^{\hspace{.1ex}j} \hspace{-0.2ex} .
\end{array}\end{equation}

\begin{tcolorbox}
\small\setlength{\abovedisplayskip}{2pt}\setlength{\belowdisplayskip}{2pt}

\emph{Example.} Using reversed Gram matrix, get cobasis for basis~$\bm{a}_i$ when
\[ \begin{array}{l}
\bm{a}_1 = \bm{e}_1 \hspace{-0.2ex} + \bm{e}_2 \hspace{.1ex},\\
\bm{a}_2 = \bm{e}_1 \hspace{-0.2ex} + \bm{e}_3 \hspace{.1ex},\\
\bm{a}_3 = \bm{e}_2 \hspace{-0.2ex} + \bm{e}_3 \hspace{.1ex}.
\end{array} \]

\[\begin{array}{c}
\textsl{g}_{ij} \hspace{-0.32ex} = \bm{a}_i \dotp \bm{a}_j \hspace{-0.12ex} = \hspace{-0.16ex}
\scalebox{0.92}[0.92]{$\left[\hspace{-0.16ex}\begin{array}{c@{\hspace{.64em}}c@{\hspace{.64em}}c}
2 & 1 & 1 \\
1 & 2 & 1 \\
1 & 1 & 2
\end{array}\hspace{-0.12ex}\right]$} \hspace{-0.2ex}, \:\:
\operatorname{det} \hspace{.12ex} \textsl{g}_{ij} \hspace{-0.25ex} = 4 \hspace{.16ex}, \\
%
\operatorname{adj} \hspace{.12ex} \textsl{g}_{ij} \hspace{-0.25ex} = \hspace{-0.16ex}
\scalebox{0.92}[0.92]{$\left[\hspace{-0.4ex}\begin{array}{r@{\hspace{.5em}}r@{\hspace{.5em}}r}
3 & -1 & -1 \\
-1 & 3 & -1 \\
-1 & -1 & 3
\end{array}\hspace{-0.12ex}\right]^{\hspace{-0.4ex}\scalebox{1.02}{$\T$}}$} \hspace{-0.8ex}, \\
%
\textsl{g}^{\hspace{.32ex}ij} \hspace{-0.25ex} = \textsl{g}_{ij}^{\hspace{.4ex}\expminusone} \hspace{-0.12ex} = \displaystyle \frac{\operatorname{adj} \hspace{.12ex} \textsl{g}_{ij}}{\operatorname{det} \hspace{.12ex} \textsl{g}_{ij}} =
\displaystyle \frac{1}{4} \hspace{.12ex}
\scalebox{0.92}[0.92]{$\left[\hspace{-0.4ex}\begin{array}{r@{\hspace{.5em}}r@{\hspace{.5em}}r}
3 & -1 & -1 \\
-1 & 3 & -1 \\
-1 & -1 & 3
\end{array}\hspace{-0.12ex}\right]^{\mathstrut}$} \hspace{-0.25ex}.
\end{array}\]

\vspace{-0.5em}Using ${\bm{a}^i = \textsl{g}^{\hspace{.32ex}ij} \bm{a}_j}$
\[\begin{array}{l}
\bm{a}^1 \hspace{-0.2ex}=\hspace{.1ex} \textsl{g}^{\hspace{.2ex}11} \bm{a}_1 \hspace{-0.2ex} + \textsl{g}^{\hspace{.2ex}12} \bm{a}_2 \hspace{-0.2ex} + \textsl{g}^{\hspace{.2ex}13} \bm{a}_3 \hspace{-0.2ex} = \smalldisplaystyleonehalf \bm{e}_1 \hspace{-0.2ex} + \smalldisplaystyleonehalf \bm{e}_2 \hspace{-0.2ex} - \smalldisplaystyleonehalf \bm{e}_3 \hspace{.1ex},\\[.5em]
%
\bm{a}^2 \hspace{-0.2ex}=\hspace{.1ex} \textsl{g}^{\hspace{.2ex}21} \bm{a}_1 \hspace{-0.2ex} + \textsl{g}^{\hspace{.2ex}22} \bm{a}_2 \hspace{-0.2ex} + \textsl{g}^{\hspace{.2ex}23} \bm{a}_3 \hspace{-0.2ex} = \smalldisplaystyleonehalf \bm{e}_1 \hspace{-0.2ex} - \smalldisplaystyleonehalf \bm{e}_2 \hspace{-0.2ex} + \smalldisplaystyleonehalf \bm{e}_3 \hspace{.1ex},\\[.5em]
%
\bm{a}^3 \hspace{-0.2ex}=\hspace{.1ex} \textsl{g}^{\hspace{.2ex}31} \bm{a}_1 \hspace{-0.2ex} + \textsl{g}^{\hspace{.2ex}32} \bm{a}_2 \hspace{-0.2ex} + \textsl{g}^{\hspace{.2ex}33} \bm{a}_3 \hspace{-0.2ex} = - \smalldisplaystyleonehalf \bm{e}_1 \hspace{-0.2ex} + \smalldisplaystyleonehalf \bm{e}_2 \hspace{-0.2ex} + \smalldisplaystyleonehalf \bm{e}_3 \hspace{.16ex}.
\end{array}\]

\par\end{tcolorbox}

...


Единичный тензор~(unit tensor, identity tensor,  metric tensor)

$\bm{E} \dotp \hspace{0.1ex} \bm{\xi} = \bm{\xi} \hspace{0.1ex} \dotp \bm{E} = \bm{\xi} \quad \forall \bm{\xi}$

$\bm{E} \dotdotp \bm{a} \bm{b} = \bm{a} \bm{b} \dotdotp \bm{E} = \bm{a} \dotp \bm{E} \dotp \bm{b} = \bm{a} \dotp \bm{b}$

$\bm{E} \dotdotp \bm{A} = \bm{A} \dotdotp \bm{E} = \operatorname{tr} \bm{A}$


$\bm{E} \dotdotp \hspace{-0.1ex} \bm{A} = \bm{A} \dotdotp \hspace{-0.1ex} \bm{E} = \operatorname{tr} \bm{A} = \operatorname{not anymore} A_{jj}$

Thus for, say, trace of some tensor ${\bm{A} = A_{ij} \bm{r}^i \bm{r}^j}$: $\bm{A} \dotdotp \bm{E} = \operatorname{tr} \bm{A}$, you have

$\bm{A} \dotdotp \bm{E} = A_{ij} \bm{r}^i \bm{r}^j \dotdotp \bm{r}_k \bm{r}^k = A_{ij} \bm{r}^i \dotp \bm{r}^j = A_{ij} \textsl{g}^{\hspace{.32ex}ij}$


...

Тензор поворота~(rotation tensor)

$\bm{P} \hspace{-0.2ex} = \hspace{-0.1ex} \bm{a}_i \hspace{.1ex} \mathcircabove{\bm{a}}^i \hspace{-0.2ex} = \hspace{-0.1ex} \bm{a}^{\hspace{-0.16ex}i} \mathcircabove{\bm{a}}_i \hspace{-0.2ex} = \hspace{-0.1ex} \bm{P}^{\expminusT}$

$\bm{P}^{\expminusone} \hspace{-0.4ex} = \hspace{-0.1ex} \mathcircabove{\bm{a}}_i \hspace{.1ex} \bm{a}^{\hspace{-0.16ex}i} \hspace{-0.2ex} = \hspace{-0.1ex} \mathcircabove{\bm{a}}^i \bm{a}_i \hspace{-0.2ex} = \hspace{-0.1ex} \bm{P}^{\T}$

$\bm{P}^{\T} \hspace{-0.4ex} = \hspace{-0.1ex} \mathcircabove{\bm{a}}^i \bm{a}_i \hspace{-0.2ex} = \hspace{-0.1ex} \mathcircabove{\bm{a}}_i \hspace{.1ex} \bm{a}^{\hspace{-0.16ex}i} \hspace{-0.2ex} = \hspace{-0.1ex} \bm{P}^{\expminusone}$

...



... Характеристическое уравнение~\eqref{chardetequation} быстро приводит к~тождеству Г\kern-0.08em\'{а}мильтона\hbox{--}К\kern-0.04em\'{э}ли~(Cayley\hbox{--}Hamilton)
\nopagebreak\vspace{.1em}\begin{equation}\label{cayley-hamilton:eq}
-\bm{B}^{3} \hspace{-0.25ex} + \mathrm{I}\hspace{0.16ex} \bm{B}^{2} \hspace{-0.25ex} - \mathrm{II}\hspace{0.16ex} \bm{B} + \mathrm{III}\hspace{0.16ex} \bm{E} = {^2\bm{0}} \hspace{0.1ex}.
\end{equation}

\end{otherlanguage}

\en{\section{Tensor functions}}

\ru{\section{Тензорные функции}}

\label{para:tensorfunctions}

\en{In~the~popular conception of~function~${y \!=\! f(x)}$ as of mapping (morphism) ${\smash{f \colon x \mapsto y}}$ an~input~(argument)~$x$ and an~output~(result)~$y$ may be tensors of any complexity.}

\ru{\noindent В~популярном представлении о~функции~${y \!=\! f(x)}$ как отображении (морфизме) ${\smash{f \colon x \mapsto y}}$ прообраз~(аргумент)~$x$ и~образ~(результат)~$y$ могут быть тензорами любой сложности.}

\begin{otherlanguage}{russian}

Рассмотрим хотя~бы скалярную функцию двухвалентного тензора ${\varphi \!=\! \varphi(\bm{B})}$. В~каждом базисе~${\bm{a}_i}$ имеем функцию девяти числовых аргументов, компонент~${\varphi(B_{ij})}$; при~переходе к~новому базису компоненты ${B_{ij}}$ могут изменяться лишь~так, чтобы сохранялся результат~$\varphi$. Дифференцирование~$\varphi$ выглядит так:

\nopagebreak\vspace{-0.1em}\begin{equation}
d \varphi = \displaystyle \frac{\partial \varphi}{\partial B_{ij}} \hspace{0.2ex} d B_{ij} = \displaystyle \frac{\partial \varphi}{\partial \bm{B}} \dotdotp d \bm{B}^{\T} \hspace{-0.25ex} , \;\;
\displaystyle \frac{\partial \varphi}{\partial \bm{B}} \hspace{0.1ex} \equiv \displaystyle \frac{\partial \varphi}{\partial B_{ij}} \hspace{0.25ex} \bm{a}_i \bm{a}_j \hspace{0.1ex}.
\end{equation}

Тензор~...

...

Но согласно опять\hbox{-}таки~\eqref{cayley-hamilton:eq}
${\hspace{-0.1ex} -\bm{B}^{2} \hspace{-0.2ex} + \mathrm{I}\hspace{0.16ex} \bm{B} \hspace{-0.1ex} - \mathrm{II}\hspace{0.16ex} \bm{E} + \mathrm{III}\hspace{0.16ex} \bm{B}^{\expminusone} \hspace{-0.25ex} = {^2\bm{0}}}$, поэтому

...



Скалярная функция~${\varphi(\bm{B})}$ называется изотропной, если она не~чувствительна к~повороту аргумента:
\nopagebreak\vspace{.1em}\begin{equation*}
\varphi(\bm{B}) \hspace{-0.12ex} = \varphi ( \rotationtensor \narrowdotp \smash{\mathcircabove{\bm{B}}} \narrowdotp \hspace{.15ex} \rotationtensor^{\hspace{-0.1ex}\T} ) \hspace{-0.2ex} = \varphi(\smash{\mathcircabove{\bm{B}}}) \;\;\,
\forall \rotationtensor \hspace{-0.2ex} = \hspace{-0.1ex} \bm{a}_i \hspace{.1ex} \mathcircabove{\bm{a}}^i \hspace{-0.25ex} = \hspace{-0.1ex} \bm{a}^{\hspace{-0.2ex}i} \mathcircabove{\bm{a}}_i \hspace{-0.16ex} = \hspace{-0.1ex} \rotationtensor^{\hspace{-0.1ex}\expminusT}
\end{equation*}
\par\vspace{-0.25em}\noindent для~любого ортогонального тензора~$\rotationtensor$ (тензора поворота, \pararef{para:rotationtensor}).

Симметричный тензор~${\bm{B}^{\mathsf{\hspace{0.1ex}S}}}$ полностью определяется тройкой инвариантов и~угловой ориентацией собственных осей (они~же взаимно ортогональны, \pararef{para:eigenvectorseigenvalues}). Ясно, что изотропная функция~${\varphi(\bm{B}^{\mathsf{\hspace{0.1ex}S}})}$ симметричного аргумента является функцией лишь инвариантов ${\mathrm{I}\hspace{0.16ex}({\bm{B}^{\mathsf{\hspace{0.1ex}S}}})}$, ${\mathrm{II}\hspace{0.16ex}({\bm{B}^{\mathsf{\hspace{0.1ex}S}}})}$, ${\mathrm{III}\hspace{0.16ex}({\bm{B}^{\mathsf{\hspace{0.1ex}S}}})}$; она дифференцируется согласно~\eqref{fonvccbnmxghjsxmnxjsdjhga}, где транспонирование излишне.

\end{otherlanguage}

\en{\section{Tensor fields. Differentiation}}

\ru{\section{Тензорные поля. Дифференцирование}}

\label{para:differentiationoftensorfields}

\begin{otherlanguage}{russian}

\begin{changemargin}{\parindent}{\parindent}
\vspace{-0.15em}
\small \flushright \textit{\inquotesx{п\'{о}ле}[---]} это тензор, меняющийся от~точки к~точке (переменный по~координатам)
\par\vspace{.25em}
\end{changemargin}

\noindent Путь в~каждой точке некоторой области трёхмерного пространства
% известно значение величины (value of value)
известна величина~$q$. Тогда имеем поле~${q \!=\! q(\bm{r})}$, где~$\bm{r}$~--- радиус\hbox{-}вектор точки. Например, поле температуры в~среде, поле давления в~идеальной жидкости. Величина~$q$ может~быть тензором любой сложности. Пример векторного поля~--- скорости частиц жидкости.

\end{otherlanguage}

% ~ ~ ~ ~ ~
\begin{wrapfigure}{R}{0.55\textwidth}
\makebox[0.55\textwidth][c]{%
\hspace{2em}
\begin{minipage}[t]{.55\textwidth}

\begin{tikzpicture}[scale=0.5]

%%\clip (-6, -6) rectangle + (12, 12) ; % crop it
\clip (0, 0) circle (6cm) ; % crop it

\tikzset{%
	tangent/.style={
		decoration={
			markings,% switch on markings
			mark=
			at position #1
			with
			{
				\def\numberoftangent{\pgfkeysvalueof{/pgf/decoration/mark info/sequence number}}
				\coordinate (tangent point-\numberoftangent) at (0, 0);
				\coordinate (tangent unit vector-\numberoftangent) at (1, 0);
				\coordinate (tangent orthogonal unit vector-\numberoftangent) at (0, 1);
			}
		},
		postaction=decorate
	},
	use tangent/.style={
		shift=(tangent point-#1),
		x=(tangent unit vector-#1),
		y=(tangent orthogonal unit vector-#1)
	},
	use tangent/.default=1
}

\tikzset{%
	show curve controls/.style={
		postaction={
			decoration={
				show path construction,
				curveto code={
					\fill [black, opacity=.5]
						(\tikzinputsegmentfirst) circle (.4ex)
						(\tikzinputsegmentlast) circle (.4ex) ;
					\draw [black, opacity=.5, line cap=round, dash pattern=on 0pt off 1.6\pgflinewidth]
						(\tikzinputsegmentfirst) -- (\tikzinputsegmentsupporta)
						(\tikzinputsegmentlast) -- (\tikzinputsegmentsupportb) ;
					\fill [magenta, opacity=.5, line cap=round, dash pattern=on 0pt off 1.6\pgflinewidth]
						(\tikzinputsegmentsupporta) circle [radius=.4ex]
						(\tikzinputsegmentsupportb) circle [radius=.4ex] ;
				}
			},
			decorate
}	}	}

%%\foreach \cycle in {0, 1, ..., 15}
%%	\draw [color=green]
%%		($ (0, 0) - (\cycle, 1.2*\cycle) $)
%%		parabola ($ (4, 3) + 0.5*(1.6*\cycle, \cycle) $);

\foreach \c in {-10, -9.5, ..., 10}
{
	\def\offset{0.2*\c, -0.1*\c}
	\pgfmathsetmacro\bottomoffsetx{-.24 * ( \c )}
	\pgfmathsetmacro\bottomoffsety{-.1 * abs( \c ) + .1 * ( \c )}
	\pgfmathsetmacro\bottomangle{12 - 1.2 * abs( \c )}
	\pgfmathsetmacro\bottomnudge{2}
	\pgfmathsetmacro\midoffsetx{-.1 * abs( \c )}
	\pgfmathsetmacro\midoffsety{.1 * abs( \c )}
	\pgfmathsetmacro\midangle{63 + 1.2 * abs( \c )}
	\pgfmathsetmacro\midnudge{4 + ( .1 * abs( \c ) )}
	\pgfmathsetmacro\topoffsetx{.32 * ( \c ) + 0 * abs( \c )}
	\pgfmathsetmacro\topoffsety{-.16 * ( \c ) + 0 * abs( \c )}
	\pgfmathsetmacro\topangle{166 + 1.6 * ( \c ) + 1.2 * abs( \c )}
	\pgfmathsetmacro\topnudge{5 + ( .25 * abs( \c ) )}
	\draw	[ line width=.4pt
		, color=blue!50
		%%, show curve controls
		]
		($ (-6, -4.5) + 5*(\offset) + (\bottomoffsetx, \bottomoffsety) $)
		.. controls ++(\bottomangle: \bottomnudge) and ++(\midangle: -\midnudge) ..
		($ 4*(\offset) + (\midoffsetx, \midoffsety) $)
		.. controls ++(\midangle: \midnudge) and ++(\topangle: \topnudge) ..
		($ (8, 6) + 2.5*(\offset) + (\topoffsetx, \topoffsety) $) ;
}

\foreach \c in {-10, -9.5, ..., 10}
{
	\def\offset{0.2*\c, 0.1*\c}
	\pgfmathsetmacro\leftoffsetx{- .1 * abs ( \c )}
	\pgfmathsetmacro\leftoffsety{.4 * ( \c )}
	\pgfmathsetmacro\leftangle{33 + .2 * abs( \c )}
	\pgfmathsetmacro\leftnudge{1.6 + .5 * abs( \c )}
	\pgfmathsetmacro\midoffsetx{-.2 * abs( \c )}
	\pgfmathsetmacro\midoffsety{.2 * abs( \c )}
	\pgfmathsetmacro\midangle{111 + 1.2 * abs( \c )}
	\pgfmathsetmacro\midnudge{5}
	\pgfmathsetmacro\rightoffsetx{.25 * abs( \c )}
	\pgfmathsetmacro\rightoffsety{.16 * ( \c )}
	\pgfmathsetmacro\rightangle{177 + 2 * ( \c )}
	\pgfmathsetmacro\rightnudge{abs( 2 - ( .5 * ( \c ) ) )}
	\draw	[ line width=.4pt
		, color=red!50
		%%, show curve controls
		]
		($ (-12, 5) + 2.5*(\offset) + (\leftoffsetx, \leftoffsety) $)
		.. controls ++(\leftangle: \leftnudge) and ++(\midangle: \midnudge) ..
		($ 5*(\offset) + (\midoffsetx, \midoffsety) $)
		.. controls ++(\midangle: -\midnudge) and ++(\rightangle: \rightnudge) ..
		($ (8, -5) + 4*(\offset) + (\rightoffsetx, \rightoffsety) $);
}

\foreach \c in {-10, -9.5, ..., 10}
{
	\def\offset{0*\c, 0.25*\c}
	\pgfmathsetmacro\midnudge{6 + .16 * ( \c )}
	\draw	[ line width=.4pt
		, color=green!50
		%%, show curve controls
		]
		($ (12, 10) + 4*(\offset) $)
		.. controls ++(88: -4) and ++(11: \midnudge) ..
		($ 4*(\offset) $)
		.. controls ++(11: -\midnudge) and ++(99: -4) ..
		($ (-12, 4) + 4*(\offset) $) ;
}

\draw	[ line width=.8pt
	, color=blue!50!black
	%%, show curve controls
	]
	(-6, -4.5)
	.. controls ++(12: 2) and ++(63: -4) ..
	(0, 0);

\draw	[ line width=.8pt
	, color=blue!50!black
	%%, show curve controls
	, tangent=0
	, tangent=0.4
	]
	(0, 0)
	.. controls ++(63: 4) and ++(166: 5) ..
	(8, 6) ;

\path [use tangent=1]
	(0, 0) -- (.4*4, 0)
	node [color=blue, pos=0.86, above left, shape=circle, fill=white, outer sep=4pt, inner sep=1pt]
		{$\bm{r}_3$} ;

\draw [line width=1.25pt, color=blue, use tangent=1, -{Latex[round, length=3.6mm, width=2.4mm]}]
	(0, 0) -- (.4*4, 0) ;

\path [use tangent=2]
	(0, 0) -- (0, -1)
	node [color=blue!50!black, pos=0.48, above, shape=circle, fill=white, outer sep=0pt, inner sep=0.25pt]
		{$q^{\hspace{.1ex}3}$} ;

%%\fill [fill=blue, use tangent=1] (0, 0) circle (1mm);

\draw	[ line width=.8pt
	, color=red!50!black
	%%, show curve controls
	]
	(-12, 5)
	.. controls ++(33: 1.6) and ++(111: 5) ..
	(0, 0);

\draw	[ line width=.8pt
	, color=red!50!black
	%%, show curve controls
	, tangent=0
	, tangent=0.5
	]
	(0, 0)
	.. controls ++(111: -5) and ++(177: 2) ..
	(8, -5) ;

\path [use tangent=1]
	(0, 0) -- (.4*5, 0)
	node [color=red, pos=0.86, below left, shape=circle, fill=white, outer sep=4pt, inner sep=1pt]
		{$\bm{r}_1$} ;

\draw [line width=1.25pt, color=red, use tangent=1, -{Latex[round, length=3.6mm, width=2.4mm]}]
	(0, 0) -- (.4*5, 0);

\path [use tangent=2]
	(0, 0) -- (0, 1)
	node [color=red!50!black, pos=0.16, above, shape=circle, fill=white, outer sep=0pt, inner sep=0.25pt]
		{$q^{1}$} ;

%%\fill [fill=red, use tangent=1] (0, 0) circle (1mm);

\draw	[ line width=.8pt
	, color=green!50!black
	%%, show curve controls
	]
	(12, 10)
	.. controls ++(88: -4) and ++(11: 6) ..
	(0, 0) ;

\draw	[ line width=.8pt
	, color=green!50!black
	%%, show curve controls
	, tangent=0
	, tangent=0.36
	]
	(0, 0)
	.. controls ++(11: -6) and ++(99: -4) ..
	(-12, 4) ;

\path [use tangent=1]
	(0, 0) -- (.4*6, 0)
	node [color=green, pos=0.92, below right, shape=circle, fill=white, outer sep=5pt, inner sep=1pt]
		{$\bm{r}_2$} ;

\draw [line width=1.25pt, color=green, use tangent=1, -{Latex[round, length=3.6mm, width=2.4mm]}]
	(0, 0) -- (.4*6, 0);

\path [use tangent=2]
	(0, 0) -- (0, -1)
	node [color=green!50!black, pos=0.12, above, shape=circle, fill=white, outer sep=0pt, inner sep=0.25pt]
		{$q^{\hspace{.1ex}2}$} ;

%%\fill [fill=green, use tangent=1] (0, 0) circle (1mm);

\coordinate (theOrigin) at (5, -2) ;
\path (0, 0) circle (1mm) node [shape=circle, inner sep=.5mm, outer sep=0] (theCircleOfO) {} ;

\draw [line width=1.5pt, black, -{Stealth[round,length=4mm,width=2.8mm]}] (theOrigin) -- (theCircleOfO)
		node [pos=0.64, above right, shape=circle, fill=white, outer sep=2pt, inner sep=1.2pt]
			{$\bm{r}$} ;

\draw [line width=1.2pt, color=black, fill=white] (0, 0) circle (1ex);

\draw [line width=1.2pt, color=black, fill=white] (theOrigin) circle (1ex);

\end{tikzpicture}

\vspace{0.1em}\caption{}\label{fig:curvilinearcoordinates}
\end{minipage}}
\end{wrapfigure}

% ~ ~ ~ ~ ~

\begin{otherlanguage}{russian}

Не~только при~решении прикладных задач, но нередко и для~\inquotes{чистой тео\-рии} вместо аргумента~$\bm{r}$ ис\-поль\-зу\-ет\-ся какая-либо трой\-ка криво\-линей\-ных координат~${q^{\hspace{.1ex}i}\hspace{-0.2ex}}$. При~этом ${\bm{r} \!=\! \bm{r}(q^{\hspace{.1ex}i}\hspace{.1ex})}$. Если непрерывно менять лишь одну координату из~трёх, получается координатная линия. Каждая точка трёхмерного пространства лежит на~пересечении трёх координатных линий (\figref{fig:curvilinearcoordinates}).

Commonly used curvilinear coordinate systems include: rectangular~(\inquotes{cartesian}), spherical, and cylindrical coordinate systems. These coordinates may be derived from a~set of cartesian coordinates by using a~transformation that is locally invertible (a~one-to-one map) at~each point. This means that one can convert a~point given in a~cartesian coordinate system to its curvilinear coordinates and~back.

...

\nopagebreak\vspace{-0.1em}\begin{equation*}
\partial_i \equiv \frac{\raisebox{-0.2em}{$\partial$}}{\raisebox{-0.1em}{$\partial q^i$}}
\end{equation*}

...

\en{Linearity}\ru{Линейность}

\nopagebreak\vspace{-0.4em}\begin{equation}\label{linearityordifferentiation}
\partial_i \hspace{-0.2ex} \left( \lambda \hspace{.1ex} p + \hspace{-0.1ex} \mu \hspace{.1ex} q \right)
= \lambda \hspace{-0.1ex} \left( \partial_i \hspace{.1ex} p \right) + \hspace{.1ex}
\mu \hspace{-0.1ex} \left( \partial_i \hspace{.1ex} q \right)
\end{equation}

\inquotes{Product rule}

\nopagebreak\vspace{-0.4em}\begin{equation}\label{productrulefordifferentiation}
\partial_i \hspace{-0.2ex} \left( \hspace{.1ex} p \circ q \right)
= \left( \partial_i \hspace{.1ex} p \right) \hspace{-0.12ex} \circ q \hspace{.16ex} +
\hspace{.16ex} p \circ \hspace{-0.12ex} \left( \partial_i \hspace{.1ex} q \right)
\end{equation}

...

Bivalent unit tensor~(identity tensor,  metric tensor), the one which is neutral~\eqref{identifyofidentitytensor} to dot product operation,
\en{can be represented as}\ru{может быть представлен как}

\nopagebreak\vspace{-0.1em}\begin{equation}
\bm{E} = \bm{r}^i \bm{r}_i = \tikzmark{beginOriginOfNabla} \bm{r}^i \partial_i \tikzmark{endOriginOfNabla} \hspace{.1ex} \bm{r} = \hspace{-0.16ex} \boldnabla \bm{r} ,
\end{equation}
\AddUnderBrace[line width=.75pt][0,-0.1ex]%
{beginOriginOfNabla}{endOriginOfNabla}%
{${\scriptstyle \boldnabla}$}

\vspace{-0.4em} \noindent \en{where appears differential \inquotes{nabla} operator} \ru{где появляется дифференциальный оператор \inquotes{набла}}

\nopagebreak\vspace{-0.1em}\begin{equation}
\boldnabla \equiv \bm{r}^i \partial_i \hspace{.1ex} .
\end{equation}

...

\en{Gradient of cross product of two vectors}\ru{Градиент векторного произведения двух векторов},
\en{applying}\ru{применяя} \inquotes{product rule}~\eqref{productrulefordifferentiation}
\en{and}\ru{и}~\eqref{crossproductoftwovectors} \en{for any two vectors}\ru{для любых двух векторов}
(\en{partial derivative}\ru{частная производная}~$\partial_i$ \en{of some~vector by scalar coordinate}\ru{некоторого вектора по скалярной координате}~$q^i\hspace{-0.1ex}$ \en{is a~vector too}\ru{это тоже вектор}):

\nopagebreak\vspace{-0.4em}\begin{multline}\label{gradientofcrossproductoftwovectors}
\boldnabla \hspace{-0.2ex} \left( \bm{a} \hspace{-0.1ex} \times \hspace{-0.1ex} \bm{b} \right) \hspace{-0.1ex}
= \hspace{.1ex} \bm{r}^i \partial_i \hspace{-0.2ex} \left( \bm{a} \hspace{-0.2ex} \times \hspace{-0.2ex} \bm{b} \right)
= \bm{r}^i \hspace{-0.32ex} \left( \partial_i \bm{a} \hspace{-0.2ex} \times \hspace{-0.2ex} \bm{b} \hspace{.1ex} +
\bm{a} \hspace{-0.2ex} \times \hspace{-0.2ex} \partial_i \bm{b} \right) =
\\[-0.1em]
%
= \bm{r}^i \hspace{-0.32ex} \left( \partial_i \bm{a} \hspace{-0.2ex} \times \hspace{-0.2ex} \bm{b} \hspace{.1ex} -
\partial_i \bm{b} \hspace{-0.2ex} \times \hspace{-0.2ex} \bm{a} \right)
= \hspace{.1ex} \bm{r}^i \partial_i \hspace{.1ex} \bm{a} \hspace{-0.2ex} \times \hspace{-0.2ex} \bm{b} \hspace{.1ex} - \hspace{.1ex}
\bm{r}^i \partial_i \hspace{.1ex} \bm{b} \hspace{-0.2ex} \times \hspace{-0.2ex} \bm{a} =
\\[-0.1em]
%
= \hspace{-0.12ex} \boldnabla \bm{a} \hspace{-0.1ex} \times \hspace{-0.1ex} \bm{b} \hspace{.12ex} - \hspace{-0.12ex}
\boldnabla \hspace{.1ex} \bm{b} \hspace{-0.1ex} \times \hspace{-0.1ex} \bm{a} \hspace{.1ex} .
\end{multline}

\newpage ...

\newpage ...



\end{otherlanguage}

\en{\section{Integral theorems}}

\ru{\section{Интегральные теоремы}}

\begin{otherlanguage}{russian}

Для векторных полей известны интегральные теоремы Gauss’а и~Stokes’а.

\noindent\leavevmode{\indent}{\small Gauss’ theorem (divergence theorem) enables an~integral taken over a~volume to be replaced by one taken over the closed surface bounding that volume, and vice versa.\par}

\noindent\leavevmode{\indent}{\small Stokes’ theorem enables an~integral taken around a closed curve to be replaced by one taken over \emph{any} surface bounded by that curve. Stokes’ theorem relates a~line integral around a closed path to a surface integral over what is called a~\emph{capping surface} of the path.\par}

Теорема Гаусса о~дивергенции~--- про~то, как заменить объёмный интеграл поверхностным~(и~наоборот). В~этой теореме рассматривается поток (ef)flux вектора через ограничивающую объём~$V$ з\'{а}мкнутую поверхность ${\mathcal{O}(\boundary V)}$ с~единичным вектором внешней нормали~$\bm{n}$
\begin{equation}
\ointegral\displaylimits_{\mathclap{\mathcal{O}(\boundary V)}} \hspace{-0.1ex} \bm{n} \dotp \bm{a} \hspace{.4ex} d\mathcal{O} \hspace{.12ex} = \integral\displaylimits_{V} \hspace{-0.3ex} \boldnabla \dotp \bm{a} \hspace{0.4ex} dV \hspace{-0.25ex}.
\end{equation}

Объём~$V$ нарезается тремя семействами координатных поверхностей на~множество бесконечно малых элементов. Поток через поверхность ${\mathcal{O}(\boundary V)}$ равен сумме потоков через края получившихся элементов. В~бесконечной малости каждый такой элемент~--- маленький локальный дифференциальный кубик~(параллелепипед). ... Поток вектора~$\bm{a}$ через грани малого кубика объёма~$dV$ есть ${\sum_{i = 1}^{6} \bm{n}_i \dotp \bm{a} \hspace{.2ex} \mathcal{O}_i}$, а~через сам этот объём поток равен ${\boldnabla \dotp \bm{a} \hspace{.32ex} dV}$.

Похожая трактовка этой теоремы есть, к примеру, в~курсе Richard’а Feynman’а~\cite{feynman-lecturesonphysics}.

\emph{( рисунок с кубиками )}

to dice~--- нарез\'{а}ть кубиками

small cube, little cube

локально ортонормальные координаты ${\bm{\xi} = \xi_i \hspace{.2ex} \bm{n}_i \hspace{.1ex}}$, ${d\bm{\xi} = d \xi_i \hspace{.2ex} \bm{n}_i}$, ${\boldnabla = \bm{n}_i \partial_i}$

разложение вектора ${\bm{a} = a_i \bm{n}_i \hspace{.1ex}}$

Теорема Стокса о~циркуляции выражается равенством

...

\newpage ...



\end{otherlanguage}



\newpage

\en{\section{Curvature tensors}}

\ru{\section{Тензоры кривизны}}

\label{para:curvaturetensors}

\begin{changemargin}{2\parindent}{\parindent}
\bgroup % to change \parindent locally
\setlength{\parindent}{\negparindent}
\setlength{\parskip}{0.8mm minus0.2mm}
\small

\leavevmode{\indent}The \emph{Riemann curvature tensor} or \emph{Riemann\hbox{--}Christoffel tensor} (after \href{https://en.wikipedia.org/wiki/Bernhard_Riemann}{\textbf{Bernhard Riemann}} and \href{https://en.wikipedia.org/wiki/Elwin_Bruno_Christoffel}{\textbf{Elwin Bruno Christoffel}}) is the most common method used to express the curvature of Riemannian manifolds. It’s a~tensor field, it assigns a~tensor to each point of a~Riemannian manifold, that measures the extent to which the~metric tensor is not locally isometric to that of \inquotes{flat} space. The curvature tensor can also be defined for any pseudo-Riemannian manifold, or any manifold equipped with an~\inquotes{affine connection} (a~choice of such connection makes a~manifold look infinitesimally like affine \inquotes{flat} space).

The \emph{Ricci curvature tensor}, named after \href{https://en.wikipedia.org/wiki/Gregorio_Ricci-Curbastro}{\textbf{Gregorio Ricci\hbox{-}Curbastro}}, represents the amount by which the~volume of a~narrow conical piece of~a~small geodesic ball in a~curved Riemannian manifold deviates from that of the standard ball in \inquotes{flat} space.
%% It~is the~contraction of~first and~third indices of~the~Riemann curvature tensor: ${\mathscr{R}_{\hspace{-0.1ex}ab} \hspace{-0.16ex} \equiv \mathfrak{R}^{\hspace{.1ex}c}_{\hspace{.1ex}\cdot\hspace{.1ex}acb} \hspace{-0.16ex} = \textsl{g}^{\hspace{.25ex}cd} \hspace{.2ex} \mathfrak{R}_{\hspace{.1ex}cadb}}$.

\vspace{.2em}
\hfill $\sim$\:\emph{from Wikipedia, the free encyclopedia}
\par
\egroup
\nopagebreak\vspace{.12em}
\end{changemargin}

\begin{otherlanguage}{russian}

\begin{changemargin}{\parindent}{\parindent}
{\small
\noindent \emph{\inquotesx{Мы рассматриваем трёхмерное пространство классической механики, а~метрика у~нас любая невырожденная без кручения}[.]}
\par}
\end{changemargin}

\noindent Рассматривая операции тензорного анализа в~криволинейных координатах, мы исходили из~представления вектора\hbox{-}радиуса функцией этих координат: ${\bm{r}(q^{\hspace{.1ex}i})}$. Этой зависимостью порождаются выражения
векторов локального касательного \hbox{базиса} ${\bm{r}_i \hspace{-0.16ex} \equiv \partial_i \hspace{.12ex} \bm{r}}$\:${( \hspace{.12ex} \partial_i \hspace{-0.1ex} \equiv \smash{\raisemath{0.16em}{\scalebox{0.88}{$\partial$}} \hspace{-0.24ex} / \hspace{-0.32ex} \raisemath{-0.32em}{\scalebox{0.88}{$\partial$} q^{\hspace{.1ex}i}}} \hspace{.16ex} )}$,
компонент ${\textsl{g}_{ij} \hspace{-0.24ex} \equiv \bm{r}_i \hspace{-0.16ex} \dotp \bm{r}_{\hspace{-0.2ex}j}}$ и~${\textsl{g}^{\hspace{.25ex}ij} \hspace{-0.32ex} \equiv \bm{r}^i \hspace{-0.32ex} \dotp \bm{r}^j \hspace{-0.32ex} = \smash{\textsl{g}_{ij}^{\hspace{.4ex}\expminusone}}}$ единичного \inquotes{метрического} тензора~${\bm{E} = \bm{r}_i \bm{r}^i \hspace{-0.2ex} = \bm{r}^i \bm{r}_i}$,
векторов локального взаимного кокасательного \hbox{базиса} ${\bm{r}^i \hspace{-0.32ex} \dotp \bm{r}_{\hspace{-0.2ex}j} \hspace{-0.16ex} = \delta_j^{\hspace{0.1ex}i}}$, ${\bm{r}^i \hspace{-0.25ex} = \textsl{g}^{\hspace{.25ex}ij} \bm{r}_{\hspace{-0.2ex}j}}$,
диф\-ферен\-циаль\-ного набла\hbox{-}оператора Hamilton’а ${\smash{\boldnabla \equiv \bm{r}^i \partial_i}}$,
${\bm{E} = \hspace{-0.25ex} \smash{\boldnabla \bm{r}}}$,
полного дифференциала ${d \bm{\xi} = d \bm{r} \dotp \hspace{-0.2ex} \boldnabla \hspace{-0.05ex} \bm{\xi}}$,
частных производных касательного \hbox{базиса} (вторых частных производных~$\bm{r}$) ${\bm{r}_{ij} \hspace{-0.16ex} \equiv \partial_i \partial_j \bm{r} \hspace{-0.1ex} = \partial_i \hspace{.12ex} \bm{r}_{\hspace{-0.2ex}j}}$,
символов \inquotes{связности} \hbox{Христоффеля}~(\hbox{Christoffel} symbols) ${\Gamma_{\hspace{-0.25ex}ij}^{\hspace{.25ex}k} \hspace{-0.1ex} \equiv \bm{r}_{ij} \hspace{-0.2ex} \dotp \bm{r}^k
%%\hspace{-0.32ex} = \Gamma_{\hspace{-0.25ex}ijn} \hspace{.25ex} \textsl{g}^{\hspace{.25ex}nk}\hspace{-0.25ex}
}$ и~${\Gamma_{\hspace{-0.25ex}ijk} \hspace{-0.16ex} \equiv \bm{r}_{ij} \hspace{-0.2ex} \dotp \bm{r}_k
%%\hspace{-0.2ex} = \Gamma_{\hspace{-0.25ex}ij}^{\hspace{.25ex}n} \hspace{.16ex} \textsl{g}_{nk}
}$.

Представим теперь, что функция~${\bm{r}(q^{\hspace{.1ex}k})}$ не~известна, но~\hbox{зат\'{о}} в~каждой точке пространства определены шесть независимых элементов положительной симметричной метрической матрицы Грама~${\textsl{g}_{ij}(q^{\hspace{.1ex}k})}$.

Билинейная форма

...

Поскольку шесть функций~${\textsl{g}_{ij}(q^{\hspace{.1ex}k})}$ происходят от векторной функции~${\bm{r}(q^{\hspace{.1ex}k})}$, то между элементами~$\textsl{g}_{ij}$ существуют некие соотношения.

\en{Expression}\ru{Выражение} ${d\bm{r} = d\bm{r} \dotp \hspace{-0.2ex} \tikzmark{beginItsE} \boldnabla \bm{r} \tikzmark{endItsE} = \bm{r}_k \hspace{0.2ex} dq^{\hspace{.1ex}k}}$~--- \en{exact differential}\ru{полный дифференциал}. Следовательно, вторые частные производные коммутируют: ${\partial_i \hspace{.12ex} \bm{r}_{\hspace{-0.2ex}j} \hspace{-0.2ex} = \partial_j \bm{r}_i}$~(${\bm{r}_{ij} \hspace{-0.2ex} = \bm{r}_{\hspace{-0.2ex}ji}}$).
Но это необходимое условие уж\'{е} обеспечено симметрией~${\textsl{g}_{ij}}$ (\inquotes{\en{connection}\ru{связностью}}~$\nabla_{\hspace{-0.32ex}i\hspace{.1ex}}$, её~же часто называют \inquotes{\en{covariant derivative}\ru{ковариантная производная}}~--- а~символы Христоффеля это \inquotes{\en{components of connection}\ru{компоненты связности}} \en{in local coordinates}\ru{в~локальных координатах}).
\AddOverBrace[line width=.75pt][0,0.1ex]{beginItsE}{endItsE}{${\scriptstyle \bm{E}}$}

${\Gamma_{\hspace{-0.25ex}ij}^{\hspace{.25ex}k} \hspace{.2ex} \bm{r}_k \hspace{-0.2ex} = \bm{r}_{ij} \hspace{-0.16ex} \dotp \hspace{.1ex} \bm{r}^k \bm{r}_k \hspace{-0.2ex} = \bm{r}_{ij}}$

$\boldnabla \bm{v} \hspace{-0.16ex}
= \bm{r}^{i} \partial_i \hspace{-0.32ex} \left( v^{\hspace{.12ex}j} \bm{r}_{\hspace{-0.2ex}j} \right) \hspace{-0.25ex}
= \bm{r}^{i} \hspace{-0.32ex} \left( \partial_i v^{\hspace{.12ex}j} \bm{r}_{\hspace{-0.2ex}j} \hspace{-0.12ex} + v^{\hspace{.12ex}j} \bm{r}_{ij} \right)$

$\boldnabla \bm{v} \hspace{-0.16ex}
= \bm{r}^{i} \bm{r}_{\hspace{-0.2ex}j} \nabla_{\hspace{-0.32ex}i\hspace{.1ex}} v^{\hspace{.12ex}j} \hspace{-0.25ex} , \:\:
\nabla_{\hspace{-0.32ex}i\hspace{.1ex}} v^{\hspace{.12ex}j} \hspace{-0.2ex} \equiv
\partial_i v^{\hspace{.12ex}j} \hspace{-0.25ex} + \Gamma_{\hspace{-0.25ex}in}^{\hspace{.25ex}j} v^{\hspace{.1ex}n}$

$\boldnabla \bm{r}_i \hspace{-0.2ex}
= \bm{r}^k \partial_k \bm{r}_i \hspace{-0.2ex}
= \bm{r}^k \bm{r}_{ki} \hspace{-0.2ex}
%%= \bm{r}^k \hspace{.2ex} \Gamma_{\hspace{-0.25ex}ki}^{\hspace{.25ex}n} \hspace{.2ex} \bm{r}_n \hspace{-0.2ex}
= \bm{r}^k \bm{r}_n \hspace{.1ex} \Gamma_{\hspace{-0.25ex}ki}^{\hspace{.25ex}n}
\hspace{.2ex} , \:\:
\nabla_{\hspace{-0.32ex}i\hspace{.1ex}} \bm{r}_{n} \hspace{-0.25ex}
= \Gamma_{\hspace{-0.25ex}in}^{\hspace{.25ex}k} \hspace{.16ex} \bm{r}_k$

\vspace{.25em} \noindent (Добавить) про симметрию $\Gamma_{\hspace{-0.25ex}ijk}$ ...

\begin{multline}
\Gamma_{\hspace{-0.25ex}ij}^{\hspace{.25ex}n} \hspace{.16ex} \textsl{g}_{nk} \hspace{-0.24ex} = \Gamma_{\hspace{-0.25ex}ijk} \hspace{-0.2ex} = \bm{r}_{ij} \hspace{-0.2ex} \dotp \bm{r}_k \hspace{-0.1ex} = \\[-0.1em]
%
= \smalldisplaystyleonehalf \hspace{-0.2ex} \left( \bm{r}_{ij} \hspace{-0.16ex} + \bm{r}_{\hspace{-0.2ex}ji} \right) \hspace{-0.1ex} \dotp \bm{r}_k \hspace{-0.1ex}
+ \smalldisplaystyleonehalf \hspace{-0.2ex} \left( \bm{r}_{\hspace{-0.2ex}jk} \hspace{-0.16ex} - \bm{r}_{kj} \right) \hspace{-0.1ex} \dotp \bm{r}_i \hspace{-0.1ex}
+ \smalldisplaystyleonehalf \hspace{-0.2ex} \left( \bm{r}_{ik} \hspace{-0.16ex} - \bm{r}_{ki} \right) \hspace{-0.1ex} \dotp \bm{r}_{\hspace{-0.2ex}j} \hspace{-0.1ex} = \\[-0.12em]
%
= \smalldisplaystyleonehalf \hspace{-0.2ex} \left( \scalebox{0.96}[1]{$\bm{r}_{ij} \hspace{-0.2ex} \dotp \bm{r}_k \hspace{-0.16ex} + \bm{r}_{ik} \hspace{-0.2ex} \dotp \bm{r}_{\hspace{-0.2ex}j}$} \right) \hspace{-0.16ex}
+ \smalldisplaystyleonehalf \hspace{-0.2ex} \left( \scalebox{0.96}[1]{$\bm{r}_{\hspace{-0.2ex}ji} \hspace{-0.2ex} \dotp \bm{r}_k \hspace{-0.16ex} + \bm{r}_{\hspace{-0.2ex}jk} \hspace{-0.2ex} \dotp \bm{r}_i$} \right) \hspace{-0.16ex}
- \smalldisplaystyleonehalf \hspace{-0.2ex} \left( \scalebox{0.96}[1]{$\bm{r}_{ki} \hspace{-0.2ex} \dotp \bm{r}_{\hspace{-0.2ex}j} \hspace{-0.16ex} + \bm{r}_{kj} \hspace{-0.2ex} \dotp \bm{r}_i$} \right) \hspace{-0.1ex} = \\[-0.25em]
%
= \smalldisplaystyleonehalf \hspace{-0.2ex} \left(^{\mathstrut} \hspace{-0.2ex}
\partial_i ( \bm{r}_{\hspace{-0.2ex}j} \hspace{-0.2ex} \dotp \bm{r}_k ) \hspace{-0.16ex}
+ \partial_j ( \bm{r}_{i} \hspace{-0.2ex} \dotp \bm{r}_k ) \hspace{-0.16ex}
- \partial_k ( \bm{r}_{i} \hspace{-0.2ex} \dotp \bm{r}_{\hspace{-0.2ex}j} )
\hspace{-0.12ex} \right) \hspace{-0.4ex} = \\[-0.25em]
%
= \smalldisplaystyleonehalf \hspace{-0.16ex} \left(
\partial_i \hspace{.12ex} \textsl{g}_{jk} \hspace{-0.2ex}
+ \partial_j \hspace{.1ex} \textsl{g}_{ik} \hspace{-0.2ex}
- \partial_k \hspace{.12ex} \textsl{g}_{ij}
\right) \hspace{-0.4ex} .
\end{multline}

Все символы Христоффеля тождественно равны нулю лишь в~ортонормальной~(декартовой) системе. (А~какие они для косоугольной?)

Пойдём дальше: ${d\bm{r}_i \hspace{-0.2ex} = d\bm{r} \dotp \hspace{-0.2ex} \boldnabla \bm{r}_i \hspace{-0.2ex} = dq^{\hspace{.1ex}k} \partial_k \bm{r}_i \hspace{-0.2ex} = \bm{r}_{ik} \hspace{0.2ex} dq^{\hspace{.1ex}k}\hspace{-0.25ex}}$~--- тоже полные дифференциалы.
Поэтому ${\partial_i \partial_j \bm{r}_k \hspace{-0.2ex} = \partial_j \partial_i \bm{r}_k}$~(${\partial_i \bm{r}_{\hspace{-0.2ex}jk} \hspace{-0.2ex} = \partial_j \bm{r}_{ik}}$),
и~трёхиндексный объект третьих частных производных
%\nopagebreak\vspace{-0.1em}
\begin{equation}
\bm{r}_{ijk} \hspace{-0.1ex} \equiv \hspace{.1ex} \partial_i \partial_j \partial_k \bm{r}
= \partial_i \hspace{.12ex} \bm{r}_{\hspace{-0.2ex}jk} %%\hspace{-0.4ex} = \partial_k \bm{r}_{ij}
\end{equation}

\vspace{-0.2em} \noindent симметричен по~первому и~второму индексам (а~не~только по~второму и~третьему). И~тогда равен нулю~${\hspace{-0.16ex}^4\bm{0}}$ следующий тензор четвёртой сложности~--- тензор кривизны Римана\hbox{--}Христоффеля
\nopagebreak\vspace{.1em}\begin{equation}\label{riemanncurvaturetensor}
{^4\bm{\mathfrak{R}}} = \hspace{.12ex} \mathfrak{R}_{\hspace{.1ex}ijkn} \hspace{.12ex} \bm{r}^i \bm{r}^j \bm{r}^k \bm{r}^n \hspace{-0.25ex}, \:\:
\mathfrak{R}_{\hspace{.1ex}ijkn} \hspace{-0.12ex} \equiv \left( \hspace{.12ex} \bm{r}_{\hspace{-0.2ex}jik} \hspace{-0.2ex} - \bm{r}_{ijk} \hspace{.12ex} \right) \hspace{-0.16ex} \dotp \bm{r}_n .
\end{equation}



{\small
The curvature tensor is given in terms of the connection $\nabla$ as
${ R(u,v)w=\nabla_u\nabla_v w - \nabla_v \nabla_u w - \nabla_{[u,v]} w }$
where [u,v] is the Lie bracket of vector fields. Occasionally, the curvature tensor is defined with the opposite sign.

If ${u=\partial /\partial x^{i}}$ and ${v=\partial /\partial x^{j}}$ are coordinate vector fields then ${[u,v]=0}$ and therefore the formula simplifies to ${\displaystyle R(u,v)w=\nabla _{u}\nabla _{v}w-\nabla _{v}\nabla _{u}w}$.

The curvature tensor measures noncommutativity of the covariant derivative, and as such is the integrability obstruction for the existence of an isometry with \inquotes{flat} space. The linear transformation ${w\mapsto R(u,v)w}$ is also called the curvature transformation or endomorphism.
\par}



Выразим компоненты~${\mathfrak{R}_{\hspace{.1ex}ijkn}}$ через метрическую матрицу~${\textsl{g}_{ij}}$. Начнём с~дифференцирования локального кобазиса:
\[ \bm{r}^i \hspace{-0.32ex} \dotp \bm{r}_k \hspace{-0.16ex} = \delta_k^{\hspace{0.1ex}i} \;\Rightarrow\:
\partial_j \bm{r}^i \hspace{-0.32ex} \dotp \bm{r}_k + \hspace{.1ex} \bm{r}^i \hspace{-0.32ex} \dotp \bm{r}_{\hspace{-0.2ex}jk} = 0 \;\Rightarrow\:
\partial_j \bm{r}^i \hspace{-0.12ex} = - \hspace{.2ex} \Gamma_{\hspace{-0.25ex}jk}^{\hspace{.25ex}i} \hspace{.2ex} \bm{r}^k
\hspace{-0.4ex}. \]

...


Равенство тензора Риччи нулю: ${\hspace{.1ex}\pmb{\scalebox{1.2}[1]{$\mathscr{R}$}} \hspace{-0.16ex} = \hspace{-0.2ex} {^2\bm{0}}}$ (в~компонентах тут шесть уравнений ${\hspace{.1ex}\mathscr{R}_{\hspace{-0.1ex}ij} \hspace{-0.3ex} = \mathscr{R}_{\hspace{-0.1ex}ji} \hspace{-0.2ex} = 0}$)~--- это достаточное условие %%(шесть условий в~компонентах)
интегрируемости~(\inquotes{совместности}, \inquotes{compatibility}) при определении радиуса-вектора~${\bm{r}(q^{\hspace{.1ex}k})}$ по~полю~${\textsl{g}_{ij}(q^{\hspace{.1ex}k})}$.

\end{otherlanguage}

\section*{\small \wordforbibliography}

\begin{changemargin}{\parindent}{0pt}
\fontsize{10}{12}\selectfont

\begin{otherlanguage}{russian}

Существует немало книг, в~которых рассматривается лишь аппарат тензорного исчисления~\cite{mcconnell-tensoranalysis, schouten-tensoranalysis, borisenko.tarapov, rashevsky-riemanniangeometry, sokolnikov-tensoranalysis}. Преобладает \inquotes{индексный} подход~--- тензоры трактуются как матрицы компонент, преобразующиеся определённым путём.
\inquotes{Прямой} подход к~тензорам излагается, например, в~приложениях книг А.\,И.\;Лурье~\cite{lurie-nonlinearelasticity, lurie-theoryofelasticity}.
Яркое описание теории векторных полей можно найти у~R.\:Feynman’а~\cite{feynman-lecturesonphysics}.
Сведения о~тензорном исчислении содержатся и~в~своеобразной и~глубокой книге C.\:Truesdell’а~\cite{truesdell-firstcourse}.

\end{otherlanguage}

\end{changemargin}
