\en{\section{Statics}}

\ru{\section{Статика}}

\label{section:statics}

\en{Let there is}\ru{Пусть есть}
\en{a~mechanical system}\ru{механическая система}
\en{with }\en{stationary}\ru{со~}\ru{стационарными}~%
(\en{constant over time}\ru{постоянными во~времени})
\en{constraints}\ru{связями}
\en{under}\ru{под}
\en{the~action}\ru{действием}
\en{of~static}\ru{статических}~%
(\en{not changing with time}\ru{не~меняющихся со~временем})
\en{active forces}\ru{активных сил}~${\bm{F}_k}$.
\en{In~equilibrium}\ru{В~равновесии}
\en{all}\ru{все}~${\locationvector_{\hspace{-0.1ex}k} \narroweq \hspace{.2ex} \boldconstant}$,
\en{hence}\ru{отсюда}
${\variation{\hspace{.1ex}\locationvector_{\hspace{-0.1ex}k}} \narroweq \hspace{.2ex} \zerovector}$,
${\scalebox{.8}{$ \displaystyle \frac{\raisemath{-0.1ex}{\partial \hspace{.1ex} \locationvector_{\hspace{-0.1ex}k}}}{\raisemath{-0.1ex}{\partial q^i}} $}
\hspace{.1ex} \narroweq \hspace{.2ex} \zerovector}$,
\en{and }\ru{и~}\en{the~principle of~virtual work}\ru{принцип виртуальной работы}
\en{is formulated}\ru{формулируется}
\en{as}\ru{как}

\nopagebreak\vspace{-0.1em}
\begin{equation}\label{statics.discrete:principleofvirtualwork}
\scalebox{.8}[.9]{$ \displaystyle \sum_{\smash{k}} $} \hspace{.25ex}
\bm{F}_k \hspace{-0.1ex} \dotp \variation{\hspace{.1ex}\locationvector_{\hspace{-0.1ex}k}} \hspace{-0.1ex} = 0
\hspace{.8em} \Leftrightarrow \hspace{.6em}
\scalebox{.92}[.96]{$ \displaystyle \sum_{\smash{k}} $} \hspace{.25ex}
\bm{F}_k \hspace{-0.1ex} \dotp \scalebox{.96}{$ \displaystyle \frac{\raisemath{-0.1ex}{\partial \hspace{.1ex} \locationvector_{\hspace{-0.1ex}k}}}{\raisemath{-0.1ex}{\partial q^i}} $}
= Q_{\hspace{-0.1ex}i} = 0
\hspace{.1ex} .
\vspace{-0.1em}\end{equation}

\vspace{-0.2em}\noindent
\en{Both pieces}\ru{Обе части}
\en{are essential}\ru{существенны}\::
\en{and }\ru{и~}\en{the~variational equation}\ru{вариационное уравнение},
\en{and }\ru{и~}\en{zeros}\ru{нули}
\en{in the~generalized forces}\ru{в~обобщённых силах}.

\en{Relations}\ru{Соотношения}~\eqref{statics.discrete:principleofvirtualwork}\en{ are}\ru{\:--- это}
\en{the~most}\ru{самые}
\en{generic and universal}\ru{общие и универсальные}
\en{equations}\ru{уравнения}
\en{of~statics}\ru{статики}.
\en{In literature}\ru{В~литературе}\en{,}
\ru{распространено }\en{the~narrow}\ru{узкое}
\en{conception}\ru{представление}
\en{of the~equilibrium equations}\ru{уравнений равновесия}
\en{as}\ru{как}
\en{the~balance}\ru{баланса}
\en{of~forces and moments}\ru{сил и~моментов}\en{ is widespread}.
\en{But}\ru{Но}
\en{in that case too}\ru{и~в~том случае},
\en{as in any other}\ru{как и~в~любом},
\en{the~set}\ru{набор}
\en{of the~equilibrium equations}\ru{уравнений равновесия}
\en{exactly matches}\ru{точно совпадает}
\en{with }\ru{с~}\en{the~generalized coordinates}\ru{обобщёнными координатами}.
\inquotes{\en{The~resultant force}\ru{Результирующая сила}}
(\en{also referred to as}\ru{также упоминаемая как}
\inquotes{\en{the~net~(full) force}\ru{равнодействующая, суммарная~(полная), \inquotes{нетто} сила}}
\en{or}\ru{или}
\inquotes{\en{the~net vector}\ru{суммарный, главный, \inquotes{нетто} вектор}})
\en{and}\ru{и}~\inquotes{\en{the~resultant couple}\ru{результирующая пара сил}}~%
(\inquotesx{\en{the~net couple}\ru{суммарная, главная, \inquotes{нетто} пара}}[,]
\inquotes{\en{the~net moment}\ru{момент суммарный, нетто, главный}})
\en{figure}\ru{фигурируют}
\en{in the~equilibrium equations}\ru{в~уравнениях равновесия}%
%%%%
\footnote{%
\en{Since}\ru{С~тех пор, как}
\bookauthor{\href{https://en.wikipedia.org/wiki/Louis_Poinsot}{Louis Poinsot}}
\en{described}\ru{описал}
\en{the~reduction}\ru{сведение (редукцию, сокращение)}
\en{of~any set of~forces}\ru{любой совокупности сил},
\en{acting}\ru{действующих}
\en{on the~same}\ru{на~одну и~ту~же}
\en{absolutely rigid}\ru{совершенно жёсткую}
\en{system}\ru{систему},
\en{into the~single force}\ru{до единственной силы}
\en{and }\ru{и~}\en{the~single couple}\ru{единственной пары}
\en{in his book}\ru{в~своей книге}
\inquotesx{Élémens de~statique}[,]
\en{first published}\ru{впервые опубликованной}
\en{in}\ru{в}~1803\:---
\href{https://books.google.com/books?id=uO3UjwKIKHUC&printsec=frontcover}{Éléments de statique, chez Calixte-Volland, 1803~(an~XII)}.
\howmanypages{267 p.}
\href{https://gallica.bnf.fr/ark:/12148/bpt6k6213152z.texteImage}{Onzième~(11ème) édition\:: Gauthier-Villars, 1873}.%
}
%%%%
\en{just because}\ru{просто потому, что}
\en{the~system has}\ru{у~системы есть}
\en{translational}\ru{трансляционные (поступательные)}
\en{and }\ru{и~}\en{rotational}\ru{поворотные (вращательные)}
\en{degrees o’freedom}\ru{степени свободы трансляции и~поворота}.
\en{The~huge popularity}\ru{Огромная популярность}
\en{of~forces}\ru{сил}
\en{and }\ru{и~}\en{moments}\ru{моментов}~(\en{force couples}\ru{пар сил})
\en{comes}\ru{идёт}
\en{not as much}\ru{не~столько}
\en{from}\ru{от}
\en{the~prevalence}\ru{преобладания}
\en{of~statics}\ru{статики}
\en{of a~perfectly non-deformable}\ru{совершенно недеформируемого}~%
(\en{ideally rigid}\ru{идеально жёсткого})
\en{solid body}\ru{твёрдого тела},
\en{but more}\ru{но больше}
\en{from the~fact that}\ru{от того, что}
\en{the~virtual work}\ru{виртуальная работа}
\en{of~internal forces}\ru{внутренних сил}
\en{on all movements}\ru{на всех движениях}
\en{of~the~system}\ru{системы}
\en{as a~rigid whole}\ru{как жёсткого целого}
\en{is always}\ru{всегда}
\en{equal to zero}\ru{равна нулю}
\en{for any medium}\ru{для любой среды}.

\en{Let}\ru{Пусть}
\en{two kinds}\ru{два вида}
\en{o’forces}\ru{сил}
\en{act}\ru{действуют}
\en{in the~system}\ru{в~системе}\::
\en{potential}\ru{потенциальные},
\en{with }\ru{с~}\en{the coordinate-dependent}\ru{зависящей от~координат}
\en{energy}\ru{энергией}~${\potentialenergyinmechanics (q^i)}$,
\en{and plus}\ru{а~также}
\en{external ones}\ru{внешние}~${{\mathcircabove{Q_{\hspace{-0.2ex}j}}^{\hspace{-0.5ex}\smthexternal}} \hspace{-0.25ex} \equiv \hspace{-0.25ex} P_{\hspace{-0.2ex}j}}$.
\en{From}\ru{Из}~\eqref{statics.discrete:principleofvirtualwork}
\en{follow}\ru{следуют}
\en{the~equilibrium equations}\ru{уравнения равновесия}

\nopagebreak\vspace{-0.2em}
\begin{equation}\label{staticequilibrium.withpotentialandexternalforces}
\scalebox{.96}{$ \displaystyle \frac{\raisemath{-0.15em}{\partial \hspace{.1ex} \potentialenergyinmechanics}}{\raisemath{-0.1em}{\partial q^i}} $} = P_{\hspace{-0.1ex}i}
%%%\hspace{.2ex} ,
\end{equation}

\vspace{-0.5em}\noindent
\en{and}\ru{и}
\en{the~exact differential}\ru{точный~(полный) дифференциал}\en{ of}~${\potentialenergyinmechanics (q^i)}$
(\en{time independent}\ru{независимого от~времени})
\en{is}\ru{есть}

\nopagebreak\vspace{-0.1em}
\begin{equation}\label{fulldifferentialoftimeindependentpotentialenergy}
d \hspace{.1ex} \potentialenergyinmechanics
= \scalebox{.95}[1]{$\displaystyle \sum_{\smash{i}}$} \hspace{.32ex}
\scalebox{.96}{$ \displaystyle \frac{\raisemath{-0.15em}{\partial \hspace{.1ex} \potentialenergyinmechanics}}{\raisemath{-0.1em}{\partial q^i}} $} \hspace{.2ex} dq^i
= \raisebox{.1em}{\scalebox{.8}[.9]{$\displaystyle \sum_{\smash{i}}$}} \hspace{.2ex} P_{\hspace{-0.1ex}i} \hspace{.2ex} dq^i
\hspace{-0.1ex} .
\end{equation}

\vspace{-0.5em}%%\noindent
\en{Equations}\ru{Уравнения}~\eqref{staticequilibrium.withpotentialandexternalforces}
\en{formulate}\ru{формулируют}
\en{the~problem of~statics}\ru{проблему статики},
\en{non-linear}\ru{нелинейную}
\en{in~overall}\ru{в~общем},
\en{about the~relation}\ru{об~отношении}
\en{of~}\href{https://en.wikipedia.org/wiki/Mechanical_equilibrium}{\en{the~equilibrium position}\ru{положения равновесия}}~%
${q_{\circ}^{\hspace{.1ex}i}}$
\en{with the~external loads}\ru{с~внешними нагрузками}~${P_{\hspace{-0.1ex}i}}$.

\en{A~linear system}\ru{Линейная система}
\en{with }\ru{с~}\en{quadratic potential}\ru{квадратичным потенциалом}~$\potentialenergyinmechanics$
\en{as a~function}\ru{как функцией}
\en{of~coordinates}\ru{координат}
%
\begin{equation}\label{potentialenergyinmechanics.as.stiffnessbycoordinates}
\potentialenergyinmechanics = \hspace{.1ex} \raisebox{.1em}{\smalldisplaystyleonehalf \hspace{-0.4ex} \scalebox{.8}[.9]{$\displaystyle \sum_{\smash{i,k}}$}} \hspace{.2ex} C_{ik} \hspace{.25ex} q^{i} \hspace{-0.1ex} q^{k}
\end{equation}

\nopagebreak\vspace{-1.25em}
\begin{equation}\label{staticequilibrium.lineardiscretesystem}
\raisebox{.1em}{\scalebox{.8}[.9]{$\displaystyle \sum_{\smash{k}}$}} \hspace{.2ex} C_{ik} \hspace{.12ex} q^k \hspace{-0.1ex}
= \hspace{.1ex} P_{\hspace{-0.1ex}i} \hspace{.1ex} .
\vspace{-0.1em}\end{equation}

\vspace{-0.2em}\noindent
\en{Here}\ru{Тут}
\en{figure}\ru{фигурируют}
\en{elements}\ru{элементы}~$C_{ik}$
\en{of }\inquotesx{\en{the~stiffness matrix}\ru{матрицы жёсткости}}[,]
\en{coordinates}\ru{координаты}~${q^k}$
\en{and}\ru{и}~\en{external loads}\ru{внешние нагрузки}~${P_{\hspace{-0.1ex}i}}$.

%% Above written is also applicable to linear elastic continua.
%% Выше написанное применимо также и к линейным упругим континуумам.

\en{Structures}\ru{Конструкции}
(\en{both human\hbox{-}made artificial}\ru{и~сделанные человеком искусственные}\ru{,}
\en{and in the~nature}\ru{и~в~природе})
\en{most often have}\ru{чаще всего имеют} %naturally
\en{a~positive-definite}\ru{положительно определённую}
\en{stiffness matrix}\ru{матрицу жёсткости}~${C_{ik}}$.
\en{Then}\ru{Тогда}
${\determinant \hspace{.16ex} C_{ik} \hspace{-0.2ex} > 0}$,
\en{the~solution}\ru{решение}
\en{of a~linear algebraic system}\ru{линейной алгебраической системы}~\eqref{staticequilibrium.lineardiscretesystem}
\en{is unique}\ru{единственно},
\en{and this solution}\ru{и~это решение}
\en{can be}\ru{может быть}
\en{substituted}\ru{заменено}
\en{by minimization}\ru{минимизацией}
\en{of the~quadratic form}\ru{квадратичной формы}

\nopagebreak\vspace{-0.1em}
\begin{equation}\label{discrete:potentialenergyofsystem}
\potentialenergyfunctional \hspace{.2ex} (q^{\hspace{.1ex}j}) %%(q^{1}\hspace{-0.5ex}, q^{2}\hspace{-0.5ex}, \ldots)
\equiv \hspace{.1ex}
\potentialenergyinmechanics - \raisebox{.1em}{\scalebox{.8}[.9]{$ \displaystyle \sum_{\smash{i}} $}} \hspace{.2ex}
P_{\hspace{-0.1ex}i} \hspace{.12ex} q^{i}
%
= \hspace{.1ex}
\raisebox{.1em}{\smalldisplaystyleonehalf \hspace{-0.4ex} \scalebox{.8}[.9]{$\displaystyle \sum_{\smash{i,k}}$}} \hspace{.2ex}
\hspace{.1ex} q^{i} \hspace{.1ex} C_{ik} \hspace{.1ex} q^{k} \hspace{-0.2ex}
- \raisebox{.1em}{\scalebox{.8}[.9]{$\displaystyle \sum_{\smash{i}}$}} \hspace{.2ex}
P_{\hspace{-0.1ex}i} \hspace{.12ex} q^{i}
%
\hspace{.1em}\to\hspace{.2em} \mathrm{min}
\hspace{.16ex} .
\vspace{-0.1em}\end{equation}

\vspace{-0.1em}
\en{However}\ru{Однако},
\en{the design}\ru{дизайн}
\en{may be}\ru{может быть}
\en{so unfortunate}\ru{столь неудачным}\ru{,}
\en{that}\ru{что}
\en{the~stiffness matrix}\ru{матрица жёсткости}
\en{becomes}\ru{выходит}
\en{singular}\ru{сингулярной}~(\en{noninvertible}\ru{необратимой})
\en{with}\ru{с}~${\determinant \hspace{.16ex} C_{ik} = \hspace{.1ex} 0}$
(\en{or}\ru{или~же}
\en{the~determinant}\ru{детерминант}
\en{is very close}\ru{очень близок}
\en{to zero}\ru{к~нулю},
${\determinant \hspace{.16ex} C_{ik} \approx \hspace{.1ex} 0}$\:---
\en{the~nearly singular}\ru{почти сингулярная}
\en{matrix}\ru{матрица}).
\en{Then}\ru{Тогда}
\en{the~solution}\ru{решение}
\en{of~the~linear problem}\ru{линейной проблемы}
\en{of~statics}\ru{статики}~\eqref{staticequilibrium.lineardiscretesystem}
\en{exists}\ru{существует}
\en{only}\ru{лишь}
\en{when}\ru{когда}
\en{external loads}\ru{внешние нагрузки}~${P_{\hspace{-0.1ex}i}}$
\en{are orthogonal}\ru{ортогональны}
\en{to all}\ru{всем}
\en{linearly independent solutions}\ru{линейно независимым решениям}
\en{of the~homogeneous conjugate system}\ru{однородной сопряжённой системы}

...


\en{The famous}\ru{Известные}
\en{theorems}\ru{теоремы}
\en{of~statics}\ru{статики}
\en{for}\ru{для}
\en{linear}\ru{линейных}
\en{continua}\ru{\rucontinuum{}ов}~%
(\chapterdotsectionref{chapter:linearclassicalelasticity}{section:theoremsofstatics})
\en{can be}\ru{могут быть}
\en{easily proved}\ru{легко доказаны}
\en{for}\ru{для}
\en{a~finite number}\ru{конечного числа}
\en{of~degrees o’freedom}\ru{степеней свободы}.
\en{The}\ru{Теорема} \href{https://en.wikipedia.org/wiki/Beno%C3%AEt_Paul_%C3%89mile_Clapeyron}{Clapeyron’\en{s}\ru{а}}\en{ theorem}
\en{looks here like}\ru{выглядит здесь как}

...


\en{The~reciprocal work theorem}\ru{Теорема о~взаимности работ}
(\inquotes{\en{the~work}\ru{работа}~${W_{\hspace{-0.1ex}12}}$
\en{of the first set’s forces}\ru{сил первого набора}
\en{on displacements}\ru{на смещениях}
\en{from}\ru{от}
\en{the~forces}\ru{сил}
\en{of the~second}\ru{второго}
\en{is equal to}\ru{равна}
\en{the~work}\ru{работе}~${W_{\hspace{-0.15ex}21}}$
\en{of the second set’s forces}\ru{сил второго набора}
\en{on displacements}\ru{на смещениях}
\en{from}\ru{от}
\en{the~forces}\ru{сил}
\en{of the~first}\ru{первого}})
\en{instantly}\ru{мгновенно}
\en{derives}\ru{выводится}
\en{from}\ru{из}~\eqref{staticequilibrium.lineardiscretesystem}\::

(....)

\noindent
\en{Here}\ru{Тут}
\ru{существенна }\en{the~symmetry}\ru{симметрия}
\en{of~the~stiffness matrix}\ru{матрицы жёсткости}~%
${C_{i\hspace{-0.1ex}j} \hspace{-0.3ex} = \hspace{-0.2ex} C_{ji}}$\en{ is essential}\:---
\en{that}\ru{то, что}
\href{https://en.wikipedia.org/wiki/Conservative_system}{\en{the~system is conservative}\ru{система консервативна}}.

.....

\en{Turning back}\ru{Возвращаясь}
\en{to the~problem}\ru{к~проблеме}~\eqref{staticequilibrium.withpotentialandexternalforces},
\en{sometimes called}\ru{иногда называемой}
\en{the}\ru{теоремой} Lagrange’\en{s}\ru{а}\en{ theorem}.
\en{Inverted}\ru{Обращённая}
\en{by}\ru{преобразованием} Legendre\ru{’а}\en{ transform(ation)},
\en{it}\ru{она}
\en{translates}\ru{переводится}
\en{into}\ru{в}

\nopagebreak\vspace{-0.4em}
\begin{equation*}
\begin{array}{c}
d \biggl( \hspace{-0.1ex} \scalebox{.8}{$\displaystyle \sum_{\smash{i}}$} \hspace{.2ex} P_{\hspace{-0.1ex}i} \hspace{.2ex} q^i \hspace{-0.12ex} \biggr) \hspace{-0.5ex}
= \scalebox{.8}{$\displaystyle \sum_{\smash{i}}$} \hspace{.25ex} d \hspace{-0.25ex} \left( \hspace{-0.1ex} P_{\hspace{-0.1ex}i} \hspace{.2ex} q^i \right) \hspace{-0.3ex}
= \scalebox{.8}{$\displaystyle \sum_{\smash{i}}$} \hspace{-0.3ex} \left( \hspace{.1ex}
q^i \hspace{.2ex} d P_{\hspace{-0.1ex}i}
+ P_{\hspace{-0.1ex}i} \hspace{.2ex} dq^i \right)
\hspace{-0.3ex} ,
\\[1em]
%
\scalebox{.8}{$\displaystyle \sum_{\smash{i}}$} \hspace{.25ex} d \hspace{-0.25ex} \left( \hspace{-0.1ex} P_{\hspace{-0.1ex}i} \hspace{.2ex} q^i \right) \hspace{-0.2ex}
- \hspace{.1ex} \tikzmark{beginDifferentialOfPotentialEnergy} \scalebox{.8}{$\displaystyle \sum_{\smash{i}}$} \hspace{.2ex} P_{\hspace{-0.1ex}i} \hspace{.2ex} dq^i \tikzmark{endDifferentialOfPotentialEnergy}
= \scalebox{.8}{$\displaystyle \sum_{\smash{i}}$} \hspace{.32ex} q^i \hspace{.2ex} dP_{\hspace{-0.1ex}i}
\hspace{.3ex} ,
\\[1.6em]
%
d \biggl( \scalebox{.8}{$\displaystyle \sum_{\smash{i}}$} \hspace{.2ex} P_{\hspace{-0.1ex}i} \hspace{.2ex} q^i - \potentialenergyinmechanics \hspace{-0.1ex} \biggr) \hspace{-0.4ex}
= \hspace{-0.1ex} \scalebox{.8}{$\displaystyle \sum_{\smash{i}}$} \hspace{.32ex} q^i \hspace{.2ex} dP_{\hspace{-0.1ex}i}
\hspace{.1ex} = \hspace{-0.2ex} \tikzmark{beginDifferentialOfComplementaryEnergy} \scalebox{.95}[1]{$\displaystyle \sum_{\smash{i}}$} \hspace{.4ex}
\scalebox{.95}{$ \displaystyle \frac{\raisemath{-0.1em}{\partial \hspace{.1ex} \complementaryenergyinmechanics}}{\raisemath{-0.1em}{\partial P_{\hspace{-0.1ex}i}}} $}
\hspace{.25ex} dP_{\hspace{-0.1ex}i} \tikzmark{endDifferentialOfComplementaryEnergy}
%
\hspace{.3ex} ,
\end{array}\end{equation*}%
\AddUnderBrace[line width=.75pt][-0.2ex,-0.7em][xshift=.1ex, yshift=.1em]%
{beginDifferentialOfPotentialEnergy}{endDifferentialOfPotentialEnergy}%
{\scalebox{.75}{${ d \hspace{.2ex} \potentialenergyinmechanics }$}}%
\AddUnderBrace[line width=.75pt][-0.1ex,-0.9em][xshift=.2ex, yshift=.1em]%
{beginDifferentialOfComplementaryEnergy}{endDifferentialOfComplementaryEnergy}%
{\scalebox{.75}{${ d \hspace{.1ex} \complementaryenergyinmechanics }$}}

\vspace{-0.1em}\noindent
\en{where}\ru{где}
\en{appears}\ru{появляется}
\en{the~exact differential}\ru{полный~(точный) дифференциал}
\en{of~the~so\hbox{-}called}\ru{так называемой}
\inquotesx{\en{complementary energy}\ru{дополнительной энергии}}\:$\complementaryenergyinmechanics$

\nopagebreak\vspace{-0.5em}
\begin{equation}\label{Castigliano:theorem}
q^i \hspace{-0.2ex} = \scalebox{.95}{$ \displaystyle \frac{\raisemath{-0.1em}{\partial \hspace{.1ex} \complementaryenergyinmechanics}}{\raisemath{-0.1em}{\partial P_{\hspace{-0.1ex}i}}} $} \hspace{.2ex} ,
\:\;
\complementaryenergyinmechanics (P_{\hspace{-0.1ex}i}) \hspace{-0.2ex}
= \hspace{-0.2ex} \scalebox{.8}{$\displaystyle \sum_{\smash{i}}$} \hspace{.2ex} P_{\hspace{-0.1ex}i} \hspace{.12ex} q^i \hspace{-0.2ex} - \potentialenergyinmechanics
\hspace{.1ex} .
\end{equation}

\vspace{-0.1em}\noindent
\en{This is known as}\ru{Это известно как}
\en{the}\ru{теорема} Castigliano\en{ theorem}\footnote{%
\bookauthor{\href{https://it.wikipedia.org/wiki/Carlo_Alberto_Castigliano}{Carlo Alberto Castigliano}}.
\href{https://architettura.unige.it/bma/PDF/Castigliano_1873_Tesi.pdf}{Intorno ai sistemi elastici,
Dissertazione presentata da Castigliano Alberto alla Commissione Esaminatrice della R.\;Scuola d’Applicazione degli Ingegneri in Torino per ottenere la Laurea di Ingegnere Civile.
Torino, Vincenzo Bona, 1873.}%
}\hbox{\hspace{-0.5ex}.}
\en{For}\ru{Для}
\en{a~linear system}\ru{линейной системы}
\eqref{staticequilibrium.lineardiscretesystem}~${\Rightarrow}$~%
${\complementaryenergyinmechanics (P_{\hspace{-0.1ex}i}) \hspace{-0.2ex} = \potentialenergyinmechanics (q^i)}$.
\en{Theorem}\ru{Теорема}~\eqref{Castigliano:theorem}
\en{is sometimes very useful}\ru{иногда бывает очень полезна}\:---
\en{when}\ru{когда}
\en{the~complementary energy}\ru{дополнительную энергию}
\en{as the~function}\ru{как функцию}
\en{of external loads}\ru{внешних нагрузок}~${\complementaryenergyinmechanics(P_{\hspace{-0.1ex}i})}$
\en{is easy to find}\ru{легко найти}.
\en{Someone may come across}\ru{Кто\hbox{-}то может встретить}
\en{the~so\hbox{-}called}\ru{так называемые}
\inquotes{\en{statically determinate}\ru{статически определимые}}
%%(\en{or}\ru{или} \inquotes{\en{isostatic}\ru{изостатические}})
\en{structures}\ru{конструкции}~(\en{systems}\ru{системы}),
\en{for which}\ru{для которых}
\en{all internal forces}\ru{все внутренние силы}
\en{can luckily be found}\ru{повезёт найти}
\en{just only}\ru{просто лишь}
\en{from}\ru{из}
\ru{уравнений }\en{the~balance}\ru{баланса}~(\en{equilibrium}\ru{равновесия})\en{ equations}
\en{for}\ru{для}
\en{forces and moments}\ru{сил и~моментов}.
\en{For such}\ru{Для таких}
\en{structures}\ru{конструкций}\en{,}
\eqref{Castigliano:theorem}
\en{is effective}\ru{эффективна}.

\en{Unlike}\ru{В~отличие от}
\en{the~linear problem}\ru{линейной задачи}~\eqref{staticequilibrium.lineardiscretesystem},
\en{the~nonlinear problem}\ru{нелинейная задача}~\eqref{staticequilibrium.withpotentialandexternalforces}
\en{may}\ru{может}
\en{have no solutions}\ru{не~иметь решений}
\en{at all}\ru{в\'{о}все}
\en{or may have}\ru{или~же иметь}
\en{several of~them}\ru{их н\'{е}сколько}.

....


\newcommand\dAlembertsTraiteDeDynamique{%
\href{https://books.google.ru/books?id=5R4OAAAAQAAJ&printsec=frontcover}{%
Traité de Dynamique,
dans lequel les Loix de l’Equilibre
\& du mouvement des Corps sont réduites au plus petit nombre possible,
\& démontrées d’une maniére nouvelle,
\& où l’on donne un Principe général pour trouver le Mouvement de plusieurs Corps qui agissent les uns sur les autres,
d’une maniére quelconque.
Paris\:: David l’aîné, MDCCXLIII~(1743).%
}%
}

\en{The~overview}\ru{Обзор}
\en{of statics}\ru{статики}
\en{in classical mechanics}\ru{в~классической механике}
\en{I am ending with}\ru{я заканчиваю}
\href{https://en.wikipedia.org/wiki/D%27Alembert%27s_principle}{\en{the}\ru{принципом} \hbox{d’\hspace{-0.2ex}Alembert’\en{s}\ru{а}}\en{ principle}}%
\footnote{%
\href{https://en.wikipedia.org/wiki/Jean_le_Rond_d'Alembert}{\bookauthor{Jean Le Rond d’Alembert}}.
\dAlembertsTraiteDeDynamique
}\,:
\en{the~dynamic equations}\ru{динамические уравнения}
\en{differ from}\ru{отличаются от}
\en{the~static ones}\ru{статических}
\en{only}\ru{лишь}
\en{in~additional}\ru{дополнительными}
\inquotes{\en{inertia forces}\ru{силами инерции}}~%
(\inquotes{\en{fictitious forces}\ru{фиктивными силами}})~%
${\hspace{-0.2ex}m_k \hspace{.1ex} \mathdotdotabove{\locationvector}_{\hspace{-0.1ex}k}}$.
\en{The~}\ru{Принцип }\hbox{d’\hspace{-0.2ex}Alembert’\en{s}\ru{а}}\en{ principle}
\en{is pretty obvious}\ru{весьма очевиден},
\en{but}\ru{но}
\en{applying it}\ru{применять его}
\en{everytime\;\&\;everywhere}\ru{всегда и~везде}\en{ is}\ru{\:---}
\en{a~mistake}\ru{ошибка}.
\en{As example}\ru{Как пример},
\href{https://en.wikipedia.org/wiki/Navier%E2%80%93Stokes_equations}{\en{the~equations}\ru{уравнения}
\en{of~motion}\ru{движения}
\en{for}\ru{для}
\en{a~viscous fluid}\ru{вязкой жидкости}
(\ru{уравнения }Navier\hbox{--}Stokes\ru{’а}\en{ equations})}
\en{in statics and in dynamics}\ru{в~статике и~в~динамике}
\en{differ}\ru{отличаются}
\en{not only}\ru{не~только лишь}
\en{in inertial adjunct}\ru{инерционной добавкой}.
\en{Nevertheless}\ru{Тем не менее},
\en{for}\ru{для}
\en{solid elastic bodies}\ru{твёрдых упругих тел}
\en{the~}\ru{принцип }\hbox{d’\hspace{-0.2ex}Alembert’\en{s}\ru{а}}\en{ principle}
\en{always apply}\ru{всегда употребим}.
