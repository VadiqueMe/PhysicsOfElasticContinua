\en{\section{Small oscillations (vibrations)}}

\ru{\section{Малые колебания (вибрации)}}

\label{section:smalloscillations}

% periodic motion

\en{If}\ru{Если}
\en{the~statics}\ru{статика}
\en{of a~linear system}\ru{линейной системы}
\en{is described}\ru{описывается}
\en{by equation}\ru{уравнением}~\eqref{staticequilibrium.lineardiscretesystem},
\en{then}\ru{то}
\en{in the dynamics}\ru{в~динамике}
\en{we have}\ru{мы имеем}

\nopagebreak\vspace{-0.4em}\begin{equation}\label{dynamicsoflineardiscretesystem}
\scalebox{0.95}[1]{$\displaystyle \sum_{\smash{k}}$} \hspace{-0.2ex} \left(^{\mathstrut} \hspace{-0.2ex} A_{ik} \hspace{.12ex} \mathdotdotabove{q}_k + C_{ik} \hspace{.12ex} q^k \right) \hspace{-0.4ex}
= \hspace{.1ex} P_{\hspace{-0.1ex}i}(t) \hspace{.1ex} ,
\vspace{-0.1em}\end{equation}

\vspace{-0.25em}\noindent
\en{where}\ru{где}
${A_{ik}}$\en{ is}\ru{\:---} \en{the }\en{symmetric and positive}\ru{симметричная и~положительная} \inquotesx{\en{matrix}\ru{матрица} \en{of kinetic energy}\ru{кинетической энергии}}[.]

\en{Any description}\ru{Любое описание} \en{of oscillations}\ru{колебаний} \en{almost always}\ru{почти всегда} \en{includes}\ru{включает} \en{the term}\ru{термин} \inquotesx{\en{mode}\ru{мода}}[.]
\en{A~mode of vibration}\ru{Мода вибрации} \en{can be defined}\ru{может быть определена} \en{as}\ru{как} \en{a~way of vibrating}\ru{способ вибрирования} \en{or}\ru{или} \en{a~pattern}\ru{паттерн} \en{of vibration}\ru{вибрации}.
\en{A~normal mode}\ru{Нормальная мода} \en{is}\ru{есть} \en{a~pattern}\ru{паттерн} \en{of periodic motion}\ru{периодического движения}, \en{when}\ru{когда} \en{all parts}\ru{все части} \en{of an~oscillating system}\ru{колеблющейся системы} \en{move sinusoidally}\ru{движутся синусоидально} \en{with the same frequency}\ru{одинаковой частотой} \en{and}\ru{и} \en{with a~fixed phase relation}\ru{с~фиксированным соотношением фаз}.
\en{The free motion}\ru{Свободное движение}\ru{,} \en{described by the normal modes}\ru{описываемое нормальными модами}\ru{,} \en{takes place}\ru{происходит} \en{at fixed frequencies}\ru{на фиксированных частотах}\:--- \en{the natural resonant frequencies}\ru{натуральных резонансных частотах} \en{of an~oscillating system}\ru{колеблющейся системы}.

\en{The most generic motion}\ru{Самое общее движение} \en{of an~oscillating system}\ru{колеблющейся системы} \en{is}\ru{есть} \ru{некоторая суперпозиция}\en{some superposition} \en{of normal modes}\ru{нормальных мод} \en{of this system}\ru{этой системы}.%
\footnote{\en{The modes}\ru{Моды} \en{are }\inquotes{\en{normal}\ru{нормальны}} \en{in the sense that}\ru{в~смысле, что} \en{they move independently}\ru{они движутся независимо}, \en{and}\ru{и} \en{an~excitation of one mode}\ru{возбуждение одной моды} \en{will never cause}\ru{никогда не вызовет} \en{a~motion of another mode}\ru{движение другой моды}.
\en{In mathematical terms}\ru{В~математических терминах}\en{,} \en{normal modes}\ru{нормальные моды} \en{are orthogonal to each other}\ru{ортогональны друг другу}.
\en{In music}\ru{В~музыке}\en{,} \en{normal modes}\ru{нормальные моды} \en{of vibrating instruments}\ru{вибрирующих инструментов}~(\en{strings}\ru{струн}, \en{air pipes}\ru{воздушных трубок}, \en{percussion}\ru{перкуссии} \en{and others}\ru{и~других}) \en{are called}\ru{называются} \inquotes{harmonics} \en{or}\ru{или} \inquotesx{overtones}[.]}

\en{A~research}\ru{Изучение} \en{of~an~oscillating system}\ru{колеблющейся системы} \en{most often begins}\ru{чаще всего начинается} \en{with }\ru{с~}\en{ortho\-gonal}\ru{орто\-гональ\-ных}~(\en{normal}\ru{нормальных}) \inquotesx{\en{modes}\ru{мод}}[---] \en{harmonics}\ru{гармоник}, \en{free}\ru{свободных} (\en{without any driving or damping force}\ru{без какой-либо движущей или демпфирующей силы}) \en{sinusoidal oscillations}\ru{синусоидальных колебаний}

\nopagebreak\vspace{-0.25em}\begin{equation*}
q^k \hspace{-0.1ex} (t) \hspace{-0.2ex} = \mathasteriskabove{q}_{\hspace{-0.1ex}k} \operatorname{sin} \omega_k \hspace{.1ex} t
\hspace{.1ex} .
\end{equation*}

\vspace{-0.2em} \noindent
\en{Multipliers}\ru{Множители}~${\mathasteriskabove{q}_{\hspace{-0.1ex}k} \hspace{-0.15ex} = \constant}$\en{ are}\ru{\:---}
\en{ortho\-gonal}\ru{орто\-гональ\-ные}~(\en{normal}\ru{нормальные})
\inquotes{\en{modes}\ru{моды}}
\en{of~oscillation}\ru{колебания},
${\omega_k\hspace{-0.1ex}}$\en{ are}\ru{\:---}
\en{natural}\ru{натуральные}~(\en{resonant}\ru{резонансные}, \en{eigen-}\ru{собственные})
\en{frequencies}\ru{частоты}.
\en{This set}\ru{Этот набор},
\en{dependent on}\ru{зависящий от}
\en{the~structure}\ru{структуры}
\en{of~an~oscillating object}\ru{колеблющегося объекта},
\en{the~materials}\ru{материалов}
\en{and }\ru{и~}\en{the~boundary conditions}\ru{краевых условий},
\en{is found}\ru{находится}
\en{from}\ru{из}
\en{the~eigenvalue problem}\ru{задачи на~собственные значения}

\nopagebreak\vspace{-0.1em}\begin{equation}
\begin{array}{c}
P_{\hspace{-0.1ex}i} \hspace{-0.15ex} = 0
\hspace{.1ex} ,
\:\;
\mathdotdotabove{q}_k \hspace{-0.12ex} = - \hspace{.2ex} \omega_k^2 \hspace{.25ex} \mathasteriskabove{q}_{\hspace{-0.1ex}k} \hspace{-0.1ex} \operatorname{sin} \omega_k \hspace{.1ex} t
\hspace{.1ex} ,
\:\;
\eqref{dynamicsoflineardiscretesystem}
\:\: \Rightarrow
\\[.3em]
%
\Rightarrow \:\,
\scalebox{0.95}[1]{$\displaystyle \sum_{\smash{k}}$} \Bigl( \hspace{-0.2ex} C_{ik} \hspace{-0.1ex} - \hspace{-0.2ex} A_{ik} \hspace{.25ex} \omega_k^2 \hspace{.1ex} \Bigr) \hspace{.12ex}
\mathasteriskabove{q}_{\hspace{-0.1ex}k} \operatorname{sin} \omega_k \hspace{.1ex} t
= 0
\end{array}
\end{equation}

...


\ru{Интеграл}\en{The} Duhamel\en{’s}\ru{’я}\en{ integral}
is a~way of calculating the response
\en{of linear systems}\ru{линейных систем}
to an~arbitrary
time-varying
external
perturbation.

...

