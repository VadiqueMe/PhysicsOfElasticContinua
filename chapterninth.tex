\en{\chapter{Thin\hbox{-}walled rods}}

\ru{\chapter{Тонкостенные стержни}}

\thispagestyle{empty}

\label{chapter:thinwalledrods}

\begin{otherlanguage}{russian}

\section{Вариационный подход}

\lettrine[lines=2, findent=2pt, nindent=0pt]{В}{\hspace{-0.25ex}} \customref[главе~]{chapter:rods} рассматривались стержни с~массивным сечением. Но в~технике широко используются иные стержни~--- тонкостенные, сечения которых представляют собой узкие полоски различного очертания~(уголок, швеллер, двутавр и~др.). Если стержни похожи на~линии~(материальные линии Коссера), то в~тонкостенных стержнях и~сам\'{о} сечение выглядит как~линия. Три размера~--- толщина и~длина сечения, а~также длина стержня~--- имеют различные порядки.

Известны прикладные теории тонкостенных стержней ...

...



\section{Уравнения с малым параметром}

Рассмотрим призматический стержень с~односвязным сечением в~виде тонкой криволинейной полоски постоянной толщины~$h$. Радиус\hbox{-}вектор в~объёме предст\'{а}вим следующим образом:

...



\section{Первый шаг асимптотической процедуры}

\subsection*{Внешнее разложение}

Из~системы

...

\subsection*{Внутреннее разложение вблизи~$s_0$}

Выпишем уравнения для

...

\subsection*{Сращивание}

Стыковка внутреннего и~внешнего разложений

...



\section{Второй шаг}

\subsection*{Внешнее разложение}

Из~системы

...

\subsection*{Внутреннее разложение вблизи~${s \narroweq s_0}$}

Из~общей системы

...

\subsection*{Сращивание}

Поскольку рассматриваются поправочные члены асимптотических разложений

...



\section{Третий шаг}

\subsection*{Внешнее разложение}

Из~системы

....

\subsection*{Внутреннее разложение около~${s \narroweq s_0}$}

Как уж\'{е} отмечалось, внутренние разложения нужны для постановки краевых условий на~концах

...

\subsection*{Сращивание}

В~плоской задаче имеем следующее двучленное внешнее разложение:

....



\section{Четвёртый шаг}

Здесь понадобится лишь внешнее разложение. Более того: в~этом приближении мы не~будем искать решения уравнений~--- будет достаточно лишь условий разрешимости. Напомним, что философия наша такова: разыскиваются лишь главные члены асимптотических разложений, но для~полного их определения могут понадобиться

...



\section{Перемещения}

Расписывая тензорное соотношение

...



\section{Итоги асимптотического анализа}

Определение главных членов асимптотики напряжений и~перемещений для~тонкостенных стержней оказалось намного сложнее, чем в~случае массивного сечения. Дадим сводку полученных выше итоговых результатов.

Перемещение:

...





\vspace{8mm}
\hfill\begin{minipage}[b]{0.95\linewidth}
\fontsize{10}{12}\selectfont

\section*{\wordforbibliography}

Помимо известных книг ... Материал главы содержится в~\cite{eliseev-models}, где можно найти и~обширный список статей.

\end{minipage}

\end{otherlanguage}
