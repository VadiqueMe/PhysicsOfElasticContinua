\en{\section{Perfectly rigid undeformable solid body}}

\ru{\section{Совершенно жёсткое недеформируемое твёрдое тело}}

\begin{changemargin}{.4\textwidth}{\parindent}
\vspace{-1em}
{\noindent\small
\setlength{\parskip}{\spacebetweenparagraphs}

\inquotesx{\en{Absolutely rigid}\ru{Абсолютно жёсткое}}[,]
\en{aka}\ru{оно~же}
\inquotes{\en{absolutely solid}\ru{абсолютно твёрдое}}
\en{and}\ru{и}~\inquotesx{\en{absolutely durable}\ru{абсолютно прочное}}[---]
\en{the~pipe dream}\ru{несбыточная мечта}
\en{of~any engineer}\ru{любого инженера}.

\par}
\vspace{.4em}
\end{changemargin}

% ~ %

\label{section:absolutelyrigidundeformablesolidbody}

\en{\dropcap{O}{ne}}\ru{\dropcap{Е}{щё}}
\en{more}\ru{один}
\en{concept}\ru{концепт},
\en{modeled}\ru{моделируемый}
\en{in~classical}\ru{в~классической}
\en{generic}\ru{общей}
\en{mechanics}\ru{механике},\en{ is}\ru{---}
\href{https://en.wikipedia.org/wiki/Rigid_body}{\en{the~}(\en{perfectly}\ru{совершенно})
\en{rigid}\ru{жёсткое}
%%\en{undeformable}\ru{недеформируемое}
\en{body}\ru{тело}}.
%
\en{That is}\ru{То есть}
\en{a~solid}\ru{твёрдое}\footnote{%
\inquotes{\en{Rigid}\ru{Жёсткое}}
\en{is inelastic}\ru{это неупругое}
\en{and not flexible}\ru{и~не~гибкое},
\en{and}\ru{а}~\inquotes{\en{solid}\ru{твёрдое}}
\en{is not fluid}\ru{это не~текучее}.
\en{A~solid substance}\ru{Твёрдое вещество}
\en{retains}\ru{сохраняет}
\en{its size and shape}\ru{свой размер и~форму}
\en{without a~container}\ru{без контейнера}
(\en{as~opposed to}\ru{в~отличие от}
\en{a~fluid substance}\ru{текучего вещества},
\en{a~liquid}\ru{жидкости}
\en{or}\ru{или}
\en{a~gas}\ru{газа}).%
}\hspace{-0.2ex}
\en{body}\ru{тело},
\en{in which}\ru{в~котором}
\en{deformation}\ru{деформация}
\en{is zero}\ru{нулевая}
(\en{or}\ru{или}
\en{is negligibly small}\ru{пренебрежимо мала}\:--- \
\en{so small}\ru{так мала}\ru{,}
\en{that it can be neglected}\ru{что ею можно пренебречь}).
\en{The~distance}\ru{Расстояние}
\en{between}\ru{между}
\en{any two points}\ru{любыми двумя точками}
\en{of~a~non-deformable}\ru{недеформируемого}
\en{rigid body}\ru{жёсткого тела}
\en{remains constant}\ru{остаётся постоянным}
\en{regardless of external forces exerted on it}\ru{независимо от действующих на~него внешних сил}.

\en{A~non-deformable}\ru{Недеформируемое}
\en{rigid body}\ru{жёсткое тело}
\en{is modeled}\ru{моделируется}\ru{,}
\en{using}\ru{используя}
\en{the~}\inquotesx{\en{continual approach}\ru{континуальный подход}}
\en{as}\ru{как}
\en{a~continuous distribution of~mass}\ru{непрерывное распределение массы}
(\en{a~material continuum, a~continuous medium}\ru{материальный \rucontinuum, сплошная среда}),
\en{rather than using}\ru{вместо использования}
\en{the~}\inquotesx{\en{discrete approach}\ru{дискретного подхода}}
(\en{that is}\ru{то есть}
\en{modeling}\ru{моделирования}
\en{a~body}\ru{т\'{е}ла}
\en{as}\ru{как}
\en{a~discrete collection}\ru{дискретной коллекции}
\en{of particles}\ru{частиц},
\sectionref{section:discreteapproach}).

\en{The~mass}\ru{Масса}
\en{of a~material continuum}\ru{материального \rucontinuum{}а}
\en{is distributed}\ru{распределяется}
\en{continuously}\ru{непрерывно}
\en{throughout its volume}\ru{по~всем\'{у} своем\'{у} объёму},

\nopagebreak\vspace{-1.1em}
\begin{equation}\label{continuousdistributionofmass:materialcontinuum}
dm \equiv \rho \hspace{.2ex} d\mathcal{V}
%%\hspace{.1ex} ,
\end{equation}

\vspace{-0.2em}\noindent
(${\rho\hspace{.1ex}}$\ru{\:---}\en{ is} \en{a~volume(tric) mass density}\ru{объёмная плотность массы} \en{and}\ru{и}~${d \mathcal{V}}$\ru{\:---}\en{ is} \en{an~infinitesimal volume}\ru{бесконечно\-м\'{а}лый объём}).

\en{A~formula}\ru{Формула}
\en{with summation}\ru{с~суммированием}
\en{over}\ru{по}
\en{discrete points}\ru{дискретным точкам}
\en{becomes}\ru{становится}
\en{a~formula}\ru{формулой}
\en{for a~continuous body}\ru{для сплошного тела}
\en{by replacing}\ru{заменой}
\en{the~masses of~particles}\ru{масс частиц}
\en{with the~mass}\ru{на~массу}~\eqref{continuousdistributionofmass:materialcontinuum}
\en{of an~infinitesimal}\ru{бесконечно\-м\'{а}лого}
\en{volume element}\ru{элемента объёма}~${d\mathcal{V}}$
\en{with integration}\ru{с~интегрированием}
\en{over}\ru{по}
\en{the~entire volume}\ru{всему объёму}
\en{of~a~body}\ru{т\'{е}ла}.
\en{In~particular}\ru{В~частности},
\en{here are}\ru{вот}
\en{the~formulas}\ru{формулы}
\en{for}\ru{для}
\en{the~}(\en{linear}\ru{линейного})
\en{momentum}\ru{импульса}

\nopagebreak\vspace{-0.2em}
\begin{equation}\label{thelinearmomentum.discreteandcontinual}
\scalebox{.83}{$ \displaystyle\sum_{\smash{k}} $} \hspace{.3ex} m_k \hspace{.2ex} \mathdotabove{\locationvector}_{\hspace{-0.1ex}k}
\hspace{.66em} \text{\en{becomes}\ru{становится}} \hspace{.55em}
\scalebox{.87}{$ \displaystyle\integral_{\mathcal{V}} $} \hspace{-0.1ex} \mathdotabove{\locationvector} \hspace{.15ex} dm
\end{equation}

\nopagebreak\vspace{-0.3em}\noindent
\en{and for}\ru{и~для}
\en{the~}\en{angular momentum}\ru{углового импульса}

\nopagebreak\vspace{-0.2em}
\begin{equation}\label{therotationalmomentum.discreteandcontinual}
\scalebox{.83}{$ \displaystyle\sum_{\smash{k}} $} \hspace{.3ex} \locationvector_{\hspace{-0.1ex}k} \hspace{-0.2ex} \times \hspace{-0.2ex} m_k \hspace{.2ex} \mathdotabove{\locationvector}_{\hspace{-0.1ex}k}
\hspace{.66em} \text{\en{becomes}\ru{становится}} \hspace{.55em}
\scalebox{.87}{$ \displaystyle\integral_{\mathcal{V}} $} \hspace{-0.1ex} \locationvector \hspace{-0.2ex} \times \hspace{-0.2ex} \mathdotabove{\locationvector} \hspace{.15ex} dm
\hspace{.2ex} .
\end{equation}

\en{To fully describe}\ru{Чтобы полностью опис\'{а}ть}
\en{the~location}\ru{положение}
(\en{position}\ru{позицию},
\en{place}\ru{место})
\en{of any non-deformable body}\ru{любого недеформируемого тела}
\en{with all its points}\ru{со всеми своими точками},
\en{it’s enough}\ru{достаточно}
\en{to choose}\ru{выбрать}
\en{some unique point}\ru{какую\hbox{-}либо уникальную точку}
\en{as}\ru{за}
\en{the~}\inquotesx{\en{pole}\ru{полюс}}[,]
\en{to~find or to~set}\ru{найти или задать}
\en{the~location}\ru{положение}
${\positionofthepole \hspace{.1ex} \narroweq \hspace{.1ex} \positionofthepole(t)}$
\en{of the~chosen point}\ru{выбранной точки},
\en{as well as the~angular orientation}\ru{а~также угловую ориентацию}
\en{of a~body}\ru{тела}
\en{relative}\ru{относительно}
\en{to the~pole}\ru{полюса}~(\figureref{fig:bodyoffsetandrotation}).
\en{As a~result}\ru{Как результат},
\en{any motion}\ru{любое движение}
\en{of an~undeformable rigid body}\ru{недеформируемого твёрдого тела}
\en{is}\ru{есть}
\en{either}\ru{либо}
\en{a~rotation}\ru{поворот}
\en{around the~chosen pole}\ru{вокруг выбранного полюса},
\en{or}\ru{либо}
\en{an~equal displacement}\ru{равное смещение}
\en{of~the~pole}\ru{полюса}
\en{and }\ru{и~}\en{all body’s points}\ru{всех точек тела}\:---
\en{a~translation}\ru{трансляция}~%
(\en{a~linear motion}\ru{линейное движение})%
\footnote{%
\en{A~translation}\ru{Трансляция}
\href{https://en.wikipedia.org/wiki/Instant_centre_of_rotation\#Pure_translation}{%
\en{can also}\ru{может также}
\en{be thought of as}\ru{быть мыслима как}
\en{a~rotation}\ru{вращение}
\en{with the~revolution center}\ru{с~центром переворота}
\en{at infinity}\ru{на~бесконечности}}.%
}\hbox{\hspace{-0.5ex},}
\en{or}\ru{либо}
\en{a~combination of~them both}\ru{комбинация их обоих}.

%%\begin{wrapfigure}{o}{.5\textwidth}
%%\makebox[.45\textwidth][c]{%
%%\begin{minipage}[t]{.45\textwidth}
\begin{figure}[htb!]
\begin{center}
\vspace{-0.2em}
\scalebox{1.1}{
\begin{tikzpicture}[scale=.63]

\def\angleofrotation{44}

\def\Opointx{-1.65}
\def\Opointy{-1.05}
\def\Oinitialpointx{-6} %-5.8
\def\Oinitialpointy{-2.3} %-2.3

\def\bodypointx{-3.5} %-2
\def\bodypointy{2.5} %1.5

\newcommand\drawnotrotatedbasis{
	\draw [line width=1pt, black!50,
		style=double, double distance=0.5mm,
		rotate around={120:(\Opointx, \Opointy)},
		-{Triangle[open, angle=60:3.2mm]}]
		(\Opointx, \Opointy) -- ++(0, 1.6) ;
	\draw [line width=1pt, black!50,
	style=double, double distance=0.5mm, rotate around={-120:(\Opointx, \Opointy)},
	-{Triangle[open, angle=60:3.2mm]}]
		(\Opointx, \Opointy) -- ++(0, 1.6) ;
	\draw [line width=1pt, black!50,
		style=double, double distance=0.5mm, -{Triangle[open, angle=60:3.2mm]}]
		(\Opointx, \Opointy) -- ++(0, 1.6)
		node [pos=.93, above, inner sep=0pt, outer sep=3.5pt]
		{$ \widetilde{\bm{e}}_i $} ;
}

%%\newcommand\setundeformablebody{
%%	\coordinate (point0) at (-4.3, 2.5);
%%	\coordinate (point1) at (-3.1, 3.2);
%%	\coordinate (point2) at (-2, 2.4);
%%	\coordinate (point3) at (-0.4, 1.6);
%%	\coordinate (point4) at (0.5, 0);
%%	\coordinate (point5) at (0, -2);
%%	\coordinate (point6) at (-1.5, -3);
%%	\coordinate (point7) at (-3, -2.2);
%%	\coordinate (point8) at (-3.5, -0.5);
%%	\coordinate (point9) at (-4.5, 1);
%%}

\newcommand\drawnotrotatedundeformablebody{
	\begin{scope}[rotate around={-\angleofrotation:(\Opointx, \Opointy)}]
	\draw [line width=1pt, black!50, opacity=50]
		plot [smooth cycle, tension=0.8] coordinates {
			(-4.3, 2.5) (-3.1, 3.2) (-2, 2.4) (-0.4, 1.6) (0.5, 0)
			(0, -2) (-1.5, -3) (-3, -2.2) (-3.5, -0.5) (-4.5, 1)
		};
	\end{scope}
}

\newcommand\drawrotatedundeformablebody{
	\draw [line width=1.6pt, black]
		plot [smooth cycle, tension=0.8] coordinates {
			(-4.3, 2.5) (-3.1, 3.2) (-2, 2.4) (-0.4, 1.6) (0.5, 0)
			(0, -2) (-1.5, -3) (-3, -2.2) (-3.5, -0.5) (-4.5, 1)
			%% (point0) (point1) (point2) (point3) (point4)
			%% (point5) (point6) (point7) (point8) (point9)
		};
}

\newcommand\drawnotrotatedtorotated{
	\tkzDefPoint(\bodypointx, \bodypointy){bodypointnotrotated}
	\begin{scope}[rotate around={-\angleofrotation:(\Opointx, \Opointy)}]
	\tkzDefPoint(\Opointx, \Opointy){centerpoint}
	\tkzDefPoint(\bodypointx, \bodypointy){bodypoint}
	\tkzDrawArc[line width=1pt, color=black!50, opacity=50](centerpoint,bodypoint)(bodypointnotrotated) ;

	\path (\bodypointx, \bodypointy) circle (2mm) node [shape=circle, inner sep=.9mm, outer sep=0] (previousbodypoint) {};

	\draw [line width=1pt, black!50, opacity=50, -{Stealth[round, length=4.5mm, width=2.8mm]}]
		(\Opointx, \Opointy) -- (previousbodypoint)
		node [pos=0.57, color=black!50, opacity=99, right, inner sep=0pt, outer sep=3.5pt]
		{$ \widetilde{\bm{x}} $} ;

	\fill [white] (\bodypointx, \bodypointy) circle (2mm) ;
	\draw [line width=1pt, color=black!50, opacity=50] (\bodypointx, \bodypointy) circle (2mm) ;
	\end{scope}
}

\newcommand\drawfirstversionvectors{
	\draw [line width=1.6pt, black, fill=white] (\bodypointx, \bodypointy) circle (2mm)
		node [shape=circle, inner sep=0.9mm, outer sep=0] (pointcirc) {} ;

	\draw [line width=1.6pt, black, -{Stealth[round, length=5mm, width=3.6mm]}] (\Oinitialpointx, \Oinitialpointy) -- (pointcirc)
		node [pos=0.5, above left, inner sep=0pt, outer sep=1.5pt] {$ \locationvector $} ;

	\path (\Opointx, \Opointy) circle (1.6mm) node [shape=circle, inner sep=.64mm, outer sep=0] (Ocirc) {} ;

	\draw [line width=1.6pt, blue, -{Stealth[round, length=5mm, width=3.6mm]}] (\Oinitialpointx, \Oinitialpointy) -- (Ocirc)
		node [pos=0.48, below, inner sep=0pt, outer sep=5pt] {$ \positionofthepole $};

	\draw [line width=1.6pt, black, -{Stealth[round, length=5mm, width=3.6mm]}] (\Opointx, \Opointy) -- (pointcirc)
		node [pos=0.63, left, inner sep=2.5pt, outer sep=3.3pt] {$ \bm{x} $} ;
}

\newcommand\drawsecondversionvectors{
	\draw [line width=1.6pt, black, fill=white] (\bodypointx, \bodypointy) circle (2mm)
		node [shape=circle, inner sep=0.9mm, outer sep=0] (pointcirc) {} ;

	\draw [line width=1.6pt, black, -{Stealth[round, length=5mm, width=3.6mm]}] (\Oinitialpointx, \Oinitialpointy) -- (pointcirc)
		node [pos=0.5, above left, inner sep=0pt, outer sep=1.5pt] {$ \locationvector $} ;

	\path (\Oinitialpointx, \Oinitialpointy) circle (1.6mm) node [shape=circle, inner sep=.64mm, outer sep=0] (Oinitialcirc) {} ;

	\draw [line width=1.6pt, blue, -{Stealth[round, length=5mm, width=3.6mm]}] (\Opointx, \Opointy) -- (Oinitialcirc)
		node [pos=0.6, below, inner sep=0pt, outer sep=4.4pt] {$ - \hspace{.2ex} \positionofthepole $};

	\draw [line width=1.6pt, black, -{Stealth[round, length=5mm, width=3.6mm]}] (\Opointx, \Opointy) -- (pointcirc)
		node [pos=0.63, left, inner sep=2.5pt, outer sep=3.3pt] {$ \bm{x} $} ;
}

\newcommand\drawbodybasis{
	\draw [line width=1pt, blue, rotate around={{\angleofrotation + 120}:(\Opointx, \Opointy)},
		style=double, double distance=0.5mm, -{Triangle[open, angle=60:3.2mm]}]
		(\Opointx, \Opointy) -- ++(0, 1.6);
	\draw [line width=1pt, blue, rotate around={{\angleofrotation - 120}:(\Opointx, \Opointy)},
		style=double, double distance=0.5mm, -{Triangle[open, angle=60:3.2mm]}]
		(\Opointx, \Opointy) -- ++(0, 1.6);
 	\draw [line width=1pt, blue, rotate around={\angleofrotation:(\Opointx, \Opointy)},
		style=double, double distance=0.5mm, -{Triangle[open, angle=60:3.2mm]}]
		(\Opointx, \Opointy) -- ++(0, 1.6);

	\draw [line width=1pt, blue, fill=white] (\Opointx, \Opointy) circle (1.6mm)
		node [below right, inner sep=0pt, outer sep=3.5pt, xshift=-.7mm, yshift=-2.5mm] {$ \bm{e}_i $} ;
}

\newcommand\drawhatchlines{
	\def\hatchlength{.3}
	\def\loopfirst{.55}
	\def\looplast{1.15}
	\pgfmathsetmacro\loopstep{(\looplast - \loopfirst) / 2}
	\pgfmathsetmacro\loopsecond{\loopfirst + \loopstep}
	\foreach \econnection in {\loopfirst, \loopsecond, ..., \looplast} {
		\draw [line width=.5pt, color=blue]
			($ (\Oinitialpointx, \Oinitialpointy) + (0, \econnection) $) -- ++(-\hatchlength, -\hatchlength) ;
		\draw [line width=.5pt, color=blue, rotate around={120:(\Oinitialpointx, \Oinitialpointy)}]
			($ (\Oinitialpointx, \Oinitialpointy) + (0, \econnection) $) -- ++(-\hatchlength, -\hatchlength) ;
		\draw [line width=.5pt, color=blue, rotate around={-120:(\Oinitialpointx, \Oinitialpointy)}]
			($ (\Oinitialpointx, \Oinitialpointy) + (0, \econnection) $) -- ++(\hatchlength, -\hatchlength) ;
	}
}

\newcommand\drawabsolutebasis{
	\draw [line width=1pt, blue,
		style=double, double distance=0.5mm, rotate around={120:(\Oinitialpointx, \Oinitialpointy)}, -{Triangle[open, angle=60:3.2mm]}]
		(\Oinitialpointx, \Oinitialpointy) -- ++(0, 1.6);
	\draw [line width=1pt, blue,
		style=double, double distance=0.5mm, rotate around={-120:(\Oinitialpointx, \Oinitialpointy)}, -{Triangle[open, angle=60:3.2mm]}]
		(\Oinitialpointx, \Oinitialpointy) -- ++(0, 1.6);
 	\draw [line width=1pt, blue,
		style=double, double distance=0.5mm, -{Triangle[open, angle=60:3.2mm]}]
		(\Oinitialpointx, \Oinitialpointy) -- ++(0, 1.6);

	\draw [line width=1pt, blue, fill=white] (\Oinitialpointx, \Oinitialpointy) circle(1.6mm)
		node [anchor=north, inner sep=0pt, outer sep=8pt, yshift=-1.1mm, xshift=.33mm]
			{$ \mathcircabove{\bm{e}}_i $};
}

	%%draw undeformable body

	\drawnotrotatedbasis

	\drawnotrotatedundeformablebody

	\drawnotrotatedtorotated

	\drawrotatedundeformablebody

	\drawfirstversionvectors

	\drawbodybasis

	\drawhatchlines
	\drawabsolutebasis

\pgfmathsetmacro\textpositionx{.5 + \Oinitialpointx}
\pgfmathsetmacro\textpositiony{\Oinitialpointy + 4}

\node [anchor=east] at (\textpositionx, \textpositiony)
	{$ \locationvector = \positionofthepole + \bm{x} $} ;

%%\node [align=center] at (\textpositionx, \textpositiony)
%%	{$ \bm{x} = - \hspace{.2ex} \positionofthepole + \locationvector $} ;

\end{tikzpicture}

\begin{comment}
\makeatletter
\newcommand\xofcoordinate[2][center]{{%
	\pgfpointanchor{#2}{#1}%
	\pgfmathparse{\pgf@x/\pgf@xx}%
	\pgfmathprintnumber[precision=2]{\pgfmathresult}%
}}
\newcommand\yofcoordinate[2][center]{{%
	\pgfpointanchor{#2}{#1}%
	\pgfmathparse{\pgf@y/\pgf@yy}%
	\pgfmathprintnumber[precision=2]{\pgfmathresult}%
}}
\makeatother

\begin{tikzpicture}
	\coordinate (point0) at (-4.3, 2.5);
	\coordinate (point1) at (-3.1, 3.2);
	\coordinate (point2) at (-2, 2.4);
	\coordinate (point3) at (-0.4, 1.6);
	\coordinate (point4) at (0.5, 0);
	\coordinate (point5) at (0, -2);
	\coordinate (point6) at (-1.5, -3);
	\coordinate (point7) at (-3, -2.2);
	\coordinate (point8) at (-3.5, -0.5);
	\coordinate (point9) at (-4.5, 1);

	\draw [line width=1.2pt, red] plot [smooth cycle, tension=0.8] coordinates {
		(point0) (point1) (point2) (point3) (point4)
		(point5) (point6) (point7) (point8) (point9)
	};

	\newcommand\xyofcoordinate[1]{\xofcoordinate{#1},\,\yofcoordinate{#1}}

	\draw [black, fill=black] (point0) circle (1mm) node [anchor=south east] {\xyofcoordinate{point0}};
	\draw [black, fill=black] (point1) circle (1mm) node [anchor=south, outer sep=4pt] {\xyofcoordinate{point1}};
	\draw [black, fill=black] (point2) circle (1mm) node [anchor=south west] {\xyofcoordinate{point2}};
	\draw [black, fill=black] (point3) circle (1mm) node [anchor=south west] {\xyofcoordinate{point3}};
	\draw [black, fill=black] (point4) circle (1mm) node [anchor=south west] {\xyofcoordinate{point4}};

	\draw [black,fill=black] (point5) circle (1mm) node [anchor=north west, outer sep=1pt] {\xyofcoordinate{point5}};
	\draw [black,fill=black] (point6) circle (1mm) node [anchor=north, outer sep=4pt] {\xyofcoordinate{point6}};
	\draw [black,fill=black] (point7) circle (1mm) node [anchor=north east, outer sep=2pt] {\xyofcoordinate{point7}};
	\draw [black,fill=black] (point8) circle (1mm) node [anchor=east, outer sep=4pt] {\xyofcoordinate{point8}};
	\draw [black,fill=black] (point9) circle (1mm) node [anchor=east, outer sep=3pt] {\xyofcoordinate{point9}};
\end{tikzpicture}
\end{comment}


}
\end{center}
\vspace{-1.5em}\caption{}\label{fig:bodyoffsetandrotation}
\vspace{-1.1em}\end{figure}
%%\end{minipage}%%}
%%\end{wrapfigure}

${\mathcircabove{\bm{e}}_i}$\:---
\en{the~triplet}\ru{тройка}
\en{of~mutually perpendicular}\ru{взаимно перпендикулярных}
\en{unit vectors}\ru{единичных векторов},
\en{called}\ru{называемых}
\en{the~}\inquotesx{\en{basis vectors}\ru{базисными векторами}}[,]
\en{immovable}\ru{неподвижная}
\en{relatively}\ru{относительно}
\en{to the~absolute}\ru{абсолютной}
(\en{or}\ru{или}
\en{to any inertial}\ru{любой инерциальной})
\en{reference system}\ru{системы отсчёта}

\begin{itemize}
   \item
${\mathcircabove{\bm{e}}_i}$\en{ is}\ru{\:---}
\en{the~}\en{immovable}\ru{неподвижный}~(\en{stationary}\ru{стационарный})
\en{basis}\ru{базис}
   \item
${\bm{e}_i}$\en{ is}\ru{\:---}
\en{the~}\en{basis}\ru{базис}\ru{,}
\en{which}\ru{который}
\en{moves}\ru{движется}
\en{along with the~body}\ru{вместе с~телом}
\end{itemize}

\en{By adding}\ru{Добавив}
\en{the~basis}\ru{базис}~${\bm{e}_i}$
(\en{it}\ru{он}
\en{moves}\ru{движется}
\en{together with the~body}\ru{вместе с~телом}),
\en{the~body’s angular orientation}\ru{угловая ориентация тела}
\en{can be}\ru{может быть}
\en{determined}\ru{определена}
\en{by the~rotation tensor}\ru{тензором поворота}~${\rotationtensor \equiv \bm{e}_i \widetilde{\bm{e}}_i}$.

\en{Then}\ru{Тогда}
\en{any motion}\ru{любое движение}
\en{of a~body}\ru{тела}
\en{is completely described}\ru{полностью описывается}
\en{by two functions}\ru{двумя функциями},
${\positionofthepole(t)}$
\en{and}\ru{и}~${\rotationtensor(t)}$.

\en{The~location vector}\ru{Вектор положения}
\en{of~some body’s point}\ru{некоторой точки тела}

\nopagebreak\vspace{-0.2em}\begin{equation}\label{completelyrigidbody.locationvectorofanypointdecomposed}
\locationvector = \positionofthepole + \bm{x}
%%\hspace{.1ex} ,
\end{equation}

${\widetilde{\bm{x}} = x_i \hspace{.1ex} \widetilde{\bm{e}}_i}$,
${\bm{x} = x_i \hspace{.1ex} \bm{e}_i}$

\eqref{rodriguesrotationformula}, \chapterdotsectionref{chapter:mathapparatus}{section:rotationtensors}

${\bm{x} = \rotationtensor \hspace{-0.15ex} \dotp \hspace{.1ex} \widetilde{\bm{x}}}$

\begin{equation*}
\mathdotabove{\locationvector} = \mathdotabove{\positionofthepole} + \mathdotabove{\bm{x}}
\hspace{.1ex} ,
\end{equation*}

\en{For a~non-deformable rigid body}\ru{Для недеформируемого жёсткого тела},
\en{components}\ru{компоненты}~${x_i}$
\en{don’t depend on time}\ru{не~зависят от времени}:
${x_i \hspace{-0.16ex} = \constant(t)}$
\en{and}\ru{и}~${\mathdotabove{\bm{x}} = x_i \hspace{.1ex} \mathdotabove{\bm{e}}_i}$

${\mathdotabove{\bm{x}} = \mathdotabove{\rotationtensor} \hspace{-0.15ex} \dotp \hspace{.1ex} \mathcircabove{\bm{x}}}$

${x_i \mathdotabove{\bm{e}}_i \hspace{-0.12ex} = \mathdotabove{\rotationtensor} \hspace{-0.15ex} \dotp x_i \hspace{.1ex} \mathcircabove{\bm{e}}_i
\:\Leftrightarrow\:
\mathdotabove{\bm{e}}_i \hspace{-0.12ex} = \mathdotabove{\rotationtensor} \hspace{-0.15ex} \dotp \mathcircabove{\bm{e}}_i}$

...

\en{The~linear momentum}\ru{Линейный импульс~(количество движения)}
\en{and }\ru{и~}\en{the~rotational~(angular) momentum}\ru{угловой импульс~(момент импульса)}
\en{of a~non-deformable continuous body}\ru{недеформируемого сплошного тела}
\en{are described}\ru{описываются}
\en{by the~following integrals}\ru{следующими интегралами}

...

...

\hspace{-0.4ex}\begin{equation*}
\displaystyle\integral_{\mathcal{V}} \hspace{-0.6ex} \positionofthepole \hspace{.2ex} dm
= \hspace{.1ex} \positionofthepole \hspace{-0.5ex} \displaystyle\integral_{\mathcal{V}} \hspace{-0.6ex} dm
= \hspace{.1ex} \positionofthepole \hspace{.2ex} m
\end{equation*}

\begin{equation*}
\hspace{-0.4ex} \displaystyle\integral_{\mathcal{V}} \hspace{-0.6ex} \bm{x} \hspace{.1ex} dm = \hspace{.15ex} \bm{\Xi} \hspace{.2ex} m
\hspace{.1ex} , \:\:
\bm{\Xi} \hspace{.1ex} \equiv m^{\hspace{-0.1ex}\expminusone} \hspace{-0.5ex} \displaystyle\integral_{\mathcal{V}} \hspace{-0.6ex} \bm{x} \hspace{.1ex} dm
\end{equation*}

\en{Three}\ru{Три} \en{inertial characteristics}\ru{инерциальных характеристики} \en{of the~body}\ru{тела}:

\nopagebreak\vspace{.2em}\begin{itemize}
\item \en{integral mass}\ru{интегральная масса} ${m = \hspace{-0.25ex}\scalebox{1.4}{$ \textstyle\integral $}_{\hspace{-0.55ex}\mathcal{V}} \hspace{.3ex} dm = \hspace{-0.25ex}\scalebox{1.4}{$ \textstyle\integral $}_{\hspace{-0.55ex}\mathcal{V}} \hspace{.3ex} \rho \hspace{.2ex} d\mathcal{V}}$\:---
\en{the~mass of the~whole body}\ru{масса всего тела},
\vspace{.2em}
\item \en{eccentricity vector}\ru{вектор экцентриситета}\hbox{~\hspace{.2ex}}$\bm{\Xi}$\:--- \en{measures}\ru{измеряет} \en{the~offset}\ru{смещение} \en{of the chosen pole}\ru{выбранного полюса} \en{from}\ru{от} \en{the body’s }{\inquotes{\en{center of mass}\ru{центра масс}}}\ru{ тела},
%%\en{and}\ru{и}
\vspace{.2em}
\item \en{inertia tensor}\ru{тензор инерции}~${\inertiatensor}$.
\end{itemize}

\en{The eccentricity vector}\ru{Вектор экцентриситета} \en{is equal to}\ru{равняется} \en{the~null vector}\ru{нуль\hbox{-}вектору} \en{only when}\ru{только когда} \en{the chosen pole}\ru{выбранный полюс} \en{coincides with}\ru{совпадает с} \en{the~}\inquotesx{\en{center of~mass}\ru{центром масс}}[---] \en{the~unique point}\ru{уникальной точкой} \en{within a~body}\ru{внутри тела} \en{with }\ru{с~}\en{location vector}\ru{вектором положения}~${\mathboldrcursive\hspace{.2ex}}$, \en{in short}\ru{короче}

\nopagebreak\vspace{-0.2em}\begin{equation*}
\bm{\Xi} = \zerovector
\hspace{.6ex} \Leftrightarrow \hspace{.5ex}
\positionofthepole = \hspace{.1ex} \mathboldrcursive
\hspace{.3ex} .
\end{equation*}

\begin{gather*}
\bm{x} = \hspace{-0.1ex} \locationvector - \positionofthepole
, \hspace{.6em}
\bm{\Xi} \hspace{.2ex} m = \hspace{-0.4ex} \integral_{\mathcal{V}} \hspace{-0.4ex} \bigl( \locationvector - \hspace{-0.1ex} \mathboldrcursive \hspace{.2ex} \bigr) dm = \hspace{.1ex} \zerovector
\hspace{.1ex} ,
\\[-0.4em]
%
\integral_{\mathcal{V}} \hspace{-0.7ex} \locationvector \hspace{.25ex} dm \hspace{.1ex}
- \hspace{.2ex} \mathboldrcursive \hspace{-0.3ex} \integral_{\mathcal{V}} \hspace{-0.7ex} dm = \hspace{.1ex} \zerovector
\hspace{.7ex} \Rightarrow \hspace{.7ex}
\mathboldrcursive = m^{\hspace{-0.1ex}\expminusone} \hspace{-0.5ex} \integral_{\mathcal{V}} \hspace{-0.7ex} \locationvector \hspace{.25ex} dm
\end{gather*}

...

\en{Introducing}\ru{Вводя}
\en{the~}(\en{pseudo}\ru{псевдо})\en{vector}\ru{вектор}
\en{of~angular velocity}\ru{угловой скорости}~$\bm{\omega}$, ...

\nopagebreak\begin{equation*}
\mathdotabove{\bm{e}}_i \hspace{-0.16ex}
= \bm{\omega} \hspace{-0.2ex} \times \hspace{-0.2ex} \bm{e}_i
\end{equation*}

...

\en{inertia tensor}\ru{тензор инерции}~${\inertiatensor}$

\nopagebreak\begin{equation*}
\inertiatensor
\equiv
- \hspace{-0.4ex} \integral_{\mathcal{V}} \hspace{-0.4ex} \bigl( \bm{x} \hspace{-0.1ex} \times \hspace{-0.22ex} \UnitDyad \hspace{.1ex} \bigr) \hspace{-0.3ex} \dotp \hspace{-0.2ex} \bigl( \bm{x} \hspace{-0.1ex} \times \hspace{-0.22ex} \UnitDyad \hspace{.1ex} \bigr) \hspace{.1ex} dm
=
\hspace{-0.4ex} \integral_{\mathcal{V}} \hspace{-0.4ex} \bigl( \bm{x} \narrowdotp \bm{x} \UnitDyad - \bm{x} \bm{x} \bigr) \hspace{.1ex} dm
\end{equation*}

It is assumed \textcolor{magenta}{(can be proven?)} that the~inertia tensor changes only due to a~rotation

\vspace{-0.1em}\begin{equation*}
\inertiatensor = \rotationtensor \hspace{-0.1ex} \dotp \inertiatensorcircabove \dotp \rotationtensor^{\hspace{-0.1ex}\T}
\end{equation*}

\vspace{-0.1em}\noindent
\en{and if}\ru{и~если}
\en{some}\ru{какой-нибудь}
\en{basis}\ru{базисе}~${\bm{e}_{\hspace{-0.1ex}j}}$
\en{is moving}\ru{движется}
\en{along}\ru{вместе}
\en{with the~body}\ru{с~телом},
\en{the~inertia components}\ru{компоненты инерции}
\en{in that basis}\ru{в~том базисе}
\en{don’t change}\ru{не~меняются}
\en{over time}\ru{со временем}

\nopagebreak\vspace{-0.1em}\begin{equation*}
\inertiatensor = \inertiatensorcomponents{ab} \hspace{.1ex} \bm{e}_a \bm{e}_b
\hspace{.1ex} , \hspace{.8ex}
\inertiatensorcomponents{ab} \hspace{-0.2ex} = \constant(t)
\vspace{.1ex} ,
\end{equation*}

\vspace{-0.1em}\noindent
\en{thus}\ru{поэтому}
\en{the~time derivative}\ru{производная по~времени}
\en{is}\ru{есть}

\nopagebreak\vspace{-0.2em}\begin{multline*}
\inertiatensordotabove
= \inertiatensorcomponents{ab} \bigl( \hspace{.1ex}
\mathdotabove{\bm{e}}_a \bm{e}_b \hspace{-0.1ex}
+ \bm{e}_a \mathdotabove{\bm{e}}_b
\hspace{.1ex} \bigr) \hspace{-0.33ex}
= \inertiatensorcomponents{ab} \hspace{-0.1ex} \bigl( \hspace{.1ex}
\bm{\omega} \hspace{-0.2ex} \times \hspace{-0.2ex} \bm{e}_a \bm{e}_b \hspace{-0.1ex}
+ \bm{e}_a \hspace{.2ex} \bm{\omega} \hspace{-0.2ex} \times \hspace{-0.2ex} \bm{e}_b
\hspace{.1ex} \bigr)
\\[-0.1em]
%
= \bm{\omega} \hspace{-0.2ex} \times \hspace{-0.2ex} \inertiatensorcomponents{ab} \hspace{.1ex} \bm{e}_a \bm{e}_b \hspace{-0.1ex}
- \inertiatensorcomponents{ab} \hspace{.1ex} \bm{e}_a \bm{e}_b \hspace{-0.2ex} \times \hspace{-0.2ex} \bm{\omega}
= \bm{\omega} \hspace{-0.1ex} \times \inertiatensor \hspace{.1ex}
- \inertiatensor \times \bm{\omega}
\end{multline*}

\textcolor{magenta}{\en{Substitution of}\ru{Подстановка}}
(....)
\en{into}\ru{в}~\eqref{balanceoftranslationalmomentum.discretepoints}
\en{and}\ru{и}~\eqref{balanceofrotationalmomentum.discretepoints}
\en{gives}\ru{даёт}
%%\en{fundamental}\ru{фундаментальные}
\en{equations}\ru{уравнения}
\en{of~balance}\ru{баланса}\:---
\en{the~balance of linear momentum}\ru{баланс количества движения (линейного импульса)}
\en{and}\ru{и}
\en{the~balance of rotational momentum}\ru{баланс момента импульса (углового импульса}\:---
\en{for}\ru{для}
\en{the~model}\ru{модели}
\en{of~a~continuous non-deformable rigid body}\ru{сплошного недеформируемого жёсткого тела}

...

\noindent
\en{here}\ru{здесь}
$\bm{f}$\en{ is}\ru{\:---} \en{the~external force}\ru{внешняя сила} \en{per mass unit}\ru{на единицу массы},
$\bm{F}$\en{ is}\ru{\:---} \en{the~resultant of external forces}\ru{результанта внешних сил} (\en{also called}\ru{также называемая} \en{the~}\inquotes{\en{equally acting force}\ru{равнодействующей силой}} \en{or}\ru{или} \en{the~}\inquotes{\en{main vector}\ru{главным вектором}}),
$\mathboldM$\en{ is}\ru{\:---} \en{the~resultant of external couples}\ru{результанта внешних пар сил} (\en{the~}\inquotes{\en{main couple}\ru{главная пара}}, \en{the~}\inquotes{\en{main moment}\ru{главный момент}}).

...

