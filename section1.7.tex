\en{\section{The cross product}}

\ru{\section{Векторное произведение}}

\label{section:crossproduct}

%%en{the~}\crossproductinquotes\hbox{-}\en{product}\ru{произведение}

\en{By common notions}\ru{По~привычным представлениям},
\en{the~}\crossproductinquotes\hbox{-}\en{product}\ru{произведение}
(\en{the~}\inquotesx{cross product}[,]
\en{the~}\inquotesx{\en{vector product}\ru{векторное произведение}}[,]
\en{sometimes}\ru{иногда}
\en{the~}\inquotes{oriented area product})
\en{of~two vectors}\ru{двух векторов}
\en{is the~vector}\ru{есть вектор},
\en{heading perpendicular}\ru{идущий перпендикулярно}
\en{to the~plane of~multipliers}\ru{плоскости сомножителей},
\en{whose length}\ru{длина которого}
\en{is equal to}\ru{равна}
\en{the~area}\ru{пл\'{о}\-ща\-ди}
\en{of the parallelogram}\ru{параллелограмма},
\en{spanned by}\ru{охватываемого}
\en{the~multipliers}\ru{сомножителями}

\nopagebreak\vspace{-0.2em}\begin{equation*}
\| \hspace{.33ex} \bm{a} \times \bm{b} \hspace{.4ex} \| \hspace{.1ex}
= \hspace{.1ex} \| \bm{a} \| \hspace{.1em} \| \bm{b} \| \hspace{.1em} \operatorname{sin} \measuredangle (\bm{a}, \bm{b})
\hspace{.1ex} .
\end{equation*}

\vspace{-0.2em}\noindent
\en{However}\ru{Однако},
\en{a~}\crossproductinquotes\hbox{-}\en{product}\ru{произведение}\en{ isn’t quite}\ru{\:--- не~вполн\'{е}}
\en{a~vector}\ru{вектор},
\en{since}\ru{поскольку}
\en{it is not completely invariant}\ru{оно не~полностью инвариантно}.

\vspace{.4em}
\hspace*{-\parindent}\begin{minipage}{\linewidth}
\setlength{\parindent}{\horizontalindent}
\setlength{\parskip}{\spacebetweenparagraphs}

\begin{wrapfigure}[8]{R}{.32\textwidth}
\makebox[.36\textwidth][c]{\hspace{.5em} \begin{minipage}[t]{.36\textwidth}
\vspace{-1.5em}
\hspace{1.25em}\scalebox{.9}{%

\begin{tikzpicture}[scale=.8]

   \draw [line width=1.6pt, black, -{Stealth[round, length=4.5mm, width=3mm]}]
      (0, 0) -- (2.4, -0.8)
      node[ above=1.7mm, xshift=-1.8mm ] {$\bm{a}$} ;
      \draw [ line width=1.6pt, black, -{Stealth[round, length=4.5mm, width=3mm]} ]
	(0, 0) -- (1.6, 0.8)
	node[midway, above=0.8mm] {$\bm{b}$};

\draw [line width=1.77pt, blue, -{Stealth[round, length=4.5mm, width=3mm]}]
	(0, 0) -- (0, -1.93);
\draw [line width=1.77pt, blue, -{Stealth[round, length=4.5mm, width=3mm]}]
	(0, 0) -- (0, -2.2)
	node[pos=0.8, right, inner sep=1pt, outer sep=5pt] {$\bm{c}$};

\draw [line width=1.77pt, blue, -{Stealth[round, length=4.5mm, width=3mm]}]
	(0, 0) -- (0, 1.93);
\draw [line width=1.77pt, blue, -{Stealth[round, length=4.5mm, width=3mm]}]
	(0, 0) -- (0, 2.2)
	node[pos=.82, right, inner sep=1pt, outer sep=5pt] {$\bm{c}$};

\draw [line width=0.5pt, black!50] (2.4, -0.8) -- (4, 0);
\draw [line width=0.5pt, black!50] (1.6, 0.8) -- (4, 0);

\end{tikzpicture}
}\vspace{-1.6em}\caption{}\label{fig:crossproduct}
\end{minipage}}
\end{wrapfigure}

\en{The multipliers}\ru{Сомножители}
\en{of the~}\hbox{ \hspace{-0.2ex}\inquotes{$ {
   \hspace{-0.25ex} \times \hspace{-0.1ex}
} $}\hspace{-0.2ex}-\en{product}\ru{произведения}}
${\bm{c} = \bm{a} \times \bm{b}}$
\en{determine}\ru{определяют}
\en{the~result’s direction}\ru{направление результата}
\en{in space}\ru{в~пространстве},
\en{with an~accuracy}\ru{с~точностью}
\en{up to the sign}\ru{до знака}
\figureref{fig:crossproduct}.

\en{Once you pick}\ru{Как только ты выбираешь}
\en{as}\ru{как}
\en{the positive}\ru{положительное}
\en{the }\inquotes{ \en{right-chiral}\ru{правостороннюю} }
\en{(\inquotes{right-handed})}\ru{}
\en{or}\ru{или}
\en{the }\inquotes{ \en{left-chiral}\ru{левостороннюю} }
\en{(\inquotes{left-handed})}
\en{orientation}\ru{ориентацию}
\en{of space}\ru{пространства},
\en{the one direction from the possible two}\ru{одно направление из двух возможных},
\en{then}\ru{тогда}
\en{the results}\ru{результаты}
\en{of the }\hbox{ \hspace{-0.2ex}\inquotes{${ \hspace{-0.25ex} \times \hspace{-0.1ex} }$ }\hspace{-0.2ex}-\en{products}\ru{произведений}}
\en{become}\ru{становятся}
\en{completely determined}\ru{полностью определёнными}.

\inquotes{\en{The chiral}\ru{Хиральный}}
\en{means}\ru{значит}
\en{asymmetric}\ru{ассиметричный}
\en{in such a~way that}\ru{таким путём, что}
\en{the thing}\ru{вещица}
\en{and}\ru{и}
\en{its mirror image}\ru{её зеркальный образ}
\en{are not superimposable}\ru{не совмещаются},
a~picture cannot be superposed
on its mirror image
by any combination
of rotations and translations.

\en{An object}\ru{Объект}
\en{is chiral}\ru{хирален}\ru{,}
\en{if}\ru{если}
\en{it is distinguishable}\ru{он отлич\'{и}м}
\en{from}\ru{от}
\en{its mirror image}\ru{своего зеркального образа}.

\end{minipage}
\vspace{.2em}

\en{Vectors}\ru{Векторы}
\en{are usually measured}\ru{обычно измеряются}
\en{via some basis}\ru{через какой-нибудь базис}~$\bm{e}_i$.
\en{They}\ru{Они}
\en{are decomposed}\ru{раскладываются}
\en{into linear combinations}\ru{на линейные комбинации}
\en{like}\ru{вида}
${\bm{a} = a_i \bm{e}_i}$.
\en{So}\ru{Так что}
\en{the orientation}\ru{ориентация}
\en{of space}\ru{пространства}
\en{is equivalent}\ru{эквивалентна}
\en{to the orientation}\ru{ориентации}
\en{of the~sequential triple of~basis vectors}\ru{последовательной тройки базисных векторов}
$\bm{e}_1$, $\bm{e}_2$, $\bm{e}_3$.
\en{It}\ru{Это}
\en{means}\ru{означает}\ru{,}
\en{that}\ru{что}
\en{the~sequence of~basis vectors}\ru{последовательность базисных векторов}
\en{becomes}\ru{становится}
\en{significant}\ru{значимой}
(\en{for linear combinations}\ru{для линейных комбинаций}\en{,}
\en{the sequence of addends}\ru{последовательность слагаемых}
\en{doesn’t affect anything}\ru{ни на что не~влияет}).

\en{If}\ru{Если}
\en{two bases}\ru{два базиса}
\en{consist}\ru{состоят}
\en{of different sequences}\ru{из разных последовательностей}
\en{of the~same vectors}\ru{одних и~тех~же векторов},
\en{then}\ru{то}
\en{orientations}\ru{ориентации}
\en{of these bases}\ru{этих базисов}
\en{differ}\ru{отличаются}
\en{by some permutation}\ru{некоторой перестановкой}.

\en{The~orientation}\ru{Ориентация}
\en{of the~space}\ru{пространства}
\en{is}\ru{есть}
\en{a }(\en{kind of}\ru{нечто вроде})
\en{asymmetry}\ru{асимметрии}.
\en{This asymmetry}\ru{Эта асимметрия}
\en{makes it impossible}\ru{делает невозможным}
\en{to replicate mirroring}\ru{повторение зеркалирования}
\en{by the means of}\ru{посредством}
\en{any rotations}\ru{любых вращений}%
\footnote{
   \en{Applying only rotations}\ru{Применяя лишь повороты},
   \en{it’s impossible}\ru{невозможно}
   \en{to replace}\ru{заменить}
   \en{the~left hand}\ru{левую руку}
   \en{with the~right hand}\ru{на правую руку}.
   \en{But}\ru{Но}
   \en{it is possible}\ru{это возможно}
   \en{by mirroring}\ru{зеркалированием}.
}

%%\textcolor{blue}{A~nonsingular linear mapping is orientation-preserving if the determinant of its matrix is positive.}

\begin{figure}[htb!]
\begin{center}

\tdplotsetmaincoords{46}{140} % orientation of camera

\begin{tikzpicture}[scale=2.5, tdplot_main_coords]

\pgfmathsetmacro{\spiralradius}{.69}
\pgfmathsetmacro{\verticaladvance}{.5}

\pgfmathsetmacro{\unitvectorlength}{1.2}

\begin{scope}[xshift=-2.8em]

\coordinate (O) at (0, 0, 0);

% basis vectors
\draw [line width=1.25pt, blue, -{Latex[round, length=3.6mm, width=2.4mm]}]
	(O) -- (\unitvectorlength, 0, 0)
	node[pos=0.9, above, xshift=-0.8em] {${\bm{e}}_2$};
\draw [line width=1.25pt, blue, -{Latex[round, length=3.6mm, width=2.4mm]}]
	(O) -- (0, \unitvectorlength, 0)
	node[pos=0.9, above, xshift=0.2em, yshift=0.1em] {${\bm{e}}_1$};
\draw [line width=1.25pt, blue, -{Latex[round, length=3.6mm, width=2.4mm]}]
	(O) -- (0, 0, \unitvectorlength)
	node[pos=0.9, left, yshift=0.5em] {${\bm{e}}_3$};

\pgfmathsetmacro\verticalshift{-0.1}
\pgfmathsetmacro\initialangle{-35}
\def\lengthincircles{2.15}
\pgfmathsetmacro\endangle{\lengthincircles * 360 + \initialangle}
\def\howmanysamples{120}

\draw[ line width=.8pt ]
	plot [ domain=\initialangle:\endangle, variable=\t, samples=\howmanysamples ]
	( {\spiralradius*sin(\t)}, {\spiralradius*cos(\t)}, {\verticaladvance*\t/360 + \verticalshift} )
	[ arrow inside = {}{0.1, 1} ] ;

\end{scope}

\begin{scope}[xshift=2.8em]

\coordinate (O) at (0, 0, 0);

% basis vectors
\draw [line width=1.25pt, blue, -{Latex[round, length=3.6mm, width=2.4mm]}]
	(O) -- (\unitvectorlength, 0, 0)
	node[pos=0.9, above, xshift=-0.8em] {${\bm{e}}_1$};
\draw [line width=1.25pt, blue, -{Latex[round, length=3.6mm, width=2.4mm]}]
	(O) -- (0, \unitvectorlength, 0)
	node[pos=0.9, above, xshift=0.2em, yshift=0.1em] {${\bm{e}}_2$};
\draw [line width=1.25pt, blue, -{Latex[round, length=3.6mm, width=2.4mm]}]
	(O) -- (0, 0, \unitvectorlength)
	node[pos=0.9, left, yshift=0.5em] {${\bm{e}}_3$};

\pgfmathsetmacro\verticalshift{.02}
\pgfmathsetmacro\initialangle{-125}
\def\lengthincircles{2.15}
\pgfmathsetmacro\endangle{\lengthincircles * 360 + \initialangle}
\def\howmanysamples{120}

\draw[ line width=.8pt ]
	plot [ domain=\initialangle:\endangle, variable=\t, samples=\howmanysamples ]
	( {-\spiralradius*sin(\t)}, {\spiralradius*cos(\t)}, {\verticaladvance*\t/360 + \verticalshift} )
	[ arrow inside = {}{0.1, 1} ] ;

\end{scope}

\end{tikzpicture}

\end{center}
\vspace{-1.2em}\caption{}\label{fig:leftrightspirals}
\vspace{-1.1em}
\end{figure}

%\en{an axial}\ru{аксиальный} \en{vector}\ru{вектор}

\en{A~pseudo\-vector}\ru{Псевдо\-вектор}
\en{is}\ru{это}
\en{a~vector\hbox{-}like}\ru{похожий на~вектор}
\en{object}\ru{объект},
\en{invariant under any rotation}\ru{инвариантный при любом повороте}.
\footnote{\ru{Повороты}\en{Rotations}
\en{cannot change}\ru{не~могут поменять}
\en{the~orientation}\ru{ориентацию}
\en{of a~triple of~basis vectors}\ru{тройки векторов базиса},
\en{it is possible}\ru{это возможно}
\en{only}\ru{лишь}
\en{via mirroring}\ru{через зеркалирование}.
}\hbox{\hspace{-.5ex}.}

\textcolor{magenta}{... put the figure here ...}




\en{Except on rare cases}\ru{Кроме редких случаев},
\en{mirroring}\ru{зеркалирование}
\en{changes}\ru{изменяет}
\en{the~direction}\ru{направление}
\en{of a~fully invariant}\ru{полностью инвариантного}~(\en{polar}\ru{полярного})
\en{vector}\ru{вектора}.

\en{A~pseudovector}\ru{Псевдовектор}
(\en{an axial vector}\ru{аксиальный вектор}),
\en{unlike}\ru{в~отличие от}
\en{a~}\en{polar}\ru{полярного}
\en{vector}\ru{вектора},
\en{doesn’t change}\ru{не~меняет}
\en{the~component}\ru{компоненту}\ru{,}
\en{that is}\ru{ту что}
\en{perpendicular}\ru{перпендикулярна}
\en{to the~mirroring plane}\ru{плоскости зеркалирования},
\en{and}\ru{и}
\en{turns out to be flipped}\ru{оказывается перевёрнутым}
\en{relatively to}\ru{относительно}
\en{the~polar vectors}\ru{полярных векторов}
\en{and}\ru{и}
\en{the~geometry}\ru{геометрии}
\en{of the~entire space}\ru{всего пространства}.
\en{This happens}\ru{Это случается}
\en{because}\ru{из\hbox{-}за того, что}
\en{the~sign}\ru{знак}
(\en{and}\ru{и},
\en{accordingly}\ru{соответственно},
\en{the~direction}\ru{направление})
\en{of each axial vector}\ru{каждого аксиального вектора}
\en{changes}\ru{меняется}
\en{along with changing}\ru{вместе c~изменением}
\en{the~sign}\ru{знака}
\en{of the~}\crossproductinquotes\hbox{-}\en{product}\ru{произведения}\:---
\en{which corresponds}\ru{что соответствует}
\en{to mirroring}\ru{зеркалированию}.

\en{The~otherness}\ru{Инаковость}
\en{of~pseudovectors}\ru{псевдовекторов}
\en{narrows}\ru{сужает}
\en{the~variety}\ru{разнообразие}
\en{of~formulas}\ru{формул}:
\en{a~pseudovector}\ru{псевдовектор}
\en{is not additive}\ru{не~складывается}
\en{with a~vector}\ru{с~вектором}.
\en{The~formula}\ru{Формула}
${\bm{v} = \bm{v}_{\raisemath{-0.1em}{0}} + \hspace{.15ex} \bm{\omega} \hspace{-0.15ex}\times\hspace{-0.15ex} \locationvector}$
\en{from}\ru{из}
\en{an~absolutely rigid undeformable body’s kinematics}\ru{кинематики абсолютно жёсткого недеформируемого тела}
\en{is correct}\ru{корректна},
\en{because}\ru{поскольку}
$\bm{\omega}$~\en{is pseudovector there}\ru{там\:--- псевдовектор},
\en{and}\ru{и}
\en{with the~cross product}\ru{с~векторным произведением}
\en{the~two}\ru{два} \inquotes{\en{pseudo}\ru{псевдо}}
\en{give}\ru{дают}
${(-1)^2 = 1}$,
\en{mutually compensating}\ru{взаимно компенсируя}
\en{each other}\ru{друг друга}.

\en{The~permutations parity tensor}\ru{Тензор чётности перестановок}
\en{is}\ru{это}
\en{the~volumetric}\ru{объёмометрический}
\en{tensor of third complexity}\ru{тензор третьей сложности}

\nopagebreak\vspace{-0.15em}
\begin{equation}\label{permutationsparityintro}
\permutationsparitytensor = \permutationsparitysymbols{i\hspace{-0.1ex}j\hspace{-0.1ex}k} \hspace{.2ex} \bm{e}_i \bm{e}_j \bm{e}_k
\hspace{.1ex}, \:\:
\permutationsparitysymbols{i\hspace{-0.1ex}j\hspace{-0.1ex}k} \hspace{-0.12ex} \equiv \hspace{.1ex} \bm{e}_i \hspace{-0.2ex} \times \hspace{-0.2ex} \bm{e}_j \hspace{-0.1ex} \dotp \hspace{.1ex} \bm{e}_k
\end{equation}

\nopagebreak\vspace{-0.1em}\noindent
\en{with the~components}\ru{с~компонентами}~${\permutationsparitysymbols{i\hspace{-0.1ex}j\hspace{-0.1ex}k}}$\ru{,}
\en{equal to}\ru{равными}
\en{the~}\inquotes{\en{triple}\ru{трой\-ным}}~(\en{the~}\inquotesx{\en{mixed}\ru{смешан\-ным}}[,]
\en{the~}\inquotes{\en{cross\hbox{-}dot}\ru{векторно\hbox{-}скалярным}})
\en{products}\ru{произведениям}
\en{of the~basis vectors}\ru{базисных векторов}.

\textcolor{magenta}{\en{The~absolute value}\ru{Абсолютная величина}~(\en{the~modulus}\ru{модуль})}
\en{of each nonzero component}\ru{всякой ненулевой компоненты}\en{ of}~$\permutationsparitytensor$
\en{is equal to}\ru{\hbox{равна}}
\en{the~volume}\ru{объёму}~${\hspace{-0.25ex}\sqrt{\hspace{-0.33ex}\mathstrut{\textsl{g}}}}$
\en{of~a~parallelopiped}\ru{параллелепипеда}\ru{,}
\en{drew upon a~basis}\ru{натянутого на \hbox{базис}}.
\en{For a~basis}\ru{Для базиса}~${\bm{e}_i}$
\en{of pairwise perpendicular}\ru{попарно перпендикулярных}
\en{one unit long vectors}\ru{векторов единичной длины}
${\hspace{-0.25ex}\sqrt{\hspace{-0.33ex}\mathstrut{\textsl{g}\hspace{.12ex}}} = \hspace{-0.1ex} 1}$.

\en{The tensor}\ru{Тензор}~$\permutationsparitytensor$
\en{is isotropic}\ru{изотропен},
\en{its components}\ru{его компоненты}
\en{are constant}\ru{постоянны}
\en{and}\ru{и}~\en{independent}\ru{независимы}
\en{of any rotations of a~basis}\ru{от любого поворота базиса}.
\en{But}\ru{Но}
\en{mirroring}\ru{зеркалирование}\:---
\en{a~change in}\ru{изменение}
\en{the orientation}\ru{ориентации}
\en{of a~triple of basis vectors}\ru{тройки базисных векторов}
(\en{a~change in}\ru{перемена}
\inquotes{\en{the~direction of screw}\ru{направления винта}})\:---
\en{changes the~sign of}\ru{меняет знак}~$\permutationsparitytensor$,
\en{so}\ru{так что}
\en{this is a~pseudotensor}\ru{это псевдотензор}
(\en{an~axial tensor}\ru{аксиальный тензор}).

\en{If}\ru{Если}
${\bm{e}_1 \hspace{-0.2ex} \times \bm{e}_2 \hspace{-0.15ex} = \bm{e}_3}$
\en{without}\ru{без}
\ru{знака }\en{the~}\inquotes{\en{minus}\ru{минус}}\en{ sign},
\en{then}\ru{то}
\en{the~basis triple}\ru{базисная тройка}~${\bm{e}_i}$
\en{is oriented positively}\ru{ориентирована положительно}.
\en{The~positive orientation}\ru{Положительная ориентация}
(\en{or}\ru{или}
\inquotes{\en{the~positive direction}\ru{положительное направление}})
\en{is chosen}\ru{выбирается}
\en{for different reasons}\ru{по~разным соображениям}
\en{from the~two possible ones}\ru{из двух возможных}~(\figureref{fig:crossproduct}).
\en{For}\ru{Для}
\en{a~positively oriented basis triplet}\ru{положительно ориентированной базисной тройки}\en{,}
\en{the~components}\ru{компоненты}\en{ of}~$\permutationsparitytensor$
\en{are equal to}\ru{равн\'{ы}}
\en{the~permutation parity symbols}\ru{символам чётности перестановки}
${\permutationsparitysymbols{i\hspace{-0.1ex}j\hspace{-0.1ex}k} \hspace{-0.1ex} = \hspace{.1ex} e_{i\hspace{-0.1ex}j\hspace{-0.1ex}k}}$.
\en{And when}\ru{Когда~же}
${\bm{e}_1 \hspace{-0.2ex} \times \bm{e}_2 \hspace{-0.15ex} = - \hspace{.25ex} \bm{e}_3}$,
\en{then}\ru{тогда}
\en{the~basis triple}\ru{базисная тройка}~${\bm{e}_i}$
\en{is oriented negatively}\ru{ориентирована отрицательно},
\en{or}\ru{или}
\inquotesx{\en{mirrored}\ru{зеркально}}[.]
\en{For}\ru{Для}
\en{mirrored triples}\ru{зеркальных троек}
${\permutationsparitysymbols{i\hspace{-0.1ex}j\hspace{-0.1ex}k} \hspace{-0.1ex} = - \hspace{.2ex} e_{i\hspace{-0.1ex}j\hspace{-0.1ex}k}}$
(\en{and}\ru{а}~${e_{i\hspace{-0.1ex}j\hspace{-0.1ex}k} \hspace{-0.12ex} = - \hspace{.33ex} \bm{e}_i \hspace{-0.2ex} \times \hspace{-0.2ex} \bm{e}_j \hspace{-0.1ex} \dotp \hspace{.12ex} \bm{e}_k}$).

\en{With}\ru{С}~\en{the~permutations parity tensor}\ru{тензором чётности перестановок}~$\permutationsparitytensor$
\en{it’s possible}\ru{возможно}
\en{to take the~fresh look}\ru{по\hbox{-}новому взглянуть}
\en{at the~cross}\ru{на~векторное}
\crossproductinquotes\hbox{-}\en{product}\ru{произведение}\::

\nopagebreak\vspace{-0.3em}
\begin{equation*}
\permutationsparitysymbols{i\hspace{-0.1ex}j\hspace{-0.1ex}k} \hspace{-0.15ex} = \hspace{.15ex}
\hspace{.1ex} \bm{e}_i \hspace{-0.2ex} \times \hspace{-0.2ex} \bm{e}_j \hspace{-0.1ex} \dotp \hspace{.12ex} \bm{e}_k
\:\Leftrightarrow\:
\bm{e}_i \hspace{-0.1ex} \times \bm{e}_j \hspace{-0.12ex}
= \permutationsparitysymbols{i\hspace{-0.1ex}j\hspace{-0.1ex}k} \hspace{.2ex} \bm{e}_k
\hspace{.1ex} ,
\end{equation*}\vspace{-1.6em}
\begin{multline}
\bm{a} \mathcolor{blue}{\times} \bm{b} \hspace{.2ex}
= \hspace{.1ex} a_i \bm{e}_i \times b_j \bm{e}_j
= \hspace{.1ex} a_i \hspace{.1ex} b_j \bm{e}_i \hspace{-0.2ex} \times \hspace{-0.2ex} \bm{e}_j
= \hspace{.1ex} a_i \hspace{.1ex} b_j \hspace{.1ex} \permutationsparitysymbols{i\hspace{-0.1ex}j\hspace{-0.1ex}k} \hspace{.2ex} \bm{e}_k =
\\[-0.12em]
\shoveright{= \hspace{.2ex} b_j \hspace{.1ex} a_i \hspace{.1ex} \bm{e}_j \bm{e}_i \dotdotp \permutationsparitysymbols{mnk} \hspace{.2ex} \bm{e}_m \bm{e}_n \bm{e}_k
= \hspace{.2ex} \bm{b} \hspace{.1ex} \bm{a} \hspace{.1ex} \dotdotp \permutationsparitytensor \hspace{.1ex} ,}
\\[-0.1em]
= \hspace{.1ex} a_i \hspace{.1ex} \permutationsparitysymbols{i\hspace{-0.1ex}j\hspace{-0.1ex}k} \hspace{.2ex} \bm{e}_k \hspace{.1ex} b_j
= - \hspace{.2ex} a_i \hspace{.1ex} \permutationsparitysymbols{ikj} \hspace{.2ex} \bm{e}_k \hspace{.1ex} b_j
= \mathcolor{blue}{-} \hspace{.2ex} \bm{a} \hspace{.1ex} \mathcolor{blue}{\dotp \permutationsparitytensor \dotp} \hspace{.12ex} \bm{b}
\hspace{.1ex} .
\end{multline}

\vspace{-0.1em}\noindent
\en{So that}\ru{Так что},
\en{the~cross product}\ru{векторное произведение}
\en{is not}\ru{не~есть}
\en{another new}\ru{ещё одно новое},
\en{entirely distinct operation}\ru{полностью отдельное действие}.
\en{With }\ru{С~}%
\en{the~permutations parity tensor}\ru{тензором чётности перестановок}
\en{it reduces}\ru{оно сводится}
\en{to the~four}\ru{к~четырём}
\en{already described}\ru{уже описанным}~(\sectionref{section:operationswithtensors})
\en{and}\ru{и}
\en{is applicable}\ru{примен\'{и}мо}
\en{to tensors}\ru{к~тензорам}
\en{of any complexity}\ru{любой сложности}.

\inquotes{\en{The cross product}\ru{Векторное произведение}}
\en{is just}\ru{это всего лишь}
\en{the }dot product\:---
\en{the combination}\ru{комбинация}
\en{of multiplication and contraction}\ru{умножения и~свёртки}
(\sectionref{section:operationswithtensors})\:---
\en{involving}\ru{с~участием}
\en{tensor}\ru{тензора}~$\permutationsparitytensor$.
\en{Such combinations}\ru{Такие комбинации}
\en{are possible}\ru{возможны}
\en{with any tensors}\ru{с~любыми тензорами}\::

\nopagebreak\vspace{-0.1em}\begin{equation*}
\begin{array}{c}
\bm{a} \hspace{.1ex} \mathcolor{blue}{\times} {^2\hspace{-0.16em}\bm{B}} = a_i \bm{e}_i \times \somebivalenttensorcomponents{\hspace{-0.1ex}j\hspace{-0.1ex}k} \bm{e}_j \bm{e}_k \hspace{-0.1ex} = \tikzmark{BeginVectorCrossTensor} a_i \somebivalenttensorcomponents{\hspace{-0.1ex}j\hspace{-0.1ex}k} \hspace{.1ex} \permutationsparitysymbols{i\hspace{-0.1ex}jn} \tikzmark{EndVectorCrossTensor} \bm{e}_n \bm{e}_k
= \mathcolor{blue}{-} \hspace{.2ex} \bm{a} \hspace{.15ex} \mathcolor{blue}{\dotp \permutationsparitytensor \dotp} \somebivalenttensor ,
\\[1.6em]
%
{^2\hspace{-0.05ex}\bm{C}} \mathcolor{blue}{\times} \hspace{-0.1ex} \bm{d} \bm{b} = C_{i\hspace{-0.1ex}j} \bm{e}_i \bm{e}_j \hspace{-0.1ex} \times d_p b_q \bm{e}_p \bm{e}_q \hspace{-0.1ex} = \bm{e}_i C_{i\hspace{-0.1ex}j} d_p \tikzmark{BeginTensorCrossTensor} \permutationsparitysymbols{j\hspace{-0.1ex}pk} \tikzmark{EndTensorCrossTensor} \bm{e}_k b_q \bm{e}_q =
\\[1.6em]
\hspace{12.5em} =
- \hspace{.2ex} {^2\hspace{-0.05ex}\bm{C}} \hspace{-0.1ex} \bm{d} \hspace{.12ex} \dotdotp \permutationsparitytensor \hspace{.1ex} \bm{b} \hspace{.2ex} =
\mathcolor{blue}{-} \hspace{.2ex} {^2\hspace{-0.05ex}\bm{C}} \mathcolor{blue}{\dotp \permutationsparitytensor \dotp} \bm{d} \bm{b} ,
\\
\end{array}
\end{equation*}%
\AddUnderBrace[line width=.75pt][0,-0.2ex]%
{BeginVectorCrossTensor}{EndVectorCrossTensor}{${\scriptstyle - a_i \permutationsparitysymbols{in\hspace{-0.1ex}j} \somebivalenttensorcomponents{\hspace{-0.1ex}j\hspace{-0.1ex}k}}$}%
\AddUnderBrace[line width=.75pt][.2ex,-0.2ex]%
{BeginTensorCrossTensor}{EndTensorCrossTensor}{${\scriptstyle \;- \permutationsparitysymbols{pj\hspace{-0.1ex}k} \:=\: - \permutationsparitysymbols{j\hspace{-0.1ex}kp}}$}%
%
\vspace{-0.32em}\begin{equation}\label{iso:twothree}
\UnitDyad \times \hspace{-0.16ex} \UnitDyad = \bm{e}_i \bm{e}_i \times \bm{e}_j \bm{e}_j = \hspace{-0.4ex} \tikzmark{BeginECrossE} - \hspace{-0.2ex} \permutationsparitysymbols{i\hspace{-0.1ex}j\hspace{-0.1ex}k} \bm{e}_i \bm{e}_j \bm{e}_k \tikzmark{EndECrossE} = - \hspace{.2ex} \permutationsparitytensor
\hspace{.1ex} .
\end{equation}
%
\AddUnderBrace[line width=.75pt][.2ex,-0.2ex]%
{BeginECrossE}{EndECrossE}{${\scriptstyle \;\;+\permutationsparitysymbols{i\hspace{-0.1ex}j\hspace{-0.1ex}k} \bm{e}_i \bm{e}_k \bm{e}_j}$}

\vspace{-0.5em}\noindent
\en{The last}\ru{Последнее}
\en{equation}\ru{равенство}
\en{links}\ru{связывает}
\en{the~isotropic tensor}\ru{изотропный тензор}
\en{of the~second complexity}\ru{второй сложности}
\en{and }\ru{и~}\en{the~isotropic tensor}\ru{изотропный тензор}
\en{of the~third complexity}\ru{третьей сложности}.

\en{Generalizing to}\ru{Обобщая на}
\en{all tensors}\ru{все тензоры}
\en{of nonzero complexity}\ru{ненулевой сложности}

\nopagebreak\vspace{-0.2em}\begin{equation}\label{crossproductforanytwotensors}
{^\mathrm{n}\hspace{-0.12ex}\bm{\xi}} \times \hspace{-0.12ex} {^\mathrm{m}\hspace{-0.12ex}\bm{\zeta}}
=
- \hspace{.25ex} {^\mathrm{n}\hspace{-0.12ex}\bm{\xi}} \dotp \permutationsparitytensor \dotp \hspace{-0.1ex} {^\mathrm{m}\hspace{-0.12ex}\bm{\zeta}}
\;\;\:
%
\forall \hspace{.4ex} {^\mathrm{n}\hspace{-0.12ex}\bm{\xi}} ,
\hspace{-0.12ex} {^\mathrm{m}\hspace{-0.12ex}\bm{\zeta}}
\;\;
\forall \hspace{.25ex} n \hspace{-0.25ex} > \hspace{-0.25ex} 0 , \,
m \hspace{-0.25ex} > \hspace{-0.25ex} 0
\hspace{.1ex} .
\end{equation}

\vspace{-0.1em}\noindent
\en{When one of the operands}\ru{Когда один из операндов}\en{ is}\ru{\:---}
\en{the~}\en{unit}\ru{единичный}~(\en{metric}\ru{метрический}) \en{tensor}\ru{тензор},
\en{from}\ru{из}~\eqref{crossproductforanytwotensors}
\en{and}\ru{и}~\eqref{definingpropertyoftheidentitytensor}
\en{for}\ru{для}~${\forall \, {^\mathrm{n}\hspace{-0.12ex}\bm{\Upsilon}}}$ ${\forall \,\mathrm{n \!>\! 0}}$

\nopagebreak\vspace{-0.2em}
\begin{equation}\label{crossproductfortheunitdyadandsometensor}
\begin{array}{c}
\UnitDyad \hspace{.1ex} \times \hspace{-0.1ex} {^\mathrm{n}\hspace{-0.12ex}\bm{\Upsilon}}
= - \hspace{.2ex} \UnitDyad \hspace{.1ex} \dotp \permutationsparitytensor \dotp {^\mathrm{n}\hspace{-0.12ex}\bm{\Upsilon}}
= - \hspace{.2ex} \permutationsparitytensor \dotp {^\mathrm{n}\hspace{-0.12ex}\bm{\Upsilon}} \hspace{-0.15ex},
\\[.2em]
%
{^\mathrm{n}\hspace{-0.12ex}\bm{\Upsilon}} \times \UnitDyad
= - \hspace{.2ex} {^\mathrm{n}\hspace{-0.12ex}\bm{\Upsilon}} \hspace{-0.1ex} \dotp \permutationsparitytensor \dotp \hspace{-0.1ex} \UnitDyad
= - \hspace{.2ex} {^\mathrm{n}\hspace{-0.12ex}\bm{\Upsilon}} \hspace{-0.1ex} \dotp \permutationsparitytensor
\hspace{.2ex} .
\end{array}
\end{equation}

\vspace{-0.1em}
\en{The~cross product}\ru{Векторное произведение}
\en{of two vectors}\ru{двух векторов}
\en{is not~commutative}\ru{не~коммутативно},
\en{but}\ru{но}
\en{is anti\-commutative}\ru{анти\-коммутативно}\::

\nopagebreak
\begin{equation}\label{crossproductoftwovectors}
\begin{array}{c}
\bm{a} \times \bm{b}
\mathcolor{magenta}{%
= \bm{a} \dotp \hspace{-0.1ex}
\left(
   \bm{b}
   \hspace{-0.1ex} \times \hspace{-0.25ex}
   \UnitDyad
   \hspace{.1ex} \right)
= \left(
   \bm{a}
   \hspace{-0.1ex} \times \hspace{-0.25ex}
   \UnitDyad
   \hspace{.1ex} \right)
\hspace{-0.1ex} \dotp \bm{b}%
}
= \tikzmark{HEADabdotdotpermutations} - \hspace{.25ex} \bm{a} \bm{b} \dotdotp \permutationsparitytensor \tikzmark{abdotdotpermutationsTAIL}
= \tikzmark{HEADpermutationsdotdotab} - \hspace{.2ex} \permutationsparitytensor \dotdotp \bm{a} \bm{b} \tikzmark{permutationsdotdotabTAIL}
\hspace{.15ex} ,
\\[.15em]
%
\bm{b} \times \bm{a}
\mathcolor{magenta}{%
= \bm{b} \dotp \hspace{-0.1ex}
\left(
   \bm{a}
   \hspace{-0.1ex} \times \hspace{-0.25ex}
   \UnitDyad \hspace{.1ex} \right)
= \left(
   \bm{b}
   \hspace{-0.1ex} \times \hspace{-0.25ex}
   \UnitDyad
   \hspace{.1ex} \right)
\hspace{-0.1ex} \dotp \bm{a}%
}
= - \hspace{.25ex} \bm{b} \bm{a} \dotdotp \permutationsparitytensor
= - \hspace{.2ex} \permutationsparitytensor \dotdotp \bm{b} \bm{a}
\hspace{.15ex} ,
\\[.2em]
%
\bm{a} \times \bm{b} \hspace{.16ex}
= - \hspace{.33ex} \bm{a} \bm{b} \dotdotp \permutationsparitytensor
= \bm{b} \bm{a} \dotdotp \permutationsparitytensor
\hspace{.5em} \Rightarrow \hspace{.44em}
\bm{a} \times \bm{b} \hspace{.2ex} = - \hspace{.3ex} \bm{b} \times \bm{a}
\hspace{.2ex} .
\end{array}
\end{equation}%
\AddOverBrace[line width=0.75pt][.2ex,0.1ex][yshift=-0.1em]%
{HEADabdotdotpermutations}{permutationsdotdotabTAIL}{${\scriptstyle
   \permutationsparitysymbols{j\hspace{-0.1ex}i\hspace{-0.1ex}k}
      = \hspace{.2ex}
         \permutationsparitysymbols{k\hspace{-0.1ex}j\hspace{-0.1ex}i}
   \hspace{.33ex} \Rightarrow \hspace{.5ex}
   a_i b_{\hspace{-0.1ex}j} \hspace{-0.1ex} \permutationsparitysymbols{j\hspace{-0.1ex}i\hspace{-0.1ex}k} \bm{e}_{k} \hspace{.1ex}
      = \hspace{.3ex}
         \permutationsparitysymbols{k\hspace{-0.1ex}j\hspace{-0.1ex}i} a_i b_{\hspace{-0.1ex}j} \bm{e}_{k}
}$}

\vspace{-0.2em}\noindent
\en{For}\ru{Для}
\en{any}\ru{любого}
\en{bivalent tensor}\ru{бивалентного тензора}~${\hspace{-0.2ex}^2\hspace{-0.16em}\bm{B}}$
\en{and}\ru{и}~\en{any}\ru{любого}
\en{vector}\ru{вектора}~$\bm{a}$

\nopagebreak\vspace{-0.3em}
\begin{multline}\label{commutativity.thecrossproductofbivalentandvector}
\somebivalenttensor \hspace{-0.1ex} \times \bm{a}
= \bm{e}_i \somebivalenttensorcomponents{i\hspace{-0.1ex}j} \bm{e}_{\hspace{-0.1ex}j} \hspace{-0.25ex} \times \hspace{-0.1ex} a_k \bm{e}_k \hspace{-0.2ex}
= - \hspace{.25ex} \bm{e}_i \somebivalenttensorcomponents{i\hspace{-0.1ex}j} \hspace{.2ex} a_k \bm{e}_k \hspace{-0.25ex} \times \hspace{-0.1ex} \bm{e}_{\hspace{-0.1ex}j}
\\[-0.1em]
= \bigl( \hspace{-0.1ex} - \hspace{.25ex} a_k \bm{e}_k \hspace{-0.25ex} \times \hspace{-0.1ex} \bm{e}_{\hspace{-0.1ex}j} \somebivalenttensorcomponents{i\hspace{-0.1ex}j} \bm{e}_i \hspace{.16ex} \bigr)^{\hspace{-0.25ex}\T} \hspace{-0.4ex}
= - \hspace{.1ex} \bigl( \hspace{.12ex} \bm{a} \times \hspace{-0.2ex} \somebivalenttensortransposed \hspace{.1ex} \bigr)^{\hspace{-0.25ex}\T}
\hspace{-0.4ex} ,
\hspace{1em}
\end{multline}

\noindent
\en{and only}\ru{и~лишь}
\en{for}\ru{для}
\en{the~unit dyad}\ru{единичной диады} %%~${\hspace{-0.2ex}\UnitDyad}$
\en{and}\ru{и}~\en{a~vector}\ru{вектора}\en{,}
\en{the~}\crossproductinquotes\hbox{-}\en{product}\ru{произведение}
\en{is commutative}\ru{коммутативно}

\nopagebreak\vspace{.6em}
\begin{equation}\label{commutativity.unitdyad-cross-vector}
- \hspace{.25ex} \UnitDyad \times \bm{a}
= \hspace{-0.2em} \tikzmark{HEADPermutationsparityDotVector} \hspace{.2em}
     \permutationsparitytensor \dotp \bm{a}
\hspace{.2em} \tikzmark{PermutationsparityDotVectorTAIL} \hspace{-0.2em}
= \hspace{-0.2em} \tikzmark{HEADVectorDotPermutationsparity} \hspace{.2em}
     \bm{a} \dotp \permutationsparitytensor
\hspace{.2em} \tikzmark{VectorDotPermutationsparityTAIL} \hspace{-0.2em}
= \hspace{.1ex} - \hspace{.3ex} \bm{a} \times \hspace{-0.25ex} \UnitDyad
\hspace{.1ex} ,
\end{equation}%
\AddOverBrace[line width=0.75pt][-0.1ex,0.1ex][xshift=-2.1em,yshift=-0.1em]%
{HEADPermutationsparityDotVector}{VectorDotPermutationsparityTAIL}{${\scriptstyle
   \permutationsparitysymbols{i\hspace{-0.1ex}j\hspace{-0.1ex}k}
      = \hspace{.2ex}
         \permutationsparitysymbols{ki\hspace{-0.1ex}j}
   \hspace{.33ex} \Rightarrow \hspace{.5ex}
   \permutationsparitysymbols{i\hspace{-0.1ex}j\hspace{-0.1ex}k} a_k \bm{e}_{i} \bm{e}_{\hspace{-0.1ex}j} \hspace{.1ex}
      = \hspace{.3ex}
         a_k \hspace{-0.1ex} \permutationsparitysymbols{ki\hspace{-0.1ex}j} \bm{e}_{i} \bm{e}_{\hspace{-0.1ex}j}
}$}

\vspace{-1.5em}\noindent
\en{plus}\ru{плюс}
\en{as}\ru{как}
\en{the~particular case}\ru{частный случай}\en{ of}~\eqref{commutativity.thecrossproductofbivalentandvector}

\nopagebreak\vspace{-0.25em}
\begin{equation}\label{theparticularcasefortheunitdyadwithvector}
\UnitDyad \times \bm{a}
= - \hspace{.25ex} \bigl( \hspace{.1ex} \bm{a} \times \hspace{-0.25ex} \UnitDyad^{\hspace{.1ex}\T} \hspace{.1ex} \bigr)^{\hspace{-0.2ex}\T} \hspace{-0.4ex}
= - \hspace{.25ex} \bigl( \bm{a} \times \hspace{-0.25ex} \UnitDyad \hspace{.15ex} \bigr)^{\hspace{-0.2ex}\T} \hspace{-0.4ex}
%%= \bm{a} \times \hspace{-0.25ex} \UnitDyad
.
\end{equation}

% the double cross product

\en{The~first}\ru{Первая}
\en{of}\ru{из~формул}~\eqref{doublepermutationscontracted}\en{~formulas}
\en{gives}\ru{даёт}
\en{the~following}\ru{следующее}
\en{representation}\ru{представление}
\en{for}\ru{для}
\en{the~double}\ru{двойного}
\crossproductinquotes\hbox{-}\en{product}\ru{произведения}

\nopagebreak\vspace{-0.5em}
\begin{multline}
\bm{a} \times \hspace{-0.15ex} \left(\hspace{.2ex}{\bm{b} \times \bm{c}}\hspace{.15ex}\right)
= a_i \bm{e}_i \times \permutationsparitysymbols{pqj} \hspace{.2ex} b_p c_q \bm{e}_j \hspace{-0.2ex}
= \permutationsparitysymbols{ki\hspace{-0.1ex}j} \permutationsparitysymbols{pqj} \hspace{.2ex} a_i b_p c_q \bm{e}_k \hspace{-0.2ex} =
\\
= \left({\delta_{kp}\delta_{iq} \hspace{-0.2ex} - \delta_{kq}\delta_{ip}}\right) a_i b_p c_q \bm{e}_k \hspace{-0.2ex}
= a_i b_k c_i \bm{e}_k \hspace{-0.15ex} - a_i b_i c_k \bm{e}_k \hspace{-0.2ex} =
\\
= \bm{a} \dotp \bm{c} \bm{b} - \bm{a} \dotp \bm{b} \bm{c}
= \bm{a} \dotp \hspace{-0.1ex} \bigl( \bm{c} \bm{b} - \bm{b} \bm{c} \hspace{.15ex} \bigr) \hspace{-0.3ex}
= \bm{a} \dotp \bm{c} \bm{b} - \bm{c} \bm{b} \dotp \bm{a}
%%= \bm{b} \bm{a} \dotp \bm{c} - \bm{c} \bm{a} \dotp \bm{b}
\hspace{.1ex} .
\end{multline}

\vspace{-0.1em}\noindent
\en{By~another}\ru{По~другой}
\en{interpretation}\ru{интерпретации},
\en{the~}dot product
\en{of~a~dyad}\ru{диады}
\en{and}\ru{и}~\en{a~vector}\ru{вектора}
\en{is not~commutative}\ru{не~коммутативен}\::
${\bm{b} \bm{d} \hspace{.1ex} \dotp \bm{c} \hspace{.25ex} \neq \hspace{.25ex} \bm{c} \hspace{.1ex} \dotp \bm{b} \bm{d}}$,
\en{and }\ru{и~}\en{the~difference}\ru{разница}
\en{can be}\ru{может быть}
\en{rendered}\ru{изображена}
\en{as}\ru{как}

\nopagebreak\vspace{-1.2em}
\begin{equation}
\bm{b} \bm{d} \hspace{.1ex} \dotp \bm{c} \hspace{.25ex} - \hspace{.25ex} \bm{c} \hspace{.1ex} \dotp \bm{b} \bm{d}
\hspace{.25ex} = \hspace{.25ex}
\bm{c} \times \hspace{-0.2ex} \bigl( \hspace{.1ex} \bm{b} \times \bm{d} \hspace{.25ex} \bigr)
\hspace{-0.1ex} .
\end{equation}

\noindent ${
\bm{a} \dotp \bm{b} \bm{c} = \bm{c} \bm{b} \dotp \bm{a} = \bm{c} \bm{a} \dotp \bm{b} = \bm{b} \dotp \bm{a} \bm{c}
}$

\noindent ${
\left(\hspace{.1ex} \bm{a} \times \bm{b} \hspace{.2ex}\right) \hspace{-0.2ex} \times \bm{c}
= - \hspace{.33ex} \bm{c} \times \hspace{-0.15ex} \left(\hspace{.2ex}{\bm{a} \times \bm{b}}\hspace{.15ex}\right) \hspace{-0.1ex}
= \bm{c} \times \hspace{-0.2ex} \left(\hspace{.2ex}{\bm{b} \times \bm{a}}\hspace{.15ex}\right)
}$

\vspace{-0.15em}\noindent
\textcolor{magenta}{\en{The~same way}\ru{Тем~же путём}
\en{it may~be derived that}\ru{выводится}}

\nopagebreak\vspace{-0.2em}
\begin{equation}
\left(\hspace{.1ex} \bm{a} \times \bm{b} \hspace{.2ex}\right) \hspace{-0.2ex} \times \bm{c}
= \hspace{-0.15ex} \bigl( \hspace{.1ex} \bm{b} \bm{a} - \bm{a} \bm{b} \hspace{.15ex} \bigr) \hspace{-0.3ex} \dotp \bm{c}
= \bm{b} \bm{a} \dotp \bm{c} - \bm{a} \bm{b} \dotp \bm{c}
\hspace{.2ex} .
\end{equation}

\vspace{-0.15em}\noindent
\en{And }\ru{И~}\en{the~following}\ru{следующие}
\en{identities}\ru{тождества}
\en{for}\ru{для}
\en{any two}\ru{любых двух}
\en{vectors}\ru{векторов}~$\bm{a}$
\en{and}\ru{и}~$\bm{b}$

\nopagebreak\vspace{-0.5em}
\begin{multline}\label{vectorcrossvectorcrossidentity}
\bigl( \bm{a} \times \bm{b} \hspace{.15ex}\bigr) \hspace{-0.33ex} \times \hspace{-0.22ex} \UnitDyad
= \permutationsparitysymbols{i\hspace{-0.1ex}j\hspace{-0.1ex}k} \hspace{.2ex} a_i \hspace{.1ex} b_j \hspace{.1ex} \bm{e}_k \hspace{-0.15ex} \times \hspace{-0.1ex} \bm{e}_n \bm{e}_n \hspace{-0.2ex}
= a_i \hspace{.1ex} b_j \hspace{.1ex} \permutationsparitysymbols{i\hspace{-0.1ex}j\hspace{-0.1ex}k} \permutationsparitysymbols{knq} \hspace{.2ex} \bm{e}_q \bm{e}_n \hspace{-0.2ex} =
\\[-0.1em]
%
= a_i b_j \bigl( \delta_{in} \delta_{j\hspace{-0.1ex}q} \hspace{-0.22ex} - \delta_{iq} \delta_{jn} \bigr) \bm{e}_q \bm{e}_n \hspace{-0.2ex}
= a_i b_j \bm{e}_j \bm{e}_i \hspace{-0.12ex} - a_i b_j \bm{e}_i \bm{e}_j \hspace{-0.2ex} =
\\[-0.1em]
%
= \bm{b} \bm{a} - \bm{a} \bm{b}
\hspace{.2ex},
\end{multline}

\vspace{-1.1em}\begin{multline}\label{vectorcrossidentitydotvectorcrossidentity}
\bigl( \bm{a} \hspace{-0.1ex}\times\hspace{-0.25ex} \UnitDyad \hspace{.2ex}\bigr) \hspace{-0.25ex} \dotp \hspace{-0.15ex} \bigl(\hspace{.1ex} \bm{b} \hspace{-0.1ex}\times\hspace{-0.25ex} \UnitDyad \hspace{.2ex}\bigr) \hspace{-0.2ex}
= \bigl( \bm{a} \hspace{.1ex} \dotp \hspace{-0.1ex} \permutationsparitytensor \hspace{.1ex}\bigr) \hspace{-0.25ex} \dotp \hspace{-0.15ex} \bigl(\hspace{.1ex} \bm{b} \hspace{.1ex} \dotp \hspace{-0.1ex} \permutationsparitytensor \hspace{.1ex}\bigr) \hspace{-0.2ex} =
\\[-0.1em]
%
= a_i \permutationsparitysymbols{ipn} \hspace{.1ex} \bm{e}_p \bm{e}_n \hspace{-0.15ex} \dotp \hspace{.2ex} b_j \permutationsparitysymbols{jsk} \hspace{.1ex} \bm{e}_s \bm{e}_k \hspace{-0.2ex}
= a_i b_j \permutationsparitysymbols{ipn} \permutationsparitysymbols{nkj} \hspace{.05ex} \bm{e}_p \bm{e}_k \hspace{-0.2ex} =
\\[-0.1em]
%
= a_i b_j \bigl( \delta_{ik} \delta_{pj} \hspace{-0.22ex} - \delta_{i\hspace{-0.1ex}j} \delta_{pk} \bigr) \bm{e}_p \bm{e}_k \hspace{-0.2ex}
= a_i b_j \bm{e}_j \bm{e}_i \hspace{-0.12ex} - a_i b_i \bm{e}_k \bm{e}_k \hspace{-0.2ex} =
\\[-0.1em]
%
= \hspace{.1ex} \bm{b} \bm{a} - \bm{a} \hspace{-0.1ex} \dotp \bm{b} \hspace{.1ex} \UnitDyad
\hspace{.1ex} .
\end{multline}

\vspace{-0.3em}
\en{Finally}\ru{Наконец},
\en{the~direct}\ru{прямое}
\en{relation}\ru{отношение}
\en{between}\ru{между}
\en{the isotropic tensors}\ru{изотропными тензорами}
\en{of the second and third}\ru{второй и~третьей}
\en{complexities}\ru{сложностей}

\nopagebreak\vspace{-0.2em}\begin{equation}\label{thedirectrelationoftheunitdyadwiththepermutationsparity}
\permutationsparitytensor \hspace{.2ex} \dotdotp \permutationsparitytensor
= \bm{e}_i \permutationsparitysymbols{i\hspace{-0.1ex}j\hspace{-0.1ex}k} \permutationsparitysymbols{kjn} \bm{e}_n \hspace{-0.2ex}
= - \hspace{.2ex} 2 \hspace{.2ex} \delta_{in} \hspace{.12ex} \bm{e}_i \bm{e}_n \hspace{-0.2ex}
= - \hspace{.1ex} 2 \hspace{.1ex} \UnitDyad
\hspace{.1ex} .
\end{equation}
