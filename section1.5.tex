\en{\section{Polyadic representation (decomposition)}}

\ru{\section{Полиадное представление (разложение)}}

\label{section:polyadicrepresentation}

\en{Before in}\ru{Ранее в}~\sectionref{section:tensoranditscomponents}\en{,}
\en{a~tensor}\ru{тензор}
\en{was presented}\ru{был представлен}
\en{as}\ru{как}
\en{some}\ru{некий}
\en{invariant object}\ru{инвариантный объект},
\en{revealing itself}\ru{проявляющий себя}
\en{in every basis}\ru{в~каждом базисе}
\en{as a~collection}\ru{совокупностью}
\en{of numbers~(components)}\ru{чисел~(компонент)}.
\hbox{\en{Such}\ru{Такое}}
\en{a~presentation}\ru{представление}
\en{is typical}\ru{типично}
\en{for}\ru{для}
\en{the~majority of~books}\ru{большинства книг}
\en{about tensors}\ru{о~тензорах}.
\en{The~index notation}\ru{Индексная запись}
\en{can be convenient}\ru{может быть удобна},
\en{especially}\ru{особенно}
\en{when}\ru{когда}
\en{only rectangular coordinates are used}\ru{используются только прямоугольные координаты},
\en{but}\ru{но}
\en{quite often}\ru{весьма часто}
\en{it is not}\ru{это не~так}.
\en{And }\ru{И~}\en{the~relevant case}\ru{подходящий случай}\en{ is}\ru{\:---}
\emph{\en{physics of~continua, elastic and not very elastic}\ru{физика \rucontinuum{}ов, упругих и~не~очень упругих}}\::
\en{it needs}\ru{ей нужен}
\en{more elegant}\ru{более изящн\hbox{ь\kern-0.066emi}й},
\en{more powerful}\ru{более мощный}
\en{and}\ru{и}~\en{perfect}\ru{совершенный}
\en{apparatus}\ru{аппарат}
\en{of the~direct tensor calculus}\ru{прямого тензорного исчисления},
\en{operating with indexless invariant objects}\ru{оперирующий с~безындексными инвариантными объектами}.

\en{Linear combination}\ru{Линейная комбинация} ${\bm{v} = v_i \bm{e}_i}$ \en{from}\ru{из} \en{decomposition}\ru{разложения}~\eqref{vectorcomponents} \en{connects}\ru{соединяет} \en{vector}\ru{вектор}~${\bm{v}}$ \en{with basis}\ru{с~базисом}~${\bm{e}_i}$ \en{and vector’s components}\ru{и~компонентами}~${v_i}$ \en{in that basis}\ru{вектора в~том базисе}.
\en{Soon}\ru{Вскоре} \en{we will get a~similar relation for a~tensor of any complexity}\ru{мы получим похожее соотношение для~тензора любой сложности}.

\en{Any bivalent tensor}\ru{Любой бивалентный тензор}~$\somebivalenttensor$
\en{has}\ru{имеет} \en{nine components}\ru{девять компонент}~$\somebivalenttensorcomponents{i\hspace{-0.1ex}j}$
\en{in~each basis}\ru{в~каждом базисе}.
\en{The~number}\ru{Число} \en{of~various dyads}\ru{различных диад}~${\bm{e}_i \bm{e}_{\hspace{-0.15ex}j}}$ \en{for the~same basis}\ru{для одного и~того~же базиса}\ru{\:---}\en{ is} \en{nine}\ru{тоже девять}~($3^2$)\en{ too}.
\en{Linear combining}\ru{Линейное комбинирование} \en{these dyads}\ru{этих диад} \en{with coefficients}\ru{с~коэффициентами}~$\somebivalenttensorcomponents{i\hspace{-0.1ex}j}$
\en{gives the~sum}\ru{даёт сумму}
${ \somebivalenttensorcomponents{i\hspace{-0.1ex}j} \bm{e}_i \bm{e}_{\hspace{-0.15ex}j} }$.
\en{This is}\ru{Это}
\en{tensor}\ru{тензор},
\en{but}\ru{но}
\en{what are its components}\ru{каков\'{ы} его компоненты},
\en{and}\ru{и}
\en{how}\ru{как}
\en{this representation}\ru{это представление}
\en{changes}
\en{or}\ru{или}
\en{doesn’t change}\ru{не~меняется}
\en{with a~rotation of~basis}\ru{с~поворотом базиса}?

\en{Components of the~constructed sum}\ru{Компоненты построенной суммы}

\nopagebreak\vspace{-0.25em}\begin{equation*}
\bigl( \somebivalenttensorcomponents{i\hspace{-0.1ex}j} \hspace{.1ex} \bm{e}_i \bm{e}_{\hspace{-0.1ex}j} \bigr)_{\hspace{-0.2ex}p\hspace{.1ex}q} \hspace{-0.3ex}
= \hspace{-0.1ex} \somebivalenttensorcomponents{i\hspace{-0.1ex}j} \hspace{.1ex} \delta_{ip} \delta_{\hspace{-0.15ex}j\hspace{-0.1ex}q} \hspace{-0.2ex}
= \hspace{-0.1ex} \somebivalenttensorcomponents{pq}
\end{equation*}

\vspace{-0.25em}\noindent
\en{are components}\ru{суть компоненты}
\en{of~tensor}\ru{тензора}~$\somebivalenttensor$.
\en{And}\ru{И}
\en{with a~rotation}\ru{с~поворотом}
\en{of~basis}\ru{базиса}

\nopagebreak\vspace{-0.25em}
\begin{equation*}
\scalebox{.99}{$
   B\hspace{.16ex}'_{\hspace{-0.32ex}i\hspace{-0.1ex}j}
   \bm{e}'_{i}
   \bm{e}'_{\hspace{-0.1ex}j} \hspace{-0.2ex}
   = \cosinematrix{i'\hspace{-0.1ex}p}
   \cosinematrix{j'\hspace{-0.1ex}q}
   \somebivalenttensorcomponents{pq}
   \cosinematrix{i'\hspace{-0.1ex}n}
   \bm{e}_n
   \cosinematrix{j'\hspace{-0.1ex}m}
   \bm{e}_m \hspace{-0.2ex}
   = \delta_{\hspace{-0.1ex}pn}
   \delta_{\hspace{-0.1ex}qm}
   \somebivalenttensorcomponents{pq} \bm{e}_n \bm{e}_m \hspace{-0.2ex}
   = \somebivalenttensorcomponents{pq} \bm{e}_p \bm{e}_q
$}
\hspace{.1ex} .
\end{equation*}

\en{Doubts are dropped}\ru{Сомнения отпали}:
\en{a~tensor of~second complexity}\ru{тензор второй сложности}
\en{can be~(re)pre\-sent\-ed}\ru{может быть представлен}
\en{as}\ru{как}
\en{the~linear combination}\ru{линейная комбинация}

\nopagebreak\vspace{-0.2em}\begin{equation}\label{polyada:2}
\somebivalenttensor = \somebivalenttensorcomponents{i\hspace{-0.1ex}j} \hspace{.1ex} \bm{e}_i \bm{e}_{\hspace{-0.1ex}j}
\end{equation}

\vspace{-0.25em}\nopagebreak\noindent
--- \en{the~dyadic decomposition}\ru{диадное разложение} \en{of~}\en{a~bivalent tensor}\ru{бивалентного тензора}.

\vspace{-0.2em} \en{For}\ru{Для} \en{the~unit tensor}\ru{единичного тензора}

\nopagebreak\vspace{-0.2em}\begin{equation*}
\UnitDyad =
E_{i\hspace{-0.1ex}j} \hspace{.16ex} \bm{e}_i \bm{e}_{\hspace{-0.1ex}j} \hspace{-0.2ex} =
\delta_{i\hspace{-0.1ex}j} \hspace{.16ex} \bm{e}_i \bm{e}_{\hspace{-0.1ex}j} \hspace{-0.2ex} =
\bm{e}_i \bm{e}_i \hspace{-0.2ex} =
\bm{e}_1 \bm{e}_1 \hspace{-0.2ex} + \bm{e}_2 \bm{e}_2 \hspace{-0.2ex} + \bm{e}_3 \bm{e}_3
\hspace{.1ex} ,
\end{equation*}

\vspace{-0.2em}\nopagebreak\noindent
\en{that’s why}\ru{вот почему}
${\hspace{-0.15ex}\UnitDyad}$
\en{is called}\ru{называется}
\en{the~unit dyad}\ru{единичной диадой}.

\en{Polyadic representations}\ru{Полиадные представления} \en{like}\ru{типа}~\eqref{polyada:2} \en{help}\ru{помогают} \en{to~operate}\ru{оперировать}
\en{with}\ru{с}
\en{the tensors}\ru{тензорами}
\en{much}\ru{намного}
\en{easier}\ru{проще}:

\nopagebreak\vspace{-0.2em}\begin{equation*}
\bm{v} \dotp \somebivalenttensor =
v_i\bm{e}_i
\hspace{-0.1ex} \dotp
\bm{e}_{\hspace{-0.1ex}j}
\somebivalenttensorcomponents{\hspace{-0.2ex}j\hspace{-0.1ex}k}
\bm{e}_k \hspace{-0.2ex}
= v_i
\delta_{i\hspace{-0.1ex}j}
\somebivalenttensorcomponents{\hspace{-0.2ex}j\hspace{-0.1ex}k} \hspace{.1ex}
\bm{e}_k \hspace{-0.2ex}
= v_i
\somebivalenttensorcomponents{ik}
\bm{e}_k
\hspace{.1ex} ,
\end{equation*}

\nopagebreak\vspace{-0.2em}
\begin{equation}\label{tensorcomponents}
\bm{e}_i
\hspace{-0.1ex} \dotp \hspace{-0.1ex}
\somebivalenttensor
\dotp
\bm{e}_{\hspace{-0.1ex}j} \hspace{-0.2ex}
=
\bm{e}_i
\hspace{-0.1ex} \dotp
\somebivalenttensorcomponents{pq}\bm{e}_p\bm{e}_q
\hspace{-0.1ex} \dotp
\bm{e}_{\hspace{-0.1ex}j} \hspace{-0.2ex}
= \hspace{-0.1ex}
\somebivalenttensorcomponents{pq} \hspace{.1ex}
\delta_{\hspace{-0.1ex}ip}
\delta_{\hspace{-0.1ex}qj} \hspace{-0.2ex}
= \hspace{-0.1ex}
\somebivalenttensorcomponents{i\hspace{-0.1ex}j} \hspace{-0.2ex}
=
\somebivalenttensor
\hspace{-0.1ex} \dotdotp
\bm{e}_{\hspace{-0.1ex}j}
\bm{e}_i
\hspace{.1ex} .
\end{equation}
