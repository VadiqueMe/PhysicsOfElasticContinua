\en{\section{Polyadic representation (decomposition)}}

\ru{\section{Полиадное представление (разложение)}}

\label{section:polyadicrepresentation}

\en{Before in}\ru{Ранее в}~\sectionref{section:tensoranditscomponents}\en{,}
\en{a~tensor}\ru{тензор}
\en{was presented}\ru{был представлен}
\en{as}\ru{как}
\en{some}\ru{некий}
\en{invariant object}\ru{инвариантный объект},
\en{revealing itself}\ru{проявляющий себя}
\en{in every basis}\ru{в~каждом базисе}
\en{as a~collection}\ru{совокупностью}
\en{of numbers~(components)}\ru{чисел~(компонент)}.
\hbox{\en{Such}\ru{Такое}}
\en{a~presentation}\ru{представление}
\en{is typical}\ru{типично}
\en{for}\ru{для}
\en{the~majority of~books}\ru{большинства книг}
\en{about tensors}\ru{о~тензорах}.
\en{The~index notation}\ru{Индексная запись}
\en{can be convenient}\ru{может быть удобна},
\en{especially}\ru{особенно}
\en{when}\ru{когда}
\en{only rectangular coordinates are used}\ru{используются только прямоугольные координаты},
\en{but}\ru{но}
\en{quite often}\ru{весьма часто}
\en{it is not}\ru{это не~так}.
\en{And }\ru{И~}\en{the~relevant case}\ru{подходящий случай}\en{ is}\ru{\:---}
\emph{\en{physics of~continua, elastic and not very elastic}\ru{физика \rucontinuum{}ов, упругих и~не~очень упругих}}\::
\en{it needs}\ru{ей нужен}
\en{more elegant}\ru{более изящн\hbox{ь\kern-0.066emi}й},
\en{more powerful}\ru{более мощный}
\en{and}\ru{и}~\en{perfect}\ru{совершенный}
\en{apparatus}\ru{аппарат}
\en{of the~direct tensor calculus}\ru{прямого тензорного исчисления},
\en{operating with indexless invariant objects}\ru{оперирующий с~безындексными инвариантными объектами}.

\en{The linear combination}\ru{Линейная комбинация}
\en{like}\ru{вида}
${\bm{v} = v_i \bm{e}_i}$
\en{from}\ru{из}~\eqref{vectorcomponents}
\en{connects}\ru{соединяет}
\en{the~vector}\ru{вектор}~${\bm{v}}$
\en{with the~basis}\ru{с~базисом}~${\bm{e}_i}$
\en{and the vector’s components}\ru{и~компонентами}~${v_i}$
\en{in that basis}\ru{вектора в~том базисе}.
\en{Is~there}\ru{Есть~ли}
\en{a~similar relation}\ru{похожее соотношение}
\en{for tensor of any complexity}\ru{для тензора любой сложности}?

\en{Any bivalent tensor}\ru{Любой бивалентный тензор}~$\somebivalenttensor$
\en{has}\ru{имеет}
\en{nine}\ru{девять}~($3^2$)
\en{components}\ru{компонент}~$\somebivalenttensorcomponents{i\hspace{-0.1ex}j}$
\en{in~each basis}\ru{в~каждом базисе}.
\en{The~number}\ru{Число}
\en{of~different dyads}\ru{различных диад}~${\bm{e}_i \bm{e}_{\hspace{-0.15ex}j}}$
\en{for the~same basis}\ru{для одного и~того~же базиса}\ru{\:---}\en{ is}
\en{nine too}\ru{тоже девять}.
\en{Linear combining}\ru{Линейное комбинирование}
\en{these dyads}\ru{этих диад}
\en{with coefficients}\ru{с~коэффициентами}~$\somebivalenttensorcomponents{i\hspace{-0.1ex}j}$
\en{gives}\ru{даёт}
${ \somebivalenttensorcomponents{i\hspace{-0.1ex}j} \bm{e}_i \bm{e}_{\hspace{-0.15ex}j} }$.
\en{Yes}\ru{Да},
\en{this is a~tensor}\ru{это тензор},
\en{like any linear combination of~tensors}\ru{как~и любая линейная комбинация тензоров}.
\en{Yet}\ru{Но}
\en{what are its components}\ru{какие у~него компоненты},
\en{and}\ru{и}~\en{how}\ru{как}
\en{such a~representation}\ru{такое представление}
\en{changes}\ru{меняется}
\en{or}\ru{или}
\en{doesn’t change}\ru{не~меняется}
\en{with a~rotation of~basis}\ru{с~поворотом базиса}?

\en{The components}\ru{Компоненты}
\en{of combination}\ru{комбинации}

\nopagebreak\vspace{-0.2em}
\begin{equation*}
\bigl( \somebivalenttensorcomponents{i\hspace{-0.1ex}j} \hspace{.1ex} \bm{e}_i \bm{e}_{\hspace{-0.1ex}j} \bigr)_{\hspace{-0.2ex}p\hspace{.1ex}q}
\hspace{-0.2ex} \equiv
\somebivalenttensorcomponents{i\hspace{-0.1ex}j} \hspace{-0.1ex}
\bigl( \bm{e}_i \hspace{-0.1ex} \dotp \bm{e}_p \bigr) \hspace{-0.2ex}
\bigl( \bm{e}_{\hspace{-0.1ex}j} \hspace{-0.1ex} \dotp \bm{e}_q \bigr)
\hspace{-0.2ex} = \hspace{-0.1ex}
\somebivalenttensorcomponents{i\hspace{-0.1ex}j} \hspace{.1ex} \delta_{ip} \delta_{\hspace{-0.15ex}j\hspace{-0.1ex}q}
\hspace{-0.2ex} = \hspace{-0.1ex}
\somebivalenttensorcomponents{pq}
\end{equation*}

\vspace{-0.5em}\noindent
\en{are the~components}\ru{суть компоненты}
\en{of~tensor}\ru{тензора}~$\somebivalenttensor$.
\en{And}\ru{И}~%
\en{with a~rotation}\ru{с~поворотом}
\en{of~basis}\ru{базиса}

\nopagebreak\vspace{-0.2em}
\begin{equation*}
\scalebox{.99}{$
   B\hspace{.16ex}'_{\hspace{-0.32ex}i\hspace{-0.1ex}j}
   \bm{e}'_{i}
   \bm{e}'_{\hspace{-0.1ex}j} \hspace{-0.2ex}
   = \cosinematrix{i'\hspace{-0.1ex}p}
   \cosinematrix{j'\hspace{-0.1ex}q}
   \somebivalenttensorcomponents{pq}
   \cosinematrix{i'\hspace{-0.1ex}n}
   \bm{e}_n
   \cosinematrix{j'\hspace{-0.1ex}m}
   \bm{e}_m \hspace{-0.2ex}
   = \delta_{\hspace{-0.1ex}pn}
   \delta_{\hspace{-0.1ex}qm}
   \somebivalenttensorcomponents{pq} \bm{e}_n \bm{e}_m \hspace{-0.2ex}
   = \somebivalenttensorcomponents{pq} \bm{e}_p \bm{e}_q
$}
\hspace{.1ex} .
\end{equation*}

\vspace{-0.2em}
\en{Doubts are dropped}\ru{Сомнения отпали}\::
\en{a~tensor of~second complexity}\ru{тензор второй сложности}
\en{can be~(re)pre\-sent\-ed}\ru{может быть представлен}
\en{as}\ru{как}
\en{a~linear combination}\ru{линейная комбинация}

\nopagebreak\vspace{-0.2em}\begin{equation}\label{polyada:2}
\somebivalenttensor = \somebivalenttensorcomponents{i\hspace{-0.1ex}j} \hspace{.1ex} \bm{e}_i \bm{e}_{\hspace{-0.1ex}j}
\end{equation}

\vspace{-0.4em}\nopagebreak\noindent
--- \en{the~dyadic decomposition}\ru{диадное разложение} \en{of~}\en{a~bivalent tensor}\ru{бивалентного тензора}.

\en{For}\ru{Для}
\en{the~unit tensor}\ru{единичного тензора}

\nopagebreak\vspace{-0.2em}\begin{equation*}
\UnitDyad =
E_{i\hspace{-0.1ex}j} \hspace{.16ex} \bm{e}_i \bm{e}_{\hspace{-0.1ex}j} \hspace{-0.2ex} =
\delta_{i\hspace{-0.1ex}j} \hspace{.16ex} \bm{e}_i \bm{e}_{\hspace{-0.1ex}j} \hspace{-0.2ex} =
\bm{e}_i \bm{e}_i \hspace{-0.2ex} =
\bm{e}_1 \bm{e}_1 \hspace{-0.2ex} + \bm{e}_2 \bm{e}_2 \hspace{-0.2ex} + \bm{e}_3 \bm{e}_3
\hspace{.1ex} ,
\end{equation*}

\vspace{-0.2em}\nopagebreak\noindent
\en{that’s why}\ru{вот почему}
${\hspace{-0.15ex}\UnitDyad}$
\en{is called}\ru{называется}
\en{the~unit dyad}\ru{единичной диадой}.

\en{Using}\ru{Используя}
\en{polyadic representations}\ru{полиадные представления}
\en{like}\ru{типа}~\eqref{polyada:2},
\en{tensors are much easier to handle}\ru{с~тензорами намного проще обращаться}\::

\nopagebreak\vspace{-0.5em}
\begin{equation*}
\bm{v} \dotp \somebivalenttensor =
v_i\bm{e}_i
\hspace{-0.1ex} \dotp
\bm{e}_{\hspace{-0.1ex}j}
\somebivalenttensorcomponents{\hspace{-0.2ex}j\hspace{-0.1ex}k}
\bm{e}_k \hspace{-0.2ex}
= v_i
\delta_{i\hspace{-0.1ex}j}
\somebivalenttensorcomponents{\hspace{-0.2ex}j\hspace{-0.1ex}k} \hspace{.1ex}
\bm{e}_k \hspace{-0.2ex}
= v_i
\somebivalenttensorcomponents{ik}
\bm{e}_k
\hspace{.1ex} ,
\end{equation*}

\nopagebreak\vspace{-1.1em}
\begin{equation}\label{tensorcomponents}
\bm{e}_i
\hspace{-0.1ex} \dotp \hspace{-0.1ex}
\somebivalenttensor
\dotp
\bm{e}_{\hspace{-0.1ex}j} \hspace{-0.2ex}
=
\bm{e}_i
\hspace{-0.1ex} \dotp
\somebivalenttensorcomponents{pq}\bm{e}_p\bm{e}_q
\hspace{-0.1ex} \dotp
\bm{e}_{\hspace{-0.1ex}j} \hspace{-0.2ex}
= \hspace{-0.1ex}
\somebivalenttensorcomponents{pq} \hspace{.1ex}
\delta_{\hspace{-0.1ex}ip}
\delta_{\hspace{-0.1ex}qj} \hspace{-0.2ex}
= \hspace{-0.1ex}
\somebivalenttensorcomponents{i\hspace{-0.1ex}j} \hspace{-0.2ex}
=
\somebivalenttensor
\hspace{-0.1ex} \dotdotp
\bm{e}_{\hspace{-0.1ex}j}
\bm{e}_i
\hspace{.1ex} .
\end{equation}

\vspace{-0.2em}\noindent
\en{The~last line here}\ru{Последняя строчка здесь}
\en{is quite interesting}\ru{весьма интересна}\::
\en{the~tensor components}\ru{компоненты тензора}
\en{are presented}\ru{представлены}
\en{through}\ru{через}
\en{the~tensor itself}\ru{сам тензор}.
\en{An~orthogonal transformation}\ru{Ортогональное преобразование}
\en{of~components}\ru{компонент}
\en{with a~rotation of~basis}\ru{с~поворотом базиса}~\eqref{orthotransform:2}
\en{turns out to be}\ru{оказывается}
\en{just a~version of}\ru{просто версией}~\eqref{tensorcomponents}.

\en{And}\ru{И}~\en{any tensor}\ru{любой тензор},
\en{of any complexity}\ru{любой сложности}
\en{above zero}\ru{выше нуля},
\en{may be decomposed}\ru{может быть разложен}
\en{into the~basis polyads}\ru{по базисным полиадам}.
\en{For a~trivalent tensor}\ru{Для трёхвалентного тензора}

\nopagebreak\begin{equation}\label{polyada:3}
\begin{array}{c}
{^3\!\,\bm{C}} = C_{i\hspace{-0.1ex}j\hspace{-0.1ex}k} \hspace{.2ex} \bm{e}_i \bm{e}_j \bm{e}_k \hspace{.1ex},
\\[.8ex]
C_{i\hspace{-0.1ex}j\hspace{-0.1ex}k} \hspace{-0.2ex} = {^3\!\,\bm{C}} \dotdotdotp \bm{e}_k \bm{e}_j \bm{e}_i = \bm{e}_i \dotp {^3\!\,\bm{C}} \dotdotp \bm{e}_k \bm{e}_j = \bm{e}_j \bm{e}_i \dotdotp {^3\!\,\bm{C}} \dotp \bm{e}_k \hspace{.1ex}.
\end{array}
\end{equation}

\en{Now}\ru{Теперь}
\en{it’s pretty easy to~see}\ru{весьма легк\'{о} увидеть}
\en{the~actuality}\ru{действительность}
\en{of~property}\ru{свойства}~\eqref{definingpropertyoftheidentitytensor}\:---
\en{the~}\inquotes{\en{unitness}\ru{единичность}}
\en{of~tensor}\ru{тензора}~$\UnitDyad$\::

\nopagebreak\vspace{-0.5em}\begin{gather*}
{^\mathrm{n}\hspace{-0.2ex}\bm{a}} = a_{i\hspace{-0.1ex}j \ldots q} \hspace{.4ex} \bm{e}_i \hspace{.2ex} \bm{e}_j \ldots \bm{e}_q , \hspace{.4em}
\UnitDyad = \bm{e}_e \bm{e}_e
\\[.4em]
%
{^\mathrm{n}\hspace{-0.2ex}\bm{a}} \dotp \UnitDyad = a_{i\hspace{-0.1ex}j \ldots q} \hspace{.4ex} \bm{e}_i \hspace{.2ex} \bm{e}_j \ldots \tikzmark{BeginDeltaEQBrace} {\bm{e}_q \hspace{-0.1ex} \dotp \hspace{.1ex} \bm{e}_e} \tikzmark{EndDeltaEQBrace} \bm{e}_e = a_{i\hspace{-0.1ex}j \ldots q} \hspace{.4ex} \bm{e}_i \hspace{.2ex} \bm{e}_j \ldots \bm{e}_q = {^\mathrm{n}\hspace{-0.2ex}\bm{a}} \hspace{.1ex},
\\[.8em]
%
\UnitDyad \dotp {^\mathrm{n}\hspace{-0.2ex}\bm{a}} = \bm{e}_e \bm{e}_e \hspace{-0.1ex} \dotp \hspace{.1ex} a_{i\hspace{-0.1ex}j \ldots q} \hspace{.4ex} \bm{e}_i \hspace{.2ex} \bm{e}_j \ldots \bm{e}_q = a_{i\hspace{-0.1ex}j \ldots q} \hspace{.2ex} \delta_{ei} \hspace{.2ex} \bm{e}_e \hspace{.2ex} \bm{e}_j \ldots \bm{e}_q = {^\mathrm{n}\hspace{-0.2ex}\bm{a}}.
\end{gather*}%
\AddUnderBrace[line width=.75pt]{BeginDeltaEQBrace}{EndDeltaEQBrace}%
{${\scriptstyle \delta_{eq}}$}

\vspace{-1.5em}
\en{The~polyadic representation}\ru{Полиадное представление}
\en{links}\ru{связывает}
\en{the direct and index notations}\ru{прямую и~индексную записи}
\en{together}\ru{воедино}.
\en{It’s not worth}\ru{Не~ст\'{о}ит}
\en{contraposing}\ru{пр\'{о}тиво\-постав\-л\'{я}ть}
\en{one another}\ru{одно другому}.
\en{The~direct notation}\ru{Прямая запись}
\en{is compact, elegant}\ru{компактна, изящна},
\en{it}\ru{она}
\en{much more than others}\ru{намного больше других}
\en{suits}\ru{подходит}
\en{for final relations}\ru{для конечных соотношений}.
\en{But}\ru{Но},
\en{sometimes}\ru{иногда},
\en{the~index notation}\ru{индексная запись}
\en{is very convenient too}\ru{тоже очень удобна},
\en{as it is}\ru{как это есть}
\en{for}\ru{для}
\en{cumbersome}\ru{громоздких}
\en{manipulations}\ru{манипуляций}
\en{with tensors}\ru{с~тензорами}.
