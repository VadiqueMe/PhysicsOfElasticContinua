\en{\section{Tensor algebra, or operations with tensors}}

\ru{\section{Тензорная алгебра, или операции с~тензорами}}

\label{para:operationswithtensors}

\en{The whole tensor algebra}\ru{Целая тензорная алгебра}
\en{can be built}\ru{может быть построена}
\en{on the basis}\ru{на основе}
\en{of the only}\ru{лишь}
\en{five}\ru{пят\'{и}}
(\en{four without the equality}\ru{четырёх без равенства}~\inquotes{$=$})
\en{operations}\ru{операций}.
\en{This paragraph is about them}\ru{Этот параграф\:--- про них}.

\en{So}\ru{Итак},
\en{the tensor algebra}\ru{тензорная алгебра}
\en{consists of the five operations}\ru{состоит из пяти операций}
(\en{actions}\ru{действий}).

% equality
\subsection*{\en{Equality}\ru{Равенство}}

\en{This operation}\ru{Эта операция}
\en{shows}\ru{показывает}\ru{,}
\ru{равен~ли}\en{whether}
\en{one tensor}\ru{один тензор}
(\inquotes{\en{on the left}\ru{слева}})
\en{is equal to another tensor}\ru{другому тензору}
(\inquotes{\en{on the right}\ru{справа}}).
\en{Tensors}\ru{Тензоры}
\en{can be equal}\ru{могут быть равны}
\en{only when}\ru{лишь тогда, когда}
\en{their}\ru{их}
\en{complexities}\ru{сложности}
(\en{valencies}\ru{валентности})
\en{are the~same}\ru{одинаковы}.
\en{Tensors of different valencies}\ru{Тензоры разных валентностей}
\en{cannot be}\ru{не~могут быть}
\en{equal}\ru{равн\'{ы}}
\en{or}\ru{или}
\en{not equal}\ru{не~равн\'{ы}}.

\begin{equation}\label{tensoralgebra:equality}
.....
\end{equation}


....



% linear combination
\subsection*{\en{Linear combination}\ru{Линейная комбинация}}

\en{The~first}\ru{Первое}
\en{is the }\textbold{\en{linear combination}\ru{линейная комбинация}},
\en{it aggregates}\ru{оно объединяет}
\en{the addition}\ru{сложение}
\en{and}\ru{и}
\en{the multiplication by a~number}\ru{умножение на~число}.
\en{Arguments of this operation and the~result}\ru{Аргументы этого действия и~результат}\en{ are}\ru{\:---}
\en{of the~same complexity}\ru{одинаковой сложности}.
\en{For a~two tensors}\ru{Для двух тензоров}:

\nopagebreak\vspace{-0.2em}
\begin{equation}\label{tensoralgebra:linearcombination}
\lambda a_{i\hspace{-0.1ex}j\ldots} \hspace{-0.2ex} + \hspace{.2ex} \mu b_{i\hspace{-0.1ex}j\ldots} \hspace{-0.2ex} = \hspace{.2ex} c_{i\hspace{-0.1ex}j\ldots} \hspace{-0.2ex}
\;\Leftrightarrow\;
\lambda \bm{a} + \mu \bm{b} = \bm{c}
\hspace{.2ex} .
\end{equation}

\vspace{-0.1em}\noindent
\en{Here}\ru{Здесь}
$\lambda$ \en{and}\ru{и}~$\mu$~\en{are}\ru{---} \en{scalar coefficients}\ru{коэффициенты\hbox{-}скаляры};
$\bm{a}$, $\bm{b}$ \en{and}\ru{и}~$\bm{c}$~\en{are}\ru{---} \en{tensors}\ru{тензоры} \en{of the~same complexity}\ru{одной и~той~же сложности}.
\en{It’s easy to show}\ru{Легко показать}\ru{,}
\en{that}\ru{что}
\en{the components}\ru{компоненты}
\en{of the result}\ru{результата}~$\bm{c}$
\en{satisfy}\ru{удовлетворяют}
\en{an~orthogonal transformation}\ru{ортогональному преобразованию}
\en{like}\ru{типа}~\eqref{orthotransform:2}.

\inquotesx{\en{Decomposition of~vector by basis}\ru{Разложение вектора по базису}}[---]
\en{a~representation}\ru{представление}
\en{of~a~vector}\ru{вектора} \en{as the~sum}\ru{суммой}~${\bm{v} = v_i \bm{e}_i}$\:--- \en{is nothing else but}\ru{это не~что~иное как} \en{the~linear combination}\ru{линейная комбинация} \en{of~basis vectors}\ru{векторов базиса}~${\bm{e}_i}$ \en{with coefficients}\ru{с~коэффициентами}~${v_i}$.

% multiplication
\subsection*{\en{Multiplication of tensors}\ru{Умножение тензоров}}

\en{The~second operation}\ru{Второе действие}
\en{is}\ru{это}
\textbold{ \en{the~multiplication}\ru{умножение}
(\en{tensor product}\ru{тензорное произведение},
\en{direct product}\ru{прямое произведение}) }.
\en{It~takes arguments}\ru{Оно принимает аргументы}
\en{of~any complexities}\ru{любых сложностей},
\en{returning}\ru{возвращая}
\en{the~result}\ru{результат}
\en{of a~cumulative complexity}\ru{суммарной сложности}.
\en{Examples}\ru{Примеры}:

\nopagebreak\vspace{-0.1em}\begin{equation}\label{tensoralgebra:multiplication}
\begin{array}{rcl}
v_i a_{j\hspace{-0.1ex}k} \hspace{-0.16ex} = C_{i\hspace{-0.1ex}j\hspace{-0.1ex}k} & \Leftrightarrow & \bm{v} \, {^2\hspace{-0.2ex}\bm{a}} = {^3\hspace{-0.15ex}\bm{C}},
\\[.1em]
a_{i\hspace{-0.1ex}j} B_{abc} \hspace{-0.16ex} = D_{i\hspace{-0.1ex}jabc} & \Leftrightarrow & {^2\hspace{-0.2ex}\bm{a}} \hspace{.25ex} {^3\hspace{-0.33ex}\bm{B}} = {^5\hspace{-0.33ex}\bm{D}}.
\end{array}\hspace{1.5em}
\end{equation}

\vspace{-0.1em}
\en{Transformation of a~collection of result’s components}\ru{Преобразование совокупности компонент результата}, \en{such as}\ru{такой как} ${C_{i\hspace{-0.1ex}j\hspace{-0.1ex}k} \hspace{-0.2ex} = v_i a_{j\hspace{-0.1ex}k}}$, \en{during a~rotation of~basis}\ru{при повороте базиса}\en{ is}\ru{\:---} \en{orthogonal}\ru{ортогональное}, \en{similar to}\ru{подобное}~\eqref{orthotransform:3}, \en{thus}\ru{посему} \en{here’s no~doubt that}\ru{тут нет сомнений, что} \en{this collection is a~set of tensor components}\ru{эта совокупность есть набор компонент тензора}.

\en{Primary}\ru{Первичный} \en{and}\ru{и} \en{already known}\ru{уж\'{е} знакомый} (\en{from}\ru{по}~\sectionref{para:tensoranditscomponents}) \en{subtype of~multiplication}\ru{подвид умножения}\en{ is}\ru{\:---} \en{the~dyadic product of two vectors}\ru{диадное произведение двух векторов}
${^2\hspace{-0.25em}\bm{A} \hspace{-0.15ex} = \bm{b}\bm{c}}$.

% contraction
\subsection*{\en{Contraction}\ru{Свёртка}}

\en{The~third operation}\ru{Третье действие} \en{is called}\ru{называется} \textbold{\ru{свёрткой~(}contraction\ru{)}}.
\en{It applies}\ru{Оно применяется} \en{to bivalent and more complex tensors}\ru{к~бивалентным и более сложным тензорам}.
\en{This operation acts upon a~single tensor}\ru{Это действие над одним тензором}, \en{without}\ru{без} \en{other}\ru{других} \inquotesx{\en{participants}\ru{участников}}[.]
\en{Roughly speaking}\ru{Грубо говоря}, \en{contracting a~tensor}\ru{свёртывание тензора} \en{is}\ru{есть} \en{summing of its components}\ru{суммирование его компонент} \en{over}\ru{по} \en{some}\ru{какой\hbox{-}либо} \en{pair of~indices}\ru{паре индексов}.
\en{As a~result,}\ru{В~результате} \en{tensor’s complexity}\ru{сложность тензора} \en{decreases by two}\ru{уменьшается на~два}.

\en{For the trivalent tensor}\ru{Для трёхвалентного тензора}~${\hspace{-0.1ex} ^3\hspace{-0.33ex}\bm{D}}$\en{,} 
ru{возможны }\en{the three variants}\ru{три варианта} \en{of~contraction}\ru{свёртки}\en{ are possible},
\en{giving vectors}\ru{дающие векторы}
${\bm{a}}$, ${\bm{b}}$ \en{and}\ru{и}~${\bm{c}}$
\en{with the components}\ru{с~компонентами}

\nopagebreak\vspace{-0.33em}\begin{equation}\label{tensoralgebra:contraction}
a_{i} = D_{kki} \hspace{.1ex} ,
\;\;
b_{i} = D_{kik} \hspace{.1ex} ,
\;\;
c_{i} = D_{ikk} \hspace{.16ex} .
\end{equation}

\vspace{-0.33em}\noindent
\en{A~rotation of~basis}\ru{Поворот базиса}

\nopagebreak\vspace{-0.25em}\begin{equation*}
a'_{i} = D\hspace{.16ex}'_{\hspace{-0.32ex}kki} \hspace{-0.16ex}
= \tikzmark{BeginDeltaPQBrace} {\cosinematrix{k'\hspace{-0.1ex}p} \hspace{.1ex} \cosinematrix{k'\hspace{-0.1ex}q}} \tikzmark{EndDeltaPQBrace} \hspace{.1ex} \cosinematrix{i'\hspace{-0.1ex}r} \hspace{.16ex} D_{pqr} \hspace{-0.16ex}
= \cosinematrix{i'\hspace{-0.1ex}r} \hspace{.16ex} D_{ppr} \hspace{-0.16ex}
= \cosinematrix{i'\hspace{-0.1ex}r} \hspace{.16ex} a_{r}
\end{equation*}
\AddUnderBrace[line width = .75pt][0, -0.22ex]{BeginDeltaPQBrace}{EndDeltaPQBrace}%
{${\scriptstyle \delta_{pq}}$}

\nopagebreak\vspace{-0.33em}\noindent
\en{shows}\ru{показывает}
\inquotes{\en{the~tensorial nature}\ru{тензорную природу}}
\en{as the~result of~contraction}\ru{как результат свёртки}.

\en{For a~tensor}\ru{Для~тензора}
\en{of~second complexity}\ru{второй сложности}\en{,} \ru{возможен }\en{the~only one}\ru{лишь один} \en{variant}\ru{вариант} \en{of~contraction}\ru{свёртки}\en{ is possible}.
\en{It gives a~scalar}\ru{дающий скаляр},
\en{known}\ru{известный}
\en{as}\ru{как}
\en{the~trace}\ru{след~(trace)}

\nopagebreak\en{\vspace{-0.2em}}\ru{\vspace{-0.8em}}\begin{equation*}
\bm{B}\tracedot \hspace{.25ex} \equiv \hspace{.3ex}
\trace{\bm{B}} \hspace{.15ex} \equiv \hspace{.4ex}
\mathrm{I}\hspace{.16ex}(\bm{B}) \hspace{-0.15ex}
= B_{kk}
\hspace{.1ex} .
\end{equation*}

\vspace{-0.16em}
\en{The~trace}\ru{След} \en{of~the~unit tensor}\ru{единичного тензора} (\inquotes{\en{contraction of~the~Kronecker delta}\ru{свёртка дельты Kronecker’а}}) \en{is equal to}\ru{равен} \en{the~dimension of~space}\ru{размерности пространства}

\nopagebreak\vspace{-0.2em}\begin{equation*}
\trace{\UnitDyad} = \hspace{-0.1ex} \UnitDyad\tracedot = \delta_{kk} \hspace{-0.2ex} = \hspace{.1ex} \delta_{1\hspace{-0.1ex}1} \hspace{-0.2ex} + \delta_{22} \hspace{-0.2ex} + \delta_{33} \hspace{-0.1ex} = \hspace{.1ex} 3
\hspace{.1ex} .
\end{equation*}

% index juggling
\subsection*{\en{Index juggling, transposing}\ru{Жонглирование индексами, транспонирование}}

\en{The~fourth operation}\ru{Четвёртое действие} \en{is also applicable}\ru{также примен\'{и}мо} \en{to a~single tensor}\ru{к~одному тензору} \en{of~second and bigger complexities}\ru{второй и~б\'{о}льших сложностей}.
\en{It}\ru{Оно} \en{is named}\ru{именуется} \en{as}\ru{как} \textbold{\ru{перестановка индексов~(}index swap\ru{)}, \ru{жонглирование индексами~(}index juggling\ru{)}, \ru{транспонирование~(}transposing\ru{)}}.
\en{From}\ru{Из} \en{components}\ru{компонент} \en{of~a~tensor}\ru{тензора}\en{,} \ru{возникает }\en{the~new collection}\ru{новая совокупность}\en{ is emerged} \en{with another}\ru{с~другой} \en{sequence of~indices}\ru{последовательностью индексов}, \en{the result’s complexity stays the~same}\ru{сложность результата остаётся той~же}.
\en{For example}\ru{Для примера}, \en{trivalent tensor}\ru{трёхвалентный тензор}~${^3\hspace{-0.16em}\bm{D}}$ \en{can give}\ru{может дать} \en{tensors}\ru{тензоры} ${^3\hspace{-0.28em}\bm{A}}$, ${^3\hspace{-0.15em}\bm{B}}$, ${^3\hspace{-0.05em}\bm{C}}$ \en{with components}\ru{с~компонентами}

\nopagebreak\vspace{-0.2em}\begin{equation}\label{tensoralgebra:transposing}
\begin{array}{rcl}
{^3\hspace{-0.3em}\bm{A}} = {^3\hspace{-0.16em}\bm{D}}_{1 \scalebox{0.6}[0.8]{$\rightleftarrows$} 2}
& \!\Leftrightarrow\!\! &
A_{i\hspace{-0.1ex}j\hspace{-0.1ex}k} = D_{j\hspace{-0.06ex}ik}
\hspace{.1ex} ,
\\
{^3\hspace{-0.15em}\bm{B}} = {^3\hspace{-0.16em}\bm{D}}_{1 \scalebox{0.6}[0.8]{$\rightleftarrows$} 3}
& \!\Leftrightarrow\!\! &
B_{i\hspace{-0.1ex}j\hspace{-0.1ex}k} = D_{kj\hspace{-0.06ex}i}
\hspace{.1ex} ,
\\
{^3\hspace{-0.05em}\bm{C}} = {^3\hspace{-0.16em}\bm{D}}_{2 \scalebox{0.6}[0.8]{$\rightleftarrows$} 3}
& \!\Leftrightarrow\!\! &
C_{i\hspace{-0.1ex}j\hspace{-0.1ex}k} = D_{ikj}
\hspace{.1ex} .
\end{array}
\end{equation}

\en{For a~bivalent tensor}\ru{Для бивалентного тензора}\en{,}
\ru{возможно }\en{the~only one transposition}\ru{лишь одно транспонирование}\en{ is possible}:
${\bm{A}^{\hspace{-0.05em}\T} \hspace{-0.15ex} \equiv \bm{A}_{1 \scalebox{0.6}[0.8]{$\rightleftarrows$} 2} = \hspace{-0.1ex} \bm{B}
\hspace{.4ex}\Leftrightarrow\hspace{.25ex}
B_{i\hspace{-0.1ex}j} \hspace{-0.1ex} = A_{j\hspace{-0.06ex}i}}$.
\en{Obviously}\ru{Очевидно},
${\bigl( \hspace{-0.1ex} \bm{A}^{\hspace{-0.05em}\T} \hspace{.15ex} \bigr)^{\hspace{-0.25ex}\T} \hspace{-0.2ex} = \bm{A}}$.

\en{For}\ru{Для} \en{the~dyadic multiplication}\ru{диадного умножения} \en{of~two vectors}\ru{двух векторов}, ${\bm{a} \bm{b} = \bm{b} \bm{a} ^{\hspace{-0.05em}\T}\hspace{-0.4ex}}$.

% combining operations
\subsection*{\en{Combining operations}\ru{Комбинирование операций}}

\en{The~four presented operations (actions)}\ru{Четыре представленных операции (действия)}
\en{can be combined}\ru{могут быть скомбинированы}
\en{in various sequences}\ru{в~разных последовательностях}.

\en{The~combination}\ru{Комбинация}
\en{of~}\en{multiplication}\ru{умножения}~\eqref{tensoralgebra:multiplication}
\en{and}\ru{и}~\en{contraction}\ru{свёртки}~\eqref{tensoralgebra:contraction}\:---
\en{the~}\hbox{\hspace{-0.2ex}\inquotes{${\dotp\hspace{.22ex}}$}\hspace{-0.2ex}}-\en{product}\ru{произведение} (dot product)\:--- \en{is the~most frequently used}\ru{самая часто используемая}.
\en{In the direct indexless notation}\ru{В~прямой безиндексной записи} \en{this is denoted}\ru{это обозначается} \en{by large dot}\ru{крупной точкой}~\hbox{\hspace{-0.2ex}\inquotes{${\dotp\hspace{.22ex}}$}\hspace{-0.2ex}}, \en{which}\ru{которая} \en{shows}\ru{показывает} \en{the~contraction}\ru{свёртку} \en{by adjacent indices}\ru{по~соседним индексам}:

\nopagebreak\vspace{-0.2em}\begin{equation}
\label{tensoralgebra:dotproductexamples}
\bm{a} = \bm{B} \dotp \bm{c}
\,\Leftrightarrow\,
a_i \hspace{-0.15ex} = B_{i\hspace{-0.1ex}j} c_j
\hspace{.1ex} , \;\:
\bm{A} = \bm{B} \dotp \bm{C}
\,\Leftrightarrow\,
A_{i\hspace{-0.1ex}j} \hspace{-0.2ex} = B_{ik} C_{kj}
\hspace{.1ex} .
\end{equation}

\en{The defining property}\ru{Определяющее свойство} \en{of the unit tensor}\ru{единичного тензора}\:--- \en{it is the neutral 
element}\ru{это нейтральный элемент} \en{for}\ru{для} \en{the~dyadic product}\ru{диадного произведения} \en{with the subsequent contraction}\ru{с~последующей свёрткой} \en{by adjacent indices}\ru{по~соседним индексам} (\inquotes{${\dotp\hspace{.25ex}}$}\hbox{\hspace{-0.2ex}-}\en{product}\ru{произведения})

\nopagebreak\vspace{-0.15em}\begin{equation}
\label{definingpropertyofidentitytensor}
{^\mathrm{n}\hspace{-0.2ex}\bm{a}} \dotp \UnitDyad
= \UnitDyad \dotp \hspace{-0.15ex} {^\mathrm{n}\hspace{-0.2ex}\bm{a}}
= {^\mathrm{n}\hspace{-0.2ex}\bm{a}} \;\:\:
\forall \, {^\mathrm{n}\hspace{-0.2ex}\bm{a}} \;\; \forall \hspace{.1ex} \mathrm{n \!>\! 0}
\hspace{.1ex} .
\end{equation}

\en{In the commutative}\ru{В~коммутативном} \en{scalar product}\ru{скалярном произведении} \en{of two vectors}\ru{двух векторов}\en{,} \en{the dot}\ru{точка} \en{represents}\ru{представляет} \en{the same}\ru{то же самое}: \en{the~dyadic product}\ru{диадное произведение} \en{and}\ru{и} \en{the subsequent contraction}\ru{последующая свёртка}

\nopagebreak\vspace{-0.2em}\begin{equation}
\label{transposeofdotproductforvectors}
\bm{a} \dotp \bm{b}
= ( \bm{a} \bm{b} )\hspace{.1ex}\tracedot \hspace{-0.1ex}
= a_i b_i \hspace{-0.2ex}
= b_i a_i \hspace{-0.2ex}
= ( \bm{b} \bm{a} )\hspace{.1ex}\tracedot \hspace{-0.1ex}
= \bm{b} \dotp \bm{a}
\hspace{.1ex} .
\end{equation}

\en{The following}\ru{Следующее} \en{identity}\ru{тождество} \en{describes}\ru{описывает} \en{how to swap multipliers}\ru{как обменять местами множители} \en{for}\ru{для} \en{the }\hbox{\hspace{-0.2ex}\inquotes{${\dotp\hspace{.22ex}}$}\hspace{-0.2ex}-\en{product}\ru{произведения}} (dot product\ru{’а}) \en{of two second complexity tensors}\ru{двух тензоров второй сложности}

\nopagebreak\vspace{-0.2em}\begin{equation}%%\label{transposeofdotproductforbivalenttensors}
\begin{array}{c}
\bm{B} \hspace{-0.1ex} \dotp \bm{Q} \hspace{.1ex} = \bigl( \bm{Q}^{\T} \hspace{-0.2ex}\dotp \bm{B}^{\T} \bigr)^{\hspace{-0.25ex}\T}
\\[.1em]
\bigl( \bm{B} \hspace{-0.1ex} \dotp \bm{Q} \bigr)^{\hspace{-0.1ex}\T} \hspace{-0.3ex} = \hspace{.1ex} \bm{Q}^{\T} \hspace{-0.2ex}\dotp \bm{B}^{\T}
\hspace{-0.2ex} .
\end{array}
\end{equation}

%%\noindent
\en{For two dyads}\ru{Для двух диад} ${\bm{B} \hspace{-0.1ex} = \bm{b}\bm{d}}$ \en{and}\ru{и}~${\bm{Q} \hspace{-0.1ex} = \bm{p}\bm{q}}$

\nopagebreak\vspace{-0.2em}\begin{equation*}
\begin{array}{r@{\hspace{.3em}}c@{\hspace{.3em}}l}
\left(\hspace{.1ex} \bm{b}\bm{d} \dotp \bm{p}\bm{q} \hspace{.1ex}\right)^{\hspace{-0.1ex}\T} & \hspace{-0.5ex} = & \hspace{-0.1ex} \bm{p}\bm{q}^{\T} \hspace{-0.25ex} \dotp \bm{b}\bm{d}^{\hspace{.1ex}\T}
\\[.2em]
d_i p_i \hspace{.25ex} \bm{b}\bm{q}^{\T} & \hspace{-0.5ex} = & \hspace{-0.1ex} \bm{q}\bm{p} \dotp \bm{d} \bm{b}
\\[.1em]
d_i p_i \hspace{.25ex} \bm{q}\bm{b} & = & p_i d_i \hspace{.25ex} \bm{q}\bm{b}
\hspace{.1ex} .
\end{array}
\end{equation*}

%%\noindent
\en{For a~vector and a~bivalent tensor}\ru{Для вектора и~бивалентного тензора}

\nopagebreak\vspace{-0.2em}\begin{equation}%%\label{vectorandbivalenttensordotproduct}
\bm{c} \hspace{.2ex} \dotp \bm{B}
= \bm{B}^{\T} \hspace{-0.4ex} \dotp \bm{c}
\hspace{-0.1ex} ,
\;\;
\bm{B} \dotp \bm{c}
= \bm{c} \hspace{.2ex} \dotp \bm{B}^{\T}
\hspace{-0.3ex} .
\end{equation}

%%\en{Tensor of~second valence \inquotes{squared} is}\ru{Тензор второй валентности \inquotes{в~квадрате} это}

%%\nopagebreak\vspace{-0.2em}\begin{equation}\label{exponentiation:two}
%%\bm{B}^2 \equiv\hspace{.2ex} \bm{B} \hspace{-0.1ex} \dotp \hspace{-0.1ex} \bm{B} .
%%\end{equation}

\en{Contraction can be repeated}\ru{Свёртка может повторяться} \en{two times or more}\ru{два раза или больше}:
${\bigl( \hspace{-0.1ex} \bm{A} \dotp \hspace{-0.1ex} \bm{B} \hspace{.1ex} \bigr)\tracedot
= \hspace{-0.1ex} \bm{A} \dotdotp \hspace{-0.1ex} \bm{B}
= \hspace{-0.1ex} A_{\hspace{.1ex}i\hspace{-0.1ex}j} B_{\hspace{-0.1ex}j\hspace{-0.06ex}i}}$,
\en{and here are useful equations}\ru{и~вот полезные равенства} \en{for second complexity tensors}\ru{для тензоров второй сложности}

\nopagebreak\vspace{-0.1em}\begin{equation}
\begin{array}{c}
\bm{A} \dotdotp \hspace{-0.1ex} \bm{B} = \bm{B} \dotdotp \hspace{-0.1ex} \bm{A}
\hspace{.12ex} ,
\;\;
\bm{d} \hspace{.1ex} \dotp \hspace{-0.15ex} \bm{A} \dotp \bm{b} = \hspace{-0.1ex} \bm{A} \dotdotp \hspace{.1ex} \bm{b}\bm{d} = \bm{b}\bm{d} \hspace{.1ex} \dotdotp \hspace{-0.1ex} \bm{A} = b_{\hspace{-0.1ex}j} d_{\hspace{.1ex}i} A_{\hspace{.1ex}i\hspace{-0.1ex}j} \hspace{.12ex} ,
\\[.2em]
%
\bm{A} \dotdotp \bm{B} = \bm{A}^{\hspace{-0.16ex}\T} \hspace{-0.2ex} \dotdotp \bm{B}^{\T} \hspace{-0.2ex} = A_{\hspace{.1ex}i\hspace{-0.1ex}j} B_{\hspace{-0.1ex}j\hspace{-0.06ex}i}
\hspace{.12ex} ,
\;\;
\bm{A} \dotdotp \bm{B}^{\T} \hspace{-0.2ex} = \bm{A}^{\hspace{-0.16ex}\T} \hspace{-0.2ex} \dotdotp \bm{B} = A_{\hspace{.1ex}i\hspace{-0.1ex}j} B_{i\hspace{-0.1ex}j}
\hspace{.12ex} ,
\\[.2em]
%
\bm{A} \hspace{-0.1ex} \dotdotp \hspace{-0.1ex} \UnitDyad = \UnitDyad \dotdotp \hspace{-0.1ex} \bm{A} = \bm{A}\hspace{.15ex}\tracedot = A_{j\hspace{-0.15ex}j} \hspace{.12ex} ,
\\[.2em]
%
\bm{A} \narrowdotp \bm{B} \narrowdotdotp \UnitDyad = \hspace{-0.16ex} A_{\hspace{.1ex}i\hspace{-0.1ex}j} B_{\hspace{-0.1ex}j\hspace{-0.1ex}k} \hspace{.2ex} \delta_{ki} \hspace{-0.1ex} = \bm{A} \narrowdotdotp \bm{B} ,
\:\:
\bm{A} \narrowdotp \bm{A} \narrowdotdotp \UnitDyad = \bm{A} \narrowdotdotp \bm{A}
\hspace{.12ex} ,
\\[.2em]
%
\bm{A} \narrowdotdotp \bm{B} \narrowdotp \bm{C} = \bm{A} \narrowdotp \bm{B} \narrowdotdotp \hspace{.1ex} \bm{C} = \hspace{.1ex} \bm{C} \narrowdotdotp \hspace{-0.1ex} \bm{A} \narrowdotp \bm{B} = A_{\hspace{.1ex}i\hspace{-0.1ex}j} B_{\hspace{-0.1ex}j\hspace{-0.1ex}k} C_{ki} \hspace{.1ex} ,
\\[.15em]
%
\bm{A} \hspace{-0.1ex}\narrowdotdotp \bm{B} \narrowdotp \bm{C} \narrowdotp \bm{D} = \bm{A} \narrowdotp \bm{B} \narrowdotdotp \bm{C} \narrowdotp \bm{D} = \bm{A} \narrowdotp \bm{B} \narrowdotp \bm{C} \narrowdotdotp \bm{D} = \hspace*{3em} \\
\hspace{11em} = \bm{D} \narrowdotdotp \bm{A} \narrowdotp \bm{B} \narrowdotp \bm{C} = A_{\hspace{.1ex}i\hspace{-0.1ex}j} B_{\hspace{-0.1ex}j\hspace{-0.1ex}k} C_{kh} D_{hi} \hspace{.1ex}.
\end{array}
\end{equation}

