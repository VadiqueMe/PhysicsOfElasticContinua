\en{\chapter{Magnetoelasticity}}

\ru{\chapter{Магнитоупругость}}

\thispagestyle{empty}

\label{chapter:magnetoelasticity}

\en{\section{Electromagnetic field}}

\ru{\section{Электромагнитное поле}}

\label{para:electromagneticfield}

\begin{otherlanguage}{russian}

\lettrine[lines=2, findent=2pt, nindent=0pt]{П}{роблемы}, решаемые с~использованием моделей упругих сред, обычно уж\'{е} содержат известные внешние нагрузки. %% Определение их~--- самостоятельная задача в~своей области.
Если в~упругом теле протекают электрические токи, то нагрузка создаётся магнитным полем. При~деформации тела магнитное поле меняется; при~большой чувствительности поля к~деформации становится необходимым совместное решение задач упругости и~магнетизма.

Вспомним положения теории электромагнетизма. В~трёхмерном пространстве имеем два векторных поля: электрическое~${\boldmathbb{E}(\bm{r},t)}$ и~магнитное~${\boldmathbb{B}(\bm{r},t)}$. Смысл этих векторов ясен из~выражения силы, действующей на~точечный заряд~$q$:

\nopagebreak\vspace{-0.1em}\begin{equation}
\bm{F} = \hspace{.12ex} q \left( \hspace{.1ex} \boldmathbb{E} \hspace{.2ex} + \hspace{.1ex} \bm{v} \times \boldmathbb{B} \right) \hspace{-0.25ex}.
\end{equation}

\vspace{-0.1em} Острый вопрос о~том, в~какой системе отсчёта определяется скорость заряда~$\bm{v}$, ведёт к~специальной теории относительности

...



\section{Электростатика}

Рассмотрение этого вопроса полезно для последующего опис\'{а}ния магнетизма. В~статике имеем

...



\section{Диэлектрики}

Начнём с~рассмотрения электростатического поля

...



\section{Магнитостатика}

Если поле~(а~с~ним ...)

...



\section{Магнетики}

Выяснив законы магнитостатики в~общем случае, обратимся к~веществу~--- некий опыт у~нас уже есть в~электростатике ди\-элект\-ри\-ков.

Начнём с~рассмотрения

...



...

Насколько соответствует поведение реальных материалов представленным здесь формальным построениям~--- этот вопрос вне~рамок нашего изложения.

\section{Магнитная жёсткость}

В~электротехнике распространены обмотки всевозможной формы, в~которых провод намотан так, что образуется некое массивное тело. Такие обмотки есть в~статоре генератора автомобиля~(да~и в~роторе), в~больших промышленных электромагнитах и~в~магнитных системах установок \href{https://ru.wikipedia.org/wiki/%D0%A2%D0%BE%D0%BA%D0%B0%D0%BC%D0%B0%D0%BA}{\inquotes{токам\'{а}к}~(\textbf{то}роидальная \textbf{ка}мера с~\textbf{ма}гнитными \textbf{к}атушками)} для управляемого термоядерного синтеза~--- примеров много. Сочетание токопровода и~изоляции образует периодический композит, и~одной из~главных нагрузок для него является пондеромоторная магнитная сила. Рассчитывая деформации и~механические напряжения в~обмотке, начинают с~определения магнитных сил. Поскольку распределение токов задано известной геометрией проводов, достаточно интегрирования по~формуле~Био\hbox{-}Савара~\eqref{law:biosavar}. Термин \inquotes{магнитоупругость} при~этом неуместен, так~как задачи магнитостатики и~упругости решаются раздельно.

Однако при~деформации обмотки меняются и~поле~$\bm{j}$, и~вызываемое им поле~$\boldmathbb{B}$. Объёмная сила становится равной
\begin{equation}
\bm{f} = \left( \hspace{0.16ex} \bm{j} \times \boldmathbb{B} \right)_0 + \dots
\end{equation}
\noindent Подчёркнутое слагаемое соответствует недеформированному состоянию. Обусловленное деформацией изменение объёмной силы линейно связано с~малым перемещением~$\bm{u}$, поэтому матричное~(после дискретизации) уравнение в~перемещениях можно представить в~виде
\begin{equation}
\left( C + C_m \right) u = F_0 .
\end{equation}
\noindent К~обычному оператору линейной упругости~$C$ добавилась магнитная жёсткость~$C_m$; $F_0$~--- силы в~недоформированном состоянии.

Добавка~$C_m$ пропорциональна квадрату тока и~может стать весьма существенной в~магнитных системах с~сильным полем. Учёт её необходим и~при~недостаточной величине~$C$; в~номинальном режиме конструкция может держать нагрузку, но дополнительная нагрузка неблагоприятного направления может оказаться \inquotesx{невыносимой}[.]

Но особенно важна роль магнитной жёсткости в~задачах устойчивости. Поскольку магнитные силы потенциальны, матрица~$C_m$ симметрична, и~критические параметры могут быть найдены статическим методом Эйлера.

Как иллюстрацию рассмотрим простую задачу о~балке в~продольном магнитном поле. Балка располагается на~декартовой оси~$z$, концы~${z \narroweq 0}$ и~${z \narroweq l}$ закреплены, магнитная индукция ${\boldmathbb{B} = B\bm{k} = \boldconst}$, по~балке течёт постоянный~(по~величине) ток~$I$. В~классической модели балки при~равных жёсткостях на~изгиб для~прогиба ${\bm{u} = u_x\bm{i} \hspace{.12ex} + u_y\bm{j}}$ легко получить следующую постановку:

...

Вводя компл\'{е}ксную комбинацию~${u \equiv u_x \hspace{-0.16ex} + \mathrm{i} u_y}$, будем иметь

...

\noindent с общим решением

...

\noindent Подстановка в~граничные условия приводит к~однородной системе для~постоянных~${A_k}$; приравняв нулю определитель, придём к~характеристическому уравнению

...


\noindent Наименьший положительный корень~${x_1 \hspace{-0.2ex} = 3.666}$, так~что критическая комбинация параметров такова:
\[
\left( I \hspace{-0.25ex} B \hspace{.1ex} {l^3} \hspace{-0.2ex} / a \hspace{.1ex} \right)_{\hspace{-0.32ex}*} \hspace{-0.32ex} = \hspace{.1ex} 394.2 \hspace{.12ex}.
\]

В~этом решении поле~$\boldmathbb{B}$ считалось внешним и~не~варьировалось. Но если собственное поле тока в~стержне сравнимо с~$\boldmathbb{B}$, решение изм\'{е}нится и~усложн\'{и}тся.

\end{otherlanguage}

\section*{\small \wordforbibliography}

\begin{changemargin}{\parindent}{0pt}
\fontsize{10}{12}\selectfont

\begin{otherlanguage}{russian}

Основы электродинамики хорошо изложены во~многих книгах~\cite{classicalelectrodynamics, feynman-lecturesonphysics}, но для~приложений в~механике выделяется курс И.\,Е.\;Тамма~\cite{tamm-electricity}. Растёт список литературы по~связанным задачам электромагнетизма и~упругости~\cite{parton-electromagneticelasticity, podstrigach.burak.kondrat-magnetothermoelasticity}. Как введение в~эту область может быть полезна книга В.\;Новацкого~\cite{nowacki-electromagneticeffects}.

\end{otherlanguage}

\end{changemargin}
