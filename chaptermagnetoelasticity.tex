\en{\chapter{Magnetoelasticity}}

\ru{\chapter{Магнитоупругость}}

\thispagestyle{empty}

\label{chapter:magnetoelasticity}

\begin{changemargin}{\parindent}{\parindent}
\vspace{-2em}
{\noindent \small

\en{Problems}\ru{Проблемы}, \en{solved using models of elastic continua}\ru{решаемые с~использованием моделей упругих сред}, \en{usually}\ru{обыкновенно} \en{include}\ru{сод\'{е}ржат} \en{already known}\ru{уж\'{е} известные} \en{external loads}\ru{внешние нагрузки} (\en{how to~find them}\ru{как найти их}\ru{\:---}\en{ is} \en{a~separate problem}\ru{отдельная проблема}).
\en{When}\ru{Когда} \en{electric currents}\ru{электрические токи} \en{flow}\ru{протекают} \en{in an~elastic body}\ru{в~упругом теле}, \en{magnetic field}\ru{магнитное поле} \en{produces}\ru{создаёт} \en{a~load}\ru{нагрузку}.
\en{With the~deformation}\ru{С~деформацией} \en{of~a~body}\ru{тела} \en{magnetic field changes too}\ru{меняется и~магнитное поле}.
\en{If}\ru{Если} \en{field’s sensitivity}\ru{чувствительность п\'{о}ля} \en{to~deformations}\ru{к~деформациям} \en{is high}\ru{высок\'{а}}, \en{then}\ru{то} \ru{нужно }\en{integrated solving}\ru{совместное решение} \en{of the~problems}\ru{проблем} \en{of~elasticity and~magnetism}\ru{упругости и~магнетизма}\en{ is~needed}.

\par}
\vspace{-1em}
\end{changemargin}

\newcommand\vacuumpermittivity{\varepsilon_{\hspace{-0.1ex}\scalebox{0.6}[0.5]{$\raisemath{.1ex}{0}$}}}

\newcommand\vacuumpermeability{\mu_{\hspace{.05ex}\scalebox{0.6}[0.5]{$\raisemath{.1ex}{0}$}}}

\newcommand\maxwellstress{{^2\hspace{-0.25ex}\mathboldM}}

\newcommand\expminustwelve{\hspace{.16ex}\scalebox{0.8}[1]{\hbox{--}}\raisemath{.1ex}{1\kern-0.07ex2}}

\en{\section{Electromagnetic field. Electromagnetic waves}}

\ru{\section{Электромагнитное поле. Электромагнитные волны}}

\label{para:electromagneticfieldandwaves}

\lettrine[lines=2, findent=2pt, nindent=0pt]{\en{H}\ru{В}}{\en{ere}\ru{от}}\en{ is} \en{an~overview}\ru{опис\'{а}ние} \en{of the~theory of electromagnetism}\ru{теории электромагнетизма}.

\noindent \hspace{\parindent} \en{This theory}\ru{Эта теория} \en{considers}\ru{рассматривает} \en{two vector fields}\ru{два векторных поля}: \en{electric}\ru{электрическое}~${\hspace{.1ex}\boldmathbb{E}(\bm{r}, t)}$ \en{and}\ru{и}~\en{magnetic}\ru{магнитное}~${\hspace{.1ex}\boldmathbb{B}(\bm{r}, t)}$.
\en{What are}\ru{Что такое} \en{vector}\ru{вектор}~${\hspace{.1ex}\boldmathbb{E}}$ \en{and}\ru{и}~\en{pseudo\-vector}\ru{псевдо\-вектор}~${\hspace{.1ex}\boldmathbb{B}}$ \en{is seen}\ru{в\'{и}дно} \en{from}\ru{из} \en{the~expression}\ru{выражения} \en{of electromagnetic force}\ru{электромагнитной силы} (\href{https://en.wikipedia.org/wiki/Lorentz_force}{\en{Lorentz force}\ru{силы Lorentz’а}}) ${\hspace{-0.1ex}\bm{F}(\bm{r}, \bm{v}, t, q)}$, \en{acting on}\ru{действующей на}~\en{a~point charge}\ru{точечный заряд}\:--- \en{a~particle}\ru{частицу} \en{of infinitesimal size}\ru{бесконечно малого размера}, \en{that contains}\ru{содержащую} \en{electric charge}\ru{электрический заряд}~$q$ \en{and}\ru{и}~\en{moves}\ru{движущуюся} \en{with velocity}\ru{со~скоростью}~${\bm{v}}$

\nopagebreak\vspace{-0.15em}\begin{equation}\label{electromagneticforce}
\bm{F} = \hspace{.12ex} q \hspace{.2ex} \bigl( \hspace{.1ex} \boldmathbb{E} \hspace{.2ex} + \hspace{.1ex} \bm{v} \hspace{-0.15ex} \times \hspace{-0.1ex} \boldmathbb{B} \bigr)
\hspace{-0.1ex} .
\end{equation}

\vspace{-0.15em} \noindent \en{In~essence}\ru{По~сути}, \en{the~part}\ru{часть} \en{of the~electromagnetic force}\ru{электромагнитной силы}\ru{,} \en{arising from interaction}\ru{возникающая от взаимо\-действия} \en{with a~moving charge}\ru{с~движущимся зарядом}\:--- \en{magnetic force}\ru{магнитная сила}~${q \hspace{.1ex} \bm{v} \hspace{-0.15ex} \times \hspace{-0.1ex} \boldmathbb{B}}$\:--- \en{reveals}\ru{являет} \en{magnetic field}\ru{магнитное поле}~${\hspace{.1ex}\boldmathbb{B}}$, \en{while}\ru{в~то время как} \en{the~other part}\ru{другая часть}\:--- \en{electric force}\ru{электрическая сила}~${q \hspace{.2ex} \boldmathbb{E}}$\:--- \en{reveals}\ru{являет} \en{electric field}\ru{электрическое поле}~${\hspace{.1ex}\boldmathbb{E}}$.

\vspace{-0.1em} \en{Acute question}\ru{Острый вопрос} \en{in~which exactly}\ru{в~какой~же именно} \en{frame of~reference}\ru{системе отсчёта} \en{is}\ru{измеряется} \en{velocity}\ru{скорость}~${\bm{v} \equiv \mathdotabove{\bm{r}}\hspace{.1ex}}$ \en{of a~charged particle}\ru{зар\'{я}женной частицы}\en{ measured}, \en{leads to}\ru{ведёт к}~\href{https://en.wikipedia.org/wiki/Special_relativity}{\en{the~special theory of~relativity}\ru{специальной теории относительности}}.

classical conception about existence of~absolute space and~time as the~preferred frame of~reference

...

\en{charge density}\ru{плотность заряда} ${\uprho\hspace{.1ex}(\bm{r}, t)}$

\en{vector of~density of~a~current}\ru{вектор плотности тока} ${\bm{j}(\bm{r}, t)}$

...

% ponderomotive force

\en{In a~continuum}\ru{В~сплошной среде} \en{with charges and~currents}\ru{с~зарядами и~токами} \en{acts}\ru{действует} \en{volume}\ru{объёмная} \ru{\inquotes{пондеромоторная} (}\inquotes{ponderomotive}\ru{)} \en{force}\ru{сила}

\nopagebreak\vspace{-0.8em}\begin{equation}\label{ponderomotiveforce}
\bm{f} = \uprho \hspace{.16ex} \boldmathbb{E} \hspace{.1ex} + \bm{j} \hspace{-0.16ex} \times \hspace{-0.2ex} \boldmathbb{B}
\end{equation}

\vspace{-0.1em} \noindent --- \en{continual version of}\ru{континуальная версия}~\eqref{electromagneticforce}.

% Maxwell stress tensor

\en{Plus there’s}\ru{Плюс существует} \en{another description}\ru{другое описание}\:--- \en{so called}\ru{так называемый} \href{https://en.wikipedia.org/wiki/Maxwell_stress_tensor}{\ru{тензор напряжения }Maxwell\ru{’а}\en{ stress tensor}}

\nopagebreak\vspace{-0.1em}\begin{equation}\label{maxwellstresstensor:definition}
\maxwellstress \equiv
\vacuumpermittivity \hspace{-0.25ex} \left(
\boldmathbb{E} \hspace{.1ex} \boldmathbb{E} + c^{\hspace{.16ex}2} \hspace{.3ex} \boldmathbb{B} \boldmathbb{B}
- \hspace{.1ex} \smalldisplaystyleonehalf \hspace{.1ex} \bigl( \hspace{.1ex} \boldmathbb{E} \hspace{-0.1ex} \dotp \hspace{-0.15ex} \boldmathbb{E} + c^{\hspace{.16ex}2} \hspace{.3ex} \boldmathbb{B} \hspace{-0.2ex} \dotp \hspace{-0.16ex} \boldmathbb{B} \bigr) \hspace{-0.06ex} \bm{E} \hspace{.05ex}
\right)
\end{equation}

...

\href{https://en.wikipedia.org/wiki/James_Clerk_Maxwell}{Maxwell}’s equations

\nopagebreak\vspace{-0.2em}\begin{equation}\label{maxwell.equations}
\begin{array}{r@{\hspace{.3em}}c@{\hspace{.3em}}l}
\boldnabla \dotp \hspace{.12ex} \boldmathbb{E} & = & \smash{\displaystyle \frac{\uprho}{\raisemath{.16em}{\vacuumpermittivity}}}
\\[.6em]
%
\boldnabla \hspace{-0.15ex} \times \hspace{-0.1ex} \boldmathbb{E} & = & - \hspace{.2ex} \mathdotabove{\boldmathbb{B}}
\\[.6em]
%
\boldnabla \dotp \hspace{.12ex} \boldmathbb{B} & = & \hspace{.1ex} 0
\\[.6em]
%
c^{\hspace{.16ex}2} \hspace{.4ex} \boldnabla \hspace{-0.15ex} \times \hspace{-0.1ex} \boldmathbb{B} & = & \smash{\displaystyle \frac{\bm{j}}{\raisemath{.16em}{\vacuumpermittivity}} + \hspace{.1ex} \mathdotabove{\boldmathbb{E}}}
\end{array}
\end{equation}

...

speed of~light in vacuum
${c = 299\:792\:458}$~\raisebox{.3em}{\en{m}\ru{м}}\hspace{-0.2ex}/\hspace{-0.25ex}\raisebox{-0.2em}{\en{s}\ru{с}}

\inquotesx{electric constant}[,] \href{https://en.wikipedia.org/wiki/Vacuum_permittivity}{vacuum permittivity} $\vacuumpermittivity$

${\vacuumpermittivity \hspace{-0.12ex} \approx 8.8541878 \hspace{-0.25ex}\cdot\hspace{-0.3ex} 10^{\expminustwelve}}$~${\text{\en{F}\ru{Ф}} \hspace{-0.33ex}\cdot\hspace{-0.33ex} \text{\en{m}\ru{м}}^{\hspace{-0.2ex}\expminusone}}$~(\href{https://en.wikipedia.org/wiki/Farad}{\en{farads}\ru{фарад}} \en{per~metre}\ru{на~метр})

\inquotesx{magnetic constant}[,] \href{https://en.wikipedia.org/wiki/Vacuum_permeability}{vacuum permeability}~${\vacuumpermeability \hspace{-0.24ex} = \hspace{-0.08ex} \smash{\scalebox{0.9}{$\displaystyle \frac{\raisemath{-0.16em}{1}}{\raisemath{-0.1em}{\vacuumpermittivity \hspace{.1ex} c^{\hspace{.16ex}2}}}$}}}$

...

Ampère’s circuital law (with Maxwell’s addition) (with Maxwell’s additional time-dependent term)

...

\eqrefwithchapdotpara{divergenceofdyadicproducoftwovectors}{chapter:elementsoftensorcalculus}{para:differentiationoftensorfields}
$\Rightarrow$
${
\boldnabla \hspace{-0.16ex} \dotp \hspace{-0.2ex} \bigl( \hspace{-0.05ex} \bm{a} \bm{a} \bigr) \hspace{-0.33ex}
= \hspace{-0.1ex} \bigl( \boldnabla \hspace{-0.18ex} \dotp \hspace{-0.1ex} \bm{a} \bigr) \hspace{-0.1ex} \bm{a} \hspace{.1ex} + \bm{a} \hspace{-0.07ex} \dotp \hspace{-0.2ex} \boldnabla \hspace{-0.1ex} \bm{a}
}$

\eqrefwithchapdotpara{gradientofdotproductoftwovectors}{chapter:elementsoftensorcalculus}{para:differentiationoftensorfields}
$\Rightarrow$
${
\boldnabla \hspace{.1ex} \bigl( \hspace{-0.1ex} \bm{a} \hspace{-0.17ex} \dotp \hspace{-0.2ex} \bm{a} \bigr) \hspace{-0.33ex}
= 2 \hspace{.16ex} \boldnabla \hspace{-0.1ex} \bm{a} \hspace{-0.12ex} \dotp \hspace{-0.15ex} \bm{a}
}$

${
\boldnabla \hspace{-0.16ex} \dotp \hspace{-0.12ex} \bigl( \hspace{-0.05ex} \bm{a} \hspace{-0.17ex} \dotp \hspace{-0.2ex} \bm{a} \hspace{.1ex} \bm{E} \hspace{.12ex} \bigr) \hspace{-0.33ex}
= \hspace{-0.15ex} \boldnabla \hspace{-0.2ex} \dotp \hspace{-0.22ex}  \bm{E} \hspace{.13ex} \bigl( \hspace{-0.1ex} \bm{a} \hspace{-0.17ex} \dotp \hspace{-0.2ex} \bm{a} \bigr) \hspace{-0.33ex}
= \hspace{-0.15ex} \boldnabla \hspace{.1ex} \bigl( \hspace{-0.1ex} \bm{a} \hspace{-0.17ex} \dotp \hspace{-0.2ex} \bm{a} \bigr) \hspace{-0.33ex}
= 2 \hspace{.16ex} \boldnabla \hspace{-0.1ex} \bm{a} \hspace{-0.12ex} \dotp \hspace{-0.15ex} \bm{a}
}$

\begin{multline*}
\boldnabla \hspace{-0.05ex} \dotp \hspace{-0.1ex} \maxwellstress =
\hspace{.1ex} \vacuumpermittivity \Bigl( \hspace{-0.2ex}
\boldnabla \hspace{-0.1ex} \dotp \hspace{-0.1ex} \bigl( \boldmathbb{E} \hspace{.1ex} \boldmathbb{E} \hspace{.1ex} \bigr) \hspace{-0.22ex}
+ \hspace{.1ex} c^{\hspace{.16ex}2} \hspace{.3ex} \boldnabla \hspace{-0.1ex} \dotp \hspace{-0.1ex} \bigl( \boldmathbb{B} \boldmathbb{B} \bigr) \hspace{-0.22ex}
- \hspace{.2ex} \smalldisplaystyleonehalf \hspace{.1ex} \boldnabla \hspace{-0.12ex} \dotp \hspace{-0.15ex} \bigl( \hspace{.1ex} \boldmathbb{E} \hspace{-0.1ex} \dotp \hspace{-0.15ex} \boldmathbb{E} \hspace{.1ex} \bm{E} \hspace{.1ex}
+ c^{\hspace{.16ex}2} \hspace{.3ex} \boldmathbb{B} \hspace{-0.2ex} \dotp \hspace{-0.16ex} \boldmathbb{B} \bm{E} \hspace{.15ex} \bigr) \hspace{-0.2ex}
\Bigr) \hspace{-0.33ex} =
\\[-0.25em]
%
= \hspace{.1ex} \vacuumpermittivity \Bigl( \hspace{-0.2ex}
\boldnabla \hspace{-0.15ex} \dotp \boldmathbb{E} \hspace{.1ex} \boldmathbb{E} \hspace{.1ex}
+ \boldmathbb{E} \dotp \hspace{-0.22ex} \boldnabla \hspace{.2ex} \boldmathbb{E} \hspace{.1ex}
+ c^{\hspace{.16ex}2} \hspace{.3ex} \boldnabla \hspace{-0.15ex} \dotp \boldmathbb{B} \boldmathbb{B}
+ c^{\hspace{.16ex}2} \hspace{.3ex} \boldmathbb{B} \hspace{-0.1ex} \dotp \hspace{-0.22ex} \boldnabla \hspace{.2ex} \boldmathbb{B} \Bigr. \hspace{.3ex} -
\\[-0.55em]
%
- \hspace{-0.2ex} \Bigl. \boldnabla \hspace{.2ex} \boldmathbb{E} \hspace{-0.1ex} \dotp \hspace{-0.15ex} \boldmathbb{E} \hspace{.1ex}
- c^{\hspace{.16ex}2} \hspace{.3ex} \boldnabla \hspace{.2ex} \boldmathbb{B} \hspace{-0.2ex} \dotp \hspace{-0.16ex} \boldmathbb{B} \hspace{-0.1ex}
\Bigr)
\end{multline*}

...

to derive wave equations

\nopagebreak\begin{equation*}
\begin{array}{r@{\hspace{.3em}}c@{\hspace{.3em}}l}
\boldnabla \hspace{-0.2ex} \times \hspace{-0.3ex} \bigl( \boldnabla \hspace{-0.15ex} \times \hspace{-0.1ex} \boldmathbb{E} \hspace{.1ex} \bigr) \hspace*{-0.3ex} &
= &
- \hspace{.1ex} \boldnabla \hspace{-0.15ex} \times \hspace{-0.1ex} \mathdotabove{\boldmathbb{B}}
\\[.1em]
\boldnabla \hspace{-0.2ex} \times \hspace{-0.3ex} \bigl( \boldnabla \hspace{-0.15ex} \times \hspace{-0.1ex} \boldmathbb{B} \bigr) \hspace*{-0.3ex} &
= &
\boldnabla \hspace{-0.12ex} \times \hspace{-0.33ex} \biggl( \displaystyle \frac{\bm{j}}{\raisemath{-0.1em}{ \vacuumpermittivity \hspace{.1ex} c^{\hspace{.16ex}2} }} + \hspace{.1ex} \displaystyle \frac{\raisemath{-0.12em}{ \mathdotabove{\boldmathbb{E}} }}{\raisemath{-0.1em}{ c^{\hspace{.16ex}2} }} \biggr)
\end{array}
\end{equation*}

${\boldnabla \hspace{-0.2ex} \times \hspace{-0.3ex} \bigl( \boldnabla \hspace{-0.15ex} \times \hspace{-0.1ex} \boldmathbb{E} \hspace{.1ex} \bigr) \hspace{-0.25ex} = \hspace{-0.1ex} \boldnabla \boldnabla \dotp \hspace{.12ex} \boldmathbb{E} \hspace{.1ex} - \Laplacian \hspace{.05ex} \boldmathbb{E}}$

${\boldnabla \hspace{-0.2ex} \times \hspace{-0.3ex} \bigl( \boldnabla \hspace{-0.15ex} \times \hspace{-0.1ex} \boldmathbb{B} \bigr) \hspace{-0.25ex} = \hspace{-0.1ex} \boldnabla \boldnabla \dotp \hspace{.12ex} \boldmathbb{B} \hspace{.1ex} - \Laplacian \hspace{.05ex} \boldmathbb{B}}$

\nopagebreak\begin{equation*}
\begin{array}{r@{\hspace{.3em}}c@{\hspace{.3em}}l}
\boldnabla \dotp \hspace{.12ex} \boldmathbb{E} & = & \displaystyle \frac{\uprho}{\raisemath{.16em}{\vacuumpermittivity}}
\\[.5em]
\boldnabla \dotp \hspace{.12ex} \boldmathbb{B} & = & \hspace{.1ex} 0
\end{array}
\end{equation*}


...

${\boldnabla \dotp \hspace{-0.1ex} \bigl( \boldnabla \hspace{-0.2ex} \times \hspace{-0.16ex} \bm{a} \bigr) \hspace{-0.24ex} = 0 \:\:\forall \bm{a}}$

${\boldnabla \hspace{-0.15ex} \times \hspace{-0.25ex} \boldnabla \alpha = \bm{0} \:\:\forall \alpha}$

vector potential $\boldmathbb{A}$

\nopagebreak\[
\boldnabla \dotp \hspace{.12ex} \boldmathbb{B} = \hspace{.1ex} 0
\hspace{.4em} \Leftrightarrow \hspace{.4em}
\boldmathbb{B} =  \hspace{-0.1ex} \boldnabla \hspace{-0.15ex} \times \hspace{-0.12ex} \boldmathbb{A}
\]

potential~${\boldmathbb{A}}$ is not unique and has gauge freedom ${\boldmathbb{A} + \hspace{-0.2ex} \boldnabla a}$

\[
\boldmathbb{B} =  \hspace{-0.1ex} \boldnabla \hspace{-0.15ex} \times \hspace{-0.3ex} \bigl( \boldmathbb{A} + \hspace{-0.2ex} \boldnabla a \hspace{.1ex} \bigr)
\hspace{.25em} \Leftrightarrow \hspace{.27em}
\boldnabla \dotp \hspace{.12ex} \boldmathbb{B} = \hspace{.1ex} 0
\]

scalar potential $\upphi$

\nopagebreak\begin{multline}
\boldnabla \hspace{-0.15ex} \times \hspace{-0.1ex} \boldmathbb{E} = - \hspace{.2ex} \mathdotabove{\boldmathbb{B}}
\hspace{.4em} \Rightarrow \hspace{.3em}
\boldnabla \hspace{-0.15ex} \times \hspace{-0.1ex} \boldmathbb{E} = - \hspace{.2ex} \boldnabla \hspace{-0.15ex} \times \hspace{-0.3ex} \bigl( \mathdotabove{\boldmathbb{A}} + \hspace{-0.2ex} \boldnabla \mathdotabove{a} \hspace{.15ex} \bigr)
\hspace{.4em} \Rightarrow
\\[-0.1em]
\Rightarrow \hspace{.3em}
\boldnabla \hspace{-0.15ex} \times \hspace{-0.3ex} \bigl( \hspace{.1ex} \boldmathbb{E} + \mathdotabove{\boldmathbb{A}} + \hspace{-0.2ex} \boldnabla \mathdotabove{a} \hspace{.15ex} \bigr) \hspace{-0.4ex} = 0
\hspace{.4em} \Rightarrow \hspace{.25em}
- \boldnabla \upphi = \hspace{.1ex} \boldmathbb{E} + \mathdotabove{\boldmathbb{A}} + \hspace{-0.2ex} \boldnabla \mathdotabove{a}
\hspace{.4em} \Rightarrow
\\[-0.1em]
\Rightarrow \hspace{.33em}
\boldmathbb{E} = - \boldnabla \hspace{.1ex} \bigl( \upphi \hspace{-0.1ex} + \mathdotabove{a} \hspace{.1ex} \bigr) \hspace{-0.3ex} - \mathdotabove{\boldmathbb{A}}
\hspace{.2ex} .
\end{multline}

And

\nopagebreak\vspace{-0.1em}\begin{equation}\label{electromagnetic.firstequationofwave}
\boldnabla \dotp \hspace{.12ex} \boldmathbb{E} = \displaystyle \frac{\uprho}{\raisemath{.16em}{\vacuumpermittivity}}
\hspace{.44em} \Rightarrow
%
\begin{array}{r@{\hspace{.3em}}c@{\hspace{.3em}}l}
- \hspace{-0.1ex} \Laplacian \bigl( \upphi \hspace{-0.1ex} + \mathdotabove{a} \hspace{.1ex} \bigr) \hspace{-0.2ex} - \hspace{-0.15ex} \boldnabla \hspace{-0.1ex} \dotp \mathdotabove{\boldmathbb{A}} & = & \scalebox{0.9}{$\displaystyle \frac{\uprho}{\raisemath{.16em}{\vacuumpermittivity}}$}
\\[.66em]
- \hspace{-0.1ex} \Laplacian \upphi - \hspace{-0.15ex} \boldnabla \hspace{-0.12ex} \dotp \hspace{-0.12ex} \bigl( \mathdotabove{\boldmathbb{A}} + \hspace{-0.2ex} \boldnabla \mathdotabove{a} \hspace{.15ex} \bigr) \hspace*{-0.33ex} & = & \scalebox{0.9}{$\displaystyle \frac{\uprho}{\raisemath{.16em}{\vacuumpermittivity}}$}
\end{array}
\end{equation}

${\boldnabla \hspace{-0.2ex} \times \hspace{-0.3ex} \bigl( \boldnabla \hspace{-0.2ex} \times \hspace{-0.2ex} \boldmathbb{A} \bigr) \hspace{-0.25ex} = \hspace{-0.1ex} \boldnabla \boldnabla \hspace{-0.1ex} \dotp \boldmathbb{A} - \Laplacian \hspace{.05ex} \boldmathbb{A}}$

${\boldnabla \boldnabla \hspace{-0.2ex} \dotp \hspace{-0.25ex} \boldnabla a - \hspace{-0.15ex} \boldnabla \hspace{-0.2ex} \dotp \hspace{-0.25ex} \boldnabla \boldnabla a = 0}$ (partial derivatives of a~smooth function commute)

\nopagebreak\vspace{-0.1em}\begin{multline}\label{electromagnetic.secondequationofwave}
c^{\hspace{.16ex}2} \hspace{.4ex} \boldnabla \hspace{-0.15ex} \times \hspace{-0.1ex} \boldmathbb{B}
= \displaystyle \frac{\bm{j}}{\raisemath{.16em}{\vacuumpermittivity}} + \hspace{.1ex} \mathdotabove{\boldmathbb{E}}
\hspace{.44em} \Rightarrow \hspace{.33em}
%
c^{\hspace{.16ex}2} \hspace{.4ex} \boldnabla \hspace{-0.2ex} \times \hspace{-0.3ex} \bigl( \boldnabla \hspace{-0.2ex} \times \hspace{-0.2ex} \boldmathbb{A} \bigr) \hspace{-0.25ex}
= \displaystyle \frac{\bm{j}}{\raisemath{.16em}{\vacuumpermittivity}} - \hspace{-0.15ex} \boldnabla \hspace{.1ex} \bigl( \mathdotabove{\upphi} \hspace{-0.1ex} + \mathdotdotabove{a} \hspace{.2ex} \bigr) \hspace{-0.3ex} - \mathdotdotabove{\boldmathbb{A}}
\hspace{.44em} \Rightarrow
%
\\[-0.4em]
%
\Rightarrow \hspace{.33em}
c^{\hspace{.16ex}2} \bigl( \boldnabla \boldnabla \hspace{-0.1ex} \dotp \boldmathbb{A} - \Laplacian \hspace{.05ex} \boldmathbb{A} \bigr) \hspace{-0.25ex}
= \displaystyle \frac{\bm{j}}{\raisemath{.16em}{\vacuumpermittivity}} - \hspace{-0.15ex} \boldnabla \mathdotabove{\upphi} - \hspace{-0.15ex} \boldnabla \hspace{.1ex} \mathdotdotabove{a} \hspace{.1ex} - \mathdotdotabove{\boldmathbb{A}}
\hspace{.2ex} .
\end{multline}

\en{With a~gauge freedom}\ru{Со~свободой калибровки} \en{it’s possible}\ru{возможно} \en{to~simplify}\ru{упростить} \en{wave equations}\ru{волновые уравнения}~\eqref{electromagnetic.secondequationofwave} \en{and}\ru{и}~\eqref{electromagnetic.firstequationofwave}, \en{assuming that}\ru{полож\'{и}в}

\nopagebreak\vspace{-0.25em}\begin{gather*}
\left.
\begin{array}{r@{\hspace{.55em}}c@{\hspace{.55em}}l}
- \boldnabla \hspace{.1ex} \mathdotdotabove{a} \hspace{.1ex} = \hspace{-0.2ex} \boldnabla \mathdotabove{\upphi} + c^{\hspace{.16ex}2} \hspace{.3ex} \boldnabla \boldnabla \hspace{-0.1ex} \dotp \boldmathbb{A}
& \Rightarrow &
%
\mathdotdotabove{a} \hspace{.1ex} = - \hspace{.4ex} \mathdotabove{\upphi} - c^{\hspace{.16ex}2} \hspace{.3ex} \boldnabla \hspace{-0.1ex} \dotp \boldmathbb{A}
%
\\[.25em]
%
- \boldnabla \hspace{-0.12ex} \dotp \hspace{-0.12ex} \bigl( \mathdotabove{\boldmathbb{A}} + \hspace{-0.2ex} \boldnabla \mathdotabove{a} \hspace{.15ex} \bigr) \hspace{-0.3ex}
= \scalebox{0.9}{$\displaystyle \frac{\raisemath{-0.16em}{1}}{\raisemath{-0.1em}{c^{\hspace{.16ex}2}}}$} \hspace{.12ex} \mathdotdotabove{\upphi}
& \Rightarrow &
%
\mathdotabove{\upphi} = \hspace{-0.1ex} - \hspace{.2ex} c^{\hspace{.16ex}2} \hspace{.33ex} \boldnabla \hspace{-0.12ex} \dotp \hspace{-0.12ex} \bigl( \boldmathbb{A} + \hspace{-0.2ex} \boldnabla a \hspace{.1ex} \bigr)
\end{array}
\hspace{-0.2em} \right\rbrace \hspace{.1em} \Rightarrow
%
\\[-0.1em]
%
\Rightarrow \hspace{.4em}
c^{\hspace{.16ex}2} \hspace{.33ex} \boldnabla \hspace{-0.12ex} \dotp \hspace{-0.12ex} \boldmathbb{A} + c^{\hspace{.16ex}2} \hspace{-0.4ex} \Laplacian \hspace{.07ex} a
- c^{\hspace{.16ex}2} \hspace{.3ex} \boldnabla \hspace{-0.1ex} \dotp \boldmathbb{A} = \hspace{.15ex} \mathdotdotabove{a}
\hspace{.16ex} ,
\end{gather*}

\vspace{-0.16em} \noindent \en{finally presenting as}\ru{наконец становясь} \en{homogeneous}\ru{однородным} \href{https://en.wikipedia.org/wiki/Wave_equation}{\en{wave equation}\ru{волновым уравнением}} \en{for}\ru{для}~$a$

\nopagebreak\vspace{-0.12em}\begin{equation}\label{gaugeadditioniswave}
\mathdotdotabove{a} \hspace{.1ex} = c^{\hspace{.16ex}2} \hspace{-0.4ex} \Laplacian \hspace{.07ex} a
\hspace{.16ex} .
\end{equation}

\vspace{-0.15em} \noindent \en{A~more popular}\ru{Более популярный} \en{option}\ru{вариант}\ru{\:---}\en{ is} \en{even more tight}\ru{ещё более жёсткое} \en{condition}\ru{условие}

\nopagebreak\vspace{-0.12em}\begin{equation*}
\Laplacian \hspace{.07ex} a = 0
\hspace{.3em} \Rightarrow \hspace{.3em}
\mathdotdotabove{a} = 0
\hspace{.1ex} , \:\:
\mathdotabove{\upphi} + c^{\hspace{.16ex}2} \hspace{.3ex} \boldnabla \hspace{-0.1ex} \dotp \boldmathbb{A} = 0
\end{equation*}

\vspace{-0.2em} \noindent --- \href{https://en.wikipedia.org/wiki/Lorenz_gauge_condition}{\ru{условие калибровки }Lorenz\ru{’а}\en{ gauge condition}}, \en{which}\ru{которое} \en{gives}\ru{даёт} \en{the~same}\ru{такой~же} \en{effect}\ru{эффект}, \en{being just a~particular}\ru{будучи лишь частным}\:--- \href{https://en.wikipedia.org/wiki/Harmonic_function}{\en{harmonic}\ru{гармоническим}}\:--- \en{case}\ru{случаем}~\en{of~}\eqref{gaugeadditioniswave}.

\en{Following from}\ru{Следующие из}~\eqref{electromagnetic.firstequationofwave} \en{and}\ru{и}~\eqref{electromagnetic.secondequationofwave} \en{with condition}\ru{с~условием}~\eqref{gaugeadditioniswave}, \en{equations}\ru{уравнения} \en{of electromagnetic waves}\ru{электромагнитных волн} \en{in the~potential formulation}\ru{в~потенциальной формулировке}\en{ are}

\nopagebreak\begin{equation}
\begin{array}{c}
- \hspace{-0.1ex} \Laplacian \upphi + \scalebox{0.9}{$\displaystyle \frac{\raisemath{-0.16em}{1}}{\raisemath{-0.1em}{c^{\hspace{.16ex}2}}}$} \hspace{.12ex} \mathdotdotabove{\upphi} = \smash{\displaystyle \frac{\uprho}{\raisemath{.16em}{\vacuumpermittivity}}}
\hspace{.1ex} ,
\\[.5em]
- \hspace{.25ex} c^{\hspace{.16ex}2} \hspace{-0.4ex} \Laplacian \hspace{.05ex} \boldmathbb{A} = \displaystyle \frac{\bm{j}}{\raisemath{.16em}{\vacuumpermittivity}} - \mathdotdotabove{\boldmathbb{A}}
\hspace{.2ex} .
\end{array}
\end{equation}

...



\en{\section{Electrostatics}}

\ru{\section{Электростатика}}

\begin{otherlanguage}{russian}

Рассмотрение этого вопроса полезно для последующего опис\'{а}ния магнетизма. В~статике

\nopagebreak\vspace{-0.1ex}\begin{equation*}
\bm{v} \equiv \mathdotabove{\bm{r}} = \bm{0}
\hspace{.44em} \Rightarrow \hspace{.4em}
\boldmathbb{B} = \bm{0}
\end{equation*}

...

Объёмная (пондеромоторная, ponderomotive) сила, с~которой электростатическое поле действует на~среду ...

\ru{Тензор напряжения }Maxwell\ru{’а}\en{ stress tensor}~\eqref{maxwellstresstensor:definition} \en{in~electrostatics}\ru{в~электростатике}

\nopagebreak\vspace{-0.1em}\begin{equation*}
\maxwellstress = \vacuumpermittivity \hspace{-0.2ex} \left( \boldmathbb{E} \boldmathbb{E} - \smalldisplaystyleonehalf \hspace{.2ex} \boldmathbb{E} \hspace{-0.1ex} \dotp \hspace{-0.16ex} \boldmathbb{E} \hspace{.1ex} \bm{E} \hspace{.1ex} \right)
\end{equation*}

...


\end{otherlanguage}

\en{\section{Dielectrics}}

\ru{\section{Диэлектрики}}

\begin{otherlanguage}{russian}

Начнём с~рассмотрения электростатического поля

...

В~диэлектриках нет свободных зарядов: \en{charge density}\ru{плотность заряда}~${\uprho = 0}$. Здесь вводится плотность дипольного момента

...


\end{otherlanguage}

\en{\section{Magnetostatics}}

\ru{\section{Магнитостатика}}

\begin{otherlanguage}{russian}

Если поле~(а~с~ним ...)

...



\end{otherlanguage}

\en{\section{Magnetics}}

\ru{\section{Магнетики}} % Магнитные материалы

\begin{otherlanguage}{russian}

Выяснив законы магнитостатики в~общем случае, обратимся к~веществу\:--- некий опыт у~нас уже есть в~электростатике ди\-элект\-ри\-ков.

Начнём с~рассмотрения

...



...

Насколько соответствует поведение реальных материалов представленным здесь формальным построениям\:--- сей вопрос is out of~scope этой книги.

\end{otherlanguage}

\en{\section{Magnetic rigidity}}

\ru{\section{Магнитная жёсткость}}

\begin{otherlanguage}{russian}

В~электротехнике распространены обмотки всевозможной формы, в~которых провод намотан так, что образуется некое массивное тело. Такие обмотки есть в~статоре генератора автомобиля~(да~и в~роторе), в~больших промышленных электромагнитах и~в~магнитных системах установок \href{https://ru.wikipedia.org/wiki/%D0%A2%D0%BE%D0%BA%D0%B0%D0%BC%D0%B0%D0%BA}{\inquotes{токам\'{а}к}~(\textbf{то}роидальная \textbf{ка}мера с~\textbf{ма}гнитными \textbf{к}атушками)} для управляемого термоядерного синтеза\:--- примеров много. Сочетание токопровода и~изоляции образует периодический композит, и~одной из~главных нагрузок для него является пондеромоторная магнитная сила. Рассчитывая деформации и~механические напряжения в~обмотке, начинают с~определения магнитных сил. Поскольку распределение токов задано известной геометрией проводов, достаточно интегрирования по~формуле~Био\hbox{-}Савара~\eqref{law:biosavar}. Термин \inquotes{магнитоупругость} при~этом неуместен, так~как задачи магнитостатики и~упругости решаются раздельно.

Однако при~деформации обмотки меняются и~поле~$\bm{j}$, и~вызываемое им поле~$\boldmathbb{B}$. Объёмная сила становится равной
\begin{equation}
\bm{f} = \left( \hspace{0.16ex} \bm{j} \times \boldmathbb{B} \right)_0 + \dots
\end{equation}
\noindent Подчёркнутое слагаемое соответствует недеформированному состоянию. Обусловленное деформацией изменение объёмной силы линейно связано с~малым перемещением~$\bm{u}$, поэтому матричное~(после дискретизации) уравнение в~перемещениях можно представить в~виде
\begin{equation}
\left( C + C_m \right) u = F_0 .
\end{equation}
\noindent К~обычному оператору линейной упругости~$C$ добавилась магнитная жёсткость~$C_m$; $F_0$\:--- силы в~недоформированном состоянии.

Добавка~$C_m$ пропорциональна квадрату тока и~может стать весьма существенной в~магнитных системах с~сильным полем. Учёт её необходим и~при~недостаточной величине~$C$; в~номинальном режиме конструкция может держать нагрузку, но дополнительная нагрузка неблагоприятного направления может оказаться \inquotesx{невыносимой}[.]

Но особенно важна роль магнитной жёсткости в~задачах устойчивости. Поскольку магнитные силы потенциальны, матрица~$C_m$ симметрична, и~критические параметры могут быть найдены статическим методом Euler’а.

Как иллюстрацию рассмотрим простую задачу о~балке в~продольном магнитном поле. Балка располагается на~декартовой оси~$z$, концы~${z \narroweq 0}$ и~${z \narroweq l}$ закреплены, магнитная индукция ${\boldmathbb{B} = B\bm{k} = \boldconstant}$, по~балке течёт постоянный~(по~величине) ток~$I$. В~классической модели балки при~равных жёсткостях на~изгиб для~прогиба ${\bm{u} = u_x\bm{i} \hspace{.12ex} + u_y\bm{j}}$ легко получить следующую постановку:

...

Вводя компл\'{е}ксную комбинацию~${u \equiv u_x \hspace{-0.16ex} + \mathrm{i} u_y}$, будем иметь

...

\noindent с общим решением

...

\noindent Подстановка в~граничные условия приводит к~однородной системе для~постоянных~${A_k}$; приравняв нулю определитель, придём к~характеристическому уравнению

...


\noindent Наименьший положительный корень~${x_1 \hspace{-0.2ex} = 3.666}$, так~что критическая комбинация параметров такова:
\[
\left( I \hspace{-0.25ex} B \hspace{.1ex} {l^3} \hspace{-0.2ex} / a \hspace{.1ex} \right)_{\hspace{-0.32ex}*} \hspace{-0.32ex} = \hspace{.1ex} 394.2 \hspace{.12ex}.
\]

В~этом решении поле~$\boldmathbb{B}$ считалось внешним и~не~варьировалось. Но если собственное поле тока в~стержне сравнимо с~$\boldmathbb{B}$, решение изм\'{е}нится и~усложн\'{и}тся.

\end{otherlanguage}

\section*{\small \wordforbibliography}

\begin{changemargin}{\parindent}{0pt}
\fontsize{10}{12}\selectfont

\begin{otherlanguage}{russian}

Основы электродинамики хорошо изложены во~многих книгах~\cite{classicalelectrodynamics, feynman-lecturesonphysics}, но для~приложений в~механике выделяется курс И.\,Е.\;Тамма~\cite{tamm-electricity}. Растёт список литературы по~связанным задачам электромагнетизма и~упругости~\cite{parton-electromagneticelasticity, podstrigach.burak.kondrat-magnetothermoelasticity}. Как введение в~эту область может быть полезна книга В.\;Новацкого~\cite{nowacki-electromagneticeffects}.

\end{otherlanguage}

\end{changemargin}
