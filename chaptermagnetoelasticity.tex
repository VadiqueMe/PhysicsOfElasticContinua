\en{\chapter{Magnetoelasticity}}

\ru{\chapter{Магнитоупругость}}

\thispagestyle{empty}

\newcommand\chargedensity{\varrho} % {\uprho}
\newcommand\currentdensity{\bm{j}} % {\bm{\jmath}}

\newcommand\velocityofpoint{\mathdotabove{\locationvector}} % {\bm{v}}
%\bm{v} \equiv \mathdotabove{\locationvector}

\newcommand\vacuumpermittivity{\varepsilon_{\hspace{-0.1ex}\scalebox{.6}[.5]{$\raisemath{.1ex}{0}$}}}
\newcommand\vacuumpermeability{\mu_{\hspace{.05ex}\scalebox{.6}[.5]{$\raisemath{.1ex}{0}$}}}

\newcommand\maxwellstress{{^2\hspace{-0.25ex}\mathboldM}}

\newcommand\expminustwelve{\hspace{.16ex}\scalebox{.8}[1]{\hbox{--}}\raisemath{.1ex}{1\kern-0.07ex2}}

%% magneto-electro-elastic materials

\label{chapter:magnetoelasticity}

\begin{changemargin}{\parindent}{\parindent}
\vspace{-2em}
{\noindent\small

\en{Much in the~modern world}\ru{Многое в~современном мире}
\en{is built upon}\ru{построено на}
\en{the~theory of~electromagnetism}\ru{теории электромагнетизма}.
%
\en{This theory}\ru{Эта теория}
\en{was created}\ru{была создана}
\en{in~the}\ru{в}~\hbox{XIX$^{\textrm{\en{th}\ru{ом}}}$\hspace{-0.2ex}}~\en{century}\ru{веке}.
%
\en{Its creators}\ru{Её создатели}\footnote{\textcolor{red}{\foreignlanguage{english}{\textsc{þeir names here}}}}
\en{relied on}\ru{опирались на}
\en{the~experiments}\ru{эксперименты}
\en{with electric circuits}\ru{с~электрическими цепями}
\en{and}\ru{и}
\en{didn’t imagine}\ru{не~представляли}
\en{the~existence}\ru{о~существовании}
\en{of electromagnetic waves}\ru{электромагнитных волн}.
%
\en{Nevertheless}\ru{Тем не~менее},
\en{the~entities}\ru{сущности}\ru{,}
\en{describing}\ru{описывающие}
\en{electricity}\ru{электричество}
\en{and}\ru{и}~\en{magnetism}\ru{магнетизм}
\en{at each point}\ru{в~каждой точке}\ru{,}
\en{were introduced}\ru{были введены}
\en{as vectors}\ru{как векторы},
\en{along with }\ru{вместе с~}\en{the~differential equations}\ru{дифференциальными уравнениями}
\en{featuring these vectors}\ru{с~участием этих векторов}.
%
\en{This happened}\ru{Это случилось}
\en{due to}\ru{из\hbox{-}за}
\emph{\ru{эфира~}\ru{\hspace{-0.2ex}(}\en{the~\hspace{-0.2ex}}æther\ru{)}},
\en{because}\ru{ведь}
\en{the~creators of~the~theory}\ru{создатели теории}
\en{were convinced}\ru{были убеждены}
\en{of its existence}\ru{в~его существовании}
\en{and thus}\ru{и~поэтому}
\en{utilized}\ru{пользовались}
\en{the~concept of~it}\ru{этим концептом}.

\vspace{1ex}

%%\en{Problems}\ru{Проблемы},
%%\en{which}\ru{которые}
%%\en{are solved}\ru{решаются}
%%\en{using models}\ru{с~использованием моделей}
%%\en{of an~elastic continua}\ru{упругих \rucontinuum{}ов},
%%\en{usually}\ru{обыкновенно}
%%\en{include}\ru{включают}
%%\en{the~already known}\ru{уж\'{е} известные}
%%\en{external loads}\ru{внешние нагрузки}.
%%\en{How to~find}\ru{Как найти}
%%\en{these external loads}\ru{эти внешние нагрузки}\en{ is}\ru{\:---}
%%\en{another problem}\ru{другая проблема}.
%
\en{When}\ru{Когда}
\en{electric currents}\ru{электрические токи}
\en{flow}\ru{текут}
\en{in a~body}\ru{в~теле}~(\en{a~medium}\ru{среде}),
\en{the~magnetic field}\ru{магнитное поле}
\en{produces}\ru{производит}
\en{a~load}\ru{нагрузку},
\en{a~body}\ru{тело}
\en{deforms}\ru{деформируется},
\en{and this deformation}\ru{и~эта деформация}
\en{alters}\ru{изменяет}
\en{the~magnetic field itself}\ru{сам\'{о} магнитное поле}.
%
\en{If}\ru{Если}
\en{the~field}\ru{поле}
\en{is highly sensitive}\ru{высок\'{о} чувствительно}
\en{to deformations}\ru{к~деформациям},
\en{then}\ru{то}
\ru{возникает }\en{a~joint problem}\ru{совместная проблема}
\en{of elasticity}\ru{упругости}
\en{and magnetism}\ru{и~магнетизма}\en{ emerges}.

\par}
\vspace{-1em}
\end{changemargin}

\en{\section{Electromagnetic field}}

\ru{\section{Электромагнитное поле}}

\label{section:electromagneticfieldandwaves}

\dropcap{\en{H}\ru{В}}{\en{ere}\ru{от}}\en{ is}
\en{the~summary}\ru{краткое изложение}
\en{of the~theory of~electromagnetism}\ru{теории электромагнетизма}.

\noindent\hspace{\parindent}\hspace{1ex}
\en{This theory}\ru{Эта теория}
\en{describes}\ru{описывает}
\en{the two vector fields}\ru{два векторных поля},
\en{the electric one}\ru{электрическое}~${\hspace{.1ex}\boldmathbb{E}(\locationvector, t)}$
\en{and}\ru{и}~\en{the magnetic one}\ru{магнитное}~${\hspace{.1ex}\boldmathbb{B}(\locationvector, t)}$.
\en{What are}\ru{Что такое}
\en{vector}\ru{вектор}~${\hspace{.1ex}\boldmathbb{E}}$
\en{and}\ru{и}~\en{pseudo\-vector}\ru{псевдо\-вектор}~${\hspace{.1ex}\boldmathbb{B}}$
\en{can be}\ru{можно}
\en{figured out}\ru{понять}
\en{from}\ru{из}
\en{the~expression}\ru{выражения}
\en{of the~electromagnetic force}\ru{электромагнитной силы}
(\href{https://en.wikipedia.org/wiki/Lorentz_force}{\en{the Lorentz force}\ru{силы Lorentz’а}})
${\hspace{-0.1ex}\bm{F}(\locationvector, \mathdotabove{\locationvector}, t, q)}$\ru{,}
\en{acting on}\ru{действующей на}~\en{a~point-like charge}\ru{точечный заряд}\:---
\en{an~infinitesimal in size particle}\ru{частицу бесконечно-м\'{а}леньких размеров}\ru{,}
\en{that contains}\ru{содержащую}
\en{an~electric charge}\ru{электрический заряд}~$q$
\en{and}\ru{и}~\en{moves}\ru{движущуюся}
\en{with velocity}\ru{со~скоростью}~${\velocityofpoint}$

\nopagebreak\vspace{-0.15em}\begin{equation}\label{electromagneticforce}
\bm{F} = \hspace{.12ex} q \hspace{.2ex} \bigl(
\hspace{.1ex} \boldmathbb{E} \hspace{.2ex} + \hspace{.2ex} \velocityofpoint \hspace{-0.1ex} \times \hspace{-0.15ex} \boldmathbb{B}
\bigr)
\hspace{-0.1ex} .
\end{equation}

\vspace{-0.1em}\noindent
\en{In~essence}\ru{По~сути},
\en{the~part}\ru{часть}
\en{of the~electromagnetic force}\ru{электромагнитной силы}\ru{,}
\en{arising from an~interaction}\ru{возникающая от взаимо\-действия}
\en{with a~moving charge}\ru{с~движущимся зарядом}\:---
\en{the magnetic force}\ru{магнитная сила}~${q \hspace{.2ex} \velocityofpoint \hspace{-0.15ex} \times \hspace{-0.1ex} \boldmathbb{B}}$\:--- \en{reveals}\ru{являет}
\en{the magnetic field}\ru{магнитное поле}~${\hspace{.1ex}\boldmathbb{B}}$,
\en{while}\ru{в~то время как}
\en{the~other part}\ru{другая часть}\:---
\en{the electric force}\ru{электрическая сила}~${q \hspace{.2ex} \boldmathbb{E}}$\:---
\en{reveals}\ru{являет}
\en{the electric field}\ru{электрическое поле}~${\hspace{.1ex}\boldmathbb{E}}$.

\vspace{-0.1em}
\en{The acute question}\ru{Острый вопрос} \emph{\inquotes{%
\en{in~which exactly}\ru{в~какой~же именно} \en{frame of~reference}\ru{системе отсчёта} \en{is}\ru{измеряется} \en{velocity}\ru{скорость}~${\velocityofpoint}$ \en{of a~charged particle}\ru{заряжённой частицы}\en{ measured}?%
}}
\en{leads to}\ru{ведёт к}~\href{https://en.wikipedia.org/wiki/Special_relativity}{\en{the~special theory of relativity}\ru{специальной теории относительности}}%
\footnote{%
\bookauthor{Albert Einstein}.
\href{http://users.physik.fu-berlin.de/~kleinert/files/1905_17_891-921.pdf}{Zur Elektrodynamik bewegter Körper.~//
Annalen der~Physik, IV\kern-0.33ex.\:Folge, Band~17, 1905. Seiten 891\hbox{--}921.}
}\hbox{\hspace{-0.55ex}.}

the~classical concept about the~existence of~absolute space and~time as the~preferred frame of~reference

a~vacuum is space without matter, \inquotes{free space}

...

\en{continuous charge distribution}\ru{непрерывное распределение заряда}

\en{a~charge density}\ru{плотность заряда}~${\chargedensity\hspace{.1ex}(\locationvector, t)}$
\en{is}\ru{это}
\en{an~electric charge}\ru{электрический заряд}
\en{per volume unit}\ru{на единицу объёма}

\nopagebreak\vspace{-0.2em}\begin{equation*}
dq = \chargedensity \hspace{.15ex} d \mathcal{V}
, \hspace{.5em}
q = \hspace{-0.5ex} \scalebox{0.9}{$\displaystyle\integral\displaylimits_{\mathcal{V}}$} \hspace{-0.3ex} \chargedensity \hspace{.15ex} d\mathcal{V}
\end{equation*}

\en{the~density of~electric current}\ru{плотность электрического тока}
\en{is}\ru{есть}
\en{the~flux of~electric charge}\ru{поток электрического заряда}
${%
   \currentdensity(\locationvector, t) \hspace{-0.1ex}
   \equiv
   \chargedensity \hspace{.15ex} \mathdotabove{\locationvector} %%\equiv \chargedensity \hspace{.15ex} \bm{v}
}$

\foreignlanguage{russian}{в ваку\kern-0.11exуме нет}
$\chargedensity$ \en{and}\ru{и}~$\currentdensity$

in a~region with no charges (${\chargedensity = 0}$) and no currents (${\currentdensity = \bm{0}}$), such as a~vacuum, ...

...

\en{the ponderomotive force}\ru{пондеромоторная сила} is the electromagnetic force (the Lorentz force) per volume unit

\en{In}\ru{В}
\en{a~material continuum}\ru{материальном \rucontinuum{}е}
\en{with }\ru{с~}\en{charges}\ru{зарядами}
\en{and}\ru{и}~\en{currents}\ru{токами}
\en{acts}\ru{действует}
\en{the~}\en{volume}\ru{объёмная}
\ru{\inquotes{пондеромоторная} (}\inquotes{ponderomotive}\ru{)} \en{force}\ru{сила}

\nopagebreak\vspace{-0.2em}\en{\vspace{-0.7em}}\begin{equation}\label{ponderomotive.electromagneticforcepervolumeunit}
\bm{f} = \chargedensity \hspace{.16ex} \boldmathbb{E} \hspace{.1ex} + \hspace{.1ex} \currentdensity \hspace{-0.16ex} \times \hspace{-0.2ex} \boldmathbb{B}
\hspace{.1ex} .
\end{equation}

\vspace{-0.22em}\noindent
\en{It is}\ru{это}
\en{the~differential}\ru{дифференциальная}
(\en{the~local}\ru{локальная},
\en{the~microscopic}\ru{микроскопическая},
\en{the~continual}\ru{континуальная})
\en{version of}\ru{версия}~\eqref{electromagneticforce}.

...

% Maxwell’s equations

\en{Electromagnetic phenomena}\ru{Электромагнитные явления}
\en{are usually described}\ru{обыкновенно описываются}
\ru{уравнениями }\en{by }\en{the }\href{https://en.wikipedia.org/wiki/James_Clerk_Maxwell}{Maxwell}’\en{s}\ru{а}\en{ equations}.
\en{Differential}\ru{Дифференциальные}
\en{versions}\ru{версии}
\en{of~}\en{these equations}\ru{этих уравнений}
\en{are}\ru{суть}

\nopagebreak\vspace{.2em}
\refstepcounter{equation}
\begin{align*}
\hspace*{5em} \boldnabla \dotp \hspace{.12ex} \boldmathbb{E} &= \smash{\displaystyle \frac{\chargedensity}{\raisemath{.16em}{\vacuumpermittivity}}}
\hspace{.1em} &
\pushright{\scalebox{0.9}{\begin{minipage}[c]{.5\textwidth}\setstretch{0.96}
\ru{теорема }Gauss’\en{s}\ru{а}\en{ theorem} \en{for elec\-tric\-i\-ty}\ru{для электричества}
\end{minipage}}}
\tag{$\theequation^{\raisemath{.15em}{\alpha}}$}\label{maxwell.equations:gauss.electric}
\\[.25em]
%
\boldnabla \hspace{-0.15ex} \times \hspace{-0.1ex} \boldmathbb{E} &= - \hspace{.2ex} \mathdotabove{\boldmathbb{B}}
\hspace{.1em} &
\pushright{\scalebox{0.9}{\begin{minipage}[c]{.5\textwidth}\setstretch{0.96}
\ru{уравнение }Maxwell\ru{’а}--Faraday\ru{’я}\en{ equation}\\
(\ru{закон индукции }Faraday’\en{s}\ru{я}\en{ law of~induction})
\end{minipage}}}
\tag{$\theequation^{\raisemath{.15em}{\beta}}$}\label{maxwell.equations:faraday.induction}
\\[.25em]
%
\boldnabla \dotp \hspace{.12ex} \boldmathbb{B} &= \hspace{.1ex} 0
\hspace{.1em} &
\pushright{\scalebox{0.9}{\begin{minipage}[c]{.5\textwidth}\setstretch{0.96}
\ru{теорема }Gauss’\en{s}\ru{а}\en{ theorem} \en{for mag\-net\-ism}\ru{для магнетизма}
\end{minipage}}}
\tag{$\theequation^{\raisemath{.15em}{\gamma}}$}\label{maxwell.equations:gauss.magnetic}
\\[.25em]
%
c^{\hspace{.16ex}2} \hspace{.22ex} \boldnabla \hspace{-0.15ex} \times \hspace{-0.1ex} \boldmathbb{B} &= \smash{\displaystyle \frac{\currentdensity}{\raisemath{.16em}{\vacuumpermittivity}} + \hspace{.1ex} \mathdotabove{\boldmathbb{E}}}
\hspace{.1em} &
\pushright{\scalebox{0.9}{\begin{minipage}[c]{.5\textwidth}\setstretch{0.96}\raggedright
\ru{циркуляционный закон }Ampère’\en{s}\ru{а}\en{ circuital law}
\en{with }\ru{со~}\ru{слагаемым }Maxwell’\en{s}\ru{а}\en{ term}~${\smash{\mathdotabove{\boldmathbb{E}}}}$
\en{for the~balance of electric charge}\ru{для баланса электрического заряда}
\end{minipage}}}
\tag{$\theequation^{\raisemath{.15em}{\delta}}$}\label{maxwell.equations:ampere.circuital.with.addition}
\end{align*}


....


\en{speed of~light in vacuum}\ru{скорость света в~ваку\kern-0.11exуме}
${c = 299\:792\:458}$~\raisebox{.3em}{\en{m}\ru{м}}\hspace{-0.2ex}/\hspace{-0.25ex}\raisebox{-0.2em}{\en{s}\ru{с}}

\inquotesx{electric constant}[,] \href{https://en.wikipedia.org/wiki/Vacuum_permittivity}{vacuum permittivity} $\vacuumpermittivity$

${\vacuumpermittivity \hspace{-0.12ex} \approx 8.8541878 \hspace{-0.25ex}\cdot\hspace{-0.3ex} 10^{\expminustwelve}}$~${\text{\en{F}\ru{Ф}} \hspace{-0.33ex}\cdot\hspace{-0.33ex} \text{\en{m}\ru{м}}^{\hspace{-0.2ex}\expminusone}}$~(\href{https://en.wikipedia.org/wiki/Farad}{\en{farads}\ru{фарад}} \en{per~metre}\ru{на~метр})

\inquotesx{magnetic constant}[,] \href{https://en.wikipedia.org/wiki/Vacuum_permeability}{vacuum permeability}~${\vacuumpermeability \hspace{-0.24ex} = \hspace{-0.08ex} \smash{\scalebox{0.9}{$\displaystyle \frac{\raisemath{-0.16em}{1}}{\raisemath{-0.1em}{\vacuumpermittivity \hspace{.1ex} c^{\hspace{.16ex}2}}}$}}}$

...

\en{With}\ru{С}~$\vacuumpermeability$ \en{equation}\ru{уравнение}~\eqref{maxwell.equations:ampere.circuital.with.addition} \en{is sometimes written as}\ru{иногда пишется как}

\nopagebreak\vspace{-0.3em}\begin{equation*}
\boldnabla \hspace{-0.15ex} \times \hspace{-0.1ex} \boldmathbb{B}
= \vacuumpermeability \hspace{.2ex} \currentdensity + \vacuumpermeability \hspace{.1ex} \vacuumpermittivity \hspace{.1ex} \mathdotabove{\boldmathbb{E}}
\hspace{.88em}\text{\en{or}\ru{или}}\hspace{.8em}
\boldnabla \hspace{-0.15ex} \times \hspace{-0.1ex} \boldmathbb{B}
= \vacuumpermeability \hspace{.2ex} \currentdensity + \hspace{.1ex} \smash{\raisemath{.07em}{\scalebox{0.84}{$\displaystyle \frac{\raisemath{-0.16em}{1}}{\raisemath{-0.1em}c^{\hspace{.16ex}2}}$}}} \hspace{.1ex} \mathdotabove{\boldmathbb{E}}
\hspace{.2ex} .
\end{equation*}

...

% balance of charge

\en{The~balance of electric charge}\ru{Баланс электрического заряда}\:--- \href{https://en.wikipedia.org/wiki/Continuity_equation}{\en{the~continuity equation}\ru{уравнение непрерывности (сплошности, неразрывности)}} \en{for}\ru{для} \en{electric charge}\ru{электрического заряда}\:--- \en{mathematically follows}\ru{математически следует} \en{from}\ru{из} \ru{уравнений }Maxwell’\en{s}\ru{а}\en{ equations}

\nopagebreak\vspace{.2em}\begin{equation*}
\left.\begin{array}{r@{\hspace{.6em}}c@{\hspace{.6em}}l}
\boldnabla \hspace{-0.1ex} \dotp \hspace{-0.1ex} \eqref{maxwell.equations:ampere.circuital.with.addition}
& \Rightarrow &
c^{\hspace{.16ex}2} \hspace{.22ex} \boldnabla \dotp \hspace{-0.1ex} \bigl( \boldnabla \hspace{-0.15ex} \times \hspace{-0.1ex} \boldmathbb{B} \hspace{.1ex} \bigr) \hspace{-0.1ex}
= \hspace{.1ex} \smash{\displaystyle \frac{\boldnabla \hspace{-0.15ex} \dotp \currentdensity}{\raisemath{.16em}{\vacuumpermittivity}}} \hspace{.1ex} + \boldnabla \dotp \hspace{.1ex} \mathdotabove{\boldmathbb{E}}
\\[.5em]
%
\eqref{maxwell.equations:gauss.electric}^{\hspace{-0.05ex}\tikz[baseline=-0.4ex]\draw[black, fill=black] (0,0) circle (.28ex);}
& \Rightarrow &
\boldnabla \dotp \hspace{.1ex} \mathdotabove{\boldmathbb{E}} \hspace{.1ex} = \smash{\displaystyle \frac{\mathdotabove{\chargedensity}}{\raisemath{.16em}{\vacuumpermittivity}}}
\\[.64em]
%
& &
\boldnabla \hspace{-0.1ex} \dotp \hspace{-0.1ex} \bigl( \boldnabla \hspace{-0.2ex} \times \hspace{-0.16ex} \bm{a} \bigr) \hspace{-0.3ex} = 0 \:\:\forall \bm{a}
\hspace{.1ex} , \hspace{.55em}
\currentdensity \equiv \chargedensity \hspace{.1ex} \mathdotabove{\locationvector}
\end{array}\hspace{.1ex}\right\rbrace
\hspace{.5ex}\Rightarrow\hspace*{4em}
\end{equation*}
\nopagebreak\vspace{-0.1em}\begin{equation}\label{continuity.balanceofelectriccharge}
\hspace*{1em}\Rightarrow\hspace{.7ex}
\boldnabla \hspace{-0.15ex} \dotp \hspace{-0.15ex} \bigl( \chargedensity \hspace{.1ex} \mathdotabove{\locationvector} \hspace{.1ex} \bigr) \hspace{-0.3ex} + \mathdotabove{\chargedensity} \hspace{.1ex} = 0
\hspace{.1ex} .
\vspace{-0.5em}\end{equation}

% Maxwell stress tensor

\en{An~electro\-magnetic field}\ru{Электро\-магнитное поле}
\en{acts}\ru{действует}
\en{upon a~continuum}\ru{на~\rucontinuum}
\en{with }\ru{с~}\en{the~vector}\ru{вектором}
\en{of a~ponderomotive force}\ru{пондеромоторной силы}~\eqref{ponderomotive.electromagneticforcepervolumeunit}.

\en{But}\ru{Но}
\en{there’s}\ru{есть}
\en{also}\ru{также}
\en{another expression of~interaction}\ru{и~другое выражение взаимо\-действия},
\en{the~bivalent}\ru{бивалентный}
\ru{тензор напряжения }\href{https://en.wikipedia.org/wiki/Maxwell_stress_tensor}{\inquotes{Maxwell\ru{’а}}\en{ stress tensor}}

\nopagebreak\vspace{\en{-0.1em}\ru{-0.25em}}
\begin{equation}\label{maxwellstresstensor:definition}
\maxwellstress \equiv \hspace{.12ex}
\vacuumpermittivity \hspace{-0.33ex} \left(
\boldmathbb{E} \hspace{.1ex} \boldmathbb{E} + c^{\hspace{.16ex}2} \hspace{.22ex} \boldmathbb{B} \boldmathbb{B}
- \hspace{.1ex} \smalldisplaystyleonehalf \hspace{.1ex} \bigl( \hspace{.1ex} \boldmathbb{E} \hspace{-0.1ex} \dotp \hspace{-0.15ex} \boldmathbb{E} + c^{\hspace{.16ex}2} \hspace{.22ex} \boldmathbb{B} \hspace{-0.2ex} \dotp \hspace{-0.16ex} \boldmathbb{B} \bigr) \hspace{-0.06ex} \UnitDyad \hspace{.05ex}
\right)
\hspace{-0.22em} .
\end{equation}

\vspace{-0.2em}\noindent
\en{It derives}\ru{Он выводится} \en{from}\ru{из}~\eqref{ponderomotive.electromagneticforcepervolumeunit} \en{and}\ru{и} \ru{уравнений }Maxwell’\en{s}\ru{а}\en{ equations}

\nopagebreak\vspace{-0.1em}\begin{align*}
\eqref{maxwell.equations:gauss.electric}
& \hspace{.27em}\Rightarrow\hspace{.35em}
\chargedensity = \vacuumpermittivity \boldnabla \hspace{-0.1ex} \dotp \boldmathbb{E}
\\[-0.1em]
%
\eqref{maxwell.equations:ampere.circuital.with.addition}
& \hspace{.27em}\Rightarrow\hspace{.35em}
\currentdensity = \vacuumpermittivity \hspace{.1ex} c^{\hspace{.16ex}2} \hspace{.22ex} \boldnabla \hspace{-0.2ex} \times \hspace{-0.2ex} \boldmathbb{B} - \vacuumpermittivity \hspace{.1ex} \mathdotabove{\boldmathbb{E}}
\end{align*}
%
\nopagebreak\vspace{-1.2em}\begin{multline*}
\eqref{ponderomotive.electromagneticforcepervolumeunit}
\hspace{.27em}\Rightarrow\hspace{.35em}
\bm{f} = \hspace{.1ex} \vacuumpermittivity \hspace{-0.1ex} \boldnabla \hspace{-0.1ex} \dotp \boldmathbb{E} \hspace{.16ex} \boldmathbb{E} \hspace{.2ex}
+ \bigl( \vacuumpermittivity \hspace{.1ex} c^{\hspace{.16ex}2} \hspace{.22ex} \boldnabla \hspace{-0.2ex} \times \hspace{-0.2ex} \boldmathbb{B} - \vacuumpermittivity \hspace{.1ex} \mathdotabove{\boldmathbb{E}} \hspace{.1ex} \bigr) \hspace{-0.44ex} \times \hspace{-0.2ex} \boldmathbb{B} =
\\[-0.15em]
%
= \hspace{.1ex} \vacuumpermittivity \hspace{-0.1ex} \boldnabla \hspace{-0.1ex} \dotp \boldmathbb{E} \hspace{.16ex} \boldmathbb{E} \hspace{.2ex}
+ \hspace{.1ex} \vacuumpermittivity \hspace{.1ex} c^{\hspace{.16ex}2} \bigl( \boldnabla \hspace{-0.2ex} \times \hspace{-0.2ex} \boldmathbb{B} \hspace{.1ex} \bigr) \hspace{-0.44ex} \times \hspace{-0.2ex} \boldmathbb{B} \hspace{.1ex}
- \hspace{.1ex} \vacuumpermittivity \hspace{.1ex} \mathdotabove{\boldmathbb{E}} \hspace{-0.15ex} \times \hspace{-0.2ex} \boldmathbb{B}
\end{multline*}

\vspace{-0.1em}\begin{equation*}
\left.
\begin{array}{c}
\bigl( \hspace{.1ex} \boldmathbb{E} \times \hspace{-0.15ex} \boldmathbb{B} \hspace{.1ex} \bigr)^{\hspace{-0.1em}\tikz[baseline=-0.4ex]\draw[black, fill=black] (0,0) circle (.28ex);} \hspace{-0.15ex}
= \hspace{.1ex} \mathdotabove{\boldmathbb{E}} \times \hspace{-0.15ex} \boldmathbb{B} \hspace{.1ex} + \hspace{.1ex} \boldmathbb{E} \times \hspace{-0.15ex} \mathdotabove{\boldmathbb{B}}
\\[.15em]
\eqref{maxwell.equations:faraday.induction}
\hspace{.27em}\Rightarrow\hspace{.33em}
\mathdotabove{\boldmathbb{B}} = - \boldnabla \hspace{-0.2ex} \times \hspace{-0.2ex} \boldmathbb{E}
\end{array}
\hspace{-0.1em} \right\rbrace
%
\hspace{.1em} \Rightarrow \hspace{.22em}
%
\mathdotabove{\boldmathbb{E}} \times \hspace{-0.15ex} \boldmathbb{B} \hspace{.1ex}
= \hspace{-0.1ex} \bigl( \hspace{.1ex} \boldmathbb{E} \times \hspace{-0.15ex} \boldmathbb{B} \hspace{.1ex} \bigr)^{\hspace{-0.1em}\tikz[baseline=-0.4ex]\draw[black, fill=black] (0,0) circle (.28ex);} \hspace{-0.15ex}
+ \hspace{.1ex} \boldmathbb{E} \hspace{-0.1ex} \times \hspace{-0.4ex} \bigl( \boldnabla \hspace{-0.2ex} \times \hspace{-0.2ex} \boldmathbb{E} \hspace{.1ex} \bigr)
\end{equation*}

\noindent
\en{Then}\ru{Тогда}

\nopagebreak\vspace{-0.2em}\begin{equation*}
\bm{f} = \hspace{.1ex} \vacuumpermittivity \hspace{-0.1ex} \boldnabla \hspace{-0.1ex} \dotp \boldmathbb{E} \hspace{.1ex} \boldmathbb{E} \hspace{.2ex}
- \hspace{.1ex} \vacuumpermittivity \hspace{.1ex} c^{\hspace{.16ex}2} \hspace{.22ex} \boldmathbb{B} \hspace{-0.2ex} \times \hspace{-0.4ex} \bigl( \boldnabla \hspace{-0.2ex} \times \hspace{-0.2ex} \boldmathbb{B} \hspace{.1ex} \bigr) \hspace{-0.2ex}
- \hspace{.1ex} \vacuumpermittivity \hspace{.12ex} \boldmathbb{E} \hspace{-0.15ex} \times \hspace{-0.4ex} \bigl( \boldnabla \hspace{-0.2ex} \times \hspace{-0.2ex} \boldmathbb{E} \hspace{.1ex} \bigr) \hspace{-0.2ex}
- \hspace{.1ex} \vacuumpermittivity \hspace{-0.07ex} \bigl( \hspace{.1ex} \boldmathbb{E} \times \hspace{-0.15ex} \boldmathbb{B} \hspace{.1ex} \bigr)^{\hspace{-0.1em}\tikz[baseline=-0.4ex]\draw[black, fill=black] (0,0) circle (.28ex);} \hspace{-0.15ex}
\end{equation*}

\vspace{-0.1em}\noindent
\en{For}\ru{Для} \en{the~symmetry}\ru{симметрии} \en{with}\ru{с}~${\boldnabla \hspace{-0.1ex} \dotp \boldmathbb{E} \hspace{.1ex} \boldmathbb{E}}$, \en{the~null vector}\ru{нуль\hbox{-}вектор}

\nopagebreak\vspace{-0.1em}\begin{align*}
\eqref{maxwell.equations:gauss.magnetic}
& \hspace{.27em}\Rightarrow\hspace{.25em}
\boldnabla \hspace{-0.1ex} \dotp \boldmathbb{B} \boldmathbb{B} = \hspace{.1ex} \bm{0}
\hspace{.1ex} , \hspace{.4em}
c^{\hspace{.16ex}2} \hspace{.22ex} \boldnabla \hspace{-0.1ex} \dotp \boldmathbb{B} \boldmathbb{B} = \hspace{.1ex} \bm{0}
\end{align*}

\nopagebreak\vspace{-0.2em}\noindent
\en{is added}\ru{добавляется} \en{to}\ru{к}~${\bm{f}\hspace{-0.1ex}}$.

...

\begin{align*}
\eqrefwithchapterdotsection{divergenceofdyadicproducoftwovectors}{chapter:mathapparatus}{section:spatialdifferentiationoftensorfields}
& \hspace{.27em}\Rightarrow\hspace{.32em}
\boldnabla \hspace{-0.16ex} \dotp \hspace{-0.2ex} \bigl( \hspace{-0.05ex} \bm{a} \bm{a} \bigr) \hspace{-0.33ex}
= \hspace{-0.1ex} \bigl( \boldnabla \hspace{-0.18ex} \dotp \hspace{-0.1ex} \bm{a} \bigr) \hspace{-0.1ex} \bm{a} \hspace{.1ex} + \bm{a} \hspace{-0.07ex} \dotp \hspace{-0.2ex} \boldnabla \hspace{-0.1ex} \bm{a}
\\[-0.1em]
%
\eqrefwithchapterdotsection{gradientofdotproductoftwovectors}{chapter:mathapparatus}{section:spatialdifferentiationoftensorfields}
& \hspace{.27em}\Rightarrow\hspace{.32em}
\boldnabla \hspace{.1ex} \bigl( \hspace{-0.1ex} \bm{a} \hspace{-0.17ex} \dotp \hspace{-0.2ex} \bm{a} \bigr) \hspace{-0.33ex}
= 2 \hspace{.16ex} \boldnabla \hspace{-0.1ex} \bm{a} \hspace{-0.12ex} \dotp \hspace{-0.15ex} \bm{a}
\end{align*}

\begin{equation*}
\boldnabla \hspace{-0.16ex} \dotp \hspace{-0.12ex} \bigl( \hspace{-0.05ex} \bm{a} \hspace{-0.17ex} \dotp \hspace{-0.2ex} \bm{a} \hspace{.1ex} \UnitDyad \hspace{.12ex} \bigr) \hspace{-0.33ex}
= \hspace{-0.15ex} \boldnabla \hspace{-0.2ex} \dotp \hspace{-0.22ex}  \UnitDyad \hspace{.13ex} \bigl( \hspace{-0.1ex} \bm{a} \hspace{-0.17ex} \dotp \hspace{-0.2ex} \bm{a} \bigr) \hspace{-0.33ex}
= \hspace{-0.15ex} \boldnabla \hspace{.1ex} \bigl( \hspace{-0.1ex} \bm{a} \hspace{-0.17ex} \dotp \hspace{-0.2ex} \bm{a} \bigr) \hspace{-0.33ex}
= 2 \hspace{.16ex} \boldnabla \hspace{-0.1ex} \bm{a} \hspace{-0.12ex} \dotp \hspace{-0.15ex} \bm{a}
\end{equation*}

\begin{multline*}
\boldnabla \hspace{-0.05ex} \dotp \hspace{-0.1ex} \maxwellstress =
\hspace{.1ex} \vacuumpermittivity \Bigl( \hspace{-0.2ex}
\boldnabla \hspace{-0.1ex} \dotp \hspace{-0.1ex} \bigl( \boldmathbb{E} \hspace{.1ex} \boldmathbb{E} \hspace{.1ex} \bigr) \hspace{-0.22ex}
+ \hspace{.1ex} c^{\hspace{.16ex}2} \hspace{.22ex} \boldnabla \hspace{-0.1ex} \dotp \hspace{-0.1ex} \bigl( \boldmathbb{B} \boldmathbb{B} \bigr) \hspace{-0.22ex}
- \hspace{.2ex} \smalldisplaystyleonehalf \hspace{.1ex} \boldnabla \hspace{-0.12ex} \dotp \hspace{-0.15ex} \bigl( \hspace{.1ex} \boldmathbb{E} \hspace{-0.1ex} \dotp \hspace{-0.15ex} \boldmathbb{E} \hspace{.1ex} \UnitDyad \hspace{.1ex}
+ c^{\hspace{.16ex}2} \hspace{.22ex} \boldmathbb{B} \hspace{-0.2ex} \dotp \hspace{-0.16ex} \boldmathbb{B} \UnitDyad \hspace{.15ex} \bigr) \hspace{-0.2ex}
\Bigr) \hspace{-0.33ex} =
\\[-0.25em]
%
= \hspace{.1ex} \vacuumpermittivity \Bigl( \hspace{-0.2ex}
\boldnabla \hspace{-0.15ex} \dotp \boldmathbb{E} \hspace{.1ex} \boldmathbb{E} \hspace{.1ex}
+ \boldmathbb{E} \dotp \hspace{-0.22ex} \boldnabla \hspace{.2ex} \boldmathbb{E} \hspace{.1ex}
+ c^{\hspace{.16ex}2} \hspace{.22ex} \boldnabla \hspace{-0.15ex} \dotp \boldmathbb{B} \boldmathbb{B}
+ c^{\hspace{.16ex}2} \hspace{.22ex} \boldmathbb{B} \hspace{-0.1ex} \dotp \hspace{-0.22ex} \boldnabla \hspace{.2ex} \boldmathbb{B} \Bigr. \hspace{.3ex} -
\\[-0.55em]
%
- \hspace{-0.2ex} \Bigl. \boldnabla \hspace{.2ex} \boldmathbb{E} \hspace{-0.1ex} \dotp \hspace{-0.15ex} \boldmathbb{E} \hspace{.1ex}
- c^{\hspace{.16ex}2} \hspace{.22ex} \boldnabla \hspace{.2ex} \boldmathbb{B} \hspace{-0.2ex} \dotp \hspace{-0.16ex} \boldmathbb{B} \hspace{-0.1ex}
\Bigr)
\end{multline*}

...

\en{\section{Electromagnetic waves}}

\ru{\section{Электромагнитные волны}}

To derive wave equations

\nopagebreak\begin{equation*}
\begin{array}{r@{\hspace{.5em}}c@{\hspace{.5em}}r@{\hspace{.3em}}c@{\hspace{.3em}}l}
\boldnabla \hspace{-0.25ex} \times \hspace{-0.25ex} \eqref{maxwell.equations:faraday.induction}
& \Rightarrow &
\boldnabla \hspace{-0.2ex} \times \hspace{-0.3ex} \bigl( \boldnabla \hspace{-0.15ex} \times \hspace{-0.1ex} \boldmathbb{E} \hspace{.1ex} \bigr) \hspace*{-0.3ex} &
= &
- \hspace{.1ex} \boldnabla \hspace{-0.15ex} \times \hspace{-0.1ex} \mathdotabove{\boldmathbb{B}}
\\[.1em]
%
\boldnabla \hspace{-0.25ex} \times \hspace{-0.25ex} \eqref{maxwell.equations:ampere.circuital.with.addition}
& \Rightarrow &
\boldnabla \hspace{-0.2ex} \times \hspace{-0.3ex} \bigl( \boldnabla \hspace{-0.15ex} \times \hspace{-0.1ex} \boldmathbb{B} \bigr) \hspace*{-0.3ex} &
= &
\boldnabla \hspace{-0.12ex} \times \hspace{-0.33ex} \biggl( \displaystyle \frac{\currentdensity}{\raisemath{-0.1em}{ \vacuumpermittivity \hspace{.1ex} c^{\hspace{.16ex}2} }} + \hspace{.1ex} \displaystyle \frac{\raisemath{-0.12em}{ \mathdotabove{\boldmathbb{E}} }}{\raisemath{-0.1em}{ c^{\hspace{.16ex}2} }} \biggr)
\end{array}
\end{equation*}

\nopagebreak\begin{equation*}
\begin{array}{r@{\hspace{.4em}}r@{\hspace{.5em}}c@{\hspace{.5em}}l}
%
\boldnabla \hspace{-0.15ex} \times \hspace{-0.1ex} \mathdotabove{\boldmathbb{E}} \hspace{.1ex}
= \hspace{-0.1ex} \bigl( \boldnabla \hspace{-0.15ex} \times \hspace{-0.1ex} \boldmathbb{E} \hspace{.1ex} \bigr)^{\hspace{-0.2ex}\tikz[baseline=-0.4ex]\draw[black, fill=black] (0,0) circle (.28ex);}
\hspace{-0.33ex} , &
\eqref{maxwell.equations:faraday.induction}
& \Rightarrow &
\boldnabla \hspace{-0.15ex} \times \hspace{-0.1ex} \mathdotabove{\boldmathbb{E}} \hspace{.1ex}
= - \hspace{.15ex} \mathdotdotabove{\boldmathbb{B}}
\\[.1em]
%
\boldnabla \hspace{-0.15ex} \times \hspace{-0.1ex} \mathdotabove{\boldmathbb{B}}
= \hspace{-0.1ex} \bigl( \boldnabla \hspace{-0.15ex} \times \hspace{-0.1ex} \boldmathbb{B} \bigr)^{\hspace{-0.2ex}\tikz[baseline=-0.4ex]\draw[black, fill=black] (0,0) circle (.28ex);}
\hspace{-0.33ex} , &
\eqref{maxwell.equations:ampere.circuital.with.addition}
& \Rightarrow &
\boldnabla \hspace{-0.15ex} \times \hspace{-0.1ex} \mathdotabove{\boldmathbb{B}}
= \displaystyle \frac{\currentdensity^{\hspace{.1ex}\tikz[baseline=-0.36ex]\draw[black, fill=black] (0,0) circle (.28ex);}}{\raisemath{-0.1em}{ \vacuumpermittivity \hspace{.1ex} c^{\hspace{.16ex}2} }} + \hspace{.1ex} \displaystyle \frac{\raisemath{-0.12em}{ \mathdotdotabove{\boldmathbb{E}} }}{\raisemath{-0.1em}{ c^{\hspace{.16ex}2} }}
\end{array}
\end{equation*}

${
\boldnabla \hspace{-0.2ex} \times \hspace{-0.3ex} \bigl( \boldnabla \hspace{-0.15ex} \times \hspace{-0.1ex} \boldmathbb{E} \hspace{.1ex} \bigr) \hspace{-0.25ex}
= \hspace{-0.1ex} \boldnabla \boldnabla \dotp \hspace{.12ex} \boldmathbb{E} \hspace{.1ex} - \Laplacian \hspace{.05ex} \boldmathbb{E}
}$

${
\boldnabla \hspace{-0.2ex} \times \hspace{-0.3ex} \bigl( \boldnabla \hspace{-0.15ex} \times \hspace{-0.1ex} \boldmathbb{B} \bigr) \hspace{-0.25ex}
= \hspace{-0.1ex} \boldnabla \boldnabla \dotp \hspace{.12ex} \boldmathbb{B} \hspace{.1ex} - \Laplacian \hspace{.05ex} \boldmathbb{B}
}$

\begin{align*}
\Laplacian \hspace{.05ex} \boldmathbb{E} - \hspace{-0.15ex} \boldnabla \tikzmark{beginElectricGaussInWaves} \boldnabla \dotp \hspace{.12ex} \boldmathbb{E} \tikzmark{endElectricGaussInWaves} \hspace{.2ex}
&= %%\hspace{-0.1ex} \boldnabla \hspace{-0.15ex} \times \hspace{-0.1ex} \mathdotabove{\boldmathbb{B}}
\hspace{.1ex} \displaystyle\frac{\raisemath{-0.12em}{ \mathdotdotabove{\boldmathbb{E}} }}{\raisemath{-0.1em}{ c^{\hspace{.16ex}2} }}
+ \hspace{.1ex} \displaystyle\frac{\currentdensity^{\hspace{.1ex}\tikz[baseline=-0.36ex]\draw[black, fill=black] (0,0) circle (.28ex);}}{\raisemath{-0.1em}{ \vacuumpermittivity \hspace{.1ex} c^{\hspace{.16ex}2} }}
\\[.2em]
%
\Laplacian \hspace{.05ex} \boldmathbb{B} - \hspace{-0.15ex} \boldnabla \tikzmark{beginMagneticGaussInWaves} \boldnabla \dotp \hspace{.12ex} \boldmathbb{B} \tikzmark{endMagneticGaussInWaves} \hspace{.1ex}
&=
\hspace{.1ex} \displaystyle\frac{\raisemath{-0.12em}{ \mathdotdotabove{\boldmathbb{B}} }}{\raisemath{-0.1em}{ c^{\hspace{.16ex}2} }}
- \hspace{.1ex} \displaystyle\frac{ \boldnabla \hspace{-0.33ex} \times \hspace{-0.25ex} \currentdensity }{\raisemath{-0.1em}{ \vacuumpermittivity \hspace{.1ex} c^{\hspace{.16ex}2} }}
\end{align*}%
\AddUnderBrace[line width=.75pt][0.2ex, 0.1ex][xshift=0.55em, yshift=.25ex]{beginElectricGaussInWaves}{endElectricGaussInWaves}{${%
\scalebox{0.77}{$ \raisemath{.36em}{\chargedensity} \hspace{-0.25ex} / \hspace{-0.3ex} \raisemath{-0.22em}{\vacuumpermittivity
} \hspace{.4ex}\eqref{maxwell.equations:gauss.electric} $}%
}$}%
\AddUnderBrace[line width=.75pt][0.2ex, 0.1ex][xshift=0.75em, yshift=.3ex]{beginMagneticGaussInWaves}{endMagneticGaussInWaves}{${%
\scalebox{0.77}{$ 0 \hspace{.7ex}\eqref{maxwell.equations:gauss.magnetic} $}%
}$}

...

% potential formulation

${\boldnabla \hspace{-0.1ex} \dotp \hspace{-0.1ex} \bigl( \boldnabla \hspace{-0.2ex} \times \hspace{-0.16ex} \bm{a} \bigr) \hspace{-0.3ex} = 0 \:\:\forall \bm{a}}$

${
\boldnabla \hspace{-0.2ex} \times \hspace{-0.28ex} \boldnabla = \hspace{.08ex} \bm{0}
\hspace{.1ex} , \hspace{.33em}
\boldnabla \hspace{-0.15ex} \times \hspace{-0.25ex} \boldnabla \alpha = \bm{0} \:\:\forall \alpha
}$

vector potential $\boldmathbb{A}$

\nopagebreak\[
\boldnabla \dotp \hspace{.12ex} \boldmathbb{B} = \hspace{.1ex} 0
\;\;\eqref{maxwell.equations:gauss.magnetic}
\hspace{.4em} \Leftrightarrow \hspace{.4em}
\boldmathbb{B} =  \hspace{-0.1ex} \boldnabla \hspace{-0.15ex} \times \hspace{-0.12ex} \boldmathbb{A}
\]

potential~${\boldmathbb{A}}$ is not unique and has gauge freedom ${\boldmathbb{A} + \hspace{-0.2ex} \boldnabla a}$

\[
\boldmathbb{B} =  \hspace{-0.1ex} \boldnabla \hspace{-0.15ex} \times \hspace{-0.3ex} \bigl( \boldmathbb{A} + \hspace{-0.2ex} \boldnabla a \hspace{.1ex} \bigr)
\hspace{.25em} \Leftrightarrow \hspace{.27em}
\boldnabla \dotp \hspace{.12ex} \boldmathbb{B} = \hspace{.1ex} 0
\;\;\eqref{maxwell.equations:gauss.magnetic}
\]

scalar potential $\upphi$

\nopagebreak\begin{multline}
\boldnabla \hspace{-0.15ex} \times \hspace{-0.1ex} \boldmathbb{E} = - \hspace{.2ex} \mathdotabove{\boldmathbb{B}}
\;\;\eqref{maxwell.equations:faraday.induction}
\hspace{.4em} \Rightarrow \hspace{.3em}
\boldnabla \hspace{-0.15ex} \times \hspace{-0.1ex} \boldmathbb{E} = - \hspace{.2ex} \boldnabla \hspace{-0.15ex} \times \hspace{-0.3ex} \bigl( \mathdotabove{\boldmathbb{A}} + \hspace{-0.2ex} \boldnabla \mathdotabove{a} \hspace{.15ex} \bigr)
\hspace{.4em} \Rightarrow
\\[-0.1em]
\Rightarrow \hspace{.3em}
\boldnabla \hspace{-0.15ex} \times \hspace{-0.3ex} \bigl( \hspace{.1ex} \boldmathbb{E} + \mathdotabove{\boldmathbb{A}} + \hspace{-0.2ex} \boldnabla \mathdotabove{a} \hspace{.15ex} \bigr) \hspace{-0.4ex} = \bm{0}
\hspace{.4em} \Rightarrow \hspace{.25em}
- \boldnabla \upphi = \hspace{.1ex} \boldmathbb{E} + \mathdotabove{\boldmathbb{A}} + \hspace{-0.2ex} \boldnabla \mathdotabove{a}
\hspace{.4em} \Rightarrow
\\[-0.1em]
\Rightarrow \hspace{.33em}
\boldmathbb{E} = - \boldnabla \hspace{.1ex} \bigl( \upphi \hspace{-0.1ex} + \mathdotabove{a} \hspace{.1ex} \bigr) \hspace{-0.3ex} - \mathdotabove{\boldmathbb{A}}
\hspace{.2ex} .
\end{multline}

And

\nopagebreak\vspace{-0.1em}\begin{equation}\label{electromagnetic.firstequationofwave}
\boldnabla \dotp \hspace{.12ex} \boldmathbb{E} = \displaystyle \frac{\chargedensity}{\raisemath{.16em}{\vacuumpermittivity}}
\;\;\eqref{maxwell.equations:gauss.electric}
\hspace{.44em} \Rightarrow
%
\begin{array}{r@{\hspace{.3em}}c@{\hspace{.3em}}l}
- \hspace{-0.1ex} \Laplacian \bigl( \upphi \hspace{-0.1ex} + \mathdotabove{a} \hspace{.1ex} \bigr) \hspace{-0.2ex} - \hspace{-0.15ex} \boldnabla \hspace{-0.1ex} \dotp \mathdotabove{\boldmathbb{A}} & = & \scalebox{0.9}{$\displaystyle \frac{\chargedensity}{\raisemath{.16em}{\vacuumpermittivity}}$}
\\[.66em]
- \hspace{-0.1ex} \Laplacian \upphi - \hspace{-0.15ex} \boldnabla \hspace{-0.12ex} \dotp \hspace{-0.12ex} \bigl( \mathdotabove{\boldmathbb{A}} + \hspace{-0.2ex} \boldnabla \mathdotabove{a} \hspace{.15ex} \bigr) \hspace*{-0.33ex} & = & \scalebox{0.9}{$\displaystyle \frac{\chargedensity}{\raisemath{.16em}{\vacuumpermittivity}}$}
\end{array}
\end{equation}

${\boldnabla \hspace{-0.2ex} \times \hspace{-0.3ex} \bigl( \boldnabla \hspace{-0.2ex} \times \hspace{-0.2ex} \boldmathbb{A} \bigr) \hspace{-0.25ex} = \hspace{-0.1ex} \boldnabla \boldnabla \hspace{-0.1ex} \dotp \boldmathbb{A} - \Laplacian \hspace{.05ex} \boldmathbb{A}}$

${\boldnabla \boldnabla \hspace{-0.2ex} \dotp \hspace{-0.25ex} \boldnabla a - \hspace{-0.2ex} \boldnabla \hspace{-0.2ex} \dotp \hspace{-0.25ex} \boldnabla \boldnabla a = \bm{0}}$ (partial derivatives of a~smooth function commute)

\nopagebreak\vspace{-0.4em}\begin{multline}\label{electromagnetic.secondequationofwave}
c^{\hspace{.16ex}2} \hspace{.22ex} \boldnabla \hspace{-0.15ex} \times \hspace{-0.1ex} \boldmathbb{B}
= \displaystyle \frac{\currentdensity}{\raisemath{.16em}{\vacuumpermittivity}} + \hspace{.1ex} \mathdotabove{\boldmathbb{E}}
\;\;\eqref{maxwell.equations:ampere.circuital.with.addition}
\hspace{.44em} \Rightarrow
%
\\[-0.4em]
%
\Rightarrow \hspace{.33em}
c^{\hspace{.16ex}2} \hspace{.22ex} \boldnabla \hspace{-0.2ex} \times \hspace{-0.3ex} \bigl( \boldnabla \hspace{-0.2ex} \times \hspace{-0.2ex} \boldmathbb{A} \bigr) \hspace{-0.15ex}
= \hspace{.1ex} \displaystyle \frac{\currentdensity}{\raisemath{.16em}{\vacuumpermittivity}} - \hspace{-0.15ex} \boldnabla \hspace{.1ex} \bigl( \mathdotabove{\upphi} \hspace{-0.1ex} + \mathdotdotabove{a} \hspace{.2ex} \bigr) \hspace{-0.3ex} - \mathdotdotabove{\boldmathbb{A}}
\hspace{.44em} \Rightarrow
%
\\[-0.4em]
%
\Rightarrow \hspace{.33em}
c^{\hspace{.16ex}2} \bigl( \boldnabla \boldnabla \hspace{-0.1ex} \dotp \boldmathbb{A} - \Laplacian \hspace{.05ex} \boldmathbb{A} \bigr) \hspace{-0.15ex}
= \hspace{.1ex} \displaystyle \frac{\currentdensity}{\raisemath{.16em}{\vacuumpermittivity}} - \hspace{-0.15ex} \boldnabla \mathdotabove{\upphi} - \hspace{-0.15ex} \boldnabla \hspace{.1ex} \mathdotdotabove{a} \hspace{.1ex} - \mathdotdotabove{\boldmathbb{A}}
\hspace{.2ex} .
\end{multline}

\en{With a~gauge freedom}\ru{Со~свободой калибровки}
\en{it’s possible}\ru{возможно}
\en{to~simplify}\ru{упростить}
\en{the wave equations}\ru{волновые уравнения}~\eqref{electromagnetic.secondequationofwave}
\en{and}\ru{и}~\eqref{electromagnetic.firstequationofwave},
\en{assuming that}\ru{полож\'{и}в}

\nopagebreak\vspace{-0.25em}\begin{gather*}
\left.
\begin{array}{r@{\hspace{.55em}}c@{\hspace{.55em}}l}
- \boldnabla \hspace{.1ex} \mathdotdotabove{a} \hspace{.1ex} = \hspace{-0.2ex} \boldnabla \mathdotabove{\upphi} + c^{\hspace{.16ex}2} \hspace{.22ex} \boldnabla \boldnabla \hspace{-0.1ex} \dotp \boldmathbb{A}
& \Rightarrow &
%
\mathdotdotabove{a} \hspace{.1ex} = - \hspace{.4ex} \mathdotabove{\upphi} - c^{\hspace{.16ex}2} \hspace{.22ex} \boldnabla \hspace{-0.1ex} \dotp \boldmathbb{A}
%
\\[.2em]
%
- \boldnabla \hspace{-0.12ex} \dotp \hspace{-0.2ex} \bigl( \mathdotabove{\boldmathbb{A}} + \hspace{-0.2ex} \boldnabla \mathdotabove{a} \hspace{.15ex} \bigr) \hspace{-0.2ex}
= \raisemath{.07em}{\scalebox{0.84}{$\displaystyle \frac{\raisemath{-0.16em}{1}}{\raisemath{-0.1em}c^{\hspace{.16ex}2}}$}} \hspace{.16ex} \mathdotdotabove{\upphi}
& \Rightarrow &
%
\mathdotabove{\upphi} = \hspace{-0.1ex} - \hspace{.3ex} c^{\hspace{.16ex}2} \hspace{.33ex} \boldnabla \hspace{-0.12ex} \dotp \hspace{-0.2ex} \bigl( \boldmathbb{A} + \hspace{-0.2ex} \boldnabla a \hspace{.1ex} \bigr)
\end{array}
\hspace{-0.2em} \right\rbrace \hspace{.1em} \Rightarrow
%
\\
%
\Rightarrow \hspace{.4em}
c^{\hspace{.16ex}2} \hspace{.22ex} \boldnabla \hspace{-0.12ex} \dotp \hspace{-0.12ex} \boldmathbb{A} + c^{\hspace{.16ex}2} \hspace{-0.4ex} \Laplacian \hspace{.07ex} a
- c^{\hspace{.16ex}2} \hspace{.22ex} \boldnabla \hspace{-0.1ex} \dotp \boldmathbb{A} = \hspace{.15ex} \mathdotdotabove{a}
\hspace{.16ex} ,
\end{gather*}

\vspace{-0.16em}\noindent
\en{finally}\ru{наконец}
\en{presenting as}\ru{становясь}
\en{the }\en{homogeneous}\ru{однородным}
\href{https://en.wikipedia.org/wiki/Wave_equation}{\en{wave equation}\ru{волновым уравнением}}
\en{for}\ru{для}~$a$

\nopagebreak\vspace{-0.12em}\begin{equation}\label{gaugeadditioniswave}
\mathdotdotabove{a} \hspace{.1ex} = c^{\hspace{.16ex}2} \hspace{-0.4ex} \Laplacian \hspace{.07ex} a
\hspace{.16ex} .
\end{equation}

\vspace{-0.15em}\noindent
\en{The more popular condition}\ru{Более популярное условие}\en{ is}\ru{\:---}
\en{even more stiff}\ru{ещё более жёсткое}

\nopagebreak\vspace{-0.1em}\begin{equation*}
\Laplacian \hspace{.07ex} a = 0
\hspace{.3em} \Rightarrow \hspace{.3em}
\mathdotdotabove{a} = 0
\hspace{.1ex} , \:\:
\mathdotabove{\upphi} + c^{\hspace{.16ex}2} \hspace{.22ex} \boldnabla \hspace{-0.1ex} \dotp \boldmathbb{A} = 0
\hspace{.1ex} .
\end{equation*}

\vspace{-0.2em}\noindent
\en{This}\ru{Это}
\href{https://en.wikipedia.org/wiki/Lorenz_gauge_condition}{\ru{условие калибровки }Lorenz\ru{’а}\en{ gauge condition}}
\en{gives}\ru{даёт}
\en{the~same}\ru{такой~же}
\en{effect}\ru{эффект},
\en{being just the~particular}\ru{будучи лишь частным}\:---
\en{the }\href{https://en.wikipedia.org/wiki/Harmonic_function}{\en{harmonic}\ru{гармоническим}}\:--- \en{case}\ru{случаем}~\en{of~}\eqref{gaugeadditioniswave}.

\en{Following from}\ru{Следующие из}~\eqref{electromagnetic.firstequationofwave} \en{and}\ru{и}~\eqref{electromagnetic.secondequationofwave} \en{with condition}\ru{с~условием}~\eqref{gaugeadditioniswave}, \en{equations}\ru{уравнения} \en{of electromagnetic waves}\ru{электромагнитных волн} \en{in the~potential formulation}\ru{в~потенциальной формулировке}\en{ are}

\nopagebreak\begin{equation}
\begin{array}{c}
- \hspace{-0.1ex} \Laplacian \upphi + \raisemath{.07em}{\scalebox{0.84}{$\displaystyle \frac{\raisemath{-0.16em}{1}}{\raisemath{-0.1em}c^{\hspace{.16ex}2}}$}} \hspace{.16ex} \mathdotdotabove{\upphi} = \smash{\displaystyle \frac{\chargedensity}{\raisemath{.16em}{\vacuumpermittivity}}}
\hspace{.1ex} ,
\\[.5em]
- \hspace{.3ex} c^{\hspace{.16ex}2} \hspace{-0.4ex} \Laplacian \hspace{.05ex} \boldmathbb{A} = \displaystyle \frac{\currentdensity}{\raisemath{.16em}{\vacuumpermittivity}} - \mathdotdotabove{\boldmathbb{A}}
\hspace{.2ex} .
\end{array}
\end{equation}

...

\en{\section{Electrostatics}}

\ru{\section{Электростатика}}

\begin{otherlanguage}{russian}

Рассмотрение этого вопроса полезно и для последующего опис\'{а}ния магнетизма.
В~статике

\nopagebreak\vspace{-0.2em}\begin{equation*}
\velocityofpoint = \bm{0}
\hspace{.44em} \Rightarrow \hspace{.33em}
\boldmathbb{B} = \bm{0}
\end{equation*}

...

\en{The~}\en{volume}\ru{Объёмная} \inquotes{\en{ponderomotive}\ru{пондеромоторная}} \en{force}\ru{сила}, с~которой электростатическое поле действует на~среду ...

...

\ru{Тензор напряжения }Maxwell\ru{’а}\en{ stress tensor}~\eqref{maxwellstresstensor:definition} \en{in~electrostatics}\ru{в~электростатике}\en{ is}

\nopagebreak\vspace{-0.2em}\begin{equation*}
\maxwellstress = \vacuumpermittivity \hspace{-0.2ex} \left( \boldmathbb{E} \boldmathbb{E} - \smalldisplaystyleonehalf \hspace{.2ex} \boldmathbb{E} \hspace{-0.1ex} \dotp \hspace{-0.16ex} \boldmathbb{E} \hspace{.1ex} \UnitDyad \hspace{.1ex} \right)
\end{equation*}

...

\end{otherlanguage}

\en{\section{Dielectrics}}

\ru{\section{Диэлектрики}}

\begin{otherlanguage}{russian}

Начнём с~рассмотрения электростатического поля

...

В~диэлектриках нет свободных зарядов: \en{charge density}\ru{плотность заряда}~${\chargedensity = 0}$.
Здесь вводится плотность дипольного момента

...


\end{otherlanguage}

\en{\section{Magnetostatics}}

\ru{\section{Магнитостатика}}

\begin{otherlanguage}{russian}

Если поле~(а~с~ним ...)

...

\end{otherlanguage}

\en{\section{Magnetics}}

\ru{\section{Магнетики}} % Магнитные материалы

\begin{otherlanguage}{russian}

Выяснив законы магнитостатики в~общем случае, обратимся к~веществу\:--- некий опыт у~нас уже есть в~электростатике ди\-элект\-ри\-ков.

Начнём с~рассмотрения

...


...

Насколько соответствует поведение реальных материалов представленным здесь формальным построениям\:--- сей вопрос is out of~scope этой книги.

\end{otherlanguage}

\en{\section{Magnetic rigidity}}

\ru{\section{Магнитная жёсткость}}

\begin{otherlanguage}{russian}

В~электротехнике распространены обмотки всевозможной формы, в~которых провод намотан так, что образуется некое массивное тело. Такие обмотки есть в~статоре генератора автомобиля~(да~и в~роторе), в~больших промышленных электромагнитах и~в~магнитных системах установок \href{https://ru.wikipedia.org/wiki/%D0%A2%D0%BE%D0%BA%D0%B0%D0%BC%D0%B0%D0%BA}{\inquotes{токам\'{а}к}~(\textboldextended{то}роидальная \textboldextended{ка}мера с~\textboldextended{ма}гнитными \textboldextended{к}атушками)} для управляемого термоядерного синтеза\:--- примеров много. Сочетание токопровода и~изоляции образует периодический композит, и~одной из~главных нагрузок для него является пондеромоторная магнитная сила. Рассчитывая деформации и~механические напряжения в~обмотке, начинают с~определения магнитных сил. Поскольку распределение токов задано известной геометрией проводов, достаточно интегрирования по~формуле~Био\hbox{-}Савара~\eqref{law:biosavar}. Термин \inquotes{магнитоупругость} при~этом неуместен, так~как задачи магнитостатики и~упругости решаются раздельно.

Однако при~деформации обмотки меняются и~поле~$\currentdensity$, и~вызываемое им поле~$\boldmathbb{B}$. Объёмная сила становится равной
\begin{equation}
\bm{f} = \left( \hspace{.16ex} \currentdensity \times \boldmathbb{B} \right)_0 + \dots
\end{equation}
\noindent \textcolor{magenta}{Подчёркнутое слагаемое} соответствует недеформированному состоянию. Обусловленное деформацией изменение объёмной силы линейно связано с~малым смещением~$\bm{u}$, поэтому матричное~(после дискретизации) уравнение в~смещениях можно представить в~виде

\noindent\begin{equation}
\left( C + C_m \right) \hspace{-0.16ex} u = \mathcircabove{F}
\hspace{.1ex} .
\end{equation}

\noindent
К~\inquotes{обычному} оператору линейной упругости~$C$ добавилась магнитная жёсткость~${C_m}$,
$\mathcircabove{F}$\:--- силы в~недоформированном состоянии.

Добавка~$C_m$ пропорциональна квадрату тока и~может стать весьма существенной в~магнитных системах с~сильным полем.
Учёт её необходим и~при~недостаточной величине~$C$.
В~номинальном \textcolor{red}{режиме}
конструкция может держать нагрузку,
но дополнительная нагрузка
неблагоприятного направления
может оказаться \inquotesx{невыносимой}[.]

\en{The role}\ru{Роль}
магнитной жёсткости
\en{is very important}\ru{очень важна}
в~задачах устойчивости.
Матрица~$C_m$ симметрична,
так как магнитные силы
\en{are potential}\ru{потенциальны}.
\en{The critical parameters}\ru{Критические параметры}
\en{can be found}\ru{могут быть найдены}
\en{using}\ru{используя}
статический метод Euler’а.

Как иллюстрацию рассмотрим простую задачу о~балке в~продольном магнитном поле.
Балка располагается на~декартовой оси~$z$, концы~${z \narroweq 0}$ и~${z \narroweq l}$ закреплены, магнитная индукция ${\boldmathbb{B} = B\bm{k} = \boldconstant}$, по~балке течёт постоянный~(по~величине) ток~$I$.
В~классической модели балки при~равных жёсткостях на~изгиб для~прогиба ${\bm{u} = u_1 \bm{e}_1 \hspace{.16ex} + u_2 \bm{e}_2}$ легко получить следующую постановку:

...

Вводя компл\'{е}ксную комбинацию
${u \equiv u_1 \hspace{-0.16ex} + \mathrm{i} u_2}$,
будем иметь

...

\noindent
с общим решением

...

\noindent
Подстановка в~граничные условия приводит к~однородной системе для постоянных ${A_k}$.
Приравняв нулю определитель, придём к~характеристическому уравнению

...

\noindent
Наименьший положительный корень~${x = 3.666}$,
так~что критическая комбинация параметров такова:

\noindent\begin{equation*}
\left(
I \hspace{-0.25ex} B \hspace{.1ex} {l^3} \hspace{-0.2ex} / a \hspace{.1ex}
\right)_{\hspace{-0.32ex}*}
\hspace{-0.32ex} = \hspace{.1ex}
394.2
\hspace{.1ex} .
\end{equation*}

Поле~$\boldmathbb{B}$
в~этом решении
считалось внешним
и~не~варьировалось.
Но
если
собственное поле тока
в~стержне
сравнимо
с~$\boldmathbb{B}$,
то решение
изм\'{е}нится
\en{and}\ru{и}~усложн\'{и}тся.

\end{otherlanguage}

\section*{\small \wordforbibliography}

\begin{changemargin}{\parindent}{0pt}
\fontsize{10}{12}\selectfont

\begin{otherlanguage}{russian}

Основы электродинамики хорошо изложены во~многих книгах~\cite{classicalelectrodynamics, feynman-lecturesonphysics}, но для~приложений в~механике выделяется книга И.\,Е.\;Тамма~\cite{tamm-electricity}.
Растёт список литературы по~связанным задачам электромагнетизма и~упругости~\cite{parton-electromagneticelasticity, podstrigach.burak.kondrat-magnetothermoelasticity}.
Как введение в~эту область может быть полезна монография В.\;Новацкого~\cite{nowacki-electromagneticeffects}.

\end{otherlanguage}

\end{changemargin}

