\en{\section{Mixed principles of stationarity}}

\ru{\section{Смешанные принципы стационарности}}

\label{section:mixedvariationalprinciples}

\vspace{.2em}\begin{changemargin}{2\parindent}{\parindent}
\bgroup % to change \parindent locally
\setlength{\parindent}{\negparindent}
\small

\hspace{\parindent}\href{https://en.wikiversity.org/wiki/Introduction_to_Elasticity/Hellinger-Reissner_principle}{\textbold{Prange\hbox{--}Hellinger\hbox{--}Reissner Variational Principle}},\\
named after \emph{Ernst Hellinger}, \emph{Georg Prange} and \emph{Eric Reissner}.
\par

\nopagebreak\vspace{.16em}
{\noindent
\scriptsize
Working independently of Hellinger and Prange, Eric Reissner published his famous six\hbox{-}page paper \inquotes{On a~variational theorem in~elasticity} in~1950. In this paper he develops\:--- without, however, considering Hamilton\hbox{--}Jacobi theory\:--- a~variational principle same to that of~Prange and~Hellinger.
\par}

\nopagebreak\vspace{.32em}
\href{https://en.wikiversity.org/wiki/Introduction_to_Elasticity/Hu-Washizu_principle}{\textbold{Hu\hbox{--}Washizu Variational Principle}},\\
\en{named as}\ru{именуемых}
\emph{Hu Haichang}
\en{and}\ru{и}
\emph{Kyuichiro\;Washizu}.
\par
\egroup
\nopagebreak\vspace{.1em}
\end{changemargin}

\noindent
\en{The~following}\ru{Следующий}
\en{functional}\ru{функционал}
\en{over}\ru{над}
\en{the displacements}\ru{смещениями}
\en{and}\ru{и}~\en{stresses}\ru{напряжениями}

\nopagebreak\vspace{-0.3em}
\begin{multline}\label{reissnerhellingervariationalprinciple}
\shoveleft{\mathcal{R}( \bm{u}, \hspace{-0.32ex} \linearstress ) \hspace{.2ex} =
\displaystyle\integral\displaylimits_{\mathcal{V}} \hspace{-0.3ex}
\Bigl[
\linearstress \dotdotp \hspace{-0.16ex} \boldnabla {\bm{u}}^{\mathsf{S}} \hspace{-0.2ex} - \hspace{.1ex} \complementaryenergydensity(\hspace{-0.1ex}\linearstress\hspace{.1ex})\hspace{-0.1ex} -
\volumeloadvector \hspace{-0.1ex} \dotp \bm{u}
\hspace{.2ex} \Bigr] \hspace{-0.1ex} d\mathcal{V} \hspace{.5ex}
- \hfill}
\\[-1.2em]
%
\hspace{8em}
- \hspace{-0.2ex} \displaystyle\integral\displaylimits_{o_1} \hspace{-0.4ex} \unitnormalvector \dotp \hspace{-0.1ex} \linearstress \dotp \hspace{-0.1ex} \bigl( \bm{u} - \bm{u}_{\raisemath{-0.1em}{0}} \bigr) do \hspace{.2ex}
- \hspace{-0.2ex} \integral\displaylimits_{o_2} \hspace{-0.4ex} \bm{p} \dotp \bm{u} \hspace{.25ex} do
\end{multline}

\nopagebreak\vspace{-0.2em}\noindent
\en{carries}\ru{н\'{о}сит} \en{names}\ru{имена} \en{of~}Reissner\ru{’а}, Prange\ru{’а} \en{and}\ru{и}~Hellinger\ru{’а}.

...

\en{The advantage}\ru{Преимущество} \ru{принципа }\en{of the }Reissner\ru{’а}\hbox{--}Hellinger\ru{’а}\en{ principle}\:--- \en{freedom of variation}\ru{свобода варьирования}.
\en{But it also has a~drawback}\ru{Но у него есть и~недостаток}: \en{on the true solution}\ru{на~истинном решении} \en{the functional has no extremum}\ru{у~функционала нет экстремума}, \en{but only stationarity}\ru{а~лишь стационарность}.


\begin{otherlanguage}{russian}

Принцип можно использовать для~построения приближённых решений методом Ritz\ru{’а}~(Ritz method).
Задавая аппроксимации

...

Принцип Hu\hbox{--}Washizu\ru{~(Ху\hbox{--}Васидзу)}~\cite{washizubook} формулируется так:

\nopagebreak\vspace{-0.2em}\begin{multline}\label{huwashizuvariationalprinciple}
\hfil \variation{\mathcal{W}} \hspace{.1ex} (\bm{u}, \hspace{-0.32ex}\infinitesimaldeformation, \hspace{-0.32ex}\linearstress) = 0
\hspace{.1ex} ,
\\[-0.1em]
%
\shoveleft{\displaystyle \mathcal{W} \hspace{.2ex} \equiv
\displaystyle\integral\displaylimits_{\mathcal{V}} \hspace{-0.3ex}
\Bigl[
\linearstress \dotdotp \hspace{-0.2ex} \Bigl( \hspace{-0.2ex} \boldnabla {\bm{u}}^{\mathsf{S}} \hspace{-0.2ex} - \infinitesimaldeformation \hspace{-0.1ex} \Bigr) \hspace{-0.2ex} + \hspace{.1ex} \potentialenergydensity (\hspace{-0.1ex}\infinitesimaldeformation\hspace{-0.1ex})\hspace{-0.1ex} -
\volumeloadvector \hspace{-0.1ex} \dotp \bm{u}
\hspace{.2ex} \Bigr] \hspace{-0.1ex} d\mathcal{V} \hspace{.5ex}
- \hfill}
\\[-1.2em]
%
\hspace{8em}
- \hspace{-0.2ex} \displaystyle\integral\displaylimits_{o_1} \hspace{-0.4ex} \unitnormalvector \dotp \hspace{-0.1ex} \linearstress \dotp \hspace{-0.1ex} \bigl( \bm{u} - \bm{u}_{\raisemath{-0.1em}{0}} \bigr) do \hspace{.2ex}
- \hspace{-0.2ex} \integral\displaylimits_{o_2} \hspace{-0.4ex} \bm{p} \dotp \bm{u} \hspace{.25ex} do
\hspace{.2ex} .
\end{multline}

Как и~в~принципе Reissner’а--Hellinger’а,
здесь нет ограничений ни~в~объёме, ни~на~поверхности,
но добавляется третий независимый аргумент~$\infinitesimaldeformation$.
Поскольку
${\complementaryenergydensity = \linearstress \dotdotp \infinitesimaldeformation - \potentialenergydensity}$,
то~\eqref{reissnerhellingervariationalprinciple}
\en{and}\ru{и}~\eqref{huwashizuvariationalprinciple}
кажутся почти одним и~тем~же.

\en{From}\ru{Из}
\ru{принципа }\en{the }Hu\hbox{--}Washizu\ru{~(Ху\hbox{--}Васидзу)}\en{ principle}
\en{ensues}\ru{вытекает}
\en{the whole complete set of~equations}\ru{весь полный набор уравнений}
\en{with boundary conditions}\ru{с~краевыми условиями},
так~как

.....

\end{otherlanguage}
