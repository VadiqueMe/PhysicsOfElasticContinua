\section*{\small \wordforbibliography}

\begin{changemargin}{\parindent}{0pt}
\fontsize{10}{12}\selectfont

\en{There are many books}\ru{Есть много книг}\ru{,}
\en{which describe}\ru{которые описывают}
\en{only}\ru{только}
\en{the apparatus}\ru{аппарат}
\en{of the tensor calculus}\ru{тензорного исчисления}
\cite{mcconnell-tensoranalysis, schouten-tensoranalysis, sokolnikoff-tensoranalysis, dimitrienko-tensorcalculus, borisenko.tarapov, rashevsky-riemanniangeometry}.

\en{However}\ru{Однако},
\en{the index notation}\ru{индексная запись}
( \en{it’s}\ru{это}
\en{when}\ru{когда}
\en{the tensors are presented}\ru{тензоры представлены}
\en{as the sets of~components}\ru{наборами компонент} )
\en{is still more popular}\ru{всё ещё более популярна}\ru{,}
\en{than the direct indexless notation}\ru{чем прямая безиндексная запись}.

\en{The direct notation}\ru{Прямая запись}
\en{is widely used}\ru{широко используется},
\en{for example}\ru{например},
\en{in the appendices}\ru{в~приложениях}
\en{to the books}\ru{к~книгам}
\en{by }Anatoliy’я I. Lurie (}\foreignlanguage{russian}{Анатолия И. Лурье}\en{)}
\cite{lurie-nonlinearelasticity, lurie-theoryofelasticity}.

\en{The}\ru{Лекции} R.\:Feynman’\en{s}\ru{а}\en{ lectures}~\cite{feynman-lecturesonphysics}
\en{contain}\ru{содержат}
\en{the vivid description}\ru{яркое описание}
\en{of the vector fields}\ru{векторных полей}.

\en{Also}\ru{Также},
\en{the information}\ru{информация}
\en{about the tensor calculus}\ru{о~тензорном исчислении}\en{ is}\ru{\:---}
\en{the part of}\ru{часть}
\en{the unusual}\ru{необычной}
\en{book}\ru{книги}
\en{by }C.\:Truesdell\ru{’а}~\cite{truesdell-firstcourse}.

\end{changemargin}
