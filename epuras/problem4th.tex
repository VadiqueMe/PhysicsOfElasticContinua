\documentclass[tikz, margin=10mm]{standalone}

\usepackage{tikz}

\usetikzlibrary{calc}
\usetikzlibrary{arrows, arrows.meta}

\makeatletter
\def\mathcolor#1#{\@mathcolor{#1}}
\def\@mathcolor#1#2#3{%
	\protect\leavevmode
	\begingroup\color#1{#2}#3\endgroup
}
\makeatother

\usepackage{xargs}
\usepackage{xifthen}

\usepackage{scalerel}
\usepackage{rotating}
\usepackage{amsopn}

\DeclareMathOperator*{\integral}{\scalerel*{\rotatebox{12}{$\!\textstyle\int\!$}}{\int}} % \rotatebox{8} for vertical line

\usepackage{accents}

\newcommand{\scircabove}{\hbox{\fontfamily{lmr}\fontsize{6}{0}\selectfont$\,\circ$}}
\DeclareRobustCommand{\mathcircabove}{\accentset{\scircabove}}

\newcommand\verynicefrac[2]{\raisebox{.11ex}{$ \raisebox{.3em}{\scalebox{.7}{$#1$}} \hspace{-0.4ex} / \hspace{-0.46ex} \raisebox{-0.25em}{\scalebox{.7}{$#2$}}\hspace{.15ex} $}}

\begin{document}

\begin{tikzpicture}[scale=1]

\def\textscale{2}
\pgfmathsetmacro\smallertextscale{.9*\textscale}
\pgfmathsetmacro\evensmallertextscale{.75*\textscale}

\def\beamlength{16}
\pgfmathsetmacro\halfbeamlength{\beamlength / 2}
\pgfmathsetmacro\quarterbeamlength{\beamlength / 4}

%
% the problem
%

\def\problempos{3.33}
\newlength\problemwidth
\setlength{\problemwidth}{12cm}

\node [above, inner sep=0pt, outer sep=0pt] at (\halfbeamlength, \problempos)
	{\scalebox{\evensmallertextscale}{ \parbox{\problemwidth}{%
An absolutely rigid bracket with a~spring on one side rests on a~hinged beam.
At what distance~$a$ should a~point force~$F$ be applied so that the~bracket does not tilt?
The~length of the~beam~$\ell$,
the~bending rigidity ${EJ}$ (constant along the~beam length)
and the~spring stiffness coefficient ${k \hspace{-0.3ex} = \hspace{-0.4ex} \frac{EJ}{\ell^3}}$
are known. }}
	} ;

%
% the beam
%

\def\beamlinewidth{3pt}
\tikzstyle{beam line} = [line width=\beamlinewidth, line cap=round, color=black]

\tikzstyle{beam length} = [line width=.5pt, color=black, line cap=round]

\newcommand{\halfell}{ \raisebox{.3em}{$\ell$} \hspace{-0.4ex} / \hspace{-0.36ex} \raisebox{-0.33em}{\scalebox{.88}{$2$}} }

\def\lengthlineoffset{0.96}
\pgfmathsetmacro\belowlengthlinelength{-\lengthlineoffset - 0.2}

\draw [beam length] (0, 0) -- (0, \belowlengthlinelength) ;

\draw [beam line] (0, 0) -- (\quarterbeamlength, 0) ;
\draw [beam length] (0, -\lengthlineoffset) -- (\quarterbeamlength, -\lengthlineoffset)
	node [pos=0.51, shape=circle, fill=white, inner sep=0pt, outer sep=11pt] {\scalebox{\textscale}{$ \ell $}} ;

\draw [beam length] (\quarterbeamlength, 0) -- (\quarterbeamlength, \belowlengthlinelength) ;

%%\draw [beam length] (\quarterbeamlength, 0) -- (\quarterbeamlength, \abovelengthlinelength) ;

\draw [beam line] (\quarterbeamlength, 0) -- (\halfbeamlength, 0) ;
\draw [beam length] (\quarterbeamlength, -\lengthlineoffset) -- (\halfbeamlength, -\lengthlineoffset)
	node [pos=0.49, shape=circle, fill=white, inner sep=0pt, outer sep=11pt] {\scalebox{\textscale}{$ \ell $}} ;

%%\draw [beam length] (\halfbeamlength, 0) -- (\halfbeamlength, \abovelengthlinelength) ;

\draw [beam length] (\halfbeamlength, 0) -- (\halfbeamlength, \belowlengthlinelength) ;

\draw [beam line] (\halfbeamlength, 0) -- (\beamlength, 0) ;
\draw [beam length] (\halfbeamlength, -\lengthlineoffset) -- (\beamlength, -\lengthlineoffset)
	node [pos=0.5, shape=circle, fill=white, inner sep=1pt, outer sep=11pt] {\scalebox{\textscale}{$ 2 \ell \hspace{-0.1ex} $}} ;

\draw [beam length] (\beamlength, 0) -- (\beamlength, \belowlengthlinelength) ;

\newcommand\redrawbeam{\draw [beam line] (0, 0) -- (\beamlength, 0) ;}

%
% the bracket
%

\def\bracketlinewidth{2pt}
\tikzstyle{that bracket} = [line width=\bracketlinewidth, line cap=round, color=blue, fill=white]

\def\bracketverticaloffset{.5}
\def\bracketheight{.4}
\pgfmathsetmacro\bracketbeyondcontact{.1 * \quarterbeamlength}

\pgfmathsetmacro\brackethardtrianglesize{.066 * \quarterbeamlength}
\draw [that bracket]
	( \halfbeamlength, .5*\beamlinewidth )
	-- ( \halfbeamlength + \brackethardtrianglesize, \bracketverticaloffset )
	-- ( \halfbeamlength - \brackethardtrianglesize, \bracketverticaloffset )
	-- cycle ;

\def\springcenterat{\quarterbeamlength}
\def\springhalfwidth{.25}
\pgfmathsetmacro\springwidth{2 * \springhalfwidth}
\def\springfirst{\bracketverticaloffset}
\def\springcoils{4}
\def\springbottom{0}
\pgfmathsetmacro\springstep{(\springfirst - \springbottom) / \springcoils}
\pgfmathsetmacro\springnext{\springfirst - \springstep}
\pgfmathsetmacro\springlast{\springbottom + \springstep}

\tikzstyle{the spring} = [line width=.66pt, color=magenta]

\foreach \ycurrent in {\springfirst, \springnext, ..., \springlast}
	\draw [the spring]
		( \springcenterat - \springhalfwidth, \ycurrent )
		-- ++( \springwidth, 0 ) -- ++( -\springwidth, -\springstep ) ;

\draw [that bracket, rounded corners=2pt]
	( \quarterbeamlength - \bracketbeyondcontact, \bracketverticaloffset )
	rectangle ( \halfbeamlength + \bracketbeyondcontact, \bracketverticaloffset + \bracketheight ) ;

%
% external loads
%

\pgfmathsetmacro\justa{.7 * \quarterbeamlength}
\pgfmathsetmacro\alength{\quarterbeamlength + \justa}

\pgfmathsetmacro\abovelengthlineoffset{\bracketheight + \lengthlineoffset - 0.1}
\pgfmathsetmacro\abovelengthlinelength{\abovelengthlineoffset + 0.23}

\draw [beam length] (\quarterbeamlength, \bracketverticaloffset) -- ++(0, \abovelengthlinelength) ;
%%\draw [beam length] (\alength, \bracketverticaloffset) -- ++(0, \abovelengthlinelength) ;
\draw [beam length]
	(\quarterbeamlength, \bracketverticaloffset + \abovelengthlineoffset) -- ++(\justa, 0)
	node [pos=0.5, shape=circle, fill=white, inner sep=2pt, outer sep=11pt] {\scalebox{\textscale}{$ a $}} ;

\def\pointloadheight{1.4}
\tikzstyle{concentrated load line} =
	[line width=2pt, red, line cap=round, -{Triangle[round, length=15pt, width=9pt]}]

\pgfmathsetmacro\loadlinestart{\pointloadheight + \bracketverticaloffset + \bracketheight}
\draw [concentrated load line]
	([yshift=.5*\bracketlinewidth] \alength, \loadlinestart ) -- ++( 0, -\pointloadheight )
	node [pos=0.4, above right, shape=circle, fill=white, inner sep=-2pt, outer sep=10pt] {\scalebox{2}{$ F $}} ;

%
% constraints
%

\input{constraints.include}

% uncomment to see more constraints
%%\addpinnedmidhingeat{(\quarterbeamlength, 0)}
%%\addmovablehingeat{(.8*\beamlength, 0)}[mid]

% redraw beam after loads and middle hinges before clamps
\redrawbeam

\addmovablehingeat{(0, 0)} %%\addpinnedendhingeat{(0, 0)}
\addpinnedendhingeat{(\beamlength, 0)}[flip]

%
% the beam again with reactions of constraints
%

\def\beamwithreactionspos{-3.5}

% extend the beam ends downwards

\tikzstyle{epure aux} = [line width=1pt, color=black, line cap=round, dash pattern=on 0pt off 1.6\pgflinewidth]

\def\auxlineoffset{.2}

\draw [epure aux] (0, \belowlengthlinelength - \auxlineoffset) -- (0, \beamwithreactionspos + \auxlineoffset) ;
\draw [epure aux] (\beamlength, \belowlengthlinelength - \auxlineoffset) -- (\beamlength, \beamwithreactionspos + \auxlineoffset) ;

%%\draw [epure aux] (\alength, \bracketverticaloffset + \bracketheight) -- (\alength, \beamwithreactionspos) ;

%%\draw [concentrated load line]
%%	(\alength, \beamwithreactionspos + \pointloadheight) -- ++(0, -\pointloadheight)
%%	node [pos=0.4, above right, shape=circle, fill=white, inner sep=-2pt, outer sep=10pt] {\scalebox{2}{$ F $}} ;

\def\leftforce{P}
\def\rightforce{R}

\pgfmathsetmacro\leftloadheight{0.63 * \pointloadheight}
\draw [epure aux]
	( \quarterbeamlength, \belowlengthlinelength - \auxlineoffset )
	-- ( \quarterbeamlength, \beamwithreactionspos + \leftloadheight + \auxlineoffset ) ;
\draw [concentrated load line]
	(\quarterbeamlength, \beamwithreactionspos + \leftloadheight) -- ++(0, -\leftloadheight)
	node [pos=0.4, above right, shape=circle, fill=white, inner sep=-2pt, outer sep=10pt] {\scalebox{2}{$ \leftforce $}} ;

\pgfmathsetmacro\rightloadheight{0.77 * \pointloadheight}
\draw [epure aux]
	( \halfbeamlength, \belowlengthlinelength - \auxlineoffset )
	-- ( \halfbeamlength, \beamwithreactionspos + \rightloadheight + \auxlineoffset ) ;
\draw [concentrated load line]
	(\halfbeamlength, \beamwithreactionspos + \rightloadheight) -- ++(0, -\rightloadheight)
	node [pos=0.4, above right, shape=circle, fill=white, inner sep=-2pt, outer sep=10pt] {\scalebox{2}{$ \rightforce $}} ;

% reactions of constraints

\def\leftreaction{\Upsilon}
\def\rightreaction{\Phi}

\def\reactionlength{1.4}

\draw [concentrated load line, color=blue]
	(\beamlength, \beamwithreactionspos - \reactionlength) -- ++(0, \reactionlength)
	node [pos=0.25, right, shape=circle, fill=white, inner sep=0pt, outer sep=1pt]
	{\scalebox{\smallertextscale}{$ \rightreaction $}} ;

\def\zeroloadlength{.7}
\draw [concentrated load line, color=blue]
	(\beamlength + \zeroloadlength, \beamwithreactionspos) -- ++(-\zeroloadlength, 0)
	node [pos=0.2, above, shape=circle, fill=white, inner sep=0pt, outer sep=8pt]
	{\scalebox{\smallertextscale}{$0$}} ;

\draw [concentrated load line, color=blue]
	(0, \beamwithreactionspos - \reactionlength) -- ++(0, \reactionlength)
	node [pos=0.25, left, shape=circle, fill=white, inner sep=0pt, outer sep=2pt]
	{\scalebox{\smallertextscale}{$ \leftreaction $}} ;

% the beam again
\draw [beam line] (0, \beamwithreactionspos) -- ++(\beamlength, 0) ;

%
% the coordinates
%

\def\beamcoordinate{\zeta} %% \xi
\pgfmathsetmacro\coordinatepos{\beamwithreactionspos + 0.5}

\def\secondcoordinate{\eta}
\def\coordinatelinelength{1.2}

\tikzstyle{coordinate line} = [line width=.5pt, color=black, fill=white]

% vertical
\draw [ coordinate line, -{To[round, length=6pt, width=8pt]} ]
	( 0, \coordinatepos ) -- ( 0, \coordinatepos + \coordinatelinelength )
	node [ pos=0.8, left, inner sep=0pt, outer sep=7pt ]
		{\scalebox{\evensmallertextscale}{$ \secondcoordinate $}} ;

% left-to-right
\draw [ coordinate line, -{To[round, length=6pt, width=8pt]} ]
	( 0, \coordinatepos ) -- ( \coordinatelinelength, \coordinatepos )
	node [ pos=0.88, above left, inner sep=0pt, outer sep=4pt ]
		{\scalebox{\evensmallertextscale}{$ \beamcoordinate $}} ;
\draw [ coordinate line ] ( 0, \coordinatepos ) circle ( 2.6pt ) ;

% right-to-left
%%\draw [ coordinate line, -{To[round, length=6pt, width=8pt]} ]
%%	( \beamlength, \coordinatepos ) -- ( \beamlength - 2, \coordinatepos )
%%	node [ pos=0.88, above right, inner sep=0pt, outer sep=4pt ]
%%		{\scalebox{\evensmallertextscale}{$ \beamcoordinate $}} ;
%%\draw [ coordinate line ] ( \beamlength, \coordinatepos ) circle ( 2.6pt ) ;

%
% text
%

\newcommand\ellsquared{ \ell^{\hspace{.1ex}\scalebox{.75}{\raisebox{.1em}{$2$}}} }

\newcommand\internalshearforce{\scalebox{.9}[1]{$ \mathcal{Q} $}}
\newcommand\internalmoment{\scalebox{.9}[1]{$ \mathcal{M} $}}
\newcommand\sectionrotationangle{\vartheta}
\newcommand\beamdeflection{v}

\pgfmathsetmacro\textpos{\beamwithreactionspos - 1.2}

\node [below, inner sep=0pt, outer sep=9pt] at (\halfbeamlength, \textpos)
	{\scalebox{\evensmallertextscale}{$ \begin{array}{c}
		\internalshearforce(\beamcoordinate) \hspace{-0.1ex}
			= \hspace{-0.3ex} \biggl[
				\leftreaction \hspace{-0.1ex} \mathcolor{gray}{\beamcoordinate^{0}} \hspace{-0.1ex}
			\biggr]_{\hspace{-0.2ex} \scalebox{.75}{$ 0 $}}^{\hspace{-0.3ex} \scalebox{.75}{$ 4\ell $}} \hspace{-1.25ex}
			- \hspace{-0.3ex} \biggl[
				\leftforce \mathcolor{gray}{( \beamcoordinate \hspace{-0.5ex} - \hspace{-0.25ex} \ell )^{\hspace{-0.1ex}0}} \hspace{-0.1ex}
			\biggr]_{\hspace{-0.3ex} \scalebox{.75}{$ \ell $}}^{\hspace{-0.3ex} \scalebox{.75}{$ 4\ell $}} \hspace{-1.25ex}
			- \hspace{-0.3ex} \biggl[
				\rightforce \mathcolor{gray}{( \beamcoordinate \hspace{-0.5ex} - \hspace{-0.25ex} 2\ell )^{\hspace{-0.1ex}0}} \hspace{-0.1ex}
			\biggr]_{\hspace{-0.3ex} \scalebox{.75}{$ 2\ell $}}^{\hspace{-0.3ex} \scalebox{.75}{$ 4\ell $}} \hspace{-1.2ex}
		\\[1em]
		\internalshearforce(4 \ell) \hspace{-0.3ex} = \hspace{-0.2ex} - \hspace{.1ex} \rightreaction
		\\[1em]
		\internalmoment(\beamcoordinate) \hspace{-0.2ex}
			= \hspace{-0.5ex}
				\scalebox{1.5}{$\integral$} \hspace{-0.2ex} \internalshearforce \hspace{.2ex} d \beamcoordinate \hspace{-0.1ex}
			= \hspace{-0.3ex} \biggl[
				\leftreaction \hspace{-0.1ex} \beamcoordinate
			\biggr]_{\hspace{-0.2ex} \scalebox{.75}{$ 0 $}}^{\hspace{-0.3ex} \scalebox{.75}{$ 4\ell $}} \hspace{-1.25ex}
			- \hspace{-0.3ex} \biggl[
				\leftforce ( \beamcoordinate \hspace{-0.5ex} - \hspace{-0.25ex} \ell )
			\biggr]_{\hspace{-0.3ex} \scalebox{.75}{$ \ell $}}^{\hspace{-0.3ex} \scalebox{.75}{$ 4\ell $}} \hspace{-1.25ex}
			- \hspace{-0.3ex} \biggl[
				\rightforce ( \beamcoordinate \hspace{-0.5ex} - \hspace{-0.25ex} 2\ell )
			\biggr]_{\hspace{-0.3ex} \scalebox{.75}{$ 2\ell $}}^{\hspace{-0.3ex} \scalebox{.75}{$ 4\ell $}} \hspace{-1.2ex}
			+ \hspace{.25ex} \mathcircabove{\hspace{-0.25ex}\internalmoment}
		\\[1em]
		\mathcircabove{\hspace{-0.25ex}\internalmoment} \hspace{-0.2ex} = \hspace{-0.2ex} \internalmoment(0) \hspace{-0.3ex} = \hspace{-0.2ex} 0
		\\[.2em]
		\internalmoment(4 \ell) \hspace{-0.3ex} = \hspace{-0.2ex} 0
		\\[1em]
		E \hspace{-0.1ex} J \hspace{.1ex} \sectionrotationangle(\beamcoordinate) \hspace{-0.2ex}
			= \hspace{-0.5ex}
				\scalebox{1.5}{$\integral$} \hspace{-0.1ex} \internalmoment \hspace{.2ex} d \beamcoordinate \hspace{-0.1ex}
			= \hspace{-0.3ex} \biggl[
				\leftreaction \frac{\beamcoordinate^2}{2}
			\biggr]_{\hspace{-0.2ex} \scalebox{.75}{$ 0 $}}^{\hspace{-0.3ex} \scalebox{.75}{$ 4\ell $}} \hspace{-1.25ex}
			- \hspace{-0.3ex} \biggl[
				\leftforce \frac{( \beamcoordinate \hspace{-0.1ex} - \ell )^{\hspace{-0.1ex}2}}{2} \hspace{-0.2ex}
			\biggr]_{\hspace{-0.3ex} \scalebox{.75}{$ \ell $}}^{\hspace{-0.3ex} \scalebox{.75}{$ 4\ell $}} \hspace{-1.25ex}
			- \hspace{-0.3ex} \biggl[
				\rightforce \frac{( \beamcoordinate \hspace{-0.1ex} - 2\ell )^{\hspace{-0.1ex}2}}{2} \hspace{-0.2ex}
			\biggr]_{\hspace{-0.3ex} \scalebox{.75}{$ 2\ell $}}^{\hspace{-0.3ex} \scalebox{.75}{$ 4\ell $}} \hspace{-1.2ex}
			+ \hspace{.25ex} \mathcircabove{\hspace{-0.25ex}\internalmoment} \beamcoordinate \hspace{-0.1ex}
			+ \hspace{-0.2ex} \mathcircabove{\hspace{.25ex}\sectionrotationangle}
		\\[1em]
		E \hspace{-0.1ex} J \hspace{.1ex} \beamdeflection(\beamcoordinate) \hspace{-0.2ex}
			= \hspace{-0.5ex}
				\scalebox{1.5}{$\integral$} \hspace{-0.2ex}
				E \hspace{-0.1ex} J \hspace{.1ex} \sectionrotationangle \hspace{.2ex} d \beamcoordinate \hspace{-0.1ex}
			= \hspace{-0.3ex} \biggl[
				\leftreaction \frac{\beamcoordinate^3}{6}
			\biggr]_{\hspace{-0.2ex} \scalebox{.75}{$ 0 $}}^{\hspace{-0.3ex} \scalebox{.75}{$ 4\ell $}} \hspace{-1.25ex}
			- \hspace{-0.3ex} \biggl[
				\leftforce \frac{( \beamcoordinate \hspace{-0.1ex} - \ell )^{\hspace{-0.1ex}3}}{6} \hspace{-0.2ex}
			\biggr]_{\hspace{-0.3ex} \scalebox{.75}{$ \ell $}}^{\hspace{-0.3ex} \scalebox{.75}{$ 4\ell $}} \hspace{-1.25ex}
			- \hspace{-0.3ex} \biggl[
				\rightforce \frac{( \beamcoordinate \hspace{-0.1ex} - 2\ell )^{\hspace{-0.1ex}3}}{6} \hspace{-0.2ex}
			\biggr]_{\hspace{-0.3ex} \scalebox{.75}{$ 2\ell $}}^{\hspace{-0.3ex} \scalebox{.75}{$ 4\ell $}} \hspace{-1.2ex}
			+ \hspace{.25ex} \mathcircabove{\hspace{-0.25ex}\internalmoment} \frac{\beamcoordinate^2}{2}
			+ \hspace{-0.2ex} \mathcircabove{\hspace{.25ex}\sectionrotationangle} \beamcoordinate \hspace{-0.1ex}
			+ \hspace{-0.2ex} \mathcircabove{\hspace{.25ex}\beamdeflection}
		\\[1em]
		\mathcircabove{\hspace{.25ex}\beamdeflection} \hspace{-0.1ex} = \hspace{-0.1ex} \beamdeflection(0) \hspace{-0.2ex} = \hspace{-0.1ex} 0
		\\[.1em]
		\beamdeflection(4 \ell) \hspace{-0.3ex} = \hspace{-0.2ex} 0
	\end{array} $}} ;

%% .........
%% ............
%%

\end{tikzpicture}

\end{document}
