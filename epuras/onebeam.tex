\documentclass[tikz, margin=10mm]{standalone}

\usepackage{tikz}

\usetikzlibrary{calc}
\usetikzlibrary{arrows, arrows.meta}

\usepackage{xargs}
\usepackage{xifthen}

\begin{document}

\begin{tikzpicture}[scale=1]

\def\textscale{2}
\pgfmathsetmacro\smallertextscale{.9*\textscale}
\pgfmathsetmacro\evensmallertextscale{.8*\textscale}

\def\beamlength{12}
\pgfmathsetmacro\halfbeamlength{\beamlength / 2}

\def\beamlinewidth{3pt}
\tikzstyle{beam line} = [line width=\beamlinewidth, line cap=round, color=black]

%
% beam
%

\newcommand{\halfell}{ \raisebox{.3em}{$\ell$} \hspace{-0.4ex} / \hspace{-0.36ex} \raisebox{-0.33em}{\scalebox{.88}{$2$}} }

\draw [beam line] (0, 0) -- (\beamlength, 0)
	node [pos=0.49, below, shape=circle, fill=white, inner sep=-2pt, outer sep=11pt] {\scalebox{\textscale}{$ \ell $}} ;

\newcommand\redrawbeam{\draw [beam line] (0, 0) -- (\beamlength, 0) ;}

% extend the beam ends downwards

\def\shearforcesepurepos{-6.5}
\def\momentsepurepos{-10}

\tikzstyle{epure aux} = [line width=1pt, color=black, line cap=round, dash pattern=on 0pt off 1.6\pgflinewidth]

\draw [epure aux] (0, 0) -- (0, \momentsepurepos) ;
\draw [epure aux] (\beamlength, 0) -- (\beamlength, \momentsepurepos) ;

%
% external loads
%

\def\continuousloadheight{.9}
\tikzstyle{continuous load line} =
	[line width=2pt, orange, line cap=round, -{Triangle[round, length=15pt, width=9pt]}]
\tikzstyle{continuous load level} =
	[line width=1.5pt, orange, line cap=round]

\tikzstyle{concentrated load line} =
	[line width=2pt, red, line cap=round, -{Triangle[round, length=15pt, width=9pt]}]

\pgfmathsetmacro\continuousstep{\halfbeamlength / 9.6}

\pgfmathsetmacro\firstxcontinuous{\halfbeamlength}
\pgfmathsetmacro\nextxcontinuous{\firstxcontinuous - \continuousstep}
\pgfmathsetmacro\lastxcontinuous{0}
\foreach \xcontinuous in {\firstxcontinuous, \nextxcontinuous, ..., \lastxcontinuous}
	\draw [continuous load line] (\xcontinuous, \continuousloadheight) -- (\xcontinuous, 0) ;
%%\draw [line width=2pt, orange, line cap=round, dash pattern=on 0pt off 1.6\pgflinewidth] (0, \continuousloadheight) -- (0, 0) ;

\pgfmathsetmacro\firstxcontinuous{\halfbeamlength + \continuousstep}
\pgfmathsetmacro\nextxcontinuous{\firstxcontinuous + \continuousstep}
\pgfmathsetmacro\lastxcontinuous{\beamlength}
\foreach \xcontinuous in {\firstxcontinuous, \nextxcontinuous, ..., \lastxcontinuous}
	\draw [continuous load line] (\xcontinuous, \continuousloadheight) -- (\xcontinuous, 0) ;
%%\draw [line width=2pt, orange, line cap=round, dash pattern=on 0pt off 1.6\pgflinewidth] (\beamlength, \continuousloadheight) -- (\beamlength, 0) ;

\draw [continuous load level] (\beamlength, \continuousloadheight) -- (0, \continuousloadheight)
	node [pos=0.49, above, shape=circle, fill=white, inner sep=-2pt, outer sep=8pt] {\scalebox{\textscale}{$q$}} ;

%
% constraints
%

\input{constraints.include}

% uncomment to see more constraints
%%\addpinnedmidhingeat{(.25*\beamlength, 0)}
%%\addmovablehingeat{(.8*\beamlength, 0)}[mid]
%%\addpinnedendhingeat{(\beamlength, -2*\hingesize)}

% redraw beam after loads and middle hinges before clamps
\redrawbeam

\addpinnedendhingeat{(\beamlength, 0)}[flip]
\addmovablehingeat{(0, 0)}

%
% reactions of constraints
%

\def\beamwithreactionspos{-3}
\def\continuousloadheight{.88}

\pgfmathsetmacro\continuousstep{\halfbeamlength / 10}

\pgfmathsetmacro\firstxcontinuous{0}
\pgfmathsetmacro\nextxcontinuous{\firstxcontinuous + \continuousstep}
\pgfmathsetmacro\lastxcontinuous{\beamlength}
\foreach \xcontinuous in {\firstxcontinuous, \nextxcontinuous, ..., \lastxcontinuous}
	\draw [continuous load line, -{Triangle[round, length=12.5pt, width=7.5pt]}]
		(\xcontinuous, \beamwithreactionspos + \continuousloadheight) -- ++(0, -\continuousloadheight) ;

\draw [continuous load level] (0, \beamwithreactionspos + \continuousloadheight) -- ++(\beamlength, 0)
	node [pos=0.5, above, shape=circle, fill=white, inner sep=-2pt, outer sep=8pt] {\scalebox{\textscale}{$q$}} ;

\def\reactionlength{1.4}

\draw [concentrated load line, color=blue]
	(\beamlength, \beamwithreactionspos - \reactionlength) -- ++(0, \reactionlength)
	node [pos=0.33, left, shape=circle, fill=white, inner sep=-5pt, outer sep=3pt]
	{\scalebox{\smallertextscale}{$\displaystyle\frac{\raisebox{-0.1em}{$q\ell$}}{\raisebox{.1em}{$2$}}$}} ;

\def\zeroloadlength{.7}
\draw [concentrated load line, color=blue]
	(\beamlength + \zeroloadlength, \beamwithreactionspos) -- ++(-\zeroloadlength, 0)
	node [pos=0.2, above, shape=circle, fill=white, inner sep=-2pt, outer sep=8pt]
	{\scalebox{\smallertextscale}{$0$}} ;

\draw [concentrated load line, color=blue]
	(0, \beamwithreactionspos - \reactionlength) -- ++(0, \reactionlength)
	node [pos=0.33, left, shape=circle, fill=white, inner sep=-5pt, outer sep=3pt]
	{\scalebox{\smallertextscale}{$\displaystyle\frac{\raisebox{-0.1em}{$q\ell$}}{\raisebox{.1em}{$2$}}$}} ;

% beam again
\draw [beam line] (0, \beamwithreactionspos) -- ++(\beamlength, 0) ;

%
% beam coordinate
%

\def\beamcoordinate{\xi} % \zeta
\pgfmathsetmacro\coordinatepos{.22 + ( \shearforcesepurepos + \momentsepurepos ) / 2}

\tikzstyle{epure parameter} = [line width=.5pt, color=black, fill=white]

\draw [ epure aux ] (0, \momentsepurepos) -- (0, \coordinatepos) ;
\draw [ epure parameter, -{To[round, length=6pt, width=8pt]} ]
	( 0, \coordinatepos ) -- ( 2, \coordinatepos )
	node [ pos=0.8, above left, inner sep=0pt, outer sep=4pt ]
		{\scalebox{\evensmallertextscale}{$ \beamcoordinate $}} ;
\draw [ epure parameter ] ( 0, \coordinatepos ) circle ( 2.6pt ) ;

%%\draw [ epure aux ] (\beamlength, \momentsepurepos) -- (\beamlength, \coordinatepos) ;
%%\draw [ epure parameter, -{To[round, length=6pt, width=8pt]} ]
%%	( \beamlength, \coordinatepos ) -- ( \beamlength - 2, \coordinatepos )
%%	node [ pos=0.8, above right, inner sep=0pt, outer sep=4pt ]
%%		{\scalebox{\evensmallertextscale}{$ \beamcoordinate $}} ;
%%\draw [ epure parameter ] ( \beamlength, \coordinatepos ) circle ( 2.6pt ) ;

%
% epuras
%

\tikzstyle{epure line} = [line width=1.5pt, color=black, line cap=round]
\tikzstyle{epure hatch} = [line width=.5pt, color=black]

\def\momentqll{12}
\pgfmathsetmacro\justq{\momentqll / (\beamlength * \beamlength)}

\pgfmathsetmacro\epurehatchstep{\beamlength / 20}
\pgfmathsetmacro\nextxhatch{\beamlength - \epurehatchstep}

\newcommand\internalshearforce{\scalebox{.9}[1]{$ \mathcal{Q} $}}
\newcommand\internalmoment{\scalebox{.9}[1]{$ \mathcal{M} $}}

%
% epure of internal shear forces
%

\pgfmathsetmacro\qforforce{2.66 * \justq}
\pgfmathsetmacro\shearforceatzero{ \qforforce * ( \beamlength / 2 ) }
\newcommand\shearforcefunction[1]{{ \shearforceatzero - ( \qforforce * #1 ) }}

\begin{scope}[shift={(0, \shearforcesepurepos)}]
\draw [epure line]
	plot [variable=\z, samples=100, domain=0:\beamlength]
		( \z, \shearforcefunction{\z} ) ;
\end{scope}

\foreach \xhatch in {\beamlength, \nextxhatch, ..., 0}
	\pgfmathsetmacro\righttoleftx{\beamlength - \xhatch}
	\pgfmathsetmacro\currentshearforce{\shearforcefunction{\righttoleftx}}
	\draw [epure hatch] (\xhatch, \shearforcesepurepos) -- ++(0, -\currentshearforce) ;

\draw [epure line] (0, \shearforcesepurepos) -- ++(\beamlength, 0) ;

\node [right, shape=circle, yshift=.2em, draw=black, line width=.5pt, fill=white, inner sep=2.5pt, outer sep=10pt]
	at (\beamlength, \shearforcesepurepos)
	{\scalebox{\textscale}{$ \internalshearforce $}} ;

%
% epure of internal bending moments
%

\newcommand\momentfunction[1]{{ ( \justq * ( \beamlength / 2 ) * #1 ) - ( \justq * #1 * #1 / 2 ) }}

\begin{scope}[shift={(0, \momentsepurepos)}]
\draw [epure line]
	plot [variable=\z, samples=100, domain=0:\beamlength]
		( \z, \momentfunction{\z} ) ;
\end{scope}

\foreach \xhatch in {\beamlength, \nextxhatch, ..., 0}
	\pgfmathsetmacro\righttoleftx{\beamlength - \xhatch}
	\pgfmathsetmacro\currentmoment{\momentfunction{\righttoleftx}}
	\draw [epure hatch] (\xhatch, \momentsepurepos) -- ++(0, \currentmoment) ;

\draw [epure line] (0, \momentsepurepos) -- ++(\beamlength, 0) ;

\node [right, shape=circle, yshift=.2em, draw=black, line width=.5pt, fill=white, inner sep=2.5pt, outer sep=10pt]
	at (\beamlength, \momentsepurepos)
	{\scalebox{\textscale}{$ \internalmoment $}} ;

%
% text
%

\newcommand\verynicefrac[2]{\raisebox{.11ex}{$ \raisebox{.3em}{\scalebox{0.7}{$#1$}} \hspace{-0.4ex} / \hspace{-0.46ex} \raisebox{-0.25em}{\scalebox{0.7}{$#2$}}\hspace{.15ex} $}}

\newcommand\ellsquared{ \ell^{\hspace{.1ex}\scalebox{.75}{\raisebox{.1em}{$2$}}} }

\node [above, inner sep=0pt, outer sep=21pt] at (\halfbeamlength, \shearforcesepurepos)
	{\scalebox{\evensmallertextscale}{$ \begin{array}{c}
		\internalshearforce(\hspace{-0.1ex}\beamcoordinate) \hspace{-0.2ex}
			= \hspace{-0.2ex} - \hspace{.1ex} q \hspace{.1ex} \beamcoordinate \hspace{-0.2ex}
			+ \hspace{-0.2ex} \textstyle\frac{q\ell}{2}
	\end{array} $}} ;

\node [below, inner sep=0pt, outer sep=9pt] at (\halfbeamlength, \momentsepurepos)
	{\scalebox{\evensmallertextscale}{$ \begin{array}{c}
		\internalmoment(\hspace{-0.1ex}\beamcoordinate) \hspace{-0.2ex}
			= \hspace{-0.2ex} - \textstyle\frac{q{\beamcoordinate}^2\hspace{-0.4ex}}{2} \hspace{-0.2ex}
			+ \hspace{-0.2ex} \textstyle\frac{q\ell}{2} \beamcoordinate
		\\[.25em]
		\internalmoment_{\mathsf{max}} \hspace{-0.25ex}
			= \hspace{-0.1ex} \internalmoment \bigl( \scalebox{.88}{$\halfell$}\hspace{.1ex} \bigr) \hspace{-0.25ex}
			= \hspace{-0.1ex} \verynicefrac{1}{8} q \ellsquared
	\end{array} $}} ;

\pgfmathsetmacro\maxmoment{\momentfunction{\halfbeamlength}}
\draw [epure line, color=orange!50!red]
	(\halfbeamlength, \momentsepurepos) -- ++(0, \maxmoment)
	node [pos=1, xshift=.33em, above, inner sep=0pt, outer sep=7pt]
	{\scalebox{\textscale}{$ \verynicefrac{1}{8} q \ellsquared $}} ;

\end{tikzpicture}

\end{document}
