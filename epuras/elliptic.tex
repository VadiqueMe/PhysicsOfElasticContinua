\documentclass[tikz, margin=10mm]{standalone}

\usepackage{tikz}

\usetikzlibrary{calc}
\usetikzlibrary{arrows, arrows.meta}

\usepackage{xargs}
\usepackage{xifthen}

\usepackage{bm}

\usepackage{graphicx}
\usepackage{scalerel}
\usepackage{amsmath}
\DeclareMathOperator*{\integral}{\scalerel*{\rotatebox{12}{$\!\textstyle\int\!$}}{\int}} % \rotatebox{8} for vertical line
\DeclareMathOperator*{\ointegral}{\scalerel*{\rotatebox{12}{$\!\textstyle\oint\!$}}{\oint}}

\makeatletter
\newcommand*\dotp{\mathpalette\dotp@{.55}}
\newcommand*\bigdot{\mathpalette\dotp@{.64}}
\newcommand*\dotp@[2]{\mathbin{\vcenter{\hbox{\scalebox{#2}{$\m@th#1\bullet$}}}}}
\makeatother
\newcommand\dotdotp{\dotp\hspace{-0.16em}\dotp\hspace{.20em}}

\newcommand\mathboldM{\scalebox{0.9}[1]{\ensuremath{\bm{M}\hspace{-0.1em}}}}
\newcommand\mathboldQ{\scalebox{0.9}[1]{\ensuremath{\bm{Q}\hspace{-0.1ex}}}}

\makeatletter
\def\mathcolor#1#{\@mathcolor{#1}}
\def\@mathcolor#1#2#3{%
	\protect\leavevmode
	\begingroup\color#1{#2}#3\endgroup
}
\makeatother

\usepackage{comfortaa}

%%\usepackage{verbatim}

\begin{document}

\begin{tikzpicture}[scale = 1]

\def\textscale{2}
\pgfmathsetmacro\smalltextscale{.84*\textscale}

\def\secondbeamxshift{20cm}

\node [ above, inner sep=0pt, outer sep=0pt ]
	at ( \secondbeamxshift / 2, 9 )
	{\scalebox{2}{ \comfortaa{Le problème de Vadique à propos d'une poutre elliptique} }} ;

\def\beamlinewidth{3pt}
\tikzstyle{beam line} = [ line width=\beamlinewidth, line cap=round, color=black ]

\def\dottedbeamlinewidth{2pt}
\tikzstyle{dotted beam line} = [ line width=\dottedbeamlinewidth, line cap=round, color=black, dash pattern=on 0pt off 1.5\pgflinewidth ]

\def\beamlength{12}
\pgfmathsetmacro\halfbeamlength{\beamlength / 2}
\def\beamlengtha{\halfbeamlength}
\pgfmathsetmacro\beamlengthb{\beamlengtha * 2/3}

\newcommand\beamcoordinate{s}

\newcommand\beamfx[1]{\beamlengtha * cos(#1)}
\newcommand\beamfy[1]{\beamlengthb * sin(#1)}

\def\qlscale{1} % forces/moments scale
\pgfmathsetmacro\justq{\qlscale / \beamlength}

\newcommand\extforcex[1]{0}
\newcommand\extforcey[1]{-\justq}

\newcommand\altextforcex[1]{-\justq * (\beamlengthb / \beamlengtha) * cos(#1)}
\newcommand\altextforcey[1]{-\justq * sin(#1)}

%
% beam
%

\newcommand\drawbeam{%
	\draw [ beam line ]
		plot [ variable=\s, domain=0:180, samples=200 ]
		( {\beamfx{\s}}, {\beamfy{\s}} ) ;
}

\drawbeam

\begin{scope}[xshift = \secondbeamxshift]
\drawbeam
\end{scope}

\tikzstyle{aux dashed} = [ line width=1pt, color=black, line cap=round, dash pattern=on 0pt off 1.6\pgflinewidth ]

\newcommand\drawlengths{%
\draw [aux dashed] ( 0, 0 ) -- ++( \beamlengtha, 0 )
	node [ pos=0.47, above, fill=white, inner sep=-2pt, outer sep=8pt ]
	{\scalebox{\textscale}{$ a $}} ;
\draw [aux dashed] ( 0, 0 ) -- ++( -\beamlengtha, 0 )
	node [ pos=0.48, above, fill=white, inner sep=-2pt, outer sep=8pt ]
	{\scalebox{\textscale}{$ a $}} ;
\draw [aux dashed] ( 0, 0 ) -- ++( 0, \beamlengthb )
	node [ pos=0.61, left, fill=white, inner sep=-2pt, outer sep=8pt ]
	{\scalebox{\textscale}{$ b $}} ;
}

\tikzstyle{location vector} =
	[line width=1.5pt, color=black, line cap=round, -{Triangle[round, length=16pt, width=8pt]}]

\pgfmathsetmacro\functionstextypos{\beamlengthb + 3}

\newcommand\drawbeamparticlelocation{%
\draw [ location vector ]
	( 0, 0 ) -- ( {\beamfx{54}}, {\beamfy{54}} )
	node [ pos=.66, above left, inner sep=0pt, outer sep=1pt ]
	{\scalebox{\textscale}{$ \bm{r} \hspace{-0.1ex} (\hspace{-0.2ex} \beamcoordinate \hspace{-0.12ex}) $}} ;

\node [ anchor=west, color=black, inner sep=0pt, outer sep=0pt ]
	at ( -\beamlengtha, \functionstextypos )
	{\scalebox{\smalltextscale}{$ \begin{array}{l}
r_{\hspace{-0.33ex}\raisebox{-0.1em}{$\scriptstyle \eta$}} \hspace{-0.1ex} (\hspace{-0.2ex} \beamcoordinate \hspace{-0.12ex}) \hspace{-0.3ex} = \hspace{-0.1ex} a \cos \beamcoordinate \\
r_{\hspace{-0.33ex}\raisebox{-0.1em}{$\scriptstyle \zeta$}} \hspace{-0.1ex} (\hspace{-0.2ex} \beamcoordinate \hspace{-0.12ex}) \hspace{-0.3ex} = \hspace{-0.1ex} b \sin \beamcoordinate
\end{array} $}} ;
}

\tikzstyle{aux line} = [ line width=1pt, color=black ]

\def\initialpositioncircleradius{2.5pt}
\def\showcoordinatelength{.1*\beamlength}
\def\showbeamcoordinatecoefficient{.88}

\newcommand\drawcoordinates{%
\draw [ aux line, -{To[round, length=8pt, width=10pt]} ]
	( 0, 0 ) -- ++( \showcoordinatelength, 0 )
	node [ pos=.8, below, inner sep=0pt, outer sep=11pt ]
		{\scalebox{\textscale}{$ \eta $}} ;
\draw [ aux line, -{To[round, length=8pt, width=10pt]} ]
	( 0, 0 ) -- ++( 0, \showcoordinatelength )
	node [ pos=.8, left, inner sep=0pt, outer sep=8pt ]
		{\scalebox{\textscale}{$ \zeta $}} ;

\draw [ aux line, fill=white ] ( 0, 0 ) circle ( \initialpositioncircleradius ) ;

\draw [ aux line, -{To[round, length=8pt, width=10pt]} ]
	plot [ variable=\s, domain=0:24, samples=100 ]
	( {\showbeamcoordinatecoefficient * \beamfx{\s}}, {\showbeamcoordinatecoefficient * \beamfy{\s}} ) ;
\node [ left, inner sep=0pt, outer sep=8pt ]
	at ( {\showbeamcoordinatecoefficient * \beamfx{9}}, {\showbeamcoordinatecoefficient * \beamfy{9}} )
	{\scalebox{\textscale}{$ \beamcoordinate $}} ;

\draw [ aux line, fill=white ] ( \showbeamcoordinatecoefficient * \beamlengtha, 0 ) circle ( \initialpositioncircleradius ) ;
}

\drawlengths
\drawbeamparticlelocation
\drawcoordinates

\begin{scope}[xshift = \secondbeamxshift]
\drawlengths
\drawbeamparticlelocation
\drawcoordinates
\end{scope}

%
% external loads
%

\def\extloadcolor{orange!66!red}

\tikzstyle{continuous load vector} =
	[ line width=2pt, \extloadcolor, line cap=round, -{Triangle[round, length=13pt, width=8pt]} ]

\tikzstyle{continuous load level} =
	[ line width=1.5pt, \extloadcolor, opacity=.8, line cap=round ]

\tikzstyle{dashed load level} =
	[ line width=1.5pt, \extloadcolor, opacity=.8, line cap=round, dash pattern=on 0pt off 1.6\pgflinewidth ]

\tikzstyle{concentrated load vector} =
	[ line width=2pt, red, line cap=round, -{Triangle[round, length=15pt, width=9pt]} ]

\pgfmathsetmacro\continuousloadscale{15}

\def\extforcexshift{0}
\def\extforceyshift{0}

\newcommand\extforcebeginsatx[1]{{ \beamfx{#1} - \continuousloadscale * ( \extforcefx{#1} ) + \extforcexshift }}
\newcommand\extforcebeginsaty[1]{{ \beamfy{#1} - \continuousloadscale * ( \extforcefy{#1} ) + \extforceyshift }}
S
\newcommand\extloadtext{\bm{q}(\hspace{-0.2ex} \beamcoordinate \hspace{-0.12ex})}

\def\howmanyforcelinesoneachhalf{12}

\newcommand\drawextload{%
\pgfmathsetmacro\qstep{90 / ( \howmanyforcelinesoneachhalf - 1)}

\pgfmathsetmacro\qfrom{90}
\pgfmathsetmacro\qnext{\qfrom + \qstep}
\pgfmathsetmacro\qlast{180 - 1*\qstep}
\foreach \q in {\qfrom, \qnext, ..., \qlast}
	\pgfmathsetmacro\sq{\q} %% 180 - \q
	\draw [ continuous load vector ]
	( \extforcebeginsatx{\sq}, \extforcebeginsaty{\sq} ) --
	++( {\continuousloadscale * ( \extforcefx{\sq} )}, {\continuousloadscale * ( \extforcefy{\sq} )} ) ;

\pgfmathsetmacro\qfrom{90 - \qstep}
\pgfmathsetmacro\qnext{90 - 2*\qstep}
\pgfmathsetmacro\qlast{0 + 1*\qstep}
\foreach \q in {\qfrom, \qnext, ..., \qlast}
	\pgfmathsetmacro\sq{\q} %% 180 - \q
	\draw [ continuous load vector ]
		( \extforcebeginsatx{\sq}, \extforcebeginsaty{\sq} ) --
		++( {\continuousloadscale * ( \extforcefx{\sq} )}, {\continuousloadscale * ( \extforcefy{\sq} )} ) ;

\draw [ continuous load level ]
	plot [ variable=\s, domain=0:180, samples=100 ] ( \extforcebeginsatx{\s}, \extforcebeginsaty{\s} ) ;
\node [ above, color=\extloadcolor, inner sep=0pt, outer sep=7pt ]
	at ( \extforcebeginsatx{89}, \extforcebeginsaty{89} )
	{\scalebox{\textscale}{$ \extloadtext $}} ;
}

\def\extforcefx{\extforcex}
\def\extforcefy{\extforcey}
\drawextload

\node [ anchor=east, color=\extloadcolor, inner sep=0pt, outer sep=0pt ]
	at ( \beamlengtha, \functionstextypos )
	{\scalebox{\smalltextscale}{$ \begin{array}{l}
q_{\raisebox{-0.1em}{$\scriptstyle \eta$}} \hspace{-0.1ex} (\hspace{-0.2ex} \beamcoordinate \hspace{-0.12ex}) \hspace{-0.3ex} = \hspace{-0.1ex} 0 \\
q_{\raisebox{-0.1em}{$\scriptstyle \zeta$}} \hspace{-0.1ex} (\hspace{-0.2ex} \beamcoordinate \hspace{-0.12ex}) \hspace{-0.3ex} = \hspace{-0.1ex} - \hspace{.1ex} q
\end{array} $}} ;

\begin{scope}[xshift = \secondbeamxshift]
\def\extforcefx{\altextforcex}
\def\extforcefy{\altextforcey}
\drawextload

\node [ anchor=east, color=\extloadcolor, inner sep=0pt, outer sep=0pt ]
	at ( \beamlengtha, \functionstextypos )
	{\scalebox{\smalltextscale}{$ \begin{array}{l}
q_{\raisebox{-0.1em}{$\scriptstyle \eta$}} \hspace{-0.1ex} (\hspace{-0.2ex} \beamcoordinate \hspace{-0.12ex}) \hspace{-0.3ex} = \hspace{-0.1ex} - \raisebox{-0.1em}{\scalebox{1.4}{$ \textstyle\frac{\scalebox{.7}{$b$}}{\scalebox{.7}{\raisebox{.15em}{$a$}}} $}} \hspace{.1ex} q \cos \beamcoordinate \\
q_{\raisebox{-0.1em}{$\scriptstyle \zeta$}} \hspace{-0.1ex} (\hspace{-0.2ex} \beamcoordinate \hspace{-0.12ex}) \hspace{-0.3ex} = \hspace{-0.1ex} - \hspace{.1ex} q \sin \beamcoordinate
\end{array} $}} ;
\end{scope}

%
% constraints and reactions of constraints
%

\input{constraints.include}

\pgfmathsetmacro\reactionstextscale{.9*\textscale}

\def\reactionlength{1.6}

\def\beamcopypos{-\beamlengthb - 3.3}

\newcommand\verticalreactionequalsto{ \raisebox{-0.2ex}{$ \raisebox{.3em}{\scalebox{.84}{$ \pi $}} \hspace{-0.54ex} / \hspace{-0.45ex} \raisebox{-0.25em}{\scalebox{.76}{$ 2 $}} $} \hspace{.2ex} q }

\newcommand\drawbeamwithreactions{%
	%%\draw [ aux dashed ] ( -\beamlengtha, 0 ) -- ++( 0, \beamcopypos ) ;
	%%\draw [ aux dashed ] ( \beamlengtha, 0 ) -- ++( 0, \beamcopypos ) ;

	\draw [ beam line ]
		plot [ variable=\s, domain=0:180, samples=200 ]
		( {\beamfx{\s}}, {\beamfy{\s} + \beamcopypos} ) ;

	% copy external load
	\def\extforceyshift{\beamcopypos}
	\renewcommand\extloadtext{\bm{q}\vphantom{()}}
	\drawextload

	\draw [ concentrated load vector, color=blue ]
		( \beamlengtha, \beamcopypos - \reactionlength ) -- ++( 0, \reactionlength )
		node [ pos=.3, left, inner sep=0pt, outer sep=7pt ]
		{\scalebox{\reactionstextscale}{$
\scalebox{.75}{$ \displaystyle\frac{\raisebox{-0.2em}{$ 1 $}}{\raisebox{.02em}{$ 2 $}} $}
\scalebox{.7}{$ \displaystyle\integral_{\scalebox{.77}{$\hspace{.2em}0$}}^{\scalebox{.88}{$\hspace{.2em}\pi$}} $}
\hspace{-0.25ex} {-} \hspace{.1ex} q_{\raisebox{-0.1em}{$\scriptstyle \zeta$}} d\beamcoordinate
\hspace{-0.1ex} = \hspace{-0.1ex}
\verticalreactionequalsto
		$}} ;
	\draw [ concentrated load vector, color=blue ]
		( -\beamlengtha, \beamcopypos - \reactionlength ) -- ++( 0, \reactionlength )
		node [ pos=.3, right, inner sep=0pt, outer sep=7pt ]
		{\scalebox{\reactionstextscale}{$
\hspace{-0.1em}\verticalreactionequalsto
\hspace{-0.1ex} = \hspace{-0.1ex}
\scalebox{.75}{$ \displaystyle\frac{\raisebox{-0.2em}{$ 1 $}}{\raisebox{.02em}{$ 2 $}} $}
\scalebox{.7}{$ \displaystyle\integral_{\scalebox{.77}{$\hspace{.2em}0$}}^{\scalebox{.88}{$\hspace{.2em}\pi$}} $}
\hspace{-0.25ex} {-} \hspace{.1ex} q_{\raisebox{-0.1em}{$\scriptstyle \zeta$}} d\beamcoordinate
		$}} ;
}

% redraw beams after loads and middle hinges before clamps
\drawbeam
\begin{scope}[xshift = \secondbeamxshift]
\drawbeam
\end{scope}

% reverse the parameter
%%\draw [ beam line, color=red ]
%%		plot [ variable=\s, domain=0:100, samples=200 ]
%%		( {\beamfx{(180 - \s)}}, {\beamfy{(180 - \s)}} ) ;

\def\extforcefx{\extforcex}
\def\extforcefy{\extforcey}
\drawbeamwithreactions

\def\zeroloadlength{.9}

\pgfmathsetmacro\zeroloadbeginx{\beamlengtha + \zeroloadlength}
\def\zeroloadendx{\beamlengtha}
\draw [ concentrated load vector, color=blue ]
	( \zeroloadbeginx, \beamcopypos ) -- ( \zeroloadendx, \beamcopypos )
	node [ pos=0, above right, inner sep=-2pt, outer sep=-9pt ]
	{\scalebox{\reactionstextscale}{$
0 \hspace{-0.1ex} = \hspace{-0.4ex}
\scalebox{.7}{$ \displaystyle\integral_{\scalebox{.77}{$\hspace{.2em}0$}}^{\scalebox{.88}{$\hspace{.2em}\pi$}} $}
\hspace{-0.1ex} q_{\raisebox{-0.1em}{$\scriptstyle \eta$}} d\beamcoordinate
	$}} ;

\addpinnedendhingeat{(\beamlengtha, 0)}[flip]
\addmovablehingeat{(-\beamlengtha, 0)}

\begin{scope}[xshift = \secondbeamxshift]
\def\extforcefx{\altextforcex}
\def\extforcefy{\altextforcey}

\renewcommand\verticalreactionequalsto{q}
\drawbeamwithreactions

\pgfmathsetmacro\zeroloadbeginx{-\beamlengtha - \zeroloadlength}
\def\zeroloadendx{-\beamlengtha}
\draw [ concentrated load vector, color=blue ]
	( \zeroloadbeginx, \beamcopypos ) -- ( \zeroloadendx, \beamcopypos )
	node [ pos=.2, below left, inner sep=-2pt, outer sep=-3pt ]
	{\scalebox{\reactionstextscale}{$
\scalebox{.7}{$ \displaystyle\integral_{\scalebox{.77}{$\hspace{.2em}0$}}^{\scalebox{.88}{$\hspace{.2em}\pi$}} $}
\hspace{-0.25ex} {-} \hspace{.1ex} q_{\raisebox{-0.1em}{$\scriptstyle \eta$}} d\beamcoordinate
\hspace{-0.15ex} = \hspace{-0.1ex} 0
	$}} ;

\addpinnedendhingeat{(-\beamlengtha, 0)}
\addmovablehingeat{(\beamlengtha, 0)}
\end{scope}

%
% balance of forces
%

\def\pointcolor{black}
\tikzstyle{point} = [ line width=1pt, color=\pointcolor, fill=\pointcolor ]
\def\endpointcolor{black}
\tikzstyle{end point} = [ line width=1pt, color=\endpointcolor, fill=\endpointcolor ]
\def\pointsize{2}
\def\endpointsize{\pointsize}

\def\partofbeamypos{\beamcopypos - \beamlengthb - 4}

\pgfmathsetmacro\internalforcetextscale{.84*\textscale}

\def\internalforcelengthx{1}
\def\internalforcelengthy{\reactionlength + .2}

\def\internalforcecolor{black!10!green}
\def\initialinternalforcecolor{green!40!blue}

\newcommand\internalforcextext{0}
\newcommand\internalforceytext{%
	q \beamcoordinate \hspace{-0.1ex}
	\mathcolor{\initialinternalforcecolor}{- \verticalreactionequalsto}
}

\newcommand\drawequilibriumofapoint{%
	\def\coordinateofapoint{109}

	\pgfmathsetmacro\xbalanceforpoint{-.6*\beamlengtha}
	\pgfmathsetmacro\ybalanceforpoint{\partofbeamypos - 3.3}

	\coordinate (point) at ($ ( {\beamfx{\coordinateofapoint}}, {\beamfy{\coordinateofapoint}} ) + ( \xbalanceforpoint, \ybalanceforpoint ) $) ;

	%%\pgfmathsetmacro\dottedbeamfirst{\coordinateofapoint - 9}
	%%\pgfmathsetmacro\dottedbeamlast{\coordinateofapoint + 10}
	%%\draw [ dotted beam line, opacity=.8 ]
		%%plot [ variable=\s, domain=\dottedbeamfirst:\dottedbeamlast, samples=100 ]
		%%( {\beamfx{\s} + \xbalanceforpoint}, {\beamfy{\s} + \ybalanceforpoint} ) ;

	\draw [ concentrated load vector, color=\internalforcecolor ]
		(point) -- ++( \internalforcelengthx, \internalforcelengthy  )
		node [ pos=.5, right, inner sep=0pt, outer sep=9pt ]
		{\scalebox{\internalforcetextscale}{$ \mathboldQ(\hspace{-0.2ex} \beamcoordinate \hspace{-0.12ex}) $}} ;

	\draw [ concentrated load vector, color=\internalforcecolor ]
		(point) -- ++( -\internalforcelengthx, -\internalforcelengthy  )
		node [ pos=.6, right, inner sep=0pt, outer sep=7pt ]
		{\scalebox{\internalforcetextscale}{$ - \hspace{.1ex} \mathboldQ(\hspace{-0.2ex} \beamcoordinate \hspace{-0.12ex}) $}} ;

	\draw [ point ] (point) circle ( \pointsize\pgflinewidth )
		node [ above left, inner sep=0pt, outer sep=4pt ]
		{\scalebox{\internalforcetextscale}{$ s $}} ;
}

\newcommand\drawhowtogetinternalforce{%
	\pgfmathsetmacro\qlast{128}

	\draw [ beam line ]
		plot [ variable=\s, domain=0:\qlast, samples=200 ]
		( {\beamfx{\s}}, {\beamfy{\s} + \partofbeamypos} ) ;

	\def\extforceyshift{\partofbeamypos}
	\pgfmathsetmacro\qfrom{0}
	\pgfmathsetmacro\qnext{\qfrom + \qstep}
	\foreach \q in {\qfrom, \qnext, ..., \qlast}
	\draw [ continuous load vector ]
		( \extforcebeginsatx{\q}, \extforcebeginsaty{\q} ) --
		++( {\continuousloadscale * ( \extforcefx{\q} )}, {\continuousloadscale * ( \extforcefy{\q} )} ) ;

	\renewcommand\extloadtext{\bm{q}(\hspace{-0.2ex} \beamcoordinate \hspace{-0.12ex})}
	\draw [ continuous load level ]
		plot [ variable=\s, domain=0:\qlast, samples=100 ]
		( \extforcebeginsatx{\s}, \extforcebeginsaty{\s} ) ;
	\node [ above right, color=\extloadcolor, inner sep=0pt, outer sep=3pt ]
		at ( \extforcebeginsatx{33}, \extforcebeginsaty{33} )
		{\scalebox{\textscale}{$ \extloadtext $}} ;

	\draw [ concentrated load vector, color=blue ]
		( \beamlengtha, \partofbeamypos - \reactionlength ) -- ++( 0, \reactionlength )
		node [ pos=0.3, left, inner sep=0pt, outer sep=7pt ]
		{\scalebox{\textscale}{$ \verticalreactionequalsto $}} ;

	\draw [ concentrated load vector, color=\internalforcecolor ]
		( {\beamfx{\qlast}}, {\beamfy{\qlast} + \partofbeamypos} ) -- ++( 0, \internalforcelengthy  )
		node [ pos=.88, left, inner sep=0pt, outer sep=6pt ]
		{\scalebox{\textscale}{$ Q_{\hspace{-0.2ex}\raisebox{-0.1em}{$\scriptstyle \zeta$}} $}} ;

	\draw [ concentrated load vector, color=\internalforcecolor ]
		( {\beamfx{\qlast}}, {\beamfy{\qlast} + \partofbeamypos} ) -- ++( \internalforcelengthx, 0  )
		node [ pos=.6, below, inner sep=0pt, outer sep=9pt ]
		{\scalebox{\textscale}{$ Q_{\hspace{-0.2ex}\raisebox{-0.1em}{$\scriptstyle \eta$}} $}} ;

	\draw [ end point ]
		( {\beamfx{0}}, {\beamfy{0} + \partofbeamypos} )
		circle ( \endpointsize\pgflinewidth ) ;
	\draw [ end point ]
		( {\beamfx{\qlast}}, {\beamfy{\qlast} + \partofbeamypos} )
		circle ( \endpointsize\pgflinewidth ) ;

	% the sum of forces
	\node [ color=\internalforcecolor, inner sep=0pt, outer sep=0pt ]
		at ( .12*\beamlengtha, \partofbeamypos + .4 )
	{\scalebox{\internalforcetextscale}{$ \begin{array}{l}
\mathcolor{black}{\scalebox{.7}{$ \displaystyle\integral_{\scalebox{.77}{$\hspace{.2em}0$}}^{\scalebox{.96}{$\hspace{.2em} \beamcoordinate$}} $}}
\hspace{-0.2ex} \mathcolor{\extloadcolor}{\bm{q}} \hspace{.15ex} \mathcolor{black}{d\beamcoordinate}
%
\mathcolor{black}{+} \mathcolor{\internalforcecolor}{\mathboldQ(\hspace{-0.2ex} \beamcoordinate \hspace{-0.12ex})} \hspace{-0.2ex}
\mathcolor{black}{-}
\mathcolor{\initialinternalforcecolor}{\mathboldQ(\hspace{-0.1ex}0\hspace{-0.1ex})} \hspace{-0.2ex}
\mathcolor{black}{=}
\mathcolor{black}{\bm{0}}
\\[-0.1em]
%
Q_{\hspace{-0.15ex}\raisebox{-0.1em}{$\scriptstyle i$}}(\hspace{-0.2ex} \beamcoordinate \hspace{-0.12ex})
\hspace{-0.2ex} \mathcolor{black}{=} \hspace{-0.1ex}
\mathcolor{black}{\scalebox{.7}{$ \hspace{-0.25ex} \displaystyle\integral_{\scalebox{.77}{$\hspace{.2em}0$}}^{\scalebox{.96}{$\hspace{.2em} \beamcoordinate$}} $}}
\hspace{-0.66ex} \mathcolor{black}{-} \hspace{-0.4ex} \mathcolor{\extloadcolor}{q_{\raisebox{-0.1em}{$\scriptstyle i$}}} \hspace{.1ex} \mathcolor{black}{d\beamcoordinate}
\mathcolor{black}{+}
\mathcolor{\initialinternalforcecolor}{Q_{\hspace{-0.15ex}\raisebox{-0.1em}{$\scriptstyle i$}}(\hspace{-0.1ex}0\hspace{-0.1ex})}
	\end{array} $}} ;

% equilibrium of end point

\coordinate (endpointequilibrium) at ( -.55*\beamlengtha, \partofbeamypos - 3 ) ;

\draw [ concentrated load vector, color=blue ]
	($ (endpointequilibrium) - ( 0, \reactionlength ) $) -- ++( 0, \reactionlength )
	node [ pos=.3, left, inner sep=0pt, outer sep=6pt ]
	{\scalebox{\internalforcetextscale}{$ \verticalreactionequalsto $}} ;

\draw [ concentrated load vector, color=\initialinternalforcecolor ]
	(endpointequilibrium) -- ++( 0, \reactionlength )
	node [ pos=.75, left, inner sep=0pt, outer sep=7pt ]
	{\scalebox{\internalforcetextscale}{$ Q_{\hspace{-0.2ex}\raisebox{-0.1em}{$\scriptstyle \zeta$}}\hspace{-0.1ex}(\hspace{-0.1ex} 0 \hspace{-0.1ex}) $}} ;

\draw [ concentrated load vector, color=\initialinternalforcecolor ]
	(endpointequilibrium) -- ++( .63*\internalforcelengthx, 0 )
	node [ pos=1, yshift=-0.2em, right, inner sep=0pt, outer sep=4pt ]
	{\scalebox{\internalforcetextscale}{$ Q_{\hspace{-0.2ex}\raisebox{-0.1em}{$\scriptstyle \eta$}}\hspace{-0.1ex}(\hspace{-0.1ex} 0 \hspace{-0.1ex}) $}} ;

\draw [ end point ] (endpointequilibrium) circle ( \endpointsize\pgflinewidth )
	node [ left, yshift=.2ex, inner sep=0pt, outer sep=8pt ]
	{\scalebox{\internalforcetextscale}{$ s \hspace{.3ex}{=}\hspace{.3ex} 0 $}} ;

\node [ right, color=\internalforcecolor, inner sep=0pt, outer sep=0pt ]
		at ($ (endpointequilibrium) + ( .54*\beamlengtha, -0.3 ) $)
	{\scalebox{\internalforcetextscale}{$ \begin{aligned}
\mathcolor{\initialinternalforcecolor}{Q_{\hspace{-0.2ex}\raisebox{-0.1em}{$\scriptstyle \eta$}}\hspace{-0.1ex}(\hspace{-0.1ex}0\hspace{-0.1ex})}
\hspace{-0.2ex} &\mathcolor{black}{= 0}
\\
\mathcolor{\initialinternalforcecolor}{Q_{\hspace{-0.2ex}\raisebox{-0.1em}{$\scriptstyle \zeta$}}\hspace{-0.1ex}(\hspace{-0.1ex}0\hspace{-0.1ex})}
\hspace{-0.2ex} \mathcolor{black}{+}
\mathcolor{blue}{\verticalreactionequalsto}
&\mathcolor{black}{= 0}
	\end{aligned} $}} ;
}

\drawequilibriumofapoint
\drawhowtogetinternalforce

\begin{scope}[xshift = \secondbeamxshift]
\def\extforcefx{\altextforcex}
\def\extforcefy{\altextforcey}

\drawequilibriumofapoint

\renewcommand\verticalreactionequalsto{q}
\drawhowtogetinternalforce
\end{scope}

%
% field of internal force
%

\tikzstyle{internal force level} =
	[ line width=1.5pt, \internalforcecolor, opacity=.8, line cap=round ] %% dash pattern=on 0pt off 1.6\pgflinewidth

\def\internalforcefieldypos{\partofbeamypos - \beamlengthb - 6.7}

\newcommand\internalforcefx[1]{0}
\newcommand\internalforcefy[1]{( #1 - 90 ) * \justq}

\pgfmathsetmacro\internalforcescale{0.5}

\def\howmanyinternalforcelines{24}

\def\intforcexshift{0}
\def\intforceyshift{\internalforcefieldypos}

\newcommand\intforcebeginsatx[1]{{ \beamfx{#1} + \intforcexshift }}
\newcommand\intforcebeginsaty[1]{{ \beamfy{#1} + \intforceyshift }}

\newcommand\intforceendsatx[1]{{ \beamfx{#1} + \intforcexshift + \internalforcescale * ( \internalforcefx{#1} ) }}
\newcommand\intforceendsaty[1]{{ \beamfy{#1} + \intforceyshift + \internalforcescale * ( \internalforcefy{#1} ) }}

\newcommand\intQloadtext{\mathboldQ(\hspace{-0.2ex} \beamcoordinate \hspace{-0.12ex})}

\newcommand\drawinternalforcefield{%
\draw [ beam line ]
	plot [ variable=\s, domain=0:180, samples=200 ]
	( {\beamfx{\s} + \intforcexshift}, {\beamfy{\s} + \intforceyshift} ) ;

\pgfmathsetmacro\Qstep{180 / ( \howmanyinternalforcelines - 1 )}
\pgfmathsetmacro\Qfrom{0}
\pgfmathsetmacro\Qnext{\Qfrom + \Qstep}
\pgfmathsetmacro\Qlast{180}

\foreach \sQ in {\Qfrom, \Qnext, ..., \Qlast}
	\draw [ continuous load vector, color=\internalforcecolor ]
		( \intforcebeginsatx{\sQ}, \intforcebeginsaty{\sQ} ) --
		( \intforceendsatx{\sQ}, \intforceendsaty{\sQ} ) ;

\def\textcoordinate{123}

\draw [ internal force level ]
	plot [ variable=\sQ, domain=0:180, samples=100 ] ( \intforceendsatx{\sQ}, \intforceendsaty{\sQ} ) ;
\node [ above right, color=\internalforcecolor, inner sep=0pt, outer sep=9pt ]
	at ( \intforceendsatx{\textcoordinate}, \intforceendsaty{\textcoordinate} )
	{\scalebox{\textscale}{$ \intQloadtext $}} ;

% components of Q
\node [ right, color=\internalforcecolor, inner sep=0pt, outer sep=0pt ]
	at ( \intforceendsatx{\textcoordinate}, \internalforcefieldypos + .66 )
	{\scalebox{\internalforcetextscale}{$ \begin{array}{l}
\begin{aligned}
Q_{\hspace{-0.2ex}\raisebox{-0.1em}{$\scriptstyle \eta$}}\hspace{-0.1ex}(\hspace{-0.2ex} \beamcoordinate \hspace{-0.12ex})
\hspace{-0.25ex} &= \hspace{-0.1ex}
\internalforcextext
\end{aligned}
\\[.4em]
\begin{aligned}
Q_{\hspace{-0.2ex}\raisebox{-0.1em}{$\scriptstyle \zeta$}}\hspace{-0.1ex}(\hspace{-0.2ex} \beamcoordinate \hspace{-0.12ex})
\hspace{-0.25ex} &= \hspace{-0.1ex}
\internalforceytext
\end{aligned}
	\end{array} $}} ;
}

\drawinternalforcefield

\begin{scope}[xshift = \secondbeamxshift]

\renewcommand\verticalreactionequalsto{q}

\renewcommand\internalforcefx[1]{(\beamlengthb / \beamlengtha) * deg(1) * \justq * sin(#1)}
\renewcommand\internalforcefy[1]{- deg(1) * \justq * cos(#1)}

\renewcommand\internalforcextext{\hspace{-0.15ex} \raisebox{-0.1em}{\scalebox{1.4}{$ \textstyle\frac{\scalebox{.7}{$b$}}{\scalebox{.7}{\raisebox{.15em}{$a$}}} $}} \hspace{.1ex} q \sin \beamcoordinate}

\renewcommand\internalforceytext{q \hspace{.1ex} (- \hspace{-0.2ex} \cos \beamcoordinate \hspace{-0.15ex} + \hspace{-0.15ex} 1) \hspace{-0.15ex} \mathcolor{\initialinternalforcecolor}{- \verticalreactionequalsto}\\[-0.3em]
&= \hspace{-0.1ex} - \hspace{.2ex} q \cos \beamcoordinate}

\drawinternalforcefield

\end{scope}

%
% tangent vectors
%

\def\tangentvectorsypos{\internalforcefieldypos - 9}

\pgfmathsetmacro\tangentvectortextscale{.84*\textscale}

\def\tangentvectorcolor{black!66}

\newcommand\tangentfx[1]{- \beamlengtha * sin(#1)}
\newcommand\tangentfy[1]{\beamlengthb * cos(#1)}

\newcommand\squaredlengthoftangent[1]{( (\tangentfx{#1})^2 + (\tangentfy{#1})^2 )}

\def\howmanytangentvectors{21}

\pgfmathsetmacro\tangentvectorscale{1}

\pgfmathsetmacro\tangentvectorsxshift{0}
\pgfmathsetmacro\tangentvectorsyshift{\tangentvectorsypos}

\newcommand\tangentvectorbeginsatx[1]{{ \beamfx{#1} + \tangentvectorsxshift }}
\newcommand\tangentvectorbeginsaty[1]{{ \beamfy{#1} + \tangentvectorsyshift }}

\newcommand\tangentvectorendsatx[1]{{ \tangentvectorbeginsatx{#1} + \tangentvectorscale * ( \tangentfx{#1} ) }}
\newcommand\tangentvectorendsaty[1]{{ \tangentvectorbeginsaty{#1} + \tangentvectorscale * ( \tangentfy{#1} ) }}

\newcommand\unittangentvectorendsatx[1]{{ \tangentvectorbeginsatx{#1} + \tangentvectorscale * ( \tangentfx{#1} ) / sqrt(\squaredlengthoftangent{#1}) }}
\newcommand\unittangentvectorendsaty[1]{{ \tangentvectorbeginsaty{#1} + \tangentvectorscale * ( \tangentfy{#1} ) / sqrt(\squaredlengthoftangent{#1}) }}

\newcommand\drawtangentvectorfield{%
\draw [ beam line ]
	plot [ variable=\s, domain=0:180, samples=200 ]
	( {\beamfx{\s} + \tangentvectorsxshift}, {\beamfy{\s} + \tangentvectorsyshift} ) ;

\pgfmathsetmacro\tanstep{180 / ( \howmanytangentvectors - 1 )}
\pgfmathsetmacro\tanfrom{0}
\pgfmathsetmacro\tannext{\tanfrom + \tanstep}
\pgfmathsetmacro\tanlast{180}

\foreach \stan in {\tanfrom, \tannext, ..., \tanlast} {
	\draw [ location vector ]
		( \tangentvectorsxshift, \tangentvectorsyshift ) --
			++( {\beamfx{\stan}}, {\beamfy{\stan}} ) ;

	\draw [ location vector, color=\tangentvectorcolor ]
		( \tangentvectorbeginsatx{\stan}, \tangentvectorbeginsaty{\stan} ) --
			( \tangentvectorendsatx{\stan}, \tangentvectorendsaty{\stan} ) ;

	%%\draw [ location vector, color=\tangentvectorcolor ]
		%%( \tangentvectorbeginsatx{\stan}, \tangentvectorbeginsaty{\stan} ) --
		%%( \unittangentvectorendsatx{\stan}, \unittangentvectorendsaty{\stan} ) ;

	\draw [ point ] ( {\beamfx{\stan} + \tangentvectorsxshift}, {\beamfy{\stan} + \tangentvectorsyshift} ) circle ( \pointsize\pgflinewidth ) ;
}

\draw [ aux line, fill=white ] ( \tangentvectorsxshift, \tangentvectorsyshift ) circle ( \initialpositioncircleradius ) ;

% components of tangent vector
\node [ below, color=\tangentvectorcolor, inner sep=0pt, outer sep=0pt ]
	at ( \tangentvectorsxshift + .5, \tangentvectorsyshift - .8 )
	{\scalebox{\tangentvectortextscale}{$ \begin{aligned}
\bm{t} \hspace{-0.1ex}
&\mathcolor{black}{=}
\hspace{-0.1ex} \mathcolor{black}{\scalebox{.82}{$\displaystyle\frac{\raisebox{-0.22em}{$ \partial \hspace{.1ex} \bm{r} $}}{\partial \hspace{-0.1ex} \beamcoordinate}$}}
\\
t_{\raisebox{-0.1em}{$\scriptstyle \eta$}} \hspace{-0.3ex}
&=
\hspace{-0.1ex} \scalebox{.7}{$\displaystyle\frac{\raisebox{-0.22em}{$ \partial $}}{\partial \hspace{-0.1ex} \beamcoordinate}$} \hspace{.1ex}
r_{\hspace{-0.33ex}\raisebox{-0.1em}{$\scriptstyle \eta$}} \hspace{-0.3ex}
=
- \hspace{.1ex} a \sin \beamcoordinate
\\[-0.1em]
t_{\hspace{-0.1ex} \raisebox{-0.1em}{$\scriptstyle \zeta$}} \hspace{-0.3ex}
&=
\hspace{-0.1ex} \scalebox{.7}{$\displaystyle\frac{\raisebox{-0.22em}{$ \partial $}}{\partial \hspace{-0.1ex} \beamcoordinate}$} \hspace{.1ex}
r_{\hspace{-0.33ex}\raisebox{-0.1em}{$\scriptstyle \zeta$}} \hspace{-0.3ex}
=
b \cos \beamcoordinate
	\end{aligned} $}} ;
}

\begin{scope}[ xshift = .5*\secondbeamxshift ]
\drawtangentvectorfield
\end{scope}

%
% epure of projection of the Q on tangent
%

\def\Qtangentepureypos{\tangentvectorsypos - 8.8}

\newcommand\Qdottangent[1]{( ((\internalforcefx{#1}) * (\tangentfx{#1})) + ((\internalforcefy{#1}) * (\tangentfy{#1})) )}

\newcommand\Qtangentfx[1]{(\tangentfx{#1}) * \Qdottangent{#1} / \squaredlengthoftangent{#1}}
\newcommand\Qtangentfy[1]{(\tangentfy{#1}) * \Qdottangent{#1} / \squaredlengthoftangent{#1}}

\newcommand\drawQtangentepure{%
\draw [ beam line ]
	plot [ variable=\s, domain=0:180, samples=200 ]
	( {\beamfx{\s}}, {\beamfy{\s} + \Qtangentepureypos} ) ;

% relation how to get the projection of Q on tangent
\node [ color=\internalforcecolor, inner sep=0pt, outer sep=0pt ]
	at ( 0, \Qtangentepureypos )
	{\scalebox{\textscale}{$
%%(\hspace{-0.2ex} \beamcoordinate \hspace{-0.12ex})
\mathboldQ_{\scalebox{.8}{$t$}} \hspace{-0.3ex}
\mathcolor{black}{=}
\hspace{-0.1ex} \mathboldQ \mathcolor{black}{\dotp} \hspace{-0.2ex} \mathcolor{black}{\scalebox{.9}{$ \displaystyle\frac{\raisebox{-0.2em}{$ \mathcolor{\tangentvectorcolor}{\bm{t}} $}}{\| \mathcolor{\tangentvectorcolor}{\bm{t}} \|} $}}
\mathcolor{black}{\scalebox{.9}{$ \displaystyle\frac{\raisebox{-0.2em}{$ \mathcolor{\tangentvectorcolor}{\bm{t}} $}}{\| \mathcolor{\tangentvectorcolor}{\bm{t}} \|} $}} \hspace{-0.2ex}
\mathcolor{black}{=}
\hspace{-0.2ex} \mathcolor{black}{\scalebox{.93}{$ \displaystyle\frac{\raisebox{-0.2em}{$ \mathcolor{\internalforcecolor}{\mathboldQ} \dotp \mathcolor{\tangentvectorcolor}{\bm{t} \hspace{.1ex} \bm{t}} $}}{\mathcolor{\tangentvectorcolor}{\bm{t}} \dotp \mathcolor{\tangentvectorcolor}{\bm{t}}} $}}
	$}} ;

\pgfmathsetmacro\Qtangentstep{180 / ( \howmanyinternalforcelines - 1 )}
\pgfmathsetmacro\Qtangentfrom{0}
\pgfmathsetmacro\Qtangentnext{\Qtangentfrom + \Qtangentstep}
\pgfmathsetmacro\Qtangentlast{180}

\foreach \sQ in {\Qtangentfrom, \Qtangentnext, ..., \Qtangentlast}
	\draw [ continuous load vector, color=\internalforcecolor ]
		( {\beamfx{\sQ}}, {\beamfy{\sQ} + \Qtangentepureypos} ) --
			++( {\internalforcescale * \Qtangentfx{\sQ}}, {\internalforcescale * \Qtangentfy{\sQ}} ) ;

\draw [ internal force level ]
	plot [ variable=\sQ, domain=0:180, samples=100 ]
	( {\beamfx{\sQ} + (\internalforcescale * \Qtangentfx{\sQ})},
		{\beamfy{\sQ} + (\internalforcescale * \Qtangentfy{\sQ}) + \Qtangentepureypos} ) ;
}

\renewcommand\internalforcefx[1]{0}
\renewcommand\internalforcefy[1]{( #1 - 90 ) * \justq}

\drawQtangentepure

\begin{scope}[xshift = \secondbeamxshift]
\renewcommand\internalforcefx[1]{(\beamlengthb / \beamlengtha) * deg(1) * \justq * sin(#1)}
\renewcommand\internalforcefy[1]{- deg(1) * \justq * cos(#1)}

\drawQtangentepure
\end{scope}

%
% balance of moments
%

\def\balanceofmomentsypos{\Qtangentepureypos - 6}

\pgfmathsetmacro\internalmomenttextscale{.84*\textscale}

\def\internalmomentcolor{magenta}
\def\initialinternalmomentcolor{magenta!40!blue}

\newcommand\drawbalanceofmomentsequation{%
\node [ color=\internalmomentcolor, inner sep=0pt, outer sep=0pt ]
	at ( 0,  \balanceofmomentsypos )
	{\scalebox{\internalmomenttextscale}{$
\mathcolor{black}{\scalebox{.7}{$ \displaystyle\integral_{\scalebox{.77}{$\hspace{.2em}0$}}^{\scalebox{.96}{$\hspace{.2em} \beamcoordinate$}} $}}
\hspace{-0.3ex} \mathcolor{black}{\big(} \mathcolor{black}{\bm{r}} \hspace{-0.25ex} \mathcolor{black}{\times} \hspace{-0.25ex} \mathcolor{\extloadcolor}{\bm{q}} \hspace{.05ex} \mathcolor{black}{\big)} \hspace{-0.1ex} \mathcolor{black}{d\beamcoordinate}
%
\mathcolor{black}{+} \hspace{.1ex}
\mathcolor{black}{\bm{r}\hspace{-0.1ex}(\hspace{-0.2ex} \beamcoordinate \hspace{-0.12ex})} \hspace{-0.4ex} \mathcolor{black}{\times} \hspace{-0.3ex} \mathcolor{\internalforcecolor}{\mathboldQ(\hspace{-0.2ex} \beamcoordinate \hspace{-0.12ex})} \hspace{-0.15ex}
\mathcolor{black}{-} \hspace{.1ex}
\mathcolor{black}{\bm{r}\hspace{-0.1ex}(\hspace{-0.1ex}0\hspace{-0.1ex})} \hspace{-0.4ex} \mathcolor{black}{\times} \hspace{-0.3ex} \mathcolor{\initialinternalforcecolor}{\mathboldQ(\hspace{-0.1ex}0\hspace{-0.1ex})} \hspace{-0.15ex}
%
\mathcolor{black}{+}
\hspace{-0.1ex}\mathboldM(\hspace{-0.2ex} \beamcoordinate \hspace{-0.12ex}) \hspace{-0.15ex}
\mathcolor{black}{-}
\textcolor{\initialinternalmomentcolor}{\hspace{-0.1ex}\mathboldM(\hspace{-0.1ex}0\hspace{-0.1ex})} \hspace{-0.15ex}
\mathcolor{black}{=}
\mathcolor{black}{\bm{0}}
$}} ;
}

\begin{scope}[ xshift = .5*\secondbeamxshift ]
\drawbalanceofmomentsequation
\end{scope}

\end{tikzpicture}

\end{document}
