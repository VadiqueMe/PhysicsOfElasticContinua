\en{\chapter{Rods}}

\ru{\chapter{Стержни}}

\thispagestyle{empty}

\newcommand\internalforce{\mathboldQ} %% \bm{Q}
\newcommand\internalmoment{\mathboldM}

\newcommand\tangenttoaxis{\bm{k}}
\newcommand\tangenttoaxistoo{\bm{e}_3}

\newcommand\rodsection{\Upomega} %% {F}
\newcommand\rodsectionexpminusone{\rodsection^{\hspace{-0.2ex}\expminusone}}

\label{chapter:rods}

\en{\section{Initial concepts}}

\ru{\section{Исходные представления}}

\label{para:overviewofrods}

\en{\dropcap{R}{od}}\ru{\dropcap{С}{тержень}}\ru{\:--- это}\en{ is} \en{a~thin long body}\ru{тонкое длинное тело}.
\en{It is thought of}\ru{Он мыслится} (\en{and modeled}\ru{и~моделируется}) \en{as}\ru{как} \en{a~spatial curve}\ru{пространственная кривая}\:--- \en{the~axis of~rod}\ru{ось стержня}, \en{coated with a~material}\ru{покрытая материалом}
( \textcolor{blue}{\en{a figure}\ru{рисунок}} ).

\en{The~axis of rod}\ru{Ось стержня} \en{is described}\ru{описывается} \en{by parameterizing}\ru{параметризацией} \en{the~location vector of curve’s points}\ru{вектора положения точек кривой} \en{as a~morphism}\ru{как морфизма}~(\en{function}\ru{функции}) \en{of one}\ru{одной} \en{variable}\ru{переменной} \en{coordinate}\ru{координаты}~$s$,

\nopagebreak\en{\vspace{-0.2em}}\ru{\vspace{-1.1em}}\begin{equation}
\locationvector \hspace{-0.4ex} = \hspace{-0.3ex} \locationvector(s)
\hspace{.1ex} .
\end{equation}

\vspace{-0.2em}\noindent
\en{Material coating}\ru{Покрытие материалом} \en{gives}\ru{даёт} \en{at each rod’s point}\ru{в~каждой точке стержня} \en{a~plane figure}\ru{плоскую фигуру}, \en{perpendicular to the~axis}\ru{перпендикулярную оси}\:--- \en{normal section}\ru{нормальное сечение}~${\rodsection(s)}$.

...

\begin{otherlanguage}{russian}

Добавив \textcolor{magenta}{единичный??} вектор касательной к~оси

\nopagebreak\vspace{-0.2em}\begin{equation*}
\locationvector{\hspace{.1ex}'} \hspace{-0.3ex}
\equiv \scalebox{.8}{$ \displaystyle\frac{\raisemath{-0.15em}{\partial \hspace{.15ex} \locationvector}}{\partial s} $}
\equiv \locationvector_\differentialindex{s} \hspace{-0.2ex}
\equiv \bm{e}_3 \hspace{-0.2ex}
\equiv \bm{k}
\hspace{.15ex} ,
\end{equation*}

\vspace{-0.1em}\noindent
получим (для каждого~$s$) тройку взаимно перпендикулярных единичных векторов, характеризующих угловую ориентацию.

\en{Curvature}\ru{Кривизна} \en{and}\ru{и}~\en{torsion}\ru{кручение} \en{of~the~rod’s axis}\ru{оси стержня} определяются вектором~$\bm{\psi}$

\nopagebreak\vspace{-0.2em}\begin{equation}
\bm{e}{\hspace{.1ex}'}_{\hspace{-0.7ex}j} \hspace{-0.2ex}
= \bm{\psi} \hspace{-0.1ex} \times \hspace{-0.1ex} \bm{e}_{j}
\hspace{.1ex} , \hspace{.5em}
\bm{\psi} \hspace{-0.1ex} = \smalldisplaystyleonehalf \hspace{.25ex} \bm{e}_{j} \hspace{-0.2ex} \times \hspace{-0.1ex} \bm{e}{\hspace{.1ex}'}_{\hspace{-0.7ex}j}
\hspace{.1ex} .
\end{equation}

...

\newpage

\en{Moreover}\ru{Кроме этого}, \en{at each point}\ru{в~каждой точке} \en{of the rod’s axis}\ru{оси стержня} \en{as a~curve line}\ru{как кривой линии} \en{there’s}\ru{есть и} \en{another}\ru{другая} \en{triple of mutually perpendicular unit vectors}\ru{тройка взаимно перпендикулярных единичных векторов}\:--- \en{with normal}\ru{с~нормалью} \en{and~binormal}\ru{и~бинормалью}.

\en{Tangent}\ru{Касательная}~$\bm{T}$, \en{normal}\ru{нормаль}~$\bm{N}$ \en{and}\ru{и} \en{binormal}\ru{бинормаль}~$\bm{B}$, \en{together called}\ru{вместе называемые} \en{the }\emph{\ru{системой }Frenet--Serret\en{ frame}}, are defined as follows:
\begin{itemize}
\item $\bm{T}$ is the unit vector tangent to the curve, pointing where the~curve continues further.
\item $\bm{N}$ is the normal unit vector, the derivative of~$\bm{T}$ by the curve’s arc length parameter, divided by its length ${ \kappa \hspace{-0.1ex} = \hspace{-0.25ex} \| \bm{N} \| }$.
\item $\bm{B}$ is the binormal unit vector, the cross product of~$\bm{T}$ and~$\bm{N}$, ${\bm{B} \hspace{-0.1ex} \equiv \bm{T} \hspace{-0.15ex} \times \hspace{-0.3ex} \bm{N}}$.
\end{itemize}

The Frenet--Serret formulas, describing derivatives of tangent, normal and binormal unit vectors in terms of each other, are

\begin{align*}
\scalebox{.88}{$ \displaystyle\frac{\raisemath{-0.2em}{ d\hspace{.1ex}\bm{T} }}{ds} $} &= \kappa \hspace{.1ex} \bm{N} ,
\\
\scalebox{.88}{$ \displaystyle\frac{\raisemath{-0.2em}{ d\bm{N} }}{ds} $} &= {}- \hspace{-0.3ex} \kappa \hspace{.25ex} \bm{T} + \tau \hspace{-0.1ex} \bm{B} ,
\\
\scalebox{.88}{$ \displaystyle\frac{\raisemath{-0.2em}{ d\bm{B} }}{ds} $} &= {}- \hspace{-0.3ex} \tau \hspace{-0.1ex} \bm{N} ,
\end{align*}

\vspace{-0.2em}\noindent
where ${\scalebox{.93}{$ \raisemath{.3em}{d} \hspace{-0.25ex} / \hspace{-0.33ex} \raisemath{-0.25em}{ds} $}}$ denotes the~derivative by arc length, $\kappa$ is the curvature and~$\tau$ is the curve’s torsion.
The associated collection\:--- ${\bm{T}\hspace{-0.2ex}}$, ${\bm{N}\hspace{-0.2ex}}$, ${\bm{B}\hspace{-0.1ex}}$, $\kappa$, $\tau$\:--- is called the \emph{Frenet--Serret apparatus}.

Two scalars $\kappa$ and~$\tau$ effectively define the~curvature and torsion of a~space curve.
Intuitively, curvature measures the deviation of a~curve from a straight line, while torsion measures the deviation of a~curve from being planar.

\[
\displaystyle \kappa = \left\| \scalebox{.88}{$ \displaystyle\frac{ d\hspace{.1ex}\bm{T} }{ds} $} \right\|
\]

\[
\kappa \ge 0
\]

\textcolor{magenta}{(nonzero curvature ${\kappa \hspace{-0.2ex} \neq \hspace{-0.1ex} 0}$ that is ${\kappa > 0}$)}

the Frenet--Serret frame is not defined if ${\kappa \hspace{-0.2ex} = \hspace{-0.1ex} 0}$

zero curvature implies the curve is a straight line, which lies in a~plane, making the torsion zero too

If the curvature is not zero, it is considered as a curve line.

...

$\bm{T}$ always has unit magnitude, and since there’s no change in length of~$\bm{T}$, then $\bm{N}$\:--- the derivative of~$\bm{T}$\:--- is always perpendicular to $\bm{T}$
(${\bm{T} \hspace{-0.15ex} \dotp \hspace{.1ex} \bm{T} = 1
\hspace{.22em} \Rightarrow \hspace{.22em}
\bm{T}{\hspace{.25ex}'} \hspace{-0.3ex} \dotp \hspace{.1ex} \bm{T} = 0
\hspace{.22em} \Rightarrow \hspace{.16em}
\bm{N} \hspace{-0.3ex} \dotp \hspace{.1ex} \bm{T} = 0}$).

Vectors of the Frenet--Serret frame make an~orthonormal basis ${\bm{f}_{\hspace{-0.1ex}i}\hspace{.2ex}}$:
${\bm{f}_{\hspace{-0.1ex}1} \hspace{-.2ex} = \bm{T}\hspace{-0.2ex}}$,
${\bm{f}_{2} \hspace{-.2ex} = \hspace{-.2ex} \bm{N}\hspace{-0.2ex}}$,
${\bm{f}_{3} \hspace{-.2ex} = \hspace{-.2ex} \bm{B}\hspace{-0.1ex}}$.

\en{The tensor version}\ru{Тензорная версия} \en{of~}\ru{формул }Frenet--Serret\en{ formulas}

\nopagebreak\vspace{-0.1em}\begin{equation}\label{frenetserretformulas.tensorversion}
\bm{f}_{\hspace{-0.1ex}i}{'} \hspace{-0.25ex} = \hspace{-0.1ex} {^2\hspace{-0.2ex}\bm{d}} \hspace{.1ex} \dotp \hspace{-0.15ex} \bm{f}_{\hspace{-0.1ex}i}
\hspace{.2ex} .
\end{equation}

Frenet--Serret formulas using matrix notation

\nopagebreak\begin{equation*}
\scalebox{.88}{$ \left[ \hspace{.1ex}
\begin{array}{c}
\bm{T}{\hspace{.25ex}'}\\
\bm{N}{\hspace{.1ex}'}\\
\bm{B}{\hspace{.15ex}'}
\end{array} \hspace{.1ex} \right] $}
\hspace{-0.4ex} = \hspace{-0.4ex}
\scalebox{.88}{$ \left[ \begin{array}{ccc}
0 & \kappa & 0 \\
-\hspace{.2ex} \kappa & 0 & \hspace{.3ex} \tau \hspace{.3ex} \\
0 & -\hspace{.2ex} \tau & 0
\end{array} \hspace{.4ex} \right] $}
\hspace{-0.4ex}
\scalebox{.88}{$ \left[ \hspace{.1ex}
\begin{array}{c}
\bm{T} \\
\bm{N} \\
\bm{B}
\end{array} \hspace{.1ex} \right] $}
\hspace{-0.1ex} .
\end{equation*}

Tensor~${^2\hspace{-0.2ex}\bm{d}}$ is skew-symmetric, therefore it can be represented by the~companion (pseudo)vector (\chapdotpararef{chapter:elementsoftensorcalculus}{para:tensors.symmetric+skewsymmetric}).
This pseudovector is known as the Darboux vector.

\begin{align*}
\bm{D} &= \tau \hspace{.2ex} \bm{T} + \kappa \hspace{.1ex} \bm{B}
\\
\bm{D} &= \tau \hspace{.2ex} \bm{T} + 0 \hspace{.1ex} \bm{N} + \kappa \hspace{.1ex} \bm{B}
\end{align*}

With the Darboux vector, Frenet--Serret formulas turn into

\begin{align*}
\bm{T}{\hspace{.25ex}'} \hspace{-0.2ex} &= \hspace{-0.15ex} \bm{D} \hspace{-0.2ex} \times \bm{T} ,
\\[-0.2em]
\bm{N}{\hspace{.1ex}'} \hspace{-0.2ex} &= \hspace{-0.15ex} \bm{D} \hspace{-0.2ex} \times \hspace{-0.2ex} \bm{N} ,
\\[-0.2em]
\bm{B}{\hspace{.15ex}'} \hspace{-0.2ex} &= \hspace{-0.15ex} \bm{D} \hspace{-0.2ex} \times \hspace{-0.15ex} \bm{B}
%\hspace{.1ex} .
\end{align*}

\noindent
\en{or}\ru{или} \en{as}\ru{как} \en{the vector version}\ru{векторная версия} \en{of~}\eqref{frenetserretformulas.tensorversion}

\nopagebreak\vspace{-0.1em}\begin{equation}\label{frenetserretformulas.vectorversion}
\bm{f}_{\hspace{-0.1ex}i}{'} \hspace{-0.25ex} = \hspace{-0.15ex} \bm{D} \hspace{-0.2ex} \times \hspace{-0.15ex} \bm{f}_{\hspace{-0.1ex}i}
\hspace{.2ex} .
\end{equation}

The Darboux vector is the angular velocity vector of the Frenet--Serret frame of a~curve.
It is also called as angular momentum vector, because it is directly proportional to angular momentum \textcolor{magenta}{(really??)}.

...

\newpage

\en{In approximate}\ru{В~приближённых} (\inquotes{\en{applied}\ru{прикладн\'{ы}х}}) \en{theories of~rods}\ru{теориях стержней} \en{like}\ru{вроде} \inquotes{\en{strength of~materials}\ru{сопротивления материалов}} \en{figure}\ru{фигурируют} \en{internal}\ru{внутренние} \en{force}\ru{сила}~${\hspace{-0.1ex}\internalforce\hspace{.1ex}}$ \en{and}\ru{и}~\en{moment}\ru{момент}~${\hspace{-0.1ex}\internalmoment}$.
\en{They are connected}\ru{Они связаны} \en{with the stress tensor}\ru{с~тензором напряжения} \en{via relations}\ru{соотношениями}

\nopagebreak\vspace{.1em}\begin{align}
\internalforce(s) \hspace{-0.2ex}
&= \hspace{-0.3ex} \scalebox{.88}{$ \displaystyle\integral_{\rodsection} $} \hspace{-0.25ex} \tractionvector{\tangenttoaxis} \hspace{.3ex} d\rodsection \hspace{-0.1ex}
= \hspace{-0.3ex} \scalebox{.88}{$ \displaystyle\integral_{\rodsection} $} \hspace{-0.25ex} \tangenttoaxis \dotp \hspace{-0.1ex} \cauchystress \hspace{.3ex} d\rodsection\hspace{.2ex} ,
\\
\internalmoment(s) \hspace{-0.2ex}
&= \hspace{-0.3ex} \scalebox{.88}{$ \displaystyle\integral_{\rodsection} $} \hspace{-0.25ex} \bm{x} \hspace{-0.2ex} \times \hspace{-0.1ex} \tractionvector{\tangenttoaxis} \hspace{.3ex} d\rodsection \hspace{-0.1ex}
= \hspace{-0.3ex} \scalebox{.88}{$ \displaystyle\integral_{\rodsection} $} \hspace{-0.25ex} \bm{x} \hspace{-0.2ex} \times \hspace{-0.1ex} \tangenttoaxis \dotp \hspace{-0.1ex} \cauchystress \hspace{.3ex} d\rodsection
\hspace{.2ex} .
\end{align}

\[
\tangenttoaxis \equiv \tangenttoaxistoo
\hspace{.1ex} , \hspace{.7em}
\bm{x} = x_{\alpha} \bm{e}_\alpha
\hspace{.1ex} , \hspace{.4em}
\alpha \hspace{-0.1ex} = \hspace{-0.12ex} 1, 2
\]

...

Такие рассуждения о~геометрии и~о~механике (внутренних силе и~моменте) относятся только лишь к~какой-то одной конфигурации стержня.
Продолжать эти соображения \en{is meaningless}\ru{бессмысленно} потому, что плоские нормальные сечения после деформирования не~остаются плоскими и~нормальными.
Добавление~же в~модель \en{an assumption-hypothesis}\ru{предположения-гипотезы} о~том, что депланации нет \en{and}\ru{и}~\inquotes{\en{plane sections remain plane}\ru{плоские сечения остаются плоскими}}\footnote{%
\en{Two very popular beam models exist}\ru{Существуют две очень популярные модели балки} \en{with }\ru{с~}\ru{гипотезой}\en{the hypothesis}\ru{,} \en{postulating the~absence of~deplanations}\ru{постулирующей отсутствие депланаций}.
In the Euler\hbox{--}Bernoulli beam theory, shear deformations are neglected, and plane sections remain plane and normal to the~axis.
In the Timoshenko beam theory, there’s a~constant \en{transverse shear}\ru{поперечный сдвиг} \en{along the section}\ru{вдоль сечения}, and plane sections still remain plane but are no longer normal to the~axis.}\hspace{-0.3ex}
вносит существенные противоречия с~реальностью.
\en{Enough to recall}\ru{Достаточно вспомнить} \en{just one fact that}\ru{лишь один факт, что} \en{without}\ru{без} \en{deplanation}\ru{депланации} \en{it’s impossible}\ru{невозможно} \en{to acceptably describe}\ru{приемлемо описать} \en{the torsion of a~rod}\ru{кручение стержня} (\en{and not only torsion}\ru{и~не~только кручение}).

\en{Very reasonable}\ru{Очень резонный} \en{approach}\ru{подход} \en{to}\ru{к}~\en{modeling}\ru{моделированию} \en{deformations}\ru{деформаций} \en{of an~elastic rod}\ru{упругого стержня} \en{consists in}\ru{состоит в} \en{the asymptotic splitting}\ru{асимптотическом расщеплении} \en{of the three-dimensional problem}\ru{трёхмерной проблемы} \en{with }\ru{с~}\en{a~small thickness}\ru{м\'{а}лой толщиной}.
\en{But}\ru{Но} \en{for}\ru{для} \en{a~complex asymptotic procedure}\ru{сложной асимптотической процедуры} \en{it would be much simpler}\ru{было бы намного проще} \en{to have}\ru{иметь} \en{whatever}\ru{какую-нибудь} \en{solution version}\ru{версию решения} \en{aforehand}\ru{заранее}.
\en{And}\ru{И} \en{the direct approach}\ru{прямой подход}, \en{when}\ru{когда} \en{the one\hbox{-}dimensional model}\ru{одномерная модель} \en{of a~rod}\ru{стержня}\en{ is}\ru{\:---} \en{a~material line}\ru{материальная линия}, \en{gives}\ru{даёт} \en{such a~version}\ru{такую версию}.

\en{The primary question}\ru{Первичный вопрос} \en{for building}\ru{для построения} \en{the one\hbox{-}dimensional model}\ru{одномерной модели}:
\en{what}\ru{какими} \en{degrees of~freedom}\ru{степенями свободы}\:--- \en{besides translation}\ru{помимо трансляции}\:--- \en{do}\ru{обладают} \en{particles}\ru{частицы} \en{of a~material line}\ru{материальной линии}\en{ possess}?

Известно, что стержни чувствительны к~моментным нагрузкам.
А~присутствие моментов среди обобщённых сил говорит о~наличии вращательных степеней свободы.
Следовательно, одномерной моделью стержня должна быть линия Коссера\:--- она состоит из бесконечномалых \textcolor{magenta}{твёрдых тел}.
Впрочем, могут проявиться и~дополнительные степени свободы\:--- как в~тонкостенных стержнях, которым посвящена отдельная глава.

В~механике упругих тел стержни занимают особенное место.
Во\hbox{-}первых, это моментные модели, и~моменты здесь играют главную роль~(не~роль малых добавок, как в~трёхмерном контину\kern-0.11exуме Cosserat).
Во\hbox{-}вторых, стержни являются как~бы \inquotes{тестовой площ\'{а}дкой} для~моделей с~дополнительными степенями свободы, поскольку наличие этих степеней можно исследовать на~трёх\-мерной модели.

Ну~а~пока сосредоточимся на~простой одномерной моментной модели типа Cosserat.

\end{otherlanguage}

\en{\section{Kinematics of Cosserat lines}} % Cosserat curves

\ru{\section{Кинематика линий Коссера}} % кривых Коссера

\begin{otherlanguage}{russian}

Рассматриваемое далее является упрощённым вариантом~\chapref{chapter:cosseratcontinuum}.
Вместо тройки материальных координат ${q^{\hspace{.1ex}i}}$ имеем одну\:--- $s$, \en{it may be}\ru{это может быть} \en{the arc length parameter}\ru{параметр длины дуги} \en{in the initial configuration}\ru{в~начальной конфигурации}.
\en{The motion}\ru{Движение} \en{of a~particle}\ru{частицы} \en{over time}\ru{со~временем} \en{is described by}\ru{описывается} \en{the location vector}\ru{вектором положения}~${\locationvector(s,t)}$ \en{and}\ru{и} \en{the rotation tensor}\ru{тензором поворота}~${\rotationtensor(s,t)}$.
\en{Linear}\ru{Линейная} \en{and}\ru{и}~\en{angular}\ru{угловая} \eqrefwithchapdotpara{angularvelocityvector}{chapter:elementsoftensorcalculus}{para:rotationtensor} \en{velocities}\ru{скорости} \en{of a~rod’s particle}\ru{частицы стержня} \en{are}\ru{суть}

\nopagebreak\vspace{-0.2em}\begin{gather}
\mathdotabove{\locationvector} = \bm{v}
\hspace{.1ex} , %\hspace{.5em}
\label{cosseratrod.linearvelocity}
\\[-0.2em]
%
\mathdotabove{\rotationtensor} \hspace{-0.1ex} = \bm{\omega} \times \hspace{-0.1ex} \rotationtensor
\hspace{.3em} \Leftrightarrow \hspace{.3em}
\bm{\omega} \hspace{-0.1ex} = \hspace{-0.1ex} - \hspace{.25ex} \smalldisplaystyleonehalf \scalebox{.96}{$ \bigl( \mathdotabove{\rotationtensor} \hspace{-0.1ex} \dotp \rotationtensor^{\T} \hspace{.1ex} \bigr)_{\hspace{-0.3ex}\Xcompanion} $}
\hspace{.1ex} .
\label{cosseratrod.angularvelocity}
\end{gather}

Деформация стержня как линии Cosserat определяется двумя векторами

\nopagebreak\vspace{-0.2em}\begin{gather}
\bm{\Gamma} \hspace{-0.1ex} = \currentlocationvector{\hspace{.1ex}'} \hspace{-0.3ex} - \rotationtensor \hspace{-0.1ex} \dotp \hspace{.15ex} \initiallocationvector{\hspace{.1ex}'}
\hspace{-0.25ex} ,
\label{cosseratrod.firstdeformationvector}
\\[.1em]
%
\bm{\kappa} \hspace{-0.1ex} = \hspace{-0.1ex} - \hspace{.25ex} \smalldisplaystyleonehalf \scalebox{.96}{$ \bigl( \rotationtensor{\hspace{.15ex}'} \hspace{-0.3ex} \dotp \rotationtensor^{\T} \hspace{.1ex} \bigr)_{\hspace{-0.3ex}\Xcompanion} $}
\hspace{.25em} \Leftrightarrow \hspace{.3em}
\rotationtensor{\hspace{.15ex}'} \hspace{-0.33ex} = \bm{\kappa} \times \hspace{-0.1ex} \rotationtensor
\hspace{.1ex} .
\label{cosseratrod.seconddeformationvector}
\end{gather}

\begin{equation*}
\Bigl( \hspace{.2em}
\initiallocationvector(s) \hspace{-0.1ex}
\equiv
\currentlocationvector(s, 0)
\hspace{.2em} \Bigr)
\vspace{-0.2em}
\end{equation*}

(*) Если ${\mathcircabove{\bm{e}}_3}$ направить по касательной ${\mathcircabove{\bm{e}}_3 \hspace{-0.2ex} \equiv \initiallocationvector{\hspace{.1ex}'}}$

\nopagebreak\begin{equation*}
\initiallocationvector{\hspace{.1ex}'} \hspace{-0.3ex}
\equiv \scalebox{.8}{$ \displaystyle\frac{\raisemath{-0.15em}{\partial \hspace{.15ex} \initiallocationvector}}{\partial s} $}
\equiv \initiallocationvector_\differentialindex{s} \hspace{-0.2ex}
\equiv \mathcircabove{\bm{e}}_3
\end{equation*}

...

То, что \eqref{cosseratrod.firstdeformationvector} \en{and}\ru{и}~\eqref{cosseratrod.seconddeformationvector} действительно векторы деформации, следует из равенства их нулю на перемещениях тела как жёсткого целого.

...

\en{The~Cosserat model doesn’t have a~section as a~plane figure.}\ru{В~модели Cosserat нет сечения как плоской фигуры.}

...


\end{otherlanguage}

\en{\section{Balance of forces and moments}}

\ru{\section{Баланс сил и моментов}}

\begin{otherlanguage}{russian}

Возможные нагрузки для стержня как линии Cosserat\:--- \hbox{силы} и~моменты: на~бесконечно-малый элемент~$ds$ действуют \en{external force}\ru{внешняя сила}~${\bm{q}ds}$ \en{and}\ru{и}~\en{external moment}\ru{внешний момент}~${\bm{m}ds}$.
Внутренними взаимо\-действиями будут сила~${\internalforce(s)}$ и~момент~${\internalmoment(s)}$\:--- это воздействие от~частицы с~координатой~${s\!+\!0}$ к~частице с~${s\!-\!0}$.
\en{The~action--re\-action principle}\ru{Принцип действия--про\-ти\-во\-действия} даёт, что перемена направления~$s$ меняет знаки ${\internalforce}$ \en{and}\ru{и}~${\internalmoment}$.

...



\end{otherlanguage}

\en{\section{Principle of virtual work and consequences}}

\ru{\section{Принцип виртуальной работы и следствия}}

\noindent
\en{For a~piece of~rod}\ru{Для куск\'{а} стержня} ${s_0 \hspace{.2em} \scalebox{0.9}{$\leq$} \hspace{.18em} s \hspace{.22em} \scalebox{0.9}{$\leq$} \hspace{.2em} s_1}$ \en{formulation of the~principle is as~follows}\ru{формулировка принципа таков\'{а}}

\begin{otherlanguage}{russian}

...


\en{Conventionally}\ru{Условно}
$\bm{a}$ \en{is}\ru{есть} \en{the~tensor of~stiffness for bending and~twisting}\ru{тензор жёсткости на~изгиб и~кручение},
$\bm{b}$\en{ is}\ru{\:---} \en{the~tensor of~stiffness for (ex)tension and~shear}\ru{тензор жёсткости на~растяжение и~сдвиг},
\en{and}\ru{а}~$\bm{c}$\en{ is}\ru{\:---} \en{the~tensor of~crosslinks}\ru{тензор перекрёстных связей}.

\en{Stiffness tensors}\ru{Тензоры жёсткости} \en{rotate together with particle}\ru{поворачиваются вместе с~частицей}:

...



\end{otherlanguage}

\en{\section{Classical Kirchhoff’s model}}

\ru{\section{Классическая модель Kirchhoff’а}}

% Kirchhoff’s rod theory

\begin{otherlanguage}{russian}

До~сих~пор функции ${\locationvector(s,t)}$ и~${\rotationtensor(s,t)}$ были независимы.
В~классической теории Kirchhoff’а существует внутренняя связь

\nopagebreak\vspace{-0.1em}\begin{equation}\label{kirchhoffmodelinternalconstraint}
\bm{\Gamma} \hspace{-0.1ex} = \bm{0}
\hspace{.4em} \Leftrightarrow \hspace{.4em}
\currentlocationvector_\differentialindex{s} \hspace{-0.2ex} = \rotationtensor \hspace{-0.1ex} \dotp \hspace{.1ex} \initiallocationvector_\differentialindex{s} \hspace{-0.2ex}
\hspace{.5em} \text{\en{or}\ru{или}} \hspace{.5em}
\currentlocationvector{\hspace{.1ex}'} \hspace{-0.33ex} = \rotationtensor \hspace{-0.1ex} \dotp \hspace{.1ex} \initiallocationvector{\hspace{.1ex}'}
\hspace{-0.33ex} .
\end{equation}

\vspace{-0.2em}\noindent
Вспоминая смысл вектора~$\bm{\Gamma}$ \textcolor{magenta}{(какой? где??)}, делаем выводы:
(1)~стержень нерастяжим, (2)~поперечных сдвигов нет.
Если в~начальном состоянии единичный вектор~${\mathcircabove{\bm{e}}_3}$ был направлен по~касательной к~оси, то он останется на~ней и~после деформирования.
Частицы стержня поворачиваются лишь вместе с~касательной и~вокруг неё.

Уравнения баланса сил и~моментов (импульса и~момента импульса) не~меняются от введения связи~\eqref{kirchhoffmodelinternalconstraint}.
Но локальное вариационное соотношение (...) становится короче:

...

\end{otherlanguage}

\en{\section{Euler’s problem}}

\ru{\section{Проблема Эйлера}}

\begin{otherlanguage}{russian}

Рассматривается прямой стержень, защемлённый на одном конце и~нагруженный силой~$\bm{P}$ на~другом (рисунок ?? 123 ??).
Сила \inquotes{мёртвая}~(не~меняется в~процессе деформирования)

...



\end{otherlanguage}

\en{\section{Variational equations}} % Equations in variations

\ru{\section{Вариационные уравнения}} % Уравнения в вариациях

\begin{otherlanguage}{russian}

В~нелинейной механике упругих тел полезны уравнения в~вариациях, описывающие малое изменение актуальной конфигурации.
Как и в~\chapdotpararef{chapter:nonlinearcontinuum}{para:variationofconfiguration}, вариации величин

...



\end{otherlanguage}

\en{\section{Non-shear model with (ex)tension}}

\ru{\section{Модель без сдвига с растяжением}}

\en{Kirchhoff’s model}\ru{Модель Kirchhoff’а} \en{with internal constraint}\ru{с~внутренней связью}~${\bm{\Gamma} \hspace{-0.25ex}=\hspace{-0.2ex} \bm{0}}$~\eqref{kirchhoffmodelinternalconstraint} \en{doesn’t describe the~simplest case of extension/compression}\ru{не~описывает простейшего случая растяжения/сжатия} \en{for a~straight rod}\ru{для прямого стержня}.
\en{This nuisance disappears with \inquotes{softening} of the~constraint}\ru{Эта неприятность исчезает со~\inquotes{смягчением} связи}:
\en{adding the~possibility of (ex)tension}\ru{добавлением возможности растяжения} \en{and}\ru{и} \en{inhibiting only transverse shear}\ru{подавлением лишь поперечного сдвига}, \en{that~is}\ru{то~есть}

\nopagebreak\vspace{-0.1em}\begin{equation}\label{kirchhoffinternalconstraintwithextension}
\bm{\Gamma} \hspace{-0.1ex} = \Gamma \bm{e}_3 \hspace{-0.2ex}
\hspace{.4em} \Leftrightarrow \hspace{.4em}
\Gamma_{\hspace{-0.2ex}\alpha} \hspace{-0.2ex} = 0
\end{equation}

\begin{otherlanguage}{russian}

...



\end{otherlanguage}

\en{\section{Mechanics of flexible thread}}

\ru{\section{Механика гибкой нити}}

\begin{otherlanguage}{russian}

\en{A~flexible thread~(chain)}\ru{Гибкая нить~(цепь)} \en{is simpler than a~rod}\ru{проще стержня}:
\en{its particles}\ru{её частицы} \en{are}\ru{суть} \inquotes{\ru{простые}\en{simple}} \en{material points}\ru{материальные точки} \en{with only translational degrees of~freedom}\ru{с~лишь трансляционными степенями свободы}.
Поэтому среди нагрузок нет моментов, только \inquotes{линейные} силы\:--- внешние распределённые~$\bm{q}$ и~внутренние сосредоточенные~$\bm{Q}$.
Движение нити полностью определяется одним вектором-радиусом~${\locationvector(s,t)}$, а~инерционные свойства\:--- линейной плотностью~${\rho\hspace{.12ex}(s)}$.
% linear density

Вот принцип виртуальной работы для куск\'{а} нити ${s_0 \leq s \leq s_1}$

\nopagebreak\vspace{-0.2em}\begin{equation}
\scalebox{0.95}{$
\displaystyle \integral\displaylimits_{\raisemath{.05em}{\mathclap{s_0}}}^{\raisemath{.15em}{\mathclap{s_1}}}
$}
\hspace{-0.25ex} \Bigl( \hspace{-0.1ex}
\bigl( \bm{q} - \hspace{-0.12ex} \rho \hspace{.2ex} \mathdotdotabove{\locationvector} \hspace{.36ex} \bigr)
\hspace{-0.25ex} \dotp \variation{\locationvector}
- \variation{\potential}
\Bigr) ds
\hspace{.1ex} + \hspace{-0.1ex} \Bigl[ \hspace{.1ex}
\bm{Q} \hspace{-0.1ex} \dotp \variation{\locationvector}
\hspace{.15ex} \Bigr]_{\hspace{-0.25ex}s_0}^{\hspace{-0.25ex}s_1}
\hspace{-0.33ex} = 0
\hspace{.1ex} .
\end{equation}

...


Механика нити детально описана в~книге~\cite{merkin-threadmechanics}.

\end{otherlanguage}

\en{\section{Linear theory}}

\ru{\section{Линейная теория}}

\begin{otherlanguage}{russian}

В~линейной теории внешние воздействия считаются малыми, а~отсчётная конфигурация\:--- ненапряжённым состоянием покоя.
Уравнения в~вариациях в~этом случае дают

...



\end{otherlanguage}

\en{\section{Case of small thickness}}

\ru{\section{Случай малой толщины}}

\begin{otherlanguage}{russian}

Когда относительная толщина стержня мал\'{а}, тогда модель типа Cosserat уступает место классической.
Понятие \inquotes{толщина} определяется соотношением жёсткостей: $\bm{a}$, $\bm{b}$ \en{and}\ru{и}~$\bm{c}$\:--- разных единиц измерения; полагая ${\bm{a} = \hcursive^{\hspace{-0.25ex}2} \hspace{.2ex} \widearc{\bm{a}}}$ \en{and}\ru{и}~${\bm{c} = \hcursive \widearc{\bm{c}}}$, \en{where}\ru{где}~$\hcursive$\en{ is}\ru{\:---} \en{some}\ru{некий} \en{scope}\ru{диапазон} \en{of~length}\ru{длины}, получим тензоры ${\widearc{\bm{a}}}$, $\bm{b}$ \en{and}\ru{и}~${\widearc{\bm{c}}\hspace{.2ex}}$ с~одной и~той~же единицей.
Подбирая $\hcursive$ так, чтобы сблизились характерные значения тензоров ${\widearc{\bm{a}}}$, $\bm{b}$ \en{and}\ru{и}~${\widearc{\bm{c}}}$, найдём \en{equivalent thickness}\ru{эквивалентную толщину}~$h$ стержня (для реальных трёхмерных стержней $h$ где-то на~уровне диаметра сечения).

Представив $\internalforce$ \en{and}\ru{и}~$\internalmoment$ через векторы бесконечномалой линейной деформации

...


Переход модели типа Cosserat в~классическую кажется более очевидным, если непосредственно интегрировать уравнения~(...)

...



\end{otherlanguage}

\en{\section{Saint\hbox{-\hspace{-0.2ex}}Venant’s problem}}

\ru{\section{Задача Сэйнт\hbox{-}Венана}}

\begin{otherlanguage}{russian}

Трудно переоценить ту роль, которую играет в~механике стержней классическое решение Saint\hbox{-\hspace{-0.2ex}}Venant\ru{’а}.
О~нём уж\'{е} шла речь в~\chapdotpararef{chapter:linearclassicalelasticity}{para:twistingofrods.saintvenant}.

Вместо условий ...

...



\end{otherlanguage}

\en{\section{Finding stiffness by energy}}

\ru{\section{Нахождение жёсткости по энергии}}

\begin{otherlanguage}{russian}

Для определения тензоров жёсткости~$\bm{a}$, $\bm{b}$ и~$\bm{c}$ одномерной модели достаточно решений трёхмерных задач для стержня.
Но тут возникают два вопроса: какие именно задачи рассматривать и что нужно взять из решений?

\ru{Проблема }Saint\hbox{-\hspace{-0.2ex}}Venant’\en{s}\ru{а}\en{ problem} выделяется среди прочих, ведь оттуда берётся жёсткость на~кручение.

Вдобавок есть много точных решений, получаемых таким путём: задаётся поле~${\bm{u}(\locationvector)}$, определяется~${\widearctoo{\cauchystress} = \stiffnesstensorC \dotdotp \hspace{-0.15ex} \boldnabla \bm{u}}$, затем находятся объёмные~${\bm{f} \hspace{-0.1ex} = - \hspace{.1ex} \boldnabla \dotp \widearctoo{\cauchystress}\hspace{.1ex}}$ и~поверхностные~${\bm{p} = \bm{n} \dotp \widearctoo{\cauchystress}\hspace{.1ex}}$ нагрузки.

Но что делать с~решением?
Ясно, что $\internalforce$ \en{and}\ru{и}~$\internalmoment$ в~стержне\:--- это интегралы по~сечению~(...).
И совсем не~ясно, что считать перемещением и~поворотом в~одномерной модели.
Если предложить, например, такой вариант (индекс у~$\bm{u}$\:--- размерность модели)

\nopagebreak\vspace{-0.1em}\begin{equation*}
\bm{u}_{\raisemath{-0.2ex}{1}}\hspace{-0.15ex}(z) \hspace{-0.1ex} = \hspace{.1ex} \rodsectionexpminusone \hspace{-0.4ex} \integral\displaylimits_{\rodsection} \hspace{-0.5ex} \bm{u}_{\raisemath{-0.2ex}{3}}\hspace{-0.05ex}(\bm{x},z) \hspace{.25ex} d\rodsection
\hspace{.15ex} ,
\:\;
\bm{\theta}(z) \hspace{-0.1ex} = \hspace{.1ex} \smalldisplaystyleonehalf \hspace{.4ex} \rodsectionexpminusone \hspace{-0.4ex} \integral\displaylimits_{\rodsection} \hspace{-0.5ex} \boldnabla \hspace{-0.15ex} \times \hspace{-0.15ex} \bm{u}_{\raisemath{-0.2ex}{3}} \hspace{.2ex} d\rodsection
\hspace{.15ex} ,
\end{equation*}

\vspace{-0.2em}\noindent
то чем другие возможные представления хуже?

Помимо $\internalforce$ \en{and}\ru{и}~$\internalmoment$, есть ещё величина, не вызывающая сомнений\:--- упругая энергия.
Естественно потребовать, чтобы в~одно\-мерной и в~трёх\-мерной моделях энергии на~единицу длины совпали.
При этом, чтобы уйти от~различий в~трактовках ${\bm{u}_{\raisemath{-0.2ex}{1}}}$ и~$\bm{\theta}$, будем исходить из \en{complementary energy}\ru{дополнительной энергии}~${\widehat{\potential}(\internalmoment, \internalforce)}$:

\nopagebreak\vspace{-0.1em}\begin{equation*}
\widehat{\potential}(\internalmoment, \internalforce) = \hspace{-0.2ex} \integral\displaylimits_{\rodsection} \hspace{-0.4ex} \potential_{\raisemath{-0.2ex}{3}} \hspace{.2ex} d\rodsection
\end{equation*}

...



\end{otherlanguage}

\en{\section{Variational method of building one-dimensional model}}

\ru{\section{Вариационный метод построения одномерной модели}}

\label{para:variationalmethodforonedimension}

\begin{otherlanguage}{russian}

Мы только~что определили жёсткости стержня, полагая, что одномерная модель линии типа Cosserat адекватно описывает поведение трёхмерной модели.
\inquotes{Одномерные} представления ассоциируются со~следующей картиной перемещений в~сечении:

\nopagebreak\vspace{-0.1em}\begin{equation}\label{rods.fieldofdisplacements}
\bm{u}(s,\bm{x}) = \bm{U}\hspace{-0.1ex}(s) + \hspace{.15ex} \bm{\theta}(s) \hspace{-0.2ex} \times \bm{x}
\hspace{.1ex} .
\end{equation}

\vspace{-0.1em}\noindent
Однако, такое поле~$\bm{u}$ не~удовлетворяет уравнениям трёхмерной теории\textcolor{red}{(??добавить, каким именно)}.
Невозможно пренебречь возникающими невязками в~дифференциальных уравнениях и краевых условиях.

Формально \inquotes{чистым} является вариационный метод сведения трёхмерной проблемы к~одномерной, называемый иногда методом внутренних связей.
Аппроксимация~\eqref{rods.fieldofdisplacements} подставляется в~трёхмерную формулировку вариационного принципа минимума потенциальной энергии~\eqrefwithchapdotpara{principleofminimumpotentialenergy.formulation}{chapter:linearclassicalelasticity}{para:principleofminimumpotentialenergy}

\nopagebreak\vspace{-0.1em}\begin{equation*}
\potentialenergyfunctional \hspace{.2ex} (\hspace{-0.1ex}\bm{u}\hspace{-0.1ex}) = \hspace{-0.2ex}
\displaystyle\integral\displaylimits_{\mathcal{V}} \hspace{-0.5ex}
\Bigl(
\potential(\hspace{-0.1ex}\bm{u}\hspace{-0.1ex}) \hspace{-0.1ex} - \bm{f} \hspace{-0.1ex} \dotp \bm{u} \Bigr) \hspace{-0.1ex} d\mathcal{V}
- \hspace{-0.3ex}
\displaystyle\integral\displaylimits_{o_2} \hspace{-0.32ex} \bm{p} \dotp \bm{u} \hspace{.33ex} do \hspace{.2ex}
\hspace{.1ex}\to\hspace{.25ex} \mathrm{min}
\hspace{.1ex} ,
\end{equation*}

\vspace{-0.1em}\noindent
которая после интегрирования по~сечению становится одномерной.
Если $\bm{U}$ и~$\bm{\theta}$ варьируются независимо, получаем модель типа Коссера.
В~случае $\bm{U}' \hspace{-0.25ex} = \bm{\theta} \hspace{-0.1ex} \times \bm{t}$ приходим к~классической модели.

Метод внутренних связей привлекателен, его продолжают \inquotesx{переоткрывать}[.]
С~его помощью возможно моделировать тела с~неоднородностью и~анизотропией, он легко обобщается на динамику, если~$\bm{f}$ дополнить неварьируемой динамической добавкой до~${\bm{f} \hspace{-0.1ex} - \rho \hspace{.1ex} \mathdotdotabove{\bm{u}}}$.
Можно рассматривать и~стержни переменного сечения, и~даже нелинейно упругие, ведь вариационная постановка есть~(\chapref{chapter:nonlinearcontinuum}).

Аппроксимацию~\eqref{rods.fieldofdisplacements} можно дополнить слагаемыми с~внутренними степенями свободы.
Понимая необходимость учёта депланаций, некоторые авторы

...

...



Для вариационного построения одномерных моделей удобен принцип Рейсснера\hbox{--}Хеллингера~(\chapdotpararef{chapter:linearclassicalelasticity}{para:mixedvariationalprinciples}) с~независимой аппроксимацией напряжений~\cite{eliseev-models}.
Но тогда необходима некая согласованность между $\bm{u}$ и~$\cauchystress$.

Множеству достоинств вариационного метода противостоит один, но очень большой недостаток.
Вводя аппроксимации по~сечению, мы навязываем реальности свои упрощённые представления.
Вариационный метод более подходит для прикладных расчётов.

\end{otherlanguage}

\en{\section{Asymptotic splitting of three-dimensional problem}}

\ru{\section{Асимптотическое расщепление трёхмерной проблемы}}

\begin{otherlanguage}{russian}

В~изложении механики стержней асимптотическое расщепление можно считать фундаментальным.
Одномерные модели составляют лишь часть картины; другая часть\:--- это двумерные задачи в~сечении, а~вместе они являются тем решением трёхмерной задачи, которое получается для м\'{а}лой толщины.

Малый параметр~$\smallparameter$ в~трёхмерную задачу проще всего ввести через представление вектора\hbox{-}радиуса~(рисунок ?? 22 ??, \pararef{para:overviewofrods}):

...



\end{otherlanguage}

\en{\section{Thermal deformation and stress}}

\ru{\section{Температурные деформация и напряжение}}

\en{The~direct approach}\ru{Прямой подход}, \en{so efficient}\ru{столь эффективный} \en{for making}\ru{для создания} \en{one-dimensional}\ru{одномерных моделей} Cosserat \en{and}\ru{и}~Kirchhoff\ru{’а}\en{ models}, \en{isn’t applicable }\en{for problems}\ru{для~проблем} \en{of~thermoelasticity}\ru{термоупругости}\ru{ непримен\'{и}м}.
\en{Then}\ru{Тогда} \en{the~transition}\ru{переход} \en{from the three-dimensional model to the one-dimensional}\ru{от трёхмерной модели к~одномерной} \en{can be realized}\ru{может быть реализован} \en{either by variational or asymptotic way}\ru{или~вариационным путём, или~асимптотическим}.

\begin{otherlanguage}{russian}

Описанный в~\pararef{para:variationalmethodforonedimension} вариационный метод целиком переносится на~термоупругость\:--- включая задачи с~неоднородностью и~анизотропией, переменным сечением, динамические и~даже нелинейные.
Для этого нужно~(\chapdotpararef{chapter:thermoelasticity}{para:variationalformulations.thermoelasticity})
\en{in}\ru{в} \en{the~}\ru{принципе }Lagrange\ru{’а}\en{ principle} \en{of~minimum potential energy}\ru{минимума потенциальной энергии}
заменить потенциал~${\potential(\hspace{-0.1ex}\infinitesimaldeformation\hspace{-0.1ex})\hspace{-0.1ex}}$
свободной энергией

...




\end{otherlanguage}

\section*{\small \wordforbibliography}

\begin{changemargin}{\parindent}{0pt}
\fontsize{10}{12}\selectfont

\en{Unlike other topics}\ru{В~отличие от других тем} \en{of the~theory of~elasticity}\ru{теории упругости}, \en{rods}\ru{стержни} \en{are presented}\ru{представлены} \en{in books}\ru{в~книгах} \en{very modestly}\ru{очень скромно}.
\ru{Стиль изложения}\en{Narration style of} \inquotes{\en{strength of~materials}\ru{сопротивления материалов}} \en{prevails}\ru{преобладает}, \en{more exact and perfect approaches}\ru{более точные и~совершенные подходы} \en{seem}\ru{кажутся} \en{impossible}\ru{невозможными} \en{or}\ru{или}~\en{unnecessary}\ru{ненужными} \en{to many authors}\ru{многим авторам}.

\begin{otherlanguage}{russian}

Но опубликовано много интересных статей, обзоры которых можно найти у~S.\:Antman’а~\cite{stuartantman-theoryofrods}, В.\,В.\;Елисеева~\cite{eliseev-models} и~А.\,А.\;Илюхина~\cite{ilyuhin-elasticrods}.

\end{otherlanguage}

\end{changemargin}
