\en{\chapter{Rods}}

\ru{\chapter{Стержни}}

\thispagestyle{empty}

\label{chapter:rods}

\begin{otherlanguage}{russian}

\section{Исходные представления}
\label{para:overviewofrods}

\lettrine[lines=2, findent=2pt, nindent=0pt]{С}{тержень}~--- это тонкое длинное тело. Он характеризуется прежде всего своей осью~--- пространственной кривой, которую \inquotes{облепляет} материал (рисунок ?? 22 ??). В~каждой точке оси имеем плоскую фигуру

...


Давно известно, что стержни чувствительны к~моментным нагрузкам. Но присутствие моментов среди обобщённых сил означает наличие вращательных степеней свободы. Следовательно, одномерной моделью стержня должна быть линия Коссера~--- она состоит из элементарных твёрдых тел. Впрочем, могут проявиться и~дополнительные степени свободы~--- как в~тонкостенных стержнях, которым посвящена отдельная глава.

В~механике упругих тел стержни занимают особенное место. Во\hbox{-}первых, это моментные модели; моменты здесь играют главную роль~(не~роль поправок, как в~трёхмерном континууме Коссера). Во\hbox{-}вторых, стержни являются как~бы \inquotes{тестовой площадкой} для~моделей с~дополнительными степенями свободы, поскольку наличие этих степеней можно достоверно исследовать на~трёх\-мерной модели. Ну~а~пока сосредоточимся на~простой одномерной модели Коссера.

\section{Кинематика линий Коссера}

Рассматриваемое далее является упрощённым вариантом~\chapref{chapter:cosseratcontinuum}. Вместо тройки материальных координат ${q^{\hspace{.1ex}i}}$ имеем одну~--- $s$; это может быть дуговая координата в~отсчётной конфигурации. Движение определяется радиусом-вектором~${\bm{r}(s,t)}$ и~тензором поворота~${\bm{P}(s,t)}$. Линейная и~угловая скорости частицы вводятся равенствами

...



\section{Силовые факторы и их баланс}

Поскольку частицы стержня~(линии Коссера)~--- твёрдые тела, то силовыми факторами являются силы и~моменты: на~элемент~$ds$ действуют внешние сила~${\bm{q}ds}$ и~момент~${\bm{m}ds}$. Внутренние взаимо\-действия тоже определяются силой~${\mathboldQ \hspace{.1ex} (s)}$ и~моментом~${\mathboldM \hspace{.1ex} (s)}$~--- это воздействие от~частицы с~координатой~${s\!+\!0}$ к~частице с~${s\!-\!0}$. Из~закона

...



\section{Принцип виртуальной работы и его следствия}

\en{\noindent For a~piece of~rod ${s_0 \leq s \leq s_1}$ formulation of the principle is as~follows}

\ru{\noindent Для куск\'{а} стержня ${s_0 \leq s \leq s_1}$ формулировка принципа таков\'{а}}

...


\en{Conventionally call $\bm{a}$ the~tensor of~stiffness for bending and~twisting, $\bm{b}$ the~tensor of~stiffness for (ex)tension and~shear, and~$\bm{c}$ the~tensor of crosslinks.}

\ru{Условно назовём $\bm{a}$ тензором жёсткости на~изгиб и~кручение, $\bm{b}$~--- тензором жёсткости на~растяжение и~сдвиг, а~$\bm{c}$~--- тензором перекрёстных связей.}

\en{Stiffness tensors rotate together with particle:}

\ru{Тензоры жёсткости поворачиваются вместе с~частицей:}

...



\section{Классическая модель Кирхгофа}

До~сих~пор функции ${\bm{r}(s,t)}$ и~${\bm{P}(s,t)}$ были независимы. В~классической теории Кирхгофа существует внутренняя связь

...



\section{Задача Эйлера}

Рассматривается прямой стержень, защемлённый на одном конце и~нагруженный силой~$\mathboldQ$ на~другом (рисунок ?? 23 ??). Сила~--- \inquotes{мёртвая}, то~есть не~меняется при~деформировании

...



\section{Уравнения в вариациях}

В~нелинейной механике упругих тел полезны уравнения в~вариациях, описывающие малое изменение актуальной конфигурации. Как и в~\chapdotpararef{chapter:nonlinearcontinuum}{para:variationofconfiguration}, вариации величин

...



\section{Модель с растяжением без сдвига}

Модель Кирхгофа с~${\Gamma \hspace{-0.25ex}=\hspace{-0.2ex} 0}$ не~описывает наипростейший случай растяжения\hbox{--}сжатия прямого стержня. Эта неприятность исчезнет, если смягчить связь: запретить лишь поперечный сдвиг, но разрешить растяжение, то~есть

...



\section{Механика нити}

Этот вопрос выходит за~рамки главы, поскольку нить проще стержня: её частицами являются \inquotes{обычные} материальные точки c~трансляционными степенями свободы. Соответственно, ...

...



\section{Линейная теория}

В~линейной теории внешние воздействия считаются малыми, а~отсчётная конфигурация~--- ненапряжённым состоянием покоя. Уравнения в~вариациях в~этом случае дают

...



\section{Случай малой толщины}

При м\'{а}лой относительной толщине стержня модель типа Коссера уступает место классической. Понятие \inquotes{толщина} определяется соотношением жёсткостей: $\bm{a}$, $\bm{b}$ и~$\bm{c}$~--- разной размерности; полагая ${\bm{a} = h^{\hspace{-0.12ex}2} \hspace{.2ex} \widearc{\bm{a}}}$ и~${\bm{c} = h \hspace{.16ex} \widearc{\bm{c}}}$, где~$h$~--- некий масштаб длины, получим тензоры

...


Переход модели Коссера в~классическую кажется более очевидным при непосредственном интегрировании

...



\section{Задача Сен-Венана}

Трудно переоценить ту роль, которую играет в~механике стержней классическое решение Сен-Венана. О~нём уж\'{е} шла речь в~...

...



\section{Определение жёсткостей по энергии}

Тензоры жёсткости~$\bm{a}$, $\bm{b}$ и~$\bm{c}$ в~одномерной модели

...



\section{Вариационный метод построения одномерных моделей}
\label{para:variationalmethodforonedimension}

Мы только~что определили жёсткости стержня, полагая, что одномерная модель Коссера правильно отражает поведение трёхмерной модели. \inquotes{Одномерные} представления ассоциируются со~следующей картиной перемещений в~сечении:

...



\section{Асимптотическое ращепление трёхмерной задачи}

В~изложении механики стержней этот вопрос я считаю основным. Одномерные модели составляют лишь часть картины; другая часть~--- это двумерные задачи в~сечении, а~вместе они являются тем решением трёхмерной задачи, которое образуется при~м\'{а}лой толщине.

Малый параметр~$\lambda$ в~трёхмерную задачу проще всего ввести через представление радиуса\hbox{-}вектора~(рисунок ?? 22 ??, \pararef{para:overviewofrods}):

...



\section{Температурные деформации и напряжения}

Прямой подход, столь эффективный при~построении одномерных моделей Коссера и~Кирхгофа, теряет силу в~задачах термоупругости. Необходимо рассматривать трёхмерную модель, что может быть реализовано или~вариационным путём, или~асимптотическим.

Описанный в~\pararef{para:variationalmethodforonedimension} вариационный метод целиком переносится на~термоупругость~--- включая задачи с~неоднородностью и~анизотропией, переменным сечением, динамические~--- и~даже нелинейные. Достаточно в~принципе Лагранжа заменить потенциал

...




\vspace{8mm}
\hfill\begin{minipage}[b]{0.95\linewidth}
\fontsize{10}{12}\selectfont

\section*{\wordforbibliography}

В~отличие от~других разделов теории упругости, механика стержней скромно представлена в~книгах. Преобладает изложение в~духе сопротивления материалов, более совершенные подходы кажутся многим авторам невозможными или~ненужными. Но есть немало интересных статей; соответствующие обзоры можно найти у~...

\end{minipage}

\end{otherlanguage}
