\en{\chapter{Rods}}

\ru{\chapter{Стержни}}

\thispagestyle{empty}

\newcommand\internalforce{\mathboldQ} %% \bm{Q}
\newcommand\internaldistributedforce{\bm{q}}
\newcommand\internalmoment{\mathboldM}
\newcommand\internaldistributedmoment{\bm{m}}

\newcommand\tangenttoaxis{\bm{k}}
\newcommand\thirdbasisvector{\bm{e}_3}

\newcommand\rodsection{\Upomega} %% {F}
\newcommand\rodsectionexpminusone{\rodsection^{\hspace{-0.2ex}\expminusone}}

\newcommand\rodfeaturesvector{\bm{\psi}}
\newcommand\rodfeaturesvectorcomponents[1]{\psi_{#1}}
\newcommand\initialrodfeaturesvector{{\mathcircabove{\bm{\psi}}}}
\newcommand\perpendicularrodfeatures{{\bm{\psi}_{\hspace{-0.2ex}\perp}}}

\newcommand\tangentvector{{\bm{T}}}
\newcommand\normalvector{{\mathboldN}}
\newcommand\binormalvector{{\bm{B}}}

\newcommand\firstdeformationvectorcosseratline{{\bm{\Gamma}}}
\newcommand\firstdeformationvectorcosseratlinecomponents[1]{{\Gamma_{\hspace{-0.2ex}#1}}}
\newcommand\seconddeformationvectorcosseratline{{\bm{\kappa}}}

\newcommand\perpendicularboldnabla{{\boldnabla_{\hspace{-0.4ex}\perp}}}

\newcommand\perpendicularstress{{\mathboldtau_{\hspace{-0.3ex}\perp}}}
\newcommand\parallelstressvector{{\bm{\tau}_{\hspace{.1ex}\parallel}}} %{^1\hspace{-0.25ex}}\mathboldtau_{\parallel}

\newcommand\RepresentedLocationVector{{\bm{R}\hspace{.1ex}}}

\newcommand\RepresentedVolume{{\mathrm{v}}}

\label{chapter:rods}

\en{\section{Initial concepts}}

\ru{\section{Исходные представления}}

\label{para:overviewofrods}

\en{\dropcap{R}{od}}\ru{\dropcap{С}{тержень}}\ru{\:--- это}\en{ is} \en{a~thin long body}\ru{тонкое длинное тело}.
\en{It is thought of}\ru{Он мыслится} (\en{and modeled}\ru{и~моделируется}) \en{as}\ru{как} \en{a~spatial curve}\ru{пространственная кривая}\:--- \en{the~axis of~rod}\ru{ось стержня}, \en{coated with a~material}\ru{покрытая материалом}
( \textcolor{blue}{\en{a figure}\ru{рисунок}} ).

\en{The~axis of rod}\ru{Ось стержня} \en{is described}\ru{описывается} \en{as a~curve}\ru{как кривая} \en{by parameterizing}\ru{параметризацией} \en{the~location vector}\ru{вектора положения} \en{of~points of~a~curve}\ru{точек кривой}.
\en{This is a~morphism}\ru{Это морфизм}~(\en{a~function}\ru{функция}) \en{of~one}\ru{одной} \en{variable}\ru{переменной} \en{coordinate}\ru{координаты}~$s$,

\nopagebreak\vspace{-0.2em}\begin{equation}
\locationvector \hspace{-0.4ex} = \hspace{-0.3ex} \locationvector(s)
\hspace{.1ex} .
\end{equation}

\en{Material coating}\ru{Покрытие материалом} \en{gives}\ru{даёт} \en{at each rod’s point}\ru{в~каждой точке стержня} \en{a~plane figure}\ru{плоскую фигуру}, \en{perpendicular to the~axis}\ru{перпендикулярную оси}\:--- \en{normal section}\ru{нормальное сечение}~${\rodsection(s)}$.

...

${\bm{c} \narroweq \bm{c}(t)}$\en{ is}\ru{\:---} \en{a~parametric curve parameterized by parameter}\ru{параметрическая кривая, параметризованная параметром}~$t$.
\en{If}\ru{Если} ${dt \neq d\ell}$, \en{then}\ru{то} \en{a~parameterization}\ru{параметризация} \en{is not natural}\ru{не натуральная}.
\en{For the natural parameterization}\ru{Для натуральной параметризации} ${dt \hspace{-0.1ex} = d\ell}$, \en{where}\ru{где}

\nopagebreak\vspace{-0.1em}\begin{equation*}
\label{lengthofdifferentialpieceofcurve}
d\ell \hspace{-0.1ex} = \hspace{-0.2ex} \scalebox{.88}{$ \displaystyle \sqrt{ \hspace{-0.2ex} \bigl( dq^1 \bigr)^{\hspace{-0.3ex}2} \hspace{-0.25ex} + \bigl( dq^2 \bigr)^{\hspace{-0.3ex}2} \hspace{-0.25ex} + \bigl( dq^3 \bigr)^{\hspace{-0.3ex}2} \hspace{.2ex} } $}
\hspace{-0.2ex} .
\end{equation*}

\en{Many different functions}\ru{Многие разные функции} \en{draw}\ru{рисуют} \en{the same curve}\ru{одну и~ту~же кривую}.
\en{But}\ru{Но} \en{among various parametrizations}\ru{среди различных параметризаций} \en{of a~curve}\ru{кривой}, \en{the parametrization by the arc length}\ru{параметризация длиной дуги} \en{is special}\ru{особенная}, \en{it is also called}\ru{её также называют} \en{the }\emph{\en{natural}\ru{естественной} \en{parametrization}\ru{параметризацией}}.

\en{The length}\ru{Длина} \en{of }\en{an~infinitesimal}\ru{бесконечно-малого} \en{piece}\ru{кусочка} \en{of~a~curve}\ru{кривой} \en{is described}\ru{описывается} \en{by the }\textgreek{Πυθαγόρας}\hbox{-}\en{formula}\ru{формулой}

\nopagebreak\vspace{-0.1em}\begin{equation*}
d\ell \hspace{-0.2ex} = \hspace{-0.2ex} \scalebox{.93}{$ \displaystyle\sqrt{ \hspace{-0.2ex} dx^{2} \hspace{-0.2ex} + dy^{2} \hspace{-0.2ex} + dz^{2} \hspace{.3ex} \hspace{.1ex} } $}
\end{equation*}

\vspace{-0.2em}\noindent
\en{where}\ru{где} ${dx \equiv dq^1\hspace{-0.4ex}}$, ${dy \equiv dq^2\hspace{-0.4ex}}$, ${dz \equiv dq^3\hspace{-0.2ex}}$\en{ are}\ru{\:---} \en{infinitesimal changes of~coordinates}\ru{бесконечно-малые изменения координат}.
$d\ell$ \en{is called}\ru{называется} \en{the differential length}\ru{дифференциальной длиной}, \en{that is}\ru{то есть} \en{the length}\ru{длиной} \en{of an~almost straight}\ru{почти прямого} \en{very small piece of a~curve}\ru{очень м\'{а}лого куска кривой}.

%%%%
\begin{comment}

\noindent
\en{Rewriting it}\ru{Переписав её}, \en{for example}\ru{например}, \en{thru}\ru{через}~${dy}$ \en{and derivatives}\ru{и~производные}

\nopagebreak\vspace{-0.4em}\begin{multline*}
d\ell \hspace{-0.1ex}
= \hspace{-0.2ex}
\scalebox{.8}{$
\displaystyle\sqrt{ \hspace{-0.2ex}
\Bigl( \displaystyle\frac{\raisemath{-0.2em}{dx}}{dy} \hspace{.2ex} dy \Bigr)^{\hspace{-0.5ex}2} \hspace{-0.6ex}
+ dy^{2} \hspace{-0.1ex}
+ \hspace{-0.2ex} \Bigl( \displaystyle\frac{\raisemath{-0.2em}{dz}}{dy} \hspace{.2ex} dy \Bigr)^{\hspace{-0.5ex}2}
\hspace{.3ex} }
$} \hspace{-0.5ex} =
%
\scalebox{.8}{$
\displaystyle\sqrt{ \hspace{-0.2ex}
\Bigl( \displaystyle\frac{\raisemath{-0.2em}{dx}}{dy} \Bigr)^{\hspace{-0.5ex}2} \hspace{-0.2ex} dy^{2}
+ dy^{2} \hspace{-0.1ex}
+ \hspace{-0.2ex} \Bigl( \displaystyle\frac{\raisemath{-0.2em}{dz}}{dy} \Bigr)^{\hspace{-0.5ex}2} \hspace{-0.2ex} dy^{2}
\hspace{.3ex} }
$} \hspace{-0.4ex} =
\\
%
= dy \hspace{.2ex}
\scalebox{.8}{$
\displaystyle\sqrt{ \hspace{-0.2ex}
\Bigl( \displaystyle\frac{\raisemath{-0.2em}{dx}}{dy} \Bigr)^{\hspace{-0.5ex}2} \hspace{-0.6ex}
+ 1
+ \hspace{-0.2ex} \Bigl( \displaystyle\frac{\raisemath{-0.2em}{dz}}{dy} \Bigr)^{\hspace{-0.5ex}2}
\hspace{.3ex} }
$} \hspace{-0.3ex} .
\end{multline*}

\end{comment}
%%%%

${\bm{c}(s)}$ is a~parametric curve parameterized by the arc length (the natural parametrization), its derivative by the arc length parameter is denoted as~${ \bm{c}\hspace{.2ex}' \hspace{-0.2ex} \equiv \scalebox{.88}{$ \displaystyle\frac{\raisemath{-0.15em}{d \hspace{.15ex} \bm{c}}}{d s} $} }$.

\en{If}\ru{Если} \ru{используется }\en{the arc length (natural) parametrization}\ru{параметризация длиной дуги (естественная)}~${\locationvector \narroweq \locationvector(s)}$\en{ is used}, \en{then}\ru{то} \en{the length}\ru{длина} \en{of derivative}\ru{производной} ${ \locationvector'(s) \hspace{-0.2ex} \equiv \scalebox{.8}{$ \displaystyle\frac{\raisemath{-0.2em}{d \hspace{.1ex} \locationvector(s)}}{\raisemath{.05em}{ds}} $} }$ (\en{of the tangent vector}\ru{касательного вектора}) \en{is always equal to}\ru{всегда равна} \en{the one unit of length}\ru{одной единице длины}:

\nopagebreak\vspace{-0.2em}\begin{gather*}
%%\tikz[baseline=-1ex] \draw[line width=.5pt, color=black, fill=white] (0, 0) circle (.8ex);
%%\hspace{2.25ex}
%%\\
%
\locationvector(s) \hspace{-0.2ex} = q_{i}(s) \hspace{.2ex} \bm{e}_{i}(s) \hspace{-0.2ex} = q_1 \bm{e}_1 \hspace{-0.2ex} + q_2 \bm{e}_2 \hspace{-0.2ex} + q_3 \bm{e}_3
\hspace{.1ex} ,
\\[.1em]
%
\locationvector'(s) \hspace{-0.2ex}
= \scalebox{.88}{$ \displaystyle\frac{\raisemath{-0.2em}{d \hspace{.1ex} \locationvector(s)}}{ds} $}
= \scalebox{.88}{$ \displaystyle\frac{\raisemath{-0.2em}{d \hspace{.1ex} q_{i}(s)}}{ds} $} \hspace{.2ex} \bm{e}_{i}(s) \hspace{-0.2ex}
= q_{i}'(s) \hspace{.2ex} \bm{e}_{i}(s)
\hspace{.1ex} ,
\\[-0.5em]
%
\| \locationvector'(s) \|^{2} \hspace{-0.2ex} \equiv \locationvector'(s) \hspace{-0.1ex} \dotp \hspace{.2ex} \locationvector'(s) \hspace{-0.2ex}
= q_{i}'(s) \hspace{.2ex} \bm{e}_{i} \hspace{-0.1ex} \dotp \hspace{.1ex} q_{\hspace{-0.1ex}j}'(s) \hspace{.2ex} \bm{e}_{\hspace{-0.1ex}j} \hspace{-0.2ex}
= q_{i}'(s) \hspace{.2ex} q_{\hspace{-0.1ex}j}'(s) \hspace{.2ex} \delta_{i\hspace{-0.1ex}j} \hspace{-0.2ex}
= \hspace{-0.2ex} \scalebox{.88}{$ \displaystyle\sum_{i=1}^{3} $} \hspace{-0.1ex} \bigl( \hspace{.1ex} q_{i}'(s) \hspace{.1ex} \bigr)^{\hspace{-0.2ex}2}
\hspace{-0.4ex} ,
\\[-0.3em]
%
ds \hspace{-0.2ex} = \hspace{-0.2ex} \scalebox{.88}{$ \displaystyle \sqrt{ \hspace{-0.2ex} \bigl( dq_1 \bigr)^{\hspace{-0.2ex}2} \hspace{-0.3ex} + \bigl( dq_2 \bigr)^{\hspace{-0.2ex}2} \hspace{-0.3ex} + \bigl( dq_3 \bigr)^{\hspace{-0.2ex}2} } $}
\hspace{.4em} \Rightarrow \hspace{.5em}
ds^2 \hspace{-0.2ex} = \hspace{-0.2ex} \scalebox{.88}{$ \displaystyle\sum_{i=1}^{3} $} \hspace{-0.1ex} \bigl( \hspace{.1ex} d \hspace{.1ex} q_{i}(s) \hspace{.1ex} \bigr)^{\hspace{-0.2ex}2}
\hspace{-0.4ex} ,
\\[-0.8em]
%
\| \locationvector'(s) \|^{2} \hspace{-0.1ex}
= \hspace{.2ex} \scalebox{.85}{$ \displaystyle\frac{\raisemath{-0.2em}{\scalebox{.88}{$ \displaystyle\sum_{i=1}^{3} $} \bigl( d \hspace{.1ex} q_{i}(s) \bigr)^{\hspace{-0.2ex}2}} }{\raisemath{-0.3em}{\bigl( ds \bigr)^{\hspace{-0.3ex}2}}} $}
= 1
\hspace{.6em} \Rightarrow \hspace{.5em}
\| \locationvector'(s) \| \hspace{-0.2ex} = 1
\hspace{.1ex} .
%%\\
%
%%\hspace{2.25ex}
%%\tikz[baseline=-0.6ex] \draw[color=black, fill=black] (0, 0) circle (.8ex);
\end{gather*}

...

\en{In each section}\ru{В~каждом сечении} \en{we select}\ru{мы выбираем} \en{two}\ru{две} \en{perpendicular}\ru{перпендикулярные} \en{axes}\ru{оси} $x_{\alpha}$ \en{with co-directed}\ru{с~сонаправленными} \en{unit vectors}\ru{единичными векторами}~${\bm{e}_{\alpha}}$ (${\alpha \narroweq 1, 2}$).
\en{The reason of selection}\ru{Причина выбора} \en{for all rod sections}\ru{для всех сечений стержня} \en{is the same}\ru{одна и~та~же}, \en{for example}\ru{для примера}, \en{the main axes of inertia of the section}\ru{главные оси инерции сечения} \en{are chosen}\ru{выбираются} \en{everywhere}\ru{везде}.

%%
\begin{tcolorbox}[enhanced, colback = orange!40, before upper={\parindent2ex}, parbox = false]
\small%
\setlength{\abovedisplayskip}{2pt}\setlength{\belowdisplayskip}{2pt}%

\begin{center}
\emph{\en{The actual axis}\ru{Актуальная ось} \en{and}\ru{и} \en{the initial axis}\ru{начальная ось} \en{are different}\ru{отличаются}}
\vspace{-1em}
\end{center}

\en{When vector}\ru{Когда вектор}~${\mathcircabove{\bm{e}}_3}$ \en{is directed}\ru{направлен} \en{along the tangent}\ru{по касательной} \en{to the initial axis}\ru{к~начальной оси} \en{with location}\ru{с~положением}~$\initiallocationvector$, \en{it is written as}\ru{это пишется как}
${%
\initiallocationvector{\hspace{.1ex}'} \hspace{-0.3ex}
\equiv \scalebox{.8}{$ \displaystyle\frac{\raisemath{-0.15em}{\partial \hspace{.15ex} \initiallocationvector}}{\partial s} $}
\equiv \initiallocationvector_\differentialindex{s} \hspace{-0.2ex}
\equiv \mathcircabove{\bm{e}}_3
}$.

\en{Vector}\ru{Вектор}~${\bm{e}_{3} \hspace{-0.2ex} \equiv \bm{k}}$ \en{is directed}\ru{направлен} \en{along the tangent}\ru{вдоль касательной} (\en{tangentially}\ru{тангенциально}) \en{to the actual axis}\ru{к~актуальной оси}.
\end{tcolorbox}
%%

\en{Together}\ru{Вместе} \en{with }\ru{с~}\en{the unit vector}\ru{единичным вектором}, \en{tangent}\ru{касательным} \en{to the actual axis}\ru{к~актуальной оси}

\nopagebreak\vspace{-0.2em}\begin{equation*}
\locationvector{\hspace{.1ex}'} \hspace{-0.3ex}
\equiv \scalebox{.8}{$ \displaystyle\frac{\raisemath{-0.15em}{d \hspace{.15ex} \locationvector}}{d s} $}
%%\equiv \scalebox{.8}{$ \displaystyle\frac{\raisemath{-0.15em}{\partial \hspace{.15ex} \locationvector}}{\partial s} $}
\equiv \locationvector_\differentialindex{s} \hspace{-0.2ex}
\equiv \bm{e}_3 \hspace{-0.2ex}
\equiv \bm{k}
\hspace{.15ex} ,
\end{equation*}

\noindent
\en{we’ll get}\ru{мы получим} \en{for each}\ru{для каждого}~$s$ \en{a~triple}\ru{тройку} \en{of mutually perpendicular}\ru{взаимно перпендикулярных} \en{unit vectors}\ru{единичных векторов}.

\en{The curvature}\ru{Кривизна} \en{and}\ru{и}~\en{the torsion}\ru{кручение} \en{of~the~rod’s axis}\ru{оси стержня} \textcolor{blue}{\en{can be described}\ru{могут быть описаны}} \en{by vector}\ru{вектором}~${\rodfeaturesvector \hspace{-0.3ex} = \hspace{-0.3ex} \rodfeaturesvectorcomponents{\hspace{-0.1ex}j} \hspace{.1ex} \bm{e}_{\hspace{-0.1ex}j}}$:

\nopagebreak\vspace{-0.2em}\begin{equation}
\label{rods.thepsivector}
\bm{e}{\hspace{.1ex}'}_{\hspace{-0.7ex}j} \hspace{-0.2ex}
= \rodfeaturesvector \hspace{-0.1ex} \times \hspace{-0.1ex} \bm{e}_{j}
\hspace{.1ex} , \hspace{.5em}
\rodfeaturesvector \hspace{-0.1ex} = \smalldisplaystyleonehalf \hspace{.25ex} \bm{e}_{j} \hspace{-0.2ex} \times \hspace{-0.1ex} \bm{e}{\hspace{.1ex}'}_{\hspace{-0.7ex}j}
\hspace{.1ex} .
\end{equation}

\en{For}\ru{Для} \en{a~}\en{cylindrical}\ru{цилиндрического} (\en{prismatic}\ru{призматического}) \en{rod}\ru{стержня} ${\rodfeaturesvector = \bm{0}}$.

\en{However}\ru{Однако}, \eqref{rods.thepsivector} \en{is}\ru{есть} \en{only}\ru{лишь} \en{an~initial concept}\ru{первоначальное понятие} \en{of~vector}\ru{о~векторе}~${\rodfeaturesvector}$ \en{as}\ru{как} \en{of geometric features}\ru{о~геометрических характеристиках}.
\en{Further}\ru{Далее} \en{in}\ru{в}~\pararef{para:rods.cosseratlines}, \en{after}\ru{после} \en{adopting}\ru{принятия} \en{the material structure}\ru{материальной структуры} \en{of a~rod}\ru{стержня}, \en{a~concept}\ru{понятие} \en{of}\ru{о}~${\rodfeaturesvector}$ \en{will change}\ru{изменится}.

\en{Moreover}\ru{Кроме этого}, \en{at each point}\ru{в~каждой точке} \en{of the rod’s axis}\ru{оси стержня}, \en{thought of as a~curve}\ru{мыслимой как кривая}, \en{there’s}\ru{есть} \en{also another}\ru{также другая} \en{triple of mutually perpendicular unit vectors}\ru{тройка взаимно перпендикулярных единичных векторов}, \en{the one}\ru{та что} \en{with the normal}\ru{с~нормальным} \en{and the binormal}\ru{и~бинормальным} \en{vectors}\ru{векторами}.

\en{Tangent}\ru{Касательный}~$\tangentvector$, \en{normal}\ru{нормальный}~$\normalvector$ \en{and}\ru{и} \en{binormal}\ru{бинормальный}~$\binormalvector$ \en{vectors}\ru{векторы}, \en{together called}\ru{вместе называемые} \en{the }\emph{\ru{системой }Frenet\hbox{--}Serret\en{ frame}}, \en{are defined as}\ru{определяются как}:
%
\begin{itemize}
\item $\tangentvector$\en{ is}\ru{\:---} \en{the unit vector tangent to a~curve}\ru{единичный вектор, касательный к~кривой}. \en{The length}\ru{Длина} \en{of the tangent vector}\ru{касательного вектора} \en{is always one unit}\ru{всегда одна единица}, \en{if}\ru{если} \en{the natural (arc length) parametrization of~a~curve is used}\ru{используется естественная параметризация кривой (длиной дуги)}. \en{The tangent vector}\ru{Касательный вектор} \en{points to where}\ru{указывает туда, где} \en{a~curve}\ru{кривая} \en{continues further}\ru{продолжается дальше}.
%
\item $\normalvector$\en{ is}\ru{\:---} \en{the normal unit vector}\ru{нормальный единичный вектор}, \en{the derivative}\ru{производная}\en{of}~$\tangentvector$ \en{by the curve’s parameter}\ru{по параметру кривой} (\en{for instance}\ru{например}, the arc length of a~curve). \en{The normal vector}\ru{Нормальный вектор} \en{is always}\ru{всегда} \en{perpendicular}\ru{перпендикулярен} \en{to the tangent vector}\ru{касательному вектору} \en{and }\ru{и~}\en{points towards the center of curvature}\ru{направлен к~центру кривизны}. \en{It is divided by its length}\ru{Он поделён на свою длину}~${\| \normalvector \|}$\ru{,} \en{to be the one unit long}\ru{чтобы быть длиной в~одну единицу}.
%
\item $\binormalvector$\en{ is}\ru{\:---} \en{the binormal unit vector}\ru{бинормальный единичный вектор}, 
\en{the~}\hbox{\hspace{-0.2ex}\inquotes{${\hspace{-0.25ex}\times\hspace{-0.1ex}}$}\hspace{-0.2ex}-\en{product}\ru{произведение}} (\inquotes{cross product}) \en{of~}$\tangentvector$ \en{and}\ru{и}~$\normalvector$, ${\binormalvector \hspace{-0.1ex} \equiv \tangentvector \hspace{-0.15ex} \times \hspace{-0.3ex} \normalvector}$.
\end{itemize}

\ru{Формулы }\en{The }Frenet\hbox{--}Serret\en{ formulas} \en{describe}\ru{описывают} \en{the derivatives}\ru{производные} \en{of }\en{tangent}\ru{касательного}, \en{normal}\ru{нормального} \en{and }\ru{и~}\en{binormal}\ru{бинормального} \en{unit vectors}\ru{единичных векторов}  \en{through relations with each other}\ru{через отношения друг с~другом}.

\begin{align*}
\scalebox{.88}{$ \displaystyle\frac{\raisemath{-0.2em}{ d \hspace{.1ex} \tangentvector }}{ds} $} &= \kappa \hspace{.1ex} \normalvector ,
\\
\scalebox{.88}{$ \displaystyle\frac{\raisemath{-0.2em}{ d \hspace{.2ex} \normalvector }}{ds} $} &= \hspace{-0.4ex} {}- \hspace{-0.3ex} \kappa \hspace{.25ex} \tangentvector + \tau \hspace{-0.1ex} \binormalvector ,
\\
\scalebox{.88}{$ \displaystyle\frac{\raisemath{-0.2em}{ d \hspace{.1ex} \binormalvector }}{ds} $} &= \hspace{-0.4ex} {}- \hspace{.1ex} \hspace{-0.3ex} \tau \hspace{-0.1ex} \normalvector ,
\end{align*}

\vspace{-0.2em}\noindent
\en{where}\ru{где} ${\scalebox{.93}{$ \raisemath{.3em}{d} \hspace{-0.25ex} / \hspace{-0.33ex} \raisemath{-0.25em}{ds} $}}$ \en{denotes}\ru{обозначает} \en{the~derivative by the arc length}\ru{производную по длине дуги}, $\kappa$ \en{is}\ru{есть} \en{the curvature}\ru{кривизна} \en{and}\ru{и}~$\tau$ \en{is}\ru{есть} \en{the curve’s torsion}\ru{кручение кривой}.
\en{The associated collection}\ru{Соединённая коллекция}\:--- ${\tangentvector\hspace{-0.2ex}}$, ${\normalvector\hspace{-0.2ex}}$, ${\binormalvector\hspace{-0.1ex}}$, $\kappa$, $\tau$\:--- \en{is called}\ru{называется} \en{the }\ru{аппаратом }Frenet--Serret\en{ apparatus}.

\en{Two scalars}\ru{Два скаляра} $\kappa$ \en{and}\ru{и}~$\tau$ \en{effectively define}\ru{эффективно определяют} \en{the~curvature}\ru{кривизну} \en{and }\ru{и~}\en{the~torsion}\ru{кручение} \en{of a~space curve}\ru{пространственной кривой}.
\en{Intuitively}\ru{Интуитивно}, \en{the curvature}\ru{кривизна} \en{measures}\ru{измеряет} \en{the deviation}\ru{отклонение} \en{of a~curve}\ru{кривой} \en{from a~straight line}\ru{от прямой линии}, \en{while}\ru{тогда как} \en{the torsion}\ru{кручение} \en{measures}\ru{измеряет} \en{the deviation}\ru{отклонение} \en{of a~curve}\ru{кривой} \en{from being planar}\ru{от плоской}.

\en{Two functions}\ru{Две функции} $\kappa(s)$ \en{and}\ru{и}~$\tau(s)$ \en{completely define the geometry of a~curve}\ru{полностью определяют геометрию кривой}, \en{because}\ru{поскольку} \en{they are}\ru{это} \en{coefficients}\ru{коэффициенты} \en{of a~system}\ru{системы} \en{of ordinary differential equations}\ru{обыкновенных дифференциальных уравнений} \en{for}\ru{для}~$\tangentvector$, $\normalvector$ \en{and}\ru{и}~$\binormalvector$.
\en{Knowing}\ru{Зная}~${\tangentvector(s)}$, \en{by integration}\ru{интегрированием} \en{we’ll obtain}\ru{мы получим}~${\locationvector(s)}$ \en{with accuracy}\ru{с~точностью} \en{up to}\ru{до} \en{a~constant}\ru{постоянного} \en{rigid displacement}\ru{жёсткого перемещения} \en{without deformations}\ru{без деформаций}.

...

\begin{align*}
\tangentvector = \scalebox{.88}{$ \displaystyle\frac{\raisemath{-0.15em}{d \hspace{.15ex} \locationvector}}{d s} $}
\hspace{1em} & \text{\en{or}\ru{или}} \hspace{1em}
\tangentvector = \locationvector{\hspace{.1ex}'}
\end{align*}

\en{The derivative}\ru{Производная} \en{of~}${\tangentvector}$ \en{consists of two multipliers}\ru{состоит из двух множителей}\:--- \en{the curvature}\ru{кривизны}~${\kappa}$ \en{and}\ru{и} \en{the unit normal vector}\ru{единичного нормального вектора}~${\normalvector}$

\begin{align*}
\scalebox{.88}{$ \displaystyle\frac{\raisemath{-0.15em}{d \hspace{.15ex} \tangentvector}}{d s} $} = \kappa \hspace{.1ex} \normalvector
\hspace{1em} & \text{\en{or}\ru{или}} \hspace{1em}
\tangentvector{\hspace{.25ex}'} \hspace{-0.4ex} = \kappa \hspace{.1ex} \normalvector
\end{align*}

\en{Curvature}\ru{Кривизна}~$\kappa$ \en{is equal to}\ru{равна} \en{the magnitude}\ru{магнитуде}~(\en{length}\ru{длине}) \en{of~vector}\ru{вектора}~$\normalvector$ (\en{the~derivative}\ru{производной} \en{of~vector}\ru{вектора}~$\tangentvector$, \en{the~second derivative}\ru{второй производной} \en{of location vector}\ru{вектора положения}~$\locationvector$)

\begin{align*}
\kappa
= \| \normalvector \|
= \| \tangentvector{\hspace{.25ex}'} \|
= \left\Vert \scalebox{.88}{$ \displaystyle\frac{\raisemath{-0.15em}{d \hspace{.15ex} \tangentvector}}{d s} $} \right\Vert
= \| \locationvector'' \|
= \left\Vert \scalebox{.88}{$ \displaystyle\frac{\raisemath{-0.15em}{d^2 \hspace{.15ex} \locationvector}}{d s^2} $} \right\Vert
\end{align*}

\en{Vector}\ru{Сам вектор}~$\normalvector$\en{ itself} \en{is divided by its length}\ru{поделён на свою длину}\ru{,} \en{thus}\ru{поэтому} \en{its length}\ru{его длина} \en{is equal to}\ru{равна} \en{the one unit}\ru{одной единице}.

\begin{align*}
\normalvector
= \scalebox{.88}{$ \displaystyle\frac{ \raisemath{.5em}{ \scalebox{.88}{$ \displaystyle\frac{\raisemath{-0.15em}{d \hspace{.15ex} \tangentvector}}{d s} $} }}%
{\raisemath{-0.5em}{\left\Vert \scalebox{.88}{$ \displaystyle\frac{\raisemath{-0.15em}{d \hspace{.15ex} \tangentvector}}{d s} $} \right\Vert} } $}
\hspace{1em} & \text{\en{or}\ru{или}} \hspace{1em}
\normalvector
= \scalebox{.88}{$ \displaystyle\frac{\raisemath{-0.15em}{\tangentvector{\hspace{.25ex}'}}}{\| \tangentvector{\hspace{.25ex}'} \|} $}
\end{align*}

\en{The~radius of~curvature}\ru{Радиус кривизны}\en{ is}\ru{\:---} \en{the~reciprocal of~curvature}\ru{число, обратное кривизне}.

\begin{align*}
\scalebox{.88}{$ \displaystyle\frac{\raisemath{-0.15em}{1}}{\kappa} \hspace{.2ex} \displaystyle\frac{\raisemath{-0.15em}{d \hspace{.15ex} \tangentvector}}{d s} $} = \normalvector
\hspace{1em} & \text{\en{or}\ru{или}} \hspace{1em}
\scalebox{.88}{$ \displaystyle\frac{\raisemath{-0.15em}{1}}{\kappa} $} \hspace{.3ex} \tangentvector{\hspace{.25ex}'} \hspace{-0.4ex} = \normalvector
\end{align*}

...

\begin{equation*}
\kappa \ge 0
\end{equation*}

\ru{Система }\en{The }Frenet\hbox{--}Serret\en{ frame} \en{is defined}\ru{определена} \en{only if}\ru{лишь если} \en{the curvature}\ru{кривизна} \en{is nonzero}\ru{отлична от нуля}~(${\kappa > 0}$), \en{it is not defined if}\ru{она не определена, если} ${\kappa \hspace{-0.2ex} = \hspace{-0.1ex} 0}$.

\en{A~line}\ru{Линия} \en{with the nonzero curvature}\ru{с~ненулевой кривизной}~${\kappa \hspace{-0.2ex} \neq \hspace{-0.1ex} 0}$ \en{is considered a~curve}\ru{считается кривой}.

\en{The zero curvature}\ru{Нулевая кривизна} \en{implies that}\ru{предполагает, что} \en{a~line is straight}\ru{линия прямая}, \en{and it lies in a~plane}\ru{и она лежит в~плоскости}, \en{making}\ru{делая} \en{the torsion}\ru{кручение} \en{equal to zero too}\ru{тоже равным нулю}
(${\tau \hspace{-0.3ex} = \hspace{-0.2ex} 0}$).

...

$\tangentvector$ \en{always}\ru{всегда} \en{has}\ru{имеет} \en{the unit magnitude}\ru{единичную магнитуду}~(\en{length}\ru{длину}).
\en{Since}\ru{Поскольку} \en{the length}\ru{длина} \en{of~}$\tangentvector$ \en{is constant}\ru{постоянна}, \en{then}\ru{то}~$\normalvector$\:--- \en{the derivative}\ru{производная} \en{of~}$\tangentvector$ \en{and}\ru{и} \en{the~second derivative}\ru{вторая производная} \en{of location vector}\ru{вектора положения}~${\locationvector}$\:--- \en{is always}\ru{всегда} \en{perpendicular}\ru{перпендикулярна} \en{to~}$\tangentvector$

\nopagebreak\vspace{-0.2em}\begin{equation*}
\tangentvector \dotp \hspace{.2ex} \tangentvector = 1
\hspace{.4em} \Rightarrow \hspace{.4em}
\tangentvector{\hspace{.25ex}'} \hspace{-0.4ex} \dotp \hspace{.2ex} \tangentvector = 0
\hspace{.4em} \Rightarrow \hspace{.4em}
\normalvector \hspace{.1ex} \dotp \hspace{.2ex} \tangentvector = 0
\hspace{.1ex} .
\end{equation*}

\en{Vectors}\ru{Векторы} \en{of the }\ru{системы }Frenet\hbox{--}Serret\en{ frame} \en{make}\ru{составляют} \en{an~orthonormal basis}\ru{ортонормальный базис} ${\bm{f}_{\hspace{-0.1ex}i}\hspace{.2ex}}$:

\nopagebreak\vspace{-0.2em}\begin{equation*}
\bm{f}_{\hspace{-0.1ex}1} \hspace{-.3ex} = \tangentvector\hspace{-0.2ex} ,
\hspace{.4em}
%%\\
\bm{f}_{2} \hspace{-.2ex} = \hspace{-.2ex} \normalvector
\hspace{-0.2ex} ,
\hspace{.4em}
%%\\
\bm{f}_{3} \hspace{-.2ex} = \hspace{-.2ex} \binormalvector\hspace{-0.1ex} .
\end{equation*}

\en{The location vector}\ru{Вектор положения} \en{in the }\ru{в~базисе }Frenet\hbox{--}Serret\en{ basis}

\nopagebreak\vspace{-0.2em}\begin{equation*}
\locationvector(s) \hspace{-0.2ex}
= q_{j}(s) \hspace{.1ex} \bm{f}_{\hspace{-0.2ex}j}(s) \hspace{-0.2ex}
= q_{1}(s) \tikzmark{beginTangentUnitVector} \hspace{.1ex} \bm{f}_{\hspace{-0.1ex}1}(s) \tikzmark{endTangentUnitVector} \hspace{-0.1ex} + q_{2}(s) \tikzmark{beginNormalUnitVector} \hspace{.1ex} \bm{f}_{2}(s) \tikzmark{endNormalUnitVector} \hspace{-0.1ex} + q_{3}(s) \tikzmark{beginBinormalUnitVector} \hspace{.1ex} \bm{f}_{3}(s) \tikzmark{endBinormalUnitVector}
\hspace{.1ex} .
\end{equation*}%
\AddUnderBrace[line width=.75pt][0, -0.1ex][yshift = -0.1ex]%
{beginTangentUnitVector}{endTangentUnitVector}%
{${\scalebox{.8}{$ \tangentvector $}}$}%
\AddUnderBrace[line width=.75pt][0, -0.1ex][yshift = -0.1ex]%
{beginNormalUnitVector}{endNormalUnitVector}%
{${\scalebox{.8}{$ \normalvector $}}$}%
\AddUnderBrace[line width=.75pt][0, -0.1ex][yshift = -0.1ex]%
{beginBinormalUnitVector}{endBinormalUnitVector}%
{${\scalebox{.8}{$ \binormalvector $}}$}%
\vspace{-0.5em}

\en{The tensor version}\ru{Тензорная версия} \en{of the }\ru{формул }Frenet\hbox{--}Serret\en{ formulas}

\nopagebreak\vspace{-0.1em}\begin{equation}\label{frenetserretformulas.tensorversion}
\bm{f}_{\hspace{-0.1ex}i}{'} \hspace{-0.25ex} = \hspace{-0.1ex} {^2\hspace{-0.2ex}\bm{d}} \hspace{.1ex} \dotp \hspace{-0.15ex} \bm{f}_{\hspace{-0.1ex}i}
\hspace{.2ex} .
\end{equation}

\ru{Формулы }\en{The }Frenet\hbox{--}Serret\en{ formulas}\ru{,} \en{written}\ru{написанные} \en{using}\ru{с~использованием} \en{the matrix notation}\ru{матричных обозначений}

\nopagebreak\begin{equation*}
\scalebox{.88}{$ \left[ \hspace{.1ex}
\begin{array}{c}
\tangentvector{\hspace{.25ex}'}\\
\normalvector{\hspace{.25ex}'}\\
\binormalvector{\hspace{.15ex}'}
\end{array} \hspace{.1ex} \right] $}
\hspace{-0.4ex} = \hspace{-0.4ex}
\scalebox{.88}{$ \left[ \begin{array}{ccc}
0 & \kappa & 0 \\
-\hspace{.2ex} \kappa & 0 & \hspace{.3ex} \tau \hspace{.3ex} \\
0 & -\hspace{.2ex} \tau & 0
\end{array} \hspace{.4ex} \right] $}
\hspace{-0.4ex}
\scalebox{.88}{$ \left[ \hspace{.1ex}
\begin{array}{c}
\tangentvector \\
\normalvector \\
\binormalvector
\end{array} \hspace{.1ex} \right] $}
\hspace{-0.1ex} .
\end{equation*} 

\en{Tensor}\ru{Тензор}~${^2\hspace{-0.2ex}\bm{d}}$\en{ is}\ru{\:---} \en{skew-symmetric}\ru{кососимметричный}, \en{therefore}\ru{поэтому} \en{it can be represented}\ru{он может быть представлен} \en{via}\ru{через} \en{the~companion}\ru{сопутствующий} \en{(pseudo)vector}\ru{(псевдо)вектор} (\chapdotpararef{chapter:mathapparatus}{para:tensors.symmetric+skewsymmetric}).
\en{This pseudovector}\ru{Этот псевдовектор} \en{is known as}\ru{известен как} \en{the }\ru{вектор }Darboux\en{ vector}.

\vspace{-0.2em}\begin{align*}
\bm{D} &= \tau \hspace{.2ex} \tangentvector + \kappa \hspace{.1ex} \binormalvector
\\
\bm{D} &= \tau \hspace{.2ex} \tangentvector + 0 \hspace{.1ex} \normalvector + \kappa \hspace{.1ex} \binormalvector
\end{align*}

\en{With}\ru{С~вектором} \en{the }Darboux\en{ vector}, \ru{формулы }\en{the }Frenet\hbox{--}Serret\en{ formulas} \en{turn into the following}\ru{превращаются в~следующее}:

\begin{align*}
\tangentvector{\hspace{.25ex}'} \hspace{-0.2ex} &= \hspace{-0.15ex} \bm{D} \hspace{-0.2ex} \times \tangentvector ,
\\[-0.2em]
\normalvector{\hspace{.25ex}'} \hspace{-0.2ex} &= \hspace{-0.15ex} \bm{D} \hspace{-0.2ex} \times \hspace{-0.2ex} \normalvector ,
\\[-0.2em]
\binormalvector{\hspace{.15ex}'} \hspace{-0.2ex} &= \hspace{-0.15ex} \bm{D} \hspace{-0.2ex} \times \hspace{-0.15ex} \binormalvector
%\hspace{.1ex} .
\end{align*}

\noindent
\en{or}\ru{или} \en{as}\ru{как} \en{the vector version}\ru{векторная версия} \en{of~}\eqref{frenetserretformulas.tensorversion}

\nopagebreak\vspace{-0.1em}\begin{equation}\label{frenetserretformulas.vectorversion}
\bm{f}_{\hspace{-0.1ex}i}{'} \hspace{-0.25ex} = \hspace{-0.15ex} \bm{D} \hspace{-0.2ex} \times \hspace{-0.15ex} \bm{f}_{\hspace{-0.1ex}i}
\hspace{.2ex} .
\end{equation}

\ru{Вектор }\en{The }Darboux\en{ vector} \en{is}\ru{есть} \en{the angular velocity vector}\ru{вектор угловой скорости} \ru{системы }\en{of the }Frenet--Serret\en{ frame}.

...

\en{Approximate}\ru{Приближённые} \en{applied}\ru{прикладн\'{ы}е} \en{theories of~rods}\ru{теории стержней} \en{like}\ru{вроде} \en{the }\inquotes{\en{strength of~materials}\ru{сопротивления материалов}} \en{use such concepts as}\ru{используют такие понятия как} \en{internal force}\ru{внутренняя сила}~${\hspace{-0.1ex}\internalforce\hspace{.1ex}}$ \en{and}\ru{и}~\en{internal moment}\ru{внутренний момент}~${\hspace{-0.1ex}\internalmoment}$.
\en{The following}\ru{Следующие} \en{relations}\ru{соотношения} \en{connect them}\ru{связывают их} \en{with the stress tensor}\ru{с~тензором напряжения}

\nopagebreak\vspace{.1em}\begin{align}
\internalforce \hspace{.1ex} (s) \hspace{-0.2ex}
&= \hspace{-0.3ex} \scalebox{.88}{$ \displaystyle\integral_{\rodsection} $} \hspace{-0.1ex} \tractionvector{\tangenttoaxis} \hspace{.3ex} d\rodsection
= \hspace{-0.3ex} \scalebox{.88}{$ \displaystyle\integral_{\rodsection} $} \hspace{-0.1ex} \tangenttoaxis \dotp \hspace{-0.1ex} \cauchystress \hspace{.3ex} d\rodsection
\hspace{.2ex} ,
\label{stresstensortointernalforce}
\\
%
\internalmoment \hspace{.1ex} (s) \hspace{-0.2ex}
&= \hspace{-0.3ex} \scalebox{.88}{$ \displaystyle\integral_{\rodsection} $} \hspace{-0.1ex} \bm{x} \hspace{-0.1ex} \times \tractionvector{\tangenttoaxis} \hspace{.3ex} d\rodsection
= \hspace{-0.3ex} \scalebox{.88}{$ \displaystyle\integral_{\rodsection} $} \hspace{-0.1ex} \bm{x} \hspace{-0.1ex} \times \tangenttoaxis \hspace{.1ex} \dotp \hspace{-0.1ex} \cauchystress \hspace{.3ex} d\rodsection
\hspace{.2ex} .
\label{stresstensortointernalmoment}
\end{align}

\begin{equation*}
\tangenttoaxis \equiv \thirdbasisvector
\hspace{.1ex} , \hspace{.6em}
\bm{x} = x_{\alpha} \bm{e}_\alpha
\hspace{.1ex} , \hspace{.6em}
\alpha \hspace{-0.1ex} = \hspace{-0.1ex} 1, 2
\end{equation*}

...

\en{These thoughts}\ru{Эти мысли} \en{about geometry}\ru{о~геометрии} \en{and }\ru{и~}\en{about mechanics}\ru{о~механике}\:--- \en{in particular}\ru{в~частности}, \en{about internal force}\ru{о~внутренней силе}~\eqref{stresstensortointernalforce} \en{and}\ru{и}~\en{internal moment}\ru{внутреннем моменте}~\eqref{stresstensortointernalmoment}\:--- \en{concern only}\ru{касаются только} \en{some unique single configuration}\ru{какой-то уникальной единственной конфигурации} \en{of a~rod}\ru{стержня}.
\en{It’s meaningless to continue these thoughts}\ru{Продолжать эти мысли бессмысленно}, \en{because}\ru{потому что} \en{in reality}\ru{в~реальности} \en{plane and normal sections}\ru{плоские и~нормальные сечения} \en{do~not remain}\ru{не~остаются} \en{plane and normal}\ru{плоскими и~нормальными} \en{after deforming}\ru{после деформирования}.

%%
\begin{tcolorbox}[enhanced, colback = cyan!50, before upper={\parindent2ex}, parbox = false]
\small%
\setlength{\abovedisplayskip}{2pt}\setlength{\belowdisplayskip}{2pt}%

\begin{center}
\emph{synonyms for \inquotes{deplanations}}
\vspace{-1em}
\end{center}

\begin{otherlanguage}{russian}

warp, warping = деформация (deform, deformation, strain), искривление, искажение (distort, distortion), перекос (skew), перекосить, перекашиваться, коробиться, покоробить, коробление, out of shape, become twisted or bent

\end{otherlanguage}

\end{tcolorbox}

\ru{А}\en{And}~\en{the~addition}\ru{добавление} \en{of some assumptions-hypotheses}\ru{некоторых предположений-гипотез} \en{to the model}\ru{в~модель}, \en{like}\ru{подобных} \inquotes{\en{there are no warping (deplanations)}\ru{искривления (депланаций) нет}} \en{and}\ru{и}~\en{initially plane sections remain plane}\ru{первоначально плоские сечения остаются плоскими}\footnote{%
\en{The two very popular beam models exist}\ru{Существуют две очень популярные модели балки}\ru{,} \en{which}\ru{которые} \en{postulate}\ru{постулируют} \en{the hypothesis}\ru{гипотезу} \en{about the~absence of~deplanations}\ru{об отсутствии депланаций}.
\en{In the }\ru{В~теории балок }Euler\ru{’а}\hbox{--}Bernoulli\en{ beam theory}, \en{shear deformations are neglected}\ru{сдвиговыми деформациями пренебрегают}, \en{plane sections remain plane}\ru{плоские сечения остаются плоскими} \en{and}\ru{и}~\en{perpendicular to the~axis}\ru{перпендикулярными оси}.
\en{In the }\ru{В~теории балок }Timoshenko\en{ beam theory} \en{there’s}\ru{имеется} \en{a~constant}\ru{постоянный} \en{transverse shear}\ru{поперечный сдвиг} \en{along the section}\ru{вдоль сечения}, \en{so}\ru{так что} \en{plane sections still remain plane}\ru{плоские сечения всё ещё остаются плоскими}, \en{but they}\ru{но они} \en{are no longer perpendicular to the~axis}\ru{больше не перпендикулярны оси}.}\hspace{-0.3ex}
%
\en{introduces}\ru{вносит} \en{essential}\ru{существенные} \en{contradictions}\ru{противоречия} \en{with reality}\ru{с~реальностью}.
\en{Enough to recall}\ru{Достаточно вспомнить} \en{just one fact that}\ru{лишь один факт, что} \en{without}\ru{без} \en{deplanations}\ru{депланаций} \en{it’s impossible}\ru{невозможно} \en{to acceptably describe}\ru{приемлемо описать} \en{the torsion of a~rod}\ru{кручение стержня} (\en{and not only torsion}\ru{и~не~только кручение}).

%%warping
%%Warping due to shear noticeably complicates the behaviour of a~rod.

\en{The very reasonable}\ru{Очень резонный} \en{approach}\ru{подход} \en{to}\ru{к}~\en{modeling}\ru{моделированию} \en{deformations}\ru{деформаций} \en{of an~elastic rod}\ru{упругого стержня} \en{consists in}\ru{состоит в} \en{asymptotic splitting}\ru{асимптотическом расщеплении} \en{of the three-dimensional problem}\ru{трёхмерной проблемы} \en{with }\ru{с~}\en{a~small thickness}\ru{м\'{а}лой толщиной}.
\en{But}\ru{Но} \en{for}\ru{для} \en{a~complex asymptotic procedure}\ru{сложной асимптотической процедуры} \en{it would be much simpler}\ru{было бы намного проще} \en{to have}\ru{иметь} \en{whatever}\ru{какую-нибудь} \en{solution version}\ru{версию решения} \en{aforehand}\ru{заранее}.
\en{And}\ru{И} \en{the direct approach}\ru{прямой подход}, \en{when}\ru{когда} \en{the one\hbox{-}dimensional model}\ru{одномерная модель} \en{of a~rod}\ru{стержня}\en{ is}\ru{\:---} \en{a~material line}\ru{материальная линия}, \en{gives}\ru{даёт} \en{such a~version}\ru{такую версию}.

\en{The primary question}\ru{Первичный вопрос} \en{for building}\ru{для построения} \en{the one\hbox{-}dimensional model}\ru{одномерной модели}:
\en{what}\ru{какими} \en{degrees of~freedom}\ru{степенями свободы}\:--- \en{besides translation}\ru{помимо трансляции}\:--- \en{do}\ru{обладают} \en{the particles}\ru{частицы} \en{of a~material line}\ru{материальной линии}\en{ possess}?

\en{It is known that}\ru{Известно, что} \en{rods}\ru{стержни} \en{are sensitive}\ru{чувствительны} \en{to the moment loads}\ru{к~моментным нагрузкам}.
\en{And }\ru{А~}\en{the presence of moments}\ru{присутствие моментов} \en{among}\ru{среди} \en{generalized forces}\ru{обобщённых сил} \en{indicates}\ru{показывает} \en{the presence}\ru{наличие} \en{of rotational degrees of freedom}\ru{вращательных степеней свободы}.
\en{Therefore}\ru{Поэтому} \en{as}\ru{как} \en{the one-dimensional model}\ru{одномерную модель} \en{of~a~rod}\ru{стержня} \en{it is reasonable to take}\ru{разумно взять} \en{the }\ru{линию }Cosserat\en{ line}\:--- \en{it}\ru{она} \en{consists of}\ru{состоит из} \en{infinitesimal}\ru{бесконечно-малых} \en{absolutely rigid bodies}\ru{абсолютно жёстких тел}.
\en{However}\ru{Впрочем}, \en{another new}\ru{другие новые} \en{degrees of~freedom}\ru{степени свободы} \en{may also appear}\ru{могут тоже появиться}\:--- \en{as for thin-walled rods}\ru{как для тонкостенных стержней}, \en{described}\ru{описанных} \en{in the dedicated chapter}\ru{в~отдельной главе}~(\chapref{chapter:thinwalledrods}).

\en{In the mechanics}\ru{В~механике} \en{of continuous elastic media}\ru{сплошных упругих сред}\en{,} \en{the place of~rods}\ru{место стержней} \en{is specific}\ru{специфическое}.
\en{First}\ru{Во\hbox{-}первых}, \en{moments}\ru{моменты} \en{play}\ru{играют} \ru{здесь }\en{the main role}\ru{главную роль}\en{ here}, \en{and not the role}\ru{а~не~роль} \en{of small additions}\ru{малых добавок} \en{as in a~three-dimensional}\ru{как в~трёхмерном контину\kern-0.11exуме} Cosserat\en{ continuum}.
\en{Second}\ru{Во\hbox{-}вторых}, \en{rods}\ru{стержни} \en{can be used to test}\ru{могут быть использованы для тестирования} \en{models}\ru{моделей} \en{with additional degrees of freedom}\ru{с~дополнительными степенями свободы}\:--- \en{and}\ru{и}~\en{before}\ru{прежде, чем} \en{the presence}\ru{наличие} \en{of these degrees}\ru{этих степеней} \en{will be researched}\ru{будет исследовано} \en{on a~three-dimensional model}\ru{на трёхмерной модели}.

\en{The following section}\ru{Следующий раздел} \en{presents and describes}\ru{представляет и~описывает} \en{the~simple}\ru{простую} \en{one-dimensional}\ru{одномерную} \en{Cosserat-like moment model}\ru{моментную модель типа Cosserat}.

\en{\section{Kinematics of Cosserat lines}} % Cosserat curves

\ru{\section{Кинематика линий Коссера}} % кривых Коссера

\label{para:rods.cosseratlines}

\en{Model described further}\ru{Модель, описанная далее}\en{ is}\ru{\:---} \en{a~simplified version of}\ru{упрощённая версия}~\chapref{chapter:cosseratcontinuum}.

\en{There’s no more a~triple}\ru{Больше нет тройки} \en{of material coordinates}\ru{материальных координат}~${q_{\hspace{.1ex}i}}$, \en{but}\ru{но} \en{the only one}\ru{лишь одна}\:--- $s$.
\en{It may be}\ru{Это может быть} \en{the arc length parameter}\ru{параметр длины дуги} \en{in the initial configuration}\ru{в~начальной конфигурации}.
\en{The motion}\ru{Движение} \en{of a~particle}\ru{частицы} \en{over time}\ru{со~временем} \en{is described}\ru{описывается} \en{by the location vector}\ru{вектором положения}~${\locationvector(s,t)}$ \en{and}\ru{и} \en{the rotation tensor}\ru{тензором поворота}~${\rotationtensor(s,t)}$.
\en{Linear}\ru{Линейная} \en{and}\ru{и}~\en{angular}\ru{угловая}~\eqrefwithchapdotpara{angularvelocityvector}{chapter:mathapparatus}{para:rotationtensor} \en{velocities}\ru{скорости} \en{of a~rod’s particle}\ru{частицы стержня} \en{are}\ru{суть}

\nopagebreak\vspace{-0.2em}\begin{gather}
\mathdotabove{\locationvector} = \bm{v}
\hspace{.1ex} , %\hspace{.5em}
\label{cosseratrod.linearvelocity}
\\[-0.2em]
%
\mathdotabove{\rotationtensor} \hspace{-0.1ex} = \bm{\omega} \times \hspace{-0.1ex} \rotationtensor
\hspace{.3em} \Leftrightarrow \hspace{.3em}
\bm{\omega} \hspace{-0.1ex} = \hspace{-0.1ex} - \hspace{.25ex} \smalldisplaystyleonehalf \scalebox{.96}{$ \bigl( \mathdotabove{\rotationtensor} \hspace{-0.1ex} \dotp \rotationtensor^{\T} \hspace{.1ex} \bigr)_{\hspace{-0.3ex}\Xcompanion} $}
\hspace{.1ex} .
\label{cosseratrod.angularvelocity}
\end{gather}

\en{Deformation of a~rod}\ru{Деформация стержня} \en{as}\ru{как} \en{a~Cosserat line}\ru{линии Cosserat} \en{is defined}\ru{определяется} \en{by two vectors}\ru{двумя векторами}

\nopagebreak\vspace{-0.2em}\begin{gather}
\firstdeformationvectorcosseratline \hspace{-0.1ex} \equiv \currentlocationvector{\hspace{.1ex}'} \hspace{-0.3ex} - \rotationtensor \hspace{-0.1ex} \dotp \hspace{.15ex} \initiallocationvector{\hspace{.1ex}'}
\hspace{-0.25ex} ,
\label{cosseratrod.firstdeformationvector}
\\[.1em]
%
\seconddeformationvectorcosseratline \hspace{-0.1ex} \equiv \hspace{-0.1ex} - \hspace{.25ex} \smalldisplaystyleonehalf \scalebox{.96}{$ \bigl( \rotationtensor{\hspace{.15ex}'} \hspace{-0.3ex} \dotp \rotationtensor^{\T} \hspace{.1ex} \bigr)_{\hspace{-0.3ex}\Xcompanion} $}
\hspace{.25em} \Leftrightarrow \hspace{.3em}
\rotationtensor{\hspace{.15ex}'} \hspace{-0.33ex} = \seconddeformationvectorcosseratline \times \hspace{-0.1ex} \rotationtensor
\hspace{.1ex} .
\label{cosseratrod.seconddeformationvector}
\end{gather}

\begin{equation*}
\Bigl( \hspace{.2em}
\initiallocationvector(s) \hspace{-0.1ex}
\equiv
\currentlocationvector(s, 0)
\hspace{.2em} \Bigr)
\vspace{-0.2em}
\end{equation*}

\begin{equation*}
\| \bm{e}_{3} \| = \| \mathcircabove{\bm{e}}_{3} \| = 1 = \constant
\end{equation*}

\noindent
\eqref{cosseratrod.firstdeformationvector} \en{and}\ru{и}~\eqref{cosseratrod.seconddeformationvector} \en{are really}\ru{реально} \en{the deformation vectors}\ru{векторы деформации}, \en{it follows}\ru{это следует} \en{from their equality}\ru{из равенства их} \en{to zero}\ru{нулю} \en{on displacements}\ru{на перемещениях} \en{of~a~body}\ru{тела} \en{as a~rigid whole}\ru{как жёсткого целого} \textcolor{magenta}{(.... add some equation(s) here describing displacements as a rigid whole .....)}.

\en{Next}\ru{Дальше} \en{we will clarify}\ru{мы проясним} \en{the idea}\ru{идею} \en{of the first deformation vector}\ru{первого вектора деформации}~$\firstdeformationvectorcosseratline$.
\en{Without loss of universality}\ru{Без потери универсальности}, \en{the parameter}\ru{параметр}~$s$ \en{is}\ru{есть} \en{the initial arc length}\ru{начальная длина дуги} \en{and}\ru{и} \en{the initial basis vector}\ru{начальный базисный вектор}~${\mathcircabove{\bm{e}}_{3}}$ \en{is directed}\ru{направлен} \en{along the tangent}\ru{вдоль касательной} \en{in the initial configuration}\ru{в~начальной конфигурации}: ${ \mathcircabove{\bm{e}}_{3} = \initiallocationvector{\hspace{.1ex}'} }$.
\en{Then}\ru{Тогда}

\nopagebreak\vspace{-0.3em}\begin{multline}
\label{cosseratrod.firstdeformationvectorexplained}
%
\firstdeformationvectorcosseratline \hspace{-0.1ex} \equiv \currentlocationvector{\hspace{.1ex}'} \hspace{-0.3ex} - \rotationtensor \hspace{-0.1ex} \dotp \hspace{.15ex} \initiallocationvector{\hspace{.1ex}'} \hspace{-0.4ex}
,
\hspace{.5em}
\initiallocationvector{\hspace{.1ex}'} \hspace{-0.2ex} = \mathcircabove{\bm{e}}_{3}
\hspace{.2ex} ,
\hspace{.5em}
\rotationtensor \hspace{-0.1ex} \dotp \hspace{.15ex} \initiallocationvector{\hspace{.1ex}'} \hspace{-0.2ex}
= \rotationtensor \hspace{-0.1ex} \dotp \hspace{.15ex} \mathcircabove{\bm{e}}_{3} \hspace{-0.2ex}
= \bm{e}_{3}
%
%%\hspace{.4em} \Rightarrow
\\[-0.25em]
\Rightarrow \hspace{.4em}
%
\firstdeformationvectorcosseratline \hspace{-0.1ex} = \currentlocationvector{\hspace{.1ex}'} \hspace{-0.3ex} - \bm{e}_{3}
\hspace{.2ex} ,
\end{multline}

\nopagebreak\vspace{-0.5em}\begin{multline*}
\currentlocationvector{\hspace{.1ex}'} \hspace{-0.3ex} = \firstdeformationvectorcosseratline \hspace{-0.1ex} + \bm{e}_{3}
\hspace{.2ex} ,
\hspace{.5em}
\| \currentlocationvector{\hspace{.1ex}'} \|^{2} \hspace{-0.3ex}
= \currentlocationvector{\hspace{.1ex}'} \hspace{-0.4ex} \dotp \hspace{.1ex} \currentlocationvector{\hspace{.1ex}'} \hspace{-0.4ex}
= \bigl( \firstdeformationvectorcosseratline \hspace{-0.1ex} + \bm{e}_{3} \bigr) \hspace{-0.3ex} \dotp \hspace{-0.2ex} \bigl( \firstdeformationvectorcosseratline \hspace{-0.1ex} + \bm{e}_{3} \bigr)
\\
%
= \firstdeformationvectorcosseratline \hspace{-0.2ex} \dotp \firstdeformationvectorcosseratline + 2 \hspace{.2ex} \firstdeformationvectorcosseratline \dotp \bm{e}_{3} \hspace{-0.2ex} + \bm{e}_{3} \hspace{-0.2ex} \dotp \bm{e}_{3}
\hspace{.2ex} ,
\end{multline*}

\nopagebreak\vspace{-0.2em}\begin{equation*}
\| \currentlocationvector{\hspace{.1ex}'} \|
= \displaystyle\sqrt{ \| \firstdeformationvectorcosseratline \|^{2} \hspace{-0.2ex} + 2 \hspace{.2ex} \firstdeformationvectorcosseratlinecomponents{3} \hspace{-0.2ex} + 1 \hspace{.2ex} }
\hspace{.1ex} ,
\end{equation*}

\nopagebreak\begin{equation}
\label{cosseratrod.firstdeformationvectordescribesextension}
%
\| \currentlocationvector{\hspace{.1ex}'} \| - 1
= \displaystyle\sqrt{ \| \firstdeformationvectorcosseratline \|^{2} \hspace{-0.2ex} + 2 \hspace{.2ex} \firstdeformationvectorcosseratlinecomponents{3} \hspace{-0.1ex} + 1 \hspace{.2ex} } - 1
= \hspace{.1ex} \firstdeformationvectorcosseratlinecomponents{3} \hspace{-0.1ex} + \infty^{\hspace{-0.3ex}\expminusone} \hspace{-0.2ex} \bigl( \| \firstdeformationvectorcosseratline \|^{2} \bigr)
\hspace{.2ex} .
\end{equation}

\noindent
\en{Equality}\ru{Равенство}~\eqref{cosseratrod.firstdeformationvectordescribesextension} \en{describes}\ru{описывает} \en{a~relative elongation}\ru{относительное удлинение}.
\en{Roughly speaking}\ru{Грубо говоря}, \en{component}\ru{компоненту}~${\firstdeformationvectorcosseratlinecomponents{3} \hspace{-0.2ex} \equiv \firstdeformationvectorcosseratline \dotp \bm{e}_{3} \hspace{-0.2ex}}$ \en{can be considered an~elongation}\ru{можно считать удлинением}, \en{and}\ru{а}~\en{components}\ru{компоненты}~${\firstdeformationvectorcosseratlinecomponents{1} \hspace{-0.2ex} \equiv \firstdeformationvectorcosseratline \dotp \bm{e}_{1}}$, ${\firstdeformationvectorcosseratlinecomponents{2} \hspace{-0.2ex} \equiv \firstdeformationvectorcosseratline \dotp \bm{e}_{2} \hspace{-0.2ex}}$ \en{present}\ru{представляют} \en{a~transverse shear}\ru{поперечный сдвиг}.
\en{It’s more accurate}\ru{Более точно} \en{to rely}\ru{полагаться} \en{on formulas}\ru{на формулы}~\eqref{cosseratrod.firstdeformationvectorexplained} \en{and}\ru{и}~\eqref{cosseratrod.firstdeformationvectordescribesextension}.

......

\en{The~Cosserat-like model of~a~rod}\ru{В~модели стержня типа Cosserat} \en{doesn’t have a~section as a~plane figure}\ru{нет сечения как плоской фигуры}.

....

%%\begin{otherlanguage}{russian}
%%\end{otherlanguage}

\en{\section{Balance of forces and moments}}

\ru{\section{Баланс сил и моментов}}

\label{para:rods.forcesandmomentsbalance}

\en{Possible loads}\ru{Возможные нагрузки}\ru{,} \en{acting}\ru{действующие} \en{on a~rod}\ru{на стержень} \en{as}\ru{как} \en{the~}\ru{линию }Cosserat\en{ line}\en{ are}\ru{\:---} \en{forces}\ru{\hbox{силы}} \en{and}\ru{и}~\en{moments}\ru{моменты}: \en{on~infinitesimal element}\ru{на~бесконечно-малый элемент}~$ds$ \en{of a~rod}\ru{стержня} \en{act}\ru{действуют} \en{external force}\ru{внешняя сила}~${\bm{q}ds}$ \en{and}\ru{и}~\en{external moment}\ru{внешний момент}~${\bm{m}ds}$.
\en{The internal interactions}\ru{Внутренними взаимо\-действиями} \en{will be}\ru{будут} \en{force}\ru{сила}~${\internalforce(s)}$ \en{and }\ru{и~}\en{moment}\ru{момент}~${\internalmoment(s)}$\:--- \en{this is the action}\ru{это действие} \en{of the particle}\ru{частицы} \en{with coordinate}\ru{с~координатой}~${s\!+\!0}$ \en{on the particle}\ru{на частицу} \en{with}\ru{с}~${s\!-\!0}$.
\en{The~action--re\-action principle}\ru{Принцип действия--про\-ти\-во\-действия} \en{gives that}\ru{даёт, что} 
\en{a~reverse}\ru{реверс} (\en{a~change of direction}\ru{перемена направления}) \en{of~}\en{coordinate}\ru{координаты}~$s$ \en{changes signs}\ru{меняет знаки} \en{of~}${\internalforce}$ \en{and}\ru{и}~${\internalmoment}$.

...

%%\begin{otherlanguage}{russian}
%%\end{otherlanguage}

\en{\section{Principle of virtual work and consequences}}

\ru{\section{Принцип виртуальной работы и следствия}}

\label{para:rods.principleofvirtualwork}

\noindent
\en{For a~piece of~rod}\ru{Для куск\'{а} стержня} ${s_0 \hspace{.2em} \scalebox{.9}{$\leq$} \hspace{.18em} s \hspace{.22em} \scalebox{.9}{$\leq$} \hspace{.2em} s_1}$ \en{formulation of the~principle is as~follows}\ru{формулировка принципа таков\'{а}}

......

\en{Conventionally}\ru{Условно}
$\bm{a}$ \en{is}\ru{есть} \en{the~tensor of~stiffness for bending and~twisting}\ru{тензор жёсткости на~изгиб и~кручение},
$\bm{b}$\en{ is}\ru{\:---} \en{the~tensor of~stiffness for (ex)tension and~shear}\ru{тензор жёсткости на~растяжение и~сдвиг},
\en{and}\ru{а}~$\bm{c}$\en{ is}\ru{\:---} \en{the~tensor of~crosslinks}\ru{тензор перекрёстных связей}.

\en{Stiffness tensors}\ru{Тензоры жёсткости} \en{rotate}\ru{поворачиваются} \en{along}\ru{вместе} \en{with a~particle}\ru{с~частицей}:

...

%%\begin{otherlanguage}{russian}
%%\end{otherlanguage}

\en{\section{Classical Kirchhoff’s model}}

\ru{\section{Классическая модель Kirchhoff’а}}

\label{para:rods.modelofkirchhoff}

\en{It is also called}\ru{Её ещё называют} \en{the }\emph{\en{Kirchhoff’s rod theory}\ru{теорией стержней Kirchhoff’а}}.

\en{Until now}\ru{До~сих~пор} \en{functions}\ru{функции} ${\locationvector(s,t)}$ \en{and}\ru{и}~${\rotationtensor(s,t)}$ \en{were independent}\ru{были независимы}.
\en{The }\ru{Классическая теория }Kirchhoff\en{’s}\ru{’а}\en{ classical theory} \en{postulates}\ru{постулирует} \en{the internal constraint}\ru{внутреннюю связь}

\nopagebreak\vspace{-0.1em}\begin{equation}\label{kirchhoffmodelinternalconstraint}
\firstdeformationvectorcosseratline \hspace{-0.1ex} = \bm{0}
\hspace{.4em} \Leftrightarrow \hspace{.4em}
\currentlocationvector_\differentialindex{s} \hspace{-0.2ex} = \rotationtensor \hspace{-0.1ex} \dotp \hspace{.1ex} \initiallocationvector_\differentialindex{s} \hspace{-0.2ex}
\hspace{.5em} \text{\en{or}\ru{или}} \hspace{.5em}
\currentlocationvector{\hspace{.1ex}'} \hspace{-0.33ex} = \rotationtensor \hspace{-0.1ex} \dotp \hspace{.1ex} \initiallocationvector{\hspace{.1ex}'}
\hspace{-0.33ex} .
\end{equation}

\vspace{-0.2em}\noindent
\en{Having}\ru{Имея} \en{the idea of vector}\ru{идею вектора}~$\firstdeformationvectorcosseratline$ \eqref{cosseratrod.firstdeformationvectorexplained}, \en{here we can tell that}\ru{здесь мы можем сказать, что}:
(1)~\en{a~rod is non-extensible}\ru{стержень нерастяжим}, (2)~\en{there are no transverse shears}\ru{поперечных сдвигов нет}.

\en{If}\ru{Если} \en{basis vector}\ru{базисный вектор}~${\mathcircabove{\bm{e}}_3}$ \en{was directed}\ru{был направлен} \en{along}\ru{вдоль} \en{the tangent to the axis}\ru{касательной к~оси} \en{in the initial configuration}\ru{в~начальной конфигурации}, \en{then}\ru{то} \en{it}\ru{он} \en{will remain}\ru{будет оставаться} \en{on the tangent}\ru{на~касательной} \en{also}\ru{также} \en{after deforming}\ru{после деформирования}.
\en{The rod particles}\ru{Частицы стержня} \en{rotate}\ru{вращаются} \en{only together}\ru{лишь вместе} \en{with }\ru{с~} \en{the tangent to the axis}\ru{касательной к~оси} \en{and around it}\ru{и~вокруг неё}.

\begin{otherlanguage}{russian}

Уравнения баланса сил и~моментов (импульса и~момента импульса) не~меняются от введения связи~\eqref{kirchhoffmodelinternalconstraint}.
Но локальное вариационное соотношение (...) становится короче:

...

\end{otherlanguage}

\en{\section{Euler’s problem about stability of rods}}

\ru{\section{Проблема Euler’а об устойчивости стержней}}

\label{para:rods.eulerstabilityproblem}

\begin{otherlanguage}{russian}

Рассматривается прямой стержень, защемлённый на одном конце и~нагруженный силой~$\bm{P}$ на~другом (рисунок ?? 123 ??).
Сила \inquotes{мёртвая}~(не~меняется в~процессе деформирования)

...

\end{otherlanguage}

\en{\section{Variational equations}} % Equations in variations

\ru{\section{Вариационные уравнения}} % Уравнения в вариациях

\label{para:rods.equationsinvariations}

\en{In the nonlinear mechanics of elastic media}\ru{В~нелинейной механике упругих сред} \ru{полезны }\en{variational equations}\ru{вариационные уравнения}\en{ are useful}, \en{which describe}\ru{которые описывают} \en{a~small change}\ru{малое изменение} \en{in the current configuration}\ru{текущей конфигурации}~(\chapdotpararef{chapter:nonlinearcontinuum}{para:variationofconfiguration}).

\en{Variating}\ru{Варьируя} \en{equations}\ru{уравнения} \en{of the complete system}\ru{полной системы} \en{of the }\ru{модели }Cosserat\en{ model}, \en{we get}\ru{мы получаем}

\nopagebreak\vspace{-0.2em}\begin{gather*}
\variation{\hspace{.1ex}\internalforce}{\hspace{.25ex}'} \hspace{-0.3ex} + \variation{\bm{q}}
= \hspace{-0.1ex} \rho \hspace{.15ex} \bigl( \fieldofdisplacements + \varbivalent{\hspace{-0.2ex}\cauchystress} \hspace{-0.2ex} \times \hspace{-0.3ex} \infinitesimaldeformation \bigr)^{ \hspace{-0.1ex} \tikz[baseline=-0.2ex] \draw[black, fill=black] (0,0) circle (.28ex); \hspace{.25ex} \tikz[baseline=-0.2ex] \draw[black, fill=black] (0,0) circle (.28ex); }
\hspace{-0.2ex} ,
\\
%
\variation{\hspace{.1ex}\internalmoment}{\hspace{.25ex}'} \hspace{-0.3ex} + \fieldofdisplacements{'} \hspace{-0.3ex} \times \hspace{-0.2ex} \internalforce + \locationvector{\hspace{.1ex}'} \hspace{-0.3ex} \times \hspace{-0.2ex} \variation{\hspace{.1ex}\internalforce} + \variation{\internaldistributedmoment}
= \ldots
\end{gather*}

...

%%\begin{otherlanguage}{russian}
%%\end{otherlanguage}

\en{\section{Non-shear model with (ex)tension}}

\ru{\section{Модель без сдвига с растяжением}}

\label{para:rods.nonshearmodelwithextension}

\en{Kirchhoff’s model}\ru{Модель Kirchhoff’а} \en{with internal constraint}\ru{с~внутренней связью}~${\firstdeformationvectorcosseratline \hspace{-0.25ex} = \hspace{-0.2ex} \bm{0}}$~\eqref{kirchhoffmodelinternalconstraint} \en{doesn’t describe}\ru{не~описывает} \en{the~simplest case}\ru{простейший случай} \en{of extension/compression}\ru{растяжения/сжатия} \en{of a~straight rod}\ru{прямого стержня}.
\en{This nuisance}\ru{Эта неприятность} \en{disappears}\ru{исчезает} \en{with}\ru{со}~\inquotes{\en{softening}\ru{смягчением}} \en{of the~constraint}\ru{связи}, \en{for example}\ru{например} \en{adding}\ru{добавляя} \en{the~possibility of (ex)tension}\ru{возможность растяжения} \en{and}\ru{и} \en{inhibiting only the transverse shear}\ru{подавляя лишь поперечный сдвиг}

\nopagebreak\vspace{-0.1em}\begin{equation}\label{kirchhoffinternalconstraintwithextension}
\firstdeformationvectorcosseratline \hspace{-0.1ex} = \Gamma \bm{e}_3 \hspace{-0.2ex}
\hspace{.4em} \Leftrightarrow \hspace{.4em}
\Gamma_{\hspace{-0.2ex}\alpha} \hspace{-0.2ex} = 0
\end{equation}

...

%%\begin{otherlanguage}{russian}
%%\end{otherlanguage}

\en{\section{Mechanics of flexible thread}}

\ru{\section{Механика гибкой нити}}

\label{para:rods.flexiblethreadmechanics}

\en{A~thread}\ru{Нить}\en{ is}\ru{\:---} \ru{это} \en{a~momentless rod}\ru{безмоментный стержень}.

\begin{otherlanguage}{russian}

\en{A~flexible thread~(chain)}\ru{Гибкая нить~(цепь)} \en{is simpler than a~rod}\ru{проще стержня}, \en{because}\ru{потому что} \en{its particles}\ru{её частицы} \en{are}\ru{суть} \inquotes{\ru{простые}\en{simple}} \en{material points}\ru{материальные точки} \en{with only translational degrees of~freedom}\ru{с~лишь трансляционными степенями свободы}.
\en{Therefore}\ru{Поэтому} среди нагрузок нет моментов, только \inquotes{линейные} силы\:--- внешние распределённые~$\bm{q}$ и~внутренние сосредоточенные~$\bm{Q}$.
Движение нити полностью определяется одним вектором-радиусом~${\locationvector(s,t)}$, а~инерционные свойства\:--- линейной плотностью~${\rho\hspace{.1ex}(s)}$.
% linear density

Вот принцип виртуальной работы для куск\'{а} нити ${s_0 \leq s \leq s_1}$

\nopagebreak\vspace{-0.2em}\begin{equation}
\scalebox{.95}{$
\displaystyle \integral\displaylimits_{\raisemath{.05em}{\mathclap{s_0}}}^{\raisemath{.15em}{\mathclap{s_1}}}
$}
\hspace{-0.25ex} \Bigl( \hspace{-0.1ex}
\bigl( \bm{q} - \hspace{-0.12ex} \rho \hspace{.2ex} \mathdotdotabove{\locationvector} \hspace{.36ex} \bigr)
\hspace{-0.25ex} \dotp \variation{\locationvector}
- \variation{\potential}
\Bigr) ds
\hspace{.1ex} + \hspace{-0.1ex} \Bigl[ \hspace{.1ex}
\bm{Q} \hspace{-0.1ex} \dotp \variation{\locationvector}
\hspace{.15ex} \Bigr]_{\hspace{-0.25ex}s_0}^{\hspace{-0.25ex}s_1}
\hspace{-0.33ex} = 0
\hspace{.1ex} .
\end{equation}

...

Механика нити детально описана в~книге~\cite{merkin-threadmechanics}.

\end{otherlanguage}

\en{\section{Linear theory}}

\ru{\section{Линейная теория}}

\label{para:rods-lineartheory}

\begin{otherlanguage}{russian}

В~линейной теории внешние воздействия считаются малыми, а~отсчётная конфигурация\:--- ненапряжённым состоянием покоя.
Уравнения в~вариациях в~этом случае дают

...



\end{otherlanguage}

\en{\section{Case of small thickness}}

\ru{\section{Случай малой толщины}}

\label{para:rods.smallthickness}

\begin{otherlanguage}{russian}

Когда относительная толщина стержня мал\'{а}, тогда модель типа Cosserat уступает место классической.
Понятие \inquotes{толщина} определяется соотношением жёсткостей: $\bm{a}$, $\bm{b}$ \en{and}\ru{и}~$\bm{c}$\:--- разных единиц измерения; полагая ${\bm{a} = \hcursive^{\hspace{-0.25ex}2} \hspace{.2ex} \widearc{\bm{a}}}$ \en{and}\ru{и}~${\bm{c} = \hcursive \widearc{\bm{c}}}$, \en{where}\ru{где}~$\hcursive$\en{ is}\ru{\:---} \en{some}\ru{некий} \en{scope}\ru{диапазон} \en{of~length}\ru{длины}, получим тензоры ${\widearc{\bm{a}}}$, $\bm{b}$ \en{and}\ru{и}~${\widearc{\bm{c}}\hspace{.2ex}}$ с~одной и~той~же единицей.
Подбирая $\hcursive$ так, чтобы сблизились характерные значения тензоров ${\widearc{\bm{a}}}$, $\bm{b}$ \en{and}\ru{и}~${\widearc{\bm{c}}}$, найдём \en{equivalent thickness}\ru{эквивалентную толщину}~$h$ стержня (для реальных трёхмерных стержней $h$ где-то на~уровне диаметра сечения).

Представив $\internalforce$ \en{and}\ru{и}~$\internalmoment$ через векторы бесконечномалой линейной деформации

...


Переход модели типа Cosserat в~классическую кажется более очевидным, если непосредственно интегрировать уравнения~(...)

...



\end{otherlanguage}

\en{\section{Saint\hbox{-\hspace{-0.2ex}}Venant’s problem}}

\ru{\section{Задача Сэйнт\hbox{-}Венана}}

\label{para:rods.problemofsaint-venant}

\begin{otherlanguage}{russian}

Трудно переоценить ту роль, которую играет в~механике стержней классическое решение Saint\hbox{-\hspace{-0.2ex}}Venant\ru{’а}.
О~нём уж\'{е} шла речь в~\chapdotpararef{chapter:linearclassicalelasticity}{para:twistingofrods.saintvenant}.

Вместо условий ...

...



\end{otherlanguage}

\en{\section{Finding stiffness by energy}}

\ru{\section{Нахождение жёсткости по энергии}}

\label{para:rods.stiffnessbyenergy}

\begin{otherlanguage}{russian}

Для определения тензоров жёсткости~$\bm{a}$, $\bm{b}$ и~$\bm{c}$ одномерной модели достаточно решений трёхмерных задач для стержня.
Но тут возникают два вопроса: какие именно задачи рассматривать и что нужно взять из решений?

\ru{Проблема }Saint\hbox{-\hspace{-0.2ex}}Venant’\en{s}\ru{а}\en{ problem} выделяется среди прочих, ведь оттуда берётся жёсткость на~кручение.

Вдобавок есть много точных решений, получаемых таким путём: задаётся поле~${\bm{u}(\locationvector)}$, определяется~${\widearctoo{\cauchystress} = \stiffnesstensorC \dotdotp \hspace{-0.15ex} \boldnabla \bm{u}}$, затем находятся объёмные~${\bm{f} \hspace{-0.1ex} = - \hspace{.1ex} \boldnabla \dotp \widearctoo{\cauchystress}\hspace{.1ex}}$ и~поверхностные~${\bm{p} = \bm{n} \dotp \widearctoo{\cauchystress}\hspace{.1ex}}$ нагрузки.

Но что делать с~решением?
Ясно, что $\internalforce$ \en{and}\ru{и}~$\internalmoment$ в~стержне\:--- это интегралы по~сечению~(...).
И совсем не~ясно, что считать перемещением и~поворотом в~одномерной модели.
Если предложить, например, такой вариант (индекс у~$\bm{u}$\:--- размерность модели)

\nopagebreak\vspace{-0.1em}\begin{equation*}
\bm{u}_{\raisemath{-0.2ex}{1}}\hspace{-0.15ex}(z) \hspace{-0.1ex} = \hspace{.1ex} \rodsectionexpminusone \hspace{-0.4ex} \integral\displaylimits_{\rodsection} \hspace{-0.5ex} \bm{u}_{\raisemath{-0.2ex}{3}}\hspace{-0.05ex}(\bm{x},z) \hspace{.25ex} d\rodsection
\hspace{.15ex} ,
\:\;
\bm{\theta}(z) \hspace{-0.1ex} = \hspace{.1ex} \smalldisplaystyleonehalf \hspace{.4ex} \rodsectionexpminusone \hspace{-0.4ex} \integral\displaylimits_{\rodsection} \hspace{-0.5ex} \boldnabla \hspace{-0.15ex} \times \hspace{-0.15ex} \bm{u}_{\raisemath{-0.2ex}{3}} \hspace{.2ex} d\rodsection
\hspace{.15ex} ,
\end{equation*}

\vspace{-0.2em}\noindent
то чем другие возможные представления хуже?

Помимо $\internalforce$ \en{and}\ru{и}~$\internalmoment$, есть ещё величина, не вызывающая сомнений\:--- упругая энергия.
Естественно потребовать, чтобы в~одно\-мерной и в~трёх\-мерной моделях энергии на~единицу длины совпали.
При этом, чтобы уйти от~различий в~трактовках ${\bm{u}_{\raisemath{-0.2ex}{1}}}$ и~$\bm{\theta}$, будем исходить из \en{complementary energy}\ru{дополнительной энергии}~${\widehat{\potential}(\internalmoment, \internalforce)}$:

\nopagebreak\vspace{-0.1em}\begin{equation*}
\widehat{\potential}(\internalmoment, \internalforce) = \hspace{-0.2ex} \integral\displaylimits_{\rodsection} \hspace{-0.4ex} \potential_{\raisemath{-0.2ex}{3}} \hspace{.2ex} d\rodsection
\end{equation*}

...

\end{otherlanguage}

\en{\section{Variational method of building one-dimensional model}}

\ru{\section{Вариационный метод построения одномерной модели}}

\label{para:variationalmethodforonedimension}

\begin{otherlanguage}{russian}

Мы только~что определили жёсткости стержня, полагая, что одномерная модель линии типа Cosserat адекватно описывает поведение трёхмерной модели.
\inquotes{Одномерные} представления ассоциируются со~следующей картиной перемещений в~сечении:

\nopagebreak\vspace{-0.1em}\begin{equation}\label{rods.fieldofdisplacements}
\bm{u}(s,\bm{x}) = \bm{U}\hspace{-0.1ex}(s) + \hspace{.15ex} \bm{\theta}(s) \hspace{-0.2ex} \times \bm{x}
\hspace{.1ex} .
\end{equation}

\vspace{-0.1em}\noindent
Однако, такое поле~$\bm{u}$ не~удовлетворяет уравнениям трёхмерной теории\textcolor{red}{(??добавить, каким именно)}.
Невозможно пренебречь возникающими невязками в~дифференциальных уравнениях и краевых условиях.

Формально \inquotes{чистым} является вариационный метод сведения трёхмерной проблемы к~одномерной, называемый иногда методом внутренних связей.
Аппроксимация~\eqref{rods.fieldofdisplacements} подставляется в~трёхмерную формулировку вариационного принципа минимума потенциальной энергии~\eqrefwithchapdotpara{principleofminimumpotentialenergy.formulation}{chapter:linearclassicalelasticity}{para:principleofminimumpotentialenergy}

\nopagebreak\vspace{-0.1em}\begin{equation*}
\potentialenergyfunctional \hspace{.2ex} (\hspace{-0.1ex}\bm{u}\hspace{-0.1ex}) = \hspace{-0.2ex}
\displaystyle\integral\displaylimits_{\mathcal{V}} \hspace{-0.5ex}
\Bigl(
\potential(\hspace{-0.1ex}\bm{u}\hspace{-0.1ex}) \hspace{-0.1ex} - \bm{f} \hspace{-0.1ex} \dotp \bm{u} \Bigr) \hspace{-0.1ex} d\mathcal{V}
- \hspace{-0.3ex}
\displaystyle\integral\displaylimits_{o_2} \hspace{-0.32ex} \bm{p} \dotp \bm{u} \hspace{.33ex} do \hspace{.2ex}
\hspace{.1ex}\to\hspace{.25ex} \mathrm{min}
\hspace{.1ex} ,
\end{equation*}

\vspace{-0.1em}\noindent
которая после интегрирования по~сечению становится одномерной.
Если $\bm{U}$ и~$\bm{\theta}$ варьируются независимо, получаем модель типа Коссера.
В~случае $\bm{U}' \hspace{-0.25ex} = \bm{\theta} \hspace{-0.1ex} \times \bm{t}$ приходим к~классической модели.

Метод внутренних связей привлекателен, его продолжают \inquotesx{переоткрывать}[.]
С~его помощью возможно моделировать тела с~неоднородностью и~анизотропией, он легко обобщается на динамику, если~$\bm{f}$ дополнить неварьируемой динамической добавкой до~${\bm{f} \hspace{-0.1ex} - \rho \hspace{.1ex} \mathdotdotabove{\bm{u}}}$.
Можно рассматривать и~стержни переменного сечения, и~даже нелинейно упругие, ведь вариационная постановка есть~(\chapref{chapter:nonlinearcontinuum}).

Аппроксимацию~\eqref{rods.fieldofdisplacements} можно дополнить слагаемыми с~внутренними степенями свободы.
Понимая необходимость учёта депланаций, некоторые авторы

...

\end{otherlanguage}

...

\en{For variational construction}\ru{Для вариационного построения} \en{of one-dimensional models}\ru{одномерных моделей} \en{it’s convenient to use}\ru{удобно использовать} \ru{принцип }\en{the }Reissner\ru{’а}\hbox{--}Hellinger\ru{’а}\en{ principle}~(\chapdotpararef{chapter:linearclassicalelasticity}{para:mixedvariationalprinciples}) \en{with independent approximation of stresses}\ru{с~независимой аппроксимацией напряжений}~\cite{eliseev-models}.
\en{In this case}\ru{В этом случае}\en{,} \ru{нужна }\en{some consistency between}\ru{некоторая согласованность между}~$\bm{u}$ \en{and}\ru{и}~$\cauchystress$\en{ is needed}.

\en{Many advantages}\ru{Множеству достоинств} \en{of the variational method}\ru{вариационного метода} \en{are opposed}\ru{противостоит} \en{by the one}\ru{один}, \en{but}\ru{но} \en{significant}\ru{значительный} \en{disadvantage}\ru{недостаток}.
\en{Introducing}\ru{Вводя} \en{approximations within cross-sections}\ru{приближения в~сечениях}, \en{we impose}\ru{мы навязываем} \en{our}\ru{наши} \en{unreal}\ru{нереальные} \en{simplifications}\ru{упрощения} \en{on reality}\ru{реальности}.
\en{The variational method}\ru{Вариационный метод} \en{is more suitable}\ru{более подходит} \en{for applied calculations}\ru{для прикладных расчётов}.

\en{\section{Asymptotic splitting of three-dimensional problem}}

\ru{\section{Асимптотическое расщепление трёхмерной проблемы}}

\label{para:rods.asymptoticsplittingofthreedimensional}

\en{Asymptotic splitting}\ru{Асимптотическое расщепление} \en{can be considered}\ru{можно считать} \en{fundamental}\ru{фундаментальным} \en{for describing}\ru{для описания} \en{the mechanics of rods}\ru{механики стержней}.
\en{One-dimensional models}\ru{Одномерные модели} \en{paint only the part of a~picture}\ru{рисуют лишь часть картины}, \en{two-dimensional problems}\ru{двумерные проблемы} \en{in cross-sections}\ru{в~поперечных сечениях} \en{paint the other part}\ru{рисуют другую часть}, \en{and together}\ru{а~вместе} \en{they present}\ru{они представляют} \en{the solution}\ru{решение} \en{of a~three-dimensional problem}\ru{трёхмерной проблемы} \en{for}\ru{для} \en{a~small thickness}\ru{м\'{а}лой толщины}.

\en{How to introduce a~small parameter}\ru{Как ввести малый параметр}~${\hspace{-0.2ex}\smallparameter}$ \en{into a~three-dimensional problem}\ru{в~трёхмерную проблему}?
\en{The easiest way to do it}\ru{Проще всего сделать это} \en{is through}\ru{через} \en{representation}\ru{представление} \en{of the location vector}\ru{вектора положения}~(\pararef{para:overviewofrods}):

\nopagebreak\vspace{-0.2em}\begin{equation*}
\RepresentedLocationVector \hspace{.1ex} (x_{\alpha}, s) \hspace{-0.2ex} = \smallparameter^{-1} \hspace{.2ex} \locationvector(s) \hspace{-0.1ex} + \hspace{.1ex} \bm{x}
\hspace{.1ex} ,
\hspace{.5em}
\bm{x} \equiv x_{\alpha} \bm{e}_{\alpha}(s)
\hspace{.1ex} ,
\hspace{.5em}
\alpha \hspace{-0.1ex} = \hspace{-0.1ex} 1, 2
\hspace{.1ex} .
\end{equation*}

\en{For an~orthonormal basis}\ru{Для ортонормального базиса}, \en{upper and lower indices}\ru{верхние и~нижние индексы} \en{do not differ}\ru{не~различаются}

\nopagebreak\vspace{-0.2em}\begin{equation*}
q_{i} \hspace{-0.2ex} = q^{i} \hspace{-0.1ex}
, \hspace{.5ex}
\RepresentedLocationVector^{\hspace{.1ex}i} \hspace{-0.2ex} = \hspace{-0.2ex} \RepresentedLocationVector_{\hspace{.1ex}i} \hspace{-0.2ex} = \hspace{-0.2ex} \RepresentedLocationVector_{\hspace{.1ex}\differentialindex{i}} \hspace{-0.2ex}
= \scalebox{.88}{$ \displaystyle\frac{\raisebox{-0.2em}{$ \partial \RepresentedLocationVector $}}{\partial q^{\hspace{.1ex}i}} $}
\hspace{.2ex} .
\end{equation*}

\vspace{-0.2em}
\en{Three coordinates}\ru{Три координаты}\en{ are}\ru{\:---}

\nopagebreak\vspace{-0.2em}\begin{equation*}
q^{1} \hspace{-0.3ex} = x_{1}
\hspace{.1ex} , \hspace{.5em}
q^{2} \hspace{-0.3ex} = x_{2}
\hspace{.1ex} , \hspace{.5em}
q^{3} \hspace{-0.3ex} = s
\hspace{.1ex} .
\end{equation*}

\vspace{-0.2em}
\en{The basis vectors}\ru{Векторы базиса} \en{are}\ru{суть}

\nopagebreak\vspace{-0.4em}\begin{equation*}
\bm{e}_{1}
\hspace{.1ex} , \hspace{.4em}
\bm{e}_{2}
\hspace{.1ex} , \hspace{.4em}
\bm{t} \equiv \bm{e}_{3}
\hspace{.1ex} .
\end{equation*}

\en{Representation}\ru{Представление} \ru{оператора }\en{of the }Hamilton\ru{’а}\en{ operator}~${\hspace{-0.3ex}\boldnabla}$

\nopagebreak\begin{gather*}
\boldnabla \hspace{-0.1ex}
= \hspace{-0.1ex} \RepresentedLocationVector^{i} \scalebox{.88}{$ \displaystyle\frac{\raisebox{-0.2em}{$ \partial $}}{\partial q^{\hspace{.1ex}i}} $}
= \hspace{-0.2ex} \perpendicularboldnabla \hspace{-0.2ex} + \hspace{.1ex} \RepresentedVolume^{-1} \hspace{.2ex} \bm{t} \hspace{.2ex} \bigl( \partial_s \hspace{-0.2ex} - \psi_{t} \hspace{.1ex} \mathrm{D} \bigr) \hspace{-0.1ex}
, \hspace{.5em}
\perpendicularboldnabla \hspace{-0.3ex} \equiv \bm{e}_{\alpha} \hspace{.2ex} \scalebox{.88}{$ \displaystyle\frac{\raisebox{-0.2em}{$ \partial $}}{\partial x_{\alpha}} $}
\hspace{.1ex} ,
\\
%
\RepresentedVolume \equiv \RepresentedLocationVector_{\hspace{.15ex}\differentialindex{1}} \hspace{-0.4ex} \times \hspace{-0.2ex} \RepresentedLocationVector_{\hspace{.15ex}\differentialindex{2}} \dotp \RepresentedLocationVector_{\hspace{.15ex}\differentialindex{3}} \hspace{-0.2ex}
= \smallparameter^{-1} \hspace{-0.1ex} + \hspace{.2ex} \bm{t} \hspace{.2ex} \dotp \hspace{.2ex} \perpendicularrodfeatures \hspace{-0.3ex} \times \bm{x}
\hspace{.1ex} ,
\hspace{.5em}
\mathrm{D} \hspace{-0.1ex} \equiv \bm{t} \hspace{.2ex} \dotp \hspace{.2ex} \bm{x} \times \hspace{-0.3ex} \perpendicularboldnabla
\hspace{.1ex} ,
\\[.2em]
%
\perpendicularrodfeatures \hspace{-0.3ex} + \psi_{t} \hspace{.2ex} \bm{t} = \rodfeaturesvector
%%\hspace{.1ex} .
\end{gather*}

\vspace{-0.2em}\noindent
(\en{the meaning}\ru{смысл} \en{of~vector}\ru{вектора}~${\rodfeaturesvector}$ \en{is the same as}\ru{тот~же как} \en{in}\ru{и~в}~\pararef{para:overviewofrods}).

\en{The }\ru{Тензор напряжения }Cauchy\en{ stress tensor}

\nopagebreak\vspace{-0.4em}\begin{equation*}
\cauchystress = \perpendicularstress \hspace{-0.2ex} + \mathsigma_{t} \hspace{.2ex} \bm{t} \bm{t} + 2 \hspace{.2ex} \bm{t} \hspace{.2ex} {\parallelstressvector}^{\hspace{-0.2ex}\mathsf{S}}
\hspace{-0.1ex} , \hspace{.5em}
\perpendicularstress \hspace{-0.2ex} \equiv \mathtau_{\alpha \beta} \hspace{.2ex} \bm{e}_{\alpha} \bm{e}_{\beta}
\hspace{.1ex} , \hspace{.5em}
\parallelstressvector \hspace{-0.1ex} \equiv \cauchystresscomponents_{3\alpha} \hspace{.2ex} \bm{e}_{\alpha}
\hspace{.1ex} .
\end{equation*}

...

%%\begin{otherlanguage}{russian}
%%\end{otherlanguage}

\en{\section{Thermal deformation and stress}}

\ru{\section{Температурные деформация и напряжение}}

\en{The~direct approach}\ru{Прямой подход}, \en{so efficient}\ru{столь эффективный} \en{for making}\ru{для создания} \en{one-dimensional}\ru{одномерных моделей} Cosserat \en{and}\ru{и}~Kirchhoff\ru{’а}\en{ models}, \en{isn’t applicable }\en{for problems}\ru{для~проблем} \en{of~thermoelasticity}\ru{термоупругости}\ru{ непримен\'{и}м}.
\en{The~transition}\ru{Переход} \en{from the three-dimensional model to the one-dimensional}\ru{от трёхмерной модели к~одномерной} \en{can be realized}\ru{может быть реализован} \en{either by variational or asymptotic way}\ru{или~вариационным путём, или~асимптотическим}.

\begin{otherlanguage}{russian}

Описанный в~\pararef{para:variationalmethodforonedimension} вариационный метод целиком переносится на~термоупругость\:--- включая задачи с~неоднородностью и~анизотропией, переменным сечением, динамические и~даже нелинейные.
Для этого нужно~(\chapdotpararef{chapter:thermoelasticity}{para:variationalformulations.thermoelasticity})
\en{in}\ru{в} \en{the~}\ru{принципе }Lagrange\ru{’а}\en{ principle} \en{of~minimum potential energy}\ru{минимума потенциальной энергии}
заменить потенциал~${\potential(\hspace{-0.1ex}\infinitesimaldeformation\hspace{-0.1ex})\hspace{-0.1ex}}$
свободной энергией

...


\end{otherlanguage}

\section*{\small \wordforbibliography}

\begin{changemargin}{\parindent}{0pt}
\fontsize{10}{12}\selectfont

\en{Unlike other topics}\ru{В~отличие от других тем} \en{of the~theory of~elasticity}\ru{теории упругости}, \en{rods}\ru{стержни} \en{are presented}\ru{представлены} \en{in books}\ru{в~книгах} \en{very modestly}\ru{очень скромно}.
\ru{Стиль изложения}\en{Narration style of} \inquotes{\en{strength of~materials}\ru{сопротивления материалов}} \en{prevails}\ru{преобладает}, \en{more exact and perfect approaches}\ru{более точные и~совершенные подходы} \en{seem}\ru{кажутся} \en{impossible}\ru{невозможными} \en{or}\ru{или}~\en{unnecessary}\ru{ненужными} \en{to many authors}\ru{многим авторам}.

\en{But}\ru{Но} \en{many interesting articles have been published}\ru{было опубликовано много интересных статей}, \en{their reviews}\ru{их обзоры} \en{are presented}\ru{представлены}, \en{for example}\ru{например}, \en{by}\ru{у}~S.\:Antman\ru{’а}~\cite{stuartantman-theoryofrods}, \foreignlanguage{russian}{В.\,В.\;Елисеев}\ru{а}~\cite{eliseev-models} \en{and}\ru{и}~\foreignlanguage{russian}{А.\,А.\;Илюхин}\ru{а}~\cite{ilyuhin-elasticrods}.

\end{changemargin}
