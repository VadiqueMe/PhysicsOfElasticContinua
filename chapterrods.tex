\en{\chapter{Rods}}

\ru{\chapter{Стержни}}

\thispagestyle{empty}

\newcommand{\surface}{\Upomega} %% {F}
\newcommand{\surfaceexpminusone}{\surface^{\hspace{-0.2ex}\expminusone}}

\label{chapter:rods}

\en{\section{Initial concepts}}

\ru{\section{Исходные представления}}

\label{para:overviewofrods}

\begin{otherlanguage}{russian}

\lettrine[lines=2, findent=2pt, nindent=0pt]{С}{тержень}~--- это тонкое длинное тело. Он характеризуется прежде всего своей осью~--- пространственной кривой, которую \inquotes{облепляет} материал (рисунок ?? 22 ??). В~каждой точке оси имеем плоскую фигуру

...


...


Наиболее логичный подход к~описанию деформации упругих стержней связан с~асимптотическим ращеплением трёхмерной задачи при малой толщине. Однако для сложной асимптотической процедуры заранее нужен некий вариант ответа. Такой вариант даёт прямой подход, основанный на одномерной модели стержня как материальной линии. Но какими степенями свободы~--- кроме трансляции~--- должны обладать частицы этой линии?

Давно известно, что стержни чувствительны к~моментным нагрузкам. А~присутствие моментов среди обобщённых сил говорит о~наличии вращательных степеней свободы. Следовательно, одномерной моделью стержня должна быть линия Коссера~--- она состоит из элементарных твёрдых тел. Впрочем, могут проявиться и~дополнительные степени свободы~--- как в~тонкостенных стержнях, которым посвящена отдельная глава.

В~механике упругих тел стержни занимают особенное место. Во\hbox{-}первых, это моментные модели, и~моменты здесь играют главную роль~(не~роль поправок, как в~трёхмерном континууме Коссера). Во\hbox{-}вторых, стержни являются как~бы \inquotes{тестовой площадкой} для~моделей с~дополнительными степенями свободы, поскольку наличие этих степеней можно достоверно исследовать на~трёх\-мерной модели.

Ну~а~пока сосредоточимся на~простой одномерной модели Коссера.

\section{Кинематика линий Коссера}

Рассматриваемое далее является упрощённым вариантом~\chapref{chapter:cosseratcontinuum}. Вместо тройки материальных координат ${q^{\hspace{.1ex}i}}$ имеем одну~--- $s$, это может быть дуговая координата в~отсчётной конфигурации. Движение определяется радиусом-вектором~${\bm{r}(s,t)}$ и~тензором поворота~${\bm{P}(s,t)}$. Линейная и~угловая скорости частицы вводятся равенствами

...



\section{Силовые факторы и их баланс}

Поскольку частицы стержня~(линии Коссера)~--- твёрдые тела, то силовыми факторами являются силы и~моменты: на~элемент~$ds$ действуют внешние сила~${\bm{q}ds}$ и~момент~${\bm{m}ds}$. Внутренние взаимо\-действия тоже определяются силой~${\mathboldQ \hspace{.1ex} (s)}$ и~моментом~${\mathboldM \hspace{.1ex} (s)}$~--- это воздействие от~частицы с~координатой~${s\!+\!0}$ к~частице с~${s\!-\!0}$. Из~закона

...



\section{Принцип виртуальной работы и его следствия}

\en{\noindent For a~piece of~rod ${s_0 \leq s \leq s_1}$ formulation of the principle is as~follows}

\ru{\noindent Для куск\'{а} стержня ${s_0 \leq s \leq s_1}$ формулировка принципа таков\'{а}}

...


\en{Conventionally call $\bm{a}$ the~tensor of~stiffness for bending and~twisting, $\bm{b}$ the~tensor of~stiffness for (ex)tension and~shear, and~$\bm{c}$ the~tensor of crosslinks.}

\ru{Условно назовём $\bm{a}$ тензором жёсткости на~изгиб и~кручение, $\bm{b}$~--- тензором жёсткости на~растяжение и~сдвиг, а~$\bm{c}$~--- тензором перекрёстных связей.}

\en{Stiffness tensors rotate together with particle:}

\ru{Тензоры жёсткости поворачиваются вместе с~частицей:}

...



\section{Классическая модель Кирхгофа}

До~сих~пор функции ${\bm{r}(s,t)}$ и~${\bm{P}(s,t)}$ были независимы. В~классической теории Кирхгофа существует внутренняя связь

...



\section{Задача Эйлера}

Рассматривается прямой стержень, защемлённый на одном конце и~нагруженный силой~$\mathboldQ$ на~другом (рисунок ?? 23 ??). Сила~--- \inquotes{мёртвая}, то~есть не~меняется при~деформировании

...



\section{Уравнения в вариациях}

В~нелинейной механике упругих тел полезны уравнения в~вариациях, описывающие малое изменение актуальной конфигурации. Как и в~\chapdotpararef{chapter:nonlinearcontinuum}{para:variationofconfiguration}, вариации величин

...



\section{Модель с растяжением без сдвига}

Модель Кирхгофа с~${\Gamma \hspace{-0.25ex}=\hspace{-0.2ex} 0}$ не~описывает наипростейший случай растяжения\hbox{--}сжатия прямого стержня. Эта неприятность исчезнет, если смягчить связь: запретить лишь поперечный сдвиг, но разрешить растяжение, то~есть

...



\section{Механика нити}

Этот вопрос выходит за~рамки главы, поскольку нить проще стержня: её частицами являются \inquotes{обычные} материальные точки c~трансляционными степенями свободы. Соответственно, ...

...



\section{Линейная теория}

В~линейной теории внешние воздействия считаются малыми, а~отсчётная конфигурация~--- ненапряжённым состоянием покоя. Уравнения в~вариациях в~этом случае дают

...



\section{Случай малой толщины}

При м\'{а}лой относительной толщине стержня модель типа Коссера уступает место классической. Понятие \inquotes{толщина} определяется соотношением жёсткостей: $\bm{a}$, $\bm{b}$ и~$\bm{c}$~--- разной размерности; полагая ${\bm{a} = h^{\hspace{-0.12ex}2} \hspace{.2ex} \widearc{\bm{a}}}$ и~${\bm{c} = h \hspace{.16ex} \widearc{\bm{c}}}$, где~$h$~--- некий масштаб длины, получим тензоры

...


Переход модели Коссера в~классическую кажется более очевидным при непосредственном интегрировании

...



\section{Задача Сен-Венана}

Трудно переоценить ту роль, которую играет в~механике стержней классическое решение Saint\hbox{-\hspace{-0.2ex}}Venant’а. О~нём уж\'{е} шла речь в~\chapdotpararef{chapter:linearclassicalelasticity}{para:twistingofrods.saintvenant}.

Вместо условий ...

...



\section{Определение жёсткостей по энергии}

Для определения тензоров жёсткости~$\bm{a}$, $\bm{b}$ и~$\bm{c}$ одномерной модели достаточно решений трёхмерных задач для стержня. Но тут возникают два вопроса: какие именно задачи рассматривать и что нужно взять из решений?

Проблема Saint\hbox{-\hspace{-0.2ex}}Venant’а выделяется среди прочих, ведь оттуда берётся жёсткость на~кручение.

Вдобавок есть много точных решений, получаемых таким путём: задаётся поле~${\bm{u}(\bm{r})}$, определяется~${\widearc{\mathboldtau} = \stiffnesstensorC \dotdotp \hspace{-0.12ex} \boldnabla \bm{u}}$, затем находятся объёмные~${\bm{f} = - \boldnabla \dotp \widearc{\mathboldtau}}$ и~поверхностные~${\bm{p} = \bm{n} \dotp \widearc{\mathboldtau}}$ нагрузки.

Но что делать с~решением? Ясно, что $\mathboldQ$ и~$\mathboldM$ в~стержне~--- это интегралы по~сечению~(...). И совсем не~ясно, что считать перемещением и~поворотом в~одномерной модели. Если предложить, например, такой вариант (индекс у~$\bm{u}$ это размерность модели)

\nopagebreak\vspace{-0.1em}\begin{equation*}
\bm{u}_1(z) = \surfaceexpminusone \hspace{-0.4ex} \integral\displaylimits_{\surface} \hspace{-0.5ex} \bm{u}_3(\bm{x},z) \hspace{.25ex} d\surface \hspace{-0.1ex},
\:\:
\bm{\theta}(z) = \smalldisplaystyleonehalf \hspace{.4ex} \surfaceexpminusone \hspace{-0.4ex} \integral\displaylimits_{\surface} \hspace{-0.5ex} \boldnabla \hspace{-0.15ex} \times \hspace{-0.15ex} \bm{u}_3 \hspace{.25ex} d\surface \hspace{-0.1ex},
\end{equation*}

\vspace{-0.2em} \noindent то чем другие возможные представления хуже?

Помимо $\mathboldQ$ и~$\mathboldM$, есть ещё величина, не вызывающая сомнений~--- упругая энергия. Естественно потребовать, чтобы в~одно\-мерной и в~трёх\-мерной моделях энергии на~единицу длины совпали. При этом, чтобы уйти от~различий в~трактовках $\bm{u}_1$ и~$\bm{\theta}$, будем исходить из дополнительной энергии~${\widehat{\Pi}(\mathboldM, \mathboldQ)}$:

\nopagebreak\vspace{-0.1em}\begin{equation*}
\widehat{\Pi}(\mathboldM, \mathboldQ) = \hspace{-0.2ex} \integral\displaylimits_{\surface} \hspace{-0.4ex} \Pi_3 \hspace{.2ex} d\surface
\end{equation*}

...



\section{Вариационный метод построения одномерных моделей}
\label{para:variationalmethodforonedimension}

Мы только~что определили жёсткости стержня, полагая, что одномерная модель Коссера правильно отражает поведение трёхмерной модели. \inquotes{Одномерные} представления ассоциируются со~следующей картиной перемещений в~сечении:

\nopagebreak\vspace{-0.1em}\begin{equation}\label{rods.fieldofdisplacements}
\bm{u}(s,\bm{x}) = \bm{U}\hspace{-0.1ex}(s) + \hspace{.15ex} \bm{\theta}(s) \hspace{-0.2ex} \times \bm{x} \hspace{.12ex} .
\end{equation}

\vspace{-0.1em} \noindent Однако, такое поле~$\bm{u}$ не~удовлетворяет уравнениям трёхмерной теории\textcolor{red}{(??добавить, каким именно)}. Невозможно пренебречь возникающими невязками в~дифференциальных уравнениях и краевых условиях.

Формально \inquotes{чистым} является вариационный метод сведения трёхмерной проблемы к~одномерной, называемый иногда методом внутренних связей. Аппроксимация~\eqref{rods.fieldofdisplacements} подставляется в~трёхмерную формулировку вариационного принципа минимума потенциальной энергии~(\chapdotpararef{chapter:linearclassicalelasticity}{para:principleofminimumpotentialenergy})

\nopagebreak\vspace{-0.1em}\begin{equation*}
\textit{Э} \hspace{0.16ex} (\hspace{-0.1ex}\bm{u}\hspace{-0.1ex}) = \hspace{-0.2ex}
\displaystyle \integral\displaylimits_{\mathcal{V}} \hspace{-0.64ex}
\left(^{\mathstrut} \hspace{-0.1ex}
\Pi(\boldnabla \bm{u}) - \bm{f} \hspace{-0.1ex} \dotp \bm{u} \right) \hspace{-0.4ex} d\mathcal{V} \hspace{.1ex}
- \hspace{-0.2ex}
\integral\displaylimits_{o_2} \hspace{-0.32ex} \bm{p} \dotp \bm{u} \hspace{0.25ex} do \hspace{0.2ex}
\hspace{.1ex}\to\hspace{.25ex} \mathrm{min} \hspace{.1ex} ,
\end{equation*}

\vspace{-0.1em} \noindent которая после интегрирования по~сечению становится одномерной. Если $\bm{U}$ и~$\bm{\theta}$ варьируются независимо, получаем модель типа Коссера. В~случае $\bm{U}' \hspace{-0.25ex} = \bm{\theta} \hspace{-0.1ex} \times \bm{t}$ приходим к~классической модели.

Метод внутренних связей привлекателен, его продолжают \inquotesx{переоткрывать}[.] С~его помощью возможно моделировать тела с~неоднородностью и~анизотропией, он легко обобщается на динамику, если~$\bm{f}$ дополнить неварьируемой динамической добавкой до~${\bm{f} \hspace{-0.1ex} - \rho \hspace{.1ex} \mathdotdotabove{\bm{u}}}$. Можно рассматривать и стержни переменного сечения, и даже нелинейно упругие, ведь вариационная постановка есть~(\chapref{chapter:nonlinearcontinuum}).

Аппроксимацию~\eqref{rods.fieldofdisplacements} можно дополнить слагаемыми с~внутренними степенями свободы. Понимая необходимость учёта депланаций, некоторые авторы

...

...



Для вариационного построения одномерных моделей удобен принцип Рейсснера\hbox{--}Хеллингера~(\chapdotpararef{chapter:linearclassicalelasticity}{para:mixedvariationalprinciples}) с~независимой аппроксимацией напряжений~\cite{eliseev-models}. Но тогда необходима некая согласованность между $\bm{u}$ и~$\mathboldtau$.

Множеству достоинств вариационного метода противостоит один, но очень большой недостаток. Вводя аппроксимации по~сечению, мы навязываем реальности свои упрощённые представления. Вариационный метод более подходит для прикладных расчётов.

\section{Асимптотическое ращепление трёхмерной задачи}

В~изложении механики стержней асимптотическое ращепление можно считать фундаментальным. Одномерные модели составляют лишь часть картины; другая часть~--- это двумерные задачи в~сечении, а~вместе они являются тем решением трёхмерной задачи, которое образуется при~м\'{а}лой толщине.

Малый параметр~$\lambda$ в~трёхмерную задачу проще всего ввести через представление вектора\hbox{-}радиуса~(рисунок ?? 22 ??, \pararef{para:overviewofrods}):

...



\section{Температурные деформации и напряжения}

Прямой подход, столь эффективный при~построении одномерных моделей Коссера и~Кирхгофа, для~проблем термоупругости невыгоден. Тут нужно рассматривать трёхмерную модель, что может быть реализовано или~вариационным путём, или~асимптотическим.

Описанный в~\pararef{para:variationalmethodforonedimension} вариационный метод целиком переносится на~термоупругость~--- включая задачи с~неоднородностью и~анизотропией, переменным сечением, динамические~--- и~даже нелинейные. Достаточно в~принципе Лагранжа заменить потенциал

...




\vspace{8mm}
\hfill\begin{minipage}[b]{0.95\linewidth}
\fontsize{10}{12}\selectfont

\section*{\wordforbibliography}

В~отличие от~других разделов теории упругости, механика стержней в~книгах представлена весьма скромно. Преобладает изложение в~стиле сопротивления материалов, более точные и~совершенные подходы большинству авторов кажутся невозможными или~ненужными. Но есть немало интересных статей; соответствующие обзоры можно найти у~...

\end{minipage}

\end{otherlanguage}
