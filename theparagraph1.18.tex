\en{\section{Curvature tensors}}

\ru{\section{Тензоры кривизны}}

\label{para:curvaturetensors}

\begin{changemargin}{2\parindent}{\parindent}
\bgroup % to change \parindent locally
\setlength{\parindent}{\negparindent}
\setlength{\parskip}{\spacebetweenparagraphs}
\small

\leavevmode{\indent}The~\href{https://en.wikipedia.org/wiki/Riemann_curvature_tensor}{\emph{Riemann curvature tensor} or \emph{Riemann\hbox{--}Christoffel tensor}} (after \href{https://en.wikipedia.org/wiki/Bernhard_Riemann}{\textbold{Bernhard Riemann}} and \href{https://en.wikipedia.org/wiki/Elwin_Bruno_Christoffel}{\textbold{Elwin Bruno Christoffel}}) is the most common method used to express the curvature of Riemannian manifolds. It’s a~tensor field, it assigns a~tensor to each point of a~Riemannian manifold, that measures the extent to which the~metric tensor is not locally isometric to that of \inquotes{flat} space. The curvature tensor measures noncommutativity of the covariant derivative, and as such is the~integrability obstruction for the~existence of an~isometry with \inquotes{flat} space.

%%%\vspace{.2em}
%%%\hfill $\sim$\:\emph{from Wikipedia, the free encyclopedia}
\par
\egroup
\nopagebreak\vspace{.12em}
\end{changemargin}

\begin{otherlanguage}{russian}

\noindent
Рассматривая тензорные поля в~криволинейных координатах~(\sectionref{para:spatialdifferentiationoftensorfields}), мы исходили из~представления вектора\hbox{-}радиуса~(вектора положения) точки функцией этих координат:
${\locationvector \hspace{-0.4ex} = \hspace{-0.4ex} \locationvector(q^{\hspace{.1ex}i}\hspace{.1ex})}$.
Этим отношением порождаются выражения

\nopagebreak\begin{itemize}
\item векторов локального касательного базиса ${%
\locationvector_\differentialindex{i} \hspace{-0.16ex} \equiv \smash{ \raisemath{.16em}{\scalebox{0.8}{$ \partial \hspace{.1ex} \locationvector $}} \hspace{-0.3ex} / \hspace{-0.4ex} \raisemath{-0.32em}{\scalebox{0.8}{$ \partial q^{\hspace{.1ex}i} $}} } \hspace{-0.15ex} \equiv \partial_i \hspace{.1ex} \locationvector%
}$,
%
\item компонент ${\textsl{g}_{i\hspace{-0.1ex}j} \hspace{-0.24ex} \equiv \locationvector_\differentialindex{i} \hspace{-0.16ex} \dotp \locationvector_\differentialindex{\hspace{-0.1ex}j}}$ и~${\textsl{g}^{\hspace{.25ex}i\hspace{-0.1ex}j} \hspace{-0.32ex} \equiv \locationvector^i \hspace{-0.32ex} \dotp \locationvector^j \hspace{-0.32ex} = \smash{\textsl{g}_{i\hspace{-0.1ex}j}^{\hspace{.33ex}\expminusone}}}$ единичного \inquotes{метрического} тензора~${\UnitDyad = \locationvector_\differentialindex{i} \locationvector^i \hspace{-0.2ex} = \locationvector^i \locationvector_\differentialindex{i} \hspace{-0.15ex} = \textsl{g}_{j\hspace{-0.1ex}k} \hspace{.1ex} \locationvector^{\hspace{.1ex}j} \hspace{-0.1ex} \locationvector^{k} \hspace{-0.25ex} = \textsl{g}^{\hspace{.25ex}j\hspace{-0.1ex}k} \hspace{.1ex} \locationvector_\differentialindex{\hspace{-0.1ex}j} \locationvector_\differentialindex{k}}$,
%
\item векторов локального взаимного кокасательного базиса ${\locationvector^i \hspace{-0.32ex} \dotp \locationvector_\differentialindex{\hspace{-0.1ex}j} \hspace{-0.22ex} = \delta_{\hspace{-0.15ex}j}^{\hspace{.2ex}i}}$, ${\locationvector^i \hspace{-0.25ex} = \textsl{g}^{\hspace{.25ex}i\hspace{-0.1ex}j} \locationvector_\differentialindex{\hspace{-0.1ex}j}}$,
%
\item диф\-ферен\-циаль\-ного набла\hbox{-}оператора Hamilton’а ${\smash{\boldnabla \equiv \locationvector^i \partial_i}}$,
${\UnitDyad = \hspace{-0.25ex} \smash{\boldnabla \locationvector}}$,
%
\item полного дифференциала ${d \bm{\xi} = d \locationvector \dotp \hspace{-0.2ex} \boldnabla \hspace{-0.05ex} \bm{\xi} \hspace{.1ex}}$,
%
\item частных производных касательного \hbox{базиса} (вторых частных производных~$\locationvector$) ${\locationvector_{\differentialindex{i}\hspace{.2ex}\differentialindex{\hspace{-0.1ex}j}} \hspace{-0.2ex} \equiv \partial_i \partial_j \locationvector \hspace{-0.1ex} = \partial_i \hspace{.12ex} \locationvector_\differentialindex{\hspace{-0.1ex}j}}$,
%
\item символов \inquotes{связности} \hbox{Христоффеля}~(\hbox{Christoffel} symbols) ${\Gamma_{\hspace{-0.25ex}i\hspace{-0.1ex}j}^{\hspace{.25ex}k} \hspace{-0.1ex} \equiv \locationvector_{\differentialindex{i}\hspace{.2ex} \differentialindex{\hspace{-0.1ex}j}} \hspace{-0.2ex} \dotp \locationvector^k
%%\hspace{-0.32ex} = \Gamma_{\hspace{-0.25ex}i\hspace{-0.1ex}j\mathdotbelow{n}} \hspace{.25ex} \textsl{g}^{\hspace{.25ex}nk}\hspace{-0.25ex}
}$ и~${\Gamma_{\hspace{-0.25ex}i\hspace{-0.1ex}j\mathdotbelow{k}} \hspace{-0.16ex} \equiv \locationvector_{\differentialindex{i}\hspace{.2ex}\differentialindex{\hspace{-0.1ex}j}} \hspace{-0.2ex} \dotp \locationvector_\differentialindex{k}
%%\hspace{-0.2ex} = \Gamma_{\hspace{-0.25ex}i\hspace{-0.1ex}j}^{\hspace{.25ex}n} \hspace{.16ex} \textsl{g}_{nk}
}$.
\vspace{-0.2em}
\end{itemize}

Представим теперь, что функция~${\locationvector(q^{\hspace{.1ex}k})}$ не~известна, но~\hbox{зат\'{о}} в~каждой точке пространства известны шесть независимых компонент положительно определённой (\en{all}\ru{все} \ru{матрицы }Gram\ru{’а}\en{ matrices} \en{are non-negative definite}\ru{определены неотрицательно}) симметричной метрической матрицы Gram\ru{’а}~${\textsl{g}_{i\hspace{-0.1ex}j}(q^{\hspace{.1ex}k})}$.

the Gram matrix (or Gramian)

Билинейная форма ...

\nopagebreak
...

Поскольку шесть функций~${\textsl{g}_{i\hspace{-0.1ex}j}(q^{\hspace{.1ex}k})}$ происходят от векторной функции~${\locationvector(q^{\hspace{.1ex}k})}$, то между элементами~$\textsl{g}_{i\hspace{-0.1ex}j}$ существуют некие соотношения.

\en{Differential}\ru{Дифференциал} ${d\locationvector}$\;\eqref{differentialoflocationvector}\en{ is}\ru{\:---} \en{exact}\ru{полный~(точный)}.
\en{This is true}\ru{Это истинно} \en{if and only if}\ru{тогда и только тогда, когда} \en{second partial derivatives}\ru{вторые частные производные} \en{commute}\ru{коммутируют}:

\nopagebreak\vspace{-0.2em}\begin{equation*}
d\locationvector \hspace{-0.1ex} = \locationvector_\differentialindex{k} \hspace{.2ex} dq^{\hspace{.1ex}k}
\hspace{.4em}\Leftrightarrow\hspace{.44em}
%%\partial_i \bigl( \partial_j \locationvector \bigr) \hspace{-0.2ex} = \hspace{.1ex} \partial_j \bigl( \partial_i \locationvector \bigr) \hspace{-0.2ex}
%%\hspace{.5em}\text{\en{or}\ru{или}}\hspace{.5em}
\partial_i \hspace{.12ex} \locationvector_\differentialindex{\hspace{-0.1ex}j} \hspace{-0.22ex} = \partial_j \locationvector_\differentialindex{i}
\hspace{.5em}\text{\en{or}\ru{или}}\hspace{.5em}
\locationvector_{\differentialindex{i}\hspace{.2ex}\differentialindex{\hspace{-0.1ex}j}} \hspace{-0.22ex} = \locationvector_{\differentialindex{\hspace{-0.1ex}j}\hspace{.2ex}\differentialindex{i}}
\hspace{.1ex} .
\end{equation*}

\vspace{-0.2em}\noindent
Но это условие уж\'{е} обеспечено симметрией~${\textsl{g}_{i\hspace{-0.1ex}j}}$

...

\en{metric}\ru{метрическая} (\inquotes{\en{affine}\ru{аффинная}}) \en{connection}\ru{связность}~$\nabla_{\hspace{-0.32ex}i\hspace{.1ex}}$, её~же называют \inquotes{\en{covariant derivative}\ru{ковариантная производная}}

\vspace{1.2em}\begin{equation*}
\locationvector_{\differentialindex{i}\hspace{.2ex}\differentialindex{\hspace{-0.1ex}j}} \hspace{-0.1ex} = \hspace{.1ex}
\tikzmark{beginChristoffelSymbolOne} \locationvector_{\differentialindex{i}\hspace{.2ex}\differentialindex{\hspace{-0.1ex}j}} \hspace{-0.15ex} \dotp \tikzmark{beginEtensorUpDown} \locationvector^{k} \hspace{-0.4ex} \tikzmark{endChristoffelSymbolOne} \hspace{.4ex} \locationvector_\differentialindex{k} \tikzmark{endEtensorUpDown} \hspace{-0.2ex}
= \tikzmark{beginChristoffelSymbolOther} \locationvector_{\differentialindex{i}\hspace{.2ex}\differentialindex{\hspace{-0.1ex}j}} \hspace{-0.15ex} \dotp \tikzmark{beginEtensorDownUp} \locationvector_\differentialindex{k} \tikzmark{endChristoffelSymbolOther} \locationvector^k \tikzmark{endEtensorDownUp}
\end{equation*}%
\AddOverBrace[line width=.75pt][0,0.2ex][yshift=-0.1em]{beginEtensorUpDown}{endEtensorUpDown}{${\scriptstyle \UnitDyad}$}%
\AddOverBrace[line width=.75pt][0,0.2ex][yshift=-0.1em]{beginEtensorDownUp}{endEtensorDownUp}{${\scriptstyle \UnitDyad}$}%
\AddUnderBrace[line width=.75pt][0,-0.1ex]{beginChristoffelSymbolOne}{endChristoffelSymbolOne}{${\scriptstyle \Gamma_{\hspace{-0.25ex}i\hspace{-0.1ex}j}^{\hspace{.25ex}k}}$}%
\AddUnderBrace[line width=.75pt][0,-0.1ex]{beginChristoffelSymbolOther}{endChristoffelSymbolOther}{${\scriptstyle \Gamma_{\hspace{-0.25ex}i\hspace{-0.1ex}j\mathdotbelow{k}}}$}

${
\Gamma_{\hspace{-0.25ex}i\hspace{-0.1ex}j}^{\hspace{.25ex}k} \hspace{.2ex} \locationvector_\differentialindex{k} \hspace{-0.2ex} = \locationvector_{\differentialindex{i}\hspace{.2ex}\differentialindex{\hspace{-0.1ex}j}} \hspace{-0.16ex} \dotp \hspace{.1ex} \locationvector^k \locationvector_\differentialindex{k} \hspace{-0.2ex} = \locationvector_{\differentialindex{i}\hspace{.2ex}\differentialindex{\hspace{-0.1ex}j}}
}$

covariant derivative (affine connection) is only defined for vector fields

${
\boldnabla \bm{v} \hspace{-0.15ex}
= \locationvector^{i} \partial_i \hspace{-0.33ex} \left( v^{\hspace{.12ex}j} \locationvector_\differentialindex{\hspace{-0.1ex}j} \right) \hspace{-0.25ex}
= \locationvector^{i} \hspace{-0.4ex} \left( \partial_i v^{\hspace{.12ex}j} \locationvector_\differentialindex{\hspace{-0.1ex}j} \hspace{-0.12ex} + v^{\hspace{.12ex}j} \locationvector_{\differentialindex{i}\hspace{.2ex}\differentialindex{\hspace{-0.1ex}j}} \right)
}$

${
\boldnabla \bm{v} \hspace{-0.15ex}
= \locationvector^{i} \locationvector_\differentialindex{\hspace{-0.1ex}j} \nabla_{\hspace{-0.32ex}i\hspace{.1ex}} v^{\hspace{.12ex}j} \hspace{-0.3ex} , \:\:
\nabla_{\hspace{-0.32ex}i\hspace{.1ex}} v^{\hspace{.12ex}j} \hspace{-0.3ex} \equiv
\partial_i v^{\hspace{.12ex}j} \hspace{-0.33ex} + \Gamma_{\hspace{-0.25ex}in}^{\hspace{.25ex}j} v^{\hspace{.1ex}n}
}$

${
\boldnabla \locationvector_\differentialindex{i} \hspace{-0.2ex}
= \locationvector^k \partial_k \locationvector_\differentialindex{i} \hspace{-0.2ex}
= \locationvector^k \locationvector_{\differentialindex{k}\hspace{.2ex}\differentialindex{i}} \hspace{-0.2ex}
%%= \locationvector^k \hspace{.2ex} \Gamma_{\hspace{-0.25ex}ki}^{\hspace{.25ex}n} \hspace{.2ex} \locationvector_\differentialindex{n} \hspace{-0.2ex}
= \locationvector^k \locationvector_\differentialindex{n} \hspace{.1ex} \Gamma_{\hspace{-0.25ex}ki}^{\hspace{.25ex}n}
\hspace{.2ex} , \:\:
\nabla_{\hspace{-0.32ex}i\hspace{.1ex}} \locationvector_\differentialindex{n} \hspace{-0.25ex}
= \Gamma_{\hspace{-0.25ex}in}^{\hspace{.25ex}k} \hspace{.16ex} \locationvector_\differentialindex{k}
}$

\vspace{.2em}
Christoffel symbols describe a~metric (\inquotes{affine}) connection, that is how the~basis changes from point to~point.

символы Christoffel’я это \inquotes{\en{components of~connection}\ru{компоненты связности}} \en{in local coordinates}\ru{в~локальных координатах}

...

\href{https://en.wikipedia.org/wiki/Torsion_tensor}{\en{torsion tensor}\ru{тензор кручения}}~${^3\bm{\mathfrak{T}}}$ \en{with components}\ru{с~компонентами}

\nopagebreak\vspace{-0.1em}\begin{equation*}
\mathfrak{T}^{k}_{i\hspace{-0.1ex}j} \hspace{-0.15ex} = \Gamma_{\hspace{-0.25ex}i\hspace{-0.1ex}j}^{\hspace{.25ex}k} \hspace{-0.1ex} - \Gamma_{\hspace{-0.33ex}j\hspace{-0.06ex}i}^{\hspace{.25ex}k}
\end{equation*}

\noindent
determines the~antisymmetric part of a~connection

...

\noindent
симметрия ${ \Gamma_{\hspace{-0.25ex}i\hspace{-0.1ex}j\mathdotbelow{k}} = \Gamma_{\hspace{-0.33ex}j\hspace{-0.06ex}i\mathdotbelow{k}} }$, поэтому ${3^3 \hspace{-0.2ex} - 3 \hspace{-0.2ex}\cdot\hspace{-0.2ex} 3 = 18}$ разных~(независимых) ${\Gamma_{\hspace{-0.25ex}i\hspace{-0.1ex}j\mathdotbelow{k}}}$

\begin{multline}
\Gamma_{\hspace{-0.25ex}i\hspace{-0.1ex}j}^{\hspace{.25ex}n} \hspace{.16ex} \textsl{g}_{nk} \hspace{-0.24ex} = \Gamma_{\hspace{-0.25ex}i\hspace{-0.1ex}j\mathdotbelow{k}} \hspace{-0.2ex} = \locationvector_{\differentialindex{i}\hspace{.2ex}\differentialindex{\hspace{-0.1ex}j}} \hspace{-0.2ex} \dotp \locationvector_\differentialindex{k} \hspace{-0.1ex} =
\\[-0.1em]
%
= \smallerdisplaystyleonehalf \hspace{-0.1ex} \bigl( \locationvector_{\differentialindex{i}\hspace{.2ex}\differentialindex{\hspace{-0.1ex}j}} \hspace{-0.16ex} + \locationvector_{\differentialindex{\hspace{-0.1ex}j}\hspace{.2ex}\differentialindex{i}} \bigr) \hspace{-0.2ex} \dotp \locationvector_\differentialindex{k} \hspace{-0.1ex}
+ \smallerdisplaystyleonehalf \hspace{-0.1ex} \bigl( \locationvector_{\differentialindex{\hspace{-0.1ex}j}\hspace{.2ex}\differentialindex{k}} \hspace{-0.16ex} - \locationvector_{\differentialindex{k}\hspace{.2ex}\differentialindex{\hspace{-0.1ex}j}} \bigr) \hspace{-0.2ex} \dotp \locationvector_\differentialindex{i} \hspace{-0.1ex}
+ \smallerdisplaystyleonehalf \hspace{-0.1ex} \bigl( \locationvector_{\differentialindex{i}\hspace{.2ex}\differentialindex{k}} \hspace{-0.16ex} - \locationvector_{\differentialindex{k}\hspace{.1ex}\differentialindex{i}} \bigr) \hspace{-0.2ex} \dotp \locationvector_\differentialindex{\hspace{-0.1ex}j} \hspace{-0.1ex} =
\\[-0.1em]
%
= \smallerdisplaystyleonehalf \hspace{-0.1ex} \bigl( \scalebox{0.93}[1]{$
	\locationvector_{\differentialindex{i}\hspace{.2ex}\differentialindex{\hspace{-0.1ex}j}} \hspace{-0.2ex} \dotp \locationvector_\differentialindex{k} \hspace{-0.16ex}
	+ \locationvector_{\differentialindex{i}\hspace{.2ex}\differentialindex{k}} \hspace{-0.2ex} \dotp \locationvector_\differentialindex{\hspace{-0.1ex}j}
$} \bigr) \hspace{-0.16ex}
+ \smallerdisplaystyleonehalf \hspace{-0.1ex} \bigl( \scalebox{0.93}[1]{$
	\locationvector_{\differentialindex{\hspace{-0.1ex}j}\hspace{.2ex}\differentialindex{i}} \hspace{-0.2ex} \dotp \locationvector_\differentialindex{k} \hspace{-0.16ex}
	+ \locationvector_{\differentialindex{\hspace{-0.1ex}j}\hspace{.2ex}\differentialindex{k}} \hspace{-0.2ex} \dotp \locationvector_\differentialindex{i}
$} \bigr) \hspace{-0.16ex}
- \smallerdisplaystyleonehalf \hspace{-0.1ex} \bigl( \scalebox{0.93}[1]{$
	\locationvector_{\differentialindex{k}\hspace{.2ex}\differentialindex{i}} \hspace{-0.2ex} \dotp \locationvector_\differentialindex{\hspace{-0.1ex}j} \hspace{-0.16ex}
	+ \locationvector_{\differentialindex{k}\hspace{.2ex}\differentialindex{\hspace{-0.1ex}j}} \hspace{-0.2ex} \dotp \locationvector_\differentialindex{i}
$} \bigr) \hspace{-0.2ex} =
\\[-0.25em]
%
= \smalldisplaystyleonehalf \hspace{-0.4ex} \left(^{\mathstrut} \hspace{-0.2ex}
\partial_i ( \locationvector_\differentialindex{\hspace{-0.1ex}j} \hspace{-0.2ex} \dotp \locationvector_\differentialindex{k} ) \hspace{-0.16ex}
+ \partial_j ( \locationvector_\differentialindex{i} \hspace{-0.2ex} \dotp \locationvector_\differentialindex{k} ) \hspace{-0.16ex}
- \partial_k ( \locationvector_\differentialindex{i} \hspace{-0.2ex} \dotp \locationvector_\differentialindex{\hspace{-0.1ex}j} )
\hspace{-0.12ex} \right) \hspace{-0.4ex} =
\\[-0.25em]
%
= \smalldisplaystyleonehalf \hspace{-0.3ex} \left(
\partial_i \hspace{.12ex} \textsl{g}_{j\hspace{-0.1ex}k} \hspace{-0.2ex}
+ \partial_j \hspace{.1ex} \textsl{g}_{ik} \hspace{-0.2ex}
- \partial_k \hspace{.12ex} \textsl{g}_{i\hspace{-0.1ex}j}
\right) \hspace{-0.4ex} .
\end{multline}

Все символы Christoffel’я тождественно равны нулю лишь в~ортонормальной~(декартовой) системе.
\textcolor{magenta}{(А~какие они для косоугольной?)}

Дальше:
${d\locationvector_\differentialindex{i} \hspace{-0.2ex}
= d\locationvector \dotp \hspace{-0.2ex} \boldnabla \locationvector_\differentialindex{i} \hspace{-0.2ex}
= dq^{\hspace{.1ex}k} \partial_k \locationvector_\differentialindex{i} \hspace{-0.2ex}
= \locationvector_{\differentialindex{k}\hspace{.2ex}\differentialindex{i}} \hspace{.2ex} dq^{\hspace{.1ex}k}\hspace{-0.25ex}}$\:--- тоже полные дифференциалы.
\[
d\locationvector_\differentialindex{k} \hspace{-0.2ex}
= \partial_i \locationvector_\differentialindex{k} \hspace{.15ex} dq^i \hspace{-0.3ex}
= \scalebox{0.84}{$ \displaystyle\frac{\raisemath{-0.2ex}{\partial \hspace{.1ex} \locationvector_\differentialindex{k}}}{\raisemath{-0.3ex}{\partial q^1}} $} \hspace{.2ex} dq^1 \hspace{-0.2ex}
+ \scalebox{0.84}{$ \displaystyle\frac{\raisemath{-0.2ex}{\partial \hspace{.1ex} \locationvector_\differentialindex{k}}}{\raisemath{-0.3ex}{\partial q^2}} $} \hspace{.2ex} dq^2 \hspace{-0.2ex}
+ \scalebox{0.84}{$ \displaystyle\frac{\raisemath{-0.2ex}{\partial \hspace{.1ex} \locationvector_\differentialindex{k}}}{\raisemath{-0.3ex}{\partial q^3}} $} \hspace{.2ex} dq^3 \hspace{-0.2ex}
\]
Поэтому ${\partial_i \partial_j \locationvector_\differentialindex{k} \hspace{-0.2ex} = \partial_j \partial_i \locationvector_\differentialindex{k}}$, ${\partial_i \locationvector_{\differentialindex{\hspace{-0.1ex}j}\hspace{.2ex}\differentialindex{k}} \hspace{-0.2ex} = \partial_j \locationvector_{\differentialindex{i}\hspace{.2ex}\differentialindex{k}}}$,
и~трёхиндексный объект из~векторов третьих частных производных

\nopagebreak\vspace{-0.25em}
\begin{equation}
\locationvector_{\differentialindex{i}\hspace{.2ex}\differentialindex{\hspace{-0.1ex}j}\hspace{.2ex}\differentialindex{k}} \hspace{-0.1ex} \equiv \hspace{.1ex} \partial_i \partial_j \partial_k \locationvector
= \partial_i \hspace{.12ex} \locationvector_{\differentialindex{\hspace{-0.1ex}j}\hspace{.2ex}\differentialindex{k}}
\end{equation}

\vspace{-0.24em} \noindent
симметричен по~первому и~второму индексам (а~не~только по~второму и~третьему).
И~тогда равен нулю~${\hspace{-0.16ex}^4\bm{0}}$ следующий тензор четвёртой сложности\:---
\href{https://en.wikipedia.org/wiki/Riemann_curvature_tensor}{ \emph{\ru{тензор кривизны }Riemann\ru{’а}\en{ curvature tensor}} (\en{or}\ru{или}~\emph{\ru{тензор }Riemann\ru{’а}\hbox{--}Christoffel\ru{’я}\en{ tensor}}) }
%% Римана\hbox{--}Христоффеля

\nopagebreak\vspace{-0.1em}\begin{equation}\label{riemanncurvaturetensor}
{^4\bm{\mathfrak{R}}} = \hspace{.12ex} \mathfrak{R}_{\hspace{.1ex}hi\hspace{-0.1ex}j\hspace{-0.1ex}k} \hspace{.12ex} \locationvector^h \locationvector^i \locationvector^j \locationvector^k \hspace{-0.25ex},
\:\:
\mathfrak{R}_{\hspace{.1ex}hi\hspace{-0.1ex}j\hspace{-0.1ex}k} \hspace{-0.12ex}
\equiv
\locationvector_\differentialindex{h} \hspace{-0.15ex} \dotp \left( \hspace{.12ex} \locationvector_{\differentialindex{\hspace{-0.1ex}j}\hspace{.2ex}\differentialindex{i}\hspace{.2ex}\differentialindex{k}} \hspace{-0.2ex} - \locationvector_{\differentialindex{i}\hspace{.2ex}\differentialindex{\hspace{-0.1ex}j}\hspace{.2ex}\differentialindex{k}} \hspace{.12ex} \right)
\hspace{-0.3ex} .
\end{equation}

Выразим компоненты~${\mathfrak{R}_{\hspace{.1ex}i\hspace{-0.1ex}j\hspace{-0.1ex}kn}}$ через метрическую матрицу~${\textsl{g}_{i\hspace{-0.1ex}j}}$.
Начнём с~дифференцирования локального кобазиса:

\[
\locationvector^i \hspace{-0.32ex} \dotp \locationvector_\differentialindex{k} \hspace{-0.16ex} = \delta_k^{\hspace{.1ex}i}
\;\Rightarrow\:
\partial_j \locationvector^i \hspace{-0.32ex} \dotp \locationvector_\differentialindex{k} \hspace{-0.15ex} + \locationvector^i \hspace{-0.32ex} \dotp \locationvector_{\differentialindex{\hspace{-0.1ex}j}\hspace{.2ex}\differentialindex{k}} \hspace{-0.15ex} = 0
\;\Rightarrow\:
\partial_j \locationvector^i \hspace{-0.12ex} = - \hspace{.2ex} \Gamma_{\hspace{-0.25ex}j\hspace{-0.1ex}k}^{\hspace{.25ex}i} \hspace{.2ex} \locationvector^k
\hspace{-0.4ex} .
\]

...

Шесть независимых компонент:
${\mathfrak{R}_{\hspace{.1ex}1212}}$, ${\mathfrak{R}_{\hspace{.1ex}1213}}$, ${\mathfrak{R}_{\hspace{.1ex}1223}}$, ${\mathfrak{R}_{\hspace{.1ex}1313}}$, ${\mathfrak{R}_{\hspace{.1ex}1323}}$, ${\mathfrak{R}_{\hspace{.1ex}2323}}$.

...

\en{Symmetric}\ru{Симметричный} \en{bivalent}\ru{бивалентный} \href{https://en.wikipedia.org/wiki/Ricci_curvature}{\emph{\ru{тензор кривизны }Ricci\en{ curvature tensor}}}
%% after \href{https://en.wikipedia.org/wiki/Gregorio_Ricci-Curbastro}{\textbold{Gregorio Ricci\hbox{-}Curbastro}}

\begin{equation*}
\hspace{.1ex}\pmb{\scalebox{1.2}[1]{$\mathscr{R}$}} \equiv
\smalldisplaystyleonefourth \hspace{.4ex} \mathfrak{R}_{\hspace{.1ex}abi\hspace{-0.1ex}j} \hspace{.2ex} \locationvector^a \hspace{-0.33ex} \times \hspace{-0.1ex} \locationvector^b \locationvector^i \hspace{-0.33ex} \times \hspace{-0.1ex} \locationvector^j \hspace{-0.25ex}
= \smalldisplaystyleonefourth \hspace{.15ex} \levicivita^{abp} \levicivita^{i\hspace{-0.1ex}j\hspace{-0.1ex}q} \hspace{.25ex} \mathfrak{R}_{\hspace{.1ex}abi\hspace{-0.1ex}j} \hspace{.2ex} \locationvector_\differentialindex{p} \locationvector_\differentialindex{q} \hspace{-0.2ex}
= \mathscr{R}^{\hspace{.1ex}pq} \hspace{.1ex} \locationvector_\differentialindex{p} \locationvector_\differentialindex{q}
\end{equation*}

\vspace{-0.2em} \noindent
(\en{coefficient}\ru{коэффициент}~$\onefourth$ \en{is used here for convenience}\ru{используется тут для удобства}) \en{with components}\ru{с~компонентами}

\vspace{.1em}\begin{equation*}
\begin{array}{ccc}
\mathscr{R}^{\hspace{.1ex}1\hspace{-0.1ex}1} \hspace{-0.3ex} =
\scalebox{0.8}{$ \displaystyle \frac{\raisemath{-0.2em}{1}}{\raisemath{.15em}{\smash{\textsl{g}}}} $} \hspace{.4ex} \mathfrak{R}_{\hspace{.1ex}2323}
\hspace{.2ex} ,
&
&
\\[.6em]
%
\mathscr{R}^{\hspace{.1ex}21} \hspace{-0.3ex} =
\scalebox{0.8}{$ \displaystyle \frac{\raisemath{-0.2em}{1}}{\raisemath{.15em}{\smash{\textsl{g}}}} $} \hspace{.4ex} \mathfrak{R}_{\hspace{.1ex}1323}
\hspace{.2ex} ,
&
\mathscr{R}^{\hspace{.1ex}22} \hspace{-0.3ex} =
\scalebox{0.8}{$ \displaystyle \frac{\raisemath{-0.2em}{1}}{\raisemath{.15em}{\smash{\textsl{g}}}} $} \hspace{.4ex} \mathfrak{R}_{\hspace{.1ex}1313}
\hspace{.2ex} ,
&
\\[.6em]
%
\mathscr{R}^{\hspace{.1ex}31} \hspace{-0.3ex} =
\scalebox{0.8}{$ \displaystyle \frac{\raisemath{-0.2em}{1}}{\raisemath{.15em}{\smash{\textsl{g}}}} $} \hspace{.4ex} \mathfrak{R}_{\hspace{.1ex}1223}
\hspace{.2ex} ,
&
\mathscr{R}^{\hspace{.1ex}32} \hspace{-0.3ex} =
\scalebox{0.8}{$ \displaystyle \frac{\raisemath{-0.2em}{1}}{\raisemath{.15em}{\smash{\textsl{g}}}} $} \hspace{.4ex} \mathfrak{R}_{\hspace{.1ex}1213}
\hspace{.2ex} ,
&
\mathscr{R}^{\hspace{.1ex}33} \hspace{-0.3ex} =
\scalebox{0.8}{$ \displaystyle \frac{\raisemath{-0.2em}{1}}{\raisemath{.15em}{\smash{\textsl{g}}}} $} \hspace{.4ex} \mathfrak{R}_{\hspace{.1ex}1212}
\hspace{.2ex} .
\end{array}
\end{equation*}

Равенство тензора Риччи нулю
${\hspace{.1ex}\pmb{\scalebox{1.2}[1]{$\mathscr{R}$}} \hspace{-0.16ex} = \hspace{-0.2ex} {^2\bm{0}}}$ (в~компонентах это шесть уравнений ${\hspace{.1ex}\mathscr{R}^{\hspace{.1ex}i\hspace{-0.1ex}j} \hspace{-0.3ex} = \mathscr{R}^{\hspace{.1ex}j\hspace{-0.06ex}i} \hspace{-0.3ex} = 0}$) \en{is}\ru{есть} \en{the~}\textcolor{magenta}{\en{necessary}\ru{необходимое}} \en{condition}\ru{условие} \en{of~integrability}\ru{интегрируемости}~(\ru{\inquotes{совместности}, }\inquotes{compatibility}) для нахождения вектора-радиуса~${\locationvector(q^{\hspace{.1ex}k})}$ по~полю~${\textsl{g}_{i\hspace{-0.1ex}j}(q^{\hspace{.1ex}k})}$.

\end{otherlanguage}

