\en{\chapter{Thin\hbox{-}walled rods}}

\ru{\chapter{Тонкостенные стержни}}

\thispagestyle{empty}

\label{chapter:thinwalledrods}

\en{\section{Variational approach}}

\ru{\section{Вариационный подход}}

\dropcap{\en{I}\ru{В}}{\en{n}\ru{\hspace{-0.2ex}}} \customref[\en{chapter}\ru{главе}~]{chapter:rods}\en{,} \ru{были описаны }\en{rods}\ru{стержни} \en{with massive cross-sections}\ru{с~массивными сечениями}\en{ were described}.
\en{But other rods are also widely used}\ru{Но широко используются и~иные стержни}\:--- \en{thin\hbox{-}walled ones}\ru{тонкостенные}, \en{with sections}\ru{с~сечениями} \en{of narrow bands}\ru{из узких полосок} \en{of various shapes}\ru{разных очертаний}: \en{corner}\ru{уголок}, Z\hbox{-}\en{beam}\ru{балка}, \en{channel}\ru{швеллер}~(C\hbox{-}\en{beam}\ru{балка}), \en{I\hbox{-}beam}\ru{двутавр}, ...
\en{If rods are like lines}\ru{Если стержни похожи на~линии}, \en{then}\ru{то} \en{in a~thin-walled rod}\ru{в~тонкостенном стержне} \ru{сам\'{о} сечение}\en{the section itself} \en{also looks like line}\ru{тоже выглядит как~линия}.
\en{The three dimensions}\ru{Три размера}\:--- \en{the thickness and the length of a~section}\ru{толщина и~длина сечения}, \en{as well as}\ru{а~также} \en{the length of a~rod}\ru{длина стержня}\:--- \en{have different decimal orders}\ru{имеют разные десятичные порядки}.

%%(\ru{материальные линии }Cosserat\en{ material lines})

\en{The applied theories of thin-walled rods are known}\ru{Известны прикладные теории тонкостенных стержней}, \en{they are described}\ru{они описаны}, \en{for example}\ru{например}, \en{in books}\ru{в~книгах} ...

\en{There is also the exact theory}\ru{Есть также точная теория}, \en{based on asymptotic splitting}\ru{основанная на асимптотическом расщеплении} \en{of the three-dimensional problem}\ru{трёхмерной проблемы}~\cite{eliseev-models}.
\en{In sum}\ru{В~итоге}, \en{the complex asymptotic analysis}\ru{сложный асимптотический анализ} \en{confirmed}\ru{подтвердил} \en{most of the hypotheses}\ru{большинство из гипотез} \en{of applied theories}\ru{прикладных теорий}.

\en{As introduction to the mechanics of thin-walled rods}\ru{Как введение в~механику тонкостенных стержней}, \en{here is}\ru{вот} \en{the following variational procedure}\ru{следующая вариационная процедура} \en{with warping (deplanation) of cross-sections}\ru{с~искажением (депланацией) поперечных сечений}.

\begin{otherlanguage}{russian}

\en{Consider}\ru{Рассмотрим} \en{the simplest case of a~cylindrical rod}\ru{простейший случай цилиндрического стержня} \en{with a~thin, simply connected section}\ru{с~тонким односвязным сечением} \textcolor{blue}{(... рисунок ...)}.
Площадь сечения $do$, боковая поверхность свободна, нагрузка в~объёме\:--- $\bm{f}$, на~торце ${z \narroweq z_{1}}$ известны поверхностные силы ${\bm{p}(\bm{x})}$ (${\bm{x} \narroweq x_{\alpha}\bm{e}_{\alpha}}$), торец ${z \narroweq z_{0}}$ закреплён.

...

\end{otherlanguage}

\en{\section{Equations with small parameter}}

\ru{\section{Уравнения с малым параметром}}

\begin{otherlanguage}{russian}

Рассмотрим призматический стержень с~односвязным сечением в~виде тонкой криволинейной полоски постоянной толщины~$h$. Радиус\hbox{-}вектор в~объёме предст\'{а}вим следующим образом:

...



\end{otherlanguage}

\en{\section{First step of the asymptotic procedure}}

\ru{\section{Первый шаг асимптотической процедуры}}

\begin{otherlanguage}{russian}

\subsection*{\en{Outer decomposition}\ru{Внешнее разложение}}

Из~системы

...

\subsection*{\en{Inner decomposition near}\ru{Внутреннее разложение вблизи}~$s_0$}

Выпишем уравнения для

...

\subsection*{\en{Coalescence}\ru{Сращивание}}

Стыковка внутреннего и~внешнего разложений

...



\end{otherlanguage}

\en{\section{Second step}}

\ru{\section{Второй шаг}}

\begin{otherlanguage}{russian}

\subsection*{\en{Outer decomposition}\ru{Внешнее разложение}}

Из~системы

...

\subsection*{\en{Inner decomposition near}\ru{Внутреннее разложение вблизи}~${s \narroweq s_0}$}

Из~общей системы

...

\subsection*{\en{Coalescence}\ru{Сращивание}}

Поскольку рассматриваются поправочные члены асимптотических разложений

...



\end{otherlanguage}

\en{\section{Third step}}

\ru{\section{Третий шаг}}

\begin{otherlanguage}{russian}

\subsection*{\en{Outer decomposition}\ru{Внешнее разложение}}

Из~системы

....

\subsection*{\en{Inner decomposition around}\ru{Внутреннее разложение около}~${s \narroweq s_0}$}

Как уж\'{е} отмечалось, внутренние разложения нужны для постановки краевых условий на~концах

...

\subsection*{\en{Coalescence}\ru{Сращивание}}

В~плоской задаче имеем следующее двучленное внешнее разложение:

....



\end{otherlanguage}

\en{\section{Fourth step}}

\ru{\section{Четвёртый шаг}}

\begin{otherlanguage}{russian}

Здесь понадобится лишь внешнее разложение. Более того: в~этом приближении мы не~будем искать решения уравнений\:--- будет достаточно лишь условий разрешимости. Напомним, что философия наша такова: разыскиваются лишь главные члены асимптотических разложений, но для~полного их определения могут понадобиться

...



\end{otherlanguage}

\section{\en{Displacements}\ru{Смещения}}

\begin{otherlanguage}{russian}

Расписывая тензорное соотношение

...



\end{otherlanguage}



\en{\section{Results of asymptotic analysis}}

\ru{\section{Результаты асимптотического анализа}}
% Итоги асимптотического анализа

\begin{otherlanguage}{russian}

Нахождение главных членов асимптотики
\en{of~stresses}\ru{напряжений}
\en{and}\ru{и}~\en{displacements}\ru{смещений}
для~тонкостенных стержней
оказалось
намного сложнее,
чем для массивных сечений.
Дадим сводку полученных выше итоговых результатов.

Смещение:

\begin{equation}
\withtheindexofperpendicularity{\fieldofdisplacements}
=
\lambda^{-4}
\bigl(
   \withtheindexofperpendicularity{\bm{U}} (z)
   +
   \theta (z) \bm{k} \times \bm{r}
\bigr)
+
\hspace{-0.1ex} \scalebox{.9}{$
   \infty^{\hspace{-0.4ex}\expminusone}
$}(\lambda^{-3})
\hspace{.1ex} ,
\end{equation}
%
\begin{equation}
\fieldofdisplacementscomponents{z}
=
\lambda^{-3}
\bigl(
   {}- \withtheindexofperpendicularity{\mathdotabove{\bm{U}}
   \dotp
   \bm{r}
   +
   \mathdotabove{\theta}
   \omega (s)
   +
   U_z (z)
\bigr)
+
\hspace{-0.1ex} \scalebox{.9}{$
   \infty^{\hspace{-0.4ex}\expminusone}
$}(\lambda^{-2})
\hspace{.1ex} .
\end{equation}

Напряжения:

...


\end{otherlanguage}

\vspace{8mm}
\hfill\begin{minipage}[b]{0.95\linewidth}
\fontsize{10}{12}\selectfont

\section*{\wordforbibliography}

\begin{otherlanguage}{russian}

Помимо книг
В.\,З.\;Власов’а~\cite{vlasov-thinwalledrods},
Г.\,Ю.\;Джанелидзе и~Я.\,Г.\;Пановко~\cite{janelidzepanovko-thinwalledrods},
отметим
статью
Я.\,Г.\;Пановко и~Е.\,А.\;Бейлина~\cite{panovko.beylin-thinwalledrods}.

Материал этой главы содержится
в~диссертации
Владимира~В.\;Елисеева~\cite{eliseev-models},
там также есть обширный список статей
на тему
тонкостенных стержней.

\end{otherlanguage}

\end{minipage}

