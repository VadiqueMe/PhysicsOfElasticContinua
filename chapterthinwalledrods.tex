\en{\chapter{Thin\hbox{-}walled rods}}

\ru{\chapter{Тонкостенные стержни}}

\thispagestyle{empty}

\label{chapter:thinwalledrods}

\en{\section{Variational approach}}

\ru{\section{Вариационный подход}}

\begin{otherlanguage}{russian}

\dropcap{В}{\hspace{-0.2ex}} \customref[главе~]{chapter:rods} рассматривались стержни с~массивным сечением. Но в~технике широко используются иные стержни\:--- тонкостенные, сечения которых представляют собой узкие полоски различного очертания~(уголок, швеллер, двутавр и~др.). Если стержни похожи на~линии~(материальные линии Коссера), то в~тонкостенных стержнях и~сам\'{о} сечение выглядит как~линия. Три размера\:--- толщина и~длина сечения, а~также длина стержня\:--- имеют различные порядки.

Известны прикладные теории тонкостенных стержней ...

...



\end{otherlanguage}

\en{\section{Equations with small parameter}}

\ru{\section{Уравнения с малым параметром}}

\begin{otherlanguage}{russian}

Рассмотрим призматический стержень с~односвязным сечением в~виде тонкой криволинейной полоски постоянной толщины~$h$. Радиус\hbox{-}вектор в~объёме предст\'{а}вим следующим образом:

...



\end{otherlanguage}

\en{\section{First step of the asymptotic procedure}}

\ru{\section{Первый шаг асимптотической процедуры}}

\begin{otherlanguage}{russian}

\subsection*{\en{Outer decomposition}\ru{Внешнее разложение}}

Из~системы

...

\subsection*{\en{Inner decomposition near}\ru{Внутреннее разложение вблизи}~$s_0$}

Выпишем уравнения для

...

\subsection*{\en{Coalescence}\ru{Сращивание}}

Стыковка внутреннего и~внешнего разложений

...



\end{otherlanguage}

\en{\section{Second step}}

\ru{\section{Второй шаг}}

\begin{otherlanguage}{russian}

\subsection*{\en{Outer decomposition}\ru{Внешнее разложение}}

Из~системы

...

\subsection*{\en{Inner decomposition near}\ru{Внутреннее разложение вблизи}~${s \narroweq s_0}$}

Из~общей системы

...

\subsection*{\en{Coalescence}\ru{Сращивание}}

Поскольку рассматриваются поправочные члены асимптотических разложений

...



\end{otherlanguage}

\en{\section{Third step}}

\ru{\section{Третий шаг}}

\begin{otherlanguage}{russian}

\subsection*{\en{Outer decomposition}\ru{Внешнее разложение}}

Из~системы

....

\subsection*{\en{Inner decomposition around}\ru{Внутреннее разложение около}~${s \narroweq s_0}$}

Как уж\'{е} отмечалось, внутренние разложения нужны для постановки краевых условий на~концах

...

\subsection*{\en{Coalescence}\ru{Сращивание}}

В~плоской задаче имеем следующее двучленное внешнее разложение:

....



\end{otherlanguage}

\en{\section{Fourth step}}

\ru{\section{Четвёртый шаг}}

\begin{otherlanguage}{russian}

Здесь понадобится лишь внешнее разложение. Более того: в~этом приближении мы не~будем искать решения уравнений\:--- будет достаточно лишь условий разрешимости. Напомним, что философия наша такова: разыскиваются лишь главные члены асимптотических разложений, но для~полного их определения могут понадобиться

...



\end{otherlanguage}

\ru{\section{Перемещения}}

\en{\section{Displacements}}

\begin{otherlanguage}{russian}

Расписывая тензорное соотношение

...



\end{otherlanguage}

\en{\section{Results of asymptotic analysis}}

\ru{\section{Итоги асимптотического анализа}}

\begin{otherlanguage}{russian}

Определение главных членов асимптотики напряжений и~перемещений для~тонкостенных стержней оказалось намного сложнее, чем в~случае массивного сечения. Дадим сводку полученных выше итоговых результатов.

Перемещение:

...




\end{otherlanguage}

\vspace{8mm}
\hfill\begin{minipage}[b]{0.95\linewidth}
\fontsize{10}{12}\selectfont

\section*{\wordforbibliography}

\begin{otherlanguage}{russian}

Помимо известных книг ... Материал главы содержится в~\cite{eliseev-models}, где можно найти и~обширный список статей.

\end{otherlanguage}

\end{minipage}
