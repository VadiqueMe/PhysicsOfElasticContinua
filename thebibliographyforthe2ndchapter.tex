\section*{\small \wordforbibliography}

\begin{changemargin}{\parindent}{0pt}
\fontsize{10}{12}\selectfont

\en{In a~long list}\ru{В~длинном списке}
\en{of the books}\ru{книг}
\en{about the classical mechanics}\ru{про классическую механику}\en{,}
\en{the~reader}\ru{читатель}
\en{can find}\ru{может найти}
\en{the works}\ru{работы}
\en{of both}\ru{и}
\en{the specialists in mechanics}\ru{специалистов по механике}~\cite{goldstein-classicalmechanics, treatiseonanalyticaldynamics-by-l.a.pars, loitsjanskiy.lurie, lurie-analyticalmechanics, olkhovskiy-theoreticalmechanicsforphysicists}\ru{,}
\en{and}\ru{и}
\en{the broadly oriented}\ru{широко ориентированных}
\en{theoretical physicists}\ru{физиков\hbox{-}теоретиков}~\cite{landau.lifshitz-shortcourse, terhaar-hamiltonianmechanics}.
%
\ru{Весьма интересна }\en{The book}\ru{книга}
\en{by Felix~R.\;Gantmacher (\russianlanguage{Феликс~Р.\;Гантмахер})}\ru{Феликса~Р.\;Гантмахер’а}~\cite{gantmacher-analyticalmechanics}
\en{with the~compact but complete}\ru{с~компактным, но~полным}
\en{narration}\ru{изложением}
\en{of the fundamentals}\ru{основ}\en{ is pretty interesting}.

\end{changemargin}

