\en{\section{Hooke’s law for an isotropic medium}}

\ru{\section{Закон Hooke’а для изотропной среды}}

\label{para:hooke.forisotropic}


\en{a~function}\ru{функция}
\en{with arguments}\ru{с~аргументами}
\en{which are only invariants}\ru{суть только инварианты}


\en{For an~isotropic material}\ru{Для изотропного материала}\en{,}
 \eqref{the Hooke's law for symmetric crystals}
 \en{is satisfied}\ru{удовлетворяется}
\en{for any orthogonal tensor}\ru{для любого ортогонального тензора}~$\orthogonaltensor$.
\en{It’s easier to accept that}\ru{Проще принять, что}
\en{the elastic potential}\ru{упругий потенциал}~${ \potential(\lineardeformation) }$
\en{becomes an isotropic function}\ru{становится изотропной функцией},
\en{which depends only on invariants}\ru{которая зависит лишь от инвариантов}
\en{of the solution}\ru{решения}
\en{of the characteristic equation}\ru{характеристического уравнения}~\eqref{eigenvalues:eq}.

"квадратичная форма" = "однородный многочлен второй степени" = функция с~только квадратичными членами

\en{the function}\ru{функция}
\en{with the members}\ru{с членами}
\en{of the second degree}\ru{второй степени}

............


for the various pairs of two elastic constants (modules)

\en{Below}\ru{Ниже}
\en{are presented}\ru{представлены}
\en{the versions}\ru{версии}
\en{of the }\ru{закона }Hooke’\ru{а}\en{s}\en{ law}
\en{for}\ru{для}
\en{the different pairs}\ru{разных пар}
\en{of elastic constants}\ru{упругих констант}
(\en{elastic moduli}\ru{модулей упругости}).
$\lambda$ \en{and}\ru{и}~$\mu$\:--- параметры Lame,
$\mu$\:--- \ru{модуль сдвига}\en{the shear modulus},
$\poissonratio$\:--- \en{the }\ru{коэффициент }Poisson’\ru{а}\en{s}\en{ ratio},
$K$\:--- \en{the volumetric modulus}\ru{объёмнометрический модуль},
$E$\:--- \en{the Young's modulus}\ru{модуль Юнга}.

\en{A~priori limits}\ru{Априорные пределы}
\en{for values}\ru{для значений}
\en{of the elastic moduli}\ru{модулей упругости}

\noindent
\begin{equation}\label{inequalitiesforelasticmoduli}
E > 0 \hspace{.2ex} , \hspace{1em}
\mu > 0 \hspace{.2ex} , \hspace{1em}
K > 0 \hspace{.2ex}
.
\end{equation}

\noindent
Растягиваемый стержень удлиняется;
сдвиг направлен в~ту~же сторону, что и касательное напряжение;
объём уменьшается от внешнего давления.
Неравенств~\eqref{inequalitiesforelasticmoduli}
достаточно для положительности~$\potential$.

..........

При~${ \poissonratio \mapsto \smalldisplaystyleonehalf }$
материал становится несжимаемым, ${ K \mapsto \infty }$.

\en{For an~isotropic material}\ru{Для изотропного материала}
( \eqref{Hooke's law for transversely isotropic material} )
\en{is satisfied}\ru{удовлетворяется}
\en{a~priori}\ru{априори}
\en{with any}\ru{с~любым}
\en{orthogonal tensor}\ru{ортогональным тензором}
$\bm{Q}$.
Проще принять\ru{,}
\en{that}\ru{что}~$\potential(\infinitesimaldeformation)$
\en{becomes an~isotropic function}\ru{становится изотропной функцией},
\en{which depends}\ru{которая зависит}
\en{only}\ru{лишь}
\en{on the invariants from the solution}\ru{от инвариантов из решения}
\en{of the characteristic equation}\ru{характеристического уравнения}~\chapdotpararef{chapter:mathapparatus}{para:eigenvectorseigenvalues}

\begin{equation}
2 \potential = \alpha \mathrm{I}_1 ^2 + \beta \mathrm{II}_2 .
\end{equation}

$\potential(\infinitesimaldeformation)$
\en{это квадратичная функция}\ru{is a~quadratic function}
\en{or}\ru{или} \en{a~}\inquotes{\en{quadratic form}\ru{квадратичная форма})
\en{with terms}\ru{с~членами}
\en{of the second degree}\ru{второй степени}.

\begin{equation}
2 \potential( \lineardeformation )
= A \lineardeformation_\tracedot \lineardeformation_\tracedot
+ B \lineardeformation \dotp \lineardeformation
\end{equation}

.........

\nopagebreak\vspace{-0.1em}
\begin{gather}\label{the Legendre transform}
\infinitesimaldeformation
= \scalebox{.9}{$
   \displaystyle
   \frac{
      { \raisemath{-0.2em}{ \partial \hspace{.1ex} \widehat{\potential} } }{ \raisemath{.04em}{\partial \linearstress} }
   }
$}
= \hspace{-0.12ex} \pliabilitytensor \dotdotp \linearstress
\hspace{.1ex} =
\linearstress \dotdotp \hspace{-0.1ex} \pliabilitytensor
\hspace{.1ex} ,
\end{gather}

%
% the complementary energy
%
\begin{equation}\label{the complementary energy}
\widehat{\potential}( \hspace{-0.1ex} \linearstress \hspace{.12ex} ) \hspace{-0.1ex}
= \linearstress \dotdotp \infinitesimaldeformation
- \hspace{.1ex} \potential(\infinitesimaldeformation)
\hspace{.1ex} .
\end{gather}

\chapdotpararef{chapter:mathapparatus}{para:tensors.symmetric+skewsymmetric}

\eqref{eigenvalues:eq}

\subsection*{The elastic potential for a linear isotropic medium (body)}

The function

\noindent
\begin{equation*}
\mathrm{I}_1,
\mathrm{II}_2
\mapsto
\potential
\end{equation*}

\noindent
\begin{equation*}
\potential =
\potential(\mathrm{I}_1, \mathrm{II}_2)
\end{equation*}

has the terms of the only 2^{nd} degree, not more.
Thus the third invariant $\mathrm{III}_3$
is out of the arguments of $\potential$.

%обратные отношения
\begin{equation}\label{The backward relations of the Hooke's law for an isotropic media}
2 \potential = \linearstress \dotdotp \infinitesimaldeformation,
\hspace{1em}
\infinitesimaldeformation = \scalebox{.92}{$
   \displaystyle
   \frac{ \raisemath{-0.133em}{
      \partial \hspace{.1ex} \potential^{\mathstrut}
   }%close \raisemath
   }%close \frac
   { \raisemath{-0.07em}{\partial \linearstress} }
$}
=
= \pliabilitytensor \dotdotp \linearstress
\end{equation}

\en{In the linear theory}\ru{В~линейной теории}\en{,}
\en{the~}\inquotes{ \en{complementary energy}\ru{дополнительная энергия} }
\en{is numerically equal to}\ru{численно равна}
\en{the }\en{elastic potential}\ru{упругому потенциалу}}

\begin{equation*}
\begintikzmark{twoElasticPotentials}
2 \potential
\endtikzmark{twoElasticPotentials}
-
\begintikzmark{oneElasticPotential}
\potential(\infinitesimaldeformation)
\endtikzmark{oneElasticPotential}
= \widehat{\potential}(\linearstress).
\end{equation*}
\AddUnderBrace{oneElasticPotential}{$ \linearstress \dotdotp \infinitesimaldeformation $}
\AddUnderBrace{twoElasticPotentials}{$ \smalldisplaystyleonehalf \linearstress \dotdotp \infinitesimaldeformation $}

\begin{equation*}
\widehat{\potential}(\linearstress) = \potential(\infinitesimaldeformation)
\end{equation*}

\en{An~isotropic medium}\ru{Изотропная среда}
\en{has}\ru{имеет}
\en{the two non-zero elastic constants}\ru{две ненулевые упругие константы},
\en{the }\inquotes{\en{elastic moduli}\ru{модули упругости}}.

\noindent
\begin{equation}\label{theelasticpotentialforanisotropicmedium}
\potential(\infinitesimaldeformation) =
\alpha \mathrm{I}_1^2 (\infinitesimaldeformation)
+ \beta \mathrm{II}_2 (\infinitesimaldeformation)
\end{equation}

\noindent
\begin{gather*}
\scalebox{.92}{$
   \displaystyle
   \frac{ \raisemath{-0.133em}{
      \partial
   }%close \raisemath
   }%close \frac
   { \raisemath{-0.07em}{\partial \linearstress} }
$}
\hspace{.1ex} \potential^{\mathstrut}
\Bigl(
   \scalebox{.92}{$
   \displaystyle
   \frac{ \raisemath{-0.133em}{
      \partial \infinitesimaldeformation_\tracedot
   }%close \raisemath
   }%close \frac
   { \raisemath{-0.07em}{\partial \infinitesimaldeformation} }
   \dotdotp
   \infinitesimaldeformation_\tracedot
   +
   \infinitesimaldeformation_\tracedot
   \dotdotp
   \scalebox{.92}{$
   \displaystyle
   \frac{ \raisemath{-0.133em}{
      \partial \infinitesimaldeformation_\tracedot
   }%close \raisemath
   }%close \frac
   { \raisemath{-0.07em}{\partial \infinitesimaldeformation} }
   $}
\Bigr)
\\
%
=
2 \infinitesimaldeformation_\tracedot \dotdotp \UnitDyad
\end{gather*}

\noindent
\begin{equation}\label{the derivative of trace by the tensor itself}
\scalebox{.92}{$
   \displaystyle
   \frac{ \raisemath{-0.133em}{
      \partial \infinitesimaldeformation_\tracedot
   }%close \raisemath
   }%close \frac
   { \raisemath{-0.07em}{\partial \infinitesimaldeformation} }
$}
=
\UnitDyad
\end{equation}

\en{In the components}\ru{В~компонентах}
\en{for an~isotropic medium}\ru{для изотропной среды}
\en{we have}\ru{имеем}

\begin{equation}\label{componentsofthestiffnesstensor.foranisotropicbody}
A_{ijpq} =
\lambda
\KroneckerDelta{ij}
\KroneckerDelta{pq}
+
\mu
\bigl(
\KroneckerDelta{ip}
\KroneckerDelta{jq}
+
\KroneckerDelta{iq}
\KroneckerDelta{jp}
\bigr)
\end{equation}\:---
\en{these are}\ru{это}
\en{components}\ru{компоненты}
\en{of an~isotropic tensor}\ru{изотропного тензора}
\en{of the fourth complexity}\ru{четвёртой сложности}.
\en{These components don’t change}\ru{Эти компоненты не меняются}
\en{when}\ru{когда}
\en{the basis rotates}\ru{базис вращается}.
