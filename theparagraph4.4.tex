\en{\section{Hooke’s law for an isotropic medium}}

\ru{\section{Закон Hooke’а для изотропной среды}}

\label{para:hooke.forisotropic}

\en{For an~isotropic material}\ru{Для изотропного материала}
( \eqref{Hooke's law for transversely isotropic material} )
\en{satisfied}\ru{удовлетворяется}
a~priori
\en{with any}\ru{с~любым}
\en{orthogonal tensor}\ru{ортогональным тензором}
$\bm{Q}$.
Проще принять,
что $\potential(\infinitesimaldeformation)$
становится изотропной функцией,
которая зависит
\en{only}\ru{лишь}
от главных инвариантов~\chapdotpararef{chapter:mathapparatus}{para:eigenvectorseigenvalues}:

\begin{equation}
\potential = A \mathrm{I}_1 ^2 + B \mathrm{II}_2
.
\end{equation}

$\potential(\infinitesimaldeformation)$ это квадратичная функция
(или \inquotes{форма})

...........

\potential = A \mathrm{I}_1 ^2 + B 


\begin{equation*}
................
\end{equation*}

.............
.............
.............
...............
...............
...............

%%\en{the potential energy density}\ru{плотность потенциальной энергии}
%%\en{from the internal forces}\ru{от внутренних сил}
%%$\potential$

%\en{the complementary energy}
%\ru{дополнительная энергия}

\en{In~the~linear theory}\en{,}\ru{В~линейной теории}
\en{the~}\inquotes{ \en{complementary energy}\ru{дополнительная энергия} }
\en{is numerically equal to}\ru{численно равна}
\en{the }\en{elastic potential}\ru{упругому потенциалу}}.

\nopagebreak\vspace{-0.1em}
\begin{equation}\label{the Legendre transform}
\begin{gathered}
\infinitesimaldeformation
= \scalebox{.9}{$
   \displaystyle
   \frac{ \raisemath{-0.2em}{
      \partial \hspace{.1ex} \widehat{\potential}
   } }
   { \raisemath{.04em}{\partial \linearstress} }
$}
= \hspace{-0.12ex} \pliabilitytensor \dotdotp \linearstress
\hspace{.1ex} =
\linearstress \dotdotp \hspace{-0.1ex} \pliabilitytensor
\hspace{.1ex} ,
\\
%the complementary energy
\widehat{\potential}( \hspace{-0.1ex}\linearstress\hspace{.12ex}) \hspace{-0.1ex}
= \linearstress \dotdotp \infinitesimaldeformation
- \hspace{.1ex} \potential(\infinitesimaldeformation)
\hspace{.1ex} .
\end{gathered}
\end{equation}\:---
\en{the }\ru{преобразование }Legendre\ru{’а}\en{ transform}\ru{ (Лежандра)}.

..............
............

\begin{equation*}
.........................
.,,......................
\end{equation*}

%\{chapter:mathapparatus}{para:tensors.symmetric+skewsymmetric})

