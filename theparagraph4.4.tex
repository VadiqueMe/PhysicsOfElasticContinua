\en{\section{Hooke’s law for an isotropic medium}}

\ru{\section{Закон Hooke’а для изотропной среды}}

\label{para:hooke.forisotropic}

\en{For an~isotropic material}\ru{Для изотропного материала}
( \eqref{Hooke's law for transversely isotropic material} )
\en{is satisfied}\ru{удовлетворяется}
\en{a~priori}\ru{априори}
\en{with any}\ru{с~любым}
\en{orthogonal tensor}\ru{ортогональным тензором}
$\bm{Q}$.
Проще принять\ru{,}
\en{that}\ru{что}~$\potential(\infinitesimaldeformation)$
\en{becomes an~isotropic function}\ru{становится изотропной функцией},
\en{which depends}\ru{которая зависит}
\en{only}\ru{лишь}
\en{on the invariants from the solution}\ru{от инвариантов из решения}
\en{of the characteristic equation}\ru{характеристического уравнения}~\chapdotpararef{chapter:mathapparatus}{para:eigenvectorseigenvalues}

\begin{equation}
2 \potential = A \mathrm{I}_1 ^2 + B \mathrm{II}_2
.
\end{equation}

$\potential(\infinitesimaldeformation)$ это квадратичная функция,
функция с членами степени не выше второй,
(или \inquotes{квадратичная форма})

\begin{equation}
2 \potential( \lineardeformation )
= A \lineardeformation_\tracedot \lineardeformation_\tracedot
+ B \lineardeformation \dotp \lineardeformation
\end{equation}

.........

\en{In~the~linear theory}\en{,}\ru{В~линейной теории}
\en{the~}\inquotes{ \en{complementary energy}\ru{дополнительная энергия} }
\en{is numerically equal to}\ru{численно равна}
\en{the }\en{elastic potential}\ru{упругому потенциалу}}.

\nopagebreak\vspace{-0.1em}
\begin{gather}\label{the Legendre transform}
\infinitesimaldeformation
= \scalebox{.9}{$
   \displaystyle
   \frac{
      { \raisemath{-0.2em}{ \partial \hspace{.1ex} \widehat{\potential} } }{ \raisemath{.04em}{\partial \linearstress} }
   }
$}
= \hspace{-0.12ex} \pliabilitytensor \dotdotp \linearstress
\hspace{.1ex} =
\linearstress \dotdotp \hspace{-0.1ex} \pliabilitytensor
\hspace{.1ex} ,
\\
%
% the complementary energy
%
\widehat{\potential}( \hspace{-0.1ex} \linearstress \hspace{.12ex} ) \hspace{-0.1ex}
= \linearstress \dotdotp \infinitesimaldeformation
- \hspace{.1ex} \potential(\infinitesimaldeformation)
\hspace{.1ex} .
\end{gather}



\chapdotpararef{chapter:mathapparatus}{para:tensors.symmetric+skewsymmetric}

\eqref{eigenvalues:eq}
