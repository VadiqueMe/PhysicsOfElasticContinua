\en{\section{Variations}}

\ru{\section{Вариации}}

\label{para:calculusofvariations}

\en{Further}\ru{Дальше}\ru{,}
\en{we will}\ru{мы будем}
\en{pretty often}\ru{довольно часто}
\en{use}\ru{использовать}
\en{the operation of varying}\ru{операцию варьирования}.
\en{It is similar}\ru{Она похожа}
\en{to the differentiation}\ru{на дифференцирование}.

\en{The variations}\ru{Вариации}
\en{are seen as}\ru{видятся как}
\en{the infinitesimal displacements}\ru{бесконечно малые перемещения},
\en{compatible with the constraints}\ru{совместимые с~ограничениями}\ru{( \inquotes{связями} )}.
\en{If}\ru{Если}
\en{there are no restrictions}\ru{ограничений}
\en{for}\ru{для}
\en{the variable}\ru{переменной} $x$\ru{ нет},
\en{then}\ru{то}
\en{the variations}\ru{вариации}
${\variation{x}}$
\en{are completely random}\ru{совершенно случайны}.
\en{But when}\ru{Но когда}
\begin{equation*}
x \!=\! x(y\hspace{-0.1ex})
\end{equation*}
\en{is}\ru{это}
\en{the function}\ru{функция}
\en{of the independent argument}\ru{независимого аргумента}~$y$,
\en{then}\ru{тогда}
\begin{equation*}
\variation{x} = x'\hspace{-0.25ex}(y\hspace{-0.1ex}) \hspace{.1ex} \variation{y} .
\end{equation*}

\en{The writings of variations}\ru{З\'{а}писи вариаций}
\en{have}\ru{имеют}
\en{the~same specifics}\ru{те~же особенности}\ru{,}
\en{as}\ru{как и}
\en{the writings}\ru{з\'{а}писи}
\en{of differentials}\ru{дифференциалов}.
\en{If}\ru{Если},
\en{as example}\ru{как пример},
${\variation{x}}$ \en{and}\ru{и}~${\variation{y}}$\en{ are}\ru{\:---}
\en{variations of}\ru{вариации}~$x$ \en{and}\ru{и}~$y$,
$u$ \en{and}\ru{и}~$v$\en{ are}\ru{\:---}
\en{the finite}\ru{кон\'{е}чные}
\en{values}\ru{значения},
\en{then we write}\ru{то мы пишем}
${u \variation{x} + v \variation{y} = \variation{w}}$
\en{even if}\ru{даже если} ${\variation{w}}$
\en{is not a~variation of}\ru{это не вариация}~$w$.

\en{In this case}\ru{В~этом случае}\en{,}
${\variation{w}}$ \en{is a~single symbol}\ru{это одиночный символ}.
\en{Surely}\ru{Разумеется}, \en{if}\ru{если} ${u \narroweq u(x,y)}$, ${v \narroweq v(x,y)}$ \en{and}\ru{и}~${\partial_x v = \partial_y u}$ (${\hspace{.16ex}\frac{\partial}{\partial x} v = \frac{\partial}{\partial y} u\hspace{.16ex}}$), \en{then}\ru{то} \en{the~sum}\ru{сумма}~${\variation{w}}$ \en{will be a~variation}\ru{будет вариацией} \en{of some}\ru{н\'{е}кой}~$w$.

Варьируя тождество~\eqref{orthogonalityofrotationtensor}, получим ${\variation{\rotationtensor} \hspace{-0.2ex} \dotp \rotationtensor^{\T} \hspace{-0.2ex} = - \hspace{.2ex} \rotationtensor \dotp \variation{\rotationtensor}^{\T}\!}$.
Этот тензор антисимметричен, и~потому выражается через свой сопутствующий вектор~${\varvector{o}}$ как~${\variation{\rotationtensor} \hspace{-0.1ex} \dotp \rotationtensor^{\T} \hspace{-0.3ex} = \varvector{o} \hspace{-0.2ex} \times \hspace{-0.2ex} \UnitDyad}$.

Приходим к~соотношениям

\nopagebreak\vspace{-0.5em}\begin{equation}
\variation{\rotationtensor} \hspace{-0.1ex} = \varvector{o} \hspace{-0.1ex} \times \hspace{-0.1ex} \rotationtensor , \:\:
\varvector{o} = - \hspace{.2ex} \scalebox{.93}{$ \displaystyle\onehalf $} \hspace{-0.1ex} \Bigl( \hspace{-0.1ex} \variation{\rotationtensor} \hspace{-0.1ex} \dotp \rotationtensor^{\T} \Bigr)_{\hspace{-0.25em}\Xcompanion}
\hspace{-0.1ex} ,
\end{equation}

\vspace{-0.5em}\noindent
аналогичным~\eqref{angularvelocityvector}.
\en{The vector}\ru{Вектор}
\en{of the infinitesimal rotation}\ru{бесконечно малого поворота}
${\varvector{o}}$
это не~\inquotesx{вариация $\bm{\mathrm{o}}$},
но единый символ
(в~отличие от~${\variation{\rotationtensor}}$).

\en{The infinitesimal rotation}\ru{Бесконечно малый поворот}
определяется вектором~${\varvector{o}}$,
но конечный поворот
тоже возможно представить
\en{as a~vector}\ru{как вектор}

...

\end{otherlanguage}
