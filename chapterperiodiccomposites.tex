\en{\chapter{Periodic composites}}

\ru{\chapter{Периодические композиты}}

\thispagestyle{empty}

\label{chapter:periodiccomposites}

\begin{changemargin}{\parindent}{\parindent}
\vspace{-2.5em}
{\noindent\small

Periodic composite structure is made of a~finite or infinite repetition of a~unit cell.

\par}
\vspace{-1.4em}
\end{changemargin}

\en{\section{One-dimensional problem}}

\ru{\section{Одномерная задача}}

\begin{otherlanguage}{russian}

\dropcap{В}{\hspace{-0.2ex}} одномерной задаче статики имеем уравнение

...



\end{otherlanguage}

\en{\section{Three-dimensional continuum}}

\ru{\section{Трёхмерный контину\kern-0.11exум}}

\begin{otherlanguage}{russian}

Исходим из уравнений в~перемещениях

...



\end{otherlanguage}

\en{\section{Fibrous structure}}

\ru{\section{Волокнистая структура}}

\begin{otherlanguage}{russian}

Тензор~${\stiffnesstensor}$ в~этом случае постоянен вдоль оси

...



\end{otherlanguage}

\en{\section{Statics of a periodic rod}}

\ru{\section{Статика периодического стержня}}

\begin{otherlanguage}{russian}

В~уравнениях линейной статики стержня

...




\end{otherlanguage}

\section*{\small \wordforbibliography}

\begin{changemargin}{\parindent}{0pt}
\fontsize{10}{12}\selectfont

\begin{otherlanguage}{russian}

Лежащий в~основе этой главы асимптотический метод представлен~(с~разной степенью математической скрупулёзности) в~книгах~\cite{bakhvalov.panasenko, asymptoticanalysisforperiodicstructures, kravchuk.mayboroda.urzhumtsev-polymericandcompositematerials, pobedrya-composites}.

\end{otherlanguage}

\end{changemargin}
