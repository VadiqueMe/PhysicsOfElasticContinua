\en{\chapter{Periodic composites}}

\ru{\chapter{Периодические композиты}}

\thispagestyle{empty}

\label{chapter:periodiccomposites}

\begin{changemargin}{\parindent}{\parindent}
\vspace{-2.5em}
{\noindent\small

\en{A~periodic composite}\ru{Периодический композит}
\en{is made of}\ru{сделан из}
\en{a~repetition}\ru{повторения},
\en{finite or infinite}\ru{конечного или бесконечного},
\en{of a~unit cell}\ru{единичной ячейки}.

\par}
\vspace{-1.4em}
\end{changemargin}

\section{%
\en{One-dimensional problem}\ru{Одномерная задача}%
}

\dropcap{\en{I}\ru{В}}{\en{n}\ru{\hspace{-0.4ex}}}
\en{one-dimensional}\ru{одномерной}
\en{problem}\ru{задаче}
\en{of statics}\ru{статики},
\en{the equation}\ru{уравнение}

...



\en{\section{Three-dimensional continuum}}

\ru{\section{Трёхмерный \rucontinuum}}

\en{From}\ru{Из}
\en{the equations}\ru{уравнений}
\en{in displacements}\ru{в~смещениях}
( ... ),

....



\en{\section{Fibrous structure}}

\ru{\section{Волокнистая структура}}

\en{In this case}\ru{В~этом случае}
\en{tensor}\ru{тензор}~${\stiffnesstensor}$
\en{is constant}\ru{постоянен}
\en{along the~axis}\ru{вдоль оси}

...



\en{\section{Statics of a periodic rod}}

\ru{\section{Статика периодического стержня}}

\en{In the~equations}\ru{В~уравнениях}
\en{of linear statics}\ru{линейной статики}
\en{of a~rod}\ru{стержня}
( ... )

.....



\section*{\small \wordforbibliography}

\begin{changemargin}{\parindent}{0pt}
\fontsize{10}{12}\selectfont

\en{The asymptotic method}\ru{Асимптотический метод},
\en{underlying this chapter}\ru{лежащий в~основе этой главы},
\en{is presented}\ru{представлен}
\en{with varying degrees}\ru{с~разной степенью}
\en{of mathematical scrupulousness}\ru{математической скрупулёзности}
\en{in the~books}\ru{в~книгах}~%
\cite{bakhvalov.panasenko, asymptoticanalysisforperiodicstructures, kravchuk.mayboroda.urzhumtsev-polymericandcompositematerials, pobedrya-composites}.

\end{changemargin}

