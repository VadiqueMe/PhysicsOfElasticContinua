

\en{\section{Tensor functions}}

\ru{\section{Тензорные функции}}

\label{para:tensorfunctions}

\noindent
\en{In~the~concept of~function}\ru{В~представлении о~функции}~${y \narroweq \hspace{-0.15ex} f(x)}$
\en{as of mapping (morphism)}\ru{как отображении (морфизме)} ${\smash{f \hspace{-0.2ex}\colon x \mapsto \hspace{-0.16ex} y}}$,
\en{an~input~(argument)}\ru{прообраз~(аргумент)}~$x$ \en{and an~output~(result)}\ru{и~образ~(результат)}~$y$ \en{may be tensors of any complexities}\ru{могут быть тензорами любых сложностей}.

\en{Consider}\ru{Рассмотрим} \en{at~least}\ru{хотя~бы} \en{a~scalar function}\ru{скалярную функцию} \en{of~a~bivalent tensor}\ru{двухвалентного тензора}~${\varphi \narroweq \varphi(\bm{B}\hspace{.1ex})}$.
\en{Examples}\ru{Примеры}\ru{\:---}\en{ are} ${\bm{B} \hspace{-0.2ex} \dotdotp \hspace{-0.25ex} \scalebox{1.1}[1]{$\bm{\mathit{\Phi}}$}}$ (\en{or}\ru{или}~${\bm{p} \dotp \hspace{-0.2ex} \bm{B} \hspace{-0.16ex} \dotp \hspace{-0.1ex} \bm{q}}$) \en{and}\ru{и}~${\bm{B} \hspace{-0.2ex} \dotdotp \hspace{-0.2ex} \bm{B}}$.
\en{Then}\ru{Тогда} \en{in~each basis}\ru{в~каждом базисе}~${\bm{a}_i}$ \en{paired with}\ru{в~паре с}~\en{cobasis}\ru{кобазисом}~${\bm{a}^{\hspace{-0.1ex}i}}$ \en{we have}\ru{имеем} \en{function}\ru{функцию}~${\varphi(B_{i\hspace{-0.1ex}j})}$ \en{of~nine numeric arguments}\ru{девяти числовых аргументов}\:--- \en{components}\ru{компонент}~$B_{i\hspace{-0.1ex}j}$ \en{of~tensor}\ru{тензора}~$\bm{B}$.
\en{For example}\ru{Для примера}

\nopagebreak\vspace{-0.2em}\begin{equation*}
\varphi(\bm{B}\hspace{.1ex}) \hspace{-0.2ex}
= \bm{B} \hspace{-0.2ex} \dotdotp \hspace{-0.28ex} \scalebox{1.1}[1]{$\bm{\mathit{\Phi}}$} \hspace{-0.1ex}
= \hspace{-0.1ex} B_{i\hspace{-0.1ex}j} \hspace{.16ex} \bm{a}^{\hspace{-0.1ex}i} \hspace{-0.16ex} \bm{a}^j \hspace{-0.3ex} \dotdotp \bm{a}_m \bm{a}_n \hspace{-0.2ex} \scalebox{1.2}[1]{$\mathit{\Phi}$}^{mn} \hspace{-0.25ex}
= \hspace{-0.1ex} B_{i\hspace{-0.1ex}j} \hspace{-0.15ex} \scalebox{1.2}[1]{$\mathit{\Phi}$}^{\hspace{.1ex}j\hspace{-0.06ex}i} \hspace{-0.25ex}
= \varphi(B_{i\hspace{-0.1ex}j})
\hspace{.1ex} .
\end{equation*}

\vspace{-0.25em} \noindent
\en{With any transition}\ru{С~любым переходом} \en{to~a~new basis,}\ru{к~новому базису} \en{the~result}\ru{результат} \en{doesn’t change}\ru{не~меняется}:
${\varphi(B_{i\hspace{-0.1ex}j}) \hspace{-0.2ex} = \varphi(B\hspace{.16ex}'_{\hspace{-0.32ex}i\hspace{-0.1ex}j}) \hspace{-0.2ex} = \varphi(\bm{B}\hspace{.1ex})}$.

\en{Differentiation of}\ru{Дифференцирование}~${\varphi(\bm{B}\hspace{.1ex})}$ \en{looks like}\ru{выглядит как}

\nopagebreak\vspace{-0.2em}\begin{equation}
d \hspace{.1ex} \varphi
= \displaystyle \frac{\partial \hspace{.1ex} \varphi}{\partial \hspace{-0.2ex} B_{i\hspace{-0.1ex}j}} \hspace{.2ex} d B_{i\hspace{-0.1ex}j} \hspace{-0.2ex}
= \displaystyle \frac{\partial \hspace{.1ex} \varphi}{\partial \hspace{-0.1ex} \bm{B}} \dotdotp d \bm{B}^{\T}
\hspace{-0.25ex} .
\end{equation}

\en{\vspace{-0.15em}}\ru{\vspace{-0.25em}}\noindent
\en{Tensor}\ru{Тензор}~${\scalebox{0.98}[1]{$\raisemath{.16em}{\scalebox{0.92}[0.92]{$\partial \hspace{.15ex} \varphi$}} \hspace{-0.1ex} / \hspace{-0.1ex} \raisemath{-0.32em}{\scalebox{0.92}[0.92]{$\partial \hspace{-0.1ex} \bm{B}$}}\hspace{.1ex}$}}$
\en{is called}\ru{называется} \en{the~derivative}\ru{производной} \en{of~function}\ru{функции}~$\varphi$ \en{by~argument}\ru{по~аргументу}~${\hspace{-0.15ex}\bm{B}\hspace{.1ex}}$;
${d \bm{B}}$\en{~is}\ru{\:---} \en{the~differential}\ru{дифференциал} \en{of~tensor}\ru{тензора}~$\bm{B}$,
${d \bm{B} \hspace{-0.1ex} = d B_{i\hspace{-0.1ex}j} \hspace{.16ex} \bm{a}^{\hspace{-0.1ex}i} \hspace{-0.16ex} \bm{a}^j}$;
${\smash{\scalebox{0.98}[1]{$\raisemath{.16em}{\scalebox{0.9}{$\partial \hspace{.15ex} \varphi$}} \hspace{-0.1ex} / \hspace{-0.2ex} \raisemath{-0.32em}{\scalebox{0.9}{$\partial \hspace{-0.1ex} B_{i\hspace{-0.1ex}j}$}}$}}}$\ru{\:---}\en{~are} \en{components}\ru{компоненты}~(\en{contra\-variant ones}\ru{контра\-вариант\-ные}) \en{of~\,}${\smash{\scalebox{0.98}[1]{$\raisemath{.16em}{\scalebox{0.92}[0.92]{$\partial \hspace{.15ex} \varphi$}} \hspace{-0.1ex} / \hspace{-0.2ex} \raisemath{-0.32em}{\scalebox{0.92}{$\partial \hspace{-0.1ex} \bm{B}$}}\hspace{.2ex}$}}}$

\nopagebreak\vspace{-0.1em}\begin{equation*}
\bm{a}^{\hspace{-0.1ex}i} \hspace{-0.15ex} \dotp \scalebox{0.92}{$\displaystyle\frac{\partial \hspace{.1ex} \varphi}{\partial \hspace{-0.1ex} \bm{B}}$} \dotp \bm{a}^j \hspace{-0.2ex}
=
\scalebox{0.92}{$\displaystyle\frac{\partial \hspace{.1ex} \varphi}{\partial \hspace{-0.1ex} \bm{B}}$} \dotdotp \bm{a}^j \hspace{-0.2ex} \bm{a}^{\hspace{-0.1ex}i} \hspace{-0.2ex}
=
\scalebox{0.92}{$\displaystyle\frac{\partial \hspace{.1ex} \varphi}{\partial \hspace{-0.15ex} B_{i\hspace{-0.1ex}j}}$}
\;\Leftrightarrow\;
\scalebox{0.92}{$\displaystyle\frac{\partial \hspace{.1ex} \varphi}{\partial \hspace{-0.1ex} \bm{B}}$}
=
\scalebox{0.92}{$\displaystyle\frac{\partial \hspace{.1ex} \varphi}{\partial \hspace{-0.15ex} B_{i\hspace{-0.1ex}j}}$} \hspace{.25ex} \bm{a}_i \bm{a}_{\hspace{-0.1ex}j}
\hspace{.1ex} .
\end{equation*}

...

\nopagebreak\begin{equation*}\begin{array}{c}
\varphi(\bm{B}\hspace{.1ex}) \hspace{-0.2ex}
= \bm{B} \hspace{-0.2ex} \dotdotp \hspace{-0.25ex} \scalebox{1.1}[1]{$\bm{\mathit{\Phi}}$}
\\[.2em]
%
d \hspace{.1ex} \varphi
= d \hspace{.2ex} ( \bm{B} \hspace{-0.2ex} \dotdotp \hspace{-0.25ex} \scalebox{1.1}[1]{$\bm{\mathit{\Phi}}$} \hspace{.1ex} ) \hspace{-0.2ex}
= d \bm{B} \hspace{-0.2ex} \dotdotp \hspace{-0.25ex} \scalebox{1.1}[1]{$\bm{\mathit{\Phi}}$}
= \hspace{-0.1ex} \scalebox{1.1}[1]{$\bm{\mathit{\Phi}}$} \hspace{-0.12ex} \dotdotp d \bm{B}
= \hspace{-0.1ex} \scalebox{1.1}[1]{$\bm{\mathit{\Phi}}$}^{\T} \hspace{-0.4ex} \dotdotp d \bm{B}^{\T}
\\[.2em]
%
d \hspace{.1ex} \varphi
= \scalebox{0.92}{$\displaystyle \frac{\partial \hspace{.1ex} \varphi}{\partial \hspace{-0.1ex} \bm{B}} \dotdotp d \bm{B}^{\T}$} \hspace{-0.3ex} ,
\:\:
\scalebox{0.92}{$\displaystyle \frac{\partial \hspace{-0.1ex} \left( \bm{B} \hspace{-0.2ex} \dotdotp \hspace{-0.25ex} \scalebox{1.1}[1]{$\bm{\mathit{\Phi}}$} \right)}{\partial \hspace{-0.1ex} \bm{B}}$}
= \hspace{-0.1ex} \scalebox{1.1}[1]{$\bm{\mathit{\Phi}}$}^{\T}
\end{array}\end{equation*}

${\bm{p} \dotp \hspace{-0.2ex} \bm{B} \hspace{-0.16ex} \dotp \hspace{-0.1ex} \bm{q} \hspace{.1ex} =
\hspace{-0.1ex} \bm{B} \hspace{-0.16ex} \dotdotp \hspace{-0.1ex} \bm{q} \bm{p}}$

\nopagebreak\begin{equation*}
\scalebox{0.92}[0.92]{$\displaystyle \frac{\partial \hspace{-0.1ex} \left( \hspace{.1ex} \bm{p} \dotp \hspace{-0.2ex} \bm{B} \hspace{-0.16ex} \dotp \hspace{-0.1ex} \bm{q} \right)}{\partial \hspace{-0.1ex} \bm{B}}$}
= \bm{p} \bm{q}
\end{equation*}

...

\nopagebreak\begin{equation*}\begin{array}{c}
\varphi(\bm{B}\hspace{.1ex}) \hspace{-0.2ex}
= \bm{B} \hspace{-0.2ex} \dotdotp \hspace{-0.2ex} \bm{B}
\\[.2em]
%
d \hspace{.1ex} \varphi
= d \hspace{.2ex} ( \bm{B} \hspace{-0.2ex} \dotdotp \hspace{-0.2ex} \bm{B} \hspace{.1ex} ) \hspace{-0.2ex}
= d ...
\end{array}\end{equation*}


...


\begin{otherlanguage}{russian}

Но согласно опять\hbox{-}таки~\eqref{cayley-hamilton:eq}
${\hspace{-0.1ex} -\bm{B}^{2} \hspace{-0.2ex} + \mathrm{I}\hspace{.16ex} \bm{B} \hspace{-0.1ex} - \mathrm{II}\hspace{.16ex} \UnitDyad + \mathrm{III}\hspace{.16ex} \bm{B}^{\expminusone} \hspace{-0.25ex} = {^2\bm{0}}}$, поэтому


...


Скалярная функция~${\varphi(\bm{B})}$ называется изотропной, если она не~чувствительна к~повороту аргумента:
\nopagebreak\vspace{.1em}\begin{equation*}
\varphi(\bm{B}) \hspace{-0.12ex} = \varphi ( \rotationtensor \narrowdotp \smash{\mathcircabove{\bm{B}}} \narrowdotp \hspace{.15ex} \rotationtensor^{\hspace{-0.1ex}\T} ) \hspace{-0.2ex} = \varphi(\smash{\mathcircabove{\bm{B}}}) \;\;\,
\forall \rotationtensor \hspace{-0.2ex} = \hspace{-0.1ex} \bm{a}_i \hspace{.1ex} \mathcircabove{\bm{a}}^i \hspace{-0.25ex} = \hspace{-0.1ex} \bm{a}^{\hspace{-0.2ex}i} \mathcircabove{\bm{a}}_i \hspace{-0.16ex} = \hspace{-0.1ex} \rotationtensor^{\hspace{-0.1ex}\expminusT}
\end{equation*}
\par\vspace{-0.25em}\noindent
для~любого ортогонального тензора~$\rotationtensor$ (тензора поворота, \pararef{para:rotationtensor}).

Симметричный тензор~${\bm{B}^{\mathsf{\hspace{.1ex}S}}}$ полностью определяется тройкой инвариантов и~угловой ориентацией собственных осей (они~же взаимно ортогональны, \pararef{para:eigenvectorseigenvalues}).
Ясно, что изотропная функция~${\varphi(\bm{B}^{\mathsf{\hspace{.1ex}S}})}$ симметричного аргумента является функцией лишь инвариантов ${\mathrm{I}\hspace{.16ex}({\bm{B}^{\mathsf{\hspace{.1ex}S}}})}$, ${\mathrm{II}\hspace{.16ex}({\bm{B}^{\mathsf{\hspace{.1ex}S}}})}$, ${\mathrm{III}\hspace{.16ex}({\bm{B}^{\mathsf{\hspace{.1ex}S}}})}$;
она дифференцируется согласно~\eqref{fonvccbnmxghjsxmnxjsdjhga}, где транспонирование излишне.

\end{otherlanguage}

\en{\section{Spatial differentiation}}

\ru{\section{Пространственное дифференцирование}}

\label{para:spatialdifferentiationoftensorfields} <<<<<< rename: remove fields

\begin{changemargin}{\parindent}{\parindent}
\vspace{-0.1em}
\small
\flushright
\textit{\en{Tensor field}\ru{Тензорное поле}}\ru{\:---}\en{ is} \en{a~tensor}\ru{это тензор}\ru{,} \en{varying from~point to~point}\ru{меняющийся от~точки к~точке} (\en{variable in~space}\ru{переменный в~пространстве}, \en{coordinate dependent}\ru{зависящий от~координат}).

\par\vspace{.3em}
\end{changemargin}

\begin{otherlanguage}{russian}

<<<<<<<

\noindent
Пусть \en{at each point}\ru{в~каждой точке} \en{of some region}\ru{некоторой области} \en{of a~three-dimensional space}\ru{трёхмерного пространства} определена величина~$\varsigma$.
Тогда говорят, что есть тензорное поле~${\varsigma \!=\! \varsigma(\locationvector)}$, \en{where}\ru{где}~$\locationvector$\en{ is}\ru{\:---} \en{location vector}\ru{вектор положения}~(\en{radius vector}\ru{вектор\hbox{-}радиус}) \en{of~a~point}\ru{точки} \en{in~space}\ru{пространства}.

Величина~$\varsigma$ может~быть тензором любой сложности.
Пример скалярного поля\:--- поле температуры в~среде, векторного поля\:--- скорости частиц жидкости.

Концепт тензорного поля никак не~связан с~концептом поля с~операциями $+$ и~$*$ с~11~свойствами этих операций.

\end{otherlanguage}

% ~ ~ ~ ~ ~
\begin{wrapfigure}{R}{0.55\textwidth}
\makebox[0.55\textwidth][c]{%
\hspace{2em}
\begin{minipage}[t]{.55\textwidth}

\begin{tikzpicture}[scale=0.5]

%%\clip (-6, -6) rectangle + (12, 12) ; % crop it
\clip (0, 0) circle (6cm) ; % crop it

\tikzset{%
	tangent/.style={
		decoration={
			markings,% switch on markings
			mark=
			at position #1
			with
			{
				\def\numberoftangent{\pgfkeysvalueof{/pgf/decoration/mark info/sequence number}}
				\coordinate (tangent point-\numberoftangent) at (0, 0);
				\coordinate (tangent unit vector-\numberoftangent) at (1, 0);
				\coordinate (tangent orthogonal unit vector-\numberoftangent) at (0, 1);
			}
		},
		postaction=decorate
	},
	use tangent/.style={
		shift=(tangent point-#1),
		x=(tangent unit vector-#1),
		y=(tangent orthogonal unit vector-#1)
	},
	use tangent/.default=1
}

\tikzset{%
	show curve controls/.style={
		postaction={
			decoration={
				show path construction,
				curveto code={
					\fill [black, opacity=.5]
						(\tikzinputsegmentfirst) circle (.4ex)
						(\tikzinputsegmentlast) circle (.4ex) ;
					\draw [black, opacity=.5, line cap=round, dash pattern=on 0pt off 1.6\pgflinewidth]
						(\tikzinputsegmentfirst) -- (\tikzinputsegmentsupporta)
						(\tikzinputsegmentlast) -- (\tikzinputsegmentsupportb) ;
					\fill [magenta, opacity=.5, line cap=round, dash pattern=on 0pt off 1.6\pgflinewidth]
						(\tikzinputsegmentsupporta) circle [radius=.4ex]
						(\tikzinputsegmentsupportb) circle [radius=.4ex] ;
				}
			},
			decorate
}	}	}

%%\foreach \cycle in {0, 1, ..., 15}
%%	\draw [color=green]
%%		($ (0, 0) - (\cycle, 1.2*\cycle) $)
%%		parabola ($ (4, 3) + 0.5*(1.6*\cycle, \cycle) $);

\foreach \c in {-10, -9.5, ..., 10}
{
	\def\offset{0.2*\c, -0.1*\c}
	\pgfmathsetmacro\bottomoffsetx{-.24 * ( \c )}
	\pgfmathsetmacro\bottomoffsety{-.1 * abs( \c ) + .1 * ( \c )}
	\pgfmathsetmacro\bottomangle{12 - 1.2 * abs( \c )}
	\pgfmathsetmacro\bottomnudge{2}
	\pgfmathsetmacro\midoffsetx{-.1 * abs( \c )}
	\pgfmathsetmacro\midoffsety{.1 * abs( \c )}
	\pgfmathsetmacro\midangle{63 + 1.2 * abs( \c )}
	\pgfmathsetmacro\midnudge{4 + ( .1 * abs( \c ) )}
	\pgfmathsetmacro\topoffsetx{.32 * ( \c ) + 0 * abs( \c )}
	\pgfmathsetmacro\topoffsety{-.16 * ( \c ) + 0 * abs( \c )}
	\pgfmathsetmacro\topangle{166 + 1.6 * ( \c ) + 1.2 * abs( \c )}
	\pgfmathsetmacro\topnudge{5 + ( .25 * abs( \c ) )}
	\draw	[ line width=.4pt
		, color=blue!50
		%%, show curve controls
		]
		($ (-6, -4.5) + 5*(\offset) + (\bottomoffsetx, \bottomoffsety) $)
		.. controls ++(\bottomangle: \bottomnudge) and ++(\midangle: -\midnudge) ..
		($ 4*(\offset) + (\midoffsetx, \midoffsety) $)
		.. controls ++(\midangle: \midnudge) and ++(\topangle: \topnudge) ..
		($ (8, 6) + 2.5*(\offset) + (\topoffsetx, \topoffsety) $) ;
}

\foreach \c in {-10, -9.5, ..., 10}
{
	\def\offset{0.2*\c, 0.1*\c}
	\pgfmathsetmacro\leftoffsetx{- .1 * abs ( \c )}
	\pgfmathsetmacro\leftoffsety{.4 * ( \c )}
	\pgfmathsetmacro\leftangle{33 + .2 * abs( \c )}
	\pgfmathsetmacro\leftnudge{1.6 + .5 * abs( \c )}
	\pgfmathsetmacro\midoffsetx{-.2 * abs( \c )}
	\pgfmathsetmacro\midoffsety{.2 * abs( \c )}
	\pgfmathsetmacro\midangle{111 + 1.2 * abs( \c )}
	\pgfmathsetmacro\midnudge{5}
	\pgfmathsetmacro\rightoffsetx{.25 * abs( \c )}
	\pgfmathsetmacro\rightoffsety{.16 * ( \c )}
	\pgfmathsetmacro\rightangle{177 + 2 * ( \c )}
	\pgfmathsetmacro\rightnudge{abs( 2 - ( .5 * ( \c ) ) )}
	\draw	[ line width=.4pt
		, color=red!50
		%%, show curve controls
		]
		($ (-12, 5) + 2.5*(\offset) + (\leftoffsetx, \leftoffsety) $)
		.. controls ++(\leftangle: \leftnudge) and ++(\midangle: \midnudge) ..
		($ 5*(\offset) + (\midoffsetx, \midoffsety) $)
		.. controls ++(\midangle: -\midnudge) and ++(\rightangle: \rightnudge) ..
		($ (8, -5) + 4*(\offset) + (\rightoffsetx, \rightoffsety) $);
}

\foreach \c in {-10, -9.5, ..., 10}
{
	\def\offset{0*\c, 0.25*\c}
	\pgfmathsetmacro\midnudge{6 + .16 * ( \c )}
	\draw	[ line width=.4pt
		, color=green!50
		%%, show curve controls
		]
		($ (12, 10) + 4*(\offset) $)
		.. controls ++(88: -4) and ++(11: \midnudge) ..
		($ 4*(\offset) $)
		.. controls ++(11: -\midnudge) and ++(99: -4) ..
		($ (-12, 4) + 4*(\offset) $) ;
}

\draw	[ line width=.8pt
	, color=blue!50!black
	%%, show curve controls
	]
	(-6, -4.5)
	.. controls ++(12: 2) and ++(63: -4) ..
	(0, 0);

\draw	[ line width=.8pt
	, color=blue!50!black
	%%, show curve controls
	, tangent=0
	, tangent=0.4
	]
	(0, 0)
	.. controls ++(63: 4) and ++(166: 5) ..
	(8, 6) ;

\path [use tangent=1]
	(0, 0) -- (.4*4, 0)
	node [color=blue, pos=0.86, above left, shape=circle, fill=white, outer sep=4pt, inner sep=1pt]
		{$\bm{r}_3$} ;

\draw [line width=1.25pt, color=blue, use tangent=1, -{Latex[round, length=3.6mm, width=2.4mm]}]
	(0, 0) -- (.4*4, 0) ;

\path [use tangent=2]
	(0, 0) -- (0, -1)
	node [color=blue!50!black, pos=0.48, above, shape=circle, fill=white, outer sep=0pt, inner sep=0.25pt]
		{$q^{\hspace{.1ex}3}$} ;

%%\fill [fill=blue, use tangent=1] (0, 0) circle (1mm);

\draw	[ line width=.8pt
	, color=red!50!black
	%%, show curve controls
	]
	(-12, 5)
	.. controls ++(33: 1.6) and ++(111: 5) ..
	(0, 0);

\draw	[ line width=.8pt
	, color=red!50!black
	%%, show curve controls
	, tangent=0
	, tangent=0.5
	]
	(0, 0)
	.. controls ++(111: -5) and ++(177: 2) ..
	(8, -5) ;

\path [use tangent=1]
	(0, 0) -- (.4*5, 0)
	node [color=red, pos=0.86, below left, shape=circle, fill=white, outer sep=4pt, inner sep=1pt]
		{$\bm{r}_1$} ;

\draw [line width=1.25pt, color=red, use tangent=1, -{Latex[round, length=3.6mm, width=2.4mm]}]
	(0, 0) -- (.4*5, 0);

\path [use tangent=2]
	(0, 0) -- (0, 1)
	node [color=red!50!black, pos=0.16, above, shape=circle, fill=white, outer sep=0pt, inner sep=0.25pt]
		{$q^{1}$} ;

%%\fill [fill=red, use tangent=1] (0, 0) circle (1mm);

\draw	[ line width=.8pt
	, color=green!50!black
	%%, show curve controls
	]
	(12, 10)
	.. controls ++(88: -4) and ++(11: 6) ..
	(0, 0) ;

\draw	[ line width=.8pt
	, color=green!50!black
	%%, show curve controls
	, tangent=0
	, tangent=0.36
	]
	(0, 0)
	.. controls ++(11: -6) and ++(99: -4) ..
	(-12, 4) ;

\path [use tangent=1]
	(0, 0) -- (.4*6, 0)
	node [color=green, pos=0.92, below right, shape=circle, fill=white, outer sep=5pt, inner sep=1pt]
		{$\bm{r}_2$} ;

\draw [line width=1.25pt, color=green, use tangent=1, -{Latex[round, length=3.6mm, width=2.4mm]}]
	(0, 0) -- (.4*6, 0);

\path [use tangent=2]
	(0, 0) -- (0, -1)
	node [color=green!50!black, pos=0.12, above, shape=circle, fill=white, outer sep=0pt, inner sep=0.25pt]
		{$q^{\hspace{.1ex}2}$} ;

%%\fill [fill=green, use tangent=1] (0, 0) circle (1mm);

\coordinate (theOrigin) at (5, -2) ;
\path (0, 0) circle (1mm) node [shape=circle, inner sep=.5mm, outer sep=0] (theCircleOfO) {} ;

\draw [line width=1.5pt, black, -{Stealth[round,length=4mm,width=2.8mm]}] (theOrigin) -- (theCircleOfO)
		node [pos=0.64, above right, shape=circle, fill=white, outer sep=2pt, inner sep=1.2pt]
			{$\bm{r}$} ;

\draw [line width=1.2pt, color=black, fill=white] (0, 0) circle (1ex);

\draw [line width=1.2pt, color=black, fill=white] (theOrigin) circle (1ex);

\end{tikzpicture}

\vspace{0.1em}\caption{}\label{fig:curvilinearcoordinates}
\end{minipage}}
\end{wrapfigure}

% ~ ~ ~ ~ ~

\begin{otherlanguage}{russian}

Не~только для~решения прикладных задач, но нередко и в~\inquotes{чистой тео\-рии} вместо аргумента~$\locationvector$ ис\-поль\-зу\-ет\-ся набор (какая-либо трой\-ка) криво\-линей\-ных координат~${q^{\hspace{.1ex}i}\hspace{-0.2ex}}$.
Если непрерывно менять лишь одну координату из~трёх, получается координатная линия.
Каждая точка трёхмерного пространства лежит на~пересечении трёх координатных линий (\figref{fig:curvilinearcoordinates}).
Вектор положения точки выражается через набор координат \en{as}\ru{как} \en{relation}\ru{отношение} ${\locationvector \hspace{-0.4ex} = \hspace{-0.4ex} \locationvector(q^{\hspace{.1ex}i}\hspace{.1ex})}$.

\end{otherlanguage}

Commonly used \en{sets of~coordinates}\ru{наборы координат}<<<<<<
\en{Rectangular}\ru{Прямоугольные} (\inquotes{\en{cartesian}\ru{декартовы}}), \en{spherical}\ru{сферические} \en{and }\ru{и~}\en{cylindrical}\ru{цилиндрические} \en{coordinates}\ru{координаты}\en{ are}\ru{\:---}

Curvilinear coordinates may be derived from a~set of~rectangular~(\inquotes{cartesian}) coordinates by using a~transformation that is locally invertible (a~one-to-one map) at~each point.
\en{Therefore}\ru{Поэтому} \en{rectangular coordinates}\ru{прямоугольные координаты} \en{of~any point}\ru{любой точки} \en{of~space}\ru{пространства} \en{can be converted}\ru{могут быть преобразованы} \en{to }\ru{в~}\en{some}\ru{какие-либо} \en{curvilinear coordinates}\ru{криволинейные координаты} \en{and}\ru{и}~\en{vice versa}\ru{обратно}.

...

The~differential of a~function presents a~change in the~linearization of this function.

...

\en{partial derivative}\ru{частная производная}

\nopagebreak\vspace{-0.4em}\begin{equation*}
\partial_i \equiv \scalebox{0.9}{$ \displaystyle\frac{\raisebox{-0.2em}{$\partial$}}{\raisebox{-0.1em}{$\partial q^i$}} $}
\end{equation*}

...

\en{differential}\ru{дифференциал} \en{of~}${\varsigma(q^i)}$

\nopagebreak\vspace{-0.4em}\begin{equation}
d\varsigma \hspace{-0.1ex}
= \scalebox{0.9}{$ \displaystyle\frac{\raisebox{-0.2em}{$\partial \hspace{.15ex} \varsigma$}}{\raisebox{-0.1em}{$\partial q^i$}} $} \hspace{.2ex} dq^i \hspace{-0.2ex}
= \partial_i \varsigma \hspace{.15ex} dq^i
\end{equation}

...

\en{Linearity}\ru{Линейность}

\nopagebreak\vspace{-0.4em}\begin{equation}\label{linearityordifferentiation}
\partial_i \bigl( \lambda \hspace{.1ex} p + \hspace{-0.2ex} \mu \hspace{.1ex} q \hspace{.1ex} \bigr) \hspace{-0.2ex}
= \lambda \bigl( \partial_i \hspace{.1ex} p \hspace{.1ex} \bigr) \hspace{-0.2ex} + \hspace{.1ex}
\mu \bigl( \partial_i \hspace{.1ex} q \hspace{.1ex} \bigr)
\end{equation}

\inquotes{Product rule}

\nopagebreak\vspace{-0.4em}\begin{equation}\label{productrulefordifferentiation}
\partial_i \bigl( \hspace{.15ex} p \circ q \hspace{.1ex} \bigr) \hspace{-0.2ex}
= \hspace{-0.2ex} \bigl( \partial_i \hspace{.1ex} p \hspace{.1ex} \bigr) \hspace{-0.25ex} \circ q \hspace{.12ex} +
\hspace{.1ex} p \circ \hspace{-0.25ex} \bigl( \partial_i \hspace{.1ex} q \hspace{.1ex} \bigr)
\end{equation}

...

Local basis ${\locationvector_\differentialindex{i}}$

\en{The~differential}\ru{Дифференциал} \en{of~location vector}\ru{вектора положения}~${\locationvector(q^{\hspace{.1ex}i}\hspace{.1ex})}$ \en{is}\ru{есть}

\nopagebreak\vspace{-0.2em}\begin{equation}\label{differentialoflocationvector}
d\locationvector
=
\scalebox{0.9}{$ \displaystyle\frac{\raisebox{-0.2em}{$ \partial \hspace{.15ex} \locationvector $}}{\partial q^{\hspace{.1ex}i}} $} \hspace{.2ex} dq^i \hspace{-0.1ex}
=
dq^i \locationvector_\differentialindex{i}
\hspace{.1ex} , \hspace{.5em}
\locationvector_\differentialindex{i} \hspace{-0.1ex} \equiv \scalebox{0.9}{$ \displaystyle\frac{\raisebox{-0.2em}{$ \partial \hspace{.15ex} \locationvector $}}{\partial q^{\hspace{.1ex}i}} $} \hspace{-0.1ex}
\equiv \partial_i \hspace{.1ex} \locationvector
\end{equation}

...

Local cobasis ${\locationvector^i}$, ${\locationvector^i \hspace{-0.32ex} \dotp \locationvector_\differentialindex{\hspace{-0.1ex}j} \hspace{-0.22ex} = \delta_{\hspace{-0.15ex}j}^{\hspace{.2ex}i}}$

...

\begin{equation*}
\displaystyle\frac{\raisebox{-0.2em}{$\partial \hspace{.15ex} \varsigma$}}{\partial \locationvector}
=
\displaystyle\frac{\raisebox{-0.2em}{$\partial \hspace{.15ex} \varsigma$}}{\raisebox{-0.1em}{$\partial q^i$}} \hspace{.1ex} \locationvector^i \hspace{-0.25ex}
=
\partial_i \varsigma \hspace{.2ex} \locationvector^i
\end{equation*}

\begin{equation}
d \varsigma \hspace{-0.1ex}
=
\scalebox{0.9}{$ \displaystyle\frac{\raisebox{-0.2em}{$\partial \hspace{.15ex} \varsigma$}}{\partial \locationvector} $} \dotp d\locationvector \hspace{-0.1ex}
=
\partial_i \varsigma \hspace{.2ex} \locationvector^i \hspace{-0.15ex} \dotp dq^{\hspace{.12ex}j} \hspace{-0.1ex} \locationvector_\differentialindex{\hspace{-0.1ex}j} \hspace{-0.2ex}
=
\partial_i \varsigma \hspace{.15ex} dq^i
\end{equation}

...

\en{The bivalent unit tensor}\ru{Бивалентный единичный тензор}~(\en{metric tensor}\ru{метрический тензор})~${\hspace{-0.1ex}\UnitDyad}$,
\en{which}\ru{который} \en{is neutral}\ru{нейтрален}~\eqref{definingpropertyofidentitytensor} \en{to the } \hbox{\hspace{-0.2ex}\inquotes{${\dotp\hspace{.22ex}}$}\hspace{-0.2ex}}-\en{product}\ru{произведению} (dot product\ru{’у}),
\en{can be represented as}\ru{может быть представлен как}

\nopagebreak\vspace{-0.1em}\begin{equation}
\UnitDyad
= \locationvector^i \locationvector_\differentialindex{i} \hspace{-0.15ex}
= \tikzmark{beginOriginOfNabla} \locationvector^i \partial_i \tikzmark{endOriginOfNabla} \hspace{.1ex} \locationvector = \hspace{-0.16ex} \boldnabla \locationvector ,
\end{equation}
\AddUnderBrace[line width=.75pt][0,-0.1ex]%
{beginOriginOfNabla}{endOriginOfNabla}%
{${\scriptstyle \boldnabla}$}

\vspace{-0.4em}\noindent
\en{where appears}\ru{где появляется} \en{the~}\en{differential}\ru{дифференциальный} \ru{оператор }\inquotes{\en{nabla}\ru{набла}}\en{ operator}

\nopagebreak\vspace{-0.2em}\begin{equation}
\boldnabla \equiv \locationvector^i \partial_i \hspace{.1ex} .
\end{equation}

...

\begin{equation}
d \varsigma \hspace{-0.1ex}
=
\scalebox{0.9}{$ \displaystyle\frac{\raisebox{-0.2em}{$\partial \hspace{.15ex} \varsigma$}}{\raisebox{-0.05em}{$\partial \locationvector$}} $} \dotp d\locationvector \hspace{-0.1ex}
=
d\locationvector \dotp \hspace{-0.11ex} \boldnabla \varsigma \hspace{-0.1ex}
=
\partial_i \varsigma \hspace{.15ex} dq^i
\end{equation}

\vspace{1.1em}${
d\locationvector = d\locationvector \dotp \hspace{-0.2ex} \tikzmark{beginItsUnitTensorE} \boldnabla \locationvector \tikzmark{endItsUnitTensorE}
}$%
\AddOverBrace[line width=.75pt][0,0.1ex]{beginItsUnitTensorE}{endItsUnitTensorE}{${\scriptstyle \UnitDyad}$}

...

\en{Divergence}\ru{Дивергенция} \en{of~the~dyadic product}\ru{диадного произведения} \en{of~two vectors}\ru{двух векторов}

\nopagebreak\vspace{-0.3em}\begin{multline}\label{divergenceofdyadicproducoftwovectors}
\boldnabla \hspace{-0.16ex} \dotp \hspace{-0.2ex} \bigl( \hspace{-0.1ex} \bm{a} \bm{b} \hspace{.05ex} \bigr) \hspace{-0.33ex}
= \locationvector^i \partial_i \hspace{-0.1ex} \dotp \hspace{-0.24ex} \bigl( \hspace{-0.1ex} \bm{a} \bm{b} \bigr) \hspace{-0.33ex}
= \locationvector^i \hspace{-0.3ex} \dotp \partial_i \bigl( \hspace{-0.1ex} \bm{a} \bm{b} \bigr) \hspace{-0.3ex}
= \locationvector^i \hspace{-0.3ex} \dotp \hspace{-0.15ex} \bigl( \partial_i \bm{a} \bigr) \bm{b} \hspace{.1ex} + \locationvector^i \hspace{-0.3ex} \dotp \bm{a} \hspace{.1ex} \bigl( \partial_i \bm{b} \bigr) \hspace{-0.33ex} =
\\[-0.1em]
%
= \hspace{-0.15ex} \bigl( \locationvector^i \hspace{-0.3ex} \dotp \partial_i \bm{a} \bigr) \bm{b} \hspace{.1ex} + \bm{a} \dotp \locationvector^i \hspace{-0.15ex} \bigl( \partial_i \bm{b} \bigr) \hspace{-0.33ex}
= \hspace{-0.15ex} \bigl( \locationvector^i \partial_i \hspace{-0.1ex} \dotp \bm{a} \bigr) \bm{b} \hspace{.1ex} + \bm{a} \dotp \hspace{-0.1ex} \bigl( \locationvector^i \partial_i \bm{b} \bigr) \hspace{-0.33ex} =
\\
%
= \hspace{-0.15ex} \bigl( \boldnabla \hspace{-0.15ex} \dotp \hspace{-0.1ex} \bm{a} \bigr) \bm{b} \hspace{.1ex} + \bm{a} \dotp \hspace{-0.12ex} \bigl( \boldnabla \hspace{.1ex} \bm{b} \bigr)
\end{multline}

\vspace{-0.2em}\noindent
--- \en{here’s no~need}\ru{тут нет нужды} \en{to~expand}\ru{разворачивать} \en{vectors}\ru{векторы}~$\bm{a}$ \en{and}\ru{и}~$\bm{b}$, \en{expanding just}\ru{развернув лишь} \en{differential operator}\ru{дифференциальный оператор}~${\hspace{-0.13ex}\boldnabla}$.

...

\en{Gradient of cross product of two vectors}\ru{Градиент векторного произведения двух векторов},
\en{applying}\ru{применяя} \inquotes{product rule}~\eqref{productrulefordifferentiation}
\en{and}\ru{и}~\en{relation}\ru{соотношение}~\eqref{crossproductoftwovectors} \en{for any two vectors}\ru{для любых двух векторов}
(\en{partial derivative}\ru{частная производная}~$\partial_i$ \en{of~some~vector by scalar coordinate}\ru{некоторого вектора по скалярной координате}~$q^i\hspace{-0.1ex}$ \en{is a~vector too}\ru{это тоже вектор})

\nopagebreak\vspace{-0.4em}\begin{multline}\label{gradientofcrossproductoftwovectors}
\boldnabla \hspace{-0.2ex} \left( \bm{a} \hspace{-0.1ex} \times \hspace{-0.1ex} \bm{b} \right) \hspace{-0.2ex}
= \hspace{.1ex} \locationvector^i \partial_i \hspace{-0.3ex} \left( \bm{a} \hspace{-0.2ex} \times \hspace{-0.2ex} \bm{b} \right) \hspace{-0.2ex}
= \locationvector^i \hspace{-0.4ex} \left( \partial_i \bm{a} \hspace{-0.2ex} \times \hspace{-0.2ex} \bm{b} \hspace{.1ex} +
\bm{a} \hspace{-0.2ex} \times \hspace{-0.2ex} \partial_i \bm{b} \right) \hspace{-0.2ex} =
\\[-0.1em]
%
= \locationvector^i \hspace{-0.4ex} \left( \partial_i \bm{a} \hspace{-0.2ex} \times \hspace{-0.2ex} \bm{b} \hspace{.1ex} -
\partial_i \bm{b} \hspace{-0.2ex} \times \hspace{-0.2ex} \bm{a} \right) \hspace{-0.2ex}
= \hspace{.1ex} \locationvector^i \partial_i \hspace{.1ex} \bm{a} \hspace{-0.2ex} \times \hspace{-0.2ex} \bm{b} \hspace{.1ex} - \hspace{.1ex}
\locationvector^i \partial_i \hspace{.1ex} \bm{b} \hspace{-0.2ex} \times \hspace{-0.2ex} \bm{a} =
\\[-0.1em]
%
= \hspace{-0.12ex} \boldnabla \bm{a} \hspace{-0.1ex} \times \hspace{-0.1ex} \bm{b} \hspace{.12ex} - \hspace{-0.12ex}
\boldnabla \hspace{.1ex} \bm{b} \hspace{-0.1ex} \times \hspace{-0.1ex} \bm{a}
\hspace{.2ex} .
\end{multline}

...

\en{Gradient}\ru{Градиент} \en{of }dot product\ru{’а} \en{of two vectors}\ru{двух векторов}

\nopagebreak\vspace{-0.4em}\begin{multline}\label{gradientofdotproductoftwovectors}
\boldnabla \hspace{.1ex} \bigl( \hspace{-0.05ex} \bm{a} \hspace{-0.1ex} \dotp \hspace{-0.1ex} \bm{b} \hspace{.05ex} \bigr) \hspace{-0.3ex}
= \hspace{.1ex} \locationvector^i \partial_i \bigl( \hspace{-0.05ex} \bm{a} \hspace{-0.1ex} \dotp \hspace{-0.1ex} \bm{b} \hspace{.05ex} \bigr) \hspace{-0.33ex}
= \hspace{.1ex} \locationvector^i \bigl( \partial_i \bm{a} \bigr) \hspace{-0.32ex} \dotp \bm{b} + \hspace{.1ex} \locationvector^i \bm{a} \hspace{-0.05ex} \dotp \hspace{-0.15ex} \bigl( \partial_i \bm{b} \bigr) \hspace{-0.33ex} =
\\[-0.1em]
%
= \hspace{-0.2ex} \bigl( \locationvector^i \partial_i \bm{a} \bigr) \hspace{-0.32ex} \dotp \bm{b} \hspace{.1ex} + \hspace{.1ex} \locationvector^i \bigl( \partial_i \bm{b} \bigr) \hspace{-0.33ex} \dotp \bm{a}
= \hspace{-0.16ex} \bigl( \boldnabla \hspace{-0.1ex} \bm{a} \bigr) \hspace{-0.3ex} \dotp \hspace{.1ex} \bm{b} \hspace{.1ex} + \hspace{-0.1ex} \bigl( \boldnabla \hspace{.1ex} \bm{b} \bigr) \hspace{-0.27ex} \dotp \hspace{.1ex} \bm{a}
\hspace{.2ex} .
\end{multline}

\newpage ...

\newpage ...




\en{\section{Integral theorems}}

\ru{\section{Интегральные теоремы}}

\begin{otherlanguage}{russian}

Для векторных полей известны интегральные теоремы Gauss’а и~Stokes’а.

\noindent\leavevmode{\indent}{\small Gauss’ theorem (divergence theorem) enables an~integral taken over a~volume to be replaced by one taken over the closed surface bounding that volume, and vice versa.\par}

\noindent\leavevmode{\indent}{\small Stokes’ theorem enables an~integral taken around a closed curve to be replaced by one taken over \emph{any} surface bounded by that curve. Stokes’ theorem relates a~line integral around a closed path to a surface integral over what is called a~\emph{capping surface} of the path.\par}

Теорема Гаусса о~дивергенции\:--- про~то, как заменить объёмный интеграл поверхностным~(\en{and vice versa}\ru{и~наоборот}). В~этой теореме рассматривается поток (ef)flux вектора через ограничивающую объём~$V$ з\'{а}мкнутую поверхность ${\mathcal{O}(\boundary V)}$ с~единичным вектором внешней нормали~$\bm{n}$

\nopagebreak\vspace{-0.1em}\begin{equation}
\ointegral\displaylimits_{\mathclap{\mathcal{O}(\boundary V)}} \hspace{-0.1ex} \bm{n} \dotp \bm{a} \hspace{.4ex} d\mathcal{O} \hspace{.12ex} = \integral\displaylimits_{V} \hspace{-0.3ex} \boldnabla \hspace{-0.12ex} \dotp \bm{a} \hspace{.4ex} dV \hspace{-0.25ex}.
\end{equation}

Объём~$V$ нарезается тремя семействами координатных поверхностей на~множество бесконечно малых элементов. Поток через поверхность ${\mathcal{O}(\boundary V)}$ равен сумме потоков через края получившихся элементов. В~бесконечной малости каждый такой элемент\:--- маленький локальный дифференциальный кубик~(параллелепипед). ... Поток вектора~$\bm{a}$ через грани малого кубика объёма~$dV$ есть ${\sum_{i = 1}^{6} \bm{n}_i \dotp \bm{a} \hspace{.2ex} \mathcal{O}_i}$, а~через сам этот объём поток равен ${\boldnabla \dotp \bm{a} \hspace{.32ex} dV}$.

Похожая трактовка этой теоремы есть, к примеру, в~курсе Richard’а Feynman’а~\cite{feynman-lecturesonphysics}.

\emph{( рисунок с кубиками )}

to dice\:--- нарез\'{а}ть кубиками

small cube, little cube

локально ортонормальные координаты ${\bm{\xi} = \xi_i \hspace{.2ex} \bm{n}_i \hspace{.1ex}}$, ${d\bm{\xi} = d \xi_i \hspace{.2ex} \bm{n}_i}$, ${\boldnabla = \bm{n}_i \partial_i}$

разложение вектора ${\bm{a} = a_i \bm{n}_i \hspace{.1ex}}$

Теорема Стокса о~циркуляции выражается равенством

...

\newpage ...



\end{otherlanguage}



\newpage

\en{\section{Curvature tensors}}

\ru{\section{Тензоры кривизны}}

\label{para:curvaturetensors}

\begin{changemargin}{2\parindent}{\parindent}
\bgroup % to change \parindent locally
\setlength{\parindent}{\negparindent}
\setlength{\parskip}{\spacebetweenparagraphs}
\small

\leavevmode{\indent}The~\href{https://en.wikipedia.org/wiki/Riemann_curvature_tensor}{\emph{Riemann curvature tensor} or \emph{Riemann\hbox{--}Christoffel tensor}} (after \href{https://en.wikipedia.org/wiki/Bernhard_Riemann}{\textbold{Bernhard Riemann}} and \href{https://en.wikipedia.org/wiki/Elwin_Bruno_Christoffel}{\textbold{Elwin Bruno Christoffel}}) is the most common method used to express the curvature of Riemannian manifolds. It’s a~tensor field, it assigns a~tensor to each point of a~Riemannian manifold, that measures the extent to which the~metric tensor is not locally isometric to that of \inquotes{flat} space. The curvature tensor measures noncommutativity of the covariant derivative, and as such is the~integrability obstruction for the~existence of an~isometry with \inquotes{flat} space.

%%%\vspace{.2em}
%%%\hfill $\sim$\:\emph{from Wikipedia, the free encyclopedia}
\par
\egroup
\nopagebreak\vspace{.12em}
\end{changemargin}

\begin{otherlanguage}{russian}

\noindent
Рассматривая тензорные поля в~криволинейных координатах~(\pararef{para:spatialdifferentiationoftensorfields}), мы исходили из~представления вектора\hbox{-}радиуса~(вектора положения) точки функцией этих координат:
${\locationvector \hspace{-0.4ex} = \hspace{-0.4ex} \locationvector(q^{\hspace{.1ex}i}\hspace{.1ex})}$.
Этим отношением порождаются выражения

\nopagebreak\begin{itemize}
\item векторов локального касательного базиса ${%
\locationvector_\differentialindex{i} \hspace{-0.16ex} \equiv \smash{ \raisemath{.16em}{\scalebox{0.8}{$ \partial \hspace{.1ex} \locationvector $}} \hspace{-0.3ex} / \hspace{-0.4ex} \raisemath{-0.32em}{\scalebox{0.8}{$ \partial q^{\hspace{.1ex}i} $}} } \hspace{-0.15ex} \equiv \partial_i \hspace{.1ex} \locationvector%
}$,
%
\item компонент ${\textsl{g}_{i\hspace{-0.1ex}j} \hspace{-0.24ex} \equiv \locationvector_\differentialindex{i} \hspace{-0.16ex} \dotp \locationvector_\differentialindex{\hspace{-0.1ex}j}}$ и~${\textsl{g}^{\hspace{.25ex}i\hspace{-0.1ex}j} \hspace{-0.32ex} \equiv \locationvector^i \hspace{-0.32ex} \dotp \locationvector^j \hspace{-0.32ex} = \smash{\textsl{g}_{i\hspace{-0.1ex}j}^{\hspace{.33ex}\expminusone}}}$ единичного \inquotes{метрического} тензора~${\UnitDyad = \locationvector_\differentialindex{i} \locationvector^i \hspace{-0.2ex} = \locationvector^i \locationvector_\differentialindex{i} \hspace{-0.15ex} = \textsl{g}_{j\hspace{-0.1ex}k} \hspace{.1ex} \locationvector^{\hspace{.1ex}j} \hspace{-0.1ex} \locationvector^{k} \hspace{-0.25ex} = \textsl{g}^{\hspace{.25ex}j\hspace{-0.1ex}k} \hspace{.1ex} \locationvector_\differentialindex{\hspace{-0.1ex}j} \locationvector_\differentialindex{k}}$,
%
\item векторов локального взаимного кокасательного базиса ${\locationvector^i \hspace{-0.32ex} \dotp \locationvector_\differentialindex{\hspace{-0.1ex}j} \hspace{-0.22ex} = \delta_{\hspace{-0.15ex}j}^{\hspace{.2ex}i}}$, ${\locationvector^i \hspace{-0.25ex} = \textsl{g}^{\hspace{.25ex}i\hspace{-0.1ex}j} \locationvector_\differentialindex{\hspace{-0.1ex}j}}$,
%
\item диф\-ферен\-циаль\-ного набла\hbox{-}оператора Hamilton’а ${\smash{\boldnabla \equiv \locationvector^i \partial_i}}$,
${\UnitDyad = \hspace{-0.25ex} \smash{\boldnabla \locationvector}}$,
%
\item полного дифференциала ${d \bm{\xi} = d \locationvector \dotp \hspace{-0.2ex} \boldnabla \hspace{-0.05ex} \bm{\xi} \hspace{.1ex}}$,
%
\item частных производных касательного \hbox{базиса} (вторых частных производных~$\locationvector$) ${\locationvector_{\differentialindex{i}\hspace{.2ex}\differentialindex{\hspace{-0.1ex}j}} \hspace{-0.2ex} \equiv \partial_i \partial_j \locationvector \hspace{-0.1ex} = \partial_i \hspace{.12ex} \locationvector_\differentialindex{\hspace{-0.1ex}j}}$,
%
\item символов \inquotes{связности} \hbox{Христоффеля}~(\hbox{Christoffel} symbols) ${\Gamma_{\hspace{-0.25ex}i\hspace{-0.1ex}j}^{\hspace{.25ex}k} \hspace{-0.1ex} \equiv \locationvector_{\differentialindex{i}\hspace{.2ex} \differentialindex{\hspace{-0.1ex}j}} \hspace{-0.2ex} \dotp \locationvector^k
%%\hspace{-0.32ex} = \Gamma_{\hspace{-0.25ex}i\hspace{-0.1ex}j\mathdotbelow{n}} \hspace{.25ex} \textsl{g}^{\hspace{.25ex}nk}\hspace{-0.25ex}
}$ и~${\Gamma_{\hspace{-0.25ex}i\hspace{-0.1ex}j\mathdotbelow{k}} \hspace{-0.16ex} \equiv \locationvector_{\differentialindex{i}\hspace{.2ex}\differentialindex{\hspace{-0.1ex}j}} \hspace{-0.2ex} \dotp \locationvector_\differentialindex{k}
%%\hspace{-0.2ex} = \Gamma_{\hspace{-0.25ex}i\hspace{-0.1ex}j}^{\hspace{.25ex}n} \hspace{.16ex} \textsl{g}_{nk}
}$.
\vspace{-0.2em}
\end{itemize}

Представим теперь, что функция~${\locationvector(q^{\hspace{.1ex}k})}$ не~известна, но~\hbox{зат\'{о}} в~каждой точке пространства известны шесть независимых компонент положительно определённой (\en{all}\ru{все} \ru{матрицы }Gram\ru{’а}\en{ matrices} \en{are non-negative definite}\ru{определены неотрицательно}) симметричной метрической матрицы Gram\ru{’а}~${\textsl{g}_{i\hspace{-0.1ex}j}(q^{\hspace{.1ex}k})}$.

the Gram matrix (or Gramian)

Билинейная форма ...

\nopagebreak
...

Поскольку шесть функций~${\textsl{g}_{i\hspace{-0.1ex}j}(q^{\hspace{.1ex}k})}$ происходят от векторной функции~${\locationvector(q^{\hspace{.1ex}k})}$, то между элементами~$\textsl{g}_{i\hspace{-0.1ex}j}$ существуют некие соотношения.

\en{Differential}\ru{Дифференциал} ${d\locationvector}$\;\eqref{differentialoflocationvector}\en{ is}\ru{\:---} \en{exact}\ru{полный~(точный)}.
\en{This is true}\ru{Это истинно} \en{if and only if}\ru{тогда и только тогда, когда} \en{second partial derivatives}\ru{вторые частные производные} \en{commute}\ru{коммутируют}:

\nopagebreak\vspace{-0.2em}\begin{equation*}
d\locationvector \hspace{-0.1ex} = \locationvector_\differentialindex{k} \hspace{.2ex} dq^{\hspace{.1ex}k}
\hspace{.4em}\Leftrightarrow\hspace{.44em}
%%\partial_i \bigl( \partial_j \locationvector \bigr) \hspace{-0.2ex} = \hspace{.1ex} \partial_j \bigl( \partial_i \locationvector \bigr) \hspace{-0.2ex}
%%\hspace{.5em}\text{\en{or}\ru{или}}\hspace{.5em}
\partial_i \hspace{.12ex} \locationvector_\differentialindex{\hspace{-0.1ex}j} \hspace{-0.22ex} = \partial_j \locationvector_\differentialindex{i}
\hspace{.5em}\text{\en{or}\ru{или}}\hspace{.5em}
\locationvector_{\differentialindex{i}\hspace{.2ex}\differentialindex{\hspace{-0.1ex}j}} \hspace{-0.22ex} = \locationvector_{\differentialindex{\hspace{-0.1ex}j}\hspace{.2ex}\differentialindex{i}}
\hspace{.1ex} .
\end{equation*}

\vspace{-0.2em}\noindent
Но это условие уж\'{е} обеспечено симметрией~${\textsl{g}_{i\hspace{-0.1ex}j}}$

...

\en{metric}\ru{метрическая} (\inquotes{\en{affine}\ru{аффинная}}) \en{connection}\ru{связность}~$\nabla_{\hspace{-0.32ex}i\hspace{.1ex}}$, её~же называют \inquotes{\en{covariant derivative}\ru{ковариантная производная}}

\vspace{1.2em}\begin{equation*}
\locationvector_{\differentialindex{i}\hspace{.2ex}\differentialindex{\hspace{-0.1ex}j}} \hspace{-0.1ex} = \hspace{.1ex}
\tikzmark{beginChristoffelSymbolOne} \locationvector_{\differentialindex{i}\hspace{.2ex}\differentialindex{\hspace{-0.1ex}j}} \hspace{-0.15ex} \dotp \tikzmark{beginEtensorUpDown} \locationvector^{k} \hspace{-0.4ex} \tikzmark{endChristoffelSymbolOne} \hspace{.4ex} \locationvector_\differentialindex{k} \tikzmark{endEtensorUpDown} \hspace{-0.2ex}
= \tikzmark{beginChristoffelSymbolOther} \locationvector_{\differentialindex{i}\hspace{.2ex}\differentialindex{\hspace{-0.1ex}j}} \hspace{-0.15ex} \dotp \tikzmark{beginEtensorDownUp} \locationvector_\differentialindex{k} \tikzmark{endChristoffelSymbolOther} \locationvector^k \tikzmark{endEtensorDownUp}
\end{equation*}%
\AddOverBrace[line width=.75pt][0,0.2ex][yshift=-0.1em]{beginEtensorUpDown}{endEtensorUpDown}{${\scriptstyle \UnitDyad}$}%
\AddOverBrace[line width=.75pt][0,0.2ex][yshift=-0.1em]{beginEtensorDownUp}{endEtensorDownUp}{${\scriptstyle \UnitDyad}$}%
\AddUnderBrace[line width=.75pt][0,-0.1ex]{beginChristoffelSymbolOne}{endChristoffelSymbolOne}{${\scriptstyle \Gamma_{\hspace{-0.25ex}i\hspace{-0.1ex}j}^{\hspace{.25ex}k}}$}%
\AddUnderBrace[line width=.75pt][0,-0.1ex]{beginChristoffelSymbolOther}{endChristoffelSymbolOther}{${\scriptstyle \Gamma_{\hspace{-0.25ex}i\hspace{-0.1ex}j\mathdotbelow{k}}}$}

${
\Gamma_{\hspace{-0.25ex}i\hspace{-0.1ex}j}^{\hspace{.25ex}k} \hspace{.2ex} \locationvector_\differentialindex{k} \hspace{-0.2ex} = \locationvector_{\differentialindex{i}\hspace{.2ex}\differentialindex{\hspace{-0.1ex}j}} \hspace{-0.16ex} \dotp \hspace{.1ex} \locationvector^k \locationvector_\differentialindex{k} \hspace{-0.2ex} = \locationvector_{\differentialindex{i}\hspace{.2ex}\differentialindex{\hspace{-0.1ex}j}}
}$

covariant derivative (affine connection) is only defined for vector fields

${
\boldnabla \bm{v} \hspace{-0.15ex}
= \locationvector^{i} \partial_i \hspace{-0.33ex} \left( v^{\hspace{.12ex}j} \locationvector_\differentialindex{\hspace{-0.1ex}j} \right) \hspace{-0.25ex}
= \locationvector^{i} \hspace{-0.4ex} \left( \partial_i v^{\hspace{.12ex}j} \locationvector_\differentialindex{\hspace{-0.1ex}j} \hspace{-0.12ex} + v^{\hspace{.12ex}j} \locationvector_{\differentialindex{i}\hspace{.2ex}\differentialindex{\hspace{-0.1ex}j}} \right)
}$

${
\boldnabla \bm{v} \hspace{-0.15ex}
= \locationvector^{i} \locationvector_\differentialindex{\hspace{-0.1ex}j} \nabla_{\hspace{-0.32ex}i\hspace{.1ex}} v^{\hspace{.12ex}j} \hspace{-0.3ex} , \:\:
\nabla_{\hspace{-0.32ex}i\hspace{.1ex}} v^{\hspace{.12ex}j} \hspace{-0.3ex} \equiv
\partial_i v^{\hspace{.12ex}j} \hspace{-0.33ex} + \Gamma_{\hspace{-0.25ex}in}^{\hspace{.25ex}j} v^{\hspace{.1ex}n}
}$

${
\boldnabla \locationvector_\differentialindex{i} \hspace{-0.2ex}
= \locationvector^k \partial_k \locationvector_\differentialindex{i} \hspace{-0.2ex}
= \locationvector^k \locationvector_{\differentialindex{k}\hspace{.2ex}\differentialindex{i}} \hspace{-0.2ex}
%%= \locationvector^k \hspace{.2ex} \Gamma_{\hspace{-0.25ex}ki}^{\hspace{.25ex}n} \hspace{.2ex} \locationvector_\differentialindex{n} \hspace{-0.2ex}
= \locationvector^k \locationvector_\differentialindex{n} \hspace{.1ex} \Gamma_{\hspace{-0.25ex}ki}^{\hspace{.25ex}n}
\hspace{.2ex} , \:\:
\nabla_{\hspace{-0.32ex}i\hspace{.1ex}} \locationvector_\differentialindex{n} \hspace{-0.25ex}
= \Gamma_{\hspace{-0.25ex}in}^{\hspace{.25ex}k} \hspace{.16ex} \locationvector_\differentialindex{k}
}$

\vspace{.2em}
Christoffel symbols describe a~metric (\inquotes{affine}) connection, that is how the~basis changes from point to~point.

символы Christoffel’я это \inquotes{\en{components of~connection}\ru{компоненты связности}} \en{in local coordinates}\ru{в~локальных координатах}

...

\href{https://en.wikipedia.org/wiki/Torsion_tensor}{\en{torsion tensor}\ru{тензор кручения}}~${^3\bm{\mathfrak{T}}}$ \en{with components}\ru{с~компонентами}

\nopagebreak\vspace{-0.1em}\begin{equation*}
\mathfrak{T}^{k}_{i\hspace{-0.1ex}j} \hspace{-0.15ex} = \Gamma_{\hspace{-0.25ex}i\hspace{-0.1ex}j}^{\hspace{.25ex}k} \hspace{-0.1ex} - \Gamma_{\hspace{-0.33ex}j\hspace{-0.06ex}i}^{\hspace{.25ex}k}
\end{equation*}

\noindent
determines the~antisymmetric part of a~connection

...

\noindent
симметрия ${ \Gamma_{\hspace{-0.25ex}i\hspace{-0.1ex}j\mathdotbelow{k}} = \Gamma_{\hspace{-0.33ex}j\hspace{-0.06ex}i\mathdotbelow{k}} }$, поэтому ${3^3 \hspace{-0.2ex} - 3 \hspace{-0.2ex}\cdot\hspace{-0.2ex} 3 = 18}$ разных~(независимых) ${\Gamma_{\hspace{-0.25ex}i\hspace{-0.1ex}j\mathdotbelow{k}}}$

\begin{multline}
\Gamma_{\hspace{-0.25ex}i\hspace{-0.1ex}j}^{\hspace{.25ex}n} \hspace{.16ex} \textsl{g}_{nk} \hspace{-0.24ex} = \Gamma_{\hspace{-0.25ex}i\hspace{-0.1ex}j\mathdotbelow{k}} \hspace{-0.2ex} = \locationvector_{\differentialindex{i}\hspace{.2ex}\differentialindex{\hspace{-0.1ex}j}} \hspace{-0.2ex} \dotp \locationvector_\differentialindex{k} \hspace{-0.1ex} =
\\[-0.1em]
%
= \smallerdisplaystyleonehalf \hspace{-0.1ex} \bigl( \locationvector_{\differentialindex{i}\hspace{.2ex}\differentialindex{\hspace{-0.1ex}j}} \hspace{-0.16ex} + \locationvector_{\differentialindex{\hspace{-0.1ex}j}\hspace{.2ex}\differentialindex{i}} \bigr) \hspace{-0.2ex} \dotp \locationvector_\differentialindex{k} \hspace{-0.1ex}
+ \smallerdisplaystyleonehalf \hspace{-0.1ex} \bigl( \locationvector_{\differentialindex{\hspace{-0.1ex}j}\hspace{.2ex}\differentialindex{k}} \hspace{-0.16ex} - \locationvector_{\differentialindex{k}\hspace{.2ex}\differentialindex{\hspace{-0.1ex}j}} \bigr) \hspace{-0.2ex} \dotp \locationvector_\differentialindex{i} \hspace{-0.1ex}
+ \smallerdisplaystyleonehalf \hspace{-0.1ex} \bigl( \locationvector_{\differentialindex{i}\hspace{.2ex}\differentialindex{k}} \hspace{-0.16ex} - \locationvector_{\differentialindex{k}\hspace{.1ex}\differentialindex{i}} \bigr) \hspace{-0.2ex} \dotp \locationvector_\differentialindex{\hspace{-0.1ex}j} \hspace{-0.1ex} =
\\[-0.1em]
%
= \smallerdisplaystyleonehalf \hspace{-0.1ex} \bigl( \scalebox{0.93}[1]{$
	\locationvector_{\differentialindex{i}\hspace{.2ex}\differentialindex{\hspace{-0.1ex}j}} \hspace{-0.2ex} \dotp \locationvector_\differentialindex{k} \hspace{-0.16ex}
	+ \locationvector_{\differentialindex{i}\hspace{.2ex}\differentialindex{k}} \hspace{-0.2ex} \dotp \locationvector_\differentialindex{\hspace{-0.1ex}j}
$} \bigr) \hspace{-0.16ex}
+ \smallerdisplaystyleonehalf \hspace{-0.1ex} \bigl( \scalebox{0.93}[1]{$
	\locationvector_{\differentialindex{\hspace{-0.1ex}j}\hspace{.2ex}\differentialindex{i}} \hspace{-0.2ex} \dotp \locationvector_\differentialindex{k} \hspace{-0.16ex}
	+ \locationvector_{\differentialindex{\hspace{-0.1ex}j}\hspace{.2ex}\differentialindex{k}} \hspace{-0.2ex} \dotp \locationvector_\differentialindex{i}
$} \bigr) \hspace{-0.16ex}
- \smallerdisplaystyleonehalf \hspace{-0.1ex} \bigl( \scalebox{0.93}[1]{$
	\locationvector_{\differentialindex{k}\hspace{.2ex}\differentialindex{i}} \hspace{-0.2ex} \dotp \locationvector_\differentialindex{\hspace{-0.1ex}j} \hspace{-0.16ex}
	+ \locationvector_{\differentialindex{k}\hspace{.2ex}\differentialindex{\hspace{-0.1ex}j}} \hspace{-0.2ex} \dotp \locationvector_\differentialindex{i}
$} \bigr) \hspace{-0.2ex} =
\\[-0.25em]
%
= \smalldisplaystyleonehalf \hspace{-0.4ex} \left(^{\mathstrut} \hspace{-0.2ex}
\partial_i ( \locationvector_\differentialindex{\hspace{-0.1ex}j} \hspace{-0.2ex} \dotp \locationvector_\differentialindex{k} ) \hspace{-0.16ex}
+ \partial_j ( \locationvector_\differentialindex{i} \hspace{-0.2ex} \dotp \locationvector_\differentialindex{k} ) \hspace{-0.16ex}
- \partial_k ( \locationvector_\differentialindex{i} \hspace{-0.2ex} \dotp \locationvector_\differentialindex{\hspace{-0.1ex}j} )
\hspace{-0.12ex} \right) \hspace{-0.4ex} =
\\[-0.25em]
%
= \smalldisplaystyleonehalf \hspace{-0.3ex} \left(
\partial_i \hspace{.12ex} \textsl{g}_{j\hspace{-0.1ex}k} \hspace{-0.2ex}
+ \partial_j \hspace{.1ex} \textsl{g}_{ik} \hspace{-0.2ex}
- \partial_k \hspace{.12ex} \textsl{g}_{i\hspace{-0.1ex}j}
\right) \hspace{-0.4ex} .
\end{multline}

Все символы Christoffel’я тождественно равны нулю лишь в~ортонормальной~(декартовой) системе.
\textcolor{magenta}{(А~какие они для косоугольной?)}

Дальше:
${d\locationvector_\differentialindex{i} \hspace{-0.2ex}
= d\locationvector \dotp \hspace{-0.2ex} \boldnabla \locationvector_\differentialindex{i} \hspace{-0.2ex}
= dq^{\hspace{.1ex}k} \partial_k \locationvector_\differentialindex{i} \hspace{-0.2ex}
= \locationvector_{\differentialindex{k}\hspace{.2ex}\differentialindex{i}} \hspace{.2ex} dq^{\hspace{.1ex}k}\hspace{-0.25ex}}$\:--- тоже полные дифференциалы.
\[
d\locationvector_\differentialindex{k} \hspace{-0.2ex}
= \partial_i \locationvector_\differentialindex{k} \hspace{.15ex} dq^i \hspace{-0.3ex}
= \scalebox{0.84}{$ \displaystyle\frac{\raisemath{-0.2ex}{\partial \hspace{.1ex} \locationvector_\differentialindex{k}}}{\raisemath{-0.3ex}{\partial q^1}} $} \hspace{.2ex} dq^1 \hspace{-0.2ex}
+ \scalebox{0.84}{$ \displaystyle\frac{\raisemath{-0.2ex}{\partial \hspace{.1ex} \locationvector_\differentialindex{k}}}{\raisemath{-0.3ex}{\partial q^2}} $} \hspace{.2ex} dq^2 \hspace{-0.2ex}
+ \scalebox{0.84}{$ \displaystyle\frac{\raisemath{-0.2ex}{\partial \hspace{.1ex} \locationvector_\differentialindex{k}}}{\raisemath{-0.3ex}{\partial q^3}} $} \hspace{.2ex} dq^3 \hspace{-0.2ex}
\]
Поэтому ${\partial_i \partial_j \locationvector_\differentialindex{k} \hspace{-0.2ex} = \partial_j \partial_i \locationvector_\differentialindex{k}}$, ${\partial_i \locationvector_{\differentialindex{\hspace{-0.1ex}j}\hspace{.2ex}\differentialindex{k}} \hspace{-0.2ex} = \partial_j \locationvector_{\differentialindex{i}\hspace{.2ex}\differentialindex{k}}}$,
и~трёхиндексный объект из~векторов третьих частных производных

\nopagebreak\vspace{-0.25em}
\begin{equation}
\locationvector_{\differentialindex{i}\hspace{.2ex}\differentialindex{\hspace{-0.1ex}j}\hspace{.2ex}\differentialindex{k}} \hspace{-0.1ex} \equiv \hspace{.1ex} \partial_i \partial_j \partial_k \locationvector
= \partial_i \hspace{.12ex} \locationvector_{\differentialindex{\hspace{-0.1ex}j}\hspace{.2ex}\differentialindex{k}}
\end{equation}

\vspace{-0.24em} \noindent
симметричен по~первому и~второму индексам (а~не~только по~второму и~третьему).
И~тогда равен нулю~${\hspace{-0.16ex}^4\bm{0}}$ следующий тензор четвёртой сложности\:---
\href{https://en.wikipedia.org/wiki/Riemann_curvature_tensor}{ \emph{\ru{тензор кривизны }Riemann\ru{’а}\en{ curvature tensor}} (\en{or}\ru{или}~\emph{\ru{тензор }Riemann\ru{’а}\hbox{--}Christoffel\ru{’я}\en{ tensor}}) }
%% Римана\hbox{--}Христоффеля

\nopagebreak\vspace{-0.1em}\begin{equation}\label{riemanncurvaturetensor}
{^4\bm{\mathfrak{R}}} = \hspace{.12ex} \mathfrak{R}_{\hspace{.1ex}hi\hspace{-0.1ex}j\hspace{-0.1ex}k} \hspace{.12ex} \locationvector^h \locationvector^i \locationvector^j \locationvector^k \hspace{-0.25ex},
\:\:
\mathfrak{R}_{\hspace{.1ex}hi\hspace{-0.1ex}j\hspace{-0.1ex}k} \hspace{-0.12ex}
\equiv
\locationvector_\differentialindex{h} \hspace{-0.15ex} \dotp \left( \hspace{.12ex} \locationvector_{\differentialindex{\hspace{-0.1ex}j}\hspace{.2ex}\differentialindex{i}\hspace{.2ex}\differentialindex{k}} \hspace{-0.2ex} - \locationvector_{\differentialindex{i}\hspace{.2ex}\differentialindex{\hspace{-0.1ex}j}\hspace{.2ex}\differentialindex{k}} \hspace{.12ex} \right)
\hspace{-0.3ex} .
\end{equation}

Выразим компоненты~${\mathfrak{R}_{\hspace{.1ex}i\hspace{-0.1ex}j\hspace{-0.1ex}kn}}$ через метрическую матрицу~${\textsl{g}_{i\hspace{-0.1ex}j}}$.
Начнём с~дифференцирования локального кобазиса:

\[
\locationvector^i \hspace{-0.32ex} \dotp \locationvector_\differentialindex{k} \hspace{-0.16ex} = \delta_k^{\hspace{.1ex}i}
\;\Rightarrow\:
\partial_j \locationvector^i \hspace{-0.32ex} \dotp \locationvector_\differentialindex{k} \hspace{-0.15ex} + \locationvector^i \hspace{-0.32ex} \dotp \locationvector_{\differentialindex{\hspace{-0.1ex}j}\hspace{.2ex}\differentialindex{k}} \hspace{-0.15ex} = 0
\;\Rightarrow\:
\partial_j \locationvector^i \hspace{-0.12ex} = - \hspace{.2ex} \Gamma_{\hspace{-0.25ex}j\hspace{-0.1ex}k}^{\hspace{.25ex}i} \hspace{.2ex} \locationvector^k
\hspace{-0.4ex} .
\]

...

Шесть независимых компонент:
${\mathfrak{R}_{\hspace{.1ex}1212}}$, ${\mathfrak{R}_{\hspace{.1ex}1213}}$, ${\mathfrak{R}_{\hspace{.1ex}1223}}$, ${\mathfrak{R}_{\hspace{.1ex}1313}}$, ${\mathfrak{R}_{\hspace{.1ex}1323}}$, ${\mathfrak{R}_{\hspace{.1ex}2323}}$.

...

\en{Symmetric}\ru{Симметричный} \en{bivalent}\ru{бивалентный} \href{https://en.wikipedia.org/wiki/Ricci_curvature}{\emph{\ru{тензор кривизны }Ricci\en{ curvature tensor}}}
%% after \href{https://en.wikipedia.org/wiki/Gregorio_Ricci-Curbastro}{\textbold{Gregorio Ricci\hbox{-}Curbastro}}

\begin{equation*}
\hspace{.1ex}\pmb{\scalebox{1.2}[1]{$\mathscr{R}$}} \equiv
\smalldisplaystyleonefourth \hspace{.4ex} \mathfrak{R}_{\hspace{.1ex}abi\hspace{-0.1ex}j} \hspace{.2ex} \locationvector^a \hspace{-0.33ex} \times \hspace{-0.1ex} \locationvector^b \locationvector^i \hspace{-0.33ex} \times \hspace{-0.1ex} \locationvector^j \hspace{-0.25ex}
= \smalldisplaystyleonefourth \hspace{.15ex} \levicivita^{abp} \levicivita^{i\hspace{-0.1ex}j\hspace{-0.1ex}q} \hspace{.25ex} \mathfrak{R}_{\hspace{.1ex}abi\hspace{-0.1ex}j} \hspace{.2ex} \locationvector_\differentialindex{p} \locationvector_\differentialindex{q} \hspace{-0.2ex}
= \mathscr{R}^{\hspace{.1ex}pq} \hspace{.1ex} \locationvector_\differentialindex{p} \locationvector_\differentialindex{q}
\end{equation*}

\vspace{-0.2em} \noindent
(\en{coefficient}\ru{коэффициент}~$\onefourth$ \en{is used here for convenience}\ru{используется тут для удобства}) \en{with components}\ru{с~компонентами}

\vspace{.1em}\begin{equation*}
\begin{array}{ccc}
\mathscr{R}^{\hspace{.1ex}1\hspace{-0.1ex}1} \hspace{-0.3ex} =
\scalebox{0.8}{$ \displaystyle \frac{\raisemath{-0.2em}{1}}{\raisemath{.15em}{\smash{\textsl{g}}}} $} \hspace{.4ex} \mathfrak{R}_{\hspace{.1ex}2323}
\hspace{.2ex} ,
&
&
\\[.6em]
%
\mathscr{R}^{\hspace{.1ex}21} \hspace{-0.3ex} =
\scalebox{0.8}{$ \displaystyle \frac{\raisemath{-0.2em}{1}}{\raisemath{.15em}{\smash{\textsl{g}}}} $} \hspace{.4ex} \mathfrak{R}_{\hspace{.1ex}1323}
\hspace{.2ex} ,
&
\mathscr{R}^{\hspace{.1ex}22} \hspace{-0.3ex} =
\scalebox{0.8}{$ \displaystyle \frac{\raisemath{-0.2em}{1}}{\raisemath{.15em}{\smash{\textsl{g}}}} $} \hspace{.4ex} \mathfrak{R}_{\hspace{.1ex}1313}
\hspace{.2ex} ,
&
\\[.6em]
%
\mathscr{R}^{\hspace{.1ex}31} \hspace{-0.3ex} =
\scalebox{0.8}{$ \displaystyle \frac{\raisemath{-0.2em}{1}}{\raisemath{.15em}{\smash{\textsl{g}}}} $} \hspace{.4ex} \mathfrak{R}_{\hspace{.1ex}1223}
\hspace{.2ex} ,
&
\mathscr{R}^{\hspace{.1ex}32} \hspace{-0.3ex} =
\scalebox{0.8}{$ \displaystyle \frac{\raisemath{-0.2em}{1}}{\raisemath{.15em}{\smash{\textsl{g}}}} $} \hspace{.4ex} \mathfrak{R}_{\hspace{.1ex}1213}
\hspace{.2ex} ,
&
\mathscr{R}^{\hspace{.1ex}33} \hspace{-0.3ex} =
\scalebox{0.8}{$ \displaystyle \frac{\raisemath{-0.2em}{1}}{\raisemath{.15em}{\smash{\textsl{g}}}} $} \hspace{.4ex} \mathfrak{R}_{\hspace{.1ex}1212}
\hspace{.2ex} .
\end{array}
\end{equation*}

Равенство тензора Риччи нулю
${\hspace{.1ex}\pmb{\scalebox{1.2}[1]{$\mathscr{R}$}} \hspace{-0.16ex} = \hspace{-0.2ex} {^2\bm{0}}}$ (в~компонентах это шесть уравнений ${\hspace{.1ex}\mathscr{R}^{\hspace{.1ex}i\hspace{-0.1ex}j} \hspace{-0.3ex} = \mathscr{R}^{\hspace{.1ex}j\hspace{-0.06ex}i} \hspace{-0.3ex} = 0}$) \en{is}\ru{есть} \en{the~}\textcolor{magenta}{\en{necessary}\ru{необходимое}} \en{condition}\ru{условие} \en{of~integrability}\ru{интегрируемости}~(\ru{\inquotes{совместности}, }\inquotes{compatibility}) для нахождения вектора-радиуса~${\locationvector(q^{\hspace{.1ex}k})}$ по~полю~${\textsl{g}_{i\hspace{-0.1ex}j}(q^{\hspace{.1ex}k})}$.

\end{otherlanguage}

\section*{\small \wordforbibliography}

\begin{changemargin}{\parindent}{0pt}
\fontsize{10}{12}\selectfont




\en{Many books exist}\ru{Существует много книг}\ru{,} \en{which describe}\ru{которые описывают} \en{only}\ru{только} \en{apparatus of~tensor calculus}\ru{аппарат тензорного исчисления}~\cite{mcconnell-tensoranalysis, schouten-tensoranalysis, sokolnikoff-tensoranalysis, borisenko.tarapov, rashevsky-riemanniangeometry}.
\en{However}\ru{Однако}, \en{an~index notation}\ru{индексная запись}\:--- \en{it’s when}\ru{это когда} \en{tensors are considered}\ru{тензоры рассматриваются} \en{as matrices of~components}\ru{как матрицы компонент}\:--- \en{is still more popular}\ru{всё ещё более популярна}\ru{,} \en{than the direct indexless notation}\ru{чем прямая безиндексная запись}.
\en{The direct notation}\ru{Прямая запись}
\en{is used}\ru{используется},
\en{for example}\ru{например},
\en{in the appendices}\ru{в~приложениях}
\en{of books}\ru{книг}
\en{by }Anatoliy I. Lurie (}\foreignlanguage{russian}{Анатолия И. Лурье}\en{)}
\cite{lurie-nonlinearelasticity, lurie-theoryofelasticity}.



\ru{Лекции }R.\:Feynman’\en{s}\ru{а}\en{ lectures}~\cite{feynman-lecturesonphysics}
\en{contain}\ru{содержат}
\en{a~vivid description}\ru{яркое описание}
\en{of vector fields}\ru{векторных полей}.
\en{Also}\ru{Также},
\en{the information}\ru{информация}
\en{about the tensor calculus}\ru{о~тензорном исчислении}\en{ is}\ru{\:---}
\en{part of}\ru{часть}
\en{original}\ru{оригинальной}
\en{and}\ru{и}
\en{profound}\ru{глубокой}
\en{book}\ru{книги}
\en{by }C.\:Truesdell\ru{’а}~\cite{truesdell-firstcourse}.

\end{changemargin}
