\en{\chapter{Micropolar three\hbox{-}dimensional continuum}}

\ru{\chapter{Микрополярная трёхмерная среда}}

\thispagestyle{empty}

\newcommand{\currentvector}{\bm{r}} %% {\bm{R}}
\newcommand{\initialvector}{\mathcircabove{\bm{r}}} %% {\bm{r}}

\newcommand{\littlerotationvector}{\varvector{\bm{\varphi}}}

\newcommand{\motiongradient}{\bm{F}}

\newcommand{\distortiontensor}{\bm{\upgamma}}
\newcommand{\wrynesstensor}{\bm{\kappa}\hspace{-0.1ex}} %% {\mathboldae}

\newcommand{\forcestresstensor}{\mathboldtau}
\newcommand{\couplestresstensor}{\bm{\mu}}

\label{chapter:cosseratcontinuum}

\en{\section{Introduction to linear micropolar theory}}

\ru{\section{Введение в линейную микрополярную теорию}}

\label{para:introtolinearmicropolar}

\begin{otherlanguage}{russian}

\lettrine[lines=2, findent=2pt, nindent=0pt]{Х}{арактерная} отличительная особенность классических упругих сред~(\chapref{chapter:nonlinearcontinuum} и~\customref{chapter:linearclassicalelasticity})\:--- то, что они состоят \inquotesx{из простых материальных точек}[.] Частица классического континуума имеет лишь трансляционные степени свободы, её~движение определяется только вектором~${\bm{R}(q^i \hspace{-0.15em}, t)}$. Поэтому нагрузки~(\inquotes{силовые факторы}) в~такой среде\:--- только силы, объёмные и~поверхностные. Моментов нет.

Но не~так~уж трудно построить более сложные модели сплошной среды, в~которых частицы обладают не~только~лишь степенями свободы трансляции, но~и н\'{е}которыми дополнительными. Новые степени свободы связаны и~с~новыми силовыми факторами, а~также новыми уравнениями.

Наиболее естественная из~неклассических моделей трёхмерной среды предложена братьями Cosserat в~1909\:год\'{у}~\cite{cosserat}. Каждая частица континуума Коссера\:--- это элементарное твёрдое тело с~шестью степенями свободы. Силовые факторы в~такой среде\:--- силы и~моменты. Работа братьев Коссера оставалась незамеченной полвека, но~затем возник интерес к~этой теме~\cite{mindlin.tiersten, nowacki-elasticity}.

\end{otherlanguage}

%% \begin{comment} %%
\vspace{2mm}

\begin{tcolorbox}[breakable, enhanced, colback = orange!8, before upper={\parindent3.2ex}, parbox = false]
\small%
\setlength{\abovedisplayskip}{2pt}\setlength{\belowdisplayskip}{2pt}%

\noindent \textit{from \textboldoblique{Nowacki~W.} The Linear Theory of~Micropolar Elasticity.} In: \textit{Micropolar Elasticity. International Centre for~Mechanical Sciences (Courses and~Lectures), vol.\:151, 1974, pp.\:1\hbox{--}43}

\vspace{.5em}

Woldemar Voigt tried to remove the shortcomings of the classical theory of elasticity [\textit{\textboldoblique{W.\:Voigt}. Theoretische Studien über die~Elasticitätsverhältnisse der~Krystalle. Abhandlungen der~Königlichen Gesellschaft der~Wissenschaften in~Göttingen, 34:\:3\hbox{--}51, 1887}] by the assumption that the interaction of two parts of the body is transmitted through an area element $do$ by means not only of the force vector $\bm{p}do$ but also by the moment vector $\bm{m}do$. Thus, besides the force stresses~$\sigma_{ji}$ also the moment stresses have been defined.

However, the complete theory of asymmetric elasticity was developed by the brothers \textboldoblique{François et~Eugène Cosserat} who published it in 1909 in the work \textit{\inquotes{Théorie des corps déformables}}.

They assumed that the body consists of interconnected particles in the form of small rigid bodies. During the deformation each particle is displaced by~$\bm{u}(\bm{x},t)$ and rotated by~$\bm{\varphi}(\bm{x},t)$, the functions of the position~$\bm{x}$ and time~$t$.

Thus an elastic continuum has been described such that its points possess the orientation (polar media) and for which we can speak of the rotation of a point. The vectors $\bm{u}$ and $\bm{\varphi}$ are mutually independent and determine the deformation of the body. The introduction of the vectors $\bm{u}$ and $\bm{\varphi}$ and the assumption that the transmission of forces through an area element $do$ is carried out by means of the force vector $\bm{p}$ and the moment vector $\bm{m}$ leads in the consequence to asymmetric stress tensors $\sigma_{ji}$ and $\mu_{ji}$.

The theory of the brothers E.~and~F.\;Cosserat remained unnoticed and was not duly appreciated during their lifetime. This was so because the presentation was very general (the theory was non-linear, including large deformations) and because its frames exceeded the frames of the theory of elasticity. They attempted to construct the unified field theory, containing mechanics, optics and electrodynamics and combined by a general principle of the least action.

The research in the field of the general theories of continuous media conducted in the last fifteen years, drew the attention of the scientists to Cosserats’ work. Looking for the new models, describing more precisely the behaviour of the real elastic media, the models similar to, or identical with that of Cosserats’ have been encountered. Here, we mention, first of all, the papers by C.\:Truesdell and R.\,A.\;Toupin [\textit{\textboldoblique{C.\:Truesdell} and \textboldoblique{R.\,A.\;Toupin}. The classical field theories. Encyclopædia of Physics, Chapter 1, Springer\hbox{-}Verlag, Berlin, 1960}], G.\:Grioli [\textit{\textboldoblique{Grioli~G.} Elasticité asymmetrique. Ann. di Mat. Pura et Appl. Ser. IV, 50 (1960)}], R.\,D.\;Mindlin and H.\,F.\;Tiersten [\textit{\textboldoblique{Mindlin, R.\,D.}; \textboldoblique{Tiersten, H.\,F.} Effects of couple-stresses in linear elasticity. Arch. Rational Mech. Anal. 11. 1962. 415\hbox{--}448}].
\par
\end{tcolorbox}

\vspace{2mm}
%% \end{comment} %%

\en{In the~truly micropolar continuum,}\ru{В~истинно микрополярном континууме}
\en{vector fields}\ru{векторные поля} \en{of~displacements}\ru{перемещений}~${\fieldofdisplacements(\bm{r},t)}$ \en{and}\ru{и}~\en{rotations}\ru{поворотов}~${\fieldofrotations(\bm{r},t)}$ \en{are mutually independent}\ru{взаимно независимы}.
\en{I\kern-0.12ext~is also called a~model with free rotation}\ru{Это также называется моделью со~свободным вращением (free rotation)}.

% pseudocontinuum, or a~model with constrained rotation
% псевдоконтинуум, или модель со~стеснённым вращением (constrained rotation)
% \pararef{para:caseoflatenttrihedron.smalldisplacementsandrotations}

\en{Consider at~first}\ru{Рассмотрим сперв\'{а}}
\en{geometrically linear model}\ru{геометрически линейную модель},
\en{that’s}\ru{то~есть} \en{the~case}\ru{случай} \en{of~}\en{small}\ru{м\'{а}лых} \en{displacements}\ru{перемещений} \en{and}\ru{и}~\en{small}\ru{м\'{а}лых} \en{rotations}\ru{поворотов}.
\en{Here}\ru{Здесь} \en{operators}\ru{операторы}
${\hspace{-0.2ex} \boldnablacircled
%%\hspace{-0.1ex} \equiv \initialvector^{\hspace{.1ex}i} \partial_i
\hspace{.1ex}}$
\en{and}\ru{и}~${\hspace{-0.2ex} \boldnabla
%%\hspace{-0.1ex} \equiv \currentvector^i \partial_i
\hspace{.1ex}}$
%%(\chapdotpararef{chapter:nonlinearcontinuum}{para:differentiation})
\en{are indistinguishable}\ru{неразличимы},
\en{equations}\ru{уравнения} \inquotesx{\en{can be written in the~initial configuration}\ru{можно пис\'{а}ть в~исходной конфигурации}}[,]
$\variation$ \en{and}\ru{и} ${\hspace{-0.2ex}\boldnabla}$ \en{commute}\ru{коммутируют}
(${\variation{\boldnabla \fieldofdisplacements} \hspace{-0.1ex} = \hspace{-0.3ex} \boldnabla \variation{\fieldofdisplacements}}$,
${\variation{\boldnabla \hspace{-0.12ex} \fieldofrotations} \hspace{-0.1ex} = \hspace{-0.3ex} \boldnabla \variation{\fieldofrotations}}$).

\en{As foundation}\ru{Как основу} \en{for building this~model}\ru{построения этой модели} \en{we put}\ru{пол\'{о}жим} \en{the~principle of~virtual work}\ru{принцип виртуальной работы}
(\inquotes{\en{the~variation of~work of~real external forces on virtual displacements is equal with negative sign to the~variation of~work of~internal forces}\ru{вариация работы реальных внешних сил на виртуальных перемещениях равна с~обратным знаком вариации работы внутренних сил}\;--- \en{real stresses on virtual deformations}\ru{реальных напряжений на~виртуальных деформациях}})

\nopagebreak\vspace{-0.1em}\begin{equation*}
\integral\displaylimits_{\mathcal{V}} \hspace{-0.64ex} \left(^{\mathstrut} \bm{f} \dotp \variation{\fieldofdisplacements} + \bm{m} \dotp \variation{\fieldofrotations} \right) \hspace{-0.4ex} d\mathcal{V}
\hspace{.25ex} +
\integral\displaylimits_o \hspace{-0.64ex} \left(^{\mathstrut} \bm{p} \dotp \variation{\fieldofdisplacements} + \mathboldM \hspace{-0.1ex} \dotp \variation{\fieldofrotations} \right) \hspace{-0.4ex} do \hspace{.12ex}
= - \hspace{-0.5ex} \integral\displaylimits_{\mathcal{V}} \hspace{-0.64ex} \variation{\internalwork} \hspace{.1ex} d\mathcal{V} .
\end{equation*}

\begin{otherlanguage}{russian}

\vspace{-0.1em} \noindent Здесь
$\bm{f}$ и~$\bm{m}$\:--- внешние силы и~моменты \en{per volume unit}\ru{на~единицу объёма};
$\bm{p}$ и~$\mathboldM$\:--- они~же, но \en{per surface unit}\ru{на~единицу поверхности} (поверхностные нагрузки действуют лишь на~н\'{е}которой части~$o$ поверхности~$\boundary{\mathcal{V}}$\hbox{\hspace{-0.12ex},} ограничивающей объём~$\mathcal{V}$);
${\variation{\internalwork}\hspace{-0.2ex}}$\:--- работа внутренних сил \en{per volume unit}\ru{на~единицу объёма}.

По\hbox{-}прежнему полагаем, что~${\variation{\internalwork}\hspace{-0.2ex}}$ обнуляется при движении тела как целого без деформации:

\nopagebreak\vspace{-0.1em}\begin{equation*}
\begin{array}{c}
\variation{\fieldofdisplacements} = \variation{\fieldofrotations} \hspace{-0.2ex} \times \hspace{-0.12ex} \bm{r} + \boldconst \hspace{.1ex} ,
\:\,
\variation{\fieldofrotations} = \boldconst
\hspace{.64ex}\Rightarrow\hspace{.5ex}
\variation{\internalwork} \hspace{-0.2ex} = 0 \hspace{.1ex},
\\[.25em]
%
\boldnabla \variation{\fieldofdisplacements}
= \hspace{-0.25ex} \boldnabla \variation{\fieldofrotations} %%{^2\bm{0}}
\hspace{-0.2ex} \times \hspace{-0.2ex} \bm{r}
- \hspace{-0.25ex} \boldnabla \bm{r} \hspace{-0.2ex} \times \hspace{-0.2ex} \variation{\fieldofrotations}
= - \bm{E} \hspace{-0.2ex} \times \hspace{-0.2ex} \variation{\fieldofrotations}
= - \hspace{.2ex} \variation{\fieldofrotations} \hspace{-0.2ex} \times \hspace{-0.2ex} \bm{E}
\hspace{.1ex} ,
\:\,
\boldnabla \variation{\fieldofrotations} = {^2\bm{0}} \hspace{.1ex} .
\end{array}
\end{equation*}

\end{otherlanguage}

\vspace{-0.1em} \en{Introducing deformation tensors}\ru{Вводя тензоры деформации}\:---
\en{tensor of relative displacement between particles}\ru{тензор относительного смещения между частицами} (distortion tensor, strain tensor)~$\distortiontensor$
\en{and}\ru{и}~\en{curvature-twist tensor}\ru{тензор искривления-скручивания} (\ru{curvature-twist tensor, }wryness tensor)~$\wrynesstensor$\:--- \en{as}\ru{как}

\nopagebreak\vspace{-0.1em}
\begin{equation}\label{deformationtensors:micropolarcontinuum}
\begin{array}{rl}
\distortiontensor \hspace{.16ex} \equiv \boldnabla \fieldofdisplacements \hspace{.1ex} + \fieldofrotations \hspace{-0.12ex} \times \hspace{-0.2ex} \bm{E} , &
\wrynesstensor \hspace{.16ex} \equiv \boldnabla \fieldofrotations \hspace{.1ex} ,
\end{array}
\end{equation}
%
\nopagebreak\vspace{-0.85em}
\begin{equation*}
\begin{array}{rl}
\distortiontensor_{\hspace{-0.25ex}\Xcompanion} \hspace{-0.16ex}
= \boldnabla \hspace{-0.16ex} \times \hspace{-0.1ex} \fieldofdisplacements \hspace{.12ex} - \hspace{.12ex} 2 \hspace{.16ex} \fieldofrotations \hspace{.1ex}
, &
\wrynesstensor_{\hspace{.1ex}\Xcompanion} \hspace{-0.16ex}
= \boldnabla \hspace{-0.16ex} \times \hspace{-0.1ex} \fieldofrotations \hspace{.1ex} ,
\\[.3em]
%
\variation{\distortiontensor}
= \hspace{-0.16ex} \boldnabla \variation{\fieldofdisplacements} \hspace{.1ex}
+ \variation{\fieldofrotations} \hspace{-0.2ex} \times \hspace{-0.2ex} \bm{E} \hspace{.1ex}
, &
\variation{\wrynesstensor} = \hspace{-0.16ex} \boldnabla \variation{\fieldofrotations} \hspace{.1ex} ,
\end{array}
\end{equation*}

\vspace{-0.56em} \noindent \en{we have the~necessary absence of virtual deformations}\ru{имеем нужное отсутствие виртуальных деформаций}
${\variation{\distortiontensor} = \hspace{-0.1ex} {^2\bm{0}}}$ \en{and}\ru{и}~${\variation{\wrynesstensor} = \hspace{-0.1ex} {^2\bm{0}}}$.

\en{In}\ru{В}~\chapdotpararef{chapter:nonlinearcontinuum}{para:stressesAsLagrangeMultipliers} \en{for momentless continuum}\ru{для безмоментной среды}, \en{stresses appear as Lagrange multipliers in the~principle of~virtual work with}\ru{напряжения появляются как множители Lagrange’а в~принципе виртуальной работы при}~${\variation{\internalwork} \hspace{-0.2ex} = 0}$. \en{The~same here}\ru{Так~же и~тут}:

\nopagebreak\vspace{-0.3em}
\begin{multline}\label{virtualworkprinciple.1:micropolarcontinuum}
\integral\displaylimits_{\mathcal{V}} \hspace{-0.64ex} \left(^{\mathstrut} \bm{f} \dotp \variation{\fieldofdisplacements}
+ \bm{m} \dotp \variation{\fieldofrotations}
- \forcestresstensor \dotdotp \variation{\distortiontensor}^{\T} \hspace{-0.4ex}
- \couplestresstensor \dotdotp \variation{\wrynesstensor}^{\T} \hspace{-0.05ex} \right) \hspace{-0.32ex} d\mathcal{V} \hspace{.5ex} + \\[-1.1em]
%
+ \integral\displaylimits_o \hspace{-0.64ex} \left(^{\mathstrut} \bm{p} \dotp \variation{\fieldofdisplacements}
+ \mathboldM \hspace{-0.1ex} \dotp \variation{\fieldofrotations} \right) \hspace{-0.32ex} do \hspace{0.12ex} = \hspace{0.1ex} 0 \hspace{0.1ex}.
\end{multline}

\begin{otherlanguage}{russian}

\vspace{-0.2em} \noindent Множители Лагранжа в~каждой точке\:--- это несимметричные тензоры второй сложности~$\forcestresstensor$ и~$\couplestresstensor$.
%%энергетически сопрягаемые с~$\distortiontensor$ и~$\wrynesstensor$

Преобразуем ${- \hspace{.2ex} \forcestresstensor \dotdotp \variation{\distortiontensor}^{\T}}$ и~${- \hspace{.2ex} \couplestresstensor \dotdotp \variation{\wrynesstensor}^{\T}}$

\nopagebreak\begin{equation*}
\begin{array}{rl}
\variation{\distortiontensor}^{\T} \hspace{-0.32ex}
= \hspace{-0.16ex} \boldnabla \variation{\fieldofdisplacements}^{\hspace{-0.05ex}\T} \hspace{-0.3ex}
- \hspace{.1ex} \variation{\fieldofrotations} \hspace{-0.2ex} \times \hspace{-0.2ex} \bm{E} \hspace{.1ex}
, &
\variation{\wrynesstensor}^{\T} \hspace{-0.32ex} = \hspace{-0.16ex} \boldnabla \variation{\fieldofrotations}^{\T} \hspace{-0.25ex} ,
\\[.25em]
%
- \hspace{.2ex} \forcestresstensor \dotdotp \variation{\distortiontensor}^{\T} \hspace{-0.32ex}
= - \hspace{.2ex} \forcestresstensor \dotdotp \hspace{-0.16ex} \boldnabla \variation{\fieldofdisplacements}^{\hspace{-0.05ex}\T} \hspace{-0.3ex}
+ \forcestresstensor \dotdotp \hspace{-0.32ex} \left( \hspace{.12ex} \variation{\fieldofrotations} \hspace{-0.2ex} \times \hspace{-0.2ex} \bm{E} \hspace{.12ex} \right) \hspace{-0.25ex}
, &
- \hspace{.2ex} \couplestresstensor \dotdotp \variation{\wrynesstensor}^{\T} \hspace{-0.32ex}
= - \hspace{.2ex} \couplestresstensor \dotdotp \hspace{-0.16ex} \boldnabla \variation{\fieldofrotations}^{\T} \hspace{-0.25ex} .
\end{array}
\end{equation*}

\noindent Используя

\nopagebreak\vspace{-1em}\begin{equation*}
\eqrefwithchapdotpara{pseudovectorinvariant}{chapter:elementsoftensorcalculus}{para:tensors.symmetric+skewsymmetric} \:\Rightarrow\,
\bm{A}_{\Xcompanion} \hspace{-0.1ex} = - \hspace{.1ex} \bm{A} \dotdotp \levicivitatensor
\hspace{.1ex} ,
\end{equation*}

\nopagebreak\vspace{-0.5em}\begin{multline*}
\bm{A} \dotdotp \hspace{-0.25ex} \left( \hspace{.1ex} \bm{b} \times \hspace{-0.2ex} \bm{E} \hspace{.1ex} \right)
= \bm{A} \dotdotp \hspace{-0.32ex} \left( \hspace{.1ex} \bm{E} \hspace{-0.12ex} \times \bm{b} \hspace{.1ex} \right)
= \bm{A} \dotdotp \hspace{-0.32ex} \left( \hspace{-0.1ex} - \hspace{.2ex} \levicivitatensor \dotp \bm{b} \hspace{.1ex} \right) =
\\[-0.2em]
%
\hspace*{\fill} = \left( \hspace{-0.1ex} - \hspace{.1ex} \bm{A} \dotdotp \levicivitatensor \hspace{.1ex} \right) \hspace{-0.2ex} \dotp \hspace{.1ex} \bm{b}
\hspace{.12ex} = \bm{A}_{\Xcompanion} \hspace{-0.1ex} \dotp \hspace{.1ex} \bm{b}
\hspace{.64ex} \Rightarrow
\end{multline*}

\nopagebreak\vspace{-0.4em}\begin{equation*}
\Rightarrow \hspace{.64ex}
\forcestresstensor \dotdotp \hspace{-0.25ex} \left( \hspace{.12ex} \variation{\fieldofrotations} \hspace{-0.2ex} \times \hspace{-0.2ex} \bm{E} \hspace{.12ex} \right)
= \forcestresstensor_{\Xcompanion} \hspace{-0.1ex} \dotp \hspace{.1ex} \variation{\fieldofrotations}
\end{equation*}

\vspace{-0.8em} \noindent и \inquotes{product rule}

\nopagebreak\vspace{-0.2em}\begin{equation*}\begin{array}{c}
\boldnabla \dotp \left( \hspace{.1ex} \forcestresstensor \dotp \variation{\fieldofdisplacements} \right)
= \left( \hspace{.12ex} \boldnabla \dotp \forcestresstensor \hspace{.12ex} \right) \hspace{-0.16ex} \dotp \variation{\fieldofdisplacements}
+ \forcestresstensor \dotdotp \hspace{-0.16ex} \boldnabla \variation{\fieldofdisplacements}^{\hspace{-0.05ex}\T} \hspace{-0.3ex} ,
\\[.25em]
%
\boldnabla \dotp \left( \hspace{.1ex} \couplestresstensor \dotp \variation{\fieldofrotations} \right)
= \left( \hspace{.12ex} \boldnabla \dotp \couplestresstensor \hspace{.12ex} \right) \hspace{-0.16ex} \dotp \variation{\fieldofrotations}
+ \couplestresstensor \dotdotp \hspace{-0.16ex} \boldnabla \variation{\fieldofrotations}^{\hspace{-0.05ex}\T} \hspace{-0.3ex} ,
\end{array}\end{equation*}

\vspace{-0.4em} \noindent получаем

\nopagebreak\vspace{-0.4em}\begin{equation*}
\begin{array}{c}
- \hspace{.2ex} \forcestresstensor \dotdotp \variation{\distortiontensor}^{\T} \hspace{-0.32ex}
= \left( \hspace{.12ex} \boldnabla \dotp \forcestresstensor \hspace{.12ex} \right) \hspace{-0.16ex} \dotp \variation{\fieldofdisplacements}
- \hspace{-0.15ex} \boldnabla \dotp \left( \hspace{.1ex} \forcestresstensor \dotp \variation{\fieldofdisplacements} \right) \hspace{.1ex}
+ \hspace{.1ex} \forcestresstensor_{\Xcompanion} \hspace{-0.1ex} \dotp \hspace{.1ex} \variation{\fieldofrotations} \hspace{.1ex} ,
\\[.25em]
%
- \hspace{.2ex} \couplestresstensor \dotdotp \variation{\wrynesstensor}^{\T} \hspace{-0.32ex}
= \left( \hspace{.12ex} \boldnabla \dotp \couplestresstensor \hspace{.12ex} \right) \hspace{-0.16ex} \dotp \variation{\fieldofrotations}
- \hspace{-0.15ex} \boldnabla \dotp \left( \hspace{.1ex} \couplestresstensor \dotp \variation{\fieldofrotations} \right) \hspace{-0.25ex} .
\end{array}
\end{equation*}

\vspace{-0.25em} \noindent После интегрирования с~применением теоремы о~дивергенции\footnote{${\bm{a} \dotp ({^2\!\bm{B}} \dotp \bm{c} \hspace{.1ex}) = (\bm{a} \dotp \hspace{-0.12ex} {^2\!\bm{B}}) \hspace{-0.1ex} \dotp \bm{c} = \bm{a} \dotp \hspace{-0.12ex} {^2\!\bm{B}} \dotp \bm{c}}$}

\nopagebreak\vspace{-0.1em}\begin{equation*}
\displaystyle
\integral\displaylimits_{\mathcal{V}} \hspace{-0.5ex} \boldnabla \dotp \left( \forcestresstensor \dotp \variation{\fieldofdisplacements} \right) \hspace{-0.1ex} d\mathcal{V}
=
\ointegral\displaylimits_{\mathclap{\widearc{o}\hspace{.15ex}(\boundary \mathcal{V})}} \hspace{-0.1ex} \bm{n} \dotp \forcestresstensor \dotp \variation{\fieldofdisplacements} \hspace{.32ex} do
\hspace{.1ex} ,
\:\:
%
\displaystyle
\integral\displaylimits_{\mathcal{V}} \hspace{-0.5ex} \boldnabla \dotp \left( \hspace{.1ex} \couplestresstensor \dotp \variation{\fieldofrotations} \right) \hspace{-0.1ex} d\mathcal{V}
=
\ointegral\displaylimits_{\mathclap{\widearc{o}\hspace{.15ex}(\boundary \mathcal{V})}} \hspace{-0.1ex} \bm{n} \dotp \couplestresstensor \dotp \variation{\fieldofrotations} \hspace{.32ex} do
%%\hspace{.1ex} ,
\end{equation*}

\vspace{-0.25em} \noindent \eqref{virtualworkprinciple.1:micropolarcontinuum} приобретает вид

\nopagebreak\vspace{-0.33em}
\begin{multline*}\label{virtualworkprinciple.2:micropolarcontinuum}
\integral\displaylimits_{\mathcal{V}} \hspace{-0.64ex} \left(^{\mathstrut} ( \hspace{.12ex} \boldnabla \hspace{-0.1ex} \dotp \forcestresstensor + \bm{f} \hspace{.12ex} ) \dotp \variation{\fieldofdisplacements} \hspace{.1ex}
+ ( \hspace{.12ex} \boldnabla \dotp \couplestresstensor \hspace{.1ex} + \forcestresstensor_{\Xcompanion} \hspace{-0.1ex} + \bm{m} \hspace{.1ex} ) \dotp \variation{\fieldofrotations} \right) \hspace{-0.32ex} d\mathcal{V} \hspace{0.5ex} + \\[-1.1em]
%
+ \integral\displaylimits_o \hspace{-0.64ex} \left(^{\mathstrut} ( \hspace{.32ex} \bm{p} - \bm{n} \dotp \forcestresstensor \hspace{.2ex} ) \dotp \variation{\fieldofdisplacements} \hspace{.1ex}
+ ( \hspace{.1ex} \mathboldM - \bm{n} \dotp \couplestresstensor \hspace{.1ex} ) \dotp \variation{\fieldofrotations} \right) \hspace{-0.32ex} do \hspace{0.12ex} = \hspace{0.1ex} 0 \hspace{0.1ex}.
\end{multline*}

Из произвольности вариаций~$\variation{\fieldofdisplacements}$ и~$\variation{\fieldofrotations}$ (и~в~объёме, и~на~поверхности) вытекают уравнения баланса сил и~моментов, а~также краевые условия:
%% формулы типа Коши Cauchy
%% раскрывающие смысл $\forcestresstensor$ и~$\couplestresstensor$:

\nopagebreak\vspace{-0.1em}\begin{equation}\label{micropolar.equilibrium:cauchy-like}
\boldnabla \hspace{-0.1ex} \dotp \forcestresstensor + \bm{f} \hspace{.12ex} = \bm{0} \hspace{.1ex} ,
\:\:
\boldnabla \dotp \couplestresstensor \hspace{.1ex} + \forcestresstensor_{\Xcompanion} \hspace{-0.1ex} + \bm{m} \hspace{.1ex} = \bm{0} \hspace{.1ex} ,
\end{equation}%
\nopagebreak\vspace{-1.1em}\begin{equation}\label{micropolar.equilibrium:boundaryconditions}
\bm{n} \dotp \forcestresstensor = \bm{p} \hspace{.1ex} ,
\:\:
\bm{n} \dotp \couplestresstensor = \mathboldM \hspace{.1ex} .
\end{equation}

\vspace{-0.5em} Тензор силового напряжения~$\forcestresstensor$ удовлетворяет тем~же дифференциальным \inquotes{уравнениям равновесия}\footnote{Кавычки здесь оттого, что \emph{уравнения равновесия} это вообще всё, что вытекает из принципа виртуальной работы в~статике.} \hspace{-0.5em} и~краевым условиям, что и в~безмоментной среде. \en{But tensor}\ru{Но~тензор}~$\forcestresstensor$ \en{is asymmetric}\ru{несимметричен}: \en{instead of}\ru{вместо} ${\forcestresstensor_{\Xcompanion} \hspace{-0.2ex} = \bm{0}}$ \en{here is}\ru{тут} ${\hspace{-0.1ex} \boldnabla \dotp \couplestresstensor \hspace{.1ex} + \forcestresstensor_{\Xcompanion} \hspace{-0.1ex} + \bm{m} \hspace{.1ex} = \bm{0}}$\:--- \ru{появляются }\en{couple stresses}\ru{моментные напряжения}~$\couplestresstensor$\en{ appear}, \en{and}\ru{и} \en{volume}\ru{объёмная} \en{moment load}\ru{моментная нагрузка}~$\bm{m}$ \ru{не~нулевая}\en{is non-zero}.

\end{otherlanguage}

\en{Meaning of~components}\ru{Смысл компонент} \en{of~the~couple stress tensor}\ru{тензора моментного напряжения}~$\couplestresstensor$ \en{is revealed similarly as for}\ru{раскрывается так~же, как и для}~$\forcestresstensor$.
\en{For an~orthonormal basis}\ru{Для~ортонормального базиса}, \en{moment}\ru{момент} ${\bm{M}_i \hspace{-0.1ex} = \bm{e}_i \hspace{-0.1ex} \dotp \couplestresstensor = \mu_{ik} \bm{e}_k}$ \en{acts}\ru{действует} \en{on an~area}\ru{на~площ\'{а}дке} \en{with normal}\ru{c~нормалью}~$\bm{e}_i$.
\en{Diagonal components}\ru{Диагональные компоненты}~$\mu_{11}$, $\mu_{22}$, $\mu_{33}$ \en{are twisting moments}\ru{это крутящие моменты}, \en{nondiagonal are bending ones}\ru{недиагональные\:--- изгибающие} (?? \figurename ??).

...




\en{\section{Relations of elasticity}}

\ru{\section{Отношения упругости}}

\begin{otherlanguage}{russian}

В~этой книге упругой называется среда с~потенциальными внутренними силами: ${\variation{\internalwork} = - \hspace{.16ex} \variation{\Pi}}$, где~$\Pi$\:--- энергия деформации на~единицу объёма (по\hbox{-}прежнему рассматриваем геометрически линейную постановку).

Располагая соотношениями

...

\[
\variation{\Pi} = - \hspace{.2ex} \variation{\internalwork} \hspace{-0.15ex}
= \forcestresstensor \dotdotp \variation{\distortiontensor}^{\T} \hspace{-0.4ex}
+ \couplestresstensor \dotdotp \variation{\wrynesstensor}^{\T}
\]

...

..., разлагая тензоры деформаций и~напряжений на~симметричные и~антисимметричные части

\nopagebreak\begin{equation}
\begin{array}{c}
\forcestresstensor = \forcestresstensor^{\hspace{.32ex}\mathsf{S}} \hspace{-0.25ex} - \displaystyle \onehalf \hspace{.2ex} \forcestresstensor_{\hspace{-0.1ex}\Xcompanion} \hspace{-0.1ex} \times \bm{E} , \;\:
%
\couplestresstensor = \couplestresstensor^{\mathsf{S}} \hspace{-0.25ex} - \displaystyle \onehalf \hspace{.32ex} \couplestresstensor_{\Xcompanion} \hspace{-0.1ex} \times \bm{E} , \;\:
%
\wrynesstensor = \wrynesstensor^{\hspace{.12ex}\mathsf{S}} \hspace{-0.25ex} - \displaystyle \onehalf \hspace{.32ex} \wrynesstensor_{\Xcompanion} \hspace{-0.1ex} \times \bm{E} ,
\\[.6em]
%
\distortiontensor = \mathboldepsilon - \displaystyle \onehalf \hspace{.32ex} \distortiontensor_{\hspace{-0.25ex}\Xcompanion} \hspace{-0.1ex} \times \bm{E} , \:\:
%
\mathboldepsilon \equiv \distortiontensor^{\hspace{.2ex}\mathsf{S}} \hspace{-0.32ex} = \hspace{-0.2ex} \boldnabla {\fieldofdisplacements}^{\hspace{0.1ex}\mathsf{S}} \hspace{-0.25ex} , \:\:
%
\distortiontensor_{\hspace{-0.25ex}\Xcompanion} \hspace{-0.16ex}
= \boldnabla \hspace{-0.16ex} \times \hspace{-0.1ex} \fieldofdisplacements \hspace{.12ex} - \hspace{.12ex} 2 \hspace{.16ex} \fieldofrotations \hspace{.16ex} ;
\\[.6em]
%
\variation{\Pi} = \ldots
\end{array}
\end{equation}

...

%% The classical isotropic linear elastic material behavior is presented by two material parameters, for example, the Young’s modulus and the Poisson’s ratio, while the isotropic Cosserat continuum needs six material parameters


...

Если устремить $h$ к~нулю, исчезает вклад $\wrynesstensor$ в~$\Pi$, а~с~ним и моментные напряжения~$\couplestresstensor$.
Когда вдобавок нет объёмной моментной нагрузки~$\bm{m}$, тогда тензор~${\hspace{-0.1ex}\forcestresstensor}$ становится симметричным:
${\boldnabla \dotp {\color{black!40}{\couplestresstensor}} \hspace{.1ex}
+ \forcestresstensor_{\Xcompanion} \hspace{-0.1ex}
+ {\color{black!40}{\bm{m}}} \hspace{.1ex}
= \bm{0}}$,
${{\color{black!40}{\couplestresstensor}} = \hspace{-0.15ex} {^2\bm{0}}}$,
${{\color{black!40}{\bm{m}}} = \bm{0} \,\Rightarrow \hspace{.1ex} \forcestresstensor_{\Xcompanion} \hspace{-0.2ex} = \bm{0}}$,
и модель превращается в~классическую.

\end{otherlanguage}

\en{Yet using of}\ru{Использование~же} \en{micropolar model}\ru{микрополярной модели} \en{is natural}\ru{естественно} \en{in~case}\ru{в~случае},
\en{when}\ru{когда} \en{the~real material}\ru{реальный материал} \en{has a~certain smallest volume}\ru{имеет некий наименьший объём},
\inquotesx{\en{which is impossible to~enter into}\ru{в~который невозможно войти}}[.]
\en{And}\ru{И}~\en{such situation}\ru{такая ситуация} \en{occurs}\ru{возникает} \ru{нередко}\en{quite often}:
\en{composites}\ru{композиты} \en{with}\ru{с}~\inquotes{\en{representative}\ru{представительным}} \en{volume}\ru{объёмом},
\en{polycrystalline materials}\ru{поликристаллические материалы},
\en{polymers}\ru{полимеры} \en{with}\ru{с}~\en{large molecules}\ru{большими молекулами}~(\en{macro\-molecules}\ru{макро\-молекулами}).

\en{\section{Compatibility equations}}

\ru{\section{Уравнения совместности}}

\label{para:compatibilityequations.cosseratcontinuum}

\begin{otherlanguage}{russian}

Из~выражений тензоров деформации~\eqref{deformationtensors:micropolarcontinuum} следует

...



\end{otherlanguage}

\en{\section{Theorems of statics}}

\ru{\section{Теоремы статики}}

\begin{otherlanguage}{russian}

Теоремы статики линейных консервативных систем, легко выводимые при конечном числе степеней свободы

...



\end{otherlanguage}

\en{\section{Cosserat pseudocontinuum}}

\ru{\section{Псевдоконтинуум Коссера}}

\label{para:caseoflatenttrihedron.smalldisplacementsandrotations}

\begin{otherlanguage}{russian}

Так называется упрощённая моментная модель\footnote{\ru{Братья }Cosserat\en{ brothers} \en{called it}\ru{называли это} cas de trièdre caché (\ru{случай скрытого трёхгранника, }case of latent trihedron).}\hspace{-0.32em},\hspace{0.24em} в~которой повороты выражаются через перемещения как в~классической среде: %%~\cite{nowacki-elasticity}:

\nopagebreak\vspace{-0.1em}\begin{equation}
\fieldofrotations = \displaystyle \onehalf \hspace{0.4ex} \boldnabla \hspace{-0.16ex} \times \hspace{-0.1ex} \fieldofdisplacements
\:\Leftrightarrow\:
\distortiontensor_{\hspace{-0.25ex}\Xcompanion} \hspace{-0.32ex} = \bm{0}
\:\Leftrightarrow\:
\distortiontensor = \mathboldepsilon = \hspace{-0.12ex} \boldnabla {\fieldofdisplacements}^{\hspace{0.1ex}\mathsf{S}} \hspace{-0.25ex}.
\end{equation}

Равенство~${\distortiontensor_{\hspace{-0.25ex}\Xcompanion} \hspace{-0.32ex} = \bm{0}}$~(симметрию~$\distortiontensor$) возможно понимать как внутреннюю связь~(\chapdotpararef{chapter:nonlinearcontinuum}{para:internalconstraints}). Аргумент~${\distortiontensor_{\hspace{-0.25ex}\Xcompanion}}$ исчезает из~энергии~$\Pi$, соотношение упругости для~${\forcestresstensor_{\hspace{-0.1ex}\Xcompanion}}$ не~может быть написано. Его место в~полной системе занимает уравнение связи.

В~классической теории упругости полная система сводится к~одному уравнению для~вектора~$\fieldofdisplacements$~(\chapdotpararef{chapter:linearclassicalelasticity}{para:equationsfordisplacement}). В~моментной теории

...



\end{otherlanguage}

\en{\section{Plane deformation}}

\ru{\section{Плоская деформация}}

\label{para:planedeformation.cosseratcontinuum}

\begin{otherlanguage}{russian}

Все переменные в~этой постановке не~зависят от декартовой координаты~${z \equiv x_3}$~(орт оси\:--- $\bm{k}$). Перемещения и~силы перпендикулярны оси~$z$, а~повороты и~моменты\:--- параллельны ей:

...


Это краткое изложение плоской задачи относится к~модели с~независимыми поворотами. Псевдоконтинуум Коссера (модель со~стеснённым вращением) получается либо при наложении внутренней связи~${\distortiontensor_{\hspace{-0.25ex}\Xcompanion} \hspace{-0.32ex} = \bm{0}}$, либо при предельном переходе ...

Подробнее о~плоской моментной задаче написано в~книгах Н.\,Ф.\;Морозова~\cite{morozov-twodimensionalproblems, morozov-fractures}.

\end{otherlanguage}

\en{\section{Nonlinear theory}}

\ru{\section{Нелинейная теория}}

\label{para:nonlinear.micropolar}

\begin{otherlanguage}{russian}

Кажущееся на~первый взгляд чрезвычайно трудным, построение теории конечных деформаций континуума Коссера становится прозрачным, если опираться на общую механику, тензорное исчисление и~нелинейную теорию безмоментной среды.

При построении модели упругого континуума обычно проходят четыре этапа:
\begin{itemize}
\item определение степеней свободы частиц,
\item выявление нагрузок~(\inquotesx{силовых факторов}[,] напряжений) и~условий их баланса,
\item подбор соответствующих мер деформации
\\
\hspace*{-\listlabelwithsep}и, наконец,
\item вывод соотношений упругости между напряжением и~деформацией.
\end{itemize}

\vspace{-0.16em} \noindent Этот традиционный путь очень сокращается, если опираться на принцип виртуальной работы.

Как и в~\chapref{chapter:nonlinearcontinuum}, среда состоит из~частиц с~материальными координатами~$q^i$ и вектором\hbox{-}радиусом~${\currentvector(q^i\hspace{-0.4ex}, t)}$.
В~начальной~(исходной, отсчётной) конфигурации ${\currentvector(q^i\hspace{-0.4ex}, 0) \equiv \initialvector(q^i)}$.
Но кроме трансляции, частицы имеют независимые степени свободы поворота, описываемого ортогональным тензором

\nopagebreak\vspace{-0.1em}\begin{equation*}
\rotationtensor(q^i\hspace{-0.4ex}, t)
\equiv
\bm{a}_{\hspace{-0.12ex}j} \hspace{.16ex} \mathcircabove{\bm{a}}^j \hspace{-0.32ex}
= \hspace{-0.1ex} \bm{a}^j \hspace{-0.12ex} \mathcircabove{\bm{a}}_{\hspace{-0.12ex}j} \hspace{-0.25ex}
= \hspace{-0.1ex} \rotationtensor^{\hspace{-0.1ex}\expminusT}
\hspace{-0.5ex} ,
\end{equation*}

\vspace{-0.1em} \noindent где тройка векторов~${\bm{a}_{\hspace{-0.12ex}j}(q^i\hspace{-0.4ex}, t)}$ жёстко связана с~каждой частицей, показывая угловую ориентацию относительно как\hbox{-}либо выбираемых\footnote{Один из вариантов: ${\mathcircabove{\bm{a}}_{\hspace{-0.12ex}j} \hspace{-0.4ex} = \initialvector_{\hspace{-0.2ex}j} \hspace{-0.1ex} \equiv \partial_j \currentvector}$.
Другое предложение: ${\mathcircabove{\bm{a}}_{\hspace{-0.12ex}j} \hspace{-0.2ex}}$ это ортонормальная тройка собственных векторов тензора инерции частицы. %% (но как обосновать такой выбор в~статике?)
Вообще, ${\mathcircabove{\bm{a}}_{\hspace{-0.12ex}j} \hspace{-0.2ex}}$ могут быть любой тройкой линейно независимых векторов.}\hspace{-0.25ex}
векторов~${\mathcircabove{\bm{a}}_{\hspace{-0.12ex}j}(q^i) \equiv \bm{a}_{\hspace{-0.12ex}j}(q^i\hspace{-0.4ex}, 0)}$,
${\bm{a}_{\hspace{-0.12ex}j} \hspace{-0.2ex} = \rotationtensor \hspace{-0.1ex} \dotp \hspace{.2ex} \mathcircabove{\bm{a}}_{\hspace{-0.12ex}j}}$;
${\bm{a}^j\hspace{-0.2ex}}$\:--- тройка взаимных векторов:
${\bm{a}_{\hspace{-0.12ex}j} \bm{a}^j \hspace{-0.25ex} = \hspace{-0.15ex} \bm{a}^j \bm{a}_{\hspace{-0.12ex}j} \hspace{-0.25ex} = \hspace{-0.2ex} \bm{E}}$
(${t\!=\!0}$, ${\mathcircabove{\bm{a}}^j}$: ${\mathcircabove{\bm{a}}_{\hspace{-0.12ex}j} \mathcircabove{\bm{a}}^j \hspace{-0.25ex} = \hspace{-0.15ex} \mathcircabove{\bm{a}}^j \mathcircabove{\bm{a}}_{\hspace{-0.12ex}j} \hspace{-0.25ex} = \hspace{-0.2ex} \bm{E}}$).
Движение среды полностью определяется функциями ${\currentvector(q^i\hspace{-0.4ex}, t)}$ и~${\rotationtensor(q^i\hspace{-0.4ex}, t)}$.

Имея представления ${\initialvector(q^i)}$ и~${\currentvector(q^i\hspace{-0.4ex}, t)}$,
вводим базис~${\currentvector_i \equiv \partial_i \currentvector}\hspace{-0.2ex}$,
взаимный базис~${\currentvector^i}$: ${\currentvector_{\hspace{-0.2ex}j} \hspace{-0.2ex} \dotp \currentvector^i \hspace{-0.25ex} = \delta^{\hspace{.1ex}i}_{\hspace{-0.2ex}j}}$,
дифференциальные операторы~$\boldnablacircled$ и~${\hspace{-0.16ex}\boldnabla}\hspace{-0.1ex}$,
а~также градиент движения~$\motiongradient$

\nopagebreak\vspace{-0.1em}\begin{equation}
\boldnablacircled \equiv \initialvector^{\hspace{.1ex}i} \partial_i
\hspace{.1ex} ,
\:\;
\boldnabla \equiv \currentvector^i \partial_i
\hspace{.1ex} ,
\:\;
\boldnabla = \motiongradient^{\expminusT} \hspace{-0.32ex} \dotp \boldnablacircled
,
\:\;
\motiongradient \equiv \hspace{-0.16ex}
\boldnablacircled \currentvector^{\T} \hspace{-0.5ex}
= \currentvector_i \hspace{.2ex} \initialvector^{\hspace{.1ex}i}
\hspace{-0.25ex} .
\end{equation}

Вариационное уравнение принципа виртуальной работы для континуума с~нагрузками в~объёме и на~поверхности:

\nopagebreak\vspace{-0.3em}\begin{multline}
\hspace*{2em}
\integral\displaylimits_{\mathcal{V}} \hspace{-0.64ex} \left(^{\mathstrut} \hspace{-0.1ex} \rho \hspace{.2ex} \bigl( \bm{f} \dotp \variation{\currentvector} + \bm{m} \dotp \littlerotationvector \hspace{.1ex} \bigr) \hspace{-0.1ex} + \hspace{.1ex} \variation{\internalwork} \hspace{.1ex} \right) \hspace{-0.3ex} d\mathcal{V}
\hspace{.5ex} + \\[-1.2em]
%
+ \integral\displaylimits_{\mathcal{O}} \hspace{-0.64ex} \left(^{\mathstrut} \bm{p} \dotp \variation{\currentvector} + \mathboldM \hspace{-0.1ex} \dotp \littlerotationvector \right) \hspace{-0.3ex} d\mathcal{O} \hspace{.1ex}
= \hspace{.1ex} 0 \hspace{.1ex} .
\hspace*{2em}
\end{multline}

\vspace{-0.2em} \noindent \en{Here}\ru{Здесь}
$\rho$~\en{is mass density}\ru{--- плотность массы};
$\bm{f}$ и~$\bm{m}$\:--- внешние сила и~момент \en{per mass unit}\ru{на~единицу массы};
$\bm{p}$ и~$\mathboldM$\:--- они~же \en{per surface unit}\ru{на~единицу поверхности};
${\variation{\internalwork}\hspace{-0.2ex}}$\:--- работа внутренних сил \en{per volume unit}\ru{на~единицу объёма} в~текущей конфигурации.
Вектор м\'{а}лого поворота~$\littlerotationvector$

\nopagebreak\vspace{-0.5em}\begin{multline*}
\scalebox{0.98}{$ \rotationtensor \hspace{-0.1ex} \dotp \rotationtensor^{\T} \hspace{-0.2ex} = \bm{E}
\:\Rightarrow\,
\variation{\rotationtensor} \hspace{-0.1ex} \dotp \rotationtensor^{\T} \hspace{-0.2ex}
= - \hspace{.2ex} \rotationtensor \hspace{-0.1ex} \dotp \variation{\rotationtensor}^{\T}
\Rightarrow $}
\\[-0.15em]
%
\scalebox{0.98}{$ \Rightarrow\,
\variation{\rotationtensor} \hspace{-0.1ex} \dotp \rotationtensor^{\T} \hspace{-0.25ex}
= \hspace{.1ex} \littlerotationvector \hspace{-0.2ex} \times \hspace{-0.2ex} \bm{E}
= \hspace{.1ex} \littlerotationvector \hspace{-0.2ex} \times \hspace{-0.2ex} \rotationtensor \hspace{-0.1ex} \dotp \rotationtensor^{\T}
\Rightarrow\,
\variation{\rotationtensor} \hspace{-0.1ex} = \littlerotationvector \hspace{-0.2ex} \times \hspace{-0.2ex} \rotationtensor , $}
\end{multline*}

\vspace{-0.4em}\begin{equation*}
\littlerotationvector = - \, \smalldisplaystyleonehalf \hspace{.1ex}
\scalebox{1}{$ \left( \variation{\rotationtensor} \hspace{-0.1ex} \dotp^{\mathstrut} \hspace{-0.1ex} \rotationtensor^{\T} \hspace{.1ex} \right)_{\hspace{-0.15em}\Xcompanion} $}
\end{equation*}

\vspace{-0.1em} При движении среды как жёсткого целого нет деформаций, и~работа~$\variation{\internalwork}\hspace{-0.2ex}$ внутренних сил равна нулю:

\nopagebreak\vspace{-0.1em}\begin{equation*}
\begin{array}{c}
\variation{\currentvector}
= \boldconst \hspace{.1ex}
+ \hspace{.1ex} \littlerotationvector \hspace{-0.2ex} \times \hspace{-0.12ex} \bm{r} ,
\:\,
\littlerotationvector = \boldconst
\hspace{.64ex}\Rightarrow\hspace{.5ex}
\variation{\internalwork} \hspace{-0.2ex}
= 0 \hspace{.1ex} ,
\\[.25em]
%
\boldnabla \littlerotationvector
= {^2\bm{0}}
\hspace{.1ex} ,
\:\,
\boldnabla \variation{\currentvector}
= \hspace{-0.25ex} \boldnabla \littlerotationvector %%{^2\bm{0}}
\hspace{-0.2ex} \times \hspace{-0.2ex} \bm{r}
- \hspace{-0.25ex} \boldnabla \bm{r} \hspace{-0.2ex} \times \hspace{-0.2ex} \littlerotationvector
= - \bm{E} \hspace{-0.2ex} \times \hspace{-0.2ex} \littlerotationvector
= - \hspace{.2ex} \littlerotationvector \hspace{-0.2ex} \times \hspace{-0.2ex} \bm{E}
\hspace{.1ex} ,
\\[.25em]
%
\boldnabla \variation{\currentvector}
+ \littlerotationvector \hspace{-0.2ex} \times \hspace{-0.2ex} \bm{E}
= {^2\bm{0}}
\hspace{.1ex} .
\end{array}
\end{equation*}

К~нагрузкам. Несимметричные тензоры напряжения, силового~$\forcestresstensor$ и моментного~$\couplestresstensor$, введём как множители Lagrange’а:

\nopagebreak\vspace{-0.3em}\begin{multline}
\integral\displaylimits_{\mathcal{V}} \hspace{-0.64ex}
\left(^{\mathstrut} \hspace{-0.1ex}
\rho \hspace{.2ex} \bigl(
\bm{f} \dotp \variation{\currentvector}
+ \bm{m} \dotp \littlerotationvector
\hspace{.1ex} \bigr) \hspace{-0.2ex}
- \forcestresstensor \dotdotp \hspace{-0.1ex} \bigl( \hspace{.16ex} \boldnabla \variation{\currentvector} + \littlerotationvector \hspace{-0.2ex} \times \hspace{-0.25ex} \bm{E} \hspace{.2ex} \bigr)^{\hspace{-0.25ex}\T} \hspace{-0.5ex}
- \couplestresstensor \dotdotp \hspace{-0.16ex} \boldnabla \littlerotationvector^{\T}
\right) \hspace{-0.3ex} d\mathcal{V}
\hspace{.5ex} + \\[-1.2em]
%
+ \integral\displaylimits_{\mathcal{O}} \hspace{-0.64ex}
\left(^{\mathstrut}
\bm{p} \dotp \variation{\currentvector}
+ \mathboldM \hspace{-0.1ex} \dotp \littlerotationvector
\right) \hspace{-0.3ex} d\mathcal{O} \hspace{.1ex}
= \hspace{.1ex} 0 \hspace{.1ex} .
\end{multline}

\textcolor{magenta}{После тех~же преобразований, что и~в~\pararef{para:introtolinearmicropolar}, получаем}

...

Отсюда вытекают уравнения баланса сил и~моментов в~объёме и~краевые условия в~виде \textcolor{magenta}{формул типа Cauchy}. Они по~существу те~же, что и в~линейной теории.

Найдём теперь тензоры деформации. Их можно вводить по\hbox{-}разному, если требовать лишь одного\:--- нечувствительности к~движению среды как жёсткого целого. Читатель найдёт не~один вариант таких тензоров. Однако, вид тензоров деформации \inquotes{подсказывает} принцип виртуальной работы.

...



\end{otherlanguage}

\en{\section{Nonlinear model with constrained rotation}}

\ru{\section{Нелинейная модель со стеснённым вращением}}

\label{para:caseoflatenttrihedron.largedisplacementsandrotations}

\begin{otherlanguage}{russian}

Вспомним переход к~модели со~стеснённым вращением в~линейной теории~(\pararef{para:caseoflatenttrihedron.smalldisplacementsandrotations}). Разделились соотношения упругости для~симметричной части тензора силового напряжения ${\forcestresstensor^{\hspace{.32ex}\mathsf{S}}}\hspace{-0.2ex}$ и~кососимметричной его части~${\forcestresstensor_{\hspace{-0.1ex}\Xcompanion}}$. Возникла внутренняя связь~${\distortiontensor_{\hspace{-0.25ex}\Xcompanion} \hspace{-0.32ex} = \bm{0}}$

...




\end{otherlanguage}

\section*{\small \wordforbibliography}

\begin{changemargin}{\parindent}{0pt}
\fontsize{10}{12}\selectfont

\begin{otherlanguage}{russian}

Все работы по~моментной теории упругости упоминают книгу братьев Eugène et~François Cosserat~\cite{cosserat}, где~трёх\-мерной среде посвящена одна глава из~шести. Переведённая монография \hbox{W\hspace{-0.2ex}.\:Nowacki}~\cite{nowacki-elasticity} была одной из первых книг на~русском языке с~изложением линейной моментной теории. Ранее эта область представлялась статьями\:--- например, R.\,D.\:Mindlin’а и~H.\,F.\:Tiersten’а~\cite{mindlin.tiersten}. Краткое изложение моментной теории, но с~подробным рассмотрением задач содержится в~книгах Н.\,Ф.\:Морозова~\cite{morozov-twodimensionalproblems, morozov-fractures}.

\end{otherlanguage}

\end{changemargin}
