\en{\chapter{Micropolar three\hbox{-}dimensional continuum}}

\ru{\chapter{Моментная трёхмерная среда}}

\thispagestyle{empty}

\label{chapter:cosseratcontinuum}

\en{\section{Introduction to linear micropolar theory}}

\ru{\section{Введение в линейную моментную теорию}}

\begin{otherlanguage}{russian}

\lettrine[lines=2, findent=2pt, nindent=0pt]{Х}{арактерной} отличительной особенностью классических сред~(\chapref{chapter:nonlinearcontinuum} и~\customref{chapter:linearclassicalelasticity}) является то, что они \inquotesx{состоят из~обычных материальных точек}[.] Частица классического континуума имеет лишь трансляционные степени свободы, её~движение определяется только вектором~${\bm{R}(q^i \hspace{-0.15em}, t)}$. Поэтому силовыми факторами являются силы~--- объёмные и~поверхностные. Моментов нет.

Но не~так~уж трудно построить более сложные модели сплошной среды, в~которых частицы обладают не~только~лишь степенями свободы трансляции, но~и н\'{е}которыми дополнительными. Новые степени свободы связаны и~с~новыми силовыми факторами, а~также новыми уравнениями.

Наиболее естественная из~неклассических моделей трёхмерной среды предложена братьями Cosserat в~1909\:год\'{у}~\cite{cosserat}. Каждая частица континуума Коссера~--- это элементарное твёрдое тело с~шестью степенями свободы. Силовые факторы в~такой среде~--- силы и~моменты. Работа братьев Коссера оставалась незамеченной полвека, но~затем возник интерес к~этой теме~\cite{mindlin.tiersten, nowacki-elasticity}.

\vspace{2mm}

\begin{tcolorbox}[breakable, enhanced, colback = orange!8, before upper={\parindent3.2ex}, parbox = false]
\small%
\setlength{\abovedisplayskip}{2pt}\setlength{\belowdisplayskip}{2pt}%

\noindent \textit{from \textboldoblique{Nowacki~W.} The Linear Theory of~Micropolar Elasticity.} In: \textit{Micropolar Elasticity. International Centre for~Mechanical Sciences (Courses and~Lectures), vol.\:151, 1974, pp.\:1\hbox{--}43}

\vspace{.5em}

Woldemar Voigt tried to remove the shortcomings of the classical theory of elasticity [\textit{\textboldoblique{W.\:Voigt}. Theoretische Studien über die~Elasticitätsverhältnisse der~Krystalle. Abhandlungen der~Königlichen Gesellschaft der~Wissenschaften in~Göttingen, 34:\:3\hbox{--}51, 1887}] by the assumption that the interaction of two parts of the body is transmitted through an area element $do$ by means not only of the force vector $\bm{p}do$ but also by the moment vector $\bm{m}do$. Thus, besides the force stresses $\sigma_{ji}$ also the moment stresses have been defined.

However, the complete theory of asymmetric elasticity was developed by the brothers \textboldoblique{François et~Eugène Cosserat} who published it in 1909 in the work \textit{“Théorie des corps déformables”}.

They assumed that the body consists of interconnected particles in the form of small rigid bodies. During the deformation each particle is displaced by $\bm{u}(\bm{x},t)$ and rotated by $\bm{\varphi}(\bm{x},t)$, the functions of the position $\bm{x}$ and time $t$.

Thus an elastic continuum has been described such that its points possess the orientation (polar media) and for which we can speak of the rotation of a point. The vectors $\bm{u}$ and $\bm{\varphi}$ are mutually independent and determine the deformation of the body. The introduction of the vectors $\bm{u}$ and $\bm{\varphi}$ and the assumption that the transmission of forces through an area element $do$ is carried out by means of the force vector $\bm{p}$ and the moment vector $\bm{m}$ leads in the consequence to asymmetric stress tensors $\sigma_{ji}$ and $\mu_{ji}$.

The theory of the brothers E.~and~F.\;Cosserat remained unnoticed and was not duly appreciated during their lifetime. This was so because the presentation was very general (the theory was non-linear, including large deformations) and because its frames exceeded the frames of the theory of elasticity. They attempted to construct the unified field theory, containing mechanics, optics and electrodynamics and combined by a general principle of the least action.

The research in the field of the general theories of continuous media conducted in the last fifteen years, drew the attention of the scientists to Cosserats’ work. Looking for the new models, describing more precisely the behaviour of the real elastic media, the models similar to, or identical with that of Cosserats’ have been encountered. Here, we mention, first of all, the papers by C.\:Truesdell and R.\,A.\;Toupin [\textit{\textboldoblique{C.\:Truesdell} and \textboldoblique{R.\,A.\;Toupin}. The classical field theories. Encyclopædia of Physics, Chapter 1, Springer\hbox{-}Verlag, Berlin, 1960}], G.\:Grioli [\textit{\textboldoblique{Grioli~G.} Elasticité asymmetrique. Ann. di Mat. Pura et Appl. Ser. IV, 50 (1960)}], R.\,D.\;Mindlin and H.\,F.\;Tiersten [\textit{\textboldoblique{Mindlin, R.\,D.}; \textboldoblique{Tiersten, H.\,F.} Effects of couple-stresses in linear elasticity. Arch. Rational Mech. Anal. 11. 1962. 415\hbox{--}448}]. At the beginning the author’s attention was concentrated on the simplified theory of elasticity, so called the Cosserat pseudo-continuum. By this name we understand a continuum for which the asymmetric force stresses and moment stresses occur, however, the deformation is determined by the displacement vector $\bm{u}$ only. Here we assume, as in the classical theory of elasticity, that $\bm{\varphi} = \onehalf \operatorname{curl} \bm{u}$. It is interesting to notice that this model was also considered by the Cosserats who called it the model with the latent trihedron.
\par
\end{tcolorbox}

\vspace{2mm}

Рассмотрим сначала геометрически линейную модель, то~есть случай малых перемещений и~поворотов. Векторные поля перемещений~${\bm{u}(\bm{r},t)}$ и~малых поворотов~${\bm{\theta}(\bm{r},t)}$ независимы. Операторы ${\hspace{-0.2ex}\boldnablacircled\hspace{0.1ex}}$ и~${\hspace{-0.2ex}\boldnabla\hspace{0.1ex}}$~(\chapdotpararef{chapter:nonlinearcontinuum}{para:differentiation}) неразличимы, уравнения \inquotesx{можно пис\'{а}ть в~отсчётной конфигурации}[.] За~основу вывода примем, как и~везде в~этой книге, принцип виртуальной работы:
\vspace{0.1em}\begin{equation}
\integral\displaylimits_{\mathcal{V}} \hspace{-0.64ex} \left(^{\mathstrut} \bm{f} \dotp \variation{\bm{u}} + \bm{m} \dotp \variation{\bm{\theta}} + \variation{\internalwork} \hspace{.15ex} \right) \hspace{-0.32ex} d\mathcal{V} +
\integral\displaylimits_o \hspace{-0.64ex} \left(^{\mathstrut} \bm{p} \dotp \variation{\bm{u}} + \mathboldM \hspace{-0.1ex} \dotp \variation{\bm{\theta}} \right) \hspace{-0.32ex} do \hspace{0.12ex} = \hspace{0.1ex} 0 \hspace{0.1ex}.
\end{equation}

\vspace{-0.25em} \noindent Здесь $\bm{f}$ и~$\bm{m}$~--- внешние силы и~моменты на~единицу объёма; $\bm{p}$ и~$\mathboldM$~--- они~же, но на~единицу поверхности; ${\variation{\internalwork}}$~--- работа внутренних сил на~единицу объёма.

По\hbox{-}прежнему считаем, что~${\variation{\internalwork} = 0}$ на~перемещениях твёрдого тела, то~есть при
\begin{equation*}
\variation{\bm{u}} = \variation{\bm{\theta}} \hspace{-0.2ex} \times \hspace{-0.12ex} \bm{r} + \boldconst , \:\: \variation{\bm{\theta}} = \boldconst \:\Leftrightarrow\;
\variation{\bm{\upgamma}} = 0 , \:\: \variation{\mathboldae} = 0 ,
\end{equation*}\vspace{-1.11em}
\begin{equation}\label{deformationtensors:micropolarcontinuum}
\bm{\upgamma} \hspace{.16ex} \equiv \boldnabla \bm{u} \hspace{.1ex} + \bm{E} \hspace{-0.2ex} \times \hspace{-0.12ex} \bm{\theta} , \:\:
\mathboldae \hspace{.2ex} \equiv \boldnabla \bm{\theta} .
\end{equation}

\vspace{-0.55em} Введённые так $\bm{\upgamma}$ и~$\mathboldae$, как нетрудно догадаться, окажутся тензорами деформации.

В~\chapdotpararef{chapter:nonlinearcontinuum}{para:stressesAsLagrangeMultipliers} было показано, что напряжения можно рассматривать как множители Lagrange’а в~принципе виртуальной работы при~${\variation{\internalwork} = 0}$. Поступим так~же и~теперь:

\nopagebreak\vspace{-0.33em}
\begin{multline}\label{virtualworkprinciple:micropolarcontinuum}
\integral\displaylimits_{\mathcal{V}} \hspace{-0.64ex} \left(^{\mathstrut} \bm{f} \dotp \variation{\bm{u}} + \bm{m} \dotp \variation{\bm{\theta}} - \mathboldtau \dotdotp \variation{\bm{\upgamma}}^{\hspace{-0.05ex}\T} \hspace{-0.4ex} - \bm{\mu} \dotdotp \variation{\mathboldae}^{\T} \hspace{-0.05ex} \right) \hspace{-0.32ex} d\mathcal{V} \hspace{0.4ex} + \\[-1em]
+ \integral\displaylimits_o \hspace{-0.64ex} \left(^{\mathstrut} \bm{p} \dotp \variation{\bm{u}} + \mathboldM \hspace{-0.1ex} \dotp \variation{\bm{\theta}} \right) \hspace{-0.32ex} do \hspace{0.12ex} = \hspace{0.1ex} 0 \hspace{0.1ex}.
\end{multline}

\vspace{-0.2em} \noindent Множители Лагранжа в~каждой точке~--- это несимметричные тензоры второй сложности~$\mathboldtau$ и~$\bm{\mu}$.

Используя равенства

...

\noindent и~теорему о~дивергенции, приведём~\eqref{virtualworkprinciple:micropolarcontinuum} к~виду

...

Из\hbox{-}за произвольности вариаций~$\variation{\bm{u}}$ и~$\variation{\bm{\theta}}$ в~объёме и~на~поверхности получаем уравнения баланса сил и~моментов, а~также формулы типа Коши, %%Cauchy,
раскрывающие смысл $\mathboldtau$ и~$\bm{\mu}$:

...

Тензор силовых напряжений удовлетворяет тем~же дифференциальным \inquotes{уравнениям равновесия}\footnote{Кавычки здесь оттого, что уравнениями равновесия вообще\hbox{-}то следует считать всё, что вытекает из принципа виртуальной работы в~статике.} \hspace{-0.5em} и~краевым условиям, что~и в~безмоментной среде. Но~тензор~$\mathboldtau$ несимметричен, поскольку отличны от~нуля моментные напряжения~$\bm{\mu}$ и~нагрузки~$\bm{m}$.

Смысл компонент

...




\en{\section{Relations of elasticity}}

\ru{\section{Отношения упругости}}

В~этой книге упругой называем среду с~потенциальными внутренними силами: ${\variation{\internalwork} = -\variation{\Pi}}$, где~$\Pi$~--- энергия деформации на~единицу объёма (по\hbox{-}прежнему рассматриваем геометрически линейную постановку).

Располагая соотношениями

...

..., разлагая тензоры деформаций и~напряжений на~симметричные и~антисимметричные части
\begin{equation}
\begin{array}{c}
\mathboldtau = \mathboldtau^{\hspace{.32ex}\mathsf{S}} \hspace{-0.25ex} - \displaystyle \onehalf \hspace{.2ex} \mathboldtau_{\hspace{-0.1ex}\Xcompanion} \hspace{-0.1ex} \times \bm{E} , \;\:
%
\bm{\mu} = \bm{\mu}^{\mathsf{S}} \hspace{-0.25ex} - \displaystyle \onehalf \hspace{.32ex} \bm{\mu}_{\Xcompanion} \hspace{-0.1ex} \times \bm{E} , \;\:
%
\mathboldae = \mathboldae^{\hspace{.12ex}\mathsf{S}} \hspace{-0.25ex} - \displaystyle \onehalf \hspace{.32ex} \mathboldae_{\Xcompanion} \hspace{-0.1ex} \times \bm{E} , \\[.5em]
%
\bm{\upgamma} = \mathboldepsilon - \displaystyle \onehalf \hspace{.32ex} \bm{\upgamma}_{\hspace{-0.16ex}\Xcompanion} \hspace{-0.1ex} \times \bm{E} , \:\:
%
\mathboldepsilon \equiv \bm{\upgamma}^{\hspace{.16ex}\mathsf{S}} \hspace{-0.4ex} = \hspace{-0.2ex} \boldnabla {\bm{u}}^{\hspace{0.1ex}\mathsf{S}} \hspace{-0.25ex} , \:\:
%
\bm{\upgamma}_{\hspace{-0.16ex}\Xcompanion} \hspace{-0.16ex}
= \boldnabla \hspace{-0.16ex} \times \hspace{-0.1ex} \bm{u} - 2 \hspace{.16ex} \bm{\theta} \hspace{.15ex}; \\[.5em]
%
\variation{\Pi} = \ldots
\end{array}
\end{equation}

...


Если формально устремить $h$ к~нулю, то исчезает вклад $\mathboldae$ в~$\Pi$, а~с~ним и моментные напряжения. При~отсутствии моментной нагрузки~$\bm{m}$ тензор~$\mathboldtau$ становится симметричным, и мы приходим к~классической модели. Использование неклассической моментной модели естественно в~тех случаях, когда в~реальном материале есть некий минимальный объём, ...

...

\en{\section{Compatibility equations}}

\ru{\section{Уравнения совместности}}

\label{para:compatibilityequations.cosseratcontinuum}

Из~выражений тензоров деформации~\eqref{deformationtensors:micropolarcontinuum} следует

...



\en{\section{Theorems of statics}}

\ru{\section{Теоремы статики}}

Теоремы статики линейных консервативных систем, легко выводимые при конечном числе степеней свободы

...



\en{\section{Cosserat pseudo-continuum}}

\ru{\section{Псевдоконтинуум Коссера}}

\label{para:caseoflatenttrihedron.smalldisplacementsandrotations}

Так называется упрощённая моментная модель\footnote{Братья Cosserat называли это cas de trièdre caché (случай скрытого трёхгранника, case of latent trihedron).}\hspace{-0.32em},\hspace{0.24em} в~которой повороты выражаются через перемещения как в~классической среде~\cite{nowacki-elasticity}:

\nopagebreak\vspace{-0.75em}\begin{equation}
\bm{\theta} = \displaystyle \onehalf \hspace{0.4ex} \boldnabla \hspace{-0.16ex} \times \hspace{-0.1ex} \bm{u} 
\:\Leftrightarrow\:
\bm{\upgamma}_{\hspace{-0.16ex}\Xcompanion} \hspace{-0.32ex} = \bm{0}
\:\Leftrightarrow\:
\bm{\upgamma} = \mathboldepsilon = \hspace{-0.12ex} \boldnabla {\bm{u}}^{\hspace{0.1ex}\mathsf{S}} \hspace{-0.25ex}.
\end{equation}

Равенство~${\bm{\upgamma}_{\hspace{-0.16ex}\Xcompanion} \hspace{-0.32ex} = \bm{0}}$~(симметрию $\bm{\upgamma}$) возможно понимать как внутреннюю связь~(\chapdotpararef{chapter:nonlinearcontinuum}{para:internalconstraints}). Аргумент~${\bm{\upgamma}_{\hspace{-0.16ex}\Xcompanion}}$ исчезает из~энергии~$\Pi$, соотношение упругости для~${\mathboldtau_{\hspace{-0.1ex}\Xcompanion}}$ не~может быть написано. Его место в~полной системе занимает уравнение связи.

В~классической теории упругости полная система сводится к~одному уравнению для~вектора~$\bm{u}$~(\chapdotpararef{chapter:linearclassicalelasticity}{para:equationsfordisplacement}). В~моментной теории

...



\en{\section{Plane deformation}}

\ru{\section{Плоская деформация}}

\label{para:planedeformation.cosseratcontinuum}

Все переменные в~этой постановке не~зависят от декартовой координаты~${z \equiv x_3}$~(орт оси~--- $\bm{k}$). Перемещения и~силы перпендикулярны оси~$z$, а~повороты и~моменты~--- параллельны ей:

...



\en{\section{Nonlinear theory}}

\ru{\section{Нелинейная теория}}

Кажущееся на~первый взгляд чрезвычайно трудным, построение теории конечных деформаций континуума Коссера становится прозрачным, если опираться на общую механику, тензорное исчисление и~нелинейную теорию классической безмоментной среды.

Как и в~\chapref{chapter:nonlinearcontinuum}, рассматривается среда из~частиц с~материальными координатами

...



\en{\section{Nonlinear model with constrained rotation}}

\ru{\section{Нелинейная модель со стеснённым вращением}}

\label{para:caseoflatenttrihedron.largedisplacementsandrotations}

Вспомним переход к~модели со~стеснённым вращением в~линейной теории~(\pararef{para:caseoflatenttrihedron.smalldisplacementsandrotations}). Разделились соотношения упругости для~симметричной части тензора силовых напряжений

...




\end{otherlanguage}

\section*{\small \wordforbibliography}

\begin{changemargin}{\parindent}{0pt}
\fontsize{10}{12}\selectfont

\begin{otherlanguage}{russian}

Все работы по~моментной теории упругости содержат ссылки на~книгу братьев Eugène et~François Cosserat~\cite{cosserat}, где~трёх\-мерной среде посвящена одна глава из~шести. Переведённая монография \hbox{W\hspace{-0.2ex}.\:Nowacki}~\cite{nowacki-elasticity} была одной из первых книг на~русском языке с~изложением линейной моментной теории. Ранее эта область представлялась статьями~--- например, R.\,D.\:Mindlin’а и~H.\,F.\:Tiersten’а~\cite{mindlin.tiersten}. Краткое изложение моментной теории, но с~подробным рассмотрением задач содержится в~книгах Н.\,Ф.\:Морозова~\cite{morozov-twodimensionalproblems, morozov-fractures}.

\end{otherlanguage}

\end{changemargin}
