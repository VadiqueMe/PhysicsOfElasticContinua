\en{\chapter{Micropolar three\hbox{-}dimensional continuum}}

\ru{\chapter{Микрополярная трёхмерная среда}}

\thispagestyle{empty}

\newcommand{\littlerotationvector}{\varvector{\bm{\varphi}}}

\newcommand{\motiongradient}{\bm{F}}

\newcommand{\distortiontensor}{\bm{\upgamma}}
\newcommand{\wrynesstensor}{\bm{\kappa}\hspace{-0.1ex}}

\newcommand{\forcestresstensor}{\mathboldtau}
\newcommand{\couplestresstensor}{\bm{\mu}}

\label{chapter:cosseratcontinuum}

\en{\section{Introduction to linear micropolar theory}}

\ru{\section{Введение в линейную микрополярную теорию}}

\label{para:introtolinearmicropolar}

%% micropolar elasticity, couple\hbox{-}stress elasticity, asymmetric (non\hbox{-}symmetric) elasticity, Cosserat elasticity

\dropcap{\en{T}\ru{Х}}{\en{he characteristic}\ru{арактерная}} \en{distinctive}\ru{отличительная} \en{feature}\ru{особенность} \en{of classical elastic media}\ru{классических упругих сред}~(\chapref{chapter:nonlinearcontinuum} \en{and}\ru{и}~\customref{chapter:linearclassicalelasticity})\:--- \en{they consist of}\ru{они состоят из} \inquotesx{\en{simple}\ru{простых} \en{material points}\ru{материальных точек}}[.]
\en{A~particle}\ru{Частица} \en{of a classical continuum}\ru{классического контину\kern-0.11exума} \en{has}\ru{имеет} \en{only}\ru{лишь} \ru{трансляционные степени свободы}\en{translational degrees of freedom}, \en{and }\ru{и~}\en{only single vector}\ru{только один вектор}~${\currentlocationvector(q^i \hspace{-0.15em}, t)}$ \en{determines}\ru{определяет} \en{its movement}\ru{её движение}.
\en{Therefore}\ru{Поэтому} \en{loads}\ru{нагрузки}~(\inquotes{\en{force factors}\ru{силовые факторы}}) \en{in such a~model}\ru{в~такой модели}\en{ are}\ru{\:---} \en{only forces}\ru{только силы}, \en{volume}\ru{объёмные} \en{and}\ru{и}~\en{surface ones}\ru{поверхностные}.
\en{No moments}\ru{Моментов нет}.

\en{But}\ru{Но} \en{it is not so hard}\ru{не~так~уж трудно} \en{to build more complex models}\ru{построить более сложные модели} \en{of a~continuous medium}\ru{сплошной среды}, \en{where}\ru{где} \en{the particles}\ru{частицы} \en{have}\ru{обладают} \en{not only}\ru{не~только~лишь} \en{degrees of translation freedom}\ru{степенями свободы трансляции}, \en{but also}\ru{но~и} \en{some additional ones}\ru{н\'{е}которыми дополнительными}.
\en{New degrees of freedom}\ru{Новые степени свободы} \en{are associated}\ru{связаны} \ru{и~}\en{with new loads}\ru{с~новыми нагрузками}.

\en{The most natural}\ru{Наиболее естественная} \en{of non-classical models}\ru{из~неклассических моделей} \en{of a~three-dimensional medium}\ru{трёхмерной среды} \en{was proposed}\ru{была предложена} \en{by }\ru{братьями }Cosserat\en{ brothers} \en{in}\ru{в}~1909~\cite{cosserat}.
\en{Every particle}\ru{Каждая частица} \en{of the }\ru{контину\kern-0.11exума }Cosserat\en{ continuum} \en{is}\ru{есть} \en{infinitesimal}\ru{бесконечно-малое} \en{absolutely rigid body}\ru{совершенно жёсткое тело} \en{with six degrees of freedom}\ru{с~шестью степенями свободы}\:--- \en{three translational and three rotational}\ru{тремя трансляционными и тремя вращательными}.
\en{Loads}\ru{Нагрузки} \en{in such a medium}\ru{в~такой среде}\en{ are}\ru{\:---} \en{forces and moments}\ru{силы и~моменты}.
\en{The work}\ru{Работа} \en{of the }\ru{братьев }Cosserat\en{ brothers} \en{remained unnoticed}\ru{оставалась незамеченной} \en{for half a~century}\ru{полвека}, \en{but then}\ru{но~затем} \en{interest in this topic arose}\ru{возник интерес к~этой теме}~\cite{mindlin.tiersten, nowacki-elasticity}.

%% \begin{comment} %%
\vspace{2mm}

\begin{tcolorbox}[breakable, enhanced, colback = orange!8, before upper={\parindent3.2ex}, parbox = false]
\small%
\setlength{\abovedisplayskip}{2pt}\setlength{\belowdisplayskip}{2pt}%

\noindent \textit{from \textboldoblique{Nowacki~W.} The Linear Theory of~Micropolar Elasticity.} In: \textit{Micropolar Elasticity. International Centre for~Mechanical Sciences (Courses and~Lectures), vol.\:151, 1974, pp.\:1\hbox{--}43}
\\[.555em]
%
\indent Woldemar Voigt tried to remove the shortcomings of the classical theory of elasticity [\textit{\textboldoblique{W.\:Voigt}. Theoretische Studien über die~Elasticitätsverhältnisse der~Krystalle. Abhandlungen der~Königlichen Gesellschaft der~Wissenschaften in~Göttingen, 34:\:3\hbox{--}51, 1887}] by the assumption that the interaction of two parts of the body is transmitted through an area element $do$ by means not only of the force vector $\bm{p}do$ but also by the moment vector $\bm{m}do$. Thus, besides the force stresses~$\sigma_{ji}$ also the moment stresses have been defined.

However, the complete theory of asymmetric elasticity was developed by the brothers \textboldoblique{François et~Eugène Cosserat} who published it in 1909 in the work \textit{\inquotes{Théorie des corps déformables}}.

\en{They assumed that}
\ru{Они предположили, что}
\en{the bodies consist}\ru{тела состоят}
\en{of the interconnected particles}\ru{из взаимосвязанных частиц}.
\en{These particles}\ru{Эти частицы}
\en{are like a~small rigid bodies}\ru{похожи на маленькие жёсткие тела}.

During the deformation each particle is displaced by~$\bm{u}(\bm{x},t)$ and rotated by~$\bm{\varphi}(\bm{x},t)$,
the functions of the position~$\bm{x}$ and time~$t$.

Thus an elastic continuum has been described such that its points possess the orientation (polar media) and for which we can speak of the rotation of a point.
\en{the vectors}~$\bm{u}$ \en{and}\ru{и}~$\bm{\varphi}$
\en{determine the strain (the deformation)}
\en{of the body}\ru{тела}.

\en{These vectors are}\ru{Эти вектора\:---}
\en{mutually independent}\ru{взаимно независимые}.
.




The introduction of~$\bm{u}$ \en{and}\ru{и}~$\bm{\varphi}$ and the assumption that the transmission of forces through an area element $do$ is carried out by means of the force vector $\bm{p}$ and the moment vector $\bm{m}$ leads in the consequence to asymmetric stress tensors~$\sigma_{\hspace{-0.25ex}qp}$ and~$\mu_{qp}$.

The theory of the brothers E.~and~F.\;Cosserat remained unnoticed and was not duly appreciated during their lifetime.
\en{It was so}\ru{Это было так}\ru{,}
\en{because}\ru{потому что}
\en{the presentation}\ru{презентация}
\en{was too general}\ru{была слишком общей}
( \en{the theory was non-linear}\ru{теория была нелинейной},
\en{including large deformations}\ru{включавшая конечные деформации} ).
Because the frames of their theory exceeded the frames of the theory of classical linear elasticity.
They attempted to construct the unified field theory, containing mechanics, optics and electrodynamics and combined by a~principle of the least action.

The research in the field of the general theories of continuous media conducted in the last fifteen years,
drew the attention of the scientists to the Cosserats’ work.
Looking for the new models,
describing the behaviour of the real elastic media more precisely,
the models similar to, or identical with that of Cosserats’
have been encountered.
Here, we mention, first of all,
the papers by C.\:Truesdell and R.\,A.\;Toupin [\textit{ \textboldoblique{C.\:Truesdell} and \textboldoblique{R.\,A.\;Toupin}.
The classical field theories.
Encyclopædia of Physics, Chapter 1, Springer\hbox{-}Verlag, Berlin, 1960 }],
%
G.\:Grioli [\textit{\textboldoblique{Grioli~G.} Elasticité asymmetrique. Ann. di Mat. Pura et Appl. Ser. IV, 50 (1960)}],
R.\,D.\;Mindlin and H.\,F.\;Tiersten
[\textit{\textboldoblique{Mindlin, R.\,D.};
\textboldoblique{Tiersten, H.\,F.}
Effects of couple-stresses in linear elasticity. Arch. Rational Mech. Anal. 11. 1962. 415\hbox{--}448}].
\par
\end{tcolorbox}

\vspace{2mm}
%% \end{comment} %%

\en{In the~truly micropolar continuum}\ru{В~истинно микрополярном контину\kern-0.11exуме}\en{,}
\en{the vector field}\ru{векторное поле}
\en{of~displacements}\ru{перемещений}
${\fieldofdisplacements(\locationvector, t)}$
\en{and}\ru{и}
\en{the field of rotations}\ru{поле поворотов}
${\fieldofrotations(\locationvector, t)}$
\en{are mutually independent}\ru{взаимно независимы}.
\en{It is also called}\ru{Это также называется}
\en{the~model with the free rotation}\ru{моделью со~свободным вращением}.
\en{At first}\ru{Сначала}
\en{I’m going}\ru{я собираюсь}
\en{to review it}\ru{рассмотреть её}.

\en{Also it is}\ru{Также это}
\en{the~geometrically}\ru{геометрически}
\en{linear model}\ru{линейная модель},
\en{that’s}\ru{то~есть}
\en{the~case}\ru{случай}
\en{of the very small}\ru{очень м\'{а}лых},
\en{the infinitesimal}\ru{бесконечно м\'{а}лых}
\en{displacements}\ru{перемещений}
\en{and}\ru{и}
\en{the infinitesimall}\ru{бесконечно м\'{а}лых}
\en{rotations}\ru{поворотов}.
\en{Here}\ru{Здесь}
\en{operators}\ru{операторы}
${\hspace{-0.2ex} \boldnablacircled \hspace{.1ex}}$
\en{and}\ru{и}~${\hspace{-0.2ex} \boldnabla \hspace{.1ex}}$
\en{are indistinguishable}\ru{неразличимы}.
${\mathcal{V} \hspace{-0.2ex} = \hspace{-0.2ex} \mathcircabove{\mathcal{V}}\hspace{-0.1ex}}$,
${\rho \hspace{-0.12ex} = \hspace{-0.22ex} \mathcircabove{\rho}\hspace{.2ex}}$
\en{and}\ru{и}
\en{Therefore}\ru{Поэтому}
\en{equations}\ru{уравнения}
\en{can be written}\ru{могут быть написаны}
\inquotes{\en{in the~initial configuration}\ru{в~начальной конфигурации}}.
\en{Also}\ru{Также}
\en{operators}\ru{операторы}~$\variation$
\en{and}\ru{и}~${\hspace{-0.2ex}\boldnabla}$
\en{commute}\ru{коммутируют}
(${ \variation{\boldnabla \fieldofdisplacements}
\hspace{-0.1ex} = \hspace{-0.3ex}
\boldnabla \variation{\fieldofdisplacements} }$,
${ \variation{\boldnabla \hspace{-0.12ex} \fieldofrotations} \hspace{-0.1ex} = \hspace{-0.3ex}
\boldnabla \variation{\fieldofrotations} }$).

\en{To build this model}\ru{Чтобы построить эту модель},
\en{I use the principle of the virtual work}\ru{я использую принцип виртуальной работы}.
\en{This principle says that}\ru{Этот принцип говорит, что}
\en{the variation of work of the real external forces on the virtual displacements}\ru{вариация работы реальных внешних сил на виртуальных перемещениях}
\en{is equal with the opposite sign}\ru{равна с~обратным знаком}
\en{to the variation of work of the internal forces}\ru{вариации работы внутренних сил},
\en{the real stresses on virtual deformations}\ru{реальных напряжений на~виртуальных деформациях}

\nopagebreak\vspace{-0.1em}\begin{equation*}
\integral\displaylimits_{\mathcal{V}} \hspace{-0.64ex}
\left(^{\mathstrut}
   \bm{f} \dotp \variation{\fieldofdisplacements}
   + \bm{m} \dotp \variation{\fieldofrotations}
\right) \hspace{-0.4ex} d\mathcal{V}
\hspace{.25ex} +
\integral\displaylimits_o \hspace{-0.64ex} \left(^{\mathstrut}
   \bm{p} \dotp \variation{\fieldofdisplacements}
   + \mathboldM \hspace{-0.1ex} \dotp \variation{\fieldofrotations} \right) \hspace{-0.4ex} do \hspace{.12ex}
= - \hspace{-0.5ex}
\integral\displaylimits_{\mathcal{V}} \hspace{-0.64ex}
\variation{\internalwork} \hspace{.1ex} d\mathcal{V}
.
\end{equation*}

\vspace{-0.1em}\noindent
\en{Here}\ru{Здесь}
$\bm{f}$ \en{and}\ru{и}~$\bm{m}$
\en{are}\ru{это}
\en{the external forces}\ru{внешние силы}
\en{and}\ru{и}~\en{moments}\ru{моменты}
\en{per}
\inquotes{\en{one volume}\ru{на~единицу объёма}},
$\bm{p}$ \en{and}\ru{и}~$\mathboldM$
\en{are external forces too}\ru{внешние силы тоже},
\en{but}\ru{но}
\en{per surface unit}\ru{на~единицу поверхности}
(\en{the surface loads}\ru{поверхностные нагрузки},
\en{they act}\ru{они действуют}
\en{only on a~certain part of}\ru{лишь на~н\'{е}которой части}~$o$
\en{on the boundary surface}\ru{на граничной поверхности}.


${\variation{\internalwork}\hspace{-0.2ex}}$ \en{is}\ru{есть}
\en{the work of internal forces density}\ru{работа внутренних сил}
\en{per volume unit}\ru{на~единицу объёма}

\en{As before}\ru{По\hbox{-}прежнему},
\en{we suppose that}\ru{мы полагаем, что}
${\variation{\internalwork}\hspace{-0.2ex}}$
\en{nullifies}\ru{обнуляется}\ru{,}
\en{when the~body moves}\ru{когда тело движется}
\en{as a~rigid whole without deformation}\ru{как жёсткое целое без деформации}:

\nopagebreak\vspace{-0.1em}
\begin{equation*}
\begin{array}{c}
\variation{\fieldofdisplacements} = \variation{\fieldofrotations} \hspace{-0.2ex} \times \hspace{-0.12ex} \locationvector + \boldconstant \hspace{.1ex} ,
\:\,
\variation{\fieldofrotations} = \boldconstant
\hspace{.64ex}\Rightarrow\hspace{.5ex}
\variation{\internalwork} \hspace{-0.2ex} = 0 \hspace{.1ex},
\\[.25em]
%
\boldnabla \variation{\fieldofdisplacements}
= \hspace{-0.25ex} \boldnabla \variation{\fieldofrotations} %%{^2\bm{0}}
\hspace{-0.2ex} \times \hspace{-0.2ex} \locationvector
- \hspace{-0.25ex} \boldnabla \locationvector \hspace{-0.2ex} \times \hspace{-0.2ex} \variation{\fieldofrotations}
= - \UnitDyad \hspace{-0.2ex} \times \hspace{-0.2ex} \variation{\fieldofrotations}
= - \hspace{.2ex} \variation{\fieldofrotations} \hspace{-0.2ex} \times \hspace{-0.2ex} \UnitDyad
\hspace{.1ex} ,
\:\,
\boldnabla \variation{\fieldofrotations} = {^2\bm{0}} \hspace{.1ex} .
\end{array}
\end{equation*}

\vspace{-0.1em}
\en{Let’s 





Introducing the strain tensors (or the deformation tensors)}\ru{Вводя тензоры деформации}\:---
( \en{the tensor of}\ru{тензор}
\en{of the relative displacement between particles}\ru )

the relative displacement between particles is the distortion
\distortiontensor
относительное смещение между частицами это дисторция


}\ru{тензор относительного смещения между частицами}
( tensor~$\$)
\en{and}\ru{и}~\en{the curvature-twist tensor}\ru{тензор искривления-скручивания}
(\ru{the curvature-twist tensor}\ru{,} the torsion-flexure tensor, or the wryness tensor)~$\wrynesstensor$

\nopagebreak\vspace{-0.1em}
\begin{equation}\label{deformationtensors:micropolarcontinuum}
\begin{array}{rl}
\distortiontensor \hspace{.16ex} \equiv \boldnabla \fieldofdisplacements \hspace{.1ex} + \fieldofrotations \hspace{-0.12ex} \times \hspace{-0.2ex} \UnitDyad \hspace{.1ex} , &
\wrynesstensor \hspace{.16ex} \equiv \boldnabla \fieldofrotations
\hspace{.1ex} ,
\end{array}
\end{equation}
%
\nopagebreak\vspace{-0.85em}
\begin{equation*}
\begin{array}{rl}
\distortiontensor_{\hspace{-0.25ex}\Xcompanion} \hspace{-0.16ex}
= \boldnabla \hspace{-0.16ex} \times \hspace{-0.1ex} \fieldofdisplacements \hspace{.12ex} - \hspace{.12ex} 2 \hspace{.16ex} \fieldofrotations \hspace{.1ex}
, &
\wrynesstensor_{\hspace{.1ex}\Xcompanion} \hspace{-0.16ex}
= \boldnabla \hspace{-0.16ex} \times \hspace{-0.1ex} \fieldofrotations \hspace{.1ex} ,
\\[.3em]
%
\variation{\distortiontensor}
= \hspace{-0.16ex} \boldnabla \variation{\fieldofdisplacements} \hspace{.1ex}
+ \variation{\fieldofrotations} \hspace{-0.2ex} \times \hspace{-0.2ex} \UnitDyad \hspace{.1ex}
, &
\variation{\wrynesstensor} = \hspace{-0.16ex} \boldnabla \variation{\fieldofrotations} \hspace{.1ex} ,
\end{array}
\end{equation*}

\vspace{-0.6em}\noindent
\en{with }\ru{с~}\en{the~needed absence}\ru{нужным отсутствием} \en{of virtual deformations}\ru{виртуальных деформаций}
${\variation{\distortiontensor} = \hspace{-0.1ex} {^2\bm{0}}}$ \en{and }\ru{и~}${\variation{\wrynesstensor} = \hspace{-0.1ex} {^2\bm{0}}}$.

\en{In}\ru{В}~\chapdotpararef{chapter:nonlinearcontinuum}{para:stressesAsLagrangeMultipliers} \en{for the~momentless continuum}\ru{для безмоментной среды}, \en{stresses appear as Lagrange multipliers}\ru{напряжения появляются как множители Lagrange’а} \en{in the~principle of~virtual work}\ru{в~принципе виртуальной работы}, \textcolor{magenta}{\en{when}\ru{когда}}~${\mathcolor{magenta}{\variation{\internalwork} \hspace{-0.2ex} = 0}}$.
\en{The~same here}\ru{Так~же и~тут}:

\nopagebreak\vspace{-0.3em}
\begin{multline}\label{virtualworkprinciple.1:micropolarcontinuum}
\integral\displaylimits_{\mathcal{V}} \hspace{-0.64ex} \left(^{\mathstrut} \bm{f} \dotp \variation{\fieldofdisplacements}
+ \bm{m} \dotp \variation{\fieldofrotations}
- \forcestresstensor \dotdotp \variation{\distortiontensor}^{\T} \hspace{-0.4ex}
- \couplestresstensor \dotdotp \variation{\wrynesstensor}^{\T} \hspace{-0.05ex} \right) \hspace{-0.32ex} d\mathcal{V} \hspace{.5ex} + \\[-1.1em]
%
+ \integral\displaylimits_o \hspace{-0.64ex} \left(^{\mathstrut} \bm{p} \dotp \variation{\fieldofdisplacements}
+ \mathboldM \hspace{-0.1ex} \dotp \variation{\fieldofrotations} \right) \hspace{-0.32ex} do \hspace{.12ex} = \hspace{.1ex} 0
\hspace{.1ex} .
\end{multline}

\begin{otherlanguage}{russian}

\vspace{-0.2em} \noindent Множители Лагранжа в~каждой точке\:--- это несимметричные тензоры второй сложности~$\forcestresstensor$ и~$\couplestresstensor$.

Преобразуем ${- \hspace{.2ex} \forcestresstensor \dotdotp \variation{\distortiontensor}^{\T}}$ и~${- \hspace{.2ex} \couplestresstensor \dotdotp \variation{\wrynesstensor}^{\T}}$

\nopagebreak\begin{equation*}
\begin{array}{rl}
\variation{\distortiontensor}^{\T} \hspace{-0.32ex}
= \hspace{-0.16ex} \boldnabla \variation{\fieldofdisplacements}^{\hspace{-0.05ex}\T} \hspace{-0.3ex}
- \hspace{.1ex} \variation{\fieldofrotations} \hspace{-0.2ex} \times \hspace{-0.2ex} \UnitDyad \hspace{.1ex}
, &
\variation{\wrynesstensor}^{\T} \hspace{-0.32ex} = \hspace{-0.16ex} \boldnabla \variation{\fieldofrotations}^{\T} \hspace{-0.25ex} ,
\\[.25em]
%
- \hspace{.2ex} \forcestresstensor \dotdotp \variation{\distortiontensor}^{\T} \hspace{-0.32ex}
= - \hspace{.2ex} \forcestresstensor \dotdotp \hspace{-0.16ex} \boldnabla \variation{\fieldofdisplacements}^{\hspace{-0.05ex}\T} \hspace{-0.3ex}
+ \forcestresstensor \dotdotp \hspace{-0.32ex} \left( \hspace{.12ex} \variation{\fieldofrotations} \hspace{-0.2ex} \times \hspace{-0.2ex} \UnitDyad \hspace{.12ex} \right) \hspace{-0.25ex}
, &
- \hspace{.2ex} \couplestresstensor \dotdotp \variation{\wrynesstensor}^{\T} \hspace{-0.32ex}
= - \hspace{.2ex} \couplestresstensor \dotdotp \hspace{-0.16ex} \boldnabla \variation{\fieldofrotations}^{\T} \hspace{-0.25ex} .
\end{array}
\end{equation*}

\noindent
\en{Using}\ru{Используя}

\nopagebreak\vspace{-1em}\begin{equation*}
\eqrefwithchapdotpara{pseudovectorinvariant}{chapter:mathapparatus}{para:tensors.symmetric+skewsymmetric}
\:\Rightarrow\,
\bm{A}_{\Xcompanion} \hspace{-0.1ex} = - \hspace{.1ex} \bm{A} \dotdotp \permutationsparitytensor
\hspace{.1ex} ,
\end{equation*}

\nopagebreak\vspace{-0.5em}\begin{multline*}
\bm{A} \dotdotp \hspace{-0.25ex} \left( \hspace{.1ex} \bm{b} \times \hspace{-0.2ex} \UnitDyad \hspace{.1ex} \right)
= \bm{A} \dotdotp \hspace{-0.32ex} \left( \hspace{.1ex} \UnitDyad \hspace{-0.12ex} \times \bm{b} \hspace{.1ex} \right)
= \bm{A} \dotdotp \hspace{-0.32ex} \left( \hspace{-0.1ex} - \hspace{.2ex} \permutationsparitytensor \dotp \bm{b} \hspace{.1ex} \right) =
\\[-0.2em]
%
\hspace*{\fill} = \left( \hspace{-0.1ex} - \hspace{.1ex} \bm{A} \dotdotp \permutationsparitytensor \hspace{.1ex} \right) \hspace{-0.2ex} \dotp \hspace{.1ex} \bm{b}
\hspace{.12ex} = \bm{A}_{\Xcompanion} \hspace{-0.1ex} \dotp \hspace{.1ex} \bm{b}
\hspace{.64ex} \Rightarrow
\end{multline*}

\nopagebreak\vspace{-0.4em}\begin{equation*}
\Rightarrow \hspace{.64ex}
\forcestresstensor \dotdotp \hspace{-0.25ex} \left( \hspace{.12ex} \variation{\fieldofrotations} \hspace{-0.2ex} \times \hspace{-0.2ex} \UnitDyad \hspace{.12ex} \right)
= \forcestresstensor_{\Xcompanion} \hspace{-0.1ex} \dotp \hspace{.1ex} \variation{\fieldofrotations}
\end{equation*}

\vspace{-0.8em}\noindent
\en{and}\ru{и} \inquotes{product rule}

\nopagebreak\vspace{-0.2em}\begin{equation*}\begin{array}{c}
\boldnabla \dotp \left( \hspace{.1ex} \forcestresstensor \dotp \variation{\fieldofdisplacements} \right)
= \left( \hspace{.12ex} \boldnabla \dotp \forcestresstensor \hspace{.12ex} \right) \hspace{-0.16ex} \dotp \variation{\fieldofdisplacements}
+ \forcestresstensor \dotdotp \hspace{-0.16ex} \boldnabla \variation{\fieldofdisplacements}^{\hspace{-0.05ex}\T} \hspace{-0.3ex} ,
\\[.25em]
%
\boldnabla \dotp \left( \hspace{.1ex} \couplestresstensor \dotp \variation{\fieldofrotations} \right)
= \left( \hspace{.12ex} \boldnabla \dotp \couplestresstensor \hspace{.12ex} \right) \hspace{-0.16ex} \dotp \variation{\fieldofrotations}
+ \couplestresstensor \dotdotp \hspace{-0.16ex} \boldnabla \variation{\fieldofrotations}^{\hspace{-0.05ex}\T} \hspace{-0.3ex} ,
\end{array}\end{equation*}

\vspace{-0.4em}\noindent
получаем

\nopagebreak\vspace{-0.4em}\begin{equation*}
\begin{array}{c}
- \hspace{.2ex} \forcestresstensor \dotdotp \variation{\distortiontensor}^{\T} \hspace{-0.32ex}
= \left( \hspace{.12ex} \boldnabla \dotp \forcestresstensor \hspace{.12ex} \right) \hspace{-0.16ex} \dotp \variation{\fieldofdisplacements}
- \hspace{-0.15ex} \boldnabla \dotp \left( \hspace{.1ex} \forcestresstensor \dotp \variation{\fieldofdisplacements} \right) \hspace{.1ex}
+ \hspace{.1ex} \forcestresstensor_{\Xcompanion} \hspace{-0.1ex} \dotp \hspace{.1ex} \variation{\fieldofrotations} \hspace{.1ex} ,
\\[.25em]
%
- \hspace{.2ex} \couplestresstensor \dotdotp \variation{\wrynesstensor}^{\T} \hspace{-0.32ex}
= \left( \hspace{.12ex} \boldnabla \dotp \couplestresstensor \hspace{.12ex} \right) \hspace{-0.16ex} \dotp \variation{\fieldofrotations}
- \hspace{-0.15ex} \boldnabla \dotp \left( \hspace{.1ex} \couplestresstensor \dotp \variation{\fieldofrotations} \right) \hspace{-0.25ex} .
\end{array}
\end{equation*}

\vspace{-0.25em}\noindent
После интегрирования с~применением теоремы о~дивергенции\footnote{${\bm{a} \dotp ({^2\!\bm{B}} \dotp \bm{c} \hspace{.1ex}) = (\bm{a} \dotp \hspace{-0.12ex} {^2\!\bm{B}}) \hspace{-0.1ex} \dotp \bm{c} = \bm{a} \dotp \hspace{-0.12ex} {^2\!\bm{B}} \dotp \bm{c}}$}

\nopagebreak\vspace{-0.1em}\begin{equation*}
\displaystyle
\integral\displaylimits_{\mathcal{V}} \hspace{-0.5ex} \boldnabla \dotp \left( \forcestresstensor \dotp \variation{\fieldofdisplacements} \right) \hspace{-0.1ex} d\mathcal{V}
=
\ointegral\displaylimits_{\mathclap{\widearc{o}\hspace{.1ex}(\boundary \mathcal{V})}} \hspace{-0.1ex} \bm{n} \dotp \forcestresstensor \dotp \variation{\fieldofdisplacements} \hspace{.32ex} do
\hspace{.1ex} ,
\:\:
%
\displaystyle
\integral\displaylimits_{\mathcal{V}} \hspace{-0.5ex} \boldnabla \dotp \left( \hspace{.1ex} \couplestresstensor \dotp \variation{\fieldofrotations} \right) \hspace{-0.1ex} d\mathcal{V}
=
\ointegral\displaylimits_{\mathclap{\widearc{o}\hspace{.1ex}(\boundary \mathcal{V})}} \hspace{-0.1ex} \bm{n} \dotp \couplestresstensor \dotp \variation{\fieldofrotations} \hspace{.32ex} do
%%\hspace{.1ex} ,
\end{equation*}

\vspace{-0.25em}\noindent
\eqref{virtualworkprinciple.1:micropolarcontinuum} приобретает вид

\nopagebreak\vspace{-0.33em}
\begin{multline*}\label{virtualworkprinciple.2:micropolarcontinuum}
\integral\displaylimits_{\mathcal{V}} \hspace{-0.64ex} \left(^{\mathstrut} ( \hspace{.12ex} \boldnabla \hspace{-0.1ex} \dotp \forcestresstensor + \bm{f} \hspace{.12ex} ) \dotp \variation{\fieldofdisplacements} \hspace{.1ex}
+ ( \hspace{.12ex} \boldnabla \dotp \couplestresstensor \hspace{.1ex} + \forcestresstensor_{\Xcompanion} \hspace{-0.1ex} + \bm{m} \hspace{.1ex} ) \dotp \variation{\fieldofrotations} \right) \hspace{-0.32ex} d\mathcal{V} \hspace{0.5ex} + \\[-1.1em]
%
+ \integral\displaylimits_o \hspace{-0.64ex} \left(^{\mathstrut} ( \hspace{.32ex} \bm{p} - \bm{n} \dotp \forcestresstensor \hspace{.2ex} ) \dotp \variation{\fieldofdisplacements} \hspace{.1ex}
+ ( \hspace{.1ex} \mathboldM - \bm{n} \dotp \couplestresstensor \hspace{.1ex} ) \dotp \variation{\fieldofrotations} \right) \hspace{-0.32ex} do \hspace{.12ex} = \hspace{.1ex} 0 \hspace{.1ex}.
\end{multline*}

\en{From randomness}\ru{Из случайности} \en{of~variations}\ru{вариаций}~$\variation{\fieldofdisplacements}$ \en{and}\ru{и}~$\variation{\fieldofrotations}$ \en{inside a~volume}\ru{в~объёме} \en{ensues}\ru{вытекает} \en{the~balance}\ru{баланс} \en{of~forces}\ru{сил} \en{and}\ru{и}~\en{moments}\ru{моментов}

\nopagebreak\vspace{-0.15em}\begin{equation}\label{micropolar.equilibrium:cauchy-like}
\boldnabla \hspace{-0.1ex} \dotp \forcestresstensor + \bm{f} \hspace{.12ex} = \hspace{.1ex} \bm{0}
\hspace{.1ex} ,
\:\;
\boldnabla \dotp \couplestresstensor \hspace{.1ex} + \forcestresstensor_{\Xcompanion} \hspace{-0.1ex} + \bm{m} \hspace{.1ex} = \hspace{.1ex} \bm{0}
\hspace{.1ex} ,
\end{equation}

\vspace{-0.2em}\noindent
\en{and}\ru{а} \en{from randomness}\ru{из случайности} \en{on a~surface}\ru{на~поверхности}\:--- \en{boundary conditions}\ru{краевые условия}

\nopagebreak\vspace{-0.15em}\begin{equation}\label{micropolar.equilibrium:boundaryconditions}
\bm{n} \dotp \forcestresstensor = \bm{p} \hspace{.1ex} ,
\:\;
\bm{n} \dotp \couplestresstensor = \mathboldM \hspace{.1ex} .
\end{equation}

\vspace{-0.1em}
\en{Force stress tensor}\ru{Тензор силового напряжения}~$\forcestresstensor$ удовлетворяет тем~же дифференциальным \inquotes{уравнениям равновесия}\footnote{%
Кавычки здесь оттого, что \emph{\loosetexttr[33]{уравнения равновесия}} это вообще всё, что вытекает из принципа виртуальной работы в~статике.%
}\hspace{-0.25ex} и~краевым условиям, что и в~безмоментной среде. \en{But tensor}\ru{Но~тензор}~$\forcestresstensor$ \en{is asymmetric}\ru{несимметричен}: \en{instead of}\ru{вместо} ${\forcestresstensor_{\Xcompanion} \hspace{-0.2ex} = \bm{0}}$ \en{here is}\ru{тут} ${\hspace{-0.1ex} \boldnabla \dotp \couplestresstensor \hspace{.1ex} + \forcestresstensor_{\Xcompanion} \hspace{-0.1ex} + \bm{m} \hspace{.1ex} = \bm{0}}$\:--- \ru{появляются }\en{couple stresses}\ru{моментные напряжения}~$\couplestresstensor$\en{ appear}, \en{and}\ru{и} \en{volume}\ru{объёмная} \en{moment load}\ru{моментная нагрузка}~$\bm{m}$ \ru{не~нулевая}\en{is non-zero}.

\end{otherlanguage}

\en{Meaning of~components}\ru{Смысл компонент} \en{of~the~couple stress tensor}\ru{тензора моментного напряжения}~$\couplestresstensor$ \en{is revealed similarly as for}\ru{раскрывается так~же, как и для}~$\forcestresstensor$.
\en{For an~orthonormal basis}\ru{Для~ортонормального базиса}, \en{moment}\ru{момент} ${\bm{M}_i \hspace{-0.1ex} = \bm{e}_i \hspace{-0.1ex} \dotp \couplestresstensor = \mu_{ik} \bm{e}_k}$ \en{acts}\ru{действует} \en{on an~area}\ru{на~площ\'{а}дке} \en{with normal}\ru{c~нормалью}~$\bm{e}_i$.
\en{Diagonal components}\ru{Диагональные компоненты}~$\mu_{11}$, $\mu_{22}$, $\mu_{33}$ \en{are twisting moments}\ru{это крутящие моменты}, \en{nondiagonal are bending ones}\ru{недиагональные\:--- изгибающие} (?? \figurename ??).

...

\en{eccentricity vector}\ru{вектор эксцентриситета}~$\mathboldoe$ \en{and}\ru{и} \en{inertia tensor}\ru{тензор инерции}~${\inertiatensor}$

\en{For an~isotropic medium}\ru{Для изотропной среды} ${\mathboldoe = \bm{0}}$, ${\inertiatensor = \inertiascalar \UnitDyad}$.

...



\en{\section{Relations of elasticity}}

\ru{\section{Отношения упругости}}

\begin{otherlanguage}{russian}

В~этой книге упругой называется среда с~потенциальными внутренними силами:
${\variation{\internalwork} = - \hspace{.16ex} \variation{\potential}}$,
где~$\potential$\:--- энергия деформации на~единицу объёма
(\en{continuing to~model}\ru{продолжая моделировать} \en{the~geometrically linear material}\ru{геометрически линейный материал}, ${\mathcal{V} \hspace{-0.2ex} = \hspace{-0.2ex} \mathcircabove{\mathcal{V}}\hspace{.12ex}}$).

Имея соотношения (...)

...

\begin{multline}\label{micropolar.constitutiveequations}
\hspace*{4em}
\variation{\potential} = - \hspace{.25ex} \variation{\internalwork} \hspace{-0.15ex}
= \forcestresstensor \dotdotp \variation{\distortiontensor}^{\T} \hspace{-0.25ex}
+ \hspace{.1ex} \couplestresstensor \dotdotp \variation{\wrynesstensor}^{\T}
\hspace{.6ex} \Rightarrow
\\[-0.1em]
%
\Rightarrow \hspace{.88ex}
\forcestresstensor = \scalebox{0.92}[0.92]{$ \displaystyle \frac{\raisemath{-0.125em}{\partial\hspace{.1ex} \potential^{\mathstrut}}}{\partial \distortiontensor}$}
\hspace{.15ex} ,
\hspace{.8ex}
\couplestresstensor = \scalebox{0.92}[0.92]{$ \displaystyle \frac{\raisemath{-0.125em}{\partial\hspace{.1ex} \potential^{\mathstrut}}}{\partial \wrynesstensor}$}
\hspace{.2ex} .
\hspace*{3.33em}
\end{multline}

Последние равенства\:--- соотношения упругости (определяющие уравнения, constitutive equations).

Разлагая тензоры деформации и~напряжения на~симметричные и~антисимметричные части

\nopagebreak\vspace{-0.1em}\begin{equation*}
\begin{array}{c}
\distortiontensor = \distortiontensor^{\mathsf{S}} \hspace{-0.1ex} - \hspace{.25ex} \smalldisplaystyleonehalf \hspace{.32ex} \distortiontensor_{\hspace{-0.25ex}\Xcompanion} \hspace{-0.1ex} \times \UnitDyad
\hspace{.1ex} ,
\:\:
%
\wrynesstensor \hspace{.1ex} = \wrynesstensor^{\hspace{.12ex}\mathsf{S}} \hspace{-0.1ex} - \hspace{.25ex} \smalldisplaystyleonehalf \hspace{.32ex} \wrynesstensor_{\hspace{.15ex}\Xcompanion} \hspace{-0.1ex} \times \UnitDyad
\hspace{.1ex} ,
\\[.6em]
%
\variation{\distortiontensor}^{\T} \hspace{-0.33ex} = \hspace{.1ex} \variation{\distortiontensor}^{\hspace{.2ex}\mathsf{S}} \hspace{-0.1ex} + \hspace{.2ex} \smalldisplaystyleonehalf \hspace{.32ex} \variation{\distortiontensor}_{\hspace{-0.25ex}\Xcompanion} \hspace{-0.1ex} \times \UnitDyad
\hspace{.1ex} ,
\:\:
%
\variation{\wrynesstensor}^{\T} \hspace{-0.33ex} = \hspace{.1ex} \variation{\wrynesstensor}^{\hspace{.12ex}\mathsf{S}} \hspace{-0.1ex} + \hspace{.2ex} \smalldisplaystyleonehalf \hspace{.32ex} \variation{\wrynesstensor}_{\hspace{.15ex}\Xcompanion} \hspace{-0.1ex} \times \UnitDyad
\hspace{.1ex} ,
\\[.6em]
%
\forcestresstensor = \forcestresstensor^{\hspace{.32ex}\mathsf{S}} \hspace{-0.1ex} - \hspace{.25ex} \smalldisplaystyleonehalf \hspace{.2ex} \forcestresstensor_{\hspace{-0.1ex}\Xcompanion} \hspace{-0.1ex} \times \UnitDyad
\hspace{.1ex} ,
\;\:
%
\couplestresstensor = \couplestresstensor^{\mathsf{S}} \hspace{-0.1ex} - \hspace{.25ex} \smalldisplaystyleonehalf \hspace{.32ex} \couplestresstensor_{\Xcompanion} \hspace{-0.1ex} \times \UnitDyad
\hspace{.1ex} ,
\end{array}
\end{equation*}

\noindent
преобразуем выражение
${\variation{\potential} = \forcestresstensor \dotdotp \variation{\distortiontensor}^{\T} \hspace{-0.3ex}
+ \couplestresstensor \dotdotp \variation{\wrynesstensor}^{\T} \hspace{-0.25ex}}$
как

\nopagebreak\vspace{-0.1em}\begin{equation}
\variation{\potential} = \ldots
\end{equation}

...

\begin{equation*}
\distortiontensor_{\hspace{-0.25ex}\Xcompanion} \hspace{-0.16ex}
= \hspace{-0.1ex} \boldnabla \hspace{-0.16ex} \times \hspace{-0.1ex} \fieldofdisplacements \hspace{.12ex} - \hspace{.12ex} 2 \hspace{.16ex} \fieldofrotations
\hspace{.1ex} ,
\end{equation*}

\begin{equation*}
\wrynesstensor_{\hspace{.15ex}\Xcompanion} \hspace{-0.16ex}
= \boldnabla \hspace{-0.16ex} \times \hspace{-0.1ex} \fieldofrotations
\hspace{.1ex} ,
\end{equation*}

\begin{equation*}
\infinitesimaldeformation \equiv \distortiontensor^{\hspace{.2ex}\mathsf{S}} \hspace{-0.32ex} = \hspace{-0.2ex} \boldnabla {\fieldofdisplacements}^{\hspace{.1ex}\mathsf{S}} \hspace{-0.25ex} , \:\:
\end{equation*}

...

{\small%
The classical isotropic linear elastic material behavior is described by two material parameters, for example, the Young’s modulus and the Poisson’s ratio, while the isotropic Cosserat continuum needs six material parameters

even when assumed to be linear, homogeneous and isotropic, it requires six independent material constants, in contrast to only two such constants for the classical continuum
\par}

...

Соотношения~\eqref{micropolar.constitutiveequations} обращаются преобразованием Лежандра

\nopagebreak\vspace{-0.2em}\begin{equation}
\begin{array}{c}
\distortiontensor = \scalebox{0.92}[0.92]{$ \displaystyle \frac{\raisemath{-0.125em}{\partial\hspace{.1ex} \widehat{\potential}^{\mathstrut}}}{\partial \forcestresstensor}$}
\hspace{.15ex} ,
\hspace{.8ex}
\wrynesstensor = \scalebox{0.92}[0.92]{$ \displaystyle \frac{\raisemath{-0.125em}{\partial\hspace{.1ex} \widehat{\potential}^{\mathstrut}}}{\partial \couplestresstensor}$}
\hspace{.2ex} ,
\\[.8em]
%
\widehat{\potential}(\hspace{-0.1ex} \forcestresstensor, \couplestresstensor \hspace{.12ex}) \hspace{-0.12ex}
= \forcestresstensor \dotdotp \distortiontensor^{\T} \hspace{-0.25ex}
+ \hspace{.1ex} \couplestresstensor \dotdotp \wrynesstensor^{\T} \hspace{-0.2ex}
- \hspace{.1ex} \potential(\distortiontensor , \wrynesstensor \hspace{.12ex})
\hspace{.1ex} .
\end{array}
\end{equation}

...

material’s intrinsic (internal) length scale $\elcursive$

Если устремить $\elcursive$ к~нулю, то исчезает вклад $\wrynesstensor$ в~$\potential$, а~с~ним и моментные напряжения~$\couplestresstensor$.
Когда вдобавок нет объёмной моментной нагрузки~$\bm{m}$, тогда тензор~${\hspace{-0.1ex}\forcestresstensor}$ становится симметричным:
${\boldnabla \dotp {\color{black!66}{\couplestresstensor}} \hspace{.1ex}
+ \forcestresstensor_{\Xcompanion} \hspace{-0.1ex}
+ {\color{black!66}{\bm{m}}} \hspace{.1ex}
= \bm{0}}$,
${{\color{black!66}{\couplestresstensor}} = \hspace{-0.15ex} {^2\bm{0}}}$,
${{\color{black!66}{\bm{m}}} = \bm{0} \,\Rightarrow \hspace{.1ex} \forcestresstensor_{\Xcompanion} \hspace{-0.2ex} = \bm{0}}$,
и модель превращается в~классическую безмоментную.

\end{otherlanguage}

\en{Yet using of}\ru{Использование~же} \en{the~micropolar model}\ru{микрополярной модели} \en{is natural}\ru{естественно} \en{in~a~case}\ru{в~случае}\ru{,}
\en{when}\ru{когда} \en{the~real material}\ru{реальный материал} \en{has a~certain smallest volume}\ru{имеет некий наименьший объём,}
\inquotesx{\en{which is impossible to~enter into}\ru{в~который невозможно войти}}[.]
\en{And}\ru{И}~\en{such a~situation}\ru{такая ситуация} \en{occurs}\ru{возникает} \ru{весьма часто}\en{quite often}:
\en{composites}\ru{композиты} \en{with}\ru{с}~\inquotes{\en{representative}\ru{представительным}} \en{volume}\ru{объёмом},
\en{polycrystalline materials}\ru{поликристаллические материалы},
\en{polymers}\ru{полимеры} \en{with}\ru{с}~\en{large molecules}\ru{большими молекулами}~(\en{macro\-molecules}\ru{макро\-молекулами}).

\en{\section{Compatibility equations}}

\ru{\section{Уравнения совместности}}

\label{para:compatibilityequations.cosseratcontinuum}

\en{Having}\ru{Имея} \en{identity}\ru{тождество}
${\boldnabla \hspace{-0.15ex} \times \hspace{-0.3ex} \boldnabla \bm{a} = {^2}\bm{0}}$~${\forall \bm{a}}$
\en{and}\ru{и}~\en{the definitions of the deformation ten\-sors}\ru{определения тензоров деформации}~\eqref{deformationtensors:micropolarcontinuum},

\nopagebreak\vspace{-0.8em}\begin{equation}
\begin{array}{r@{\hspace{1ex}}c@{\hspace{1ex}}l}
\wrynesstensor \hspace{.16ex} \equiv \boldnabla \fieldofrotations
& \hspace{.1ex} \Rightarrow &
\boldnabla \hspace{-0.15ex} \times \hspace{-0.1ex} \wrynesstensor \hspace{.1ex} = {^2}\bm{0}
\hspace{.1ex} ,
\\[.3em]
%
\distortiontensor \hspace{.1ex} - \hspace{.1ex} \fieldofrotations \hspace{-0.12ex} \times \hspace{-0.25ex} \UnitDyad \hspace{.12ex} = \hspace{-0.1ex} \boldnabla \fieldofdisplacements \hspace{.1ex}
& \hspace{.1ex} \Rightarrow &
\boldnabla \hspace{-0.15ex} \times \hspace{-0.2ex} \bigl(
\distortiontensor \hspace{.1ex} - \hspace{.1ex} \fieldofrotations \hspace{-0.12ex} \times \hspace{-0.25ex} \UnitDyad \hspace{.12ex}
\bigr) \hspace{-0.22ex}
= {^2}\bm{0}
\hspace{.8ex} ...
\end{array}
\end{equation}

%%\begin{otherlanguage}{russian}

...

%%\end{otherlanguage}

\en{\section{Theorems of statics}}

\ru{\section{Теоремы статики}}

\begin{otherlanguage}{russian}

Теоремы статики линейных консервативных систем, легко выводимые для конечного числа степеней свободы
(минимальность энергии, теорема Clapeyron’а, \en{reciprocal work theorem}\ru{теорема о~взаимности работ} \en{et~al.}\ru{и~др.}\:--- \chapdotpararef{chapter:genericmechanics}{para:statics}),
справедливы и~для континуальных линейных упругих сред~(\chapref{chapter:linearclassicalelasticity}),
\en{including}\ru{включая} \en{the~micropolar model of~continuum}\ru{микрополярную модель контину\kern-0.11exума}~(\en{a~medium with force couples}\ru{среду с~парами сил}, \en{moments}\ru{моментами}).

...



\end{otherlanguage}

\en{\section{Cosserat pseudocontinuum}}

\ru{\section{Псевдоконтину\kern-0.11exум Коссера}}

\label{para:caseoflatenttrihedron.smalldisplacementsandrotations}

\en{Besides}\ru{Помимо} \en{the~model with free rotation}\ru{модели со~свободным вращением} (\inquotes{\en{the~truly micropolar continuum}\ru{истинно микрополярного контину\kern-0.11exума}}), \en{there is}\ru{существует} \en{the~simplified model}\ru{упрощённая модель} \en{of a~medium with force couples}\ru{среды с~парами сил}, \en{in which}\ru{в~которой} \en{rotations}\ru{повороты} \en{are expressed}\ru{выражаются} \en{via displacements}\ru{через перемещения} \en{as in the~classical momentless continuum}\ru{как в~классическом безмоментном контину\kern-0.11exуме}:

\nopagebreak\vspace{-0.15em}\begin{equation}
\fieldofrotations = \smalldisplaystyleonehalf \hspace{.2ex} \boldnabla \hspace{-0.16ex} \times \hspace{-0.1ex} \fieldofdisplacements
\hspace{.6ex}\Leftrightarrow\hspace{.6ex}
\distortiontensor_{\hspace{-0.25ex}\Xcompanion} \hspace{-0.32ex} = \bm{0}
\hspace{.6ex}\Leftrightarrow\hspace{.6ex}
\distortiontensor = \infinitesimaldeformation = \hspace{-0.12ex} \boldnabla {\fieldofdisplacements}^{\hspace{.1ex}\mathsf{S}}
%%\hspace{-0.25ex} .
\end{equation}

\vspace{-0.2em}\noindent
--- \en{the~model}\ru{модель} \en{with }\ru{со~стеснённым вращением (}constrained rotation\ru{)}\footnote{\ru{Братья }Cosserat\en{ brothers} \en{called it}\ru{называли это} cas de trièdre caché (\ru{случай скрытого трёхгранника, }case of latent trihedron).}\hspace{-0.32em}.

\begin{otherlanguage}{russian}

Равенство~${\distortiontensor_{\hspace{-0.25ex}\Xcompanion} \hspace{-0.32ex} = \bm{0}}$~(симметрию~$\distortiontensor$) возможно понимать как внутреннюю связь~(\chapdotpararef{chapter:nonlinearcontinuum}{para:internalconstraints}).
Аргумент~${\distortiontensor_{\hspace{-0.25ex}\Xcompanion}}$ исчезает из~энергии~$\potential$, соотношение упругости для~${\forcestresstensor_{\hspace{-0.1ex}\Xcompanion}}$ не~может быть написано.
Его место в~полной системе занимает уравнение связи.

В~классической (линейной безмоментной) теории упругости полная система сводится к~одному уравнению для~вектора~$\fieldofdisplacements$~(\chapdotpararef{chapter:linearclassicalelasticity}{para:equationsfordisplacements.linearelasticity}).
В~моментной теории

...



\end{otherlanguage}

\en{\section{Plane deformation}}

\ru{\section{Плоская деформация}}

\label{para:planedeformation.cosseratcontinuum}

\begin{otherlanguage}{russian}

Все переменные в~этой постановке не~зависят от декартовой координаты~${z \equiv x_3}$~(орт оси\:--- $\bm{k}$).
Перемещения и~силы перпендикулярны оси~$z$, а~повороты и~моменты\:--- параллельны ей:

...


Это краткое изложение плоской задачи относится к~модели с~независимыми поворотами.
Псевдоконтину\kern-0.11exум Коссера (модель со~стеснённым вращением) получается либо при наложении внутренней связи~${\distortiontensor_{\hspace{-0.25ex}\Xcompanion} \hspace{-0.32ex} = \bm{0}}$, либо при предельном переходе ...

Подробнее о~плоской моментной задаче написано в~книгах Н.\,Ф.\;Морозова~\cite{morozov-twodimensionalproblems, morozov-fractures}.

\end{otherlanguage}

\en{\section{Nonlinear theory}}

\ru{\section{Нелинейная теория}}

\label{para:nonlinear.micropolar}

\begin{otherlanguage}{russian}

Кажущееся на~первый взгляд чрезвычайно трудным, построение теории конечных деформаций контину\kern-0.11exума Коссера становится прозрачным, если опираться на общую механику, тензорное исчисление и~нелинейную теорию безмоментной среды.

При построении модели упругого контину\kern-0.11exума обычно проходят четыре этапа:
\begin{itemize}
\item определение степеней свободы частиц,
\item выявление нагрузок~(\inquotesx{силовых факторов}[,] напряжений) и~условий их баланса,
\item подбор соответствующих мер деформации
\\
\hspace*{-\listlabelwithsep}и, наконец,
\item вывод соотношений упругости между напряжением и~деформацией.
\end{itemize}

\vspace{-0.16em}\noindent
Этот традиционный путь очень сокращается, если опираться на принцип виртуальной работы.

Как и в~\chapref{chapter:nonlinearcontinuum}, среда состоит из~частиц с~материальными координатами~$q^i$ и вектором\hbox{-}радиусом~${\currentlocationvector(q^i\hspace{-0.4ex}, t)}$.
В~начальной~(исходной, отсчётной) конфигурации ${\currentlocationvector(q^i\hspace{-0.4ex}, 0) \hspace{-0.12ex} \equiv \hspace{.12ex} \initiallocationvector(q^i)}$.
Но кроме трансляции, частицы имеют независимые степени свободы поворота, описываемого ортогональным тензором

\nopagebreak\vspace{-0.2em}\begin{equation*}
\rotationtensor(q^i\hspace{-0.4ex}, t) \hspace{-0.1ex}
\equiv
\bm{a}_{\hspace{-0.12ex}j} \hspace{.16ex} \mathcircabove{\bm{a}}^j \hspace{-0.32ex}
= \hspace{-0.1ex} \bm{a}^j \hspace{-0.12ex} \mathcircabove{\bm{a}}_{\hspace{-0.12ex}j} \hspace{-0.25ex}
= \hspace{-0.1ex} \rotationtensor^{\hspace{-0.1ex}\expminusT}
\hspace{-0.5ex} ,
\end{equation*}

\vspace{-0.1em} \noindent где тройка векторов~${\bm{a}_{\hspace{-0.12ex}j}(q^i\hspace{-0.4ex}, t)}$ жёстко связана с~каждой частицей, показывая угловую ориентацию относительно как\hbox{-}либо выбираемых\footnote{%
Один из вариантов: ${\mathcircabove{\bm{a}}_{\hspace{-0.12ex}j} \hspace{-0.4ex} = \initiallocationvector_{\hspace{-0.2ex}j} \hspace{-0.1ex} \equiv \partial_j \currentlocationvector}$.
Другое предложение: ${\mathcircabove{\bm{a}}_{\hspace{-0.12ex}j} \hspace{-0.2ex}}$ это ортонормальная тройка собственных векторов тензора инерции частицы. %% (но как обосновать такой выбор в~статике?)
Вообще, ${\mathcircabove{\bm{a}}_{\hspace{-0.12ex}j} \hspace{-0.2ex}}$ могут быть любой тройкой линейно-независимых векторов.
}\hspace{-0.2ex}
векторов
${\mathcircabove{\bm{a}}_{\hspace{-0.12ex}j}(q^i) \hspace{-0.1ex} \equiv \bm{a}_{\hspace{-0.12ex}j}(q^i\hspace{-0.4ex}, 0)}$,
${\bm{a}_{\hspace{-0.12ex}j} \hspace{-0.2ex} = \rotationtensor \hspace{-0.1ex} \dotp \hspace{.2ex} \mathcircabove{\bm{a}}_{\hspace{-0.12ex}j}}$;
${\bm{a}^j\hspace{-0.2ex}}$\:--- тройка взаимных векторов:
${\bm{a}_{\hspace{-0.12ex}j} \bm{a}^j \hspace{-0.25ex}
= \hspace{-0.15ex} \bm{a}^j \hspace{-0.2ex} \bm{a}_{\hspace{-0.12ex}j} \hspace{-0.25ex}
= \hspace{-0.2ex} \UnitDyad}$
(${t\!=\!0}$, ${\mathcircabove{\bm{a}}^j}$: ${\mathcircabove{\bm{a}}_{\hspace{-0.12ex}j} \mathcircabove{\bm{a}}^j \hspace{-0.25ex} = \hspace{-0.15ex} \mathcircabove{\bm{a}}^j \mathcircabove{\bm{a}}_{\hspace{-0.12ex}j} \hspace{-0.25ex} = \hspace{-0.2ex} \UnitDyad}$).
Движение среды полностью определяется функциями ${\currentlocationvector(q^i\hspace{-0.4ex}, t)}$ \en{and}\ru{и}~${\rotationtensor(q^i\hspace{-0.4ex}, t)}$.

Имея представления ${\initiallocationvector(q^i)}$ и~${\currentlocationvector(q^i\hspace{-0.4ex}, t)}$,
вводим базис~${\currentlocationvector_i \equiv \partial_i \currentlocationvector}\hspace{-0.2ex}$,
взаимный базис~${\currentlocationvector^i}$: ${\currentlocationvector_{\hspace{-0.2ex}j} \hspace{-0.2ex} \dotp \currentlocationvector^i \hspace{-0.25ex} = \delta^{\hspace{.1ex}i}_{\hspace{-0.2ex}j}}$,
\en{differential operators}\ru{дифференциальные операторы}~$\boldnablacircled$ \en{and}\ru{и}~${\hspace{-0.16ex}\boldnabla}\hspace{-0.1ex}$,
а~также \en{motion gradient}\ru{градиент движения}~$\motiongradient$

\nopagebreak\vspace{-0.1em}\begin{equation}
\boldnablacircled \equiv \initiallocationvector^{\hspace{.1ex}i} \partial_i
\hspace{.1ex} ,
\:\;
\boldnabla \equiv \currentlocationvector^i \partial_i
\hspace{.1ex} ,
\:\;
\boldnabla = \motiongradient^{\expminusT} \hspace{-0.32ex} \dotp \boldnablacircled
,
\:\;
\motiongradient \equiv \hspace{-0.16ex}
\boldnablacircled \currentlocationvector^{\T} \hspace{-0.5ex}
= \currentlocationvector_i \hspace{.2ex} \initiallocationvector^{\hspace{.1ex}i}
\hspace{-0.25ex} .
\end{equation}

Вариационное уравнение принципа виртуальной работы для контину\kern-0.11exума с~нагрузками в~объёме и на~поверхности:

\nopagebreak\vspace{-0.3em}\begin{multline}
\hspace*{2em}
\integral\displaylimits_{\mathcal{V}} \hspace{-0.64ex} \left(^{\mathstrut} \hspace{-0.1ex} \rho \hspace{.2ex} \bigl( \bm{f} \dotp \variation{\currentlocationvector} + \bm{m} \dotp \littlerotationvector \hspace{.1ex} \bigr) \hspace{-0.1ex} + \hspace{.1ex} \variation{\internalwork} \hspace{.1ex} \right) \hspace{-0.3ex} d\mathcal{V}
\hspace{.5ex} + \\[-1.2em]
%
+ \integral\displaylimits_{\mathcal{O}} \hspace{-0.64ex} \left(^{\mathstrut} \bm{p} \dotp \variation{\currentlocationvector} + \mathboldM \hspace{-0.1ex} \dotp \littlerotationvector \right) \hspace{-0.3ex} d\mathcal{O} \hspace{.1ex}
= \hspace{.1ex} 0 \hspace{.1ex} .
\hspace*{2em}
\end{multline}

\vspace{-0.2em} \noindent \en{Here}\ru{Здесь}
$\rho$\ru{\:---}\en{~is} \en{mass density}\ru{плотность массы};
$\bm{f}$ \en{and}\ru{и}~$\bm{m}$\ru{\:---}\en{~are} \en{external}\ru{внешние} \en{force}\ru{сила} \en{and}\ru{и}~\en{moment}\ru{момент} \en{per mass unit}\ru{на~единицу массы};
$\bm{p}$ \en{and}\ru{и}~$\mathboldM$\:--- они~же \en{per surface unit}\ru{на~единицу поверхности};
${\variation{\internalwork}\hspace{-0.2ex}}$\:--- работа внутренних сил \en{per volume unit}\ru{на~единицу объёма} в~текущей конфигурации.
Вектор м\'{а}лого поворота~$\littlerotationvector$

\nopagebreak\vspace{-0.5em}\begin{multline*}
\scalebox{0.98}{$ \rotationtensor \hspace{-0.1ex} \dotp \rotationtensor^{\T} \hspace{-0.2ex} = \UnitDyad
\:\Rightarrow\,
\variation{\hspace{.1ex}\rotationtensor} \hspace{-0.1ex} \dotp \rotationtensor^{\T} \hspace{-0.2ex}
= - \hspace{.2ex} \rotationtensor \hspace{-0.1ex} \dotp \variation{\hspace{.1ex}\rotationtensor}^{\T}
\Rightarrow $}
\\[-0.15em]
%
\scalebox{0.98}{$ \Rightarrow\,
\variation{\hspace{.1ex}\rotationtensor} \hspace{-0.1ex} \dotp \rotationtensor^{\T} \hspace{-0.25ex}
= \hspace{.1ex} \littlerotationvector \hspace{-0.2ex} \times \hspace{-0.2ex} \UnitDyad
= \hspace{.1ex} \littlerotationvector \hspace{-0.2ex} \times \hspace{-0.2ex} \rotationtensor \hspace{-0.1ex} \dotp \rotationtensor^{\T}
\Rightarrow\,
\variation{\hspace{.1ex}\rotationtensor} \hspace{-0.1ex} = \littlerotationvector \hspace{-0.2ex} \times \hspace{-0.2ex} \rotationtensor , $}
\end{multline*}

\vspace{-0.4em}\begin{equation*}
\littlerotationvector = - \, \smalldisplaystyleonehalf \hspace{.1ex}
\scalebox{1}{$ \left( \variation{\hspace{.1ex}\rotationtensor} \hspace{-0.1ex} \dotp^{\mathstrut} \hspace{-0.1ex} \rotationtensor^{\T} \hspace{.1ex} \right)_{\hspace{-0.15em}\Xcompanion} $}
\end{equation*}

\vspace{-0.1em}
При движении среды как жёсткого целого нет деформаций, и~работа~$\variation{\internalwork}\hspace{-0.2ex}$ внутренних сил равна нулю:

\nopagebreak\vspace{-0.1em}\begin{equation*}
\begin{array}{c}
\variation{\currentlocationvector}
= \boldconstant \hspace{.1ex}
+ \hspace{.1ex} \littlerotationvector \hspace{-0.2ex} \times \hspace{-0.12ex} \locationvector ,
\:\,
\littlerotationvector = \boldconstant
\hspace{.64ex}\Rightarrow\hspace{.5ex}
\variation{\internalwork} \hspace{-0.2ex}
= 0 \hspace{.1ex} ,
\\[.25em]
%
\boldnabla \littlerotationvector
= {^2\bm{0}}
\hspace{.1ex} ,
\:\,
\boldnabla \variation{\currentlocationvector}
= \hspace{-0.25ex} \boldnabla \littlerotationvector %%{^2\bm{0}}
\hspace{-0.2ex} \times \hspace{-0.2ex} \locationvector
- \hspace{-0.25ex} \boldnabla \locationvector \hspace{-0.2ex} \times \hspace{-0.2ex} \littlerotationvector
= - \UnitDyad \hspace{-0.2ex} \times \hspace{-0.2ex} \littlerotationvector
= - \hspace{.2ex} \littlerotationvector \hspace{-0.2ex} \times \hspace{-0.2ex} \UnitDyad
\hspace{.1ex} ,
\\[.25em]
%
\boldnabla \variation{\currentlocationvector}
+ \littlerotationvector \hspace{-0.2ex} \times \hspace{-0.2ex} \UnitDyad
= {^2\bm{0}}
\hspace{.1ex} .
\end{array}
\end{equation*}

К~нагрузкам.
Несимметричные тензоры напряжения, силового~$\forcestresstensor$ и моментного~$\couplestresstensor$, введём как множители Lagrange’а:

\nopagebreak\vspace{-0.3em}\begin{multline}
\integral\displaylimits_{\mathcal{V}} \hspace{-0.64ex}
\left(^{\mathstrut} \hspace{-0.1ex}
\rho \hspace{.2ex} \bigl(
\bm{f} \dotp \variation{\currentlocationvector}
+ \bm{m} \dotp \littlerotationvector
\hspace{.1ex} \bigr) \hspace{-0.2ex}
- \forcestresstensor \dotdotp \hspace{-0.1ex} \bigl( \hspace{.16ex} \boldnabla \variation{\currentlocationvector} + \littlerotationvector \hspace{-0.2ex} \times \hspace{-0.25ex} \UnitDyad \hspace{.2ex} \bigr)^{\hspace{-0.25ex}\T} \hspace{-0.5ex}
- \couplestresstensor \dotdotp \hspace{-0.16ex} \boldnabla \littlerotationvector^{\T}
\right) \hspace{-0.3ex} d\mathcal{V}
\hspace{.5ex} + \\[-1.2em]
%
+ \integral\displaylimits_{\mathcal{O}} \hspace{-0.64ex}
\left(^{\mathstrut}
\bm{p} \dotp \variation{\currentlocationvector}
+ \mathboldM \hspace{-0.1ex} \dotp \littlerotationvector
\right) \hspace{-0.3ex} d\mathcal{O} \hspace{.1ex}
= \hspace{.1ex} 0 \hspace{.1ex} .
\end{multline}

\textcolor{magenta}{После тех~же преобразований, что и~в~\pararef{para:introtolinearmicropolar}, получаем}

...

Отсюда вытекают уравнения баланса сил и~моментов в~объёме и~краевые условия в~виде \textcolor{magenta}{формул типа Cauchy}.
Они по~существу те~же, что и в~линейной теории.

Найдём теперь тензоры деформации.
Их можно вводить по\hbox{-}разному, если требовать лишь одного\:--- нечувствительности к~движению среды как жёсткого целого.
Читатель найдёт не~один вариант таких тензоров.
Однако, вид тензоров деформации \inquotes{подсказывает} принцип виртуальной работы.

...



\end{otherlanguage}

\en{\section{Nonlinear model with constrained rotation}}

\ru{\section{Нелинейная модель со стеснённым вращением}}

\label{para:caseoflatenttrihedron.largedisplacementsandrotations}

\begin{otherlanguage}{russian}

Вспомним переход к~модели со~стеснённым вращением в~линейной теории~(\pararef{para:caseoflatenttrihedron.smalldisplacementsandrotations}).
Разделились соотношения упругости для~симметричной части тензора силового напряжения ${\forcestresstensor^{\hspace{.32ex}\mathsf{S}}}\hspace{-0.2ex}$ и~кососимметричной его части~${\forcestresstensor_{\hspace{-0.1ex}\Xcompanion}}$.
Возникла внутренняя связь~${\distortiontensor_{\hspace{-0.25ex}\Xcompanion} \hspace{-0.32ex} = \bm{0}}$

...




\end{otherlanguage}

\section*{\small \wordforbibliography}

\begin{changemargin}{\parindent}{0pt}
\fontsize{10}{12}\selectfont

\begin{otherlanguage}{russian}

Все работы по~моментной теории упругости упоминают книгу братьев Eugène et~François Cosserat~\cite{cosserat}, где~трёх\-мерной среде посвящена одна глава из~шести.
Переведённая монография \hbox{W\hspace{-0.2ex}.\:Nowacki}~\cite{nowacki-elasticity} была одной из первых книг на~русском языке с~изложением линейной моментной теории.
Ранее эта область представлялась статьями\:--- например, R.\,D.\:Mindlin’а и~H.\,F.\:Tiersten’а~\cite{mindlin.tiersten}.
Краткое изложение моментной теории, но с~подробным рассмотрением задач содержится в~книгах Н.\,Ф.\:Морозова~\cite{morozov-twodimensionalproblems, morozov-fractures}.

\end{otherlanguage}

\end{changemargin}
