\en{\section{Theorems of statics}}

\ru{\section{Теоремы статики}}

\label{section:theoremsofstatics}

\subsection*{\ru{Теорема }Clapeyron’\ru{а}\en{s theorem}}

% Benoît Paul Émile Clapeyron
% Mémoire sur le travail des forces élastiques dans un corps solide élastique déformé par l’action de forces extérieures
% Comptes rendus hebdomadaires des séances de l’Académie des sciences, Tome XLVI, Janvier–Juin 1858

\en{In~equilibrium}\ru{В~равновесии}
\en{with the external forces}\ru{с~внешними силами},
\ru{объёмными}\en{the volume ones}~$\volumeloadvector$
\ru{и}\en{and}
\en{the surface ones}\ru{поверхностными}
$\bm{p}$,
\en{the work}\ru{работа}
\en{of~these}\ru{этих}
\inquotesx{\en{statically frozen}\ru{статически замороженных}}
(\en{that is constant along time}\ru{то есть постоянных во~времени})
\en{forces}\ru{сил}
\en{on the actual displacements}\ru{на~актуальных смещениях}
\ru{равна}\en{is equal to}
\en{the double of}\ru{удвоенной}\footnote{%
%%\emph{(i)}~\bookauthor{Gabriel Lam\'{e}} et \bookauthor{Benoît Paul \'{E}mile Clapeyron}.
%%\href{https://gallica.bnf.fr/ark:/12148/bpt6k33093/f475.image}{M\'{e}moire sur l'\'{e}quilibre int\'{e}rieur des~corps solides homog\'{e}nes. \emph{Memoires present\'{e}s par Divers Savants}, IV, 1833. Pagine~465\hbox{--}562.}
%%\emph{(ii)}~

\inquotesx{Ce produit repr\'{e}sentait d’ailleurs le~double de la~force~vive que le~ressort pouvait absorber par l’effet de sa flexion et qui \'{e}tait la~mesure naturelle de~sa~puissance.}[---]\\
\bookauthor{Benoît Paul \'{E}mile Clapeyron}.
\href{https://gallica.bnf.fr/ark:/12148/bpt6k3003h/f208.image}{ M\'{e}moire sur le travail des forces \'{e}lastiques dans un corps solide \'{e}lastique d\'{e}form\'{e} par l’action de~forces ext\'{e}rieures.
\emph{Comptes rendus},
Tome\;XLVI, Janvier--Juin 1858.
Pagine~208\hbox{--}212.}
}\hspace{-0.4ex}
\en{the energy of~deformation}\ru{энергии деформации}

\nopagebreak\vspace{-0.1em}\begin{equation}\label{clapeyron:elasticitytheorem}
2 \hspace{-0.2em}
\integral\displaylimits_{\mathcal{V}} \hspace{-0.5ex} \potentialenergydensity \hspace{.2ex} d\mathcal{V} \hspace{.12ex}
= \hspace{-0.2ex}
\integral\displaylimits_{\mathcal{V}} \hspace{-0.4ex} \volumeloadvector \hspace{-0.1ex} \dotp \bm{u} \hspace{.25ex} d\mathcal{V}
+ \hspace{-0.3ex}
\integral\displaylimits_{o_2} \hspace{-0.4ex} \bm{p} \dotp \bm{u} \hspace{.25ex} do
\hspace{.2ex} .
\end{equation}

\vspace{-0.55em}\begin{multline*}
\tikz[baseline=-1ex] \draw [line width=.5pt, color=black, fill=white] (0, 0) circle (.8ex);
\hspace{2.25ex}
2 \hspace{.1ex} \potentialenergydensity \hspace{.1ex} = \linearstress \dotdotp \infinitesimaldeformation =
\linearstress \hspace{.1ex} \dotdotp \hspace{-0.12ex} \boldnabla \bm{u}^{\mathsf{S}} \hspace{-0.12ex} =
\boldnabla \hspace{-0.1ex} \dotp \left( \hspace{.1ex} \linearstress \dotp \bm{u} \hspace{.1ex} \right) -
\tikzmark{beginMinusLoad} \boldnabla \dotp \linearstress \tikzmark{endMinusLoad} \dotp \bm{u} \;\Rightarrow
\\[.5em]
%
\Rightarrow\;\,
\displaystyle 2 \hspace{-0.16em}
\integral\displaylimits_{\mathcal{V}} \hspace{-0.5ex}
  \potentialenergydensity \hspace{.2ex} d\mathcal{V} \hspace{.12ex}
= \integral\displaylimits_{o_2} \hspace{-0.4ex}
\tikzmark{beginSurfaceLoad}
  \unitnormalvector \dotp \linearstress
\tikzmark{endSurfaceLoad} \hspace{.1ex}
  \dotp \bm{u} \hspace{.25ex} do \hspace{.4ex}
+ \integral\displaylimits_{\mathcal{V}} \hspace{-0.4ex}
  \volumeloadvector \hspace{-0.1ex} \dotp \bm{u} \hspace{.25ex}
  d\mathcal{V}
\hspace{2.25ex}
\tikz[baseline=-0.6ex] \draw [color=black, fill=black] (0, 0) circle (.8ex);
\end{multline*}%
\AddUnderBrace[line width=.75pt][.2ex, 0][yshift = .1em]%
{beginMinusLoad}{endMinusLoad}{$
    \scalebox{.75}{$ - \volumeloadvector $}
$}%
\AddUnderBrace[line width=.75pt][-0.1ex, 0][yshift = .1em]%
{beginSurfaceLoad}{endSurfaceLoad}{$
    \scalebox{.75}{$ \bm{p} $}
$}

\vspace{-0.2em}
\en{From}\ru{Из}~\eqref{clapeyron:elasticitytheorem}
\en{also follows}\ru{следует также},
\en{that without loading}\ru{что без нагрузки}
${\hspace{-0.25ex}
\scalebox{1.4}{$ \integral $}_
{ \hspace{-0.5ex}
   \raisemath{.1em}{\mathcal{V}}
} \hspace{.25ex}
\potentialenergydensity \hspace{.2ex} d\mathcal{V}
= 0
}$.
\en{Because}\ru{Поскольку} $\potentialenergydensity$
\en{is positive}\ru{положителен},
\en{then the stress}\ru{то~и~напряжение}~$\linearstress$,
\en{and}\ru{и}
\en{deformation}\ru{деформация}~$\infinitesimaldeformation$
\en{without a~load}\ru{без нагрузки}
\en{are equal to zero}\ru{равны нулю}.

\begin{align*}
2 \hspace{.1ex} \potentialenergydensity \hspace{.1ex} &= \linearstress \dotdotp \infinitesimaldeformation
\\
\mathdotabove{\potentialenergydensity} \hspace{.1ex} &= \linearstress \dotdotp \mathdotabove{\infinitesimaldeformation}
\\
\variation{\potentialenergydensity} &= \linearstress \dotdotp \variation{\infinitesimaldeformation}
\end{align*}

$\potentialenergydensity$
\en{is equal to}\ru{равна}
\en{only}\ru{лишь}
\en{the~half}\ru{половине}
\en{of~the~work}\ru{работы}
\en{of~the~external forces}\ru{внешних сил}.

\en{The~accumulated}\ru{Накопленная}
\en{potential energy of~deformation}\ru{потенциальная энергия деформации}~$\potentialenergydensity$
\en{is equal}\ru{равна}
\en{to only}\ru{только}
\en{the half}\ru{половине}
\en{of the work}\ru{работы}\ru{,}
\en{done}\ru{совершённой}
\en{by the external forces}\ru{внешними силами},
\en{acting}\ru{действующими}
\en{from}\ru{из}
\en{the~unstressed}\ru{ненагруженной}
\en{configuration}\ru{конфигурации}
\en{to the~equilibrium}\ru{к~равновесию}
\en{with}\ru{со}
\en{the external forces}\ru{внешними силами}.

\ru{Теорема }Clapeyron’\en{s}\ru{а}\en{ theorem}
\en{implies that}\ru{подразумевает, что}
\en{the~accumulated elastic energy}\ru{накопленная упругая энергия}
\en{accounts for}\ru{составляет}
\en{only the~half}\ru{лишь половину}
\en{of the energy}\ru{энергии}\ru{,}
\en{spent on the deformation}\ru{потраченной на деформацию}.
\en{The~remaining}\ru{Оставшаяся}
\en{half}\ru{половина}
\en{of the~work}\ru{работы},
\en{done by the external forces}\ru{совершённой внешними силами},
\en{is lost somewhere}\ru{теряется где\hbox{-}то}
\en{before reaching}\ru{до достижения}
\en{the~equilibrium}\ru{равновесия}.

\bookauthor{Roger Fosdick} and \bookauthor{Lev Truskinovsky}.
\href{http://www.cityu.edu.hk/ma/ws2007/notes/FTJElast.pdf}{About Clapeyron’s Theorem in Linear Elasticity. \emph{Journal of~Elasticity}, Volume~72, July 2003. Pages 145\hbox{--}172.}

\en{In theory}\ru{В~теории}\en{,}
\ru{распространена }\en{the concept}\ru{концепция}
\en{of~the~}\inquotesx{\en{static loading}\ru{статического нагружения}}\en{ is common}.
\en{It‘s}\ru{Это}
\en{when}\ru{когда}
\en{the external load}\ru{внешняя нагрузка}
\en{is applied}\ru{применяется}
\en{infinitely slow}\ru{бесконечно медленно}
(\en{sounds like forever}\ru{звучит как вечность},
\en{yeah}\ru{да}).

\en{The work}\ru{Работа}
\en{of the external forces}\ru{внешних сил}
\en{on the actual displacements}\ru{на актуальных смещениях}
\ru{равна}\en{is equal to}
\en{the double}\ru{удвоенной}
\en{of the potential energy density}\ru{плотности потенциальной энергии}
$2\potentialenergydensity$.

\begin{otherlanguage}{russian}
Если снять внешние воздействия
мгновенно (бесконечно быстро),
то тело будет колебаться.
Но \textcolor{magenta}{из-за сопротивления среды (внутреннего трения)}
спустя некоторое время
тело придёт в состояние равновесия.
\end{otherlanguage}

\en{Yes}\ru{Да},
\en{only the half}\ru{всего половина}
\en{of the linear elastic energy}\ru{линейно упругой энергии}
\en{is stored}\ru{запасается}.
\en{The second half}\ru{Вторая половина}
\en{is}\ru{это}
\en{the }\inquotes{ \en{additional energy}\ru{дополнительная энергия} },
\en{which is lost}\ru{которая теряется}
\en{before}\ru{до}
\en{reaching}\ru{достижения}
\en{of the equilibrium}\ru{равновесия}
\en{on}\ru{на}
\en{the dynamics}\ru{динамику}\:---
\en{on}\ru{на}
\en{the internal energy}\ru{внутреннюю энергию}
\en{of the particles}\ru{частиц}
( \en{of the dissipation}\ru{диссипацию} ),
\en{on}\ru{на}
\en{the vibrations}\ru{колебания}
\en{and waves}\ru{и~волны}.

\en{But}\ru{Но}
\en{any real loading}\ru{любое реальное нагружение}
\en{would be}\ru{не будет}
\en{neither a~sudden loading}\ru{ни внезапным нагружением}\ru{,}
\en{nor an~infinitely slow loading}\ru{ни бесконечно медленным}.
\en{These are the two extremes}\ru{Это две крайности}.
\en{The real dynamics}\ru{Реальная динамика}
\en{of applying the loads}\ru{приложения нагрузок}
\en{will always be}\ru{всегда будет}
\en{different}\ru{отличной}
\en{from the theory}\ru{от теории}.

%(there’s always heating due to the dissipation of the energy)

\en{Within}\ru{В~пределах}
\en{infinitesimal variations}\ru{бесконечномалых вариаций},
\en{applied to an~elastic medium}\ru{приложенные к~упругой среде}
\en{real external forces}\ru{реальные внешние силы}
\en{work}\ru{работают}
\en{on virtual displacements}\ru{на виртуальных смещениях}~${\variation{\infinitesimaldeformation}}$
\en{and}\ru{и}
\en{produce work}\ru{производят работу}~${\variation{\externalwork} \hspace{-0.25ex}}$\ru{,}
\en{that}\ru{которая}
\en{is exactly equal}\ru{в~точности равна}
\textcolor{magenta}{(for linear elastic only or for non-linear too??)}
\en{to the~variation}\ru{вариации}
\en{of the elastic potential energy density}\ru{плотности упругой потенциальной энергии}

\nopagebreak\vspace{-0.5em}
\begin{equation*}
\variation{\externalwork} \hspace{-0.25ex}
= \linearstress \dotdotp \variation{\infinitesimaldeformation}
= \variation{\potentialenergydensity}
\hspace{.1ex}
.
\end{equation*}

\en{A~linear}\ru{Линейная}
\en{elastic medium}\ru{упругая среда}
\en{is medium}\ru{это среда},
\ru{где}\en{where}
\en{the~variation}\ru{вариация}
\en{of~work}\ru{работы}
\en{of the internal forces}\ru{внутренних сил}
(\en{that is stresses}\ru{то есть напряжений})
\en{is}\ru{есть}
\en{the~variation}\ru{вариация}
\en{of the potential energy density}\ru{плотности потенциальной энергии}
\en{with the opposite sign}\ru{с~противоположным зн\'{а}ком}
${
   - \hspace{.1ex} \variation{\internalwork} \hspace{-0.2ex}
   = \variation{\potentialenergydensity} = \variation{\externalwork} \hspace{-0.25ex}
}$,
\en{when}\ru{когда}
\ru{варьируются }\en{only}\ru{только}
\en{the~displacements}\ru{смещения}\en{ vary}
(\en{while}\ru{пока}
\en{the stress loads}\ru{нагрузки\hbox{-}напряжения}
\en{do not}\ru{нет}).

\en{It is necessary that}\ru{Необходимо, чтобы}
\en{the~virtual work}\ru{виртуальная работа}
\en{of the~real}\ru{реальных}
\en{external forces}\ru{внешних сил}
\en{on variations}\ru{на~вариациях}
\en{of displacements}\ru{смещений}
\en{be equal}\ru{была равна}
\en{to the variation}\ru{вариации}
\en{of internal energy}\ru{внутренней энергии}
\en{with the~opposite sign}\ru{с~обратным знаком}
( \en{for}\ru{для}
\en{an elastic media}\ru{упругой среды}\:---
\en{the~variation}\ru{вариации}
\en{of internal energy}\ru{внутренней энергии} ).

\subsection*{\en{The uniqueness of the solution theorem}\ru{Теорема единственности решения}}

\en{As in dynamics}\ru{Как и в~динамике}
(\sectionref{section:uniquenessfordynamicproblem}),
\en{we}\ru{мы}
\en{suppose}\ru{допускаем}
\en{the existænce}\ru{существование}
\en{of the~two solutions}\ru{двух решений}
\en{and}\ru{и}
\en{are looking for}\ru{ищем}
\en{their difference}\ru{их~разность}

\begin{equation}\label{kirchhoff:uniquenessforstatics}
..........
\end{equation}

%%%%%%%%%%
\newpage
%%%%%%%%%%

%%%\hspace*{-\parindent}
%%%\begin{minipage}{\linewidth}

% ~ ~ ~ ~ ~
\begin{wrapfigure}[0]{o}{.4\textwidth}
%%\makebox[.45\textwidth][c]{%
\begin{minipage}[t]{.5\textwidth}
%%\vspace{-1em}
\scalebox{.88}{
    %%\begin{center}

\begin{wrapfigure}{o}{.4\textwidth}
\makebox[.45\textwidth][c]{\begin{minipage}[t]{.5\textwidth}
\vspace{-1em}
\scalebox{0.88}{

\def\cameraangle{166}
\tdplotsetmaincoords{44}{\cameraangle} % orientation of camera

\def\rodheight{10}
\def\rodradius{.2}

\pgfmathsetmacro{\beginangle}{\cameraangle}
\pgfmathsetmacro{\endangle}{\cameraangle - 180}

\tikzset{pics/rod/.style={code={

	% draw rod

	\draw [line width=1pt, color=black, fill=yellow!50!white, opacity=.9]
		plot [domain=\beginangle:\endangle]
			( {\rodradius*cos(\x)}, {\rodradius*sin(\x)}, 0 )
		-- plot [domain=\endangle:\beginangle]
			( {\rodradius*cos(\x)}, {\rodradius*sin(\x)}, \rodheight )
		-- cycle ;

	\draw [line width=1pt, color=black, fill=yellow!50!white, opacity=.9, domain=0:360]
		plot ( {-\rodradius*cos(\x)}, {-\rodradius*sin(\x)}, \rodheight ) ;

}}}

\tikzset{pics/dottedrod/.style={code={

	% draw rod dotted

	\draw [line width=1pt, color=black, line cap=round, dash pattern=on 0pt off 1.6\pgflinewidth]
		plot [domain=\beginangle:\endangle]
			( {\rodradius*cos(\x)}, {\rodradius*sin(\x)}, 0 )
		-- plot [domain=\endangle:\beginangle]
			( {\rodradius*cos(\x)}, {\rodradius*sin(\x)}, \rodheight )
		-- cycle ;

	\draw [line width=1pt, color=black, line cap=round, dash pattern=on 0pt off 1.6\pgflinewidth, opacity=.9, domain=0:360]
		plot ( {-\rodradius*cos(\x)}, {-\rodradius*sin(\x)}, \rodheight ) ;

}}}

\tikzset{pics/rodaxis/.style={code={

	% draw axis
	\draw [line width=0.5pt, blue, line cap=round, dash pattern=on 12pt off 2pt on \the\pgflinewidth off 2pt]
		( 0, 0, -0.4pt ) -- ( 0, 0, \rodheight + 0.4pt ) ;

}}}

\tikzset{pics/externalforce/.style={code={

	% draw force
	\def\forcelength{1.2}

	\draw [line width=1.5pt, red, line cap=round, -{Triangle[round, length=3.6mm, width=2.4mm]}]
		( 0, 0, \rodheight + \forcelength ) -- ( 0, 0, \rodheight )
		node [ pos=0.4, left, inner sep=0, outer sep=4.4pt ]
			{\scalebox{1.2}[1.2]{${\bm{F}}$}} ;

}}}

\begin{tikzpicture}[scale=1, tdplot_main_coords] % use 3dplot

	\coordinate (O) at ( 0, 0, 0 ) ;
	\coordinate (rodTopCenter) at ($ (O) + ( 0, 0, \rodheight ) $) ;

	% draw circle
	\def\circleradius{0.8}
	\def\heightofhatch{0.5}

	\pgfmathsetmacro{\stepangleforcircle}{\beginangle - 10}
	\foreach \angle in { \beginangle, \stepangleforcircle, ..., \endangle }
		\draw [line width=0.4pt, color=black]
			( \angle:\circleradius ) -- ($ ( \angle:\circleradius ) - ( 0, 0, \heightofhatch ) $) ;

	\draw [line width=1pt, color=black, fill=white] (O) circle ( \circleradius ) ;

	% draw rod, axis and force
	\pic (initial) {rod} ;
	\pic (initial) {rodaxis} ;
	\pic (initial) {externalforce} ;

	% draw deformed rod
	\scoped {
		\pgfsetcurvilinearbeziercurve
			{\pgfpointxyz{0}{0}{0}}
			{\pgfpointxyz{0}{0}{0.5cm}}
			{\pgfpointxyz{0.25cm}{0}{1cm}}
			{\pgfpointxyz{1.25cm}{0}{1.25cm}}
		\pgftransformnonlinear{\pgfgetlastxy\x\y\pgfpointcurvilinearbezierorthogonal{\y}{\x}}
			\pic (deformed) {dottedrod} ;
			\pic (deformed) {rodaxis} ;
	}

\end{tikzpicture}
}
\vspace{-0.2em}\caption{}\label{fig:deformedrod}
\end{minipage}}
\end{wrapfigure}

%%\end{center}

}
\vspace{-0.2em}\caption{}\label{fig:deformedrod}
\end{minipage}%
%%}
\end{wrapfigure}
% ~ ~ ~ ~ ~

%%%%%%%%%%
\newpage
%%%%%%%%%%

\hspace{\horizontalindent}%
\en{The~uniqueness}\ru{Уникальность}
\en{of the~solution}\ru{решения},
\en{dis\-covered}\ru{открытая}
\en{by~}Gustav\ru{’ом} Kirchhoff\ru{’ом}
\en{for bodies}\ru{для тел}
\en{with }\ru{с~}\en{the simply connected}\ru{одно\-св\'{я}зным}
\en{contour}\ru{контуром}%
\stepcounter{footnote}\setcounter{auxfootnotecounter}{\value{footnote}}\footnotemark[\value{auxfootnotecounter}]\hbox{\hspace{-0.4ex},}
\en{is contrary to}\ru{противоречит},
\en{as it seems}\ru{\hbox{казалось~бы}},
\en{the everyday experience}\ru{повседневному опыту}.
\en{Imagine}\ru{Вообразим}
\en{a~straight rod}\ru{прямой стержень}\en{,}
\en{clamped}\ru{зажатый}
\en{at the one end}\ru{на одном конц\'{е}}
(the \inquotes{ \en{cantilever}\ru{конс\'{о}льный} })
\en{and compressed}\ru{и~сжимаемый}
\en{at the second end}\ru{на~втором конц\'{е}}
\en{with a~longitudinal force}\ru{продольной силой}~(\figureref{fig:deformedrod}).
\en{When}\ru{Когда}
\en{the~load}\ru{нагрузка}
\en{is large enough}\ru{достаточно больш\'{а}я},
\en{the~problem of statics}\ru{задача статики}
\en{has}\ru{имеет}
\en{the two solutions}\ru{два решения},
\inquotes{\en{straight}\ru{прямое}} \en{and}\ru{и}~\inquotesx{\en{bent}\ru{изогнутое}}.
\en{Such a~contradiction}\ru{такое противоречие}
\en{with the~uniqueness theorem}\ru{с~теоремой единственности}
\en{comes from}\ru{происходит от}
\en{the nonlinearity of this problem}\ru{нелинейности этой задачи}.
\en{If a~load is small}\ru{Если нагрузка мал\'{а}}
(\en{infinitesimal}\ru{бесконечно мал\'{а}}),
\en{then}\ru{то}
\en{the~solution}\ru{решение}
\en{is described}\ru{описывается}
\en{by the linear equations}\ru{линейными уравнениями}
\en{and}\ru{и}
\en{is unique}\ru{единственно}.


...

...

%%%\end{minipage}

\nicefootnotetext{auxfootnotecounter}{\bookauthor{Gustav Robert Kirchhoff}. \href{https://opacplus.bsb-muenchen.de/Vta2/bsb10525510/bsb:2960444?page=291}{Über das~Gleich\-gewicht und die~Bewe\-gung eines unendlich dünnen elastischen Stabes. \emph{Journal für die reine und angewandte Mathematik (Crelle’s journal)}, 56.\:Band (1859). Seiten 285\hbox{--}313.} (Seite 291)}

\subsection*{\en{Reciprocal work theorem}\ru{Теорема взаимности работ}}

Proposed by Enrico Betti\footnote{\bookauthor{Enrico Betti}.
\href{https://play.google.com/books/reader?id=RHDzLUBIlUUC&pg=GBS.PA69}{Teoria della elasticit\`{a}.
\emph{Il Nuovo Cimento}~(1869--1876), VII\:e\:VIII\;(1872). Pagina 69.}}{\hspace{-0.5ex}.}

%%Il Nuovo Cimento (1869--1876)
%%Betti, E. Teoria della elasticità. Nuovo Cimento 7\hbox{--}8 (1872)
%%Betti, Nuovo Cimento VII--VIII (1872): 5--21, 69--97, 158--180

\en{For a~body}\ru{Для т\'{е}ла}
\en{with a~fixed part}\ru{с~фиксированной частью}~${o_1}$
\en{of the surface}\ru{поверхности}\en{,}
\en{the two cases are considered}\ru{рассматриваются два случая}\::
\en{the~first one}\ru{первая}
\en{with loads}\ru{с~нагрузками}
$\volumeloadvector_1$, $\bm{p}_1$
\en{and the~second one}\ru{и~вторая}
\en{with loads}\ru{с~нагрузками} $\volumeloadvector_2$, $\bm{p}_2$.
\en{The~verbal formulation}\ru{Словесная формулировка}
\en{of the~theorem}\ru{теоремы}
\en{is the~same as in}\ru{та~же, что и~в}~\chapterdotsectionref{chapter:classicalmechanics}{section:statics}.
\en{The~mathematical notation}\ru{Математическая запись}

\nopagebreak\vspace{1.3em}\begin{equation}\label{betti:reciprocalworktheorem}
\tikzmark{beginReciprocalFirstVariant}
\scalebox{.96}{$ \displaystyle %
   \integral\displaylimits_{\mathcal{V}} \hspace{-0.4ex} \volumeloadvector_1 \hspace{-0.2ex} \dotp \bm{u}_2 \hspace{.25ex} d\mathcal{V} $}
+ \hspace{-0.3ex}
\scalebox{.96}{$ \displaystyle %
   \integral\displaylimits_{o_2} \hspace{-0.4ex} \bm{p}_1 \hspace{-0.2ex} \dotp \bm{u}_2 \hspace{.25ex} do $}
\tikzmark{endReciprocalFirstVariant}
%
\hspace{.2ex} = \hspace{-0.2ex}
%
\tikzmark{beginReciprocalSecondVariant}
\scalebox{.96}{$ \displaystyle %
   \integral\displaylimits_{\mathcal{V}} \hspace{-0.4ex} \volumeloadvector_2 \hspace{-0.2ex} \dotp \bm{u}_1 \hspace{.25ex} d\mathcal{V} $}
+ \hspace{-0.3ex}
\scalebox{.96}{$ \displaystyle %
   \integral\displaylimits_{o_2} \hspace{-0.4ex} \bm{p}_2 \hspace{-0.2ex} \dotp \bm{u}_1 \hspace{.25ex} do $}
\tikzmark{endReciprocalSecondVariant}
\hspace{.1ex} .
\end{equation}%
\AddOverBrace[line width=.75pt][.1ex, .66em][yshift=-0.11ex]%
{beginReciprocalFirstVariant}{endReciprocalFirstVariant}{\scalebox{0.88}{$ W_{\hspace{-0.1ex}12} $}}%
\AddOverBrace[line width=.75pt][.1ex, .66em][yshift=-0.11ex]%
{beginReciprocalSecondVariant}{endReciprocalSecondVariant}{\scalebox{0.88}{$ W_{\hspace{-0.15ex}21} $}}

...

{\small
The reciprocal work theorem, also known as the Betti’s theorem, claims that for a~linear elastic structure subject to the two sets of forces, $P$ and $Q$, the~work done by set~$P$ through displacements produced by set~$Q$ is equal to the~work done by set~$Q$ through displacements produced by set~$P$.
This theorem has applications in structural engineering where it is used to define influence lines and derive the boundary element method.
\par}

...

%%%\hspace*{-\parindent}
%%%\begin{minipage}{\linewidth}

\begin{wrapfigure}[12]{o}{.42\textwidth}
\makebox[.42\textwidth][c]{\begin{minipage}[t]{.42\textwidth}
\vspace{-0.5em}

% parameter #1 is the height of beam
\newcommand\drawrodbeam[1]%
{
       \draw [line width=1.2pt, black] (-0.5, 0) -- (0.5, 0) ;
       \foreach \xground in {-0.36, -0.16, ..., 0.5}
               \draw [line width=0.4pt, black!80] (\xground, 0) -- (\xground - 0.2, -0.2) ;

       \draw [line width=2pt, line cap=round, black] (0, 0) -- (0, #1) ;
}

\tikzstyle{force line} =
       [line width=1.25pt, red, line cap=round, -{Triangle[round, length=3.2mm, width=2mm]}]

\tikzstyle{unitforce line} =
       [line width=1.25pt, blue, line cap=round, dash pattern=on 0pt off 1.6\pgflinewidth, -{Triangle[round, length=3.2mm, width=2mm]}]

\begin{tikzpicture}[scale=0.8]

       \def\beamheight{5}
       \def\beamxshift{1.9}
       \def\firstforcelength{0.9}
       \def\secondforcelength{1.1}
       \def\firstunitforcelength{0.69}
       \pgfmathsetmacro\secondunitforcelength{\firstunitforcelength * 3 / 4}

       \pgfmathsetmacro{\firstforceposition}{\beamheight}
       \pgfmathsetmacro{\secondforceposition}{0.6 * \beamheight}

       \drawrodbeam{\beamheight}

       \foreach \loadpos in { 0, 0.3, ..., 0.6 }
               \draw [force line]
                       ( -\firstforcelength, \firstforceposition - \loadpos ) -- ( 0, \firstforceposition - \loadpos ) ;

       \node at ( -\firstforcelength, \firstforceposition - 0.3 )
               [left, inner sep=0, outer sep=3.3pt, red]
                       {${P_{\hspace{-0.25ex}\raisemath{-0.4ex}{1}}}$} ;

       \foreach \loadpos in { 0, 0.3, ..., 1 }
               \draw [force line]
                       ( -\secondforcelength, \secondforceposition - \loadpos ) -- ( 0, \secondforceposition - \loadpos ) ;

       \node at ( -\secondforcelength, \secondforceposition - 0.5 )
               [left, inner sep=0, outer sep=3.3pt, red]
                       {${P_{\hspace{-0.25ex}\raisemath{-0.4ex}{2}}}$} ;

       \pgfmathsetmacro\secondxshift{\beamxshift}

       \begin{scope}[xshift=\secondxshift cm, yshift=0cm]
               \drawrodbeam{\beamheight}
       \end{scope}

       \foreach \loadpos in { 0, 0.3, ..., 0.6 }
               \draw [unitforce line]
                       ( \secondxshift - \firstunitforcelength, \firstforceposition - \loadpos )
                       -- ( \secondxshift, \firstforceposition - \loadpos ) ;

       \node at ( \secondxshift - \firstunitforcelength, \firstforceposition - 0.3 )
               [left, inner sep=0, outer sep=3.2pt, blue] {$ 1 $} ;

       \pgfmathsetmacro\thirdxshift{\beamxshift + \beamxshift}

       \begin{scope}[xshift=\thirdxshift cm, yshift=0cm]
               \drawrodbeam{\beamheight}
       \end{scope}

       \foreach \loadpos in { 0, 0.3, ..., 1 }
               \draw [unitforce line]
                       ( \thirdxshift - \secondunitforcelength, \secondforceposition - \loadpos )
                       -- ( \thirdxshift, \secondforceposition - \loadpos ) ;

       \node at ( \thirdxshift - \secondunitforcelength, \secondforceposition - 0.5 )
               [left, inner sep=0, outer sep=3.2pt, blue] {$ 1 $} ;

\end{tikzpicture}
\vspace{-1.5em}\caption{}\label{fig:exampleofapplicationofreciprocalworktheorem}
\end{minipage}}
\end{wrapfigure}

\en{The reciprocal work theorem}\ru{Теорема взаимности работ}
\en{finds}\ru{находит}
\en{unexpected and effective}\ru{неожиданные и~эффективные}
\en{applications}\ru{применения}.
\en{For example}\ru{Для примера}\en{,}
\en{consider}\ru{рассмотрим}
\en{a~rod-beam}\ru{стержень-балку}\ru{,}
\en{clamped}\ru{защёмленную}
\en{at one end}\ru{на одн\'{о}м конц\'{е}}~(\inquotes{\en{cantilever}\ru{консольную}})
\en{and}\ru{и}
\en{bent}\ru{изгибаемую}
\en{by the two forces}\ru{двумя силами}
\en{with }\ru{с~}\en{the integral values}\ru{интегральными значениями}~$P_{\hspace{-0.25ex}\raisemath{-0.4ex}{1}}$
\en{and}\ru{и}~$P_{\hspace{-0.25ex}\raisemath{-0.4ex}{2}}$~(\figureref{fig:exampleofapplicationofreciprocalworktheorem}).
\en{While the~linear theory is applied}\ru{Тогда как применяется линейная теория},
\en{displacements}\ru{перемещения}-\en{deflections}\ru{пр\'{о}гибы}
\en{can be represented as}\ru{могут быть представлены как}

\nopagebreak\vspace{-0.1em}\begin{equation*}\begin{array}{c}
u_1 \hspace{-0.22ex} =
    \alpha_{1\hspace{-0.1ex}1} P_{\hspace{-0.25ex}\raisemath{-0.4ex}{1}} \hspace{-0.2ex}
  + \alpha_{12} P_{\hspace{-0.25ex}\raisemath{-0.4ex}{2}}
\hspace{.2ex} ,
\\[.1em]
u_2 \hspace{-0.22ex} = \alpha_{21} P_{\hspace{-0.25ex}\raisemath{-0.4ex}{1}} \hspace{-0.2ex} + \alpha_{22} P_{\hspace{-0.25ex}\raisemath{-0.4ex}{2}}
\hspace{.2ex} .
\end{array}\end{equation*}

...

%%%\end{minipage}
