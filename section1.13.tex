\en{\section{Rotations via quaternions}}

\ru{\section{Повороты кватернионами}}

\label{section:quaternions}

\en{The~other way}\ru{Другой путь}
\en{to describe}\ru{описать}
\en{a~rotation}\ru{поворот~(вращение)}
\en{in 3\hbox{-}dimensional space}\ru{в~трёхмерном пространстве}\en{ is}\ru{\:---}
\en{using quaternions}\ru{использовать кватернионы}.
\en{They}\ru{Они}
\en{are very popular}\ru{очень популярны}
\en{for}\ru{для}
\en{computer graphics}\ru{компьютерной графики}.

\en{Quaternions}\ru{Кватернионы}
\en{were invented}\ru{были изобретены}
\en{by~}\href{https://en.wikipedia.org/wiki/William_Rowan_Hamilton}{William\ru{’ом} Rowan\ru{’ом} Hamilton\ru{’ом}}
\en{in}\ru{в}~1843\footnote{%
\href{https://www.maths.tcd.ie/pub/HistMath/People/Hamilton/OnQuat/OnQuat.pdf}{\emph{On Quaternions; or on a new System of Imaginaries in Algebra}}
\en{by~}\href{https://en.wikipedia.org/wiki/William_Rowan_Hamilton}{\boldauthor{William\ru{’а} Rowan\ru{’а} Hamilton\ru{’а}}}
\en{appeared}\ru{появилась}
\en{in 18 publications}\ru{в~18 публикациях}
\en{in}\ru{в~журнале}
\insinglequotes{The London, Edinburgh and Dublin Philosophical Magazine and Journal of Science}\en{\hbox{\hspace{-0.5ex},}}
\en{volumes}\ru{в~томах}
xxv\hbox{--}xxxvi\en{,}
3\en{rd}\ru{ей}
\en{Series}\ru{серии},
1844\hbox{--}1850.%
%%Each publication, including the last one, ended with the~words
%%\insinglequotes{To be continued}\hbox{\hspace{-0.5ex}.}
}\hbox{\hspace{-0.5ex}.}


Lorem ipsum ....


\begin{equation}\label{theformulaforquaternions}
i^2 \hspace{-0.1ex} = j^2 \hspace{-0.1ex} = k^2 \hspace{-0.1ex} = i \hspace{-0.1ex} j k = -1
\hspace{.2ex} .
\end{equation}

\begin{equation}\label{moredefinitionalequationsforquaternions}
ij = k = - \hspace{.1ex} ji \hspace{.1ex} , \hspace{.7em}
jk = i = - \hspace{.1ex} kj \hspace{.1ex} , \hspace{.7em}
k \hspace{.1ex} i = \hspace{-0.1ex} j = - \hspace{.2ex} ik \hspace{.1ex} ,
\end{equation}


...


