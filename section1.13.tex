\en{\section{Rotations via quaternions}}

\ru{\section{Повороты кватернионами}}

\label{section:quaternions}

\en{The~other way}\ru{Другой путь}
\en{to describe}\ru{описать}
\en{a~rotation}\ru{поворот~(вращение)}
\en{in 3\hbox{-}dimensional space}\ru{в~трёхмерном пространстве}\en{ is}\ru{\:---}
\en{using quaternions}\ru{использовать кватернионы}.
\en{They}\ru{Они}
\en{are very popular}\ru{очень популярны}
\en{for}\ru{для}
\en{computer graphics}\ru{компьютерной графики}.

\en{Quaternions}\ru{Кватернионы}
\en{were invented}\ru{были изобретены}
\en{by~}\href{https://en.wikipedia.org/wiki/William_Rowan_Hamilton}{William\ru{’ом} Rowan\ru{’ом} Hamilton\ru{’ом}}
\en{in}\ru{в}~1843\footnote{%
\href{https://www.maths.tcd.ie/pub/HistMath/People/Hamilton/OnQuat/OnQuat.pdf}{\emph{On Quaternions; or on a new System of Imaginaries in Algebra}}
\en{by~}\href{https://en.wikipedia.org/wiki/William_Rowan_Hamilton}{\boldauthor{William\ru{’а} Rowan\ru{’а} Hamilton\ru{’а}}}
\en{appeared}\ru{появилась}
\en{in 18 publications}\ru{в~18 публикациях}
\en{in}\ru{в~журнале}
\insinglequotes{The London, Edinburgh and Dublin Philosophical Magazine and Journal of Science}\en{\hbox{\hspace{-0.5ex},}}
\en{volumes}\ru{в~томах}
xxv\hbox{--}xxxvi\en{,}
3\en{rd}\ru{ей}
\en{series}\ru{серии},
1844\hbox{--}1850.%
%%Each publication, including the last one, ended with the~words
%%\insinglequotes{To be continued}\hbox{\hspace{-0.5ex}.}
}\hbox{\hspace{-0.5ex}.}

..........

\subsection*{\en{Complex numbers}\ru{Комплексные числа}}

\begin{gather*}
p,q \in \mathbb{C}
\\
\begin{array}{r@{\hspace{.22em}}c@{\hspace{.22em}}l}
p = & %%\scalebox{.8}{$\mathcolor{black!40}{(1\cdot)}\hspace{.4ex}$}
\exponentofimaginary{\varphi} & = \cosine \varphi + i \sine \varphi
\\
q = & %%\scalebox{.8}{$\mathcolor{black!40}{(1\cdot)}\hspace{.4ex}$}
\exponentofimaginary{\psi} & = \cosine \psi + i \sine \psi
\end{array}
\end{gather*}

the~composition of two rotations
\begin{equation*}
p \hspace{.1ex} q = \exponentofimaginary{\varphi} \exponentofimaginary{\psi} \hspace{-0.2ex} = \exponentofimaginary{( \varphi + \psi )}
\end{equation*}

\begin{align*}
\exponentofimaginary{(\varphi + \psi)} \hspace{-0.3ex}
= & \cosine ( \varphi {\hspace{.1ex} + \hspace{.1ex}} \psi ) + i \sine ( \varphi {\hspace{.1ex} + \hspace{.1ex}} \psi )
\\
%
\exponentofimaginary{\varphi} \exponentofimaginary{\psi} \hspace{-0.2ex}
= & \hspace{.5ex} ( \cosine \varphi + i \sine \varphi ) ( \cosine \psi + i \sine \psi )
\\
%
= & \hspace{.5ex} \begin{multlined}[t]
( \cosine \varphi \cosine \psi - \sine \varphi \sine \psi )
\\
\hspace{2em} + i \hspace{.2ex} ( \sine \varphi \cosine \psi + \cosine \varphi \sine \psi )
\end{multlined}
\end{align*}

\noindent
from there also follows

\begin{equation*}
\begin{array}{r@{\hspace{.22em}}c@{\hspace{.25em}}l}
\cosine ( \varphi {\hspace{.1ex} + \hspace{.1ex}} \psi )
& = &
\cosine \varphi \cosine \psi - \sine \varphi \sine \psi
\\[.2em]
%
\sine ( \varphi {\hspace{.1ex} + \hspace{.1ex}} \psi )
& = &
\sine \varphi \cosine \psi + \cosine \varphi \sine \psi
\end{array}
\end{equation*}

.............

All rotations are \en{a~}\inquotes{\en{group}\ru{группа}}
\en{in the~sense}\ru{в~смысле}
\en{of~}\href{https://en.wikipedia.org/wiki/Group_theory}{\en{the~group theory}\ru{теории групп}}
\en{in }\ru{в~}\href{https://en.wikipedia.org/wiki/Abstract_algebra}{\en{abstract algebra}\ru{абстрактной алгебре}}.

\href{https://en.wikipedia.org/wiki/%C3%89variste_Galois}{Évariste Galois}

\href{https://en.wikipedia.org/wiki/Permutation_group}{permutations are group} too

\en{A~}\emph{\en{group}\ru{Группа}}~%
\{\hspace{.3ex}$\mathcal{G}$,\hspace{-0.4ex} \inquotes{$\circ\hspace{.1ex}$}\hspace{.3ex}\}
\en{is}\ru{это}
\en{a~set}\ru{множество}~$\mathcal{G}$
\en{with }\ru{с~}\en{a~binary operation}\ru{бинарной операцией}~\inquotes{$\circ\hspace{.1ex}$},
\en{where}\ru{где}
\textit{(i)}~\inquotes{$\circ\hspace{.1ex}$}~\en{is associative}\ru{ассоциативна},
${( h \circ g ) \circ \hspace{-0.1ex} f \hspace{-0.1ex} = h \circ ( g \circ \hspace{-0.1ex} f )}$
\en{for}\ru{для}
\en{any}\ru{любых}
${h, g, \hspace{-0.1ex} f \hspace{-0.1ex} \in \mathcal{G}}$,
\textit{(ii)}~$\mathcal{G}$~\en{has}\ru{имеет} \en{a~two\hbox{-}sided}\ru{двухсторонний} \en{identity element}\ru{тождественный элемент},
${\exists \hspace{.4ex} e {\hspace{.3ex} \in \hspace{.3ex}} \mathcal{G} \colon g \circ e = e \circ g = g \hspace{1ex} \forall g {\hspace{.2ex} \in \hspace{.2ex}} \mathcal{G}}$,
\en{and }\ru{и~}\textit{(iii)}~\en{every element}\ru{каждый элемент} \en{of~}$\mathcal{G}$ \en{has}\ru{имеет} \en{an~inverse}\ru{обратный} \en{element}\ru{элемент},
${\forall g {\hspace{.2ex} \in \hspace{.2ex}} \mathcal{G} \hspace{.8ex} \exists \hspace{.2ex} g^{\littleminusone} \colon g \circ g^{\littleminusone} \hspace{-0.25ex} = g^{\littleminusone} \hspace{-0.25ex} \circ g = e}$.

Group operation~\inquotes{$\circ\hspace{.1ex}$}
can be
commutative
or non-commutative.
%
A~group
with a~commutative operation
is called
\en{commutative group}\ru{коммутативная группа},
\en{or}\ru{или}
\en{abelian group}\ru{абелева группа}.
%
If there exists
at least one
pair of elements ${a, b \in \mathcal{G}}$
for which ${a \circ b \neq b \circ a}$,
then this group is non-abelian (non-commutative).

\en{A~}\emph{\en{field}\ru{Поле}}~%
\{\hspace{.3ex}$\mathcal{F}$\hspace{-0.2ex},\hspace{-0.4ex} \inquotes{$+\hspace{.1ex}$},\hspace{-0.4ex} \inquotes{$*\hspace{.1ex}$}\hspace{.3ex}\}
\en{is}\ru{это}
\en{a~set}\ru{множество}~$\mathcal{F}$
\en{with two}\ru{с~двумя}
\en{binary operations}\ru{бинарными операциями}
\inquotes{$+\hspace{.1ex}$}~and~\inquotes{$*\hspace{.1ex}$},
\en{when}\ru{когда}
\textit{(i)}~\{\hspace{.3ex}$\mathcal{F}$,\hspace{-0.4ex} \inquotes{$+\hspace{.1ex}$}\hspace{.3ex}\}\en{ is}\ru{\:---}
\en{commutative group}\ru{коммутативная группа},
${a + b = b + a}$ ${\forall a, \hspace{-0.2ex} b \hspace{-0.2ex} \in \hspace{-0.3ex} \mathcal{F}}$,
\en{with }\ru{с~}\inquotes{\en{additive}\ru{аддитивным}} \en{identity}\ru{тождественным}
${e_{0} {\hspace{.2ex} \in \hspace{.3ex}} \mathcal{F}}$,
\textit{(ii)}~\{\hspace{.3ex}$\mathcal{F} \hspace{-0.3ex} \setminus \hspace{-0.5ex} e_{0}$,\hspace{-0.4ex} \inquotes{$*\hspace{.1ex}$}\hspace{.3ex}\}\en{ is}\ru{\:---}
\en{commutative group}\ru{коммутативная группа},
${a * b = b * a}$ ${\forall a, \hspace{-0.2ex} b \hspace{-0.2ex} \in \hspace{-0.3ex} \mathcal{F} \hspace{-0.3ex} \setminus \hspace{-0.5ex} e_{0}}$,
\en{with }\ru{с~}\inquotes{\en{multiplicative}\ru{мультипликативным}} \en{identity}\ru{тождественным}
${e_{1} {\hspace{.2ex} \in \hspace{.3ex}} \mathcal{F} \hspace{-0.3ex} \setminus \hspace{-0.5ex} e_{0}}$
(${e_{1} \hspace{-0.3ex} \neq \hspace{-0.2ex} e_{0}}$),
${a, \hspace{-0.2ex} b \hspace{-0.2ex} \in \hspace{-0.3ex} \mathcal{F} \hspace{-0.3ex} \setminus \hspace{-0.5ex} e_{0}}$
$\Rightarrow$
${a * b \hspace{-0.2ex} \neq \hspace{-0.2ex} e_{0}}$
\en{for}\ru{для}~${\forall a {\hspace{.4ex} \neq \hspace{.4ex}} e_{0}}$
\en{and}\ru{и}~${\forall \hspace{.2ex} b {\hspace{.4ex} \neq \hspace{.4ex}} e_{0}}$
\textit{(iii)}~......


..............

Two imaginary units
${j \neq i}$

\begin{equation*}
( a + b \hspace{.1ex} i ) ( c + d \hspace{.1ex} j )
= ac + cb \hspace{.1ex} i + ad \hspace{.1ex} j + bd \hspace{.2ex} i \hspace{-0.1ex} j
\end{equation*}

By introducing
third
imaginary unit~$k$\::
${i \hspace{-0.1ex} j = k}$

\begin{equation}\label{theformulaforquaternions}
i^2 \hspace{-0.1ex} = j^2 \hspace{-0.1ex} = k^2 \hspace{-0.1ex} = i \hspace{-0.1ex} j k = -1
\hspace{.2ex} .
\end{equation}

\noindent
and making multiplication
not commutative but anticommutative

\begin{equation}\label{moredefinitionalequationsforquaternions}
i \hspace{-0.1ex} j = k = - \hspace{.1ex} ji \hspace{.1ex} , \hspace{.7em}
jk = i = - \hspace{.1ex} kj \hspace{.1ex} , \hspace{.7em}
k \hspace{.1ex} i = \hspace{-0.1ex} j = - \hspace{.2ex} ik \hspace{.1ex} ,
\end{equation}

\noindent
meet quaternions.

In fact there’s only two different imaginary units,
and a third one is there just for short notation (${k = i \hspace{-0.1ex} j}$),
say they’re only $i$~\en{and}\ru{и}~$j$, ${i^2 = j^2 = -1}$, ${i \neq j}$,
and a~quaternion is ${q = a + b \hspace{.1ex} i + cj + d \hspace{.2ex} i \hspace{-0.1ex} j}$,
with the really-easy-to-memorize equalities for multiplication
${j i \hspace{-0.1ex} j = i}$, ${i \hspace{-0.1ex} j i = j}$
and obvious
${i \hspace{-0.1ex} j \hspace{-0.1ex} j = -i}$, ${ii \hspace{-0.1ex} j = -j}$,
from there follows the anticommutativity itself,
${ji = -i \hspace{-0.1ex} j}$.

The following are absolutely the same,
just a result of denoting ${k = i \hspace{-0.1ex} j}$,
${i \hspace{-0.1ex} j i \hspace{-0.1ex} j = i \hspace{-0.1ex} j k = k^2}$
and the \inquotes{really-easy-to-memorize} are derivable as well
${i \hspace{-0.1ex} j i \hspace{-0.1ex} j = (i \hspace{-0.1ex} ji)j = j^2 = i(ji \hspace{-0.1ex} j) = i^2}$.

...


