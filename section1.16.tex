\newpage

\en{\section{In the oblique basis}}

\ru{\section{В~косоугольном базисе}}

\en{Until now}\ru{До~сих~пор}
\ru{использовался }\en{a~basis}\ru{базис}
\en{of the three}\ru{из трёх}
\en{mutually perpendicular}\ru{взаимно перпендикулярных}
\en{unit vectors}\ru{единичных векторов}
$\bm{e}_i$\en{ was used}.
\en{Now}\ru{Теперь}
\en{we will take}\ru{мы возьмём}
\en{a~basis}\ru{базис}
\en{of any three}\ru{из~любых трёх}
\en{linearly independent}\ru{линейно независимых}
(\en{non\hbox{-}coplanar}\ru{некомпланарных})
\en{vectors}\ru{векторов}
$\bm{a}_i$.

%%\begin{figure}[!htbp]
%%\begin{center}
\begin{wrapfigure}[20]{R}{.48\textwidth}
\makebox[.5\textwidth][c]{
\begin{minipage}[t]{.5\textwidth}


% some vector to draw
\pgfmathsetmacro{\lengthofvector}{2.66}
	\pgfmathsetmacro{\vectoranglefromz}{33}
	\pgfmathsetmacro{\vectoranglefromx}{59}

\tdplotsetmaincoords{33}{109} % orientation of camera

% vectors of basis
\pgfmathsetmacro{\firstlength}{0.8}
	\pgfmathsetmacro{\firstanglefromz}{90} % first and second are xy plane
	\pgfmathsetmacro{\firstanglefromx}{77} % first is not orthogonal to second
\pgfmathsetmacro{\secondlength}{1.1}
	\pgfmathsetmacro{\secondanglefromz}{90} % first and second are xy plane
	\pgfmathsetmacro{\secondanglefromx}{0}
\pgfmathsetmacro{\thirdlength}{1}
	\pgfmathsetmacro{\thirdanglefromz}{-15}
	\pgfmathsetmacro{\thirdanglefromx}{50}

\vspace{-0.9em}
\hspace{1.1em}\scalebox{.96}[.96]{%
\begin{tikzpicture}[scale=2.5, tdplot_main_coords]

	\coordinate (O) at (0,0,0);

	% draw axes and vectors of basis
	\tdplotsetcoord{A1}{\firstlength}{\firstanglefromz}{\firstanglefromx}
	\tdplotsetcoord{A2}{\secondlength}{\secondanglefromz}{\secondanglefromx}
	\tdplotsetcoord{A3}{\thirdlength}{\thirdanglefromz}{\thirdanglefromx}

	\draw [line width=0.4pt, blue] (O) -- ($ 2.2*(A1) $);
	\draw [line width=1.25pt, blue, -{Latex[round, length=3.6mm, width=2.4mm]}]
		(O) -- (A1)
		node[pos=0.93, above, inner sep=0pt, outer sep=6pt] {${\bm{a}}_1$};

	\draw [line width=0.4pt, blue] (O) -- ($ 1.08*(A2) $);
	\draw [line width=1.25pt, blue, -{Latex[round, length=3.6mm, width=2.4mm]}]
		(O) -- (A2)
		node[pos=0.97, above left, inner sep=0pt, outer sep=3.3pt] {${\bm{a}}_2$};

	\draw [line width=0.4pt, blue] (O) -- ($ 1.12*(A3) $);
	\draw [line width=1.25pt, blue, -{Latex[round, length=3.6mm, width=2.4mm]}]
		(O) -- (A3)
		node[pos=1.02, below right, inner sep=0pt, outer sep=7pt] {${\bm{a}}_3$};

	% define vector by sperical coordinates {length}{angle from z}{angle from x}
	% (plus it defines its projections)
	\tdplotsetcoord{V}{\lengthofvector}{\vectoranglefromz}{\vectoranglefromx}

	% get components of vector
	\coordinate (ParallelToThird) at ($ (V) - (A3) $);
	\coordinate (VcomponentXY) at (intersection of V--ParallelToThird and O--Vxy);

	\coordinate (ParallelToSecond) at ($ (VcomponentXY) - (A2xy) $);
	\coordinate (ParallelToFirst) at ($ (VcomponentXY) - (A1xy) $);

	\coordinate (Vcomponent1) at (intersection of VcomponentXY--ParallelToSecond and O--A1);
	\coordinate (Vcomponent2) at (intersection of VcomponentXY--ParallelToFirst and O--A2);

	\draw [line width=0.4pt, dotted, color=black] (O) -- (VcomponentXY); % projection on first & second vectors’ plane
	\draw [line width=0.4pt, dotted, color=black] (V) -- (VcomponentXY);
	\draw [line width=0.4pt, dotted, color=black] (VcomponentXY) -- (Vcomponent1);
	\draw [line width=0.4pt, dotted, color=black] (VcomponentXY) -- (Vcomponent2);

	% draw parallelopiped
	\coordinate (onPlane23) at ($ (Vcomponent2) + (V) - (VcomponentXY) $);
	\draw [line width=0.4pt, dotted, color=black] (Vcomponent2) -- (onPlane23);
	\draw [line width=0.4pt, dotted, color=black] (V) -- (onPlane23);

	\coordinate (onPlane13) at ($ (Vcomponent1) + (V) - (VcomponentXY) $);
	\draw [line width=0.4pt, dotted, color=black] (Vcomponent1) -- (onPlane13);
	\draw [line width=0.4pt, dotted, color=black] (V) -- (onPlane13);

	\coordinate (onAxis3) at ($ (V) - (VcomponentXY) $);
	\draw [line width=0.4pt, dotted, color=black] (O) -- (onAxis3);
	\draw [line width=0.4pt, dotted, color=black] (onPlane13) -- (onAxis3);
	\draw [line width=0.4pt, dotted, color=black] (onPlane23) -- (onAxis3);

	\draw [line width=0.4pt, dotted, color=black] (O) -- (onPlane13);
	\draw [line width=0.4pt, dotted, color=black] (O) -- (onPlane23);

	% draw components of vector
	\draw [color=black, line width=1.6pt, line cap=round, dash pattern=on 0pt off 1.6\pgflinewidth,
		-{Stealth[round, length=4mm, width=2.4mm]}]
		(O) -- (Vcomponent1)
		node[pos=0.6, below, shape=circle, fill=white, inner sep=-2pt, outer sep=2pt] {${v^1 \hspace{-0.1ex} \bm{a}_1}$};

	\draw [color=black, line width=1.6pt, line cap=round, dash pattern=on 0pt off 1.6\pgflinewidth,
		-{Stealth[round, length=4mm, width=2.4mm]}]
		(Vcomponent1) -- (VcomponentXY)
		node[pos=0.4, below right, fill=white, shape=circle, inner sep=0pt, outer sep=4pt] {${v^2 \hspace{-0.1ex} \bm{a}_2}$};

	\draw [color=black, line width=1.6pt, line cap=round, dash pattern=on 0pt off 1.6\pgflinewidth,
		-{Stealth[round, length=4mm, width=2.4mm]}]
		(VcomponentXY) -- (V)
		node[pos=0.55, above right, shape=circle, fill=white, inner sep=0.3pt, outer sep=6.2pt] {${v^3 \hspace{-0.1ex} \bm{a}_3}$};

	% draw vector
	\draw [line width=1.6pt, black, -{Stealth[round, length=5mm, width=2.8mm]}]
		(O) -- (V)
		node[pos=0.68, above, fill=white, inner sep=1pt, outer sep=5pt] {\scalebox{1.2}[1.2]{${\bm{v}}$}};

\end{tikzpicture}}



\vspace{-1.6em}\caption{}\label{fig:ObliqueCoordinates}
\end{minipage}}
\end{wrapfigure}
%%\end{center}
%%\end{figure}\vspace{-1.5em}

\en{The decomposition}\ru{Декомпозиция~(разложение)}\
\en{of~vector}\ru{вектора}~$\bm{v}$
\en{in the basis}\ru{в~базисе}~${\bm{a}_i}$~(\figureref{fig:ObliqueCoordinates})
\en{is the linear combination}\ru{есть линейная комбинация}

\nopagebreak\vspace{-0.2em}\en{\vspace{-0.8em}}
\begin{equation}\label{decompositionbyobliquebasis}
\bm{v} = v^{i} \hspace{-0.1ex} \bm{a}_i
\hspace{.1ex} .
\end{equation}

{\small\setlength{\parindent}{0pt}
\begin{leftverticalbar}[oversize]
%% if an~index is repeated once at the upper (superscript) level and once at the lower (subscript) level in the~same monomial, it implies summation over this index
%% если индекс повторяется один раз на верхнем (надстрочном) уровне и один раз на нижнем (подстрочном) уровне в~том~же одночлене, это подразумевает суммирование по этому индексу
%%
\en{The~summation convention}\ru{Соглашение о~суммировании} \en{gains}\ru{обретает} \en{the~new conditions}\ru{новые положения}:
\en{a~summation index}\ru{индекс суммирования} \en{is repeated}\ru{повторяется} \en{at different levels}\ru{на разных уровнях} \en{of the~same monomial}\ru{того~же одночлена},
\en{and }\ru{а~}\en{a~free}\ru{свободный}
\en{index}\ru{индекс}
\en{stays}\ru{остаётся}
\en{at the~same height}\ru{на той~же высоте}
\en{in any part}\ru{в~любой части}
\en{of~the~expression}\ru{выражения}
(${a_i \hspace{-0.2ex} = \hspace{.1ex} b_{i\hspace{-0.1ex}j} c^{\hspace{.2ex}j}}$\ru{\:---}\en{ is} \en{correct}\ru{корректно},
${a_i \hspace{-0.2ex} = \hspace{.1ex} b_{kk}^{\hspace{.1ex}i}}$\ru{\:---}\en{ is} \en{wrong twice}\ru{дважды ошибочно}).
\end{leftverticalbar}
\par}

\begin{otherlanguage}{russian}

\vspace{-0.16em}
В~таком базисе уж\'{е}
${\bm{v} \dotp \bm{a}_i \hspace{-0.1ex} = \hspace{.1ex} v^{k} \bm{a}_k \hspace{-0.1ex} \dotp \bm{a}_i \neq\vspace{.2ex} v^{i}}$\hbox{\hspace{-0.25ex},}
ведь~тут ${\bm{a}_i \hspace{-0.1ex} \dotp \bm{a}_k \neq\vspace{.2ex} \delta_{ik}}$.

Дополним~же
\hbox{\en{basic}\ru{базис}}~${\bm{a}_i}$ ещё другой тройкой векторов~\hbox{${\bm{a}^{\hspace{-0.05ex}i}}$\hspace{-0.25ex},}
\hbox{называемых}
кобазисом или~взаимным базисом, чтобы

\nopagebreak\vspace{-0.3em}\begin{equation}\label{fundamentalpropertyofcobasis}
\begin{array}{c}
\bm{a}_i \dotp \bm{a}^{\hspace{.1ex}j} \hspace{-0.1ex} = \hspace{.1ex} \delta_i^{\hspace{.1ex}j} , \:\:
\bm{a}^{\hspace{-0.05ex}i} \hspace{-0.1ex} \dotp \bm{a}_j \hspace{-0.1ex} = \hspace{.1ex} \delta_{\hspace{-0.1ex}j}^{\hspace{.1ex}i}
\hspace{.16ex} ,
\\[.2em]
\UnitDyad = \bm{a}^{\hspace{-0.1ex}i} \hspace{-0.1ex} \bm{a}_i \hspace{-0.1ex} = \bm{a}_i \hspace{.1ex} \bm{a}^{\hspace{-0.1ex}i}
\hspace{-0.2ex} .
\end{array}
\end{equation}

\vspace{-0.1em}\noindent
Это\:--- основное свойство кобазиса.
Орто\-нормирован\-ный~(орто\-нормаль\-ный) базис может быть определён как совпад\'{а}ющий со~своим кобазисом: ${\bm{e}^{\hspace{.05ex}i} \hspace{-0.2ex} = \bm{e}_{i}}$.

%%\inquotes{В~декартовых координатах}, когда базис\:--- отронормальный правый:
%%\begin{itemize}
%%\item компоненты единичного~(\inquotes{метрического}) тензора\:--- символы Кронекера (дельта Кронекера),
%%\item компоненты (псевдо)тензора чётности перестановок (Л\'{е}ви\hbox{-\!}Чив\'{и}ты)\:--- символы Веблена
%%      c~\inquotes{+} для чётного и с~\inquotes{-} для нечётного числ\'{а} перестановок
%%     (c~\inquotes{+} для \inquotes{правой} и с~\inquotes{-} для \inquotes{левой} тройки базисных векторов)
%%\end{itemize}

\begin{comment} %%
\vspace{-0.5em}\[
\bm{a}_i \dotp \bm{a}^{\hspace{.1ex}j} \hspace{-0.1ex} = \hspace{-0.2ex}
\scalebox{0.8}[0.8]{$\left[ \begin{array}{ccc}
\bm{a}_1 \hspace{-0.1ex} \dotp \bm{a}^{\hspace{-0.1ex}1} & \bm{a}_1 \hspace{-0.1ex} \dotp \bm{a}^2 & \bm{a}_1 \hspace{-0.1ex} \dotp \bm{a}^3 \\
\bm{a}_2 \hspace{-0.1ex} \dotp \bm{a}^{\hspace{-0.1ex}1} & \bm{a}_2 \hspace{-0.1ex} \dotp \bm{a}^2 & \bm{a}_2 \hspace{-0.1ex} \dotp \bm{a}^3 \\
\bm{a}_3 \hspace{-0.1ex} \dotp \bm{a}^{\hspace{-0.1ex}1} & \bm{a}_3 \hspace{-0.1ex} \dotp \bm{a}^2 & \bm{a}_3 \hspace{-0.1ex} \dotp \bm{a}^3
\end{array} \right]$} \!=\!
\scalebox{0.8}[0.8]{$\left[ \begin{array}{ccc}
1 & 0 & 0 \\
0 & 1 & 0 \\
0 & 0 & 1
\end{array} \right]$} \!=
\hspace{.1ex} \delta_i^{\hspace{.1ex}j}
\]
\end{comment} %%

\en{For}\ru{Для},
\en{as example}\ru{к~примеру},
\en{the~first}\ru{первого}
\en{cobasis vector}\ru{вектора кобазиса}~${\bm{a}^{\hspace{-0.1ex}1}\hspace{-0.1ex}}$

\nopagebreak\vspace{-0.1em}\begin{equation*}
\scalebox{.9}{$\left\{\hspace{-0.16em}\begin{array}{l}
\bm{a}^{\hspace{-0.1ex}1} \hspace{-0.2ex} \dotp \bm{a}_{1} = 1 \\
\bm{a}^{\hspace{-0.1ex}1} \hspace{-0.2ex} \dotp \bm{a}_{2} = 0 \\
\bm{a}^{\hspace{-0.1ex}1} \hspace{-0.2ex} \dotp \bm{a}_{3} = 0 \\[.05em]
\end{array}\right.$}
\Rightarrow\hspace{.33em}
%
\scalebox{.92}{$\left\{\hspace{-0.12em}\begin{array}{l}
\bm{a}^{\hspace{-0.1ex}1} \hspace{-0.2ex} \dotp \hspace{.2ex} \bm{a}_{1} = 1 \\[.1em]
\gamma \hspace{.1ex} \bm{a}^{\hspace{-0.1ex}1} \hspace{-0.2ex} = \hspace{.1ex} \bm{a}_2 \hspace{-0.1ex} \times \bm{a}_3 \\[.08em]
\end{array}\right.$}
\Rightarrow\hspace{.33em}
%
\scalebox{.96}{$\left\{\hspace{-0.11em}\begin{array}{l}
\bm{a}^{\hspace{-0.1ex}1} \hspace{-0.2ex} =
\displaystyle \nicefrac{\scalebox{0.95}{$1$}\hspace{.1ex}}{\scalebox{1.02}{$\gamma$}} \hspace{.5ex}
\bm{a}_2 \hspace{-0.1ex} \times \bm{a}_3 \hspace{.1ex} \\[.08em]
\gamma = \hspace{.1ex} \bm{a}_2 \hspace{-0.1ex} \times \bm{a}_3 \hspace{.1ex} \dotp \hspace{.25ex} \bm{a}_{1} \\[.08em]
\end{array}\right.$}
\end{equation*}

\vspace{-0.1em}\noindent
Коэффициент~$\gamma$ получился равным (с~точностью до~знака для \inquotes{левой} тройки ${\bm{a}_i}$) объёму параллелепипеда, построенного на~векторах~$\bm{a}_i$.
\en{In}\ru{В}~\sectionref{section:crossproduct} \en{the~same volume}\ru{тот~же объём} \en{was presented as}\ru{был представлен как}~${\hspace{-.22ex}\sqrt{\hspace{-0.36ex}\mathstrut{\textsl{g}}}\hspace{.16ex}}$, \en{and this is}\ru{и~это} \en{not without reason}\ru{не без причины}, \en{because}\ru{ведь} \en{it coincides with the~square root of}\ru{он совпад\'{а}ет с~квадратным корнем из}~\href{https://en.wikipedia.org/wiki/Gramian_matrix}{\en{gramian}\ru{грамиана}}~\hbox{$\textsl{g} \hspace{.1ex} \hspace{.25ex} \equiv \hspace{.2ex} \operatorname{det} \textsl{g}_{i\hspace{-0.1ex}j}$\hspace{-0.12ex}}\:--- \en{determinant of}\ru{определителя} \en{the~symmetric}\ru{симметричной} \ru{матрицы }\hbox{\href{https://en.wikipedia.org/wiki/J\%C3\%B8rgen_Pedersen_Gram}{J.\,P.\:Gram}\ru{’а}}\en{ matrix}~${\textsl{g}_{i\hspace{-0.1ex}j} \hspace{-0.1ex} \equiv \hspace{.2ex} \bm{a}_i \dotp \bm{a}_j}$.

\noindent
${ \tikz[baseline=-1ex] \draw [line width=.5pt, color=black, fill=white] (0, 0) circle (.8ex);
\hspace{.6em} }$
\en{The~proof}\ru{Доказательство}
\en{resembles}\ru{напоминает}
\en{the~derivation}\ru{вывод}\en{ of}~\eqref{doublepermutationsymbols}.
\en{The~}\inquotes{\en{triple product}\ru{Тройное произведение}}
${\bm{a}_i \hspace{-0.25ex} \times \hspace{-0.25ex} \bm{a}_j \dotp \hspace{.1ex} \bm{a}_k}$
\en{in some}\ru{в~каком\hbox{-}нибудь}
\en{orthonormal basis}\ru{орто\-нормаль\-ном базисе}~${\bm{e}_i}$
\en{is computable}\ru{вычисл\'{и}мо}
\en{as}\ru{как}
\en{the~determinant}\ru{детерминант}
(\en{with}\ru{с}~\inquotes{$-$}
\en{for}\ru{для}
\en{a~}\inquotes{\en{left}\ru{левой}}
\en{triplet}\ru{тройки}~${\bm{a}_i}$)
\en{by the~rows}\ru{по~строкам}

\nopagebreak\vspace{-0.2em}
\begin{equation*}
\permutationsparitysymbols{i\hspace{-0.1ex}j\hspace{-0.1ex}k} \hspace{-0.1ex}
\equiv \hspace{.16ex} \bm{a}_i \hspace{-0.25ex} \times \hspace{-0.25ex} \bm{a}_j \dotp \hspace{.1ex} \bm{a}_k =
\hspace{.1ex} \pm \hspace{.1ex} \operatorname{det}\hspace{-0.1ex}
\scalebox{0.92}[0.92]{$\left[\hspace{-0.2ex}\begin{array}{c@{\hspace{.64em}}c@{\hspace{.64em}}c}
\bm{a}_i \narrowdotp\hspace{.12ex} \bm{e}_1 & \bm{a}_i \narrowdotp\hspace{.12ex} \bm{e}_2 & \bm{a}_i \narrowdotp\hspace{.12ex} \bm{e}_3 \\
\bm{a}_j \narrowdotp\hspace{.12ex} \bm{e}_1 & \bm{a}_j \narrowdotp\hspace{.12ex} \bm{e}_2 & \bm{a}_j \narrowdotp\hspace{.12ex} \bm{e}_3 \\
\bm{a}_k \narrowdotp\hspace{.12ex} \bm{e}_1 & \bm{a}_k \narrowdotp\hspace{.12ex} \bm{e}_2 & \bm{a}_k \narrowdotp\hspace{.12ex} \bm{e}_3
\end{array}\hspace{-0.12ex}\right]$}
\end{equation*}

\vspace{-0.2em}\noindent
\en{or}\ru{или}
\en{by the~columns}\ru{по~столбцам}

\nopagebreak\vspace{-0.4em}
\begin{equation*}
\permutationsparitysymbols{pqr} \hspace{-0.15ex}
\equiv \hspace{.16ex} \bm{a}_p \hspace{-0.25ex} \times \hspace{-0.25ex} \bm{a}_q \dotp \hspace{.1ex} \bm{a}_r =
\hspace{.1ex} \pm \hspace{.1ex} \operatorname{det}\hspace{-0.1ex}
\scalebox{0.92}[0.92]{$\left[\hspace{-0.2ex}\begin{array}{c@{\hspace{.64em}}c@{\hspace{.64em}}c}
\bm{a}_p \narrowdotp\hspace{.12ex} \bm{e}_1 & \bm{a}_q \narrowdotp\hspace{.12ex} \bm{e}_1 & \bm{a}_r \narrowdotp\hspace{.12ex} \bm{e}_1 \\
\bm{a}_p \narrowdotp\hspace{.12ex} \bm{e}_2 & \bm{a}_q \narrowdotp\hspace{.12ex} \bm{e}_2 & \bm{a}_r \narrowdotp\hspace{.12ex} \bm{e}_2 \\
\bm{a}_p \narrowdotp\hspace{.12ex} \bm{e}_3 & \bm{a}_q \narrowdotp\hspace{.12ex} \bm{e}_3 & \bm{a}_r \narrowdotp\hspace{.12ex} \bm{e}_3
\end{array}\hspace{-0.12ex}\right]$} .
\end{equation*}

\vspace{.1em}\noindent
Произведение определителей~${\permutationsparitysymbols{i\hspace{-0.1ex}j\hspace{-0.1ex}k} \permutationsparitysymbols{pqr}}$
равно
определителю произведения матриц,
\en{and}\ru{а}~\en{elements of~the~latter}\ru{элементы последнего}\ru{\:---}\en{ are}
\en{sums}\ru{суммы}
\en{like}\ru{вида}
${\bm{a}_i \hspace{-0.1ex} \dotp \bm{e}_s \bm{a}_p \hspace{-0.2ex} \dotp \bm{e}_s \hspace{-0.2ex}
= \bm{a}_i \hspace{-0.1ex} \dotp \bm{e}_s \bm{e}_s \hspace{-0.2ex} \dotp \bm{a}_p \hspace{-0.15ex}
= \bm{a}_i \hspace{-0.1ex} \dotp \hspace{-0.1ex} \UnitDyad \dotp \bm{a}_p \hspace{-0.15ex}
= \bm{a}_i \hspace{-0.1ex} \dotp \bm{a}_p}$,
в~результате

\nopagebreak\vspace{-0.2em}
\begin{equation*}
\permutationsparitysymbols{i\hspace{-0.1ex}j\hspace{-0.1ex}k} \permutationsparitysymbols{pqr} = \hspace{0.25ex}
\operatorname{det}\hspace{-0.1ex}
\scalebox{0.92}[0.92]{$\left[\hspace{-0.16ex}\begin{array}{c@{\hspace{.64em}}c@{\hspace{.64em}}c}
\bm{a}_i \narrowdotp\hspace{.12ex} \bm{a}_p & \bm{a}_i \narrowdotp\hspace{.12ex} \bm{a}_q & \bm{a}_i \narrowdotp\hspace{.12ex} \bm{a}_r \\
\bm{a}_j \narrowdotp\hspace{.12ex} \bm{a}_p & \bm{a}_j \narrowdotp\hspace{.12ex} \bm{a}_q & \bm{a}_j \narrowdotp\hspace{.12ex} \bm{a}_r \\
\bm{a}_k \narrowdotp\hspace{.12ex} \bm{a}_p & \bm{a}_k \narrowdotp\hspace{.12ex} \bm{a}_q & \bm{a}_k \narrowdotp\hspace{.12ex} \bm{a}_r
\end{array}\hspace{-0.12ex}\right]$} ;
\end{equation*}

\vspace{-0.1em} \noindent
${i \narroweq p \narroweq 1}$, ${j \narroweq q \narroweq 2}$, ${k \narroweq r \narroweq 3}$ ${\,\Rightarrow}$ ${\permutationsparitysymbols{123} \hspace{.2ex} \permutationsparitysymbols{123} \hspace{-0.15ex} = \underset{\raisebox{.15em}{\scalebox{.7}{$i$,$\hspace{.15ex}j$}}}{\operatorname{det}} \left( \bm{a}_i \dotp \bm{a}_j \right) \hspace{-0.1ex} = \underset{\raisebox{.15em}{\scalebox{.7}{$i$,$\hspace{.15ex}j$}}}{\operatorname{det}} \, \textsl{g}_{i\hspace{-0.1ex}j}}$.
${ \hspace{.6em}
\tikz[baseline=-0.6ex] \draw [color=black, fill=black] (0, 0) circle (.8ex); }$

\en{Representing}\ru{Представляя}~${\bm{a}^{\hspace{-0.1ex}1}\hspace{-0.1ex}}$ \en{and other}\ru{и~другие} \en{cobasis vectors}\ru{векторы кобазиса} \en{as the sum}\ru{как сумму}

\nopagebreak\vspace{.8em}\begin{equation*}
\begin{array}{l@{\hspace{.3em}}c@{\hspace{.36em}}r}
\pm \hspace{.33ex} 2 \hspace{.1ex} \scalebox{0.95}[0.96]{$\sqrt{\hspace{-0.36ex}\mathstrut{\textsl{g}}}$} \hspace{.5ex} \bm{a}^{\hspace{-0.1ex}1} & = & \bm{a}_2 \times \bm{a}_3 \hspace{.2ex} \tikzmark{BeginPlusToMinus} - \hspace{.2ex} \bm{a}_3 \times \bm{a}_2 \tikzmark{EndPlusToMinus}
\hspace{.2ex} ,
\end{array}
\end{equation*}%
\AddOverBrace[line width=.75pt][0,-0.2ex]{BeginPlusToMinus}{EndPlusToMinus}%
{${\scriptstyle {+ \hspace{.4ex} \bm{a}_2 \hspace{.1ex} \times \hspace{.2ex} \bm{a}_3}}$}

\vspace{-1.3em}\noindent
приходим к~общей формуле (с~\inquotes{$-$} для \inquotes{левой} тройки ${\bm{a}_i}$)

\nopagebreak\begin{equation}\label{basisvectorstocobasisvectors}
\bm{a}^{\hspace{-0.05ex}i} \hspace{-0.2ex}
= \displaystyle \pm \hspace{.2ex} \frac{\raisemath{-0.4ex}{1}}{2 \hspace{.1ex} \scalebox{0.95}[0.96]{$\sqrt{\hspace{-0.36ex}\mathstrut{\textsl{g}}}$}} \hspace{.5ex} e^{i\hspace{-0.1ex}j\hspace{-0.1ex}k} \hspace{.1ex} \bm{a}_j \hspace{-0.1ex} \times \bm{a}_k \hspace{.1ex},
\:\:
\sqrt{\hspace{-0.36ex}\mathstrut{\textsl{g}}} \hspace{.25ex}
\equiv \hspace{.15ex} \pm \hspace{.4ex} \bm{a}_1 \hspace{-0.2ex} \times \bm{a}_2 \hspace{.1ex} \dotp \hspace{.25ex} \bm{a}_3
> 0
\hspace{.2ex} .
\end{equation}

\vspace{-0.2em}\noindent
Здесь ${e^{i\hspace{-0.1ex}j\hspace{-0.1ex}k}}$ по\hbox{-}прежнему символы чётности перестановки (${\pm 1}$ или~$0$):
${e^{i\hspace{-0.1ex}j\hspace{-0.1ex}k} \hspace{-0.1ex} \equiv e_{i\hspace{-0.1ex}j\hspace{-0.1ex}k}}$.
Произведение~${\bm{a}_j \hspace{-0.1ex} \times \bm{a}_k = \hspace{.1ex} \permutationsparitysymbols{j\hspace{-0.1ex}kn} \hspace{.2ex} \bm{a}^n\hspace{-0.12ex}}$, компоненты тензора Л\'{е}ви\hbox{-\!}Чив\'{и}ты~${ \permutationsparitysymbols{j\hspace{-0.1ex}kn} \hspace{-0.2ex} = \pm \hspace{.33ex} e_{j\hspace{-0.1ex}kn} \hspace{-0.1ex} \sqrt{\hspace{-0.36ex}\mathstrut{\textsl{g}}} }$,
\en{and}\ru{а}~\en{by}\ru{по}~\eqref{doublepermutationscontracted}
${e^{i\hspace{-0.1ex}j\hspace{-0.1ex}k} e_{j\hspace{-0.1ex}kn} \hspace{-0.25ex} = 2 \hspace{.1ex} \delta_n^{\hspace{.1ex}i}}$.
\en{Thus}\ru{Так что}

\nopagebreak\vspace{-0.1em}\begin{equation*}\scalebox{0.96}[0.96]{$%
\begin{array}{l@{\hspace{.25em}}c@{\hspace{.33em}}r}
\bm{a}^{\hspace{-0.1ex}1} & = & \pm \hspace{.25ex} \displaystyle \nicefrac{\scalebox{0.95}{$1$}}{\hspace{-0.25ex}\sqrt{\hspace{-0.2ex}\scalebox{0.96}{$\mathstrut{\textsl{g}}$}}} \hspace{.2ex} \left( \hspace{.1ex} \bm{a}_2 \hspace{-0.1ex} \times \hspace{-0.1ex} \bm{a}_3 \hspace{.1ex} \right) \hspace{-0.3ex},
\end{array}
\begin{array}{l@{\hspace{.25em}}c@{\hspace{.33em}}r}
\bm{a}^2 & = & \pm \hspace{.25ex} \displaystyle \nicefrac{\scalebox{0.95}{$1$}}{\hspace{-0.25ex}\sqrt{\hspace{-0.2ex}\scalebox{0.96}{$\mathstrut{\textsl{g}}$}}} \hspace{.2ex} \left( \hspace{.1ex} \bm{a}_3 \hspace{-0.1ex} \times \hspace{-0.1ex} \bm{a}_1 \hspace{.1ex} \right) \hspace{-0.3ex},
\end{array}
\begin{array}{l@{\hspace{.25em}}c@{\hspace{.33em}}r}
\bm{a}^3 & = & \pm \hspace{.25ex} \displaystyle \nicefrac{\scalebox{0.95}{$1$}}{\hspace{-0.25ex}\sqrt{\hspace{-0.2ex}\scalebox{0.96}{$\mathstrut{\textsl{g}}$}}} \hspace{.2ex} \left( \hspace{.1ex} \bm{a}_1 \hspace{-0.16ex} \times \hspace{-0.1ex} \bm{a}_2 \hspace{.1ex} \right) \hspace{-0.28ex}.
\end{array}%
$}\end{equation*}

\begin{tcolorbox}
\small\setlength{\abovedisplayskip}{2pt}\setlength{\belowdisplayskip}{2pt}

\emph{Example.} Get cobasis for basis~$\bm{a}_i$ when
\[ \begin{array}{l}
\bm{a}_1 \hspace{-0.2ex} = \bm{e}_1 \hspace{-0.2ex} + \bm{e}_2 \hspace{.1ex} , \\
\bm{a}_2 \hspace{-0.2ex} = \bm{e}_1 \hspace{-0.2ex} + \bm{e}_3 \hspace{.1ex} , \\
\bm{a}_3 \hspace{-0.2ex} = \bm{e}_2 \hspace{-0.2ex} + \bm{e}_3 \hspace{.1ex} .
\end{array} \]

\[
\sqrt{\hspace{-0.36ex}\mathstrut{\textsl{g}}} \hspace{.32ex} =
- \hspace{.4ex} \bm{a}_1 \hspace{-0.2ex} \times \bm{a}_2 \hspace{.1ex} \dotp \hspace{.25ex} \bm{a}_3 \hspace{.2ex} =
- \operatorname{det}\hspace{-0.1ex}
\scalebox{0.92}[0.92]{$\left[\hspace{-0.16ex}\begin{array}{c@{\hspace{.64em}}c@{\hspace{.64em}}c}
1 & 1 & 0 \\
1 & 0 & 1 \\
0 & 1 & 1
\end{array}\hspace{-0.12ex}\right]$} \hspace{-0.5ex} = 2 \hspace{.25ex};
\]
\[
- \hspace{.4ex} \bm{a}_2 \hspace{-0.2ex} \times \bm{a}_3 = \operatorname{det}\hspace{-0.1ex}
\scalebox{0.92}[0.92]{$\left[\hspace{-0.16ex}\begin{array}{c@{\hspace{.6em}}c@{\hspace{.5em}}c}
1 & \bm{e}_1 & 0 \\
0 & \bm{e}_2 & 1 \\
1 & \bm{e}_3 & 1
\end{array}\hspace{-0.2ex}\right]$} \hspace{-0.5ex} = \bm{e}_1 \hspace{-0.2ex} + \bm{e}_2 \hspace{-0.2ex} - \bm{e}_3 \hspace{.1ex},
\]
\[
- \hspace{.4ex} \bm{a}_3 \hspace{-0.2ex} \times \bm{a}_1 = \operatorname{det}\hspace{-0.1ex}
\scalebox{0.92}[0.92]{$\left[\hspace{-0.16ex}\begin{array}{c@{\hspace{.6em}}c@{\hspace{.5em}}c}
0 & \bm{e}_1 & 1 \\
1 & \bm{e}_2 & 1 \\
1 & \bm{e}_3 & 0
\end{array}\hspace{-0.2ex}\right]$} \hspace{-0.5ex} = \bm{e}_1 \hspace{-0.2ex} + \bm{e}_3 \hspace{-0.2ex} - \bm{e}_2 \hspace{.1ex},
\]
\[
- \hspace{.4ex} \bm{a}_1 \hspace{-0.2ex} \times \bm{a}_2 = \operatorname{det}\hspace{-0.1ex}
\scalebox{0.92}[0.92]{$\left[\hspace{-0.16ex}\begin{array}{c@{\hspace{.6em}}c@{\hspace{.5em}}c}
1 & \bm{e}_1 & 1 \\
1 & \bm{e}_2 & 0 \\
0 & \bm{e}_3 & 1
\end{array}\hspace{-0.2ex}\right]$} \hspace{-0.5ex} = \bm{e}_2 \hspace{-0.2ex} + \bm{e}_3 \hspace{-0.2ex} - \bm{e}_1 \hspace{.1ex}
\]

\vspace{-0.4em}and finally
\vspace{-0.4em}\[\begin{array}{l}
\bm{a}^1 \hspace{-0.2ex}=\hspace{.1ex} \smalldisplaystyleonehalf \hspace{-0.2ex} \left(^{\mathstrut} \bm{e}_1 \hspace{-0.2ex} + \bm{e}_2 \hspace{-0.2ex} - \bm{e}_3 \right)
\hspace{-0.5ex} ,
\\[.5em]
\bm{a}^2 \hspace{-0.2ex}=\hspace{.1ex} \smalldisplaystyleonehalf \hspace{-0.2ex} \left(^{\mathstrut} \bm{e}_1 \hspace{-0.2ex} - \bm{e}_2 \hspace{-0.2ex} + \bm{e}_3 \right)
\hspace{-0.5ex} ,
\\[.5em]
\bm{a}^3 \hspace{-0.2ex}=\hspace{.1ex} \smalldisplaystyleonehalf \hspace{-0.2ex} \left(^{\mathstrut} \hspace{-0.2ex} {- \bm{e}_1} \hspace{-0.2ex} + \bm{e}_2 \hspace{-0.2ex} + \bm{e}_3 \right)
\hspace{-0.5ex} .
\end{array}\]

\par\end{tcolorbox}

Имея кобазис, возможно не~только разложить по~нему любой вектор~(\figureref{fig:DecompositionOfVector}), но~и найти коэффициенты разложения~\eqref{decompositionbyobliquebasis}:
\begin{equation}\begin{array}{c}
\bm{v} = v^{i} \hspace{-0.1ex} \bm{a}_i = v_{i} \hspace{.1ex} \bm{a}^{\hspace{-0.05ex}i} \hspace{-0.25ex},
\\[.16em]
\bm{v} \dotp \bm{a}^{\hspace{-0.05ex}i} = v^{k} \hspace{-0.1ex} \bm{a}_k \hspace{-0.1ex} \dotp \bm{a}^{\hspace{-0.05ex}i} = v^{i} \hspace{-0.25ex}, \:\;
v_{i} \hspace{-0.1ex} = \bm{v} \dotp \bm{a}_i \hspace{.1ex} .
\end{array}\end{equation}
\noindent
Коэффициенты~${v_i}$ называются ко\-вариант\-ными компонентами вектора~$\bm{v}$, а~${v^i \hspace{-0.25ex}}$\:--- его контра\-вариант\-ными%
\footnote{Потому что они меняются обратно~(contra) изменению длин базисных векторов~${\bm{a}_i}$.}\hspace{-0.2ex}
компонентами.

Есть литература о~тензорах,
где introducing existænce
and различают ко\-вариант\-ные
и~контра\-вариант\-ные...
векторы~(\en{and}\ru{и}~\inquotes{\en{covectors}\ru{ковекторы}}, \inquotes{dual vectors}).
Не~ст\'{о}ит вводить читателя в~заблуждение:
\href{https://www.physicsforums.com/threads/is-a-vector-itself-contra-covariant-or-just-its-components.994318/}{вектор\hbox{-}то один и~тот~же}, просто разложение по~двум разным базисам даёт два набора компонент.

% ~ ~ ~ ~ ~
% converts spherical coordinates to cartesian
\newcommand{\tdsphericaltocartesian}[6]{%
\def\thecostheta{cos(#2)}%
\def\thesintheta{sin(#2)}%
\def\thecosphi{cos(#3)}%
\def\thesinphi{sin(#3)}%
\pgfmathsetmacro{#4}{ #1 * \thesintheta * \thecosphi }%
\pgfmathsetmacro{#5}{ #1 * \thesintheta * \thesinphi }%
\pgfmathsetmacro{#6}{ #1 * \thecostheta }%
}

% takes two points as cartesian {x}{y}{z} and calculates cross product of their location vectors
% placing the result into last three arguments
\newcommand{\tdcrossproductcartesian}[9]{%
\def\crossz{ #1 * #5 - #2 * #4 }%
\def\crossx{ #2 * #6 - #3 * #5 }%
\def\crossy{ #3 * #4 - #1 * #6 }%
\pgfmathsetmacro{#7}{\crossx}%
\pgfmathsetmacro{#8}{\crossy}%
\pgfmathsetmacro{#9}{\crossz}%
}

% takes two points as spherical {length}{anglefromz}{anglefromx} and calculates cross product of their location vectors
% placing the result as cartesian {x}{y}{z} into last three arguments
\newcommand{\tdcrossproductspherical}[9]{%
%
\tdplotsinandcos{\firstsintheta}{\firstcostheta}{#2}%
\tdplotsinandcos{\firstsinphi}{\firstcosphi}{#3}%
\def\firstx{ #1 * \firstsintheta * \firstcosphi }%
\def\firsty{ #1 * \firstsintheta * \firstsinphi }%
\def\firstz{ #1 * \firstcostheta }%
%
\tdplotsinandcos{\secondsintheta}{\secondcostheta}{#5}%
\tdplotsinandcos{\secondsinphi}{\secondcosphi}{#6}%
\def\secondx{ #4 * \secondsintheta * \secondcosphi }%
\def\secondy{ #4 * \secondsintheta * \secondsinphi }%
\def\secondz{ #4 * \secondcostheta }%
%
\def\crossz{ \firstx * \secondy - \firsty * \secondx }%
\def\crossx{ \firsty * \secondz - \firstz * \secondy }%
\def\crossy{ \firstz * \secondx - \firstx * \secondz }%
\pgfmathsetmacro{#7}{\crossx}%
\pgfmathsetmacro{#8}{\crossy}%
\pgfmathsetmacro{#9}{\crossz}%
}

% calculates dot product of location vectors of two 3D points specified by cartesian coordinates
\newcommand{\tddotproductcartesian}[7]{%
\edef\tddotproductcartesianxint{ \xinttheexpr round( #1 * #4 + #2 * #5 + #3 * #6 , 10 ) \relax }%
\pgfmathsetmacro{#7}{\tddotproductcartesianxint}%
}

% calculates dot product of location vectors of two 3D points specified by spherical coordinates
\newcommand{\tddotproductspherical}[7]{%
%
\tdplotsinandcos{\firstsintheta}{\firstcostheta}{#2}%
\tdplotsinandcos{\firstsinphi}{\firstcosphi}{#3}%
\def\firstx{ ( #1 * \firstsintheta * \firstcosphi ) }%
\def\firsty{ ( #1 * \firstsintheta * \firstsinphi ) }%
\def\firstz{ ( #1 * \firstcostheta ) }%
%
\tdplotsinandcos{\secondsintheta}{\secondcostheta}{#5}%
\tdplotsinandcos{\secondsinphi}{\secondcosphi}{#6}%
\def\secondx{ ( #4 * \secondsintheta * \secondcosphi ) }%
\def\secondy{ ( #4 * \secondsintheta * \secondsinphi ) }%
\def\secondz{ ( #4 * \secondcostheta ) }%
%
\edef\tddotproductsphericalxint{ \xinttheexpr round( \firstx * \secondx + \firsty * \secondy + \firstz * \secondz , 10 ) \relax }%
\pgfmathsetmacro{#7}{\tddotproductsphericalxint}%
}

% takes three points as spherical {length}{anglefromz}{anglefromx}
% and calculates triple product r1 × r2 • r3 of their location vectors
% the result is placed into \LastThreeDTripleProduct
\newcommand{\tdtripleproductspherical}[9]{%
%
\tdsphericaltocartesian{#1}{#2}{#3}{\firstx}{\firsty}{\firstz}
\tdsphericaltocartesian{#4}{#5}{#6}{\secondx}{\secondy}{\secondz}
\tdsphericaltocartesian{#7}{#8}{#9}{\thirdx}{\thirdy}{\thirdz}
%
\def\crossz{ ( \firstx * \secondy - \firsty * \secondx ) }%
\def\crossx{ ( \firsty * \secondz - \firstz * \secondy ) }%
\def\crossy{ ( \firstz * \secondx - \firstx * \secondz ) }%
%
\edef\LastThreeDTripleProduct{ \xinttheexpr round( \crossx * \thirdx + \crossy * \thirdy + \crossz * \thirdz , 10 ) \relax }%
}

% orientation of camera
\def\cameraTheta{36} \def\cameraPhi{98}
%% \def\cameraTheta{89.99} \def\cameraPhi{120}
	% 90 gives “You asked me to calculate `1/0.0', but I cannot divide any number by zero.”
\tdplotsetmaincoords{\cameraTheta}{\cameraPhi}

% vectors of basis
\pgfmathsetmacro{\firstlength}{0.69}
	\pgfmathsetmacro{\firstanglefromz}{71}
	\pgfmathsetmacro{\firstanglefromx}{-16}
\pgfmathsetmacro{\secondlength}{0.88}
	\pgfmathsetmacro{\secondanglefromz}{86}
	\pgfmathsetmacro{\secondanglefromx}{77}
\pgfmathsetmacro{\thirdlength}{0.96}
	\pgfmathsetmacro{\thirdanglefromz}{-19}
	\pgfmathsetmacro{\thirdanglefromx}{45}

\tdsphericaltocartesian%
	{\firstlength}{\firstanglefromz}{\firstanglefromx}%
	{\firstcartesianx}{\firstcartesiany}{\firstcartesianz}
\tdsphericaltocartesian%
	{\secondlength}{\secondanglefromz}{\secondanglefromx}%
	{\secondcartesianx}{\secondcartesiany}{\secondcartesianz}
\tdsphericaltocartesian%
	{\thirdlength}{\thirdanglefromz}{\thirdanglefromx}%
	{\thirdcartesianx}{\thirdcartesiany}{\thirdcartesianz}

% some but very important vector
\pgfmathsetmacro{\lengthofvector}{3.33}
	\pgfmathsetmacro{\vectoranglefromz}{33}
	\pgfmathsetmacro{\vectoranglefromx}{44}

\tdsphericaltocartesian%
	{\lengthofvector}{\vectoranglefromz}{\vectoranglefromx}%
	{\vectorcartesianx}{\vectorcartesiany}{\vectorcartesianz}

%%\begin{comment} %%
\begin{minipage}{\textwidth}
\hfill\[\scalebox{0.9}[0.9]{$\begin{array}{l@{\hspace{0.2\textwidth}}l@{\hspace{1.2em}}l@{\hspace{0.8em}}l}
\theta = \pgfmathprintnumber{\cameraTheta}\degree \hspace{0.8em}
\phi = \pgfmathprintnumber{\cameraPhi}\degree
& \scalebox{1.05}{$\bm{v}^{\varrho} = \pgfmathprintnumber{\lengthofvector}$} &
	\scalebox{1.05}{$\bm{v}^{\theta} = \pgfmathprintnumber{\vectoranglefromz}\degree$} &
	\scalebox{1.05}{$\bm{v}^{\phi} = \pgfmathprintnumber{\vectoranglefromx}\degree$} \\[0.25em]
%
& {\bm{a}_{1}^{\varrho} = \pgfmathprintnumber{\firstlength}} &
	{\bm{a}_{1}^{\theta} = \pgfmathprintnumber{\firstanglefromz}\degree} &
		{\bm{a}_{1}^{\phi} = \pgfmathprintnumber{\firstanglefromx}\degree} \\[0.1em]
& {\bm{a}_{2}^{\varrho} = \pgfmathprintnumber{\secondlength}} &
	{\bm{a}_{2}^{\theta} = \pgfmathprintnumber{\secondanglefromz}\degree} &
		{\bm{a}_{2}^{\phi} = \pgfmathprintnumber{\secondanglefromx}\degree} \\[0.1em]
& {\bm{a}_{3}^{\varrho} = \pgfmathprintnumber{\thirdlength}} &
	{\bm{a}_{3}^{\theta} = \pgfmathprintnumber{\thirdanglefromz}\degree} &
		{\bm{a}_{3}^{\phi} = \pgfmathprintnumber{\thirdanglefromx}\degree}
\end{array}$}\]
\end{minipage}
%%\end{comment} %%

\begin{figure}[!htbp]
\begin{center}

\vspace{0.1em}
\begin{tikzpicture}[scale=3.2, tdplot_main_coords] % tdplot_main_coords style to use 3dplot

	\coordinate (O) at (0,0,0);

	% define axes
	\tdplotsetcoord{A1}{\firstlength}{\firstanglefromz}{\firstanglefromx}
	\tdplotsetcoord{A2}{\secondlength}{\secondanglefromz}{\secondanglefromx}
	\tdplotsetcoord{A3}{\thirdlength}{\thirdanglefromz}{\thirdanglefromx}

	% define vector
	\tdplotsetcoord{V}{\lengthofvector}{\vectoranglefromz}{\vectoranglefromx} % {length}{angle from z}{angle from x}

	% square root of Gram matrix’ determinant is a1 × a2 • a3
	\tdtripleproductspherical%
		{\firstlength}{\firstanglefromz}{\firstanglefromx}%
		{\secondlength}{\secondanglefromz}{\secondanglefromx}%
		{\thirdlength}{\thirdanglefromz}{\thirdanglefromx}
	\edef\sqrtGramian{\xinttheexpr round( \LastThreeDTripleProduct, 10 )\relax}
	\edef\inverseOfSqrtGramian{\xinttheexpr round( 1 / \sqrtGramian, 10 )\relax}

	\node[fill=white!50, inner sep=0pt, outer sep=2pt] at (1.2,0,-1.45)
		{$\scalebox{0.9}{$\begin{array}{r}\bm{a}_1 \hspace{-0.4ex} \times \hspace{-0.3ex} \bm{a}_2 \dotp \hspace{0.2ex} \bm{a}_3 \hspace{-0.2ex} = \hspace{-0.2ex} \sqrt{\hspace{-0.36ex}\mathstrut{\textsl{g}}} \hspace{0.1ex} = \hspace{-0.2ex} \pgfmathprintnumber[fixed, precision=5]{\sqrtGramian} \\[0.25em]
		\displaystyle \nicefrac{\scalebox{0.95}{$1$}}{\hspace{-0.25ex}\sqrt{\hspace{-0.2ex}\scalebox{0.96}{$\mathstrut{\textsl{g}}$}}} \hspace{0.1ex} = \hspace{-0.2ex} \pgfmathprintnumber[fixed, precision=5]{\inverseOfSqrtGramian}\end{array}$}$};

	% calculate vectors of cobasis
	\tdcrossproductspherical%
		{\firstlength}{\firstanglefromz}{\firstanglefromx}%
		{\secondlength}{\secondanglefromz}{\secondanglefromx}%
		{\firstsecondcrossx}{\firstsecondcrossy}{\firstsecondcrossz}
	\coordinate (cross12) at (\firstsecondcrossx, \firstsecondcrossy, \firstsecondcrossz);
	\draw [line width=1.25pt, orange, -{Latex[round, length=3.6mm, width=2.4mm]}]
		(O) -- (cross12)
		node[pos=0.64, above right, inner sep=0pt, outer sep=6pt]
		{$\scalebox{0.8}{$\bm{a}_1 \hspace{-0.4ex} \times \hspace{-0.3ex} \bm{a}_2$}$};

	\tdcrossproductspherical%
		{\thirdlength}{\thirdanglefromz}{\thirdanglefromx}%
		{\firstlength}{\firstanglefromz}{\firstanglefromx}%
		{\thirdfirstcrossx}{\thirdfirstcrossy}{\thirdfirstcrossz}
	\coordinate (cross31) at (\thirdfirstcrossx, \thirdfirstcrossy, \thirdfirstcrossz);
	\draw [line width=1.25pt, orange, -{Latex[round, length=3.6mm, width=2.4mm]}]
		(O) -- (cross31)
		node[pos=0.86, above, inner sep=0pt, outer sep=5pt]
		{$\scalebox{0.8}{$\bm{a}_3 \hspace{-0.4ex} \times \hspace{-0.3ex} \bm{a}_1$}$};

	\tdcrossproductspherical%
		{\secondlength}{\secondanglefromz}{\secondanglefromx}%
		{\thirdlength}{\thirdanglefromz}{\thirdanglefromx}%
		{\secondthirdcrossx}{\secondthirdcrossy}{\secondthirdcrossz}
	\coordinate (cross23) at (\secondthirdcrossx, \secondthirdcrossy, \secondthirdcrossz);
	\draw [line width=1.25pt, orange, -{Latex[round, length=3.6mm, width=2.4mm]}]
		(O) -- (cross23)
		node[pos=0.88, below right, inner sep=0pt, outer sep=2.5pt]
		{$\scalebox{0.8}{$\bm{a}_2 \hspace{-0.4ex} \times \hspace{-0.3ex} \bm{a}_3$}$};

	\coordinate (coA3) at ($ \inverseOfSqrtGramian*(cross12) $);
	\coordinate (coA2) at ($ \inverseOfSqrtGramian*(cross31) $);
	\coordinate (coA1) at ($ \inverseOfSqrtGramian*(cross23) $);

	% get vector’s projection on a1 & a2 plane (third co-vector a^3 is normal to that plane)
	% it’s as deep down parallel to a3 as v^3 = v • a^3 in units of a3
	\tddotproductcartesian%
		{\vectorcartesianx}{\vectorcartesiany}{\vectorcartesianz}%
		{\inverseOfSqrtGramian*\firstsecondcrossx}%
			{\inverseOfSqrtGramian*\firstsecondcrossy}%
				{\inverseOfSqrtGramian*\firstsecondcrossz}%
		{\vectorthirdcoco}
	% get third co-component and translate it to vector’s head
	\coordinate (Vcomponent3) at ($ \vectorthirdcoco*(A3) $);
	\coordinate (VcomponentXY) at ($ (V) - (Vcomponent3) $);

	% decompose vector via initial basis
	\coordinate (ParallelToSecond) at ($ (VcomponentXY) - (A2) $);
	\coordinate (ParallelToFirst) at ($ (VcomponentXY) - (A1) $);
	\coordinate (Vcomponent1) at (intersection of VcomponentXY--ParallelToSecond and O--A1);
	\coordinate (Vcomponent2) at (intersection of VcomponentXY--ParallelToFirst and O--A2);

	\draw [line width=0.4pt, dotted, color=blue] (O) -- (VcomponentXY); % projection on first & second vectors’ plane

	\draw [line width=0.4pt, dotted, color=blue] (V) -- (VcomponentXY);
	\draw [line width=0.4pt, dotted, color=blue] (VcomponentXY) -- (Vcomponent1);
	\draw [line width=0.4pt, dotted, color=blue] (VcomponentXY) -- (Vcomponent2);

	% check a^1 × a^2 direction to be the same as a3
	\tdcrossproductcartesian%
		{\inverseOfSqrtGramian*\secondthirdcrossx}%
			{\inverseOfSqrtGramian*\secondthirdcrossy}%
				{\inverseOfSqrtGramian*\secondthirdcrossz}%
		{\inverseOfSqrtGramian*\thirdfirstcrossx}%
			{\inverseOfSqrtGramian*\thirdfirstcrossy}%
				{\inverseOfSqrtGramian*\thirdfirstcrossz}%
		{\CofirstCosecondOrthoX}{\CofirstCosecondOrthoY}{\CofirstCosecondOrthoZ}
	\coordinate (co1co2ortho) at (\CofirstCosecondOrthoX, \CofirstCosecondOrthoY, \CofirstCosecondOrthoZ);
	\draw [line width=1.25pt, blue!50, -{Latex[round, length=3.6mm, width=2.4mm]}]
		(O) -- ($ \sqrtGramian*(co1co2ortho) $);

	% length of a^3
	\tddotproductcartesian%
		{\inverseOfSqrtGramian*\thirdfirstcrossx}%
			{\inverseOfSqrtGramian*\thirdfirstcrossy}%
				{\inverseOfSqrtGramian*\thirdfirstcrossz}%
		{\inverseOfSqrtGramian*\thirdfirstcrossx}%
			{\inverseOfSqrtGramian*\thirdfirstcrossy}%
				{\inverseOfSqrtGramian*\thirdfirstcrossz}%
		{\squaredlengthofthirdcovector}
	%%\node[fill=white!50, inner sep=0pt, outer sep=4pt] at (0,0,-2.25)
		%%{$\scalebox{0.9}{$ | \hspace{0.1ex} \bm{a}^{\hspace{-0.1ex}3} \hspace{0.06ex} | \hspace{0.1ex} =
			%%\sqrt{\pgfmathprintnumber[fixed, precision=5]{\squaredlengthofthirdcovector}} $}$};

	% get vector’s projection on a^1 & a^2 plane (third basis vector a3 is normal to that plane)
	% it’s as deep down parallel to a^3 as v3 = v • a3 in units of a^3
	\tddotproductspherical%
		{\lengthofvector}{\vectoranglefromz}{\vectoranglefromx}%
		{\thirdlength}{\thirdanglefromz}{\thirdanglefromx}%
		{\vectorthirdcomponent}
	% get third co-component and translate it to vector’s head
	\coordinate (Vcoco3) at ($ \vectorthirdcomponent*(coA3) $);
	\coordinate (VcocoXY) at ($ (V) - (Vcoco3) $);

	% decompose vector via cobasis
	%%\coordinate (ParallelToCothird) at ($ (V) - (coA3) $);
	\coordinate (ParallelToCosecond) at ($ (VcocoXY) - (coA2) $);
	\coordinate (ParallelToCofirst) at ($ (VcocoXY) - (coA1) $);
	\coordinate (Vcoco1) at (intersection of VcocoXY--ParallelToCosecond and O--coA1);
	\coordinate (Vcoco2) at (intersection of VcocoXY--ParallelToCofirst and O--coA2);

	\draw [line width=0.4pt, red] (O) -- (Vcoco2);

	\draw [line width=0.4pt, dotted, color=red] (O) -- (VcocoXY);

	\draw [line width=0.4pt, dotted, color=red] (V) -- (VcocoXY);
	\draw [line width=0.4pt, dotted, color=red] (VcocoXY) -- (Vcoco1);
	\draw [line width=0.4pt, dotted, color=red] (VcocoXY) -- (Vcoco2);

	% draw parallelepiped of decomposition
	\coordinate (onPlane23) at ($ (Vcomponent2) + (V) - (VcomponentXY) $);
	\draw [line width=0.4pt, dotted, color=blue] (Vcomponent2) -- (onPlane23);
	\draw [line width=0.4pt, dotted, color=blue] (V) -- (onPlane23);

	\coordinate (onPlane13) at ($ (Vcomponent1) + (V) - (VcomponentXY) $);
	\draw [line width=0.4pt, dotted, color=blue] (Vcomponent1) -- (onPlane13);
	\draw [line width=0.4pt, dotted, color=blue] (V) -- (onPlane13);

	\coordinate (onAxis3) at ($ (V) - (VcomponentXY) $);
	\draw [line width=0.4pt, dotted, color=blue] (O) -- (onAxis3);
	\draw [line width=0.4pt, dotted, color=blue] (onPlane13) -- (onAxis3);
	\draw [line width=0.4pt, dotted, color=blue] (onPlane23) -- (onAxis3);

	\draw [line width=0.4pt, dotted, color=blue] (O) -- (onPlane13);
	\draw [line width=0.4pt, dotted, color=blue] (O) -- (onPlane23);

	% draw co-parallelepiped of co-decomposition
	\coordinate (onCoplane23) at ($ (Vcoco2) + (V) - (VcocoXY) $);
	\draw [line width=0.4pt, dotted, color=red] (Vcoco2) -- (onCoplane23);
	\draw [line width=0.4pt, dotted, color=red] (V) -- (onCoplane23);

	\coordinate (onCoplane13) at ($ (Vcoco1) + (V) - (VcocoXY) $);
	\draw [line width=0.4pt, dotted, color=red] (Vcoco1) -- (onCoplane13);
	\draw [line width=0.4pt, dotted, color=red] (V) -- (onCoplane13);

	\coordinate (onCoAxis3) at ($ (V) - (VcocoXY) $);
	\draw [line width=0.4pt, dotted, color=red] (O) -- (onCoAxis3);
	\draw [line width=0.4pt, dotted, color=red] (onCoplane13) -- (onCoAxis3);
	\draw [line width=0.4pt, dotted, color=red] (onCoplane23) -- (onCoAxis3);

	\draw [line width=0.4pt, dotted, color=red] (O) -- (onCoplane13);
	\draw [line width=0.4pt, dotted, color=red] (O) -- (onCoplane23);

	% draw vectors of cobasis
	\draw [line width=0.4pt, red] (O) -- ($ 1.01*(coA3) $);
	\draw [line width=1.25pt, red, -{Latex[round, length=3.6mm, width=2.4mm]}]
		(O) -- (coA3)
		node[pos=0.8, above right, shape=circle, fill=white, inner sep=-1pt, outer sep=11pt]
		{${\bm{a}}^{\hspace{-0.1ex}3}$};

	\draw [line width=0.4pt, red] (O) -- ($ 1.01*(coA2) $);
	\draw [line width=1.25pt, red, -{Latex[round, length=3.6mm, width=3mm]}]
		(O) -- (coA2)
		node[pos=0.88, above, shape=circle, fill=white, inner sep=-1pt, outer sep=4pt]
		{${\bm{a}}^{\hspace{-0.1ex}2}$};

	\draw [line width=0.4pt, red] (O) -- ($ 1.01*(coA1) $);
	\draw [line width=1.25pt, red, -{Latex[round, length=3.6mm, width=2.4mm]}]
		(O) -- (coA1)
		node[pos=0.92, below right, shape=circle, fill=white, inner sep=-1pt, outer sep=4pt]
		{${\bm{a}}^{\hspace{-0.16ex}1}$};

	% draw vectors of basis
	\draw [line width=0.4pt, blue] (O) -- ($ 1.01*(A1) $);
	\draw [line width=1.25pt, blue, -{Latex[round, length=3.6mm, width=2.4mm]}]
		(O) -- (A1)
		node[pos=0.84, above left, shape=circle, fill=white, inner sep=-1pt, outer sep=6pt]
		{${\bm{a}}_{\hspace{-0.08ex}1}$};

	\draw [line width=0.4pt, blue] (O) -- ($ 1.01*(A2) $);
	\draw [line width=1.25pt, blue, -{Latex[round, length=3.6mm, width=2.4mm]}]
		(O) -- (A2)
		node[pos=0.88, below, shape=circle, fill=white, inner sep=-1pt, outer sep=6pt]
		{${\bm{a}}_2$};
	\draw [line width=0.4pt, blue] (O) -- (Vcomponent2);

	\draw [line width=0.4pt, blue] (O) -- ($ 1.01*(A3) $);
	\draw [line width=1.25pt, blue, -{Latex[round, length=3.6mm, width=2.4mm]}]
		(O) -- (A3)
		node[pos=0.71, above left, shape=circle, fill=white, inner sep=-1pt, outer sep=16pt]
		{${\bm{a}}_3$};

	% draw components of vector
	\draw [color=blue!50!black, line width=1.6pt, line cap=round, dash pattern=on 0pt off 1.6\pgflinewidth,
		-{Stealth[round, length=4mm, width=2.4mm]}]
		(O) -- (Vcomponent1)
		node[pos=0.67, above left, fill=white, shape=circle, inner sep=0pt, outer sep=4pt]
	{${v^{\hspace{-0.08ex}1} \hspace{-0.1ex} \bm{a}_{\hspace{-0.08ex}1}}$};

	\draw [color=blue!50!black, line width=1.6pt, line cap=round, dash pattern=on 0pt off 1.6\pgflinewidth,
		-{Stealth[round, length=4mm, width=2.4mm]}]
		(Vcomponent1) -- (VcomponentXY)
		node[pos=0.48, above, shape=circle, fill=white, inner sep=-2pt, outer sep=1pt]
	{${v^2 \hspace{-0.1ex} \bm{a}_2}$};

	\draw [color=blue!50!black, line width=1.6pt, line cap=round, dash pattern=on 0pt off 1.6\pgflinewidth,
		-{Stealth[round, length=4mm, width=2.4mm]}]
		(VcomponentXY) -- (V)
		node[pos=0.77, above right, shape=circle, fill=white, inner sep=-1pt, outer sep=7pt]
	{${v^3 \hspace{-0.1ex} \bm{a}_3}$};

	% draw co-components of vector
	\draw [color=red!50!black, line width=1.6pt, line cap=round, dash pattern=on 0pt off 1.6\pgflinewidth,
		-{Stealth[round, length=4mm, width=2.4mm]}]
		(O) -- (Vcoco1)
		node[pos=1.02, above left, fill=white, shape=circle, inner sep=-1pt, outer sep=5pt]
	{${v_{\raisemath{-0.2ex}{1}} \bm{a}^{\hspace{-0.16ex}1}}$};

	\draw [color=red!50!black, line width=1.6pt, line cap=round, dash pattern=on 0pt off 1.6\pgflinewidth,
		-{Stealth[round, length=4mm, width=2.4mm]}]
		(Vcoco1) -- (VcocoXY)
		node[pos=0.5, below right, shape=circle, fill=white, inner sep=-2pt, outer sep=7pt]
	{${v_{\raisemath{-0.2ex}{2}}  \hspace{0.1ex} \bm{a}^{\hspace{-0.1ex}2}}$};

	\draw [color=red!50!black, line width=1.6pt, line cap=round, dash pattern=on 0pt off 1.6\pgflinewidth,
		-{Stealth[round, length=4mm, width=2.4mm]}]
		(VcocoXY) -- (V)
		node[pos=0.37, above left, shape=circle, fill=white, inner sep=-1pt, outer sep=9pt]
	{${v_{\raisemath{-0.2ex}{3}} \hspace{0.1ex} \bm{a}^{\hspace{-0.1ex}3}}$};

	% draw vector
	\draw [line width=1.6pt, black, -{Stealth[round, length=5mm, width=2.8mm]}]
		(O) -- (V)
		node[pos=0.69, above, shape=circle, fill=white, inner sep=0pt, outer sep=3.33pt]
			{\scalebox{1.2}[1.2]{${\bm{v}}$}};

	% calculate a_i • a^j
	\tddotproductcartesian%
		{\firstcartesianx}{\firstcartesiany}{\firstcartesianz}%
		{\inverseOfSqrtGramian*\secondthirdcrossx}%
			{\inverseOfSqrtGramian*\secondthirdcrossy}%
				{\inverseOfSqrtGramian*\secondthirdcrossz}%
		{\FirstDotCofirst}
	\tddotproductcartesian%
		{\secondcartesianx}{\secondcartesiany}{\secondcartesianz}%
		{\inverseOfSqrtGramian*\thirdfirstcrossx}%
			{\inverseOfSqrtGramian*\thirdfirstcrossy}%
				{\inverseOfSqrtGramian*\thirdfirstcrossz}%
		{\SecondDotCosecond}
	\tddotproductcartesian%
		{\thirdcartesianx}{\thirdcartesiany}{\thirdcartesianz}%
		{\inverseOfSqrtGramian*\firstsecondcrossx}%
			{\inverseOfSqrtGramian*\firstsecondcrossy}%
				{\inverseOfSqrtGramian*\firstsecondcrossz}%
		{\ThirdDotCothird}
	%
	\tddotproductcartesian%
		{\secondcartesianx}{\secondcartesiany}{\secondcartesianz}%
		{\inverseOfSqrtGramian*\secondthirdcrossx}%
			{\inverseOfSqrtGramian*\secondthirdcrossy}%
				{\inverseOfSqrtGramian*\secondthirdcrossz}%
		{\SecondDotCofirst}
	\tddotproductcartesian%
		{\firstcartesianx}{\firstcartesiany}{\firstcartesianz}%
		{\inverseOfSqrtGramian*\thirdfirstcrossx}%
			{\inverseOfSqrtGramian*\thirdfirstcrossy}%
				{\inverseOfSqrtGramian*\thirdfirstcrossz}%
		{\FirstDotCosecond}
	\tddotproductcartesian%
		{\thirdcartesianx}{\thirdcartesiany}{\thirdcartesianz}%
		{\inverseOfSqrtGramian*\secondthirdcrossx}%
			{\inverseOfSqrtGramian*\secondthirdcrossy}%
				{\inverseOfSqrtGramian*\secondthirdcrossz}%
		{\ThirdDotCofirst}
	\tddotproductcartesian%
		{\firstcartesianx}{\firstcartesiany}{\firstcartesianz}%
		{\inverseOfSqrtGramian*\firstsecondcrossx}%
			{\inverseOfSqrtGramian*\firstsecondcrossy}%
				{\inverseOfSqrtGramian*\firstsecondcrossz}%
		{\FirstDotCothird}
	\tddotproductcartesian%
		{\secondcartesianx}{\secondcartesiany}{\secondcartesianz}%
		{\inverseOfSqrtGramian*\firstsecondcrossx}%
			{\inverseOfSqrtGramian*\firstsecondcrossy}%
				{\inverseOfSqrtGramian*\firstsecondcrossz}%
		{\SecondDotCothird}
	\tddotproductcartesian%
		{\thirdcartesianx}{\thirdcartesiany}{\thirdcartesianz}%
		{\inverseOfSqrtGramian*\thirdfirstcrossx}%
			{\inverseOfSqrtGramian*\thirdfirstcrossy}%
				{\inverseOfSqrtGramian*\thirdfirstcrossz}%
		{\ThirdDotCosecond}

	% show a_i • a^j as matrix
	\node[fill=white!50, inner sep=0pt, outer sep=2pt] at (0,0.45,-3.8)
		{$\scalebox{0.9}[0.9]{$
			\bm{a}_i \dotp \hspace{0.1ex} \bm{a}^{\hspace{0.1ex}j} \hspace{-0.1ex} =
			\hspace{-0.2ex}\scalebox{0.9}[0.9]{$\left[ \begin{array}{ccc}
				\bm{a}_1 \hspace{-0.1ex} \dotp \bm{a}^{\hspace{-0.1ex}1} &
					\bm{a}_1 \hspace{-0.1ex} \dotp \bm{a}^{\hspace{-0.06ex}2} &
						\bm{a}_1 \hspace{-0.1ex} \dotp \bm{a}^{\hspace{-0.06ex}3} \\
				\bm{a}_2 \hspace{-0.1ex} \dotp \bm{a}^{\hspace{-0.1ex}1} &
					\bm{a}_2 \hspace{-0.1ex} \dotp \bm{a}^{\hspace{-0.06ex}2} &
						\bm{a}_2 \hspace{-0.1ex} \dotp \bm{a}^{\hspace{-0.06ex}3} \\
				\bm{a}_3 \hspace{-0.1ex} \dotp \bm{a}^{\hspace{-0.1ex}1} &
					\bm{a}_3 \hspace{-0.1ex} \dotp \bm{a}^{\hspace{-0.06ex}2} &
						\bm{a}_3 \hspace{-0.1ex} \dotp \bm{a}^{\hspace{-0.06ex}3}
			\end{array} \right]$} \!=\!
			\scalebox{0.9}[0.9]{$\left[ \begin{array}{ccc}
				\pgfmathprintnumber[fixed, precision=3]{\FirstDotCofirst} &
					\pgfmathprintnumber[fixed, precision=3]{\FirstDotCosecond} &
						\pgfmathprintnumber[fixed, precision=3]{\FirstDotCothird} \\
				\pgfmathprintnumber[fixed, precision=3]{\SecondDotCofirst} &
					\pgfmathprintnumber[fixed, precision=3]{\SecondDotCosecond} &
						\pgfmathprintnumber[fixed, precision=3]{\SecondDotCothird} \\
				\pgfmathprintnumber[fixed, precision=3]{\ThirdDotCofirst} &
					\pgfmathprintnumber[fixed, precision=3]{\ThirdDotCosecond} &
						\pgfmathprintnumber[fixed, precision=3]{\ThirdDotCothird}
			\end{array} \right]$}
			\hspace{-0.1em} = %%\approx
			\hspace{0.1ex} \delta_i^{\hspace{0.1ex}j}
		$}$};

\end{tikzpicture}

\vspace{0.2em}\caption{\inquotes{Decomposition of vector}}\label{fig:DecompositionOfVector}

\end{center}
\end{figure}

% ~ ~ ~ ~ ~

От~векторов перейдём к~тензорам второй сложности.
Имеем четыре комплекта диад:
${\bm{a}_i \hspace{.1ex} \bm{a}_j}$,
\hbox{$\bm{a}^{\hspace{-0.05ex}i} \hspace{-0.1ex} \bm{a}^{\hspace{0.05ex}j}$\hspace{-0.25ex},}
\hbox{$\bm{a}_i \hspace{.1ex} \bm{a}^{\hspace{0.05ex}j}$\hspace{-0.25ex},}
${\bm{a}^{\hspace{-0.05ex}i} \hspace{-0.1ex} \bm{a}_j}$.
Согласующиеся коэффициенты в~декомпозиции тензора называются его контра\-вариант\-ными, ко\-вариант\-ными и~смешан\-ными компонентами:
\vspace{.1em}\begin{equation}\begin{array}{c}
\somebivalenttensor \hspace{.1ex} =
\somebivalenttensorcomponentsupper{i\hspace{-0.1ex}j} \hspace{-0.1ex} \bm{a}_i \hspace{.1ex} \bm{a}_j \hspace{-0.05ex} =
\somebivalenttensorcomponents{i\hspace{-0.1ex}j} \hspace{.1ex} \bm{a}^{\hspace{-0.05ex}i} \hspace{-0.1ex} \bm{a}^{\hspace{0.05ex}j} \hspace{-0.15ex} =
\somebivalenttensorcomponents{\hspace{-0.2ex} \cdot j}^{\hspace{.1ex}i} \hspace{.1ex} \bm{a}_i \hspace{.1ex} \bm{a}^{\hspace{0.05ex}j} \hspace{-0.15ex} =
\somebivalenttensorcomponents{\hspace{-0.1ex}i}^{\hspace{-0.05ex} \cdot j} \hspace{-0.2ex} \bm{a}^{\hspace{-0.05ex}i} \hspace{-0.1ex} \bm{a}_j \hspace{.1ex}, \\[0.4em]
%
\somebivalenttensorcomponentsupper{i\hspace{-0.1ex}j} \hspace{-0.25ex} = \bm{a}^{\hspace{-0.05ex}i} \dotp \somebivalenttensor \dotp \hspace{.1ex} \bm{a}^{\hspace{.05ex}j} \hspace{-0.2ex}, \:\,
\somebivalenttensorcomponents{i\hspace{-0.1ex}j} = \bm{a}_i \dotp \somebivalenttensor \dotp \hspace{.1ex} \bm{a}_j \hspace{.1ex}, \\[0.25em]
%
\somebivalenttensorcomponents{\hspace{-0.2ex} \cdot j}^{\hspace{.1ex}i} = \bm{a}^{\hspace{-0.05ex}i} \dotp \somebivalenttensor \dotp \hspace{.1ex} \bm{a}_j \hspace{.1ex}, \:\,
\somebivalenttensorcomponents{\hspace{-0.1ex}i}^{\hspace{-0.05ex} \cdot j} \hspace{-0.2ex} = \bm{a}_i \dotp \somebivalenttensor \dotp \hspace{.1ex} \bm{a}^{\hspace{.1ex}j} \hspace{-0.1ex}.
\end{array}\end{equation}

\vspace{-0.1em}\noindent
Для~двух видов смешанных компонент точка в~индексе это просто свободное место:
у~${ \somebivalenttensorcomponents{\hspace{-0.2ex} \cdot j}^{\hspace{.1ex}i} }$
верхний индекс~\inquotesx{$i\hspace{.25ex}$}[---]
первый,
а~ниж\-ний~\inquotesx{$\hspace{-0.1ex}j\hspace{.25ex}$}[---]
второй.

Компоненты
единичного~(\inquotes{метрического})
тензора~$\UnitDyad$

\nopagebreak\vspace{-0.1em}\begin{equation}\begin{array}{c}
\UnitDyad = \bm{a}^{k} \hspace{-0.1ex} \bm{a}_{k} \hspace{-0.1ex} = \bm{a}_{k} \hspace{.1ex} \bm{a}^{k} \hspace{-0.2ex} = \textsl{g}_{j\hspace{-0.1ex}k} \hspace{.1ex} \bm{a}^{\hspace{.1ex}j} \hspace{-0.1ex} \bm{a}^{k} \hspace{-0.2ex} = \textsl{g}^{\hspace{0.25ex}j\hspace{-0.1ex}k} \hspace{-0.1ex} \bm{a}_j \bm{a}_k \hspace{-0.32ex}: \\[0.2em]
%
\bm{a}_i \dotp \UnitDyad \dotp \hspace{.1ex} \bm{a}^{\hspace{.1ex}j} \hspace{-0.2ex} = \bm{a}_i \hspace{.1ex} \dotp \hspace{.1ex} \bm{a}^{\hspace{.1ex}j} \hspace{-0.2ex} = \delta_i^{\hspace{.1ex}j} , \hspace{0.32em}
\bm{a}^{\hspace{-0.05ex}i} \dotp \UnitDyad \dotp \hspace{.1ex} \bm{a}_j = \bm{a}^{\hspace{-0.05ex}i} \hspace{-0.2ex} \dotp \hspace{.1ex} \bm{a}_j \hspace{-0.2ex} = \delta_{\hspace{-0.1ex}j}^{\hspace{.1ex}i} \hspace{0.2ex},
\\[0.2em]
%
\bm{a}_i \dotp \UnitDyad \dotp \hspace{.1ex} \bm{a}_j = \bm{a}_i \dotp \hspace{.1ex} \bm{a}_j \equiv \hspace{.16ex} \textsl{g}_{i\hspace{-0.1ex}j} \hspace{.1ex} , \hspace{0.32em}
\bm{a}^{\hspace{-0.05ex}i} \dotp \UnitDyad \dotp \hspace{.1ex} \bm{a}^{\hspace{.1ex}j} \hspace{-0.15ex} = \bm{a}^{\hspace{-0.05ex}i} \dotp \hspace{.1ex} \bm{a}^{\hspace{.1ex}j} \hspace{-0.15ex} \hspace{.1ex} \equiv \hspace{.16ex} \textsl{g}^{\hspace{0.2ex}i\hspace{-0.1ex}j} \hspace{.1ex} ;
\\[0.2em]
%
\scalebox{0.96}[0.97]{$\UnitDyad \dotp \hspace{-0.1ex} \UnitDyad = \textsl{g}_{i\hspace{-0.1ex}j} \hspace{.1ex} \bm{a}^{\hspace{-0.05ex}i} \hspace{-0.1ex} \bm{a}^{\hspace{0.05ex}j} \hspace{-0.1ex} \dotp \hspace{.1ex} \textsl{g}^{\hspace{0.2ex}nk} \bm{a}_n \bm{a}_k \hspace{-0.1ex} = \textsl{g}_{i\hspace{-0.1ex}j} \hspace{.1ex} \textsl{g}^{\hspace{0.2ex}j\hspace{-0.1ex}k} \bm{a}^{\hspace{-0.05ex}i} \bm{a}_k \hspace{-0.1ex} = \UnitDyad$}
\:\Rightarrow\: \textsl{g}_{i\hspace{-0.1ex}j} \hspace{.1ex} \textsl{g}^{\hspace{.2ex}j\hspace{-0.1ex}k} \hspace{-0.25ex} = \delta_i^{\hspace{0.05ex}k} \hspace{-0.1ex}.
\end{array}\end{equation}

\vspace{.1em}\noindent
Вдобавок к~\eqref{fundamentalpropertyofcobasis} и~\eqref{basisvectorstocobasisvectors} открылся ещё~один способ найти векторы кобазиса\:--- через матрицу~\hbox{$\textsl{g}^{\hspace{.2ex}i\hspace{-0.1ex}j}$\hspace{-0.3ex}}, обратную матрице Грама~${\textsl{g}_{i\hspace{-0.1ex}j}}$.
И~наоборот:

\nopagebreak\vspace{-0.12em}
\begin{equation}\begin{array}{c}
\bm{a}^{\hspace{-0.05ex}i} \hspace{-0.1ex}
= \UnitDyad \dotp \bm{a}^{\hspace{-0.05ex}i} \hspace{-0.15ex}
= \textsl{g}^{\hspace{.2ex}j\hspace{-0.1ex}k} \bm{a}_j \bm{a}_k \hspace{-0.1ex} \dotp \bm{a}^{\hspace{-0.05ex}i} \hspace{-0.15ex}
= \textsl{g}^{\hspace{.2ex}j\hspace{-0.1ex}k} \bm{a}_j \hspace{.16ex} \delta_{k}^{\hspace{.1ex}i}
= \textsl{g}^{\hspace{.2ex}j\hspace{-0.06ex}i} \bm{a}_j \hspace{.1ex} ,
\\[.25em]
%
\bm{a}_i
= \UnitDyad \dotp \bm{a}_i \hspace{-0.1ex}
= \textsl{g}_{j\hspace{-0.1ex}k} \hspace{.1ex} \bm{a}^{\hspace{.1ex}j} \hspace{-0.1ex} \bm{a}^{k} \hspace{-0.2ex} \dotp \bm{a}_i \hspace{-0.1ex}
= \textsl{g}_{j\hspace{-0.1ex}k} \hspace{.1ex} \bm{a}^{\hspace{.1ex}j} \hspace{.1ex} \delta_{i}^{k}
= \textsl{g}_{j\hspace{-0.06ex}i} \hspace{.16ex} \bm{a}^{\hspace{.1ex}j} \hspace{-0.2ex} .
\end{array}\end{equation}

\begin{tcolorbox}
\small\setlength{\abovedisplayskip}{2pt}\setlength{\belowdisplayskip}{2pt}

\emph{Example.} Using reversed Gram matrix, get cobasis for basis~$\bm{a}_i$ when
\[ \begin{array}{l}
\bm{a}_1 \hspace{-0.2ex} = \bm{e}_1 \hspace{-0.2ex} + \bm{e}_2 \hspace{.1ex} , \\
\bm{a}_2 \hspace{-0.2ex} = \bm{e}_1 \hspace{-0.2ex} + \bm{e}_3 \hspace{.1ex} , \\
\bm{a}_3 \hspace{-0.2ex} = \bm{e}_2 \hspace{-0.2ex} + \bm{e}_3 \hspace{.1ex} .
\end{array} \]

\[\begin{array}{c}
\textsl{g}_{i\hspace{-0.1ex}j} \hspace{-0.32ex} = \bm{a}_i \dotp \bm{a}_j \hspace{-0.12ex} = \hspace{-0.16ex}
\scalebox{0.92}[0.92]{$\left[\hspace{-0.16ex}\begin{array}{c@{\hspace{.64em}}c@{\hspace{.64em}}c}
2 & 1 & 1 \\
1 & 2 & 1 \\
1 & 1 & 2
\end{array}\hspace{-0.12ex}\right]$} \hspace{-0.2ex}, \:\:
\operatorname{det} \hspace{.12ex} \textsl{g}_{i\hspace{-0.1ex}j} \hspace{-0.25ex} = 4 \hspace{.16ex}, \\
%
\operatorname{adj} \hspace{.12ex} \textsl{g}_{i\hspace{-0.1ex}j} \hspace{-0.25ex} = \hspace{-0.16ex}
\scalebox{0.92}[0.92]{$\left[\hspace{-0.4ex}\begin{array}{r@{\hspace{.5em}}r@{\hspace{.5em}}r}
3 & -1 & -1 \\
-1 & 3 & -1 \\
-1 & -1 & 3
\end{array}\hspace{-0.12ex}\right]^{\hspace{-0.4ex}\scalebox{1.02}{$\T$}}$} \hspace{-0.8ex}, \\
%
\textsl{g}^{\hspace{.32ex}i\hspace{-0.1ex}j} \hspace{-0.25ex} = \textsl{g}_{i\hspace{-0.1ex}j}^{\hspace{.4ex}\expminusone} \hspace{-0.12ex} = \displaystyle \frac{\operatorname{adj} \hspace{.12ex} \textsl{g}_{i\hspace{-0.1ex}j}}{\operatorname{det} \hspace{.12ex} \textsl{g}_{i\hspace{-0.1ex}j}} =
\displaystyle \frac{1}{4} \hspace{.12ex}
\scalebox{0.92}[0.92]{$\left[\hspace{-0.4ex}\begin{array}{r@{\hspace{.5em}}r@{\hspace{.5em}}r}
3 & -1 & -1 \\
-1 & 3 & -1 \\
-1 & -1 & 3
\end{array}\hspace{-0.12ex}\right]^{\mathstrut}$} \hspace{-0.25ex}.
\end{array}\]

\vspace{-0.5em}Using ${\bm{a}^i = \textsl{g}^{\hspace{.32ex}i\hspace{-0.1ex}j} \bm{a}_j}$
\[\begin{array}{l}
\bm{a}^1 \hspace{-0.2ex}=\hspace{.1ex} \textsl{g}^{\hspace{.2ex}1\hspace{-0.1ex}1} \bm{a}_1 \hspace{-0.2ex} + \textsl{g}^{\hspace{.2ex}12} \bm{a}_2 \hspace{-0.2ex} + \textsl{g}^{\hspace{.2ex}13} \bm{a}_3 \hspace{-0.2ex} = \smalldisplaystyleonehalf \bm{e}_1 \hspace{-0.2ex} + \smalldisplaystyleonehalf \bm{e}_2 \hspace{-0.2ex} - \smalldisplaystyleonehalf \bm{e}_3
\hspace{.1ex} ,
\\[.5em]
%
\bm{a}^2 \hspace{-0.2ex}=\hspace{.1ex} \textsl{g}^{\hspace{.2ex}21} \bm{a}_1 \hspace{-0.2ex} + \textsl{g}^{\hspace{.2ex}22} \bm{a}_2 \hspace{-0.2ex} + \textsl{g}^{\hspace{.2ex}23} \bm{a}_3 \hspace{-0.2ex} = \smalldisplaystyleonehalf \bm{e}_1 \hspace{-0.2ex} - \smalldisplaystyleonehalf \bm{e}_2 \hspace{-0.2ex} + \smalldisplaystyleonehalf \bm{e}_3
\hspace{.1ex} ,
\\[.5em]
%
\bm{a}^3 \hspace{-0.2ex}=\hspace{.1ex} \textsl{g}^{\hspace{.2ex}31} \bm{a}_1 \hspace{-0.2ex} + \textsl{g}^{\hspace{.2ex}32} \bm{a}_2 \hspace{-0.2ex} + \textsl{g}^{\hspace{.2ex}33} \bm{a}_3 \hspace{-0.2ex} = - \smalldisplaystyleonehalf \bm{e}_1 \hspace{-0.2ex} + \smalldisplaystyleonehalf \bm{e}_2 \hspace{-0.2ex} + \smalldisplaystyleonehalf \bm{e}_3
\hspace{.16ex} .
\end{array}\]

\par\end{tcolorbox}

...


Единичный тензор
(unit tensor, identity tensor, metric tensor)

$\UnitDyad \dotp \hspace{.1ex} \bm{\xi} = \bm{\xi} \hspace{.1ex} \dotp \hspace{-0.12ex} \UnitDyad = \bm{\xi} \quad \forall \bm{\xi}$

$\UnitDyad \dotdotp \bm{a} \bm{b} = \bm{a} \bm{b} \dotdotp \hspace{-0.12ex} \UnitDyad = \bm{a} \dotp \UnitDyad \dotp \bm{b} = \bm{a} \dotp \bm{b}$

$\UnitDyad \dotdotp \bm{A} = \bm{A} \dotdotp \hspace{-0.12ex} \UnitDyad = \trace{\hspace{-0.1ex}\bm{A}}$

$\UnitDyad \dotdotp \hspace{-0.1ex} \bm{A} = \bm{A} \dotdotp \hspace{-0.12ex} \UnitDyad = \trace{\hspace{-0.1ex}\bm{A}} \neq \operatorname{not anymore} A_{j\hspace{-0.12ex}j}$

Thus for, say, trace of some tensor ${\bm{A} = A_{i\hspace{-0.1ex}j} \locationvector^i \locationvector^j}$: ${\bm{A} \dotdotp \hspace{-0.12ex} \UnitDyad = \trace{\hspace{-0.1ex}\bm{A}}}$, you have

${
\bm{A} \dotdotp \hspace{-0.12ex} \UnitDyad
= A_{i\hspace{-0.1ex}j} \locationvector^i \locationvector^j \hspace{-0.25ex} \dotdotp \locationvector_\differentialindex{k} \locationvector^k \hspace{-0.1ex}
= A_{i\hspace{-0.1ex}j} \locationvector^i \hspace{-0.25ex} \dotp \locationvector^j \hspace{-0.1ex}
= A_{i\hspace{-0.1ex}j} \textsl{g}^{\hspace{.33ex}i\hspace{-0.1ex}j}
}$


...

Тензор поворота
(the rotation tensor)

$\bm{P} \hspace{-0.2ex} = \hspace{-0.1ex} \bm{a}_i \hspace{.1ex} \mathcircabove{\bm{a}}^i \hspace{-0.2ex} = \hspace{-0.1ex} \bm{a}^{\hspace{-0.16ex}i} \mathcircabove{\bm{a}}_i \hspace{-0.2ex} = \hspace{-0.1ex} \bm{P}^{\expminusT}$

$\bm{P}^{\expminusone} \hspace{-0.4ex} = \hspace{-0.1ex} \mathcircabove{\bm{a}}_i \hspace{.1ex} \bm{a}^{\hspace{-0.16ex}i} \hspace{-0.2ex} = \hspace{-0.1ex} \mathcircabove{\bm{a}}^i \bm{a}_i \hspace{-0.2ex} = \hspace{-0.1ex} \bm{P}^{\T}$

$\bm{P}^{\T} \hspace{-0.4ex} = \hspace{-0.1ex} \mathcircabove{\bm{a}}^i \bm{a}_i \hspace{-0.2ex} = \hspace{-0.1ex} \mathcircabove{\bm{a}}_i \hspace{.1ex} \bm{a}^{\hspace{-0.16ex}i} \hspace{-0.2ex} = \hspace{-0.1ex} \bm{P}^{\expminusone}$

...



... Характеристическое уравнение~\eqref{chardetequation} быстро приводит к~тождеству К\kern-0.04em\'{э}ли\hbox{--}Г\kern-0.08em\'{а}мильтона~(Cayley\hbox{--}Hamilton)

\nopagebreak\vspace{-0.2em}\begin{equation}\label{cayley-hamilton equation}
\begin{array}{c}
-\bm{B} \hspace{-0.2ex} \dotp \hspace{-0.2ex} \bm{B} \hspace{-0.2ex} \dotp \hspace{-0.2ex} \bm{B}
+ \hspace{.1ex} \mathrm{I} \hspace{.2ex} \bm{B} \hspace{-0.2ex} \dotp \hspace{-0.2ex} \bm{B}
- \hspace{.1ex} \mathrm{II} \hspace{.2ex} \bm{B}
+ \hspace{.1ex} \mathrm{III} \hspace{.2ex} \UnitDyad
= \zerobivalent
\hspace{.1ex} ,
\\[.16em]
-\bm{B}^{3} \hspace{-0.2ex}
+ \hspace{.1ex} \mathrm{I} \hspace{.2ex} \bm{B}^{2} \hspace{-0.2ex}
- \hspace{.1ex} \mathrm{II} \hspace{.2ex} \bm{B}
+ \hspace{.1ex} \mathrm{III} \hspace{.2ex} \UnitDyad
= \zerobivalent
\hspace{.1ex} .
\end{array}
\end{equation}

\end{otherlanguage}
