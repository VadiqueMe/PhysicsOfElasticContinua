\en{\chapter{Nonlinear elastic momentless continuum}}

\ru{\chapter{Нелинейно\hbox{-\hspace{-0.2ex}}упругая безмоментная среда}}

\thispagestyle{empty}

\label{chapter:nonlinearcontinuum}

\en{\section{Continuum and the two approaches to describe it}}

\ru{\section{\ruContinuum{} и два подхода к описанию его}}

\label{section:continuum}

\dropcap{\en{A}\ru{С}}{\en{ccording}\ru{огласно}}
\en{to~}\href{https://en.wikipedia.org/wiki/Atomic_theory}{\en{the~atomic theory}\ru{атомной теории}},
\en{a~substance}\ru{вещество}
\en{is composed}\ru{состоит}
\en{of discrete particles}\ru{из дискретных частиц}\:---
\en{atoms}\ru{атомов}.
\en{Therefore}\ru{Поэтому}
\en{a~model}\ru{модель}
\en{of a~system of~particles}\ru{системы частиц}
\en{with }\ru{с~}\en{masses}\ru{массами}~${m_k}$
\en{and}\ru{и}~\en{location vectors}\ru{векторами положения}~${\currentlocationvector_k (t)}$
\en{may seem suitable}\ru{может показаться подходящей}
\en{yet despite}\ru{даже несмотр\'{я} на}
\en{an~unimaginable}\ru{невообразимое}
\en{number of degrees of freedom}\ru{число степеней свободы},
\en{because}\ru{так как}
\en{amounts of~memory}\ru{объёмы памяти}
\en{and}\ru{и}~\en{the~speed}\ru{быстрота}
\en{of~modern computers}\ru{современных компьютеров}
\en{are characterized}\ru{характеризуются}
\en{also}\ru{тоже}
\en{by }\en{astronomical numbers}\ru{астрономическими числами}.

\en{But anyway}\ru{Но~всё~же},
\en{maybe}\ru{может быть}\ru{,}
\en{it’s worth choosing}\ru{ст\'{о}ит выбрать}
\en{a~}\en{fundamentally}\ru{фундаментально}
\en{and}\ru{и}~\en{qualitatively}\ru{качественно}
\en{different model}\ru{иную модель}\:---
\en{a~model}\ru{модель}
\en{of the~material continuum}\ru{материального \rucontinuum{}а},
\en{or}\ru{или}
\en{the~continuous medium}\ru{сплошной среды},
\en{where}\ru{где}
\en{the~mass}\ru{масса}
\en{is distributed}\ru{распределена}
\en{continuously}\ru{континуально~(непрерывно)}
\en{within a~volume}\ru{в~объёме},
\en{and}\ru{и}
\en{the~finite volume}\ru{конечный объём}~$\mathcal{V}$
\en{contains}\ru{содержит}
\en{the~mass}\ru{массу}

\nopagebreak\vspace{-0.2em}
\begin{equation}\label{themassofthevolume}
m = \hspace{-0.5ex} \scalebox{.9}{$\displaystyle\integral\displaylimits_{\mathcal{V}}$} \hspace{-0.2ex} \rho \hspace{.25ex} d\mathcal{V}
, \hspace{.5em}
dm = \rho \hspace{.2ex} d \mathcal{V} ,
\end{equation}

\vspace{-0.25em}\noindent
\en{here}\ru{здесь}
${\rho\hspace{.1ex}}$\ru{\:---}\en{ is}
\en{the~volume(tric) mass density}\ru{объёмная плотность массы}
\en{and}\ru{и}~${d \mathcal{V}}$\ru{\:---}\en{ is}
\en{the~infinitesimal volume element}\ru{бесконечно\-малый элемент объёма}.

\en{A~real matter}\ru{Реальная материя}
\en{is modeled}\ru{моделируется}
\en{as}\ru{как}
\en{a~continuum}\ru{\rucontinuum},
\en{which can be thought of as}\ru{который может быть мыслим как}
\en{an~infinite set}\ru{бесконечное множество}
\en{of~vanishingly small}\ru{исчезающе м\'{а}лых}
\en{particles}\ru{частиц},
\en{joined together}\ru{соединённых вместе}.

\en{A~space of material points}\ru{Пространство материальных точек}\en{ is}\ru{\:---}
\en{only the first and simple}\ru{лишь первая и~простая}
\en{idea}\ru{идея}
\en{of a~continuous distribution}\ru{непрерывного распределения}
\en{of~mass}\ru{массы}.
\en{More complex models are possible too}\ru{Возможны более сложные модели}, \en{where}\ru{где} \en{particles}\ru{частицы} \en{have}\ru{имеют} \en{more degrees of freedom}\ru{больше степеней свободы}: \en{not only of~translation}\ru{не~только трансляции}, \en{but also}\ru{но также} \en{of~rotation}\ru{поворота}, \en{of~internal deformation}\ru{внутренней деформации}\en{,} \en{and others}\ru{и~другие}.
\en{Knowing that}\ru{Зная, что} \en{such models}\ru{такие модели} \en{are attracting}\ru{привлекают} \en{more and more interest}\ru{всё б\'{о}льший интерес}, \en{in this chapter}\ru{в~этой главе} \en{we will consider}\ru{мы рассмотрим} \en{the classical concept}\ru{классический концепт} \en{of a~continuous medium}\ru{о~сплошной среде} \en{as}\ru{как} \inquotesx{\en{made}\ru{сделанной} \en{of}\ru{из}~\en{simple points}\ru{простых точек}}[.]

\en{At~every moment of~time}\ru{В~каждый момент времени}~$t$\en{,}
\en{a~deformable continuum}\ru{деформируемый \rucontinuum}
\en{occupies}\ru{занимает}
\en{a~certain volume}\ru{некий объём}~${\mathcal{V}\hspace{.1ex}}$
\en{of the space}\ru{пространства}.
\en{This volume}\ru{Этот объём}
\en{moves}\ru{движется}
\en{and}\ru{и}~\en{deforms}\ru{деформируется},
\en{but}\ru{но}
\en{the~set of particles}\ru{набор частиц}
\en{inside this volume}\ru{внутри этого объёма}
\en{is constant}\ru{постоянен}.
\en{It is}\ru{Это}
\en{the~balance of~mass}\ru{баланс массы}
(\inquotes{\en{matter is neither created nor annihilated}\ru{материя не~создаётся и~не~аннигилируется}})

\nopagebreak\vspace{-0.2em}
\begin{equation}\label{thebalanceofmass}
dm \hspace{-0.15ex}
= \hspace{-0.2ex} \rho \hspace{.2ex} d\mathcal{V} \hspace{-0.2ex}
= \hspace{-0.2ex} \rho\hspace{.1ex}' \hspace{-0.2ex} d\mathcal{V}\hspace{.2ex}' \hspace{-0.5ex}
= \hspace{-0.2ex} \mathcircabove{\rho} \hspace{.22ex} d\mathcircabove{\mathcal{V}} ,
\hspace{.6em}
%
m = \hspace{-0.4ex} \scalebox{0.9}{$\displaystyle\integral\displaylimits_{\mathcal{V}}$} \hspace{-0.2ex} \rho \hspace{.2ex} d\mathcal{V} \hspace{-0.1ex}
= \hspace{-0.4ex} \scalebox{0.9}{$\displaystyle\integral\displaylimits_{\mathcal{V}\hspace{.12ex}'}$} \hspace{-0.2ex} \rho\hspace{.1ex}' \hspace{-0.2ex} d\mathcal{V}\hspace{.2ex}' \hspace{-0.4ex}
= \hspace{-0.4ex} \scalebox{0.9}{$\displaystyle\integral\displaylimits_{\mathcircabove{\mathcal{V}}}$} \hspace{-0.2ex} \mathcircabove{\rho} \hspace{.22ex} d\mathcircabove{\mathcal{V}} .
\vspace{-0.5em}\end{equation}

\en{Introducing}\ru{Вводя} \en{some}\ru{какие\hbox{-}либо} \en{variable parameters}\ru{переменные параметры}~${q^{\hspace{.1ex}i}\hspace{-0.3ex}}$\:--- \en{the curvilinear coordinates}\ru{криволинейные координаты}, \en{we have a~relation}\ru{имеем отношение} \en{for locations of particles}\ru{для положений частиц}

\nopagebreak\vspace{-0.2em}\begin{equation}\label{particlelocationbycoordinatesandtime}
\currentlocationvector \hspace{-0.4ex} = \hspace{-0.3ex} \currentlocationvector \hspace{.1ex} (q^{\hspace{.1ex}i} \hspace{-0.33ex}, t)
\hspace{.1ex} .
\end{equation}

...

\begin{otherlanguage}{russian}

% the material (or Lagrangian) description

\emph{\en{Material}\ru{Материальное} \en{description}\ru{описание}}

\textcolor{magenta}{\en{at the~initial moment}\ru{в~начальный момент}, \en{in~the~so\hbox{-}called}\ru{в~так называемой} \en{initial}\ru{начальной}~(\en{original}\ru{оригинальной}, \en{\sout{reference}}\ru{\sout{отсчётной}}, \inquotes{\en{material}\ru{материальной}}) \en{configuration}\ru{конфигурации}}

\en{at some initial moment}\ru{в~какой\hbox{-}то начальный момент} ${t \narroweq 0}$

\inquotes{запоминается} начальная~(\inquotes{материальная}) конфигурация\:---
locations in space of~particles at some arbitrarily chosen \inquotes{initial} moment ${t \narroweq 0}$

\nopagebreak\vspace{-0.1em}\begin{equation*}
\initiallocationvector \hspace{.1ex} (q^{\hspace{.1ex}i}) \hspace{-0.2ex}
\equiv
\locationvector \hspace{.1ex} (q^{\hspace{.1ex}i} \hspace{-0.33ex}, 0)
\end{equation*}

\en{Morphism}\ru{Морфизм}~(\en{function}\ru{функция})~${\initiallocationvector \hspace{-0.4ex} = \hspace{-0.3ex} \initiallocationvector(q^{i})}$

isomorphism (bijective mapping) (invertible one\hbox{-}to\hbox{-}one relation) (взаимно однозначное)

%%\en{material}\ru{материальное}~(\en{or}\ru{или} \inquotes{\en{Lagrangian}\ru{лагранжево}})

Subsequent locations in space of particles are then dependent variables\:--- functions of time and of the~initial~(material, \inquotes{Lagrangian}) \textcolor{magenta}{coordinates}/location~${\initiallocationvector}$

\nopagebreak\vspace{-1em}\begin{equation*}
\currentlocationvector \hspace{-0.4ex} = \hspace{-0.3ex} \currentlocationvector \hspace{.1ex} (\initiallocationvector, t)
\hspace{.1ex} .
\end{equation*}

\noindent
Для пространственного дифференцирования (постоянных во~времени) отношений \en{like}\ru{типа} ${\varphi \hspace{-0.4ex} = \hspace{-0.4ex} \varphi(\initiallocationvector)}$

вводится локальный касательный базис~${\initiallocationvector_\differentialindex{i}}$ и~взаимный базис~${\smash{\initiallocationvector^i}}$

\nopagebreak\vspace{-0.2em}\begin{equation*}
\initiallocationvector_\differentialindex{i} \hspace{-0.2ex} \equiv \partial_i \initiallocationvector
\hspace{.3em} \bigl( \partial_i \hspace{-0.1ex} \equiv \hspace{-0.1ex} \scalebox{0.8}{$ \displaystyle\frac{\raisemath{-0.2em}{\partial}}{\partial q^i} $} \hspace{.1ex} \bigr) \hspace{-0.1ex}
, \hspace{.5em}
\initiallocationvector^i \hspace{-0.3ex} \dotp \initiallocationvector_\differentialindex{\hspace{-0.1ex}j} \hspace{-0.3ex} = \hspace{-0.1ex} \delta^{i}_{\hspace{-0.2ex}j}
\end{equation*}

\inquotes{материальный} оператор Hamilton’а~${\hspace{-0.25ex}\smash{\boldnablacircled}}$

\nopagebreak\vspace{-0.2em}\begin{equation}
\UnitDyad = \initiallocationvector^i \initiallocationvector_\differentialindex{i} \hspace{-0.1ex} = \initiallocationvector^i \partial_i \initiallocationvector = \hspace{-0.25ex} \boldnablacircled \initiallocationvector
, \hspace{.5em}
\boldnablacircled \equiv \hspace{.1ex} \initiallocationvector^i \partial_i \hspace{.1ex} ,
\end{equation}

\vspace{-0.2em}\noindent
тогда ${d\varphi = d\initiallocationvector \dotp \hspace{-0.2ex} \boldnablacircled \varphi}$.

...

\end{otherlanguage}

% the spatial (or Eulerian) description
% introduced by d’Alembert

\en{But}\ru{Но} \ru{может быть эффективен }\en{yet another approach}\ru{ещё иной подход}\en{ may be effective}\:---
\emph{\en{the~spatial}\ru{пространственное}~(\en{or}\ru{или} \inquotes{\hbox{\en{Eulerian}\ru{эйлерово}}}) \en{description}\ru{опис\'{а}ние}},
\en{when}\ru{когда}
\en{instead of focusing on}\ru{вместо фокусирования на~том,} \en{how particles}\ru{как частицы} \en{of~a~continuum}\ru{\rucontinuum{}а} \en{move}\ru{движутся} \en{from the~initial configuration}\ru{из начальной конфигурации} \en{through space and~time}\ru{в~пространстве и~времени},
\en{processes}\ru{процессы} \en{are~considered}\ru{рассматриваются} \en{at fixed points in~space}\ru{в~неподвижных точках пространства} \en{as~time progresses}\ru{с~течением времени}.
\en{With relations}\ru{C~отношениями} \en{like}\ru{типа} ${\rho \hspace{-0.3ex} = \hspace{-0.44ex} \rho \hspace{.16ex} (\hspace{-0.1ex}\currentlocationvector, t)}$,
\en{we track}\ru{мы следим} \en{what’s happening}\ru{за про\-ис\-хо\-дя\-щим} \en{exactly in this place}\ru{именно в~этом месте}.
\en{Various particles}\ru{Разные частицы}, \en{continuously}\ru{непрерывно} \en{leaving and coming here}\ru{уходящие и~приходящие сюда}, \en{do not confuse us}\ru{не смущают нас}.

...

\en{the~balance of~mass}\ru{баланс массы} \en{in~spatial description}\ru{в~пространственном описании}
(\en{the~continuity equation}\ru{уравнение непрерывности~(сплошности, неразрывности)} \en{for}\ru{для} \en{mass}\ru{массы})

...

Jaumann derivative (\inquotes{\textcolor{magenta}{corotational} time derivative}) was first introduced by \href{https://en.wikipedia.org/wiki/Gustav_Jaumann}{Gustav Jaumann}\footnote{%
\bookauthor{Gustav Jaumann}. \href{http://www.physikdidaktik.uni-karlsruhe.de/download/jaumann_1911.pdf}{Geschlossenes System phy\-si\-ka\-li\-scher und che\-mi\-scher Differential\-gesetze (I.\;Mit\-teilung)~//~Sitzungs\-berichte der~Kaiser\-lichen Akademie der~Wissen\-schaften in~Wien, Mathematisch\hbox{-}natur\-wissen\-schaftliche Klasse, Abteilung~IIa, Band~CXX, 1911. Seiten 385\hbox{--}530.}}

\nopagebreak\begin{tcolorbox}[breakable, enhanced, colback = orange!8, before upper={\parindent3.2ex}, parbox = false]
\small%
\setlength{\abovedisplayskip}{2pt}\setlength{\belowdisplayskip}{2pt}%

Es sei ${\frac{\partial}{\partial t}}$ der~Operator der~\loosetexttr[80]{lokalen Fluxion}, d.\:i. der~partiellen Fluxion in einem gegen das~Koordinatensystem ruhenden Punkte des~Raumes. Ferner sei ${\frac{d}{dt}}$ der~Operator der~\loosetexttr[80]{totalen Fluxion}, welcher definiert wird durch

\nopagebreak\vspace{.1em}\begin{equation*}
\begin{array}{r@{\hspace{.66ex}}c@{\hspace{.75ex}}l}
\scalebox{0.92}[0.92]{$\displaystyle\frac{\raisemath{-0.125em}{da}}{dt}$} & = & \scalebox{0.92}[0.92]{$\displaystyle\frac{\raisemath{-0.125em}{\partial a}}{\partial t}$} + \bm{v} \dotp \hspace{-0.2ex} \boldnabla a
\hspace{.1ex} ,
\\[.8em]
\scalebox{0.92}[0.92]{$\displaystyle\frac{\raisemath{-0.125em}{d \bm{a}}}{dt}$} & \stackrel{3}{=} & \scalebox{0.92}[0.92]{$\displaystyle\frac{\raisemath{-0.125em}{\partial \bm{a}}}{\partial t}$} + \bm{v} \dotp \hspace{-0.2ex} \boldnabla\bm{;} \hspace{-0.26ex} \bm{a} - \hspace{.1ex} \smalldisplaystyleonehalf \bigl( \hspace{.1ex} \operatorname{rot}\hspace{.1ex} \bm{v} \hspace{.1ex} \bigr) \hspace{-0.4ex} \times \hspace{-0.2ex} \bm{a}
\hspace{.1ex} ,
\\[.8em]
\scalebox{0.92}[0.92]{$\displaystyle\frac{\raisemath{-0.125em}{d \mathboldalpha}}{dt}$} & \stackrel{9}{=} & \scalebox{0.92}[0.92]{$\displaystyle\frac{\raisemath{-0.125em}{\partial \mathboldalpha}}{\partial t}$} + \bm{v} \dotp \hspace{-0.2ex} \boldnabla\bm{;} \hspace{-0.28ex} \mathboldalpha - \hspace{.1ex} \smalldisplaystyleonehalf \bigl( \hspace{.1ex} \operatorname{rot}\hspace{.1ex} \bm{v} \hspace{-0.2ex} \times \hspace{-0.2ex} \mathboldalpha - \mathboldalpha \hspace{-0.2ex} \times \hspace{-0.2ex} \operatorname{rot}\hspace{.1ex} \bm{v} \hspace{.1ex} \bigr)
\hspace{-0.1ex} .
\end{array}
\end{equation*}

Endlich verwenden wir die~\loosetexttr[80]{körperliche Fluxion} eines Skalars:

\nopagebreak\vspace{-0.1em}\begin{equation*}
\scalebox{0.92}[0.92]{$\displaystyle\frac{\raisemath{-0.125em}{\delta}}{\delta t}$} \hspace{.15ex} a
= \scalebox{0.92}[0.92]{$\displaystyle\frac{\raisemath{-0.125em}{\partial}}{\partial t}$} \hspace{.15ex} a + \operatorname{div}\hspace{.1ex} a \bm{v}
= \scalebox{0.92}[0.92]{$\displaystyle\frac{\raisemath{-0.125em}{d}}{dt}$} \hspace{.15ex} a + a \operatorname{div}\hspace{.1ex} \bm{v}
\hspace{.1ex} .
\end{equation*}
\par
\end{tcolorbox}

\begin{otherlanguage}{russian}

körperliche\:--- bodily/телесная, material/вещественная(материальная), physical/физическая

${
\boldnabla \hspace{-0.1ex} \dotp \hspace{-0.15ex} \bigl( a \bm{v} \bigr) \hspace{-0.2ex} = \hspace{.1ex} a \hspace{.2ex} \boldnabla \hspace{-0.2ex} \dotp \hspace{-0.1ex} \bm{v} + \bm{v} \dotp \hspace{-0.2ex} \boldnabla a
}$

...

Пусть~${\upupsilon(\initiallocationvector, t)}$\:--- какое\hbox{-}либо поле \textcolor{magenta}{(?? только в~материальном описании от~${\initiallocationvector}$ ??)}.
Найдём скорость изменения интеграла по~объёму

\end{otherlanguage}

\nopagebreak\vspace{-0.25em}\begin{equation*}
\Upsilon
\equiv \hspace{-0.4ex} \scalebox{0.9}{$\displaystyle\integral\displaylimits_{\mathcal{V}}$} \hspace{-0.1ex} \rho \hspace{.19ex} \upupsilon \hspace{.11ex} d\mathcal{V}
\end{equation*}

\vspace{-0.4em}\noindent
(\inquotes{$\upupsilon$ \en{is}\ru{есть} $\Upsilon$ \en{per mass unit}\ru{на~единицу массы}}).
\en{Seemingly difficult}\ru{Кажущееся сложным}
\en{calculation of}\ru{вычисление}~${\mathdotabove{\Upsilon}}$
(\en{since}\ru{ведь}
$\mathcal{V}$
\en{is deforming}\ru{деформируется})
\en{is actually}\ru{на~с\'{а}мом деле}
\en{quite simple}\ru{весьма простое}
\en{with the}\ru{с}~%
\en{balance of~mass}\ru{балансом массы}~\eqref{thebalanceofmass}\::

\nopagebreak\vspace{-0.1em}\begin{equation}\label{rateofvolumeintegralchange}
\Upsilon = \hspace{-0.4ex} \scalebox{0.9}{$\displaystyle\integral\displaylimits_{\mathcircabove{\mathcal{V}}}$} \hspace{-0.1ex} \mathcircabove{\rho} \hspace{.19ex} \upupsilon \hspace{.11ex} d\mathcircabove{\mathcal{V}}
\;\Rightarrow\;
\mathdotabove{\Upsilon}
= \hspace{-0.4ex} \scalebox{0.9}{$\displaystyle\integral\displaylimits_{\mathcircabove{\mathcal{V}}}$} \hspace{-0.1ex} \mathcircabove{\rho} \hspace{.19ex} \mathdotabove{\upupsilon} \hspace{.11ex} d\mathcircabove{\mathcal{V}}
= \hspace{-0.4ex} \scalebox{0.9}{$\displaystyle\integral\displaylimits_{\mathcal{V}}$} \hspace{-0.1ex} \rho \hspace{.19ex} \mathdotabove{\upupsilon} \hspace{.11ex} d\mathcal{V} .
\end{equation}

\nopagebreak\begin{equation*}
\Psi = \hspace{-0.25ex}\scalebox{1.4}{$\integral$}_{\hspace{-0.5ex}\raisemath{-0.05em}{\mathcal{V}}} \hspace{.3ex} \rho \hspace{.2ex} \psi \hspace{.2ex} d\mathcal{V} \hspace{-0.1ex}
= \hspace{-0.2ex} \scalebox{1.4}{$\integral$}_{\hspace{-0.5ex}\raisemath{-0.05em}{\mathcal{V}\hspace{.12ex}'}} \hspace{.2ex} \rho \hspace{.1ex}' \psi \hspace{.2ex} d\mathcal{V}\hspace{.2ex}'
\hspace{.2ex} \Rightarrow \hspace{.33ex}
\mathdotabove{\Psi} = \hspace{-0.25ex}\scalebox{1.4}{$\integral$}_{\hspace{-0.5ex}\raisemath{-0.05em}{\mathcal{V}}} \hspace{.3ex} \rho \hspace{.2ex} \mathdotabove{\psi} \hspace{.2ex} d\mathcal{V} \hspace{-0.1ex}
= \hspace{-0.2ex} \scalebox{1.4}{$\integral$}_{\hspace{-0.5ex}\raisemath{-0.05em}{\mathcal{V}\hspace{.12ex}'}} \hspace{.2ex} \rho \hspace{.1ex}' \mathdotabove{\psi} \hspace{.2ex} d\mathcal{V}\hspace{.2ex}'
%%\hspace{-0.5ex} ,
\end{equation*}

...

\en{It is not worth it}\ru{Не~ст\'{о}ит} \en{to contrapose}\ru{противопоставлять} \en{the material}\ru{материальное} \en{and}\ru{и}~\en{the spatial}\ru{пространственное} \en{descriptions}\ru{описания}.
\en{In this book}\ru{В~этой книге} \en{both are used}\ru{используются оба}, \en{depending on the situation}\ru{в~зависимости от~ситуации}.

\newpage

\en{\section{Motion gradient}}

\ru{\section{Градиент движения}}

\label{section:motiongradient}

\en{Having}\ru{Имея}
\en{the motion function}\ru{функцию движения}~%
${\currentlocationvector \narroweq \hspace{-0.1ex} \currentlocationvector(q^{\hspace{.1ex}i} \hspace{-0.3ex}, t)}$,
${\initiallocationvector(q^{\hspace{.1ex}i}) \equiv \currentlocationvector(q^{\hspace{.1ex}i} \hspace{-0.3ex}, 0)}$,
\en{the }\ru{операторы }\inquotes{\en{nabla}\ru{набла}}\en{ operators}
${\boldnabla \equiv \currentlocationvector^{i} \partial_i}$,
${\boldnablacircled \equiv \initiallocationvector^{i} \partial_i}$
\en{and}\ru{и}
\en{looking}\ru{гл\'{я}дя}
\en{at differential relations}\ru{на дифференциальные отношения}
\en{for}\ru{для}
\en{a~certain}\ru{какого\hbox{-}либо}
\en{infinitesimal vector}\ru{бесконечномалого вектора}
\en{in two configurations}\ru{в~двух конфигурациях},
\en{the~current}\ru{текущей}
\en{with}\ru{с}~${d\currentlocationvector}$
\en{and}\ru{и}~\en{the~initial}\ru{начальной}
\en{with}\ru{с}~${d\initiallocationvector}$

\nopagebreak\en{\vspace{1.3em}}\ru{\vspace{2.1em}}
\begin{equation}
\begin{array}{c}
d\currentlocationvector = d\initiallocationvector \dotp \hspace{-0.2ex} \tikzmark{beginFtransposed} \boldnablacircled \currentlocationvector \tikzmark{endFtransposed} = \hspace{-0.2ex} \tikzmark{beginMotionGradient} \boldnablacircled \currentlocationvector^{\T} \tikzmark{endMotionGradient} \hspace{-0.44ex} \dotp d\initiallocationvector
\\[.2em]
%
d\initiallocationvector = d\currentlocationvector \dotp \hspace{-0.2ex} \tikzmark{beginFtransposedinverse} \boldnabla \initiallocationvector \tikzmark{endFtransposedinverse} = \hspace{-0.2ex} \tikzmark{beginFinverse} \boldnabla \initiallocationvector^{\T} \hspace{-0.4ex} \tikzmark{endFinverse} \dotp d\currentlocationvector
\end{array}
\end{equation}%
\AddOverBrace[line width=.75pt][0,0.6ex][yshift=.1em]%
{beginFtransposed}{endFtransposed}{${\begin{array}{c}
\hspace{.12em} \scalebox{0.85}{$\bm{F}^{\hspace{.1ex}\T}$}
\\[-0.33em]
\scriptstyle \initiallocationvector^{i} \hspace{-0.2ex} \currentlocationvector_\differentialindex{i}
\\[-0.36em]
\end{array}}$}
\AddOverBrace[line width=.75pt][0,0.6ex][yshift=.1em]%
{beginMotionGradient}{endMotionGradient}{${\begin{array}{c}
\scalebox{0.85}{$\bm{F}$}
\\[-0.33em]
\scriptstyle \currentlocationvector_\differentialindex{i} \initiallocationvector^{i}
\\[-0.36em]
\end{array}}$}
\AddUnderBrace[line width=.75pt][0,0.2ex][yshift=.2em]%
{beginFtransposedinverse}{endFtransposedinverse}{${\begin{array}{c}
\scriptstyle \currentlocationvector^{i} \initiallocationvector_\differentialindex{i}
\\[-0.1em]
\scalebox{0.85}{$\bm{F}^{\hspace{.1ex}\expminusT}$}
\end{array}}$}
\AddUnderBrace[line width=.75pt][0,0.2ex][yshift=.2em]%
{beginFinverse}{endFinverse}{${\begin{array}{c}
\scriptstyle \initiallocationvector_\differentialindex{i} \currentlocationvector^{i}
\\[-0.1em]
\scalebox{0.85}{$\bm{F}^{\hspace{.1ex}\expminusone}$}
\end{array}}$}

\vspace{1.5em}

\noindent
\en{here comes to~mind}\ru{приходит на~ум} \en{to introduce}\ru{ввести} \en{the }\inquotes{\en{motion gradient}\ru{градиент движения}}\footnote{%
\en{Tensor}\ru{Тензору}~$\bm{F}$ \en{doesn’t well suit}\ru{не~вполне подходит} \en{its}\ru{его} \en{more popular}\ru{более популярное} \en{name}\ru{название} \inquotesx{\en{deformation gradient}\ru{градиент деформации}}[,] \en{because}\ru{поскольку} \en{this tensor}\ru{этот тензор} \en{describes}\ru{описывает} \en{not only the~deformation itself}\ru{не~только сам\'{у} деформацию}, \en{but also the rotation of a~body as a~whole without de\-for\-ma\-tion}\ru{но~и поворот тела как~целого без деформации}.
}%
\hbox{\hspace{-0.6ex},} \en{picking}\ru{взяв} one of these tensor multipliers for it:
${\bm{F} \equiv \hspace{-0.2ex} \boldnablacircled \currentlocationvector^{\T} \hspace{-0.33ex} = \currentlocationvector_\differentialindex{i} \initiallocationvector^{i} \hspace{-0.4ex}}$.

\en{Why this one}\ru{Почему именно этот}?
\en{The~reason}\ru{Причина}
\en{to choose}\ru{выбрать}~${\hspace{-0.25ex}\boldnablacircled \currentlocationvector^{\T}\hspace{-0.2ex}}$\ru{\:---}\en{ is}
\en{another expression}\ru{другое выражение}
\en{for the~differential}\ru{для дифференциала}

\begin{gather*}
\begin{array}{c@{\hspace{2em}}c}
d\currentlocationvector = \scalebox{0.9}{$ \displaystyle \frac{\raisemath{-0.2em}{\partial \currentlocationvector}}{\partial \initiallocationvector} $} \dotp d\initiallocationvector
&
\bm{F} \hspace{-0.1ex} = \scalebox{0.9}{$ \displaystyle \frac{\raisemath{-0.2em}{\partial \currentlocationvector}}{\partial \initiallocationvector} $}
\end{array}
\\
%
\begin{array}{c@{\hspace{2em}}c}
d\initiallocationvector = \scalebox{0.9}{$ \displaystyle \frac{\raisemath{-0.2em}{\partial \initiallocationvector}}{\partial \currentlocationvector} $} \dotp d\currentlocationvector
&
\bm{F}^{\expminusone} \hspace{-0.3ex} = \scalebox{0.9}{$ \displaystyle \frac{\raisemath{-0.2em}{\partial \initiallocationvector}}{\partial \currentlocationvector} $}
\end{array}
\end{gather*}

\begin{equation*}
\scalebox{.9}{$ \displaystyle \frac{\raisemath{-0.2em}{\partial \bm{\zeta}}}{\partial \initiallocationvector} $} = \partial_i \bm{\zeta} \hspace{.1ex} \initiallocationvector^i
\hspace{2em}
\scalebox{.9}{$ \displaystyle \frac{\raisemath{-0.2em}{\partial \bm{\zeta}}}{\partial \currentlocationvector} $} = \partial_i \bm{\zeta} \currentlocationvector^{i}
\end{equation*}


....

\nopagebreak\vspace{-0.2em}\begin{equation*}
\UnitDyad
= \tikzmark{unitTensorAsOriginalDerivativeBegin} \hspace{-0.25ex} \boldnablacircled \initiallocationvector \tikzmark{unitTensorAsOriginalDerivativeEnd}
= \tikzmark{unitTensorAsCurrentDerivativeBegin} \hspace{-0.25ex} \boldnabla \currentlocationvector \tikzmark{unitTensorAsCurrentDerivativeEnd}
\end{equation*}%
\AddUnderBrace[line width=.75pt][0.1ex,0.2ex]%
{unitTensorAsOriginalDerivativeBegin}{unitTensorAsOriginalDerivativeEnd}%
{${ \scalebox{0.8}{$ \displaystyle \frac{\raisemath{-0.2em}{\partial \initiallocationvector}}{\partial \initiallocationvector} $} }$}%
\AddUnderBrace[line width=.75pt][0.1ex,0.2ex]%
{unitTensorAsCurrentDerivativeBegin}{unitTensorAsCurrentDerivativeEnd}%
{${ \scalebox{0.8}{$ \displaystyle \frac{\raisemath{-0.2em}{\partial \currentlocationvector}}{\partial \currentlocationvector} $} }$}

...

\en{For cartesian coordinates}\ru{Для декартовых координат} \en{with orthonormal basis}\ru{с~ортонормальным базисом} ${\bm{e}_i \hspace{-0.16ex} = \boldconstant}$

\nopagebreak\vspace{-0.2em}\begin{equation*}
\currentlocationvector = \hspace{-0.15ex} x_{i}(t) \hspace{.2ex} \bm{e}_i
\hspace{.1ex} , \:\;
\initiallocationvector = \hspace{-0.15ex} x_{i}(0) \hspace{.2ex} \bm{e}_i \hspace{-0.16ex} = \mathcircabove{x}_i \hspace{.1ex} \bm{e}_i
\hspace{.1ex} , \:\:
\mathcircabove{x}_i \hspace{-0.15ex} \equiv x_{i}(0)
\hspace{.1ex} ,
\end{equation*}

\nopagebreak\vspace{-0.25em}\begin{equation*}
\boldnablacircled \hspace{-0.1ex}
= \bm{e}_i \hspace{.2ex} \scalebox{0.9}{$ \displaystyle \frac{\raisemath{-0.2em}{\partial}}{\partial \mathcircabove{x}_i} $} \hspace{-0.1ex}
= \bm{e}_i \hspace{.15ex} \mathcircabove{\partial}_i
\hspace{.1ex} , \:\:
%
\boldnabla \hspace{-0.1ex}
= \bm{e}_i \hspace{.2ex} \scalebox{0.9}{$ \displaystyle \frac{\raisemath{-0.2em}{\partial}}{\partial x_i} $} \hspace{-0.1ex}
= \bm{e}_i \hspace{.15ex} \partial_i
\hspace{.1ex} ,
\end{equation*}

\nopagebreak\vspace{-0.4em}\begin{equation*}
\begin{array}{c}
\boldnablacircled \currentlocationvector
= \bm{e}_i \hspace{.2ex} \scalebox{.9}{$ \displaystyle \frac{\raisemath{-0.2em}{\partial \currentlocationvector}}{\partial \mathcircabove{x}_i} $} \hspace{-0.1ex}
= \bm{e}_i \hspace{.2ex} \scalebox{.9}{$ \displaystyle \frac{\raisemath{-0.2em}{\partial \hspace{.1ex} ( \hspace{-0.1ex} x_{\hspace{-0.1ex}j} \hspace{.2ex} \bm{e}_j )}}{\partial \mathcircabove{x}_i} $}
=  \hspace{.2ex} \scalebox{.9}{$ \displaystyle \frac{\raisemath{-0.2em}{\partial x_{\hspace{-0.1ex}j}}}{\partial \mathcircabove{x}_i} $} \hspace{.25ex} \bm{e}_i \bm{e}_{\hspace{-0.1ex}j} \hspace{-0.2ex}
= \mathcircabove{\partial}_i \hspace{.1ex} x_{\hspace{-0.1ex}j} \hspace{.1ex} \bm{e}_i \bm{e}_{\hspace{-0.1ex}j}
\hspace{.1ex} ,
\\[.66em]
%
\boldnabla \initiallocationvector
= \bm{e}_i \hspace{.2ex} \scalebox{.9}{$ \displaystyle \frac{\raisemath{-0.2em}{\partial \hspace{.1ex} \initiallocationvector}}{\partial x_{i}} $} \hspace{-0.1ex}
= \hspace{.2ex} \scalebox{.9}{$ \displaystyle \frac{\raisemath{-0.2em}{\partial \mathcircabove{x}_{\hspace{-0.1ex}j}}}{\partial x_{i}} $} \hspace{.25ex} \bm{e}_i \bm{e}_{\hspace{-0.1ex}j} \hspace{-0.2ex}
= \partial_i \hspace{.1ex} \mathcircabove{x}_{\hspace{-0.1ex}j} \hspace{.1ex} \bm{e}_i \bm{e}_{\hspace{-0.1ex}j}
\end{array}
\end{equation*}

...

\en{By the polar decomposition theorem}\ru{По теореме о~полярном разложении}~(\chapterdotsectionref{chapter:mathapparatus}{section:polardecomposition}), \en{the motion gradient}\ru{градиент движения} \en{decomposes into}\ru{разлож\'{и}м на} \en{the rotation tensor}\ru{тензор поворота}~$\rotationtensor$ \en{and}\ru{и}~\en{the symmetric}\ru{симметричные} \en{positive}\ru{положительные} \en{stretch tensors}\ru{тензоры искажений}~${\bm{U}\hspace{-0.25ex}}$ \en{and}\ru{и}~${\bm{V}\hspace{-0.1ex}}$:

\nopagebreak\vspace{-0.1em}\begin{equation*}
 \bm{F} \hspace{-0.1ex} = \rotationtensor \dotp \hspace{.25ex} \bm{U} \hspace{-0.2ex} = \bm{V} \hspace{-0.3ex} \dotp \hspace{.15ex} \rotationtensor
\end{equation*}

...

\en{When}\ru{Когда} \en{there's no rotation}\ru{поворота нет}~(${\rotationtensor = \hspace{-0.1ex} \UnitDyad \hspace{.1ex}}$), \en{then}\ru{тогда} ${\bm{F} \hspace{-0.1ex} = \hspace{.1ex} \bm{U} \hspace{-0.25ex} = \bm{V}\hspace{-0.3ex}}$.

...

\en{\section{Measures (tensors) of deformation}}

\ru{\section{Меры (тензоры) деформации}}

\label{section:deformationtensors}

\en{And this}\ru{А это}\en{ is}\ru{\:---} \en{where}\ru{где} \ru{возникает }\en{the extra complexity}\ru{сверх сложность}\en{ arose}.
Although, the multivariance is often seen as a big gift.

\en{Motion gradient}\ru{Градиент движения}~$\bm{F}$ \en{characterizes}\ru{характеризует}
\en{both the~deformation of a~body}\ru{и~деформацию тела,}
\en{and the~rotation of a~body as a~whole}\ru{и~поворот тела как~целого}.
%
\en{The deformation-only tensors}\ru{Тензорами лишь-деформации}
\en{are}\ru{являются}
\en{the stretch tensors}\ru{тензоры искажений}~${\bm{U}\hspace{-0.25ex}}$
\en{and}\ru{и}~${\bm{V}\hspace{-0.1ex}}$
\en{from the polar decomposition}\ru{из полярного разложения}
${\bm{F} \hspace{-0.1ex} = \rotationtensor \dotp \hspace{.25ex} \bm{U} \hspace{-0.2ex} = \bm{V} \hspace{-0.3ex} \dotp \hspace{.15ex} \rotationtensor}$,
\en{as well as}\ru{так~же как и}~\en{another tensors}\ru{другие тензоры},
\en{originating}\ru{происходящие}
\en{from}\ru{от}~${\bm{U}\hspace{-0.25ex}}$
\ru{или\,(и)}\en{or\,(and)}~${\bm{V}\hspace{-0.3ex}}$.

\en{The widely used ones are}\ru{Широко используются}
\en{the~}\inquotes{\en{squares}\ru{квадраты}}
\en{of~}${\bm{U}\hspace{-0.25ex}}$
\en{and}\ru{и}~${\bm{V}\hspace{-0.1ex}}$

\nopagebreak\vspace{-0.1em}\begin{equation}\label{deformationtensors.nonlinear}
\begin{array}{c}
\hspace*{-2.33em} \bigl( \hspace{.1ex} \bm{U}^{\hspace{.1ex}2} \hspace{-0.3ex} = \hspace{.22ex} \bigr) \hspace{.8ex}
\bm{U} \hspace{-0.3ex} \dotp \hspace{.1ex} \bm{U} \hspace{-0.2ex}
= \bm{F}^{\hspace{.1ex}\T} \hspace{-0.4ex} \dotp \bm{F}
\equiv \bm{G}
\hspace{.1ex} ,
\\[.1em]
%
\hspace*{-2.33em} \bigl( \hspace{.1ex} \bm{V}^{\hspace{.04ex}2} \hspace{-0.3ex} = \hspace{.22ex} \bigr) \hspace{.8ex}
\bm{V} \hspace{-0.3ex} \dotp \bm{V} \hspace{-0.25ex}
= \bm{F} \dotp \bm{F}^{\hspace{.1ex}\T} \hspace{-0.36ex}
\equiv \mathboldPhi
\hspace{.1ex} .
\end{array}
\end{equation}

\vspace{-0.25em}\noindent
\en{These are}\ru{Это}
\en{the }\ru{тензор деформации }Green’\en{s}\ru{а}\en{ deformation tensor}
(\en{or}\ru{или}
\en{the right}\ru{правый}
\ru{тензор }Cauchy--Green\ru{’а}\en{ tensor})~$\bm{G}$
\en{and}\ru{и}
\en{the }\ru{тензор деформации }Finger’\en{s}\ru{а}\en{ deformation tensor}
(\en{or}\ru{или}
\en{the left}\ru{левый}
\ru{тензор }Cauchy--Green\ru{’а}\en{ tensor})~$\mathboldPhi$.
%
\en{They have}\ru{У~них есть}
\en{the~convenient link}\ru{удобная связь}
\en{with}\ru{с}~\en{the motion gradient}\ru{градиентом движения}~$\bm{F}$,
\en{without}\ru{без}
\en{calculating square roots}\ru{вычисления квадратных корней}
(\en{as}\ru{как}
\en{it’s needed}\ru{это нужно}
\en{for}\ru{для}~${\bm{U}\hspace{-0.25ex}}$
\en{and}\ru{и}~${\bm{V}\hspace{-0.25ex}}$%
).
%
\en{That’s}\ru{Таков\'{а}}
\en{the big reason}\ru{больш\'{а}я причина}\ru{,}
\en{why}\ru{почему}
\en{tensors}\ru{тензоры}~$\bm{G}$
\en{and}\ru{и}~$\mathboldPhi$
\en{are so widely used}\ru{так широк\'{о} используются}.

\en{Tensor}\ru{Тензор}~$\bm{G}$
\en{was first used}\ru{впервые использовал}
\en{by }George Green\hspace{-0.1ex}%
\footnote{%
\href{https://en.wikipedia.org/wiki/George_Green_(mathematician)}{\bookauthor{Green, George}}.
\href{https://hdl.handle.net/2027/mdp.39015027059651?urlappend=\%3Bseq=133}{(1839) On the~propagation of~light in crystallized media. \emph{Transactions of the~Cambridge Philosophical Society.} 1842, vol.\:7, part~II, pages 121\hbox{--}140.}
}\hspace{-0.5ex}.

\en{An~inversion}\ru{Обращение}
\en{of~}$\mathboldPhi$
\en{and}\ru{и}~$\bm{G}$
\en{gives}\ru{даёт} \en{the two more}\ru{ещё два} \en{deformation tensors}\ru{тензора деформации}

\nopagebreak\vspace{-0.2em}\begin{equation}\label{moredeformationtensors.nonlinear}
\begin{array}{c}
\bm{V}^{\expminustwo} \hspace{-0.25ex}
= \mathboldPhi^{\expminusone} \hspace{-0.2ex}
= \hspace{-0.1ex} \left( \bm{F} \dotp \bm{F}^{\hspace{.1ex}\T} \hspace{.1ex} \right)^{\hspace{-0.33ex}\expminusone} \hspace{-0.4ex}
= \bm{F}^{\expminusT} \hspace{-0.3ex} \dotp \bm{F}^{\expminusone} \hspace{-0.25ex}
\equiv {^2\hspace{-0.2ex}\bm{c}}
\hspace{.2ex} ,
\\
%
\bm{U}^{\expminustwo} \hspace{-0.25ex}
= \bm{G}^{\hspace{.12ex}\expminusone} \hspace{-0.2ex}
= \hspace{-0.1ex} \left( \bm{F}^{\hspace{.1ex}\T} \hspace{-0.4ex} \dotp \bm{F} \hspace{.2ex} \right)^{\hspace{-0.33ex}\expminusone} \hspace{-0.4ex}
= \bm{F}^{\expminusone} \hspace{-0.3ex} \dotp \bm{F}^{\expminusT} \hspace{-0.3ex}
\equiv {^2\hspace{-0.4ex}\bm{f}}
\hspace{-0.1ex} ,
\end{array}
\end{equation}

\vspace{-0.2em}\noindent
\en{each of~which}\ru{каждый из~которых} \en{is sometimes called}\ru{иногда называется} \ru{тензором }\en{the~}\href{https://en.wikipedia.org/wiki/Gabrio_Piola}{Piola}\en{ tensor} \en{or}\ru{или} \ru{тензором }\en{the~}\href{https://en.wikipedia.org/wiki/Josef_Finger}{Finger}\ru{’а}\en{ tensor}.
\en{The~inverse}\ru{Обратный} \en{of the left}\ru{к~левому} \ru{тензору }Cauchy--Green\ru{’а}\en{ tensor}~${\hspace{-0.1ex}\mathboldPhi\hspace{.1ex}}$\en{ is}\ru{\:---} \ru{тензор деформации }\en{the }Cauchy\en{ deformation tensor}~${\hspace{-0.2ex}{^2}\hspace{-0.2ex}\bm{c}}$.

\en{The components}\ru{Компоненты}
\en{of these tensors}\ru{этих тензоров}\en{ are}

\nopagebreak\vspace{-0.1em}
\begin{equation*}\label{componentsofdeformationtensors}
\begin{array}{r@{\hspace{.5em}}l}
\bm{G} = \initiallocationvector^i \hspace{-0.25ex} \currentlocationvector_\differentialindex{i} \hspace{-0.1ex} \dotp \currentlocationvector_\differentialindex{\hspace{-0.1ex}j} \initiallocationvector^j \hspace{-0.25ex}
= G_{\hspace{-0.15ex}i\hspace{-0.1ex}j} \hspace{.1ex} \initiallocationvector^i \initiallocationvector^j
\hspace{-0.3ex} , &
G_{\hspace{-0.15ex}i\hspace{-0.1ex}j} \hspace{-0.2ex} \equiv
\currentlocationvector_\differentialindex{i} \hspace{-0.1ex} \dotp \currentlocationvector_\differentialindex{\hspace{-0.1ex}j}
\hspace{.1ex} ,
\\[.25em]
%
{{^2}\hspace{-0.4ex}\bm{f}} \hspace{-0.2ex} =  \initiallocationvector_\differentialindex{i} \currentlocationvector^{i} \hspace{-0.2ex} \dotp \hspace{-0.15ex} \currentlocationvector^j \hspace{-0.1ex} \initiallocationvector_\differentialindex{\hspace{-0.1ex}j} \hspace{-0.2ex}
= G^{\hspace{.1ex}i\hspace{-0.1ex}j} \initiallocationvector_\differentialindex{i} \initiallocationvector_\differentialindex{\hspace{-0.1ex}j} \hspace{-0.2ex}
\hspace{.2ex} , &
G^{\hspace{.1ex}i\hspace{-0.1ex}j} \hspace{-0.2ex} \equiv
\currentlocationvector^{i} \hspace{-0.2ex} \dotp \hspace{-0.15ex} \currentlocationvector^j
\hspace{-0.3ex} ,
\\[.25em]
%
{{^2}\hspace{-0.2ex}\bm{c}} = \hspace{-0.1ex} \currentlocationvector^{i} \initiallocationvector_\differentialindex{i} \hspace{-0.1ex} \dotp \hspace{.1ex} \initiallocationvector_\differentialindex{\hspace{-0.1ex}j} \currentlocationvector^{j} \hspace{-0.25ex}
= \textsl{g}_{i\hspace{-0.1ex}j} \hspace{.1ex} \currentlocationvector^{i} \hspace{-0.2ex} \currentlocationvector^{j}
\hspace{-0.3ex} , &
\textsl{g}_{i\hspace{-0.1ex}j} \hspace{-0.15ex} \equiv
\initiallocationvector_\differentialindex{i} \hspace{-0.1ex} \dotp \hspace{.1ex} \initiallocationvector_\differentialindex{\hspace{-0.1ex}j}
\hspace{.1ex} ,
\\[.25em]
%
\mathboldPhi = \hspace{-0.1ex} \currentlocationvector_\differentialindex{i} \hspace{.1ex} \initiallocationvector^i \hspace{-0.15ex} \dotp \hspace{.1ex} \initiallocationvector^j \hspace{-0.25ex} \currentlocationvector_\differentialindex{\hspace{-0.1ex}j} \hspace{-0.2ex}
= \textsl{g}^{\hspace{.2ex}i\hspace{-0.1ex}j} \currentlocationvector_\differentialindex{i} \currentlocationvector_\differentialindex{\hspace{-0.1ex}j}
\hspace{.1ex} , &
\textsl{g}^{\hspace{.2ex}i\hspace{-0.1ex}j} \hspace{-0.15ex} \equiv
\initiallocationvector^{\hspace{.1ex}i} \hspace{-0.15ex} \dotp \hspace{.1ex} \initiallocationvector^{\hspace{.1ex}j}
\hspace{.1ex} ,
\end{array}
\end{equation*}

\vspace{-0.2em}\noindent
\en{and }\ru{и~}\en{they}\ru{они}
\en{coincide}\ru{совпадают}
\en{with the components}\ru{с~компонентами}
\en{of the unit}\ru{единичного}~(\en{metric}\ru{метрического})
\en{tensor}\ru{тензора}

\nopagebreak\vspace{-0.4em}
\begin{multline*}
\shoveleft{
   \hspace{3em} \UnitDyad
   = \hspace{-0.2ex}
   \currentlocationvector_\differentialindex{i} \hspace{-0.1ex} \currentlocationvector^{i}
   \hspace{-0.2ex} =
   G_{\hspace{-0.15ex}i\hspace{-0.1ex}j} \currentlocationvector^{i} \hspace{-0.2ex} \currentlocationvector^j
   \hspace{-0.25ex} = \hspace{-0.2ex}
   \currentlocationvector^{i} \hspace{-0.2ex} \currentlocationvector_\differentialindex{i}
   \hspace{-0.2ex} =
   G^{\hspace{.1ex}i\hspace{-0.1ex}j} \hspace{-0.2ex} \currentlocationvector_\differentialindex{i} \currentlocationvector_\differentialindex{\hspace{-0.1ex}j}
\hfill }
\\[-0.2em]
%
= \initiallocationvector^i \initiallocationvector_\differentialindex{i} \hspace{-0.2ex}
= \textsl{g}^{\hspace{.2ex}i\hspace{-0.1ex}j} \initiallocationvector_\differentialindex{i} \initiallocationvector_\differentialindex{\hspace{-0.1ex}j} \hspace{-0.2ex}
= \initiallocationvector_\differentialindex{i} \initiallocationvector^i \hspace{-0.25ex}
= \textsl{g}_{i\hspace{-0.1ex}j} \hspace{.1ex} \initiallocationvector^i \initiallocationvector^j
\hspace{-0.2ex} ,
\end{multline*}

\vspace{-0.2em}\noindent
\en{but}\ru{но}
\en{the components’ bases}\ru{базисы компонент}
\en{are different}\ru{разные}.
%
\en{Using}\ru{Пользуясь}
\en{only}\ru{только}
\en{the~index notation}\ru{индексной записью},
\en{it’s easy to get confused}\ru{легко запутаться}
\en{due to the~differences}\ru{из\hbox{-}за различий}
\en{between}\ru{между}
\en{the~unit}\ru{единичным}
\en{tensor}\ru{тензором}~$\UnitDyad$
\en{and the~strain tensors}\ru{и~тензорами деформации}
$\bm{G}$,
$\mathboldPhi$,
${{^2}\hspace{-0.4ex}\bm{f}\hspace{-0.1ex}}$,
${{^2}\hspace{-0.2ex}\bm{c}}$.
%
\en{The direct indexless notation}\ru{Прямая безиндексная запись}
\en{has}\ru{имеет}
\ru{тут }\en{the obvious}\ru{явное}
\en{advantage}\ru{преимущество}\en{ here}.

\en{As was mentioned}\ru{Как упоминалось}
\en{in}\ru{в}~\chapterdotsectionref{chapter:mathapparatus}{section:polardecomposition},
\en{the~invariants}\ru{инварианты}
\en{of~the~stretch tensors}\ru{тензоров искажений}
$\bm{U}\hspace{-0.1em}$
\en{and}\ru{и}~$\bm{V}\hspace{-0.1em}$
\en{are the~same}\ru{одинаковые}.
%
\en{If}\ru{Если}
${w_{i}}$~\en{are}\ru{это}
\en{the~three eigenvalues}\ru{три собственных значения}
\en{of~}${\bm{U}\hspace{-0.1em}}$
\en{and}\ru{и}~${\bm{V}\hspace{-0.2em}}$,
\en{that is}\ru{то есть}
\en{the~roots}\ru{корни}
\en{of the characteristic equations}\ru{характеристических уравнений}
\en{for these tensors}\ru{для этих тензоров},
\en{then}\ru{то}
\en{here are}\ru{вот}
\en{their}\ru{их}
\en{invariants}\ru{инварианты}:

\nopagebreak\vspace{-0.3em}
\begin{gather*}
\anyfirstinvariantof{\bm{U}}
= \anyfirstinvariantof{\bm{V}}
= \trace{\bm{U}} \hspace{-0.25ex}
= \trace{\hspace{-0.2ex}\bm{V}} \hspace{-0.25ex}
= \textstyle\sum \hspace{-0.2ex} U_{\hspace{-0.2ex}j\hspace{-0.2ex}j}
= \textstyle\sum \hspace{-0.2ex} V_{\hspace{-0.1ex}j\hspace{-0.2ex}j}
= \textstyle\sum \hspace{-0.2ex} w_{i}
\hspace{.1ex} ,
\\
%
\anysecondinvariantof{\bm{U}} \hspace{-0.2ex}
= \anysecondinvariantof{\bm{V}} \hspace{-0.2ex}
= \vphantom{\textstyle\sum}
w_1 w_2 \hspace{-0.2ex}
+ w_1 w_3 \hspace{-0.2ex}
+ w_2 \hspace{.15ex} w_3
\hspace{.1ex} ,
\\
%
\anythirdinvariantof{\bm{U}} \hspace{-0.2ex}
= \anythirdinvariantof{\bm{V}} \hspace{-0.2ex}
= \vphantom{\textstyle\sum}
w_1 w_2 \hspace{.15ex} w_3
\hspace{.1ex} .
\end{gather*}

\en{The invariants}\ru{Инварианты}
\en{of~}$\bm{G}$
\en{and}\ru{и}~$\mathboldPhi$
\en{coincide too}\ru{тоже совпадают}:

\nopagebreak
\begin{equation*}
\anyfirstinvariantof{\bm{G}}
\hspace{-0.2ex}
= \anyfirstinvariantof{\mathboldPhi}
\hspace{.1ex} , \dots
\end{equation*}

\en{Without a~deformation}\ru{Без деформации}

\nopagebreak\vspace{-0.2em}\begin{equation*}
\bm{F} = \bm{U} \hspace{-0.3ex} = \bm{V} \hspace{-0.3ex} = \bm{G} = \mathboldPhi = {^2}\hspace{-0.4ex}\bm{f} = {^2}\hspace{-0.2ex}\bm{c} = \UnitDyad
\hspace{.1ex}
,
\end{equation*}

\vspace{-0.2em}\noindent
\en{thus}\ru{поэтому}
\en{as characteristics of~deformation}\ru{как характеристики деформации}
\en{it’s worth taking}\ru{ст\'{о}ит взять}
\en{the differences}\ru{разности}
\en{like}\ru{типа}
${\bm{U} \hspace{-0.2ex} - \UnitDyad}$,
${\bm{U} \hspace{-0.3ex} \dotp \bm{U} \hspace{-0.2ex} - \UnitDyad}$, \dots

...

\subsection*{The right Cauchy\hbox{--}Green deformation tensor}

George Green discovered a deformation tensor known as the right Cauchy\hbox{--}Green deformation tensor or Green’s deformation tensor

\nopagebreak\begin{equation*}
\bm{G}
= \bm{F}^{\hspace{.1ex}\T} \hspace{-0.5ex} \dotp \bm{F}
= \bm{U}^{2}
\hspace{1em} \text{\en{or}\ru{или}} \hspace{1em}
G_{i\hspace{-0.1ex}j} \hspace{-0.2ex}
= F_{k' i} \hspace{.25ex} F_{k' \hspace{-0.15ex}j} \hspace{-0.15ex}
= \frac{\partial \hspace{-0.1ex} x_{\hspace{-0.1ex}k'}}{\partial \mathcircabove{x}_{i}} \hspace{.15ex} \frac{\partial \hspace{-0.1ex} x_{\hspace{-0.1ex}k'}}{\partial \mathcircabove{x}_{\hspace{-0.2ex}j}}
\hspace{.1ex} .
\end{equation*}

This tensor \textcolor{magenta}{gives the~\inquotes{square} of local change in distances} due to deformation:
${\displaystyle d\currentlocationvector \dotp d\currentlocationvector = d\initiallocationvector \dotp \bm{G} \dotp d\initiallocationvector}$

The most popular invariants of~${\bm{G}}$ are
%%used in expressions for the potential energy of elastic deformations of an~isotropic body.
\[
\begin{array}{r@{\hspace{.25em}}c@{\hspace{.33em}}l}
\anyfirstinvariantof{\bm{G}} & \equiv &
\trace{\bm{G}}
= G_{ii} \hspace{-0.2ex} = \gamma_{1}^{2} + \gamma_{2}^{2} + \gamma_{3}^{2}
\\[.25em]
%
\anysecondinvariantof{\bm{G}} & \equiv &
\smalldisplaystyleonehalf \bigl( G_{\hspace{-0.2ex}j\hspace{-0.1ex}j}^{\hspace{.25ex}2} \hspace{-0.1ex} - G_{ik} G_{ki} \hspace{.1ex} \bigr) \hspace{-0.25ex}
= \gamma_{1}^{2}\gamma_{2}^{2} + \gamma_{2}^{2}\gamma_{3}^{2} + \gamma_{3}^{2}\gamma_{1}^{2}
\\[.4em]
%
\anythirdinvariantof{\bm{G}} & \equiv &
\determinant \hspace{.1ex} \bm{G}
= \gamma_{1}^{2}\gamma_{2}^{2}\gamma_{3}^{2}
\end{array}
\]
where ${\gamma_{i}\hspace{-0.2ex}}$ are stretch ratios for unit fibers that are initially oriented along directions of eigenvectors of the right stretch tensor~${\bm{U}\hspace{-0.2ex}}$.

\subsection*{The inverse of Green’s deformation tensor}

Sometimes called the Finger tensor or the Piola tensor, the~inverse of the right Cauchy\hbox{--}Green deformation tensor

\nopagebreak\vspace{-0.25em}\begin{equation*}
{^2\hspace{-0.4ex}\bm{f}}
= \bm{G}^{\expminusone} \hspace{-0.25ex}
= \bm{F}^{\expminusone} \hspace{-0.4ex} \dotp \bm{F}^{\expminusT}
\hspace{1em} \text{\en{or}\ru{или}} \hspace{1em}
f_{i\hspace{-0.1ex}j} \hspace{-0.2ex} = \frac{\partial \mathcircabove{x}_{i}}{\partial x_{\hspace{-0.1ex}k'}} \hspace{.15ex} \frac{\partial \mathcircabove{x}_{\hspace{-0.2ex}j}}{\partial x_{\hspace{-0.1ex}k'}}
\end{equation*}

\subsection*{The left Cauchy\hbox{--}Green or Finger deformation tensor}

Swapping multipliers in the formula for the right Green–Cauchy deformation tensor leads to the left Cauchy\hbox{--}Green deformation tensor, defined as

\nopagebreak\vspace{-0.2em}\begin{equation*}
\mathboldPhi
= \bm{F} \dotp \bm{F}^{\hspace{.1ex}\T} \hspace{-0.4ex}
= \bm{V}^{2}
\hspace{1em} \text{\en{or}\ru{или}} \hspace{1em}
\Phi_{i\hspace{-0.1ex}j} \hspace{-0.2ex}
= \frac{\partial x_{i}}{\partial \mathcircabove{x}_{k}} \hspace{.15ex} \frac{\partial x_{\hspace{-0.15ex}j}}{\partial \mathcircabove{x}_{k}}
\end{equation*}

The left Cauchy\hbox{--}Green deformation tensor is often called the Finger’s deformation tensor, named after Josef Finger (1894).

Invariants of ${\mathboldPhi}$ are also used in expressions for strain energy density functions.
The conventional invariants are defined as

\nopagebreak\begin{equation*}
\begin{aligned}
I_{1} & \equiv \Phi_{ii} = \lambda_{1}^{2} + \lambda_{2}^{2} + \lambda_{3}^{2}
\\
%
I_{2} & \equiv \tfrac{1}{2} \bigl( \Phi_{ii}^{2} - \Phi_{jk}\Phi_{kj} \bigr) = \lambda_{1}^{2}\lambda_{2}^{2} + \lambda_{2}^{2}\lambda_{3}^{2} + \lambda_{3}^{2}\lambda_{1}^{2}
\\
%
I_{3} & \equiv \det \mathboldPhi = J^{2} \hspace{-0.4ex} = \lambda_{1}^{2}\lambda_{2}^{2}\lambda_{3}^{2}
\end{aligned}
\end{equation*}

\vspace{-0.2em}\noindent
(${J \equiv \det{\bm{F}}}$\en{ is}\ru{\:---} \en{the Jacobian}\ru{якобиан}, \en{determinant of the motion gradient}\ru{определитель градиента движения})

\subsection*{The Cauchy deformation tensor}

The Cauchy deformation tensor is defined as the~inverse of the left Cauchy\hbox{--}Green deformation tensor

\nopagebreak\vspace{-0.4em}\begin{equation*}
{^2\hspace{-0.2ex}\bm{c}} = \mathboldPhi^{\expminusone} \hspace{-0.25ex}
= \bm{F}^{\expminusT} \hspace{-0.4ex} \dotp \bm{F}^{\expminusone}
\hspace{1em} \text{\en{or}\ru{или}} \hspace{1em}
c_{i\hspace{-0.1ex}j} \hspace{-0.2ex}
= \frac{\partial \mathcircabove{x}_{k}}{\partial \hspace{-0.1ex} x_{i}} \hspace{.15ex} \frac{\partial \mathcircabove{x}_{k}}{\partial \hspace{-0.1ex} x_{\hspace{-0.15ex}j}}
\end{equation*}

${\displaystyle d\initiallocationvector \dotp d\initiallocationvector = d\currentlocationvector \dotp {^2\hspace{-0.2ex}\bm{c}} \dotp d\currentlocationvector}$

This tensor is also called the Piola tensor or the Finger tensor in rheology and fluid dynamics literature.

\subsection*{Finite strain tensors}

The concept of \emph{strain} is used to evaluate how much a~given displacement differs locally from a~body displacement as a~whole (a~\inquotes{rigid body displacement}). One of such strains for large deformations is the \emph{Green strain tensor} (\emph{Green\hbox{--}Lagrangian strain tensor}, \emph{Green\hbox{--}Saint\hbox{-\hspace{-0.2ex}}Venant strain tensor}), defined as

\nopagebreak\begin{equation*}
\displaystyle \bm{C} = \smalldisplaystyleonehalf \bigl( \bm{G} - \UnitDyad \hspace{.1ex} \bigr)
\hspace{1em} \text{\en{or}\ru{или}} \hspace{1em}
C_{i\hspace{-0.1ex}j} \hspace{-0.2ex} = \onehalf \Bigl( \frac{\partial x_{k'}}{\partial \mathcircabove{x}_{i}} \hspace{.15ex} \frac{\partial x_{k'}}{\partial \mathcircabove{x}_{\hspace{-0.2ex}j}} - \delta_{i\hspace{-0.1ex}j} \Bigr)
\end{equation*}

\noindent
or as the function of the displacement gradient tensor

\nopagebreak\begin{equation*}
\displaystyle \bm{C} = \smalldisplaystyleonehalf \hspace{-0.3ex} \left( \hspace{-0.1ex}
\boldnablacircled\bm{u}
+ \hspace{-0.1ex} \boldnablacircled\bm{u}^{\hspace{-0.1ex}\T} \hspace{-0.3ex}
+ \hspace{-0.1ex} \boldnablacircled\bm{u} \dotp \hspace{-0.1ex} \boldnablacircled\bm{u}^{\hspace{-0.1ex}\T}
\right)
\end{equation*}

\noindent
in cartesian coordinates

\nopagebreak\begin{equation*}
\displaystyle C_{i\hspace{-0.1ex}j} \hspace{-0.2ex} = \onehalf \hspace{-0.25ex} \left(
\frac{\partial u_{\hspace{-0.1ex}j}}{\partial \mathcircabove{x}_{i}}
+ \frac{\partial u_{i}}{\partial \mathcircabove{x}_{\hspace{-0.2ex}j}}
+ \frac{\partial u_{k}}{\partial \mathcircabove{x}_{i}} \frac{\partial u_{k}}{\partial \mathcircabove{x}_{\hspace{-0.2ex}j}}
\right)
\end{equation*}

The Green strain tensor measures how much $\bm{G}$ differs from~$\UnitDyad$.

The \emph{Almansi\hbox{--}Hamel strain tensor}, referenced to the deformed configuration (\inquotes{Eulerian description}), is defined as

\nopagebreak\vspace{-0.5em}\begin{equation*}
{^2\hspace{-0.2ex}\bm{a}} = \smalldisplaystyleonehalf \bigl( \UnitDyad - \hspace{-0.15ex} {^2\hspace{-0.2ex}\bm{c}} \hspace{.3ex} \bigr) \hspace{-0.3ex}
= \smalldisplaystyleonehalf \bigl( \UnitDyad - \mathboldPhi^{\expminusone} \hspace{.2ex} \bigr)
\hspace{1em} \text{\en{or}\ru{или}} \hspace{1em}
a_{i\hspace{-0.1ex}j} \hspace{-0.2ex}
= \onehalf \hspace{-0.25ex} \left( \hspace{-0.4ex} \delta _{i\hspace{-0.1ex}j} - \frac{\partial \mathcircabove{x}_{k}}{\partial \hspace{-0.1ex} x_{i}} \hspace{.15ex} \frac{\partial \mathcircabove{x}_{k}}{\partial \hspace{-0.1ex} x_{\hspace{-0.15ex}j}} \right)
\end{equation*}

\vspace{-0.4em}\noindent
or as function of the displacement gradient

\nopagebreak\begin{equation*}
{^2}\hspace{-0.2ex}\bm{a} = \smalldisplaystyleonehalf \bigl(
\boldnabla\bm{u}^{\hspace{-0.1ex}\T} \hspace{-0.3ex}
+ \hspace{-0.1ex} \boldnabla\bm{u}
- \hspace{-0.1ex} \boldnabla\bm{u} \dotp \hspace{-0.1ex} \boldnabla\bm{u}^{\hspace{-0.1ex}\T}
\bigr)
\end{equation*}

\nopagebreak\vspace{-0.2em}\begin{equation*}
\displaystyle a_{i\hspace{-0.1ex}j} \hspace{-0.2ex} = \onehalf \hspace{-0.25ex} \left(
\frac{\partial u_{i}}{\partial x_{\hspace{-0.15ex}j}}
+ \frac{\partial u_{\hspace{-0.1ex}j}}{\partial \hspace{-0.1ex} x_{i}}
- \frac{\partial u_{k}}{\partial x_{i}} \frac{\partial u_{k}}{\partial x_{\hspace{-0.15ex}j}}
\right)
\end{equation*}

\subsection*{Seth\hbox{--}Hill family of abstract strain tensors}

B. R. Seth was the first to show that the Green and Almansi strain tensors are special cases of a more abstract measure of deformation.
The idea was further expanded upon by Rodney Hill in~1968 \textcolor{red}{(publication??)}.
The Seth\hbox{--}Hill family of strain measures (also called Doyle\hbox{--}Ericksen tensors) is expressed as

\nopagebreak\vspace{-0.1em}\begin{equation*}
\displaystyle \bm{D}_{(m)} \hspace{-0.2ex}
= \frac{\raisebox{-0.2em}{1}}{2m} \left( \hspace{.1ex} \bm{U}^{2m} \hspace{-0.4ex} - \UnitDyad \hspace{.2ex} \right)
= \frac{\raisebox{-0.2em}{1}}{2m} \left( \bm{G}^{m} \hspace{-0.4ex} - \UnitDyad \hspace{.1ex} \right) \end{equation*}

\vspace{.1em} \noindent \en{For various}\ru{Для разных}~$m$
\en{it gives}\ru{это даёт}

\nopagebreak\begin{equation*}
\begin{array}{r@{\hspace{0.1em}}ll}
\bm{D}_{(1)} & = \smalldisplaystyleonehalf \hspace{-0.25ex} \left( \bm{U}^{2} \hspace{-0.25ex} - \UnitDyad \right) = \smalldisplaystyleonehalf (\bm{G} - \UnitDyad) & \text{\scalebox{0.9}{Green strain tensor}}
\\[.4em]
\bm{D}_{(\nicefrac{1}{2})} & = \bm{U} \hspace{-0.15ex} - \UnitDyad = \bm{G}^{\hspace{.1ex}\nicefrac{1}{2}} \hspace{-0.25ex} - \UnitDyad & \text{\scalebox{0.9}{Biot strain tensor}}
\\[.4em]
\bm{D}_{(0)} & = \ln \bm{U} = \smalldisplaystyleonehalf \ln \bm{G} & \text{\scalebox{0.9}{logarithmic strain, Hencky strain}}
\\[.4em]
\bm{D}_{(-\hspace{-0.1ex}1)} & = \smalldisplaystyleonehalf \hspace{-0.25ex} \left( \hspace{-0.1ex} \UnitDyad - \bm{U}^{-2} \hspace{.1ex} \right) & \text{\scalebox{0.9}{Almansi strain}}
\end{array}
\end{equation*}

The second\hbox{-}order approximation of these tensors is
\[ \bm{D}_{(m)} \hspace{-0.2ex} =
\infinitesimaldeformation
+ \smalldisplaystyleonehalf \hspace{.1ex} \boldnabla\bm{u} \dotp \hspace{-0.1ex} \boldnabla\bm{u}^{\hspace{-0.1ex}\T} \hspace{-0.3ex}
- (1 - m) \hspace{.2ex} \infinitesimaldeformation \dotp \infinitesimaldeformation \]

\vspace{-0.25em}\noindent
where ${\infinitesimaldeformation \equiv \hspace{-0.2ex} \boldnabla {\bm{u}}^{\hspace{.1ex}\mathsf{S}}}$ is the infinitesimal deformation tensor.

Many other different definitions of measures~$\bm{D}$ are possible, provided that they satisfy these conditions:

\begin{itemize}
\item $\bm{D}$ vanishes for any movement of a~body as a~rigid whole
\item dependence of~$\bm{D}$ on displacement gradient tensor~${\nabla \bm{u}}$ is continuous, continuously differentiable and monotonic
\item it’s desired that $\bm{D}$ reduces to the infinitesimal linear deformation tensor~${\infinitesimaldeformation}$ when ${\boldnabla \bm{u} \to 0}$
\end{itemize}

\noindent For example, tensors from the set
\[ \displaystyle \bm{D}^{(n)} \hspace{-0.32ex} = \left( {\bm{U}}^{n} \hspace{-0.4ex} - {\bm{U}}^{-n} \right) \hspace{-0.4ex} / \hspace{.25ex} 2n \]
aren’t from the Seth\hbox{--}Hill family, but for any~$n$ they have the same 2nd\hbox{-}order approximation as Seth\hbox{--}Hill measures with~${m=0}$.

\vspace{.4em} \noindent \hfill \textboldoblique{Wikipedia, the free encyclopedia}\:--- \href{https://en.wikipedia.org/wiki/Finite_strain_theory}{Finite strain theory}

...


\subsection*{\en{Logarithmic strain, Hencky’s strain}\ru{Логарифмическая деформация, деформация Hencky}}

\href{https://en.wikipedia.org/wiki/Heinrich_Hencky}{%
\bookauthor{Heinrich Hencky}%
}.
Über die Form des Elastizitätsgesetzes bei ideal elastischen Stoffen.
Zeitschrift für technische Physik, Vol.\:9~(1928),
Seiten~215\hbox{--}220.


....



\en{\section{Velocity field}}

\ru{\section{Поле скоростей}}

\label{section:velocityfield}

\en{This topic}\ru{Эта тема} \en{is discussed}\ru{обсуждается} \en{in nearly any}\ru{в~почти любой} \en{book}\ru{книге} \en{about~continuum mechanics}\ru{о~механике сплошной среды}, \en{however}\ru{однако} \en{for}\ru{для} \en{solid elastic continua}\ru{твёрдых упругих сред} \en{it’s not very vital}\ru{она не~столь насущна}.
\en{Among}\ru{Среди} \en{various}\ru{разных} \en{models}\ru{моделей} \en{of~a~material continuum}\ru{материального \rucontinuum{}а}, \en{an elastic solid body}\ru{упругое твёрдое тело} \en{is distinguished}\ru{выделяется} \en{by interesting possibility}\ru{интересной возможностью} \en{of~deriving}\ru{вывода} \en{the~complete set}\ru{полного набора}~(\en{system}\ru{системы}) \en{of~equations}\ru{уравнений} \en{for it}\ru{для него} \en{via the single logically flawless procedure}\ru{единой логически безупречной процедурой}.
\en{But now}\ru{Но пока} \en{we follow the~way}\ru{мы идём путём}, \en{usual}\ru{обычным} \en{for}\ru{для} \en{fluid continuum mechanics}\ru{механики сплошной текучей среды}.

\en{So}\ru{Итак}, \en{there’s}\ru{есть} \en{velocity field}\ru{поле скоростей} \en{in spatial description}\ru{в~пространственном описании}
${\bm{v} \equiv \hspace{-0.1ex} \mathdotabove{\currentlocationvector} = \bm{v}(\currentlocationvector, \hspace{-0.1ex} t)}$.
\en{Decomposition}\ru{Разложение} \en{of~tensor}\ru{тензора} ${\hspace{-0.1ex}\boldnabla \bm{v} \hspace{-0.16ex} = \hspace{-0.2ex} \boldnabla \mathdotabove{\currentlocationvector} = \hspace{-0.16ex} \currentlocationvector^{i} \partial_i \mathdotabove{\currentlocationvector} = \hspace{-0.16ex} \currentlocationvector^{i} \mathdotabove{\currentlocationvector}_\differentialindex{i}}$%
\kern-0.15ex\footnote{For sufficiently smooth functions, partial derivatives always commute, space and time ones too.
Thus

\nopagebreak\vspace{-0.8em}\begin{equation*}
\scalebox{0.92}{$\displaystyle
\frac{\raisemath{-0.2em}{\partial}}{\partial q^i} \frac{\raisemath{-0.2em}{\partial \hspace{.1ex} \currentlocationvector}}{\partial t}
= \frac{\raisemath{-0.2em}{\partial}}{\partial t} \frac{\raisemath{-0.2em}{\partial \hspace{.1ex} \currentlocationvector}}{\partial q^i}
$}
\hspace{.8em} \text{\en{or}\ru{или}} \hspace{.8em}
\partial_i \mathdotabove{\currentlocationvector} = \hspace{-0.1ex} \mathdotabove{\currentlocationvector}_\differentialindex{i}
\end{equation*}
}
\en{into symmetric and skewsymmetric parts}\ru{на симметричную и~кососимметричную части}~(\chapterdotsectionref{chapter:mathapparatus}{section:tensors.symmetric+skewsymmetric})

\nopagebreak\vspace{-0.1em}\begin{equation*}
\boldnabla \mathdotabove{\currentlocationvector}
= \hspace{-0.2ex} \boldnabla \mathdotabove{\currentlocationvector}^{\hspace{.3ex}\mathsf{S}} \hspace{-0.25ex}
- \hspace{.1ex} \smalldisplaystyleonehalf \bigl( \boldnabla \hspace{-0.25ex}\times\hspace{-0.25ex} \mathdotabove{\currentlocationvector} \bigr) \hspace{-0.4ex} \times \hspace{-0.2ex} \UnitDyad
\end{equation*}

\vspace{-0.1em}\noindent
\en{or}\ru{или},
\en{introducing}\ru{вводя}
\en{the~rate of~deformation tensor}\ru{тензор скорости деформации}~(\ru{rate of~deformation tensor, }rate of~stretching tensor, strain rate tensor)~${\rateofdeformationtensor}$
\en{and }\ru{и~}\en{the~vorticity tensor}\ru{тензор вихря}~(\ru{vorticity tensor, }rate of~rotation tensor, spin tensor)~${\vorticitytensor}$

\nopagebreak\vspace{-0.1em}\begin{equation}
\begin{array}{c}
\boldnabla \bm{v} = \rateofdeformationtensor - \hspace{-0.1ex} \vorticitytensor
,
\\[.333em]
%
\rateofdeformationtensor \equiv \hspace{-0.12ex} \boldnabla {\bm{v}}^{\hspace{.2ex}\mathsf{S}} \hspace{-0.32ex}
=  \hspace{-0.2ex} \boldnabla \mathdotabove{\currentlocationvector}^{\hspace{.3ex}\mathsf{S}} \hspace{-0.32ex}
= \smalldisplaystyleonehalf \bigl( \hspace{-0.1ex} \currentlocationvector^{i} \mathdotabove{\currentlocationvector}_\differentialindex{i} \hspace{-0.1ex} + \hspace{-0.1ex} \mathdotabove{\currentlocationvector}_\differentialindex{i} \currentlocationvector^{i} \bigr)
\hspace{.1ex} ,
\\[.44em]
%
- \hspace{.1ex} \vorticitytensor \equiv \hspace{-0.12ex} \boldnabla {\bm{v}}^{\hspace{.2ex}\mathsf{A}} \hspace{-0.25ex}
= \hspace{-0.1ex} - \hspace{.24ex} \vorticityvector \hspace{-0.12ex} \times \hspace{-0.2ex} \UnitDyad
\hspace{.1ex} , \hspace{.6em}
\vorticityvector \equiv \hspace{.1ex} \smalldisplaystyleonehalf \hspace{.15ex} \boldnabla \hspace{-0.2ex} \times \hspace{-0.1ex} \bm{v}
= \hspace{.1ex} \smalldisplaystyleonehalf \hspace{.25ex} \currentlocationvector^{i} \hspace{-0.4ex}\times\hspace{-0.3ex} \mathdotabove{\currentlocationvector}_\differentialindex{i}
\hspace{.1ex} ,
\end{array}
\end{equation}

\vspace{-0.1em}\noindent
\en{where}\ru{где}
\en{also figures}\ru{также фигурирует}
\en{the~vorticity }(\en{pseudo}\ru{псевдо})\en{vector}\ru{вектор}\ru{ вихря}~${\vorticityvector}$,
\en{the~companion of}\ru{сопутствующий}~${\vorticitytensor}$.

\ru{Компоненты}\en{Components}
\en{of the~rate of~deformation tensor}\ru{тензора скорости деформации}
\en{in the current configuration’s basis}\ru{в~базисе текущей конфигурации}

\nopagebreak\vspace{-0.4em}\begin{multline*}
\rateofdeformationtensor = \rateofdeformationcomponents{i\hspace{-0.1ex}j} \currentlocationvector^{i} \hspace{-0.2ex} \currentlocationvector^j
\hspace{-0.3ex} , \hspace{.6em}
\rateofdeformationcomponents{i\hspace{-0.1ex}j} \hspace{-0.2ex}
= \hspace{-0.1ex} \currentlocationvector_\differentialindex{i} \hspace{-0.1ex} \dotp \rateofdeformationtensor \dotp \currentlocationvector_\differentialindex{j} \hspace{-0.1ex}
= \hspace{.1ex} \smalldisplaystyleonehalf \hspace{.25ex} \currentlocationvector_\differentialindex{i} \hspace{-0.1ex} \dotp \hspace{-0.1ex} \bigl( \currentlocationvector^{k} \mathdotabove{\currentlocationvector}_{k} \hspace{-0.1ex} + \hspace{-0.1ex} \mathdotabove{\currentlocationvector}_k \currentlocationvector^{k} \bigr) \hspace{-0.4ex} \dotp \hspace{-0.15ex} \currentlocationvector_\differentialindex{\hspace{-0.1ex}j} \hspace{-0.2ex}
=
\\[-0.3em]
\shoveright{ \hfill \hspace{12em}
= \hspace{.1ex} \smalldisplaystyleonehalf \hspace{.1ex} \bigl( \mathdotabove{\currentlocationvector}_\differentialindex{i} \hspace{-0.1ex} \dotp \hspace{-0.15ex} \currentlocationvector_\differentialindex{\hspace{-0.1ex}j} \hspace{-0.1ex} + \hspace{-0.1ex} \currentlocationvector_\differentialindex{i} \hspace{-0.1ex} \dotp \hspace{-0.15ex} \mathdotabove{\currentlocationvector}_\differentialindex{\hspace{-0.1ex}j} \bigr) \hspace{-0.3ex}
= \hspace{.1ex} \smalldisplaystyleonehalf \Bigl( \hspace{-0.2ex}
\currentlocationvector_\differentialindex{i} \hspace{-0.1ex} \dotp \currentlocationvector_\differentialindex{\hspace{-0.1ex}j}
\hspace{-0.2ex} \Bigr)^{\hspace{-0.3ex}\tikz[baseline=-0.5ex]\draw[black, fill=black] (0,0) circle (.266ex);}
}
\end{multline*}

...

\[
\mathdotabove{G}_{\hspace{-0.15ex}i\hspace{-0.1ex}j}
\]

\[
G_{\hspace{-0.15ex}i\hspace{-0.1ex}j} \hspace{-0.2ex} \equiv
\currentlocationvector_\differentialindex{i} \hspace{-0.1ex} \dotp \currentlocationvector_\differentialindex{\hspace{-0.1ex}j}
\]

...


\en{For}\ru{Для}
\en{elastic solid media}\ru{упругих твёрдых сред}\en{,}
\en{there’s no need}\ru{нет нужды}
\en{to discuss}\ru{дискутировать}
\en{about rotations}\ru{о~поворотах}\::
\en{the true representation}\ru{истинное представление}
\en{appears}\ru{появляется}
\en{along the~way}\ru{по~пути}
\en{of~logically harmonious}\ru{логически гармоничных}
\en{conclusions}\ru{выводов}
\en{and without}\ru{и~без}
\en{additional}\ru{добавочных}
\en{hypotheses}\ru{гипотез}.

\en{\section{Area vector. Surface change}} % Nanson’s formula

\ru{\section{Вектор пл\'{о}щади. Изменение площ\'{а}дки}} % формула Нансона

\en{Take an~infinitesimal surface}\ru{Возьмём бесконечно м\'{а}лую площ\'{а}дку}.
\en{The~area vector}\ru{Вектор пл\'{о}щади~(area vector)} \en{by length}\ru{по~длине} \en{is equal to}\ru{равен} \en{the~surface’s area}\ru{пл\'{о}щади площ\'{а}дки} \en{and}\ru{и} \en{is directed along the~normal}\ru{направлен вдоль нормали} \en{to~this surface}\ru{к~этой площ\'{а}дке}.

\en{In~the~initial}\ru{В~начальной}~(\en{original}\ru{оригинальной}, \en{undeformed}\ru{недеформированной}, \inquotes{\en{material}\ru{материальной}}, \sout{\en{reference}\ru{отсчётной}}) \en{configuration}\ru{конфигурации}\en{,} \en{the~area vector}\ru{вектор пл\'{о}щади} \en{can be represented}\ru{может быть представлен} \en{as}\ru{как}~${\initialunitnormal \hspace{.1ex} do}$.
\en{Surface’s area}\ru{Пл\'{о}щадь}~$do$ \en{is infinitely small}\ru{бесконечно мал\'{а}}, \en{and}\ru{а}~$\initialunitnormal$\en{ is}\ru{\:---} \en{unit normal vector}\ru{единичный вектор нормали}.

\en{In~the~present}\ru{В~текущей}~(\en{current, }\en{actual}\ru{актуальной}, \en{deformed}\ru{деформированной}, \inquotes{\en{spatial}\ru{пространственной}}) \en{configuration}\ru{конфигурации}, \en{the~same surface}\ru{та~же площ\'{а}дка} \en{has area vector}\ru{имеет вектор пл\'{о}щади}~${\currentunitnormal \hspace{.1ex} d\mathcal{O}\hspace{-0.2ex}}$.

\en{With differential precision}\ru{С~дифференциальной точностью}, \en{these infinitesimal surfaces}\ru{эти бесконечномалые площ\'{а}дки} \en{are parallelograms}\ru{суть параллелограммы}, \en{thus}\ru{поэтому}

\nopagebreak\vspace{-0.1em}\en{\vspace{-0.32em}}
\begin{equation}\label{areavectorascrossproduct}
\begin{array}{c}
\initialunitnormal \hspace{.1ex} do = d \initiallocationvector{'} \hspace{-0.5ex} \times \hspace{-0.1ex} d \initiallocationvector{''} \hspace{-0.5ex}
= \scalebox{0.9}{$ \displaystyle\frac{\raisemath{-0.2em}{\partial \initiallocationvector}}{\partial q^i} $} \hspace{.2ex} d q^i \hspace{-0.1ex} \times \scalebox{0.9}{$ \displaystyle\frac{\raisemath{-0.2em}{\partial \initiallocationvector}}{\partial q^{j}} $} \hspace{.2ex} d q^{j}
= \initiallocationvector_\differentialindex{i} \hspace{-0.16ex} \times \hspace{-0.1ex} \initiallocationvector_\differentialindex{\hspace{-0.1ex}j} \hspace{.2ex} dq^{i} dq^{j}
\hspace{-0.2ex} ,
\\[.8em]
%
\currentunitnormal \hspace{.1ex} d\mathcal{O} \hspace{-0.1ex} = d \currentlocationvector{'} \hspace{-0.4ex} \times \hspace{-0.1ex} d \currentlocationvector{''} \hspace{-0.5ex}
= \scalebox{0.9}{$ \displaystyle\frac{\raisemath{-0.23em}{\partial \currentlocationvector}}{\partial q^i} $} \hspace{.2ex} d q^i \hspace{-0.12ex} \times \scalebox{0.9}{$ \displaystyle\frac{\raisemath{-0.23em}{\partial \currentlocationvector}}{\partial q^{j}} $} \hspace{.2ex} d q^{j}
= \currentlocationvector_\differentialindex{i} \hspace{-0.1ex} \times \hspace{-0.2ex} \currentlocationvector_\differentialindex{\hspace{-0.1ex}j} \hspace{.2ex} dq^{i} dq^{j}
\hspace{-0.2ex} .
\end{array}
\end{equation}

\en{Applying}\ru{Применяя} \en{the~transformation of~volume}\ru{преобразование объёма}~\eqref{volumechange}, \en{we have}\ru{имеем}

\nopagebreak\vspace{-0.4em}\begin{multline*}
d\mathcal{V} \hspace{-0.1ex} = \hspace{-0.1ex} J d\mathcircabove{\mathcal{V}}
\:\Rightarrow\:
\currentlocationvector_\differentialindex{i} \hspace{-0.1ex} \times \hspace{-0.2ex} \currentlocationvector_\differentialindex{\hspace{-0.1ex}j} \dotp \currentlocationvector_\differentialindex{k}
= J \hspace{.1ex} \initiallocationvector_\differentialindex{i} \hspace{-0.16ex} \times \hspace{-0.1ex} \initiallocationvector_\differentialindex{\hspace{-0.1ex}j} \dotp \initiallocationvector_\differentialindex{k}
\:\Rightarrow
\\
%
\Rightarrow\:
\currentlocationvector_\differentialindex{i} \hspace{-0.1ex} \times \hspace{-0.2ex} \currentlocationvector_\differentialindex{\hspace{-0.1ex}j} \dotp \currentlocationvector_\differentialindex{k} \currentlocationvector^k
= J \hspace{.1ex} \initiallocationvector_\differentialindex{i} \hspace{-0.16ex} \times \hspace{-0.1ex} \initiallocationvector_\differentialindex{\hspace{-0.1ex}j} \dotp \initiallocationvector_\differentialindex{k} \currentlocationvector^k
\:\Rightarrow
\\
%
\Rightarrow\:
\currentlocationvector_\differentialindex{i} \hspace{-0.1ex} \times \hspace{-0.2ex} \currentlocationvector_\differentialindex{\hspace{-0.1ex}j}
= J \hspace{.1ex} \initiallocationvector_\differentialindex{i} \hspace{-0.16ex} \times \hspace{-0.1ex} \initiallocationvector_\differentialindex{\hspace{-0.1ex}j} \dotp \bm{F}^{\hspace{.1ex}\expminusone}
\hspace{-0.1ex} .
\end{multline*}

\en{Hence}\ru{Отсюда} \en{with}\ru{с}~\eqref{areavectorascrossproduct} \en{we come to~the~relation}\ru{мы приходим к~соотношению}

\nopagebreak\vspace{-0.2em}\begin{equation}\label{areachange:nansonformula}
\currentunitnormal \hspace{.1ex} d\mathcal{O} = J \hspace{.1ex} \initialunitnormal \hspace{.1ex} do \dotp \bm{F}^{\hspace{.1ex}\expminusone}
\hspace{-0.2ex} ,
\end{equation}

\nopagebreak\vspace{-0.25em}\en{\vspace{-0.15em}}\noindent
\en{called}\ru{называемому} \en{the }Nanson’\en{s}\en{ formula}\ru{формулой Nanson’а}.

\en{\section{Forces in continuum. Existence of the~Cauchy stress tensor}}

\ru{\section{Силы в \rucontinuum{}е. Существование тензора напряжения Cauchy}}

\label{section:stressviatetrahedron}

% forces are
% • linear force
% • angular torque (moment, moment of force, couple)

% forces are
% • body force, within a force field
% • contact (or surface) force, via direct physical contact

\begin{changemargin}{\parindent}{\parindent}
\vspace{-0.5em}
{\noindent\small
\setlength{\parskip}{\spacebetweenparagraphs}

Augustin-Louis Cauchy
\en{founded}\ru{основал}
\en{the~}\emph{\en{continuum mechanics}\ru{механику \rucontinuum{}а}}\ru{,}
\en{with the~idea that}\ru{исходя из идеи, что}
\en{two adjoining parts of a~body}\ru{две соседние части тела}
\en{interact with each other}\ru{взаимодействуют друг с~другом}
\en{by means of}\ru{посредством}
\en{contact forces}\ru{контактных сил}
\en{on a~dividing surface}\ru{на разделяющей поверхности}.

\en{Assuming that}\ru{Полагая, что}
\en{these contact forces}\ru{эти контактные силы}
\en{depend only}\ru{зависят только}
\en{on the~perpendicular}\ru{от перпендикуляра}
\en{to the dividing surface}\ru{к~разделяющей поверхности}
\en{and that}\ru{и что}
\en{surface contact forces}\ru{поверхностные контактные силы}
\en{are balanced}\ru{балансируются}
\en{by some}\ru{некоторой}
\en{volumetric force density}\ru{объёмной плотностью силы},
\en{including inertia}\ru{включая инерцию},
Cauchy
\en{played with tetrahedrons}\ru{поиграл с~тетраэдрами}
\en{and}\ru{и}~\en{proved}\ru{доказал}
\en{the~existænce}\ru{существование}
\en{of the~stress tensor}\ru{тензора напряжения}.

%% Math exercises
%% Математические упражнения

\noindent
\inquotes{De la pression ou tension dans un corps solide.}
dans
(i) \href{https://books.google.com/books?id=w-741ZwVBnAC&pg=GBS.PA42}{\emph{Exercices de~ma\-th\'{e}\-ma\-tiques}, par M.\hspace{-0.2ex} \href{https://en.wikipedia.org/wiki/Augustin-Louis_Cauchy}{\bookauthor{Augustin\hbox{-}Louis Cauchy}}. Seconde ann\'{e}e: 1827. Paris, Chez de~Bure fr\`{e}res. Pages 42~à~59.}
(ii) \href{https://gallica.bnf.fr/ark:/12148/bpt6k901990/f63}{\emph{Œuvres complètes} d’\bookauthor{Augustin Cauchy}. Série 2, tome 7. Pages 60~à~78.}

%%\noindent
%%\href{https://en.wikipedia.org/wiki/Augustin-Louis_Cauchy}{\bookauthor{Augustin\hbox{-}Louis Cauchy}}. \emph{Exercices de math\'{e}matiques.} Troisi\`{e}me ann\'{e}e: 1828. Paris, Chez de~Bure fr\`{e}res.

\par}
\vspace{.8em}
\end{changemargin}

\en{The particles}\ru{Частицы}
\en{of a~momentless model}\ru{безмоментной модели}
\en{of a~continuum}\ru{\rucontinuum{}а}
\en{are}\ru{суть}
\en{points}\ru{точки}\ru{,}
\en{that have}\ru{которые имеют}
\en{only}\ru{только}
\en{translational degrees of~freedom}\ru{трансляционные степени свободы}%
\footnote{%
\en{The translational}\ru{Трансляционные (или поступательные)}
\en{degrees of~freedom}\ru{степени свободы}
\en{come}\ru{происходят}
\en{from}\ru{из}
\en{the particle’s ability}\ru{способности частицы}
\en{to~move}\ru{двигаться}
\en{freely}\ru{свободно}
\en{in space}\ru{в~пространстве}.%
}\hbox{\hspace{-0.5ex}.}
\en{Thus}\ru{Поэтому}
\en{there’re no moments among generalized forces}\ru{среди обобщённых сил нет моментов},
\en{and there can’t be any external force couples}\ru{и~никаких внешних пар сил быть не~может}.

\en{Force}\ru{Сила}~${\rho \hspace{-0.1ex} \massloadvector d\mathcal{V}}$
\en{acts}\ru{действует}
\en{on infinitesimal}\ru{на бесконечно-малый}
\en{volume}\ru{объём}~$d\mathcal{V}$.
\en{If}\ru{Если}~$\massloadvector$\en{ is}\ru{\:---}
\en{a~mass force}\ru{массовая сила}
(\en{acting per unit of~mass}\ru{действующая на~единицу массы}),
\en{then}\ru{то}~${\rho \hspace{-0.1ex} \massloadvector}$\en{ is}\ru{\:---}
\en{a~volume one}\ru{объёмная}.
\en{Such forces}\ru{Такие силы}
\en{originate}\ru{происходят}
\en{from force fields}\ru{от силовых полей},
\en{for example}\ru{например}:
\en{the~}\en{gravitational forces}\ru{гравитационные силы}
(\inquotes{\en{weight}\ru{силы тяжести}}),
\en{the~}\en{forces of~inertia}\ru{силы инерции}
\en{in a~non-inertial}\ru{в~неинерциальной}
\en{reference system}\ru{системе отсчёта},
\en{the~}\en{electromagnetic forces}\ru{электромагнитные силы}
\en{in a~medium}\ru{в~среде}
\en{with}\ru{с}~\en{charges}\ru{зарядами}
\en{and}\ru{и}~\en{currents}\ru{токами}.

\en{Surface force}\ru{Поверхностная сила}~${\bm{p} \hspace{.2ex} d\mathcal{O}\hspace{-0.12ex}}$ \en{acts}\ru{действует} \en{on}\ru{на} \en{infinitesimal surface}\ru{бесконечно-малую поверхность}~${d\mathcal{O}\hspace{-0.12ex}}$.
\en{It may be}\ru{Это может быть} \en{a~contact pressure}\ru{контактное давление} \en{or/and}\ru{или/и}~\en{a~friction}\ru{трение}, \en{an electrostatic force}\ru{электростатическая сила} \en{with}\ru{с}\en{ charges} \en{concentrated}\ru{сосредоточенными} \en{on the surface}\ru{на~поверхности}\ru{ зарядами}.

\en{In a~material continuum}\ru{В~материальном \rucontinuum{}е},
\en{like in any mechanical system}\ru{как и в~любой механической системе},
\ru{различаются }\en{the external}\ru{внешние}
\en{and the internal}\ru{и~внутренние}
\en{forces}\ru{силы}\en{ are distinguished}.
\en{The internal forces}\ru{Внутренние силы}
\en{balance}\ru{уравновешивают}
\en{the~action}\ru{действие}
\en{of the external forces}\ru{внешних сил},
\en{and}\ru{и}~\en{they are transmitted}\ru{они передаются}
\en{continuously}\ru{непрерывно}
\en{from point to point}\ru{от~точки к~точке}.
\en{Since the times}\ru{Со~времён}
\en{of~}Euler\ru{’а} \en{and}\ru{и}~Cauchy,
\en{the internal forces}\ru{внутренние силы}
\en{are assumed to be}\ru{предполагаются}
\en{the surface short-range contact forces}\ru{поверхностными контактными силами близкодействия}:
\en{on an~infinitesimal surface}\ru{на бесконечномалой площ\'{а}дке}~${\hspace{-0.1ex}\currentunitnormal d\mathcal{O}\hspace{-0.1ex}}$
\en{acts}\ru{действует}
\en{the~force}\ru{сила}~${\tractionvector{\currentunitnormal} \hspace{.2ex} d\mathcal{O}\hspace{-0.2ex}}$.
\textcolor{magenta}{%
   \en{It acts}\ru{Она действует}
   \en{from?? that side}\ru{с~той?? стороны}
   \en{of the~two}\ru{из двух}\ru{,}
   \en{where}\ru{куда}
   \ru{направлена }\en{the unit normal}\ru{единичная нормаль}~$\currentunitnormal$\en{ is directed}.%
}

\begin{wrapfigure}[13]{o}{.48\textwidth}
\makebox[.48\textwidth][c]{\begin{minipage}[t]{.5\textwidth}
\vspace{-0.8em}
\scalebox{.93}{
\tdplotsetmaincoords{44}{155} % orientation of camera

\begin{tikzpicture}[scale=1, tdplot_main_coords]

% draw perpendicular vectors

\def\unitlength{3}

\draw [line width=1.5pt, black, -{Stealth[round, length=5mm, width=2.8mm]}]
	(0, 0, 0) -- (0, 0, \unitlength)
	node [pos=.94, right, inner sep=0pt, outer sep=6pt]
	{\scalebox{1.2}{${ \bm{n} }$}} ;

\draw [line width=1.5pt, black, -{Stealth[round, length=5mm, width=2.8mm]}]
	(0, 0, 0) -- (0, 0, -\unitlength)
	node [pos=.94, right, inner sep=0pt, outer sep=4.4pt]
	{\scalebox{1.2}{${ - \bm{n} }$}} ;

% draw negative traction vector

\def\tractionvectorlengthx{2.3}
\def\tractionvectorlengthy{0.4}
\def\tractionvectorlengthz{5}

\draw [line width=1.5pt, blue, -{Stealth[round, length=5mm, width=2.8mm]}]
	(0, 0, 0) -- (- \tractionvectorlengthx, - \tractionvectorlengthy, - \tractionvectorlengthz)
	node [pos=.92, right, inner sep=0pt, outer sep=4pt]
	{\scalebox{1.2}{$ \begin{array}{c}
\tractionvector{-\currentunitnormal}
\\[-0.1em]
- \tractionvector{\currentunitnormal}
\end{array} $}} ;

% for the shape of plane and stuff
\def\xlength{1.5}
\def\ylength{1.5}

% draw opacity

\draw [fill=white, fill opacity=.8, line cap=round, line width=0pt, draw=white, draw opacity=.8]
	(\xlength, \ylength, 0)
	-- (0, 0, 0)
	-- (-\xlength, \ylength, 0)
	-- cycle ;

% draw plane

\draw [draw=black, draw opacity=1.0, line cap=round, line width=1.25pt, rounded corners=2pt]
	(-\xlength, -\ylength, 0)
	-- (-\xlength, \ylength, 0)
	-- (\xlength, \ylength, 0)
	-- (\xlength, -\ylength, 0)
	-- cycle ;

\node [inner sep=0pt, outer sep=6pt]
	at (-0.66*\xlength, .66*\ylength, 0)
		{\scalebox{1.2}{${ d\mathcal{O} }$}} ;

% draw positive traction vector

\draw [line width=1.5pt, blue, -{Stealth[round, length=5mm, width=2.8mm]}]
	(0, 0, 0) -- (\tractionvectorlengthx, \tractionvectorlengthy, \tractionvectorlengthz)
	node [pos=.92, right, inner sep=0pt, outer sep=8pt]
	{\scalebox{1.2}{$ \tractionvector{\currentunitnormal} $}} ;

% draw axes
%%\draw [line width=.5pt, blue] (0, 0, 0) -- (1, 0, 0);
%%\draw [line width=.5pt, blue] (0, 0, 0) -- (0, 1, 0);
%%\draw [line width=.5pt, blue] (0, 0, 0) -- (0, 0, 1);

% draw point
\draw [line width=1pt, draw=black, fill=black, opacity=1.0]
	(0, 0, 0) circle (2pt) ;

\end{tikzpicture}}
\vspace{-0.5em}\caption{}\label{fig:perpendicularandtractionvectors}
\end{minipage}}
\end{wrapfigure}

\en{By the~action\hbox{--}reaction principle}\ru{По принципу действия и~противодействия},
\en{a~traction vector}\ru{вектор тракции~(тяги)}~$\tractionvector{\currentunitnormal}$
\en{is reversed}\ru{переворачивается}
(\en{alters direction}\ru{меняет направление})
\en{together}\ru{вместе}
\en{with a~unit normal vector}\ru{с~единичным нормальным вектором}~$\currentunitnormal$:
${\tractionvector{-\currentunitnormal} = - \hspace{.2ex} \tractionvector{\currentunitnormal}\hspace{.2ex}}$.
%
\en{Sometimes}\ru{Иногда}
\en{this thesis}\ru{этот тезис}
\en{is called}\ru{называется}
\inquotes{\en{the}\ru{аргумент-коробка} Cauchy\en{ pillbox argument}}
\en{and is proved}\ru{и~доказывается}
\en{thru}\ru{через}
\en{the~balance of~momentum}\ru{баланс импульса}
\en{for}\ru{для}
\en{an~infinitely short cylinder}\ru{бесконечно короткого цилиндра}
\en{with bases}\ru{с~основаниями}
${\currentunitnormal \hspace{.1ex} d\mathcal{O}\hspace{-0.1ex}}$
\en{and}\ru{и}~${- \hspace{.1ex} \currentunitnormal \hspace{.1ex} d\mathcal{O}\hspace{-0.2ex}}$.

\en{Traction vector}\ru{Вектор тракции}~${\tractionvector{\currentunitnormal}}$ \en{on the~surface}\ru{на~площ\'{а}дке} \en{with the~unit normal}\ru{с~единичной нормалью}~$\currentunitnormal$ \en{is called}\ru{называется} \en{the~}\ru{вектором поверхностной тракции (}surface traction vector\ru{)} \en{or}\ru{или} \en{force\hbox{-}stress vector}\ru{вектором силового напряжения}.
\en{However}\ru{Однако}, ${\tractionvector{\currentunitnormal}}$\en{ is}\ru{\:---} \textcolor{red}{\en{not a~vector field}\ru{не векторное поле}}: \en{traction}\ru{тракция} ${\tractionvectoritself \hspace{.1ex} \narroweq \hspace{.2ex} \tractionvectoritself(\currentunitnormal, \locationvector, t)}$ \en{depends}\ru{зависит} \en{not only}\ru{не~только} \en{on location}\ru{от положения}~$\locationvector$ \en{of~the~point}\ru{точки}, \en{but also}\ru{но также} \en{on the~local direction}\ru{от локального направления} (\en{defined by}\ru{определяемой}~$\currentunitnormal$) \en{of the~surface element}\ru{элемента поверхности}.
\en{An~infinite number}\ru{Бесконечное число}
\en{of~surfaces}\ru{площ\'{а}док}
\en{of any direction}\ru{любого направления}
\en{contain}\ru{содержат}
\en{the~same point}\ru{одну и~ту~же точку},
\en{and there are}\ru{и~имеется}
\en{infinitely many}\ru{бесконечно много}
\en{traction vectors}\ru{векторов тяги~(тракции)}~$\tractionvector{\currentunitnormal}$
\en{at~each point}\ru{в~каждой точке}.

\en{The stress}\ru{Напряжение}
\en{at a~point}\ru{в~точке}
\en{of~continuum}\ru{\rucontinuum{}а}
\en{is}\ru{это}
\emph{\en{not a~vector field}\ru{не~векторное поле}}.
\en{Such a~field}\ru{Такое поле}
\en{is more complex}\ru{более сложное},
\en{it is}\ru{это}
\en{an~infinite collection}\ru{бесконечная коллекция}
\en{of~all}\ru{всех}
\en{traction vectors}\ru{векторов тяги}
\en{for all}\ru{для всех}
\en{infinitesimal surfaces}\ru{бесконечномалых площ\'{а}док}
\en{of any direction}\ru{любого направления},
\en{containing}\ru{содержащих}
\en{that point}\ru{ту точку}.

\en{And in~fact}\ru{И~фактически},
\en{an~infinite collection}\ru{бесконечная коллекция}
\en{of~all traction vectors}\ru{всех векторов тяги}%%~$\tractionvector{\currentunitnormal}$
\en{at a~point}\ru{в~точке}
\en{is completely defined}\ru{полностью определяется}
\en{by the~only one single}\ru{одним-единственным}
\en{second complexity tensor}\ru{тензором второй сложности}\:---
\en{the}\ru{тензором напряжения} Cauchy\en{ stress tensor\hspace{-0.2ex}}~$\cauchystress$.

\en{The~derivation}\ru{Вывод}
\en{of this thesis}\ru{этого тезиса}
\en{is described in many books}\ru{описан во~многих книгах}.
\en{It is known as}\ru{Он известен как}
\en{the~theorem}\ru{теорема}
\en{about existænce}\ru{о~существовании}
\en{of the}\ru{тензора напряжения}~Cauchy\en{ stress tensor}\:---
\en{the~one}\ru{та самая}\ru{,}
\en{with the impressive}\ru{с~впечатляющим}
\en{tetrahedron argument}\ru{аргументом-тетраэдром}.

\emph{%
\ru{Аргумент-тетраэдр}\en{The} Cauchy\en{ tetrahedron argument}
\en{and}\ru{и}
\en{the~proof}\ru{доказательство}
\en{of the~existænce}\ru{существования}
\ru{тензора напряжений }\en{of the }Cauchy\en{ stress tensor}.}

\en{On the~surface}\ru{На~поверхности}
\en{of an~infinitesimal}\ru{бесконечномалого}
\en{material tetrahedron}\ru{материального тетраэдра}
...

......

\en{The traction vector}\ru{Вектор тракции}~$\bm{t}$
\en{and its projections}\ru{и~его проекции},
$\withtheindexofperpendicularity{\bm{t}}$ \en{and}\ru{и}~$\withtheindexofparallelism{\bm{t}}$

\begin{itemize}
%%\item \en{the traction vector}\ru{вектор тяги~(тракции)}~$\bm{t}$,
\item \en{the projection}\ru{проекция}
\en{of the~traction vector}\ru{вектора тяги} %~$\bm{t}$
\en{on the unit normal vector}\ru{на вектор единичной нормали} %~$\bm{n}$
\begin{equation}\label{theprojection.ofthetractionvector.onthenormal}
\withtheindexofperpendicularity{\bm{t}} = \bm{t}_{\bm{n}} = \bm{t} \dotp \bm{n}
\end{equation}
(\en{is perpendicular}\ru{перпендикулярна}
\en{to the cross-section area}\ru{к~площадке поперечного сечения}),
\item \en{the projection}\ru{проекция}
\en{of the~traction vector}\ru{вектора тяги} %~$\bm{t}$
\en{on the~surface}\ru{на площ\'{а}дку}
\begin{equation}\label{theprojection.ofthetractionvector.ontheplane}
\withtheindexofparallelism{\bm{t}} = \bm{t} - \withtheindexofperpendicularity{\bm{t}}
\end{equation}
\end{itemize}

....

\en{\section{Balance of momentum and angular momentum}}

\ru{\section{Баланс импульса и момента импульса}}

\label{section:balance.elasticcontinuum}

\en{Consider}\ru{Рассмотрим} \en{some random}\ru{какой\hbox{-}либо случайный} \en{finite volume}\ru{кон\'{е}чный объём}~$\mathcal{V}$ \en{of an~elastic medium}\ru{упругой среды},
\en{contained within}\ru{содержащийся внутри} \en{surface}\ru{поверхности}~${\mathcal{O}(\boundary \mathcal{V})}$.
\en{It is loaded with external forces}\ru{Он нагружен внешними силами}, \en{surface contact ones}\ru{поверхностными контактными}~${\bm{p} \hspace{.2ex} d\mathcal{O}}$ \en{and}\ru{и} \en{body (mass or volume) ones}\ru{объёмными (или массовыми)}~${\massloadvector dm \hspace{-0.15ex} = \hspace{-0.1ex} \rho \hspace{-0.1ex} \massloadvector d\mathcal{V}}$.

\en{The integral formulation}\ru{Интегральная формулировка}
\en{of the balance of~momentum}\ru{баланса импульса (количества движения)}
\en{is as follows}\ru{таков\'{а}}

\nopagebreak\vspace{-0.2em}\begin{equation}\label{balanceoftranslationalmomentum.integral}
\displaystyle\left( \integral\displaylimits_{\mathcal{V}} \hspace{-0.3ex} \rho \hspace{.2ex} \bm{v} \hspace{.1ex} d\mathcal{V} \hspace{-0.2ex} \right)^{\hspace{-0.3em}\tikz[baseline=-0.2ex]\draw[black, fill=black] (0,0) circle (.28ex);} \hspace{-0.1ex}
=
\integral\displaylimits_{\mathcal{V}} \hspace{-0.33ex} \rho \hspace{-0.1ex} \massloadvector \hspace{.1ex} d\mathcal{V}
\hspace{.25ex} + \hspace{.25ex}
\ointegral\displaylimits_{\mathclap{\mathcal{O}(\boundary \mathcal{V})}} \hspace{-0.2ex} \bm{p} \hspace{.25ex} d\mathcal{O} .
\end{equation}

\noindent
... ${\bm{p} = \hspace{-0.1ex} \tractionvector{\currentunitnormal} \hspace{-0.1ex} = \currentunitnormal \dotp \cauchystress}$ ...

\noindent
\en{The~derivative}\ru{Производная}
\en{of the~momentum}\ru{импульса}
\en{on the~left}\ru{слева}
\en{can be found}\ru{может быть найдена}
\en{as in}\ru{как в}~\eqref{rateofvolumeintegralchange},
\en{and }\ru{а~}%
\en{the~integral over the~surface}\ru{интеграл по поверхности}
\en{turns into }\ru{превращается в~}%
\en{the~volume integral}\ru{объёмный интеграл}
\en{by}\ru{по}
\en{the~divergence theorem}\ru{теореме о~дивергенции}.
\en{This gives}\ru{Это даёт}

\nopagebreak\vspace{-0.1em}\begin{equation*}
\scalebox{.96}{$ \displaystyle\integral\displaylimits_{\mathcal{V}} $} \hspace{-0.3ex} \Bigl( \hspace{-0.1ex} \boldnabla \dotp \cauchystress \hspace{.15ex}
+ \rho \bigl( \massloadvector \hspace{-0.2ex} - \mathdotabove{\bm{v}} \bigr) \hspace{-0.25ex} \Bigr) d\mathcal{V}
= \hspace{.1ex} \zerovector
\hspace{.1ex} .
\end{equation*}

\vspace{-0.25em}\noindent
\en{But}\ru{Но}
\en{volume}\ru{объём}~$\mathcal{V}$
\en{is random}\ru{случаен},
\en{and therefore}\ru{и~поэтому}
\en{the integrand itself}\ru{сам\'{о} подынтегральное выражение}
\en{is also equal}\ru{также равн\'{о}}
\en{to the null vector}\ru{нулевому вектору}\:---
\en{the equation}\ru{уравнение}
\en{of balance}\ru{баланса}
\en{of~momentum}\ru{импульса}
(\en{forces}\ru{сил})
\en{in }\ru{в~}\en{local}\ru{локальной}~(\en{differential}\ru{дифференциальной})
\en{form}\ru{форме}

\nopagebreak\vspace{-0.1em}\begin{equation}\label{balanceoftranslationalmomentum.local}
\boldnabla \dotp \cauchystress \hspace{.15ex}
+ \rho \bigl( \massloadvector \hspace{-0.2ex} - \mathdotabove{\bm{v}} \bigr) \hspace{-0.2ex}
= \hspace{.1ex} \zerovector
\hspace{.1ex} .
\end{equation}

....

\begin{otherlanguage}{russian}

\en{Now}\ru{Теперь}
\en{about the balance}\ru{о~балансе}
\en{of the angular (rotational) momentum}\ru{момента импульса (момента количества движения)}.
\en{Here is the integral formulation}\ru{Вот интегральная формулировка}\::

\nopagebreak\vspace{-0.3em}\begin{equation}\label{balanceofrotationalmomentum.integral}
\displaystyle\left( \integral\displaylimits_{\mathcal{V}} \hspace{-0.3ex} \currentlocationvector \times \hspace{-0.2ex} \rho \hspace{.2ex} \bm{v} \hspace{0.1ex} d\mathcal{V} \hspace{-0.25ex} \right)^{\hspace{-0.32em}\tikz[baseline=-0.2ex]\draw[black, fill=black] (0,0) circle (.28ex);} \hspace{-0.1ex}
= \integral\displaylimits_{\mathcal{V}} \hspace{-0.5ex} \currentlocationvector \times \hspace{-0.2ex} \rho \hspace{-0.1ex} \massloadvector \hspace{.1ex} d\mathcal{V}
\hspace{.25ex} + \hspace{.25ex}
\ointegral\displaylimits_{\mathclap{\mathcal{O}(\boundary \mathcal{V})}} \hspace{-0.25ex} \currentlocationvector \times \hspace{-0.05ex} \bm{p} \hspace{.25ex} d\mathcal{O} .
\end{equation}

\vspace{-0.1em}
Дифференцируя левую часть (${\bm{v} \equiv \hspace{-0.1ex} \mathdotabove{\currentlocationvector}\hspace{.2ex}}$)

\nopagebreak\vspace{-0.2em}\begin{equation*}
\displaystyle\left( \integral\displaylimits_{\mathcal{V}} \hspace{-0.4ex} \currentlocationvector \times \hspace{-0.2ex} \rho \hspace{.1ex} \mathdotabove{\currentlocationvector} \hspace{.4ex} d\mathcal{V} \hspace{-0.2ex} \right)^{\hspace{-0.3em}\tikz[baseline=-0.2ex]\draw[black, fill=black] (0,0) circle (.28ex);} \hspace{-0.1ex}
=
\integral\displaylimits_{\mathcal{V}} \hspace{-0.32ex} \currentlocationvector \times \hspace{-0.2ex} \rho \hspace{.1ex} \mathdotdotabove{\currentlocationvector} \hspace{.4ex} d\mathcal{V}
\hspace{.3ex} +
\integral\displaylimits_{\mathcal{V}} \hspace{-0.3ex} \tikzmark{RdotCrossRdotBegin} \mathdotabove{\currentlocationvector} \times \hspace{-0.2ex} \rho \hspace{.1ex} \mathdotabove{\currentlocationvector} \hspace{.1ex} \tikzmark{RdotCrossRdotEnd} \hspace{.3ex} d\mathcal{V} ,
\end{equation*}
\AddUnderBrace[line width=.75pt][0, -0.1ex][yshift=0.11em]%
{RdotCrossRdotBegin}{RdotCrossRdotEnd}{${\scalebox{.8}{$ \zerovector $}}$}

\vspace{-0.6em}\noindent
применяя теорему о~дивергенции к~поверхностному интегралу ....

(... ${\bm{p} = \hspace{-0.1ex} \tractionvector{\currentunitnormal} \hspace{-0.1ex} = \currentunitnormal \dotp \cauchystress}$ ...)

\nopagebreak\vspace{-0.3em}\begin{multline*}
\currentlocationvector \times \hspace{-0.2ex} \left( \currentunitnormal \dotp \cauchystress \hspace{.1ex} \right)
= \hspace{.1ex} - \hspace{-0.1ex} \left( \currentunitnormal \dotp \cauchystress \hspace{.1ex} \right) \hspace{-0.2ex} \times \hspace{-0.24ex} \currentlocationvector
%
= \hspace{.1ex} - \hspace{.2ex} \currentunitnormal \hspace{.1ex} \dotp \hspace{.1ex} \left( \cauchystress \times \hspace{-0.24ex} \currentlocationvector \hspace{.2ex} \right)
\: \Rightarrow \\
%
\Rightarrow \:
\ointegral\displaylimits_{\mathclap{\mathcal{O}(\boundary \mathcal{V})}} \hspace{-0.25ex} \currentlocationvector \times \hspace{-0.2ex} \left( \currentunitnormal \dotp \cauchystress \hspace{.1ex} \right) \hspace{-0.1ex} d\mathcal{O}
= - \hspace{-0.4ex} \integral\displaylimits_{\mathcal{V}} \hspace{-0.5ex} \boldnabla \dotp \left( \cauchystress \times \hspace{-0.25ex} \currentlocationvector \hspace{.2ex} \right) \hspace{-0.1ex} d\mathcal{V} ,
\end{multline*}

...

\nopagebreak\vspace{-0.2em}\begin{equation*}
\integral\displaylimits_{\mathcal{V}} \hspace{-0.5ex} \currentlocationvector \times \hspace{-0.2ex} \rho \hspace{.1ex} \mathdotdotabove{\currentlocationvector} \hspace{.4ex} d\mathcal{V} \hspace{.22ex}
= \hspace{-0.2ex} \integral\displaylimits_{\mathcal{V}} \hspace{-0.5ex} \currentlocationvector \times \hspace{-0.2ex} \rho \hspace{-0.1ex} \massloadvector \hspace{0.1ex} d\mathcal{V} \hspace{.2ex}
- \hspace{-0.2ex} \integral\displaylimits_{\mathcal{V}} \hspace{-0.5ex} \boldnabla \dotp \left( \cauchystress \times \hspace{-0.24ex} \currentlocationvector \hspace{.2ex} \right) \hspace{-0.1ex} d\mathcal{V} ,
\end{equation*}

\nopagebreak\vspace{-0.2em}\begin{equation*}
\integral\displaylimits_{\mathcal{V}} \hspace{-0.5ex} \currentlocationvector \times \hspace{-0.2ex} \rho \hspace{.15ex} \bigl( \massloadvector \hspace{-0.1ex} - \hspace{-0.1ex} \mathdotdotabove{\currentlocationvector} \hspace{.2ex} \bigr) \hspace{.1ex} d\mathcal{V} \hspace{.12ex}
- \hspace{-0.2ex} \integral\displaylimits_{\mathcal{V}} \hspace{-0.5ex} \boldnabla \dotp \left( \cauchystress \times \hspace{-0.24ex} \currentlocationvector \hspace{.2ex} \right) \hspace{-0.1ex} d\mathcal{V} \hspace{.1ex}
= \hspace{.15ex} \zerovector
\hspace{.1ex} ,
\end{equation*}

...

\nopagebreak\vspace{-0.2em}\begin{equation*}
\tikzmark{divergenceOfStressCrossLocationBegin} \hspace{-0.2ex} \boldnabla \dotp \hspace{.1ex} \left( \cauchystress \times \hspace{-0.25ex} \currentlocationvector \hspace{.2ex} \right) \hspace{-0.33ex} \tikzmark{divergenceOfStressCrossLocationEnd}
= \hspace{-0.33ex} \tikzmark{firstTermOfStressCrossLocationDivergenceBegin} \hspace{.1ex} \left( \hspace{.1ex} \boldnabla \dotp \cauchystress \hspace{.2ex} \right) \hspace{-0.25ex} \times \hspace{-0.2ex} \currentlocationvector \hspace{.2ex} \tikzmark{firstTermOfStressCrossLocationDivergenceEnd} \hspace{.15ex}
+ \hspace{.1ex} \currentlocationvector^{i} \hspace{-0.25ex} \dotp \hspace{-0.1ex} \left( \cauchystress \hspace{-0.1ex} \times \hspace{-0.12ex} \partial_i \currentlocationvector \hspace{.24ex} \right)
\end{equation*}
\AddUnderBrace[line width=.75pt][0.1ex,-0.1ex][yshift=0.11em]%
{divergenceOfStressCrossLocationBegin}{divergenceOfStressCrossLocationEnd}{${\scalebox{0.8}{$ \currentlocationvector^{i} \hspace{-0.25ex} \dotp \partial_i \hspace{-0.25ex} \left( \cauchystress \times \hspace{-0.24ex} \currentlocationvector \hspace{.24ex} \right) $}}$}%
\AddUnderBrace[line width=.75pt][0.2ex,-0.1ex][xshift=-0.1ex, yshift=0.11em]%
{firstTermOfStressCrossLocationDivergenceBegin}{firstTermOfStressCrossLocationDivergenceEnd}{${\scalebox{0.8}{$ \currentlocationvector^{i} \hspace{-0.25ex} \dotp \hspace{-0.1ex} \left( \hspace{.1ex} \partial_i \cauchystress \hspace{.2ex} \right) \hspace{-0.3ex} \times \hspace{-0.2ex} \currentlocationvector $}}$}

\noindent
${\cauchystress = \bm{e}_{i} \tractionvector{i}\hspace{.1ex}}$, ${\bm{e}_i \hspace{-0.2ex} = \boldconstant}$

\nopagebreak\vspace{-0.2em}\begin{multline*}
\currentlocationvector^{i} \hspace{-0.25ex} \dotp \hspace{-0.1ex} \left( \cauchystress \hspace{-0.2ex} \times \hspace{-0.2ex} \partial_i \currentlocationvector \hspace{.25ex} \right) \hspace{-0.1ex}
= \currentlocationvector^{i} \hspace{-0.2ex} \dotp \hspace{-0.1ex} \left( \hspace{.1ex} \bm{e}_{j} \tractionvector{j} \hspace{-0.2ex} \times \hspace{-0.2ex} \currentlocationvector_{\hspace{.1ex}i} \hspace{.1ex} \right) \hspace{-0.15ex}
= \currentlocationvector^{i} \hspace{-0.25ex} \dotp \bm{e}_{j} \tractionvector{j} \hspace{-0.2ex} \times \hspace{-0.2ex} \currentlocationvector_{\hspace{.1ex}i} \hspace{.2ex} =
\\[-0.15em]
%%= - \hspace{.2ex} \currentlocationvector^{i} \hspace{-0.25ex} \dotp \bm{e}_{j} \currentlocationvector_{\hspace{.1ex}i} \hspace{-0.25ex} \times \hspace{-0.25ex} \tractionvector{j}
= - \hspace{.25ex} \bm{e}_{j} \hspace{-0.15ex} \dotp \hspace{-0.1ex} \currentlocationvector^{i} \hspace{-0.2ex} \currentlocationvector_{\hspace{.1ex}i} \hspace{-0.25ex} \times \hspace{-0.25ex} \tractionvector{j}
= - \hspace{.25ex} \bm{e}_{j} \hspace{-0.15ex} \dotp \hspace{-0.1ex} \UnitDyad \hspace{-0.2ex} \times \hspace{-0.25ex} \tractionvector{j}
= - \hspace{.25ex} \bm{e}_{j} \hspace{-0.25ex} \times \hspace{-0.2ex} \tractionvector{j}
= - \hspace{.2ex} \cauchystress_{\hspace{-0.1ex}\Xcompanion}
\end{multline*}

...

\end{otherlanguage}

\en{\section{Eigenvalues of the Cauchy stress tensor. Mohr’s circles}}

\ru{\section{Собственные числа тензора напряжения Cauchy. Круги Mohr’а}}

\begin{otherlanguage}{russian}

\en{Like any}\ru{Как любой}
\en{symmetric bivalent tensor}\ru{симметричный бивалентный тензор},
\ru{тензор напряжения }\en{the }Cauchy\en{ stress tensor}~$\cauchystress$
\en{has}\ru{имеет}
три вещественных собственных числ\'{а}~$\mathsigma_i$, а~также тройку взаимно перпендикулярных собственных векторов единичной длины~(\chapterdotsectionref{chapter:mathapparatus}{section:eigenvectorseigenvalues}).
Собственные числа тензора~$\cauchystress$ называются главными напряжениями (principal stresses).

\en{In~representation}\ru{В~представлении} ${\cauchystress = \hspace{-0.2ex} \sum \hspace{-0.15ex} \mathsigma_i \hspace{.15ex} \bm{e}_i \bm{e}_i}$ \en{most often}\ru{чаще всего} \en{indices}\ru{индексы} \en{are sorted descending}\ru{сортируются по убыванию} ${\mathsigma_1 \hspace{-0.1ex} \geq \mathsigma_2 \geq \mathsigma_3}$, а~тройка~${\bm{e}_i}$ ориентирована как \inquotesx{правая}[.]

Известна теорема о~кругах Мора (Mohr’s circles)%
\footnote{Mohr’s circles, named after Christian Otto Mohr, is a~two-dimensional graphical representation of transformation for the Cauchy stress tensor.}

...



Чтобы замкнуть набор (систему) уравнений модели сплошной среды, нужно добавить определяющие отношения~(constitutive relations)\:--- уравнения, связывающие напряжение с~деформацией (и~другие необходимые связи).
\en{However}\ru{Однако},
\en{for}\ru{для}
\en{a~solid elastic continuum}\ru{твёрдого упругого \rucontinuum{}а}
такой длинный путь построения модели излишен,
что читатель и~увидит ниже.

\end{otherlanguage}

\en{\section{Principle of virtual work (without Lagrange multipliers)}}

\ru{\section{Принцип виртуальной работы (без множителей Lagrange’а)}}

\label{section:virtualworkprinciple.elastic}

\en{According}\ru{Согласно}
\en{to the~principle of~virtual work}\ru{принципу виртуальной работы}
\en{for}\ru{для}
\en{some}\ru{некоего}
\en{finite}\ru{конечного}
\en{volume}\ru{объёма}
\en{of~a~continuous medium}\ru{сплошной среды} %%\en{of~a~continuum}\ru{\rucontinuum{}а}

\nopagebreak\ru{\vspace{-0.1em}}\begin{equation}\label{princlipleofvirtualwork.integral:nonlinearmomentlesscontinuum}
\integral\displaylimits_{\mathcal{V}} \hspace{-0.5ex} \Bigl( \massdensity \hspace{-0.1ex} \massloadvector \dotp \variation{\currentlocationvector} + \variation{\internalwork} \Bigr) \hspace{-0.1ex} d\mathcal{V}
+ \ointegral\displaylimits_{\mathclap{\mathcal{O}(\boundary \mathcal{V})}} \hspace{-0.2ex} \currentunitnormal \dotp \cauchystress \dotp \variation{\currentlocationvector} \hspace{.2ex} d\mathcal{O} = \hspace{.1ex} 0
\hspace{.1ex} .
\end{equation}

\vspace{-0.1em}\noindent
\en{Here}\ru{Здесь}
${\variation{\internalwork}}$\ru{\:---}\en{ is}
\en{the~work of~internal forces}\ru{работа внутренних сил}
\en{per volume unit}\ru{на~единицу объёма}
\en{in the~current configuration}\ru{в~текущей конфигурации},
%
$\massloadvector$\ru{\:---}\en{ is}
\en{the mass force}\ru{массовая сила}
(\en{including dynamics}\ru{включая динамику},
${\massloadvector \hspace{-0.2ex} \equiv \hspace{-0.2ex} \massloadvector_{\hspace{-0.25ex}*} \hspace{-0.3ex} - \mathdotdotabove{\currentlocationvector}\hspace{.25ex}}$),
%
${\bm{p} = \hspace{-0.1ex} \tractionvector{\currentunitnormal} \hspace{-0.1ex} = \currentunitnormal \dotp \cauchystress}$\ru{\:---}\en{ is}
\en{the surface force}\ru{поверхностная сила}.

\en{Applying the~divergence theorem to the~surface integral}\ru{Применяя к~поверхностному интегралу теорему о~дивергенции}, \en{using}\ru{используя}\footnote{%
${ \currentlocationvector^{i} \hspace{-0.25ex} \dotp \partial_i \hspace{-0.1ex} \bigl( \hspace{-0.1ex} \cauchystress \dotp \variation{\currentlocationvector} \hspace{-0.1ex} \bigr) \hspace{-0.25ex}
= \currentlocationvector^{i} \hspace{-0.25ex} \dotp \hspace{-0.2ex} \bigl( \partial_i \cauchystress \hspace{.1ex} \bigr) \hspace{-0.25ex} \dotp \variation{\currentlocationvector}
+ \currentlocationvector^{i} \hspace{-0.25ex} \dotp \hspace{-0.1ex} \cauchystress \dotp \partial_i \hspace{-0.1ex} \bigl( \variation{\currentlocationvector} \hspace{-0.1ex} \bigr)
\hspace{-0.1ex} , }$

\hspace*{\fill}
${ \currentlocationvector^{i} \hspace{-0.25ex} \dotp \hspace{-0.1ex} \cauchystress \dotp \partial_i \hspace{-0.1ex} \bigl( \variation{\currentlocationvector} \hspace{-0.1ex} \bigr) \hspace{-0.25ex}
= \cauchystress \dotdotp \partial_i \hspace{-0.1ex} \bigl( \variation{\currentlocationvector} \hspace{-0.1ex} \bigr) \currentlocationvector^{i} \hspace{-0.2ex}
= \cauchystress \dotdotp \hspace{-0.12ex} \bigl( \currentlocationvector^{i} \partial_i \hspace{.12ex} \variation{\currentlocationvector} \bigr)^{\raisemath{-0.1em}{\hspace{-0.4ex}\T}} }$%
}

\nopagebreak\vspace{-0.1em}\begin{equation*}
\boldnabla \hspace{-0.1ex} \dotp \hspace{-0.1ex} \left( \cauchystress \dotp \variation{\currentlocationvector} \right) \hspace{-0.1ex}
= \boldnabla \hspace{-0.1ex} \dotp \cauchystress \dotp \variation{\currentlocationvector}
+ \cauchystress \dotdotp \hspace{-0.12ex} \boldnabla \hspace{.1ex} \variation{\currentlocationvector}^{\T}
\end{equation*}

\vspace{-0.4em}\noindent
\en{and the~randomness of}\ru{и~случайность}~${\mathcal{V}\hspace{-0.2ex}}$,
\en{here comes}\ru{получается}
\en{the~local differential version of}\ru{локальную дифференциальную версию}~\eqref{princlipleofvirtualwork.integral:nonlinearmomentlesscontinuum}

\nopagebreak\vspace{-0.2em}\begin{equation}\label{princlipleofvirtualwork.local:nonlinearmomentlesscontinuum}
\Bigl( \hspace{-0.2ex} \boldnabla \dotp \cauchystress \hspace{.1ex} + \hspace{-0.1ex} \massdensity \hspace{-0.1ex} \massloadvector \Bigr) \hspace{-0.4ex} \dotp \variation{\currentlocationvector}
+ \cauchystress \dotdotp \hspace{-0.12ex} \boldnabla \hspace{.1ex} \variation{\currentlocationvector}^{\T} \hspace{-0.33ex}
+ \hspace{.1ex} \variation{\internalwork}
= \hspace{.1ex} 0 \hspace{.1ex}.
\end{equation}

\vspace{-0.25em}
\en{When}\ru{Когда}
\en{a~body}\ru{тело}
\en{virtually moves}\ru{виртуально движется}
\en{as a~rigid whole}\ru{как жёсткое целое},
\en{the~work}\ru{работа}
\en{of~internal forces}\ru{внутренних сил}
\en{nullifies}\ru{обнуляется}

\nopagebreak\vspace{-0.25em}\begin{equation*}
%\label{princlipleofvirtualwork.worknullifies:nonlinearmomentlesscontinuum}
\begin{array}{c}
\variation{\currentlocationvector} = \constvarvector{\hspace{-0.1ex}\bm{\rho}} \hspace{.1ex} + \constvarvector{o} \hspace{-0.33ex} \times \hspace{-0.33ex} \currentlocationvector
%%\hspace{.1ex} , \:
%%\constvarvector{\hspace{-0.1ex}\bm{\rho}} = \boldconstant
%%\hspace{.1ex} , \:
%%\constvarvector{o} = \boldconstant
\hspace{.66ex}\Rightarrow\hspace{.4ex}
\variation{\internalwork} \hspace{-0.2ex} = 0
\hspace{.1ex} ,
\\[.2em]
%
\bigl( \hspace{.1ex} \boldnabla \dotp \cauchystress \hspace{.1ex} + \hspace{-0.1ex} \massdensity \hspace{-0.1ex} \massloadvector \hspace{.16ex} \bigr)
\hspace{-0.25ex} \dotp \hspace{-0.25ex}
\bigl(  \constvarvector{\hspace{-0.1ex}\bm{\rho}} \hspace{.1ex} + \constvarvector{o} \hspace{-0.33ex} \times \hspace{-0.33ex} \currentlocationvector \hspace{.2ex} \bigr) \hspace{-0.16ex}
+ \cauchystress^{\hspace{.16ex}\T} \hspace{-0.5ex}
\dotdotp
\hspace{-0.16ex} \boldnabla \hspace{.1ex} \bigl(
\constvarvector{\hspace{-0.1ex}\bm{\rho}} \hspace{.1ex} +
\constvarvector{o} \hspace{-0.33ex} \times \hspace{-0.33ex} \currentlocationvector \hspace{.2ex} \bigr) \hspace{-0.25ex}
= 0 \hspace{.1ex} ,
\\[.2em]
%
\constvarvector{\hspace{-0.1ex}\bm{\rho}} = \boldconstant
\hspace{.4ex} \Rightarrow \hspace{.2ex}
\boldnabla \hspace{.1ex} \constvarvector{\hspace{-0.1ex}\bm{\rho}} =  \hspace{-0.12ex} \zerobivalent \hspace{.1ex} ,
\;\:
\constvarvector{o} = \boldconstant
\hspace{.4ex} \Rightarrow \hspace{.2ex}
\boldnabla \hspace{.1ex} \constvarvector{o} = \hspace{-0.12ex} \zerobivalent \hspace{.1ex} ,
\\[.2em]
%
\boldnabla \hspace{.1ex} \bigl(
\constvarvector{\hspace{-0.1ex}\bm{\rho}} \hspace{.1ex} +
\constvarvector{o} \hspace{-0.33ex} \times \hspace{-0.33ex} \currentlocationvector \hspace{.2ex} \bigr) \hspace{-0.25ex}
= \hspace{-0.1ex} \boldnabla \hspace{.1ex} \bigl(
\constvarvector{o} \hspace{-0.33ex} \times \hspace{-0.33ex} \currentlocationvector \hspace{.2ex} \bigr) \hspace{-0.25ex}
= \boldnabla \hspace{.1ex} \constvarvector{o} \hspace{-0.33ex} \times \hspace{-0.33ex} \currentlocationvector
\hspace{.1ex} - \hspace{-0.2ex}
\boldnabla \currentlocationvector \times \hspace{-0.2ex} \constvarvector{o} =
\\[.1em] %
\hspace*{\fill}
= - \boldnabla \currentlocationvector \hspace{-0.1ex} \times \hspace{-0.2ex} \constvarvector{o}
= - \hspace{.1ex} \UnitDyad \hspace{-0.16ex} \times \hspace{-0.2ex} \constvarvector{o}
%%= - \hspace{.1ex} \constvarvector{o} \hspace{-0.3ex} \times \hspace{-0.33ex} \UnitDyad
\\[.2em]
%
\cdots
%%\hspace{.1ex} .
\end{array}
\end{equation*}

\vspace{-0.2em}
\en{Assuming}\ru{Полагая}
${\constvarvector{o} \hspace{-0.1ex} = \zerovector}$
(\en{just a~translation}\ru{лишь трансляция})
${\Rightarrow}$~${\hspace{-0.1ex} \boldnabla \hspace{.1ex} \variation{\currentlocationvector}
= \hspace{-0.2ex} \boldnabla \hspace{.1ex} \constvarvector{\hspace{-0.1ex}\bm{\rho}}
= \hspace{-0.12ex} \zerobivalent}$,
\en{it turns into the~balance of~forces~(of~momentum)}\ru{оно превращается в~баланс сил~(импульса)}

\nopagebreak\vspace{-0.2em}\begin{equation*}
\boldnabla \dotp \cauchystress \hspace{.1ex} + \hspace{-0.1ex} \massdensity \hspace{-0.1ex} \massloadvector \hspace{-0.1ex} = \zerovector
\hspace{.1ex} .
\end{equation*}

\vspace{-0.1em}
\en{If}\ru{Если} ${\variation{\currentlocationvector} = \constvarvector{o} \hspace{-0.33ex} \times \hspace{-0.33ex} \currentlocationvector}$ (\en{just rotation}\ru{лишь поворот}) \en{with}\ru{с}~${\constvarvector{o} \hspace{-0.1ex} = \boldconstant}$, \en{then}\ru{то}

\nopagebreak\vspace{-0.1em}\begin{equation*}
\begin{array}{r@{\hspace{.8ex}}l}
\eqrefwithchapterdotsection{gradientofcrossproductoftwovectors}{chapter:mathapparatus}{section:spatialdifferentiation}
\,\Rightarrow &
\boldnabla \hspace{.1ex} \variation{\currentlocationvector}
= \boldnabla \hspace{.1ex} \constvarvector{o} \hspace{-0.33ex} \times \hspace{-0.33ex} \currentlocationvector
\hspace{.1ex} - \hspace{-0.2ex}
\boldnabla \currentlocationvector \times \hspace{-0.2ex} \constvarvector{o}
= - \hspace{.1ex} \UnitDyad \hspace{-0.16ex} \times \hspace{-0.2ex} \constvarvector{o}
%%= - \hspace{.2ex} \constvarvector{o} \hspace{-0.3ex} \times \hspace{-0.33ex} \UnitDyad
\hspace{.1ex} ,
\\[.25em]
%
& \boldnabla \hspace{.1ex} \variation{\currentlocationvector}^{\T} \hspace{-0.32ex}
= \UnitDyad \hspace{-0.16ex} \times \hspace{-0.2ex} \constvarvector{o}
%%= \constvarvector{o} \hspace{-0.3ex} \times \hspace{-0.33ex} \UnitDyad
\end{array}
\end{equation*}

\noindent
With

\nopagebreak\vspace{-0.8em}\begin{equation*}
\begin{array}{c}
\eqrefwithchapterdotsection{pseudovectorinvariant}{chapter:mathapparatus}{section:tensors.symmetric+skewsymmetric} \:\Rightarrow\,
\cauchystress_{\hspace{-0.1ex}\Xcompanion} \hspace{-0.1ex} = - \hspace{.1ex} \cauchystress \hspace{.1ex} \dotdotp \permutationsparitytensor
\hspace{.1ex} ,
\\[.2em]
%
\cauchystress \dotdotp \hspace{-0.32ex} \left( \hspace{.1ex} \UnitDyad \hspace{-0.16ex} \times \hspace{-0.2ex} \constvarvector{o} \hspace{.1ex} \right) \hspace{-0.1ex}
= \cauchystress \dotdotp \hspace{-0.4ex} \left( \hspace{-0.1ex} - \hspace{.2ex} \permutationsparitytensor \dotp \constvarvector{o} \hspace{.1ex} \right) \hspace{-0.1ex}
= \left( \hspace{-0.1ex} - \hspace{.1ex} \cauchystress \dotdotp \permutationsparitytensor \hspace{.1ex} \right) \hspace{-0.3ex} \dotp \constvarvector{o} \hspace{.1ex}
= \cauchystress_{\hspace{-0.1ex}\Xcompanion} \dotp \hspace{.1ex} \constvarvector{o}
\end{array}
\end{equation*}

...

\en{In an~elastic continuum}\ru{В~упругой среде}\en{,} \en{the~internal forces}\ru{внутренние силы}
\en{are potential}\ru{потенциальны}

\nopagebreak\vspace{-0.2em}\begin{equation*}
\variation{\internalwork} = - \massdensity \hspace{.2ex} \variation{\widetilde{\potentialenergydensity}}
%%\hspace{.1ex} .
\end{equation*}


...




%----------
{\small
\setlength{\parindent}{0pt}

\begin{leftverticalbar}%%[oversize]

\inquotes{The elastic potential energy density per volume unit},
becomes when shorting
\inquotes{The elastic potential ....}

\begin{otherlanguage}{russian}
Плотность
упругой потенциальной энергии,
запасённой|накопленной
в~единице объёма тела (сред\'{ы}).
\end{otherlanguage}

\begin{otherlanguage}{russian}
Дословный перевод
c~english на~русский
фразы
\inquotesx{the elastic potential}
даёт
\inquotesx{упругий потенциал}[.]
\end{otherlanguage}

\end{leftverticalbar}
\par}
%----------


....


\begin{equation}
\cauchystress \dotdotp \hspace{-0.12ex} \boldnabla \hspace{.1ex} \variation{\currentlocationvector}^{\mathsf{S}} \hspace{-0.25ex}
= \hspace{-0.2ex} - \hspace{.2ex} \variation{\internalwork}
= \massdensity \hspace{.2ex} \variation{\widetilde{\potentialenergydensity}}
\end{equation}

...

\begin{otherlanguage}{russian}

Вид потенциала ${\widetilde{\potentialenergydensity}}$ \en{per mass unit}\ru{на~единицу \hbox{массы}} пока неизвестен, но очевидно что ${\widetilde{\potentialenergydensity}}$ определяется деформацией.

\en{With}\ru{С}~\en{the~balance of~mass}\ru{балансом массы}
${\massdensity \hspace{.2ex} J \hspace{-0.1ex} = \initialmassdensity \hspace{.5ex} \Leftrightarrow \hspace{.1ex} \massdensity = \hspace{-0.1ex} J^{\expminusone} \initialmassdensity}$
(${J \hspace{-0.1ex} \equiv \determinant \bm{F}\hspace{-0.12ex}}$\en{ is}\ru{\:---} \en{the }Jacobian, \en{determinant of the motion gradient}\ru{определитель градиента движения}),
потенциал на~единицу объёма в~недеформированной конфигурации~$\smash{\mathcircabove{\potentialenergydensity}}$ имеет вид

\nopagebreak\vspace{-0.2em}\begin{equation}
\begin{array}{c}
\mathcircabove{\potentialenergydensity} \equiv \initialmassdensity \hspace{.4ex} \widetilde{\potentialenergydensity}
\hspace{.4ex} \Rightarrow \hspace{.2ex}
\variation{\mathcircabove{\potentialenergydensity}} = \initialmassdensity \hspace{.25ex} \variation{\widetilde{\potentialenergydensity}}
\hspace{.1ex} ,
\\[.1em]
%
\massdensity \hspace{.2ex} \variation{\widetilde{\potentialenergydensity}} = \hspace{-0.1ex} J^{\expminusone} \hspace{.1ex} \variation{\mathcircabove{\potentialenergydensity}}
\hspace{.1ex} .
\end{array}
\end{equation}

\vspace{-0.25em}
Полным аналогом~(...) является равенство

...



\end{otherlanguage}


\en{\section{Constitutive relations of elasticity}}

\ru{\section{Определяющие отношения упругости}}

\en{The fundamental relation of~elasticity}\ru{Фундаментальное соотношение упругости}~\eqref{fundamentalrelationofelasticity}

...

{\small

\[ \potentialenergydensity (\bm{C}) = \displaystyle \integral_{\raisemath{-0.25em}{\hspace{-0.1ex}\scalebox{0.85}{$0$}}}^{\raisemath{.15em}{\bm{C}}} \hspace{-0.25ex} \cauchystress \hspace{.1ex} \dotdotp d \hspace{.1ex} \boldsymbol{\bm{C}} \]

If the strain energy density is path independent, then it acts as a~potential for stress, that is
\[ \displaystyle \cauchystress = {\frac{\partial \potentialenergydensity (\bm{C})}{\partial \bm{C}}} \]

For adiabatic processes, ${\potentialenergydensity}$ is equal to the change in internal energy per unit of volume.

For isothermal processes, ${\potentialenergydensity}$ is equal to the Helmholtz free energy per unit of volume.

The natural configuration of a~body is defined as the configuration in which the body is in stable thermal equilibrium with no external loads and zero stress and strain.

When we apply energy methods in elasticity, we implicitly assume that a~body returns to its natural configuration after loads are removed. This implies that the Gibbs’ condition is satisfied:
\[ \potentialenergydensity (\bm{C}) \geq 0~~{\text{with}}~~\potentialenergydensity (\bm{C}) = 0~~{\text{iff}}~~\bm{C} = 0 \]

\par}

...

\begin{otherlanguage}{russian}

\noindent
Начальная конфигурация
считается
естественной (natural configuration)\:---
недеформированной ненапряжённой\::
${\bm{C} = \hspace{-0.1ex} \zerobivalent \hspace{.4ex} \Leftrightarrow \hspace{.2ex} \cauchystress
= \hspace{-0.2ex} \zerobivalent}$,
поэтому
в~$\potentialenergydensity$
нет линейных членов.

Тензор жёсткости~$\stiffnesstensor$

...

A~rubber\hbox{-}like material (an elastomer)

Для материала типа резины~(эластомера)
характерны больш\'{и}е деформации.
Функция~${\potentialenergydensity\hspace{.12ex}(\anyfirstinvariant, \anysecondinvariant, \anythirdinvariant)}$
для такого материала
бывает весьма сложной%
\footnote{\bookauthor{Harold Alexander}. \href{https://kundoc.com/pdf-a-constitutive-relation-for-rubber-like-materials-.html}{A~constitutive relation for rubber-like ma\-te\-ri\-als~// International Journal of~Engineering Science, volume~6 (September 1968), pages 549\hbox{--}563.}}\hbox{\hspace{-0.5ex}.}
% https://www.researchgate.net/publication/232329906_A_constitutive_relation_for_rubber-like_materials

Преимущества использования~$\bm{u}$ и~$\bm{C}$
исчезают,
если деформации
больш\'{и}е (кон\'{е}чные)\:---
проще остаться с~вектором\hbox{-}радиусом~$\currentlocationvector$ ...

...


\end{otherlanguage}

\newpage

\en{\section{Piola\hbox{--}Kirchhoff stress tensors and other measures of~stress}}

\ru{\section{Тензоры напряжения Piola\hbox{--}Kirchhoff’а и~другие меры напряжения}}

\label{section:piolakirchhoffstresstensor}

\begin{otherlanguage}{russian}

Соотношение Nanson’а ${\currentunitnormal \hspace{.1ex} d\mathcal{O} = J \hspace{.1ex} \initialunitnormal \hspace{.1ex} do \dotp \bm{F}^{\hspace{.1ex}\expminusone} \hspace{-0.1ex}}$ между векторами бесконечно малой площ\'{а}дки в~начальной~(${\initialunitnormal \hspace{.1ex} do}$) и~в~текущей~(${\currentunitnormal \hspace{.1ex} d\mathcal{O}}$) конфигурациях%
\footnote{\en{Like before}\ru{По\hbox{-}прежнему},
${\bm{F} \hspace{-0.1ex}
= \scalebox{0.8}{$ \displaystyle \frac{\raisemath{-0.2em}{\partial \currentlocationvector}}{\raisemath{-0.1em}{\partial \initiallocationvector}} $}
= \hspace{-0.1ex} \currentlocationvector_\differentialindex{i} \initiallocationvector^{i} \hspace{-0.2ex}
= \hspace{-0.2ex} \boldnablacircled \currentlocationvector^{\T} \hspace{-0.2ex}}$\ru{\:---}\en{ is} \en{motion gradient}\ru{градиент движения}, ${J \hspace{-0.1ex} \equiv \determinant \bm{F}\hspace{-0.12ex}}$\en{ is}\ru{\:---} \en{the Jacobian}\ru{якобиан} (\en{the Jacobian determinant}\ru{определитель Якоби}).}

\nopagebreak\vspace{-0.12em}\begin{equation*}
\eqref{areachange:nansonformula}
\:\Rightarrow\,
\currentunitnormal \hspace{.1ex} d\mathcal{O} \dotp \cauchystress
= J \hspace{.1ex} \initialunitnormal do \dotp \bm{F}^{\hspace{.1ex}\expminusone} \hspace{-0.2ex} \dotp \cauchystress
\:\Rightarrow\,
\currentunitnormal \dotp \cauchystress \hspace{.25ex} d\mathcal{O}
= \hspace{.1ex} \initialunitnormal \dotp J \bm{F}^{\hspace{.1ex}\expminusone} \hspace{-0.2ex} \dotp \cauchystress \hspace{.2ex} do
\end{equation*}

\vspace{-0.2em} \noindent \en{gives the~dual expression of a~surface force}\ru{даёт двоякое выражение поверхностной силы}

\nopagebreak\vspace{-0.16em}\begin{equation}\label{dualexpressionofsurfaceforce}
\currentunitnormal \dotp \cauchystress \hspace{.25ex} d\mathcal{O}
= \initialunitnormal \dotp \hspace{.12ex} \firstpiolakirchhoffstress \hspace{.1ex} do
\hspace{.1ex}, \:\:
\firstpiolakirchhoffstress \hspace{.1ex} \equiv J \bm{F}^{\hspace{.1ex}\expminusone} \hspace{-0.2ex} \dotp \cauchystress \hspace{.1ex}.
\end{equation}

\vspace{-0.2em}
Тензор~${\hspace{.1ex}\firstpiolakirchhoffstress}$ называется первым~(несимметричным) тензором напряжения Piola--Kirchhoff\ru{’а}, иногда\:--- \inquotes{номинальным напряжением} (\inquotes{nominal stress}) или \inquotes{инженерным напряжением} (\inquotes{engineering stress}). Бывает и~когда какое\hbox{-}либо из этих (на)именований даётся транспонированному тензору

\nopagebreak\vspace{-0.1em}\begin{equation*}
\firstpiolakirchhoffstress^{\hspace{.1ex}\T} \hspace{-0.32ex}
= J \cauchystress^{\hspace{.16ex}\T} \hspace{-0.32ex} \dotp \bm{F}^{\hspace{.1ex}\expminusT} \hspace{-0.32ex}
= J \cauchystress \hspace{.16ex} \dotp \bm{F}^{\hspace{.1ex}\expminusT} \hspace{-0.25ex}.
\end{equation*}

Обращение~\eqref{dualexpressionofsurfaceforce}

\begin{equation*}
J^{\hspace{.12ex}\expminusone} \bm{F} \dotp \hspace{.16ex} \firstpiolakirchhoffstress = J^{\hspace{.12ex}\expminusone} \bm{F} \dotp J \bm{F}^{\hspace{.16ex}\expminusone} \hspace{-0.2ex} \dotp \cauchystress
\:\,\Rightarrow\:
\cauchystress = J^{\hspace{.12ex}\expminusone} \bm{F} \dotp \hspace{.16ex} \firstpiolakirchhoffstress
\end{equation*}

...



\nopagebreak\vspace{-0.2em}\begin{equation}
\variation{\potentialenergydensity} = \hspace{.1ex} \firstpiolakirchhoffstress \hspace{-0.12ex} \dotdotp \hspace{.1ex} \variation{\hspace{.1ex} \boldnablacircled \currentlocationvector^{\hspace{.1ex}\T}}
\hspace{.1ex} \Rightarrow \hspace{.32ex}
\potentialenergydensity \hspace{-0.32ex}=\hspace{-0.25ex} \potentialenergydensity (\boldnablacircled \currentlocationvector\hspace{.1ex})
\end{equation}

\vspace{-0.2em} \noindent
--- этот немного неожиданный результат получился благодаря коммутативности $\variation$ и~$\smash{\hspace{-0.1ex}\boldnablacircled\hspace{.1ex}}$: ${\boldnablacircled \hspace{.1ex} \variation{\currentlocationvector}^{\hspace{.1ex}\T} \hspace{-0.4ex} = \variation{\hspace{.1ex} \boldnablacircled \currentlocationvector^{\hspace{.1ex}\T}}\hspace{-0.25ex}}$ ($\boldnabla$ \en{and}\ru{и}~$\variation$ \en{don’t commute}\ru{не~коммутируют}).

% энергетически сопряжённый с ... = energy conjugate to ...

Тензор~$\firstpiolakirchhoffstress$ оказался энергетически сопряжённым с~${\bm{F} \equiv \hspace{-0.1ex} \smash{\boldnablacircled} \currentlocationvector^{\hspace{.1ex}\T}}$

\nopagebreak\vspace{-0.12em}\begin{equation}
\firstpiolakirchhoffstress \hspace{-0.1ex}
= \scalebox{0.92}{$ \displaystyle \frac{\partial \hspace{.1ex} \potentialenergydensity}{\raisemath{-0.4em}{\partial \hspace{.1ex} \smash{\boldnablacircled} \currentlocationvector^{\hspace{.1ex}\T}}} $}
= \scalebox{0.92}{$ \displaystyle \frac{\hspace{.2ex} \partial \hspace{.1ex} \potentialenergydensity \hspace{.2ex}}{\raisemath{-0.25em}{\partial \bm{F}}} $} \hspace{.25ex}.
\vspace{.1em}\end{equation}

Второй~(симметричный) тензор напряжения Piola--Kirch\-hoff\ru{’а} $\secondpiolakirchhoffstress$ энергетически сопряжён с~${\bm{G} \equiv \hspace{-0.1ex} \bm{F}^{\hspace{.1ex}\T} \hspace{-0.36ex} \dotp \bm{F}}$ и~${\bm{C} \hspace{-0.1ex} \equiv \smalldisplaystyleonehalf \hspace{.1ex} (\bm{G} - \hspace{-0.12ex} \UnitDyad \hspace{.1ex})}$

\nopagebreak\vspace{-0.4em}\begin{equation}
\begin{array}{c}
\variation{\potentialenergydensity}(\bm{C}\hspace{.1ex}) \hspace{-0.2ex} = \secondpiolakirchhoffstress \dotdotp \variation{\hspace{.1ex} \bm{C}}
\hspace{.25ex} \Rightarrow \hspace{.32ex}
%%\potentialenergydensity \hspace{-0.32ex}=\hspace{-0.25ex} \potentialenergydensity (\bm{C}\hspace{.1ex}) \hspace{.1ex} ,
%%\:\:
\secondpiolakirchhoffstress = \scalebox{0.92}{$ \displaystyle \frac{\partial \hspace{.1ex} \potentialenergydensity}{\raisemath{-0.1em}{\partial \hspace{.1ex} \bm{C}}} $} \hspace{.24ex} , \\[.5em]
%
d\bm{G} \hspace{-0.12ex} = 2 \hspace{.2ex} d \bm{C}
\hspace{.25ex} \Rightarrow \hspace{.32ex}
\variation{\potentialenergydensity}(\bm{G}\hspace{.1ex}) \hspace{-0.2ex} = \hspace{.1ex} \smalldisplaystyleonehalf \hspace{.2ex} \secondpiolakirchhoffstress \dotdotp \variation{\hspace{.1ex} \bm{G}} ,
\:\,
\secondpiolakirchhoffstress \hspace{-0.1ex} = 2 \hspace{.25ex} \scalebox{0.92}{$ \displaystyle \frac{\partial \hspace{.1ex} \potentialenergydensity}{\raisemath{-0.1em}{\partial \hspace{.1ex} \bm{G}}} $} \hspace{.25ex}.
\end{array}
\end{equation}

Связь между первым и~вторым тензорами

\nopagebreak\vspace{-0.12em}\begin{equation*}
\secondpiolakirchhoffstress \hspace{-0.1ex}
= \firstpiolakirchhoffstress \hspace{-0.1ex} \dotp \bm{F}^{\hspace{.1ex}\expminusT} \hspace{-0.4ex}
= \bm{F}^{\hspace{.1ex}\expminusone} \hspace{-0.3ex} \dotp \hspace{.1ex} \firstpiolakirchhoffstress^{\hspace{.1ex}\T}
%
\hspace{.25ex} \Leftrightarrow \hspace{.75ex}
%
\firstpiolakirchhoffstress = \secondpiolakirchhoffstress \dotp \bm{F}^{\hspace{.1ex}\T}
\hspace{-0.4ex} , \:\:
\firstpiolakirchhoffstress^{\hspace{.1ex}\T} \hspace{-0.4ex} = \bm{F} \dotp \secondpiolakirchhoffstress
\end{equation*}

\vspace{-0.2em}\noindent
и между тензором~$\secondpiolakirchhoffstress$ и~тензором напряжения Cauchy~$\cauchystress$

\nopagebreak\vspace{-0.12em}\begin{equation*}
\secondpiolakirchhoffstress = J \bm{F}^{\hspace{.1ex}\expminusone} \hspace{-0.2ex} \dotp \cauchystress \hspace{.16ex} \dotp \bm{F}^{\hspace{.1ex}\expminusT}
\hspace{.2ex} \Leftrightarrow \hspace{.5ex}
J^{\hspace{.12ex}\expminusone} \bm{F} \hspace{-0.1ex} \dotp \secondpiolakirchhoffstress \dotp \bm{F}^{\hspace{.1ex}\T} \hspace{-0.32ex}
= \cauchystress \hspace{.1ex}.
\end{equation*}

...

\begin{equation*}
\firstpiolakirchhoffstress \hspace{-0.1ex}
= \scalebox{0.92}{$ \displaystyle \frac{\partial \hspace{.1ex} \potentialenergydensity}{\raisemath{-0.1em}{\partial \hspace{.1ex} \bm{C}}} $} \dotp \hspace{-0.1ex} \bm{F}^{\hspace{.1ex}\T} \hspace{-0.5ex}
= 2 \hspace{.25ex} \scalebox{0.92}{$ \displaystyle \frac{\partial \hspace{.1ex} \potentialenergydensity}{\raisemath{-0.1em}{\partial \hspace{.1ex} \bm{G}}} $} \dotp \hspace{-0.1ex} \bm{F}^{\hspace{.1ex}\T}
\end{equation*}

\begin{equation*}
\variation{\secondpiolakirchhoffstress}
= \scalebox{0.92}{$ \displaystyle \frac{\partial \hspace{.1ex} \bm{S}}{\raisemath{-0.1em}{\partial \hspace{.1ex} \bm{C}}} $} \dotdotp \variation{\hspace{.12ex}\bm{C}}
= \scalebox{0.92}{$ \displaystyle \frac{\partial^2 \hspace{0.1ex} \potentialenergydensity}{\raisemath{-0.1em}{\partial \hspace{0.1ex} \bm{C} \hspace{0.1ex} \partial \hspace{0.1ex} \bm{C}}} $} \dotdotp \variation{\hspace{.12ex}\bm{C}}
\end{equation*}

\begin{equation*}
\variation{\hspace{.1ex}\firstpiolakirchhoffstress} \hspace{-0.2ex} =
\variation{\secondpiolakirchhoffstress} \dotp \bm{F}^{\hspace{.1ex}\T} \hspace{-0.4ex} + \hspace{.1ex}
\secondpiolakirchhoffstress \dotp \variation{\bm{F}}^{\hspace{.1ex}\T}
\end{equation*}

...

{\small
The quantity ${\bm{\kappa} = J \cauchystress}$ is called the \emph{Kirchhoff stress tensor} and is used widely in numerical algorithms in metal plasticity (where there’s no change in volume during plastic deformation).
Another name for it is \emph{weighted Cauchy stress tensor}.
\par}

...

\end{otherlanguage}

\en{Here’s balance of~forces~(of~momentum) with tensor~${\hspace{.1ex}\firstpiolakirchhoffstress}$ for any undeformed volume~$\mathcircabove{\mathcal{V}}$}

\ru{Вот баланс сил~(импульса) с~тензором~${\hspace{.1ex}\firstpiolakirchhoffstress}$ для любого недеформированного объёма~$\mathcircabove{\mathcal{V}}$}

\nopagebreak\ru{\vspace{-0.12em}}\begin{equation*}
\scalebox{0.96}[0.94]{$ \displaystyle \integral\displaylimits_{\mathcal{V}} \hspace{-0.4ex} \rho \hspace{-0.1ex} \massloadvector \hspace{.1ex} d\mathcal{V} $} + \hspace{-0.2ex}
\scalebox{0.96}[0.94]{$ \displaystyle \integral\displaylimits_{\mathclap{\mathcal{O}(\boundary \mathcal{V})}} \hspace{-0.5ex} \currentunitnormal \hspace{-0.12ex} \dotp \hspace{-0.1ex} \cauchystress \hspace{.2ex} d\mathcal{O} $}
= \hspace{-0.2ex}
\scalebox{0.96}[0.94]{$ \displaystyle \integral\displaylimits_{\mathcircabove{\mathcal{V}}} \hspace{-0.4ex} \mathcircabove{\rho} \hspace{-0.1ex} \massloadvector \hspace{.12ex} d \mathcircabove{\mathcal{V}} $} + \hspace{-0.2ex}
\scalebox{0.96}[0.94]{$ \displaystyle \integral\displaylimits_{\mathclap{o \hspace{.1ex} (\boundary \smash{\mathcircabove{\mathcal{V}}})}} \hspace{-0.5ex} \initialunitnormal \hspace{-0.12ex} \dotp \firstpiolakirchhoffstress \hspace{.2ex} do $}
= \hspace{-0.2ex}
\scalebox{0.96}[0.94]{$ \displaystyle \integral\displaylimits_{\mathcircabove{\mathcal{V}}} \hspace{-0.5ex}
\left(^{\mathstrut} \hspace{-0.1ex} \mathcircabove{\rho} \hspace{-0.1ex} \massloadvector \hspace{-0.12ex} + \hspace{-0.4ex} \boldnablacircled \dotp \firstpiolakirchhoffstress \right) \hspace{-0.4ex} d \mathcircabove{\mathcal{V}} $} \hspace{-0.25ex}
= \zerovector
\vspace{-0.25em}\end{equation*}

\noindent
\en{or}\ru{или} \en{in the~local~(differential) version}\ru{в~локальной~(дифференциальной) версии}

\nopagebreak\vspace{-0.25em}
\begin{equation}\label{balanceoftranslationalmomentum.local.withfirstpiolakirchhoffstress}
\boldnablacircled \dotp \hspace{.12ex} \firstpiolakirchhoffstress + \hspace{.1ex} \mathcircabove{\rho} \hspace{-0.1ex} \massloadvector
= \hspace{.1ex} \zerovector \hspace{.12ex}.
\end{equation}

\en{Advantages of this equation}\ru{Преимущества этого уравнения}
\en{in comparison with}\ru{в~сравнении с}~\eqref{balanceoftranslationalmomentum.local}\en{ are}\::
\en{here}\ru{здесь}
\en{figures}\ru{фигурирует}
\en{the~known}\ru{известная}
\en{mass density}\ru{плотность}~${\hspace{-0.1ex}\mathcircabove{\rho}}$\ru{ массы}
\en{of an~undeformed volume}\ru{недеформированного объёма}~${\hspace{-0.1ex}\mathcircabove{\mathcal{V}}\hspace{-0.25ex}}$,
\en{and }\ru{и~}\en{the~operator}\ru{оператор}~${\hspace{-0.16ex}\boldnablacircled \equiv \initiallocationvector^i \partial_i}$
\en{is defined}\ru{определяется}
\en{through}\ru{через}
\en{the~known vectors}\ru{известные векторы}~${\initiallocationvector^i\hspace{-0.25ex}}$.
\en{The~appearance of}\ru{Появление}~${\hspace{.16ex}\firstpiolakirchhoffstress}$
\en{presents}\ru{являет}
\en{the~specific property}\ru{специфическое свойство}
\en{of an~elastic solid body}\ru{упругого твёрдого тела}\:---
\inquotes{\en{to~retain}\ru{помнить}}
\en{its}\ru{свою}
\en{initial configuration}\ru{начальную конфигурацию}.
\en{Tensor}\ru{Тензор}~${\hspace{.16ex}\firstpiolakirchhoffstress}$
\en{is unlikely useful}\ru{едва~ли полезен}
\en{in~fluid mechanics}\ru{в~механике текучих сред}.

\en{The~principle of~virtual work}\ru{Принцип виртуальной работы}
\en{for an~arbitrary volume}\ru{для произвольного объёма}~$\mathcircabove{\mathcal{V}}$
\en{of elastic}\ru{упругой} (${\variation{\internalwork} = - \hspace{.2ex} \variation{\potentialenergydensity}}$)
\en{continuum}\ru{среды}\::

\nopagebreak\vspace{-0.16em}\ru{\vspace{-0.2em}}\begin{equation*}
\begin{array}{c}
\scalebox{0.96}[0.94]{$ \displaystyle \integral\displaylimits_{\mathcircabove{\mathcal{V}}} \hspace{-0.5ex}
\left(^{\mathstrut} \hspace{-0.1ex} \mathcircabove{\rho} \hspace{-0.1ex} \massloadvector \dotp \variation{\currentlocationvector} - \variation{\potentialenergydensity} \right) \hspace{-0.4ex} d \mathcircabove{\mathcal{V}} $}
\hspace{-0.1ex} + \hspace{-0.25ex}
\scalebox{0.96}[0.94]{$ \displaystyle \integral\displaylimits_{\mathclap{o \hspace{.1ex} (\boundary \smash{\mathcircabove{\mathcal{V}}})}} \hspace{-0.4ex} \initialunitnormal \dotp \firstpiolakirchhoffstress \hspace{-0.16ex} \dotp \variation{\currentlocationvector} \hspace{.4ex} do $}
= 0 \hspace{.1ex} ,
\\[.1em]
%
\boldnablacircled \hspace{-0.1ex} \dotp \hspace{-0.1ex} \left( \hspace{.12ex} \firstpiolakirchhoffstress \hspace{-0.12ex} \dotp \variation{\currentlocationvector} \hspace{.12ex} \right)
= \boldnablacircled \hspace{-0.1ex} \dotp \firstpiolakirchhoffstress \hspace{-0.16ex} \dotp \variation{\currentlocationvector} \hspace{.1ex}
+ \hspace{.1ex} \firstpiolakirchhoffstress^{\hspace{.1ex}\T} \hspace{-0.5ex} \dotdotp \hspace{-0.2ex} \boldnablacircled \hspace{.1ex} \variation{\currentlocationvector} \hspace{.1ex},
\:\,
\firstpiolakirchhoffstress^{\hspace{.1ex}\T} \hspace{-0.5ex} \dotdotp \hspace{-0.2ex} \boldnablacircled \hspace{.1ex} \variation{\currentlocationvector} \hspace{.1ex}
= \firstpiolakirchhoffstress \hspace{-0.1ex} \dotdotp \hspace{-0.2ex} \boldnablacircled \hspace{.1ex} \variation{\currentlocationvector}^{\hspace{.1ex}\T}
\\[.25em]
%
\variation{\potentialenergydensity}
= \hspace{-0.15ex} \left(^{\mathstrut} \hspace{-0.1ex} \mathcircabove{\rho} \hspace{-0.1ex} \massloadvector + \hspace{-0.25ex} \boldnablacircled \hspace{-0.1ex} \dotp \firstpiolakirchhoffstress \right) \hspace{-0.4ex} \dotp \variation{\currentlocationvector}
\hspace{.1ex}
+ \firstpiolakirchhoffstress \hspace{-0.1ex} \dotdotp \hspace{-0.2ex} \boldnablacircled \hspace{.1ex} \variation{\currentlocationvector}^{\hspace{.1ex}\T}
\end{array}
\end{equation*}

....

The~first one is non-symmetric, it links forces in the~deformed stressed configuration to the underfomed geometry and mass (volumes, areas, densities as they were initially), and it is energetically conjugate \en{to the }\en{motion gradient}\ru{градиент движения} (\en{often mistakenly called}\ru{часто ошибочно называемый} \en{the }\inquotes{\en{deformation gradient}\ru{градиент деформации}}, \en{forgetting}\ru{забывая} \en{about rigid rotations}\ru{о~жёстких вращениях}).
The~first (or sometimes its transpose) is also known as \inquotes{nominal stress} and \inquotes{engineering stress}.

The~second one is symmetric, it links loads in the~initial undeformed configuration to the initial mass and geometry, and it is conjugate to the right Cauchy\hbox{--}Green deformation tensor (and thus to the Cauchy\hbox{--}Green\hbox{--}Venant measure of deformation).

The~first is simplier when you use just the motion gradient and is more universal, but the~second is simplier when you prefer right Cauchy\hbox{--}Green deformation and its offsprings.

There’s also popular Cauchy stress, which relates forces in the deformed configuration to the deformed geometry and mass.

\inquotes{energetically conjugate} means that their product is kind of energy, here: elastic potential energy per unit of volume

......

\en{In the case of finite deformations}\ru{В~случае конечных деформаций}, \ru{тензоры }\en{the }Piola\hbox{--}Kirchhoff\ru{’а}\en{ tensors}~${\firstpiolakirchhoffstress}$ \en{and}\ru{и}~${\secondpiolakirchhoffstress}$ \en{describe}\ru{описывают} \en{the stress}\ru{напряжение} \en{relative to the initial configuration}\ru{относительно начальной конфигурации}.
\en{In contrast with them}\ru{В~отличие от них}, \ru{тензор напряжения }\en{the }Cauchy\en{ stress tensor}~${\cauchystress}$ \en{describes}\ru{описывает} \en{the stress}\ru{напряжение} \en{relative to the current configuration}\ru{относительно текущей конфигурации}.
\en{For infinitesimal deformations}\ru{Для бесконечно-малых деформаций}\en{,} \ru{тензоры напряжения }\en{the }Cauchy \en{and }\ru{и~}Piola\hbox{--}Kirchhoff\ru{’а}\en{ stress tensors} \en{are identical}\ru{идентичны}.

\subsection*{1st Piola\hbox{--}Kirchhoff stress tensor}

The 1st Piola\hbox{--}Kirchhoff stress tensor~$\firstpiolakirchhoffstress$ relates forces in the current~(present, \inquotes{spatial}) configuration with areas in the initial~(\inquotes{material}) configuration

\noindent\vspace{-0.2em}\begin{equation*}
\firstpiolakirchhoffstress = J \, \cauchystress \dotp \bm{F}^{\expminusT}
\end{equation*}

\vspace{-0.2em}\noindent
\en{where}\ru{где} $\bm{F}$ \en{is}\ru{есть} \en{the motion gradient}\ru{градиент движения} \en{and}\ru{и}~${J \equiv \determinant \bm{F}}$ \en{is}\ru{есть} \ru{определитель }\en{the }Jacobi\en{ determinant}, Jacobian.

Because it relates different coordinate systems, the 1st~Piola\hbox{--}Kirchhoff stress is a~two\hbox{-}point tensor.
Commonly, it’s not symmetric.

The 1st~Piola\hbox{--}Kirchhoff stress is the 3D generalization of the 1D concept of engineering stress.

If the material rotates without a~change in stress (rigid rotation), the components of the 1st Piola\hbox{--}Kirchhoff stress tensor will vary with material orientation.

The 1st~Piola\hbox{--}Kirchhoff stress is energy conjugate to the motion gradient.

\subsection*{2nd Piola\hbox{--}Kirchhoff stress tensor}

The 2nd~Piola\hbox{--}Kirchhoff stress tensor~$\secondpiolakirchhoffstress$ relates forces in the initial configuration to areas in the initial configuration.
The force in the initial configuration is obtained via mapping that preserves the relative relationship between the force direction and the area normal in the initial configuration.

\noindent\vspace{-0.2em}\begin{equation*}
\secondpiolakirchhoffstress = J \, \bm{F}^{\expminusone} \dotp \cauchystress \dotp \bm{F}^{\expminusT}
\end{equation*}

This tensor is a~one\hbox{-}point tensor and it is symmetric.

If the material rotates without a change in stress (rigid rotation), the 2nd~Piola\hbox{--}Kirchhoff stress tensor remain constant, irrespective of material orientation.

The 2nd~Piola\hbox{--}Kirchhoff stress tensor is energy conjugate to the Green\hbox{--}Lagrange finite strain tensor.

....

%%-------
\newpage
%%-------

\en{\section{Variation of the present configuration}}

\ru{\section{Варьирование текущей конфигурации}}

\label{section:variationofconfiguration}

\en{Usually}\ru{Обыкновенно}
\ru{рассматриваются }\en{the two configurations}\ru{две конфигурации} \en{of a~nonlinear elastic medium}\ru{нелинейной упругой среды}\en{ are considered}\::
\en{the initial one}\ru{начальная}
\en{with location vectors}\ru{с~векторами положения}~$\initiallocationvector$
\en{and}\ru{и}
\en{the present (current) one}\ru{текущая (актуальная)}
\en{with}\ru{с}~$\currentlocationvector$.

\en{The following}\ru{Следующие}
\en{equations}\ru{уравнения}
\en{describe}\ru{описывают}
\en{a~small change}\ru{малое изменение}
\en{of the current configuration}\ru{текущей конфигурации}
\en{with infinitesimal changes}\ru{с~бесконечно-малыми изменениями}
\en{to the location vector}\ru{вектора положения}~$\variation{\currentlocationvector}$,
\en{to the vector of mass forces}\ru{вектора массовых сил}~${\variation{\hspace{-0.2ex}\massloadvector}\hspace{-0.2ex}}$,
\en{to the first}\ru{первого} \ru{тензора напряжения }Piola\hbox{--}Kirchhoff\ru{’а}\en{ stress tensor}~${\variation{\hspace{.1ex}\firstpiolakirchhoffstress}}$
\en{and}\ru{и}
\en{to the deformation tensor}\ru{тензора деформации}~${\variation{\hspace{.1ex}\bm{C}}}$.

\en{By varying}\ru{Варьируя}
\eqref{balanceoftranslationalmomentum.local.withfirstpiolakirchhoffstress}, (......)\footnote{%
${\boldnabla \hspace{-0.08ex}
= \hspace{-0.2ex} \boldnabla \dotp \hspace{-0.15ex} \boldnablacircled \initiallocationvector \hspace{-0.08ex}
= \currentlocationvector^{i} \hspace{-0.1ex} \partial_{i} \hspace{-0.15ex} \dotp \initiallocationvector^j \hspace{-0.1ex} \partial_{\hspace{-0.1ex}j} \initiallocationvector \hspace{-0.1ex}
\stackrel{?}{=} \currentlocationvector^{i} \hspace{-0.1ex} \partial_{i} \initiallocationvector \hspace{-0.1ex} \dotp \initiallocationvector^j \hspace{-0.1ex} \partial_{\hspace{-0.1ex}j} \hspace{-0.3ex}
= \hspace{-0.2ex} \boldnabla \initiallocationvector \dotp \hspace{-0.15ex} \boldnablacircled \hspace{-0.1ex}
= \hspace{-0.1ex} \bm{F}^{\hspace{.1ex}\expminusT} \hspace{-0.3ex} \dotp \hspace{-0.1ex} \boldnablacircled}$ \\
%
${\boldnablacircled \hspace{-0.1ex}
= \hspace{-0.2ex} \boldnablacircled \dotp \hspace{-0.2ex} \boldnabla \currentlocationvector
= \initiallocationvector^{i} \partial_{i} \hspace{-0.15ex} \dotp \hspace{-0.1ex} \currentlocationvector^{j} \hspace{-0.1ex} \partial_{\hspace{-0.1ex}j} \hspace{-0.1ex} \currentlocationvector
\stackrel{?}{=} \initiallocationvector^{i} \partial_{i} \currentlocationvector \dotp \hspace{-0.25ex} \currentlocationvector^{j} \hspace{-0.1ex} \partial_{\hspace{-0.1ex}j} \hspace{-0.3ex}
= \hspace{-0.3ex} \boldnablacircled \currentlocationvector \hspace{.1ex} \dotp \hspace{-0.15ex} \boldnabla \hspace{-0.25ex}
= \hspace{-0.1ex} \bm{F}^{\hspace{.1ex}\T} \hspace{-0.3ex} \dotp \hspace{-0.1ex} \boldnabla}$}
\en{and}\ru{и}~(......),
\en{we get}\ru{мы получаем}

\nopagebreak\vspace{-0.4em}
\begin{equation}\label{variationsforthecurrentconfiguration}
\begin{array}{c}
\boldnablacircled \hspace{-0.1ex} \dotp \variation{\hspace{.1ex}\firstpiolakirchhoffstress} \hspace{-0.1ex}
+ \mathcircabove{\rho} \hspace{.25ex} \variation{\hspace{-0.2ex}\massloadvector} \hspace{-0.1ex}
= \zerovector
\hspace{.1ex} , \:\,
%
\variation{\hspace{.1ex}\firstpiolakirchhoffstress} \hspace{-0.1ex}
= \hspace{-0.2ex} \left( \hspace{.1ex} \scalebox{.93}{$ \displaystyle\frac{\partial^2 \hspace{.1ex} \potentialenergydensity}{\raisemath{-0.1em}{\partial \hspace{.1ex} \bm{C} \hspace{.1ex} \partial \hspace{.1ex} \bm{C}}} $} \dotdotp \variation{\hspace{.1ex}\bm{C}} \hspace{-0.15ex} \right) \hspace{-0.3ex} \dotp \bm{F}^{\hspace{.1ex}\T} \hspace{-0.3ex}
+ \hspace{.1ex}
\displaystyle\frac{\partial \hspace{.1ex} \potentialenergydensity}{\raisemath{-0.1em}{\partial \hspace{.1ex} \bm{C}}} \dotp \variation{\bm{F}}^{\hspace{.1ex}\T}
\hspace{-0.4ex} ,
\\[1em]
%
\variation{\bm{F}}^{\hspace{.1ex}\T} \hspace{-0.5ex}
= \variation{\hspace{.1ex} \boldnablacircled \currentlocationvector}
= \hspace{-0.2ex} \boldnablacircled \hspace{.1ex} \variation{\currentlocationvector}
= \hspace{-0.1ex} \bm{F}^{\hspace{.1ex}\T} \hspace{-0.25ex} \dotp \boldnabla \hspace{.1ex} \variation{\currentlocationvector} \hspace{.15ex} ,
\:\,
\variation{\bm{F}} \hspace{-0.2ex} = \variation{\hspace{.1ex} \boldnablacircled \currentlocationvector^{\T}} \hspace{-0.4ex}
= \hspace{-0.2ex} \boldnabla \hspace{.1ex} \variation{\currentlocationvector}^{\hspace{.1ex}\T} \hspace{-0.25ex} \dotp \bm{F}
\hspace{-0.1ex} ,
\\[.5em]
%
\variation{\hspace{.1ex}\bm{C}}
= \smalldisplaystyleonehalf \hspace{.2ex} \variation{ \bigl( \bm{F}^{\hspace{.1ex}\T} \hspace{-0.4ex} \dotp \bm{F} \bigr) } \hspace{-0.25ex}
= \bm{F}^{\hspace{.1ex}\T} \hspace{-0.4ex} \dotp \infinimentpetitedeformationvariation \dotp \bm{F} ,
\:\,
\infinimentpetitedeformationvariation \equiv \insideinfinitesimalstrainvariation
\hspace{.1ex} .
\end{array}
\end{equation}

.....

\begin{equation*}\begin{array}{c}
\eqref{areachange:nansonformula}
\hspace{.4em} \Rightarrow \hspace{.4em}
%
\initialunitnormal \hspace{.1ex} do = \inverseJacobian \currentunitnormal \hspace{.1ex} d\mathcal{O} \hspace{-0.1ex} \dotp \bm{F}
\hspace{.4em} \Rightarrow \hspace{.4em}
%
\initialunitnormal \dotp \variation{\firstpiolakirchhoffstress} \hspace{.1ex} do
= \inverseJacobian \currentunitnormal \hspace{-0.1ex} \dotp \bm{F} \hspace{-0.1ex} \dotp \variation{\firstpiolakirchhoffstress} \hspace{.1ex} d\mathcal{O}
\\[.4em]
%
\text{\en{or}\ru{или}} \hspace{1.5ex}
\initialunitnormal \dotp \variation{\firstpiolakirchhoffstress} \hspace{.1ex} do
= \currentunitnormal \dotp \variedcauchystress \hspace{.25ex} d\mathcal{O} ,
\:\:
\variedcauchystress \equiv \inverseJacobian \bm{F} \hspace{-0.1ex} \dotp \variation{\firstpiolakirchhoffstress}
\end{array}\end{equation*}

\vspace{-0.2em}\noindent
---
\ru{введённый здесь }\en{tensor}\ru{тензор}~${\variedcauchystress}$\en{ introduced here}
\en{is related to}\ru{связан с}~\en{variation}\ru{вариацией}~$\variation{\hspace{.1ex}\firstpiolakirchhoffstress}$
\en{just alike}\ru{так же, как}
${\hspace{-0.1ex} \cauchystress \hspace{.1ex}}$
\en{is related to}\ru{связан с}~$\firstpiolakirchhoffstress$
(${\cauchystress \hspace{.1ex} = \hspace{-0.1ex} \inverseJacobian \bm{F} \dotp \hspace{.15ex} \firstpiolakirchhoffstress\hspace{.2ex}}$).
%
\en{From}\ru{Из}~\eqref{variationsforthecurrentconfiguration}
\en{and}\ru{и} ...

.....

...
\en{and}\ru{и}~%
\en{adjusting}\ru{адаптируя}
\en{the~coefficients}\ru{коэффициенты}
\en{of the linear function}\ru{линейной функции}
${\variedcauchystress\hspace{.15ex}( \infinimentpetitedeformationvariation )}$ (...)



\en{\section{Internal constraints}}

\ru{\section{Внутренние связи}}

\label{section:internalconstraints}

\begin{otherlanguage}{russian}

До~сих~пор деформация считалась свободной, мера деформации~$\bm{C}$ могла быть любой.
Однако, существуют материалы со~значительным сопротивлением некоторым видам деформации.
Резина, например, изменению формы сопротивляется намного меньше, чем изменению объёма\:--- некоторые виды резины можно считать несжимаемым материалом.

Понятие геометрической связи, развитое в~общей механике ...

\end{otherlanguage}


...



for incompressible materials
${\potentialenergydensity \hspace{-0.4ex} = \hspace{-0.33ex} \potentialenergydensity (\anyfirstinvariant, \anysecondinvariant)}$

Mooney\hbox{--}Rivlin model of incompressible material
\[
\potentialenergydensity = c_1 \bigl( \anyfirstinvariant - 3 \bigr) + c_2 \bigl( \anysecondinvariant - 3 \bigr)
\]

incompressible Treloar (neo-Hookean) material
\[
c_2 \hspace{-0.16ex} = 0
\;\;\Rightarrow\;\;
\potentialenergydensity = c_1 \bigl( \anyfirstinvariant - 3 \bigr)
\]


...



\en{\section{Hollow sphere under pressure}}

\ru{\section{Полая сфера под давлением}}

\label{section:hollowsphereunderpressure}

\begin{otherlanguage}{russian}

Решение этой относительно простой задачи описано во~многих книгах.
В~начальной~(ненапряжённой) конфигурации имеем сферу с~внутренним радиусом~${r_0}$ и~наружным~${r_1}$.
Давление равно $p_0$~внутри и~$p_1$~снаружи.

Введём удобную для этой задачи сферическую систему координат в~отсчётной конфигурации ${q^1 = \theta}$, ${q^2 = \phi}$, ${q^3 = r}$~(\figureref{sphericalcoordinates}).
Эти~же координаты будут и~материальными.
Имеем

...

\end{otherlanguage}

%%-------
\newpage
%%-------

\en{\section{Stresses as Lagrange multipliers}}

\ru{\section{Напряжения как множители Lagrange’а}}

\label{section:stressesAsLagrangeMultipliers}

\newcommand\bivalentlagrangemultiplier{{^2\hspace{-0.2ex}\bm{\lambda}}}

\en{The~application}\ru{Применению}
\en{of the~principle of~virtual work}\ru{принципа виртуальной работы},
\en{described}\ru{описанному}
\en{in}\ru{в}~\sectionref{section:virtualworkprinciple.elastic},
\en{was preceded by}\ru{предшествовало}
\en{the~introduction}\ru{введение}
\ru{тензора напряжения}\en{of the} Cauchy\en{ stress tensor}
\en{through}\ru{через}
\en{the~balance of~forces}\ru{баланс сил}
\en{for}\ru{для}
\en{an~infinitesimal}\ru{бесконечномалого}
\en{tetrahedron}\ru{тетраэдра}~(\sectionref{section:stressviatetrahedron}).
%
\en{But now}\ru{Но теперь}
\en{the~reader}\ru{читатель}
\en{will see}\ru{увидит}\ru{,}
\en{that}\ru{что}
\en{this principle}\ru{этот принцип}
\en{may be}\ru{может быть}
\en{as well applied without any}\ru{применён и~в\'{о}все без}
\en{tetrahedrons}\ru{тетраэдров}.

\en{Considering}\ru{Рассматривая}
\en{a~continuum}\ru{\rucontinuum}/\en{body}\ru{тело}\:---
\en{not~only}\ru{не~только}
\en{elastic}\ru{упругое},
\en{with any}\ru{с~любой}
\en{virtual work}\ru{виртуальной работой}
\en{of internal forces}\ru{внутренних сил}~${\variation{\internalwork} \hspace{-0.33ex}}$
(\en{per unit mass}\ru{на~единицу массы})\:---
\en{loaded with}\ru{нагруженное}
\en{external forces}\ru{внешними силами},
\en{mass ones}\ru{массовыми}~${\massloadvector dm \hspace{-0.1ex} = \hspace{-0.3ex} \massloadvector \hspace{-0.2ex} \massdensity \hspace{.2ex} d\mathcal{V} \hspace{-0.2ex}}$
(\en{for}\ru{для}
\en{brevity}\ru{краткости}
\en{just}\ru{просто}~${\hspace{-0.2ex} \massloadvector \hspace{-0.33ex}}$,
\en{meaning}\ru{имея в виду}
${\massloadvector \hspace{-0.2ex} \equiv \hspace{-0.2ex} \massloadvector_{\hspace{-0.25ex}*} \hspace{-0.3ex} - \mathdotdotabove{\currentlocationvector}\hspace{.25ex}}$
\en{in}\ru{в}~\en{dynamics}\ru{динамике})
\en{and }\ru{и~}\en{surface ones}\ru{поверхностными}~${\bm{p} \hspace{.25ex} d\mathcal{O}}$.
%
\en{Then}\ru{Тогда}
\en{the variational equa\-tion}\ru{вариационное уравнение}
\en{of the~principle of~virtual work}\ru{принципа виртуальной работы}
\en{is}\ru{есть}

\nopagebreak\vspace{-0.4em}
\begin{equation}\label{stressesAsLagrangeMultipliers:variations}
\integral\displaylimits_{\mathcal{V}} \hspace{-0.4ex}
\massdensity
\Bigl(
   \massloadvector \dotp \variation{\currentlocationvector}
   \hspace{.1ex} +
   \variation{\internalwork}
\hspace{.15ex} \Bigr)
d\mathcal{V}
\hspace{.1ex} + \hspace{-0.4ex}
\integral\displaylimits_{\mathclap{\mathcal{O}(\boundary \mathcal{V})}} \hspace{-0.4ex}
\bm{p} \dotp \variation{\currentlocationvector}
\hspace{.2ex} d\mathcal{O}
= \hspace{.1ex} 0
\hspace{.1ex} .
\end{equation}

\vspace{-0.7em}
\en{Further}\ru{Далее}\en{,}
\en{it’s assumed that}\ru{предполагается, что}
\en{internal forces}\ru{внутренние силы}
(\inquotes{\en{stresses}\ru{напряжения}})
\en{do~not produce work}\ru{не~производят работу}\ru{,}
\en{when}\ru{когда}
\en{a~continuum}\ru{\rucontinuum}/\en{body}\ru{тело}
\en{virtually}\ru{виртуально}
\en{moves}\ru{движется}
(\en{with}\ru{с}~${\variation{\currentlocationvector}}$)
\en{as a~whole}\ru{как целое}
\en{without}\ru{без}
\en{deformations}\ru{деформаций}
(\en{when}\ru{когда}
${\infinimentpetitedeformationvariation \hspace{-0.1ex} \equiv \hspace{-0.1ex} \insideinfinitesimalstrainvariation = \hspace{-0.2ex} \zerobivalent}$),
\en{that is}\ru{то есть}

\nopagebreak\vspace{-0.3em}
\begin{equation}\label{stressesAsLagrangeMultipliers:zerovirtualmovements}
\insideinfinitesimalstrainvariation
= \hspace{-0.2ex}
\zerobivalent
\hspace{1ex} \Rightarrow \hspace{.8ex}
\variation{\internalwork}
\hspace{-0.2ex} =
0
\hspace{.1ex} .
\end{equation}

\vspace{-0.4em}\noindent
\eqref{stressesAsLagrangeMultipliers:variations}
\en{with condition}\ru{с~условием}~\eqref{stressesAsLagrangeMultipliers:zerovirtualmovements}
\en{and }\ru{и~}\en{without}\ru{без}~${\variation{\internalwork}}$
\en{becomes}\ru{становится}
\en{a~variational equation}\ru{вариационным уравнением}
\en{with constraint}\ru{со~связью}.

\en{The~method of Lagrange multipliers}\ru{Метод множителей Lagrange’а}
\en{makes}\ru{делает}~${\variation{\currentlocationvector}}$
\en{random}\ru{случайными}
(\en{independent}\ru{независимыми})
\en{variations}\ru{вариациями}.
%
\en{Since}\ru{Поскольку}
\en{at~each point}\ru{в~каждой точке}
\en{the~constraint}\ru{связь}
\en{appears as}\ru{представляет собой}
\en{a~symmetric}\ru{симметричный}
\en{bivalent}\ru{бивалентный} %\en{of second complexity}\ru{второй сложности}
\en{tensor}\ru{тензор},
\en{the}\ru{множитель} Lagrange\ru{’а}\en{ multiplier}~${\hspace{-0.16ex} \bivalentlagrangemultiplier}$
\en{will likewise be}\ru{тоже будет}
\en{such}\ru{таким~же}
\en{a~tensor}\ru{тензором},
\en{bivalent}\ru{двухвалентным}
\en{and}\ru{и}~\en{symmetric}\ru{симметричным}.
%
\en{The~equation}\ru{Уравнение}
\en{with this multiplier}\ru{с~этим множителем}
\en{looks like}\ru{выглядит как}

\nopagebreak\vspace{-0.4em}
\begin{equation}\label{stressesAsLagrangeMultipliers:equationwiththemultiplier}
\integral\displaylimits_{\mathcal{V}} \hspace{-0.5ex} \Bigl( \massdensity \hspace{-0.1ex} \massloadvector \hspace{-0.15ex} \dotp \variation{\currentlocationvector} \hspace{.1ex} - \hspace{-0.1ex} \bivalentlagrangemultiplier \dotdotp \hspace{-0.15ex} \insideinfinitesimalstrainvariation \hspace{.2ex} \Bigr) \hspace{-0.1ex} d\mathcal{V}
\hspace{.1ex} + \hspace{-0.4ex}
\integral\displaylimits_{\mathclap{\mathcal{O}(\boundary \mathcal{V})}} \hspace{-0.4ex} \bm{p} \dotp \variation{\currentlocationvector} \hspace{.2ex} d\mathcal{O}
= \hspace{.1ex} 0
\hspace{.1ex} .
\vspace{-0.25em}\end{equation}

\vspace{-0.5em}
\en{The~symmetry}\ru{Симметрия}\en{ of}~$\bivalentlagrangemultiplier$
\en{gives}\ru{даёт}\footnote{${%
\bm{\Lambda}^{\hspace{-0.16ex}\mathsf{S}} \hspace{-0.1ex} \dotdotp \hspace{-0.1ex} \bm{X} \hspace{-0.2ex} =
\hspace{.1ex} \bm{\Lambda}^{\hspace{-0.16ex}\mathsf{S}} \hspace{-0.1ex} \dotdotp \hspace{-0.1ex} \bm{X}^{\T} \hspace{-0.4ex} =
\hspace{.1ex} \bm{\Lambda}^{\hspace{-0.16ex}\mathsf{S}} \hspace{-0.1ex} \dotdotp \hspace{-0.1ex} \bm{X}^{\hspace{.1ex}\mathsf{S}}
\hspace{-0.25ex}}$,
\hspace{.66em}
${%
\boldnabla \hspace{-0.1ex} \dotp \hspace{-0.2ex} \bigl( \hspace{.1ex} {\bm{B}} \hspace{-0.1ex} \dotp \bm{a} \hspace{.16ex} \bigr)
\hspace{-0.2ex} = \hspace{-0.2ex}
\bigl( \hspace{.1ex} \boldnabla \dotp \hspace{-0.15ex} \bm{B} \hspace{.1ex} \bigr) \hspace{-0.2ex} \dotp \bm{a} \hspace{.1ex}
+ \bm{B}^{\T} \hspace{-0.3ex} \dotdotp \boldnabla \hspace{-0.15ex} \bm{a}
}$}

\nopagebreak\vspace{-0.2em}\begin{equation*}
\bivalentlagrangemultiplier = \hspace{-0.2ex} \bivalentlagrangemultiplier^{\hspace{-0.33ex}\T}
\hspace{.466em} \Rightarrow \hspace{.4em}
\hspace{.12ex}\bivalentlagrangemultiplier \dotdotp \hspace{-0.15ex} \insideinfinitesimalstrainvariation
= \bivalentlagrangemultiplier \dotdotp \hspace{-0.15ex} \boldnabla \variation{\currentlocationvector}^{\T}
\hspace{-0.4ex} ,
\end{equation*}

\nopagebreak\vspace{-0.2em}\begin{equation*}
\bivalentlagrangemultiplier \dotdotp \hspace{-0.15ex} \insideinfinitesimalstrainvariation
=
\boldnabla \hspace{-0.1ex} \dotp \hspace{-0.1ex} \bigl( \hspace{.1ex} \bivalentlagrangemultiplier \dotp \variation{\currentlocationvector} \hspace{.1ex} \bigr) \hspace{-0.2ex}
- \hspace{-0.1ex} \boldnabla \hspace{.1ex} \dotp \hspace{-0.1ex} \bivalentlagrangemultiplier \dotp \variation{\currentlocationvector}
\hspace{.1ex} .
\end{equation*}

\vspace{-0.2em}\noindent
\en{Substituting}\ru{Подставляя}
\en{this}\ru{это}
\en{into}\ru{в}~\eqref{stressesAsLagrangeMultipliers:equationwiththemultiplier}
\en{and }\ru{и~}%
\en{applying}\ru{применяя}
\en{the~divergence theorem}\ru{теорему о~дивергенции}\footnote{${%
\scalebox{.93}{$ \displaystyle \integral\displaylimits_{\mathcal{V}} $} \hspace{-0.2ex}
\boldnabla \hspace{-0.1ex} \dotp \hspace{-0.2ex} \bigl( \hspace{.1ex} \bivalentlagrangemultiplier \dotp \variation{\currentlocationvector} \bigr)
d\mathcal{V}
\hspace{.1ex} = \hspace{-0.4ex}
\scalebox{.93}{$ \displaystyle \integral\displaylimits_{\mathclap{\mathcal{O}(\boundary \mathcal{V})}} $} \hspace{-0.1ex}
\currentunitnormal \dotp \hspace{-0.2ex} \bigl( \hspace{.1ex} \bivalentlagrangemultiplier \dotp \variation{\currentlocationvector} \bigr)
d\mathcal{O}
\hspace{.1ex} ,
\hspace{.66em}
%
\currentunitnormal \dotp \hspace{-0.2ex} \bigl( \hspace{.1ex} \bivalentlagrangemultiplier \dotp \variation{\currentlocationvector} \bigr)
\hspace{-0.2ex} = \hspace{-0.2ex} \bigl( \currentunitnormal \dotp \bivalentlagrangemultiplier \bigr) \hspace{-0.3ex} \dotp \variation{\currentlocationvector}
= \currentunitnormal \dotp \hspace{-0.1ex} \bivalentlagrangemultiplier \dotp \variation{\currentlocationvector}
}$}\hbox{\hspace{-0.5ex},}
\en{and}\ru{и}
\en{the variational equation with multiplier}\ru{вариационное уравнение с~множителем}~${\hspace{-0.2ex} \bivalentlagrangemultiplier}$
\en{becomes}\ru{становится}

\nopagebreak\vspace{-0.4em}
\begin{equation}\label{stressesAsLagrangeMultipliers:equationwiththemultipliertoo}
\integral\displaylimits_{\mathcal{V}} \hspace{-0.5ex} \Bigl( \massdensity \hspace{-0.1ex} \massloadvector \hspace{.12ex} + \boldnabla \dotp \hspace{-0.1ex} \bivalentlagrangemultiplier \Bigr) \hspace{-0.33ex} \dotp \variation{\currentlocationvector} \hspace{.2ex} d\mathcal{V}
\hspace{.1ex} + \hspace{-0.4ex}
\integral\displaylimits_{\mathclap{\mathcal{O}(\boundary \mathcal{V})}} \hspace{-0.5ex} \Bigl( \bm{p} \hspace{.2ex} - \currentunitnormal \hspace{.1ex} \dotp \hspace{-0.1ex} \bivalentlagrangemultiplier \Bigr) \hspace{-0.33ex} \dotp \variation{\currentlocationvector} \hspace{.2ex} d\mathcal{O}
= \hspace{.1ex} 0
\hspace{.1ex} .
\vspace{-0.25em}\end{equation}

\vspace{-0.4em}\noindent
\en{But}\ru{Но}~$\variation{\currentlocationvector}$
\en{is random}\ru{случайна}
\en{both on a~surface}\ru{и~на~поверхности}\ru{,}
\en{and}\ru{и}~\en{in a~volume}\ru{в~объёме},
\en{thus}\ru{поэтому}

\nopagebreak\vspace{-0.3em}
\begin{equation}\label{stressesAsLagrangeMultipliers:veryfamousequations}
\bm{p} \hspace{.1ex} = \currentunitnormal \hspace{.1ex} \dotp \hspace{-0.1ex} \bivalentlagrangemultiplier \hspace{.2ex},
\hspace{1em}
%
\boldnabla \dotp \hspace{-0.1ex} \bivalentlagrangemultiplier \hspace{.1ex} + \massdensity \hspace{-0.1ex} \massloadvector \hspace{-0.1ex} = \hspace{.1ex} \zerovector
\end{equation}

\vspace{-0.33em}\noindent
---
\en{the symmetric multiplier}\ru{симметричный множитель}~%
${\hspace{-0.2ex} \bivalentlagrangemultiplier}$,
\en{introduced}\ru{введённый}
\en{formally}\ru{формально},
\en{is}\ru{это}
\en{in fact}\ru{на с\'{а}мом деле}
\en{precisely}\ru{именно что}
\ru{тензор напряжения}\en{the} Cauchy\en{ stress tensor}\:!

\en{A~similar}\ru{Похожее}
\en{introduction of~stresses}\ru{введение напряжений}
\en{was presented}\ru{было представлено}
\en{in the~book}\ru{в~книге}~\cite{rabotnov-mechanicsofdeformable}.
\en{Here are no new results}\ru{Тут нет новых результатов},
\en{but}\ru{но}
\en{the~very possibility}\ru{сам\'{а} возможность}
\en{of simultaneously deriving}\ru{одно\-времен\-ного вывода}
\en{those equations}\ru{тех уравнений}
\en{of continuum mechanics}\ru{механики \rucontinuum{}а},
\en{that were previously considered}\ru{которые прежде считались}
\en{independent}\ru{независимыми},
\en{is quite interesting}\ru{весьма интересна}.
%
\en{In}\ru{В}~\en{subsequent chapters}\ru{последующих главах}
\en{this technique}\ru{эта техника}
\en{is used}\ru{используется}
\en{for building}\ru{для построения}
\en{new}\ru{новых}
\en{continuum models}\ru{континуальных моделей}.


%-----------------

\section*{\small \wordforbibliography}

\begin{changemargin}{\parindent}{0pt}
\fontsize{10}{12}\selectfont

\begin{otherlanguage}{russian}

Глубина изложения нелинейной безмоментной упругости характерна для книг А.\,И.\:Лурье~\cite{lurie-nonlinearelasticity, lurie-theoryofelasticity}.
%
Оригинальность как~основных идей, так~и~стиля присуща книге Clifford’а Truesdell’а~\cite{truesdell-firstcourse}.
%
Монография Юрия Работнова~\cite{rabotnov-mechanicsofdeformable}, где напряжения представлены как множители Лагранжа, очень интересна и~своеобразна.
%
Много ценной информации можно найти у~К.\,Ф.\:Черн\'{ы}х~\cite{chernyh-nonlinearelasticity}.
%
\en{The}\ru{Книгу}
\russianlanguage{Л.\,М.\:Зубов}\ru{а}\en{’s book}~\cite{zubov}
\en{is worthy of~mention too}\ru{тоже ст\'{о}ит упомянуть}.
%
О~применении нелинейной теории упругости в~смежных областях рассказано в~книге Cristian’а Teodosiu~\cite{teodosiu-crystaldefects}.
%
Повышенным математическим уровнем отличается монография Philippe’а Ciarlet~\cite{ciarlet-mathematicalelasticity}.
\par

\end{otherlanguage}

\end{changemargin}

