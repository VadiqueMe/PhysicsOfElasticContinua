\en{\chapter{Nonlinear elastic momentless continuum}}

\ru{\chapter{Нелинейно\hbox{-\hspace{-0.2ex}}упругая безмоментная среда}}

\thispagestyle{empty}

\label{chapter:nonlinearcontinuum}

\en{\section{Model of continuum. Descriptions of processes}}

\ru{\section{Модель сплошной среды. Описания процессов}}

\begin{otherlanguage}{russian}

\lettrine[lines=2, findent=2pt, nindent=0pt]{В}{ещества} имеют дискретное строение, и~модель системы частиц с~массами~${m_k}$ и~радиусами\hbox{-}векторами~${\bm{R}_k (t)}$ может показаться подходящей, несмотря на невообразимое число степеней свободы\:--- тем~более что объёмы памяти и~быстродействие компьютеров характеризуются тоже астрономическими числами.

И~всё~же предпочтение ст\'{о}ит отдать качественно иной модели\:--- модели сплошной среды~(материального контину\kern-0.11exума), в~которой масса распределена по~объёму непрерывно: в~объёме~$\mathcal{V}$ содержится масса

\nopagebreak\vspace{-0.5em}\begin{equation}
m = \hspace{-0.5ex} \integral\displaylimits_{\mathcal{V}} \hspace{-0.5ex} \rho \hspace{.2ex} d\mathcal{V} ,
\end{equation}

\vspace{-0.25em}\noindent \en{where}\ru{где} ${\rho\hspace{.1ex}}$\ru{\:---}\en{ is} \en{volume(tric) mass density}\ru{объёмная плотность массы}.

Real matter is modelled as a~continuum, \en{which can be thought of as}\ru{который может быть мыслим как} \en{an~infinite set}\ru{бесконечное множество} \en{of~vanishingly small}\ru{исчезающе м\'{а}лых} \en{particles}\ru{частиц}, joined together.

С~непрерывным распределением массы связано лишь первое и~простое представление о~сплошной среде как множестве~(пространстве) материальных точек. Возможны и~более сложные модели, в~которых частицы обладают степенями свободы не~только трансляции, но~и поворота, внутренней деформации и~другими. Отметив, что подобные модели притягивают всё б\'{о}льший интерес, в~этой главе ограничимся классическим представлением о~среде как состоящей из~\inquotesx{простых точек}[.]

В~каждый момент времени~$t$ среда~(деформируемый контину\kern-0.11exум) занимает некий объём~${\mathcal{V}\hspace{.1ex}}$ пространства. Этот объём движется и~деформируется, но набор частиц в~нём постоянен\:--- \en{the~balance of~mass}\ru{баланс массы} (\inquotes{\en{matter is neither created nor annihilated}\ru{материя не~создаётся и~не~аннигилируется}})

\nopagebreak\vspace{-0.2em}\begin{equation}\label{balanceofmass}
dm \hspace{-0.15ex} = \hspace{-0.2ex} \rho \hspace{.2ex} d\mathcal{V} \hspace{-0.1ex} = \hspace{-0.2ex} \mathcircabove{\rho} \hspace{.22ex} d\mathcircabove{\mathcal{V}} ,
\hspace{.8em}
%
m = \hspace{-0.4ex} \scalebox{0.9}{$\displaystyle\integral\displaylimits_{\mathcal{V}}$} \hspace{-0.2ex} \rho \hspace{.2ex} d\mathcal{V} \hspace{-0.1ex}
= \hspace{-0.4ex} \scalebox{0.9}{$\displaystyle\integral\displaylimits_{\mathcircabove{\mathcal{V}}}$} \hspace{-0.2ex} \mathcircabove{\rho} \hspace{.22ex} d\mathcircabove{\mathcal{V}} ,
\end{equation}

\vspace{-0.6em} \noindent \en{where}\ru{где} $\mathcircabove{\rho}$ \en{and}\ru{и}~${d\mathcircabove{\mathcal{V}}}$\ru{\:---}\en{ are} \en{mass density}\ru{плотность массы} \en{and}\ru{и}~\en{infinitesimal volume}\ru{бесконечно\-малый объём} \en{at the~initial moment}\ru{в~начальный момент}, \en{in~the~so\hbox{-}called initial~(original, reference, \inquotes{material}) configuration}\ru{в~так называемой начальной~(оригинальной, отсчётной, \inquotes{материальной}) конфигурации}.

% the Lagrangian description

Вводя какие\hbox{-}либо криволинейные координаты~${q^{\hspace{.1ex}i}\hspace{-0.4ex}}$, считаем

\nopagebreak\vspace{-0.2em}\begin{equation}\label{materialcoordinates}
\bm{R} \hspace{-0.25ex} = \hspace{-0.4ex} \bm{R}(q^{\hspace{.1ex}i} \hspace{-0.3ex}, t) .
\end{equation}

...

% the Eulerian description

Но~может~быть эффективно и~иное опис\'{а}ние\:--- пространственное~(или эйлерово), при~котором процессы рассматриваются не~в~движущихся частицах среды, а~в~неподвижных точках пространства. Полагая, например, ${\rho \!=\! \rho \hspace{.16ex} (\hspace{-0.1ex}\bm{R}, t)}$, мы следим за~происходящим в~этом месте и~не~смущаемся непрерывным уходом и~приходом частиц.

\end{otherlanguage}

\en{\section{Differentiation}}

\ru{\section{Дифференцирование}}

\label{para:differentiation}

\begin{otherlanguage}{russian}

Имея зависимости ${\varphi \!=\! \varphi(\bm{r})}$, ${\bm{r} \!=\! \bm{r}(q^{i})}$, вводится базис ${\bm{r}_i \equiv \partial_i \bm{r}}$~${(\partial_i \equiv \frac{\partial}{\partial q^i})}$, взаимный базис ${\bm{r}^i}$ и~оператор Hamilton’а

\nopagebreak\vspace{-0.4em}\begin{equation}
\bm{E} = \bm{r}^i \bm{r}_i \hspace{-0.1ex} = \bm{r}^i \partial_i \bm{r} = \hspace{-0.25ex} \boldnablacircled \bm{r} , \:\,
\boldnablacircled \equiv \hspace{.1ex} \bm{r}^i \partial_i \hspace{.1ex} ,
\end{equation}

\vspace{-0.2em}\noindent так что ${d\varphi = d\bm{r} \dotp \hspace{-0.1ex} \boldnablacircled \varphi}$.

Если

...

Jaumann derivative (\inquotes{corotational time derivative}). The corotational time derivative was first introduced by Jaumann\footnote{%
\bibauthor{Gustav Jaumann}. \href{http://www.physikdidaktik.uni-karlsruhe.de/download/jaumann_1911.pdf}{Geschlossenes System phy\-si\-ka\-li\-scher und che\-mi\-scher Differential\-gesetze (I.\;Mit\-teilung)~//~Sitzungs\-berichte der~Kaiser\-lichen Akademie der~Wissen\-schaften in~Wien, Mathematisch\hbox{-}naturwissenschaftliche Klasse, Abteilung~IIa, Band~CXX, 1911. Seiten 385\hbox{--}530.}}

\nopagebreak\begin{tcolorbox}[breakable, enhanced, colback = orange!8, before upper={\parindent3.2ex}, parbox = false]
\small%
\setlength{\abovedisplayskip}{2pt}\setlength{\belowdisplayskip}{2pt}%

Es sei ${\frac{\partial}{\partial t}}$ der~Operator der~\loosetexttr[80]{lokalen Fluxion}, d.\:i. der~partiellen Fluxion in einem gegen das~Koordinatensystem ruhenden Punkte des~Raumes. Ferner sei ${\frac{d}{dt}}$ der~Operator der~\loosetexttr[80]{totalen Fluxion}, welcher definiert wird durch

\nopagebreak\vspace{.1em}\begin{equation*}
\begin{array}{r@{\hspace{.66ex}}c@{\hspace{.75ex}}l}
\scalebox{0.92}[0.92]{$\displaystyle\frac{\raisemath{-0.125em}{da}}{dt}$} & = & \scalebox{0.92}[0.92]{$\displaystyle\frac{\raisemath{-0.125em}{\partial a}}{\partial t}$} + \bm{v} \dotp \hspace{-0.2ex} \boldnabla a
\hspace{.1ex} ,
\\[.8em]
\scalebox{0.92}[0.92]{$\displaystyle\frac{\raisemath{-0.125em}{d \bm{a}}}{dt}$} & \stackrel{3}{=} & \scalebox{0.92}[0.92]{$\displaystyle\frac{\raisemath{-0.125em}{\partial \bm{a}}}{\partial t}$} + \bm{v} \dotp \hspace{-0.2ex} \boldnabla\bm{;} \hspace{-0.25ex} \bm{a} - \hspace{.1ex} \smalldisplaystyleonehalf \bigl( \hspace{.1ex} \operatorname{rot}\hspace{.1ex} \bm{v} \hspace{.1ex} \bigr) \hspace{-0.4ex} \times \hspace{-0.2ex} \bm{a}
\hspace{.1ex} ,
\\[.8em]
\scalebox{0.92}[0.92]{$\displaystyle\frac{\raisemath{-0.125em}{d \mathboldalpha}}{dt}$} & \stackrel{9}{=} & \scalebox{0.92}[0.92]{$\displaystyle\frac{\raisemath{-0.125em}{\partial \mathboldalpha}}{\partial t}$} + \bm{v} \dotp \hspace{-0.2ex} \boldnabla\bm{;} \hspace{-0.25ex} \mathboldalpha - \hspace{.1ex} \smalldisplaystyleonehalf \bigl( \hspace{.1ex} \operatorname{rot}\hspace{.1ex} \bm{v} \hspace{-0.2ex} \times \hspace{-0.2ex} \mathboldalpha - \mathboldalpha \hspace{-0.2ex} \times \hspace{-0.2ex} \operatorname{rot}\hspace{.1ex} \bm{v} \hspace{.1ex} \bigr)
\hspace{-0.1ex} .
\end{array}
\end{equation*}

Endlich verwenden wir die~\loosetexttr[80]{körperliche Fluxion} eines Skalars:

\nopagebreak\vspace{-0.1em}\begin{equation*}
\scalebox{0.92}[0.92]{$\displaystyle\frac{\raisemath{-0.125em}{\delta}}{\delta t}$} \hspace{.15ex} a
= \scalebox{0.92}[0.92]{$\displaystyle\frac{\raisemath{-0.125em}{\partial}}{\partial t}$} \hspace{.15ex} a + \operatorname{div}\hspace{.1ex} a \bm{v}
= \scalebox{0.92}[0.92]{$\displaystyle\frac{\raisemath{-0.125em}{d}}{dt}$} \hspace{.15ex} a + a \operatorname{div}\hspace{.1ex} \bm{v}
\hspace{.1ex} .
\end{equation*}
\par
\end{tcolorbox}

körperliche\:--- bodily/телесная, material/вещественная(материальная), physical/физическая

${
\boldnabla \hspace{-0.1ex} \dotp \hspace{-0.15ex} \bigl( a \bm{v} \bigr) \hspace{-0.2ex} = \hspace{.1ex} a \hspace{.2ex} \boldnabla \hspace{-0.2ex} \dotp \hspace{-0.1ex} \bm{v} + \bm{v} \dotp \hspace{-0.2ex} \boldnabla a
}$

...

Пусть~${\varphi(\bm{r},t)}$\:--- какое\hbox{-}либо поле. Найдём скорость изменения интеграла по~объёму

\nopagebreak\vspace{-0.5em}\begin{equation*}
\Upsilon \equiv \hspace{-0.2ex} \scalebox{0.9}{$\displaystyle\integral\displaylimits_{\mathcal{V}}$} \hspace{-0.1ex} \rho \hspace{.1ex} \varphi \hspace{.2ex} d\mathcal{V}
\end{equation*}

\vspace{-0.25em}\noindent (\inquotes{$\varphi$ есть~$\Upsilon$ на~единицу массы}). Кажущееся сложным вычисление~${\mathdotabove{\Upsilon}}$ (ведь $\mathcal{V}$ деформируется) на~с\'{а}мом деле весьма простое благодаря~\eqref{balanceofmass}:
\begin{equation}\label{volumeintegralinbothconfigurations}
\Upsilon = \hspace{-0.2ex} \integral\displaylimits_{\mathcircabove{\mathcal{V}}} \! \mathcircabove{\rho} \hspace{0.1ex} \varphi \hspace{0.2ex} d\mathcircabove{\mathcal{V}}
\;\Rightarrow\;
\mathdotabove{\Upsilon} = \hspace{-0.2ex} \integral\displaylimits_{\mathcircabove{\mathcal{V}}} \! \mathcircabove{\rho} \hspace{0.1ex} \mathdotabove{\varphi} \hspace{0.2ex} d\mathcircabove{\mathcal{V}} =
\hspace{-0.2ex} \integral\displaylimits_{\mathcal{V}} \! \rho \hspace{0.1ex} \mathdotabove{\varphi} \hspace{0.2ex} d\mathcal{V} .
\end{equation}

\vspace{-0.1em} Не~ст\'{о}ит противопоставлять материальное и~пространственное описания. Далее будут использоваться оба в~зависимости от~ситуации.

\end{otherlanguage}

\newpage

\en{\section{Motion gradient}}

\ru{\section{Градиент движения}}

\label{para:motiongradient}

\en{Having}\ru{Имея} \en{motion function}\ru{функцию движения}~${\bm{R} \narroweq \hspace{-0.1ex} \bm{R}(q^{\hspace{.1ex}i} \hspace{-0.3ex}, t)}$, ${\bm{r}(q^{\hspace{.1ex}i}) \equiv \bm{R}(q^{\hspace{.1ex}i} \hspace{-0.3ex}, 0)}$,
\ru{операторы }\inquotes{\en{nabla}\ru{набла}}\en{ operators} ${\boldnabla \equiv \bm{R}^{\hspace{.1ex}i} \partial_i}$, ${\boldnablacircled \equiv \bm{r}^{\hspace{-0.05ex}i} \partial_i}$
\en{and}\ru{и} \en{looking}\ru{гл\'{я}дя} \en{at differential relations}\ru{на дифференциальные отношения} \en{for}\ru{для} \en{a~certain}\ru{какого\hbox{-}либо} \en{infinitesimal vector}\ru{бесконечномалого вектора} \en{in two configurations}\ru{в~двух конфигурациях}, \en{current}\ru{текущей} \en{with}\ru{с}~${d\bm{R}}$ \en{and}\ru{и}~\en{initial}\ru{начальной} \en{with}\ru{с}~${d\bm{r}}$

\nopagebreak\en{\vspace{1.3em}}\ru{\vspace{2.1em}}
\begin{equation}
\begin{array}{c}
d\bm{R} = d\bm{r} \dotp \hspace{-0.2ex} \tikzmark{beginFtransposed} \boldnablacircled \hspace{-0.16ex} \bm{R} \tikzmark{endFtransposed} = \hspace{-0.2ex} \tikzmark{beginMotionGradient} \boldnablacircled \hspace{-0.16ex} \bm{R}^{\hspace{.1ex}\T} \tikzmark{endMotionGradient} \hspace{-0.44ex} \dotp d\bm{r} \\[.2em]
%
d\bm{r} = d\bm{R} \dotp \hspace{-0.2ex} \tikzmark{beginFtransposedinverse} \boldnabla \bm{r} \tikzmark{endFtransposedinverse} = \hspace{-0.2ex} \tikzmark{beginFinverse} \boldnabla \bm{r}^{\T} \hspace{-0.4ex} \tikzmark{endFinverse} \dotp d\bm{R}
\end{array}
\end{equation}%
\AddOverBrace[line width=.75pt][0,0.6ex]%
{beginFtransposed}{endFtransposed}{${\begin{array}{c}
\hspace{.12em} \scalebox{0.85}{$\bm{F}^{\hspace{.1ex}\T}$} \\[-0.4em]
\scriptstyle \bm{r}^{\hspace{-0.05ex}i} \hspace{-0.2ex} \bm{R}_i \\[-0.36em]
\end{array}}$}
\AddOverBrace[line width=.75pt][0,0.6ex]%
{beginMotionGradient}{endMotionGradient}{${\begin{array}{c}
\scalebox{0.85}{$\bm{F}$} \\[-0.4em]
\scriptstyle \bm{R}_i \bm{r}^{\hspace{-0.05ex}i} \\[-0.36em]
\end{array}}$}
\AddUnderBrace[line width=.75pt][0,0.2ex][yshift=.32em]%
{beginFtransposedinverse}{endFtransposedinverse}{${\begin{array}{c}
\scriptstyle \bm{R}^{\hspace{.05ex}i} \bm{r}_{\hspace{-0.12ex}i} \\[-0.16em]
\scalebox{0.85}{$\bm{F}^{\hspace{.1ex}\expminusT}$}
\end{array}}$}
\AddUnderBrace[line width=.75pt][0,0.2ex][yshift=.32em]%
{beginFinverse}{endFinverse}{${\begin{array}{c}
\scriptstyle \bm{r}_{\hspace{-0.12ex}i} \bm{R}^{\hspace{.05ex}i} \\[-0.16em]
\scalebox{0.85}{$\bm{F}^{\hspace{.1ex}\expminusone}$}
\end{array}}$}

\vspace{1.5em}

\noindent \en{here comes to~mind}\ru{приходит на~ум} \en{to introduce}\ru{ввести} \en{the }\inquotes{\en{motion gradient}\ru{градиент движения}}\footnote{%
\en{Tensor}\ru{Тензору}~$\bm{F}$ \en{doesn’t well suit}\ru{не~вполне подходит} \en{its}\ru{его} \en{more popular}\ru{более популярное} \en{name}\ru{название} \inquotesx{\en{deformation gradient}\ru{градиент деформации}}[,] \en{because}\ru{поскольку} \en{this tensor}\ru{этот тензор} \en{describes}\ru{описывает} \en{not only the~deformation itself}\ru{не~только сам\'{у} деформацию}, \en{but also the rotation of a~body as a~whole without deformation}\ru{но~и поворот тела как~целого без деформации}.
}%
\hbox{\hspace{-0.6ex},} \en{picking}\ru{взяв} one of these tensor multipliers for it: ${\bm{F} \equiv \hspace{-0.2ex} \boldnablacircled \hspace{-0.16ex} \bm{R}^{\hspace{.1ex}\T} \hspace{-0.32ex} = \bm{R}_{\hspace{.1ex}i} \bm{r}^{\hspace{-0.05ex}i} \hspace{-0.4ex}}$.

\en{Why this one}\ru{Почему именно этот}?
\en{The~reason}\ru{Причина} \en{to choose}\ru{выбрать}~${\hspace{-0.25ex}\boldnablacircled \hspace{-0.16ex} \bm{R}^{\hspace{.1ex}\T}\hspace{-0.2ex}}$\ru{\:---}\en{ is} \en{another expression}\ru{другое выражение} \en{for the~differential}\ru{для дифференциала}

\[
d\bm{R} = \scalebox{0.9}{$ \displaystyle \frac{\raisemath{-0.2em}{\partial \hspace{-0.15ex} \bm{R}}}{\partial \bm{r}} $} \dotp d\bm{r}
\]

\[
\bm{F} \hspace{-0.1ex} = \scalebox{0.9}{$ \displaystyle \frac{\raisemath{-0.2em}{\partial \hspace{-0.15ex} \bm{R}}}{\partial \bm{r}} $}
\]

\[
d\bm{r} = \scalebox{0.9}{$ \displaystyle \frac{\raisemath{-0.2em}{\partial \bm{r}}}{\partial \hspace{-0.15ex} \bm{R}} $} \dotp d\bm{R}
\]

\[
\bm{F}^{\expminusone} \hspace{-0.3ex} = \scalebox{0.9}{$ \displaystyle \frac{\raisemath{-0.2em}{\partial \bm{r}}}{\partial \hspace{-0.15ex} \bm{R}} $}
\]

\[
\scalebox{0.9}{$ \displaystyle \frac{\raisemath{-0.2em}{\partial \bm{\zeta}}}{\partial \bm{r}} $} = \partial_i \bm{\zeta} \hspace{.1ex} \bm{r}^i
\hspace{2em}
\scalebox{0.9}{$ \displaystyle \frac{\raisemath{-0.2em}{\partial \bm{\zeta}}}{\partial \hspace{-0.15ex} \bm{R}} $} = \partial_i \bm{\zeta} \bm{R}^{\hspace{.1ex}i}
\]



...

\nopagebreak\vspace{-0.2em}\begin{equation*}
\bm{E}
= \tikzmark{unitTensorAsOriginalDerivativeBegin} \hspace{-0.25ex} \boldnablacircled \bm{r} \tikzmark{unitTensorAsOriginalDerivativeEnd}
= \tikzmark{unitTensorAsCurrentDerivativeBegin} \hspace{-0.25ex} \boldnabla \hspace{-0.16ex} \bm{R} \tikzmark{unitTensorAsCurrentDerivativeEnd}
\end{equation*}%
\AddUnderBrace[line width=.75pt][0.1ex,0.2ex]%
{unitTensorAsOriginalDerivativeBegin}{unitTensorAsOriginalDerivativeEnd}%
{${ \scalebox{0.8}{$ \displaystyle \frac{\raisemath{-0.2em}{\partial \bm{r}}}{\partial \bm{r}} $} }$}%
\AddUnderBrace[line width=.75pt][0.1ex,0.2ex]%
{unitTensorAsCurrentDerivativeBegin}{unitTensorAsCurrentDerivativeEnd}%
{${ \scalebox{0.8}{$ \displaystyle \frac{\raisemath{-0.2em}{\partial \hspace{-0.15ex} \bm{R}}}{\partial \hspace{-0.15ex} \bm{R}} $} }$}

...

\en{For cartesian coordinates}\ru{Для декартовых координат} \en{with orthonormal basis}\ru{с~ортонормальным базисом} ${\bm{e}_i \hspace{-0.16ex} = \boldconstant}$

\nopagebreak\vspace{-0.2em}\begin{equation*}
\bm{R} = \hspace{-0.15ex} X_{\hspace{-0.1ex}i}(t) \hspace{.2ex} \bm{e}_i
\hspace{.1ex} , \:\;
\bm{r} = \hspace{-0.15ex} X_{\hspace{-0.1ex}i}(0) \hspace{.2ex} \bm{e}_i \hspace{-0.16ex} = x_i \hspace{.1ex} \bm{e}_i
\hspace{.1ex} , \:\:
x_i \hspace{-0.1ex} \equiv \hspace{-0.1ex} X_{\hspace{-0.1ex}i}(0)
\hspace{.1ex} ,
\end{equation*}

\nopagebreak\vspace{-0.25em}\begin{equation*}
\boldnablacircled \hspace{-0.1ex}
= \bm{e}_i \hspace{.2ex} \scalebox{0.9}{$ \displaystyle \frac{\raisemath{-0.2em}{\partial}}{\partial x_i} $} \hspace{-0.1ex}
= \bm{e}_i \hspace{.15ex} \mathcircabove{\partial}_i
\hspace{.1ex} , \:\:
%
\boldnabla \hspace{-0.1ex}
= \bm{e}_i \hspace{.2ex} \scalebox{0.9}{$ \displaystyle \frac{\raisemath{-0.2em}{\partial}}{\partial \hspace{-0.2ex} X_{\hspace{-0.1ex}i}} $} \hspace{-0.1ex}
= \bm{e}_i \hspace{.15ex} \partial_i
\hspace{.1ex} ,
\end{equation*}

\nopagebreak\vspace{-0.4em}\begin{equation*}
\begin{array}{c}
\boldnablacircled \hspace{-0.16ex} \bm{R}
= \bm{e}_i \hspace{.2ex} \scalebox{.9}{$ \displaystyle \frac{\raisemath{-0.2em}{\partial \bm{R}}}{\partial x_i} $} \hspace{-0.1ex}
= \bm{e}_i \hspace{.2ex} \scalebox{.9}{$ \displaystyle \frac{\raisemath{-0.2em}{\partial \hspace{.1ex} ( \hspace{-0.1ex} X_{\hspace{-0.3ex}j} \hspace{.2ex} \bm{e}_j )}}{\partial x_i} $}
=  \hspace{.2ex} \scalebox{.9}{$ \displaystyle \frac{\raisemath{-0.2em}{\partial \hspace{-0.1ex} X_{\hspace{-0.3ex}j}}}{\partial x_i} $} \hspace{.25ex} \bm{e}_i \bm{e}_{\hspace{-0.1ex}j} \hspace{-0.2ex}
= \mathcircabove{\partial}_i \hspace{.1ex} X_{\hspace{-0.3ex}j} \hspace{.1ex} \bm{e}_i \bm{e}_{\hspace{-0.1ex}j}
\hspace{.1ex} ,
\\[.66em]
%
\boldnabla \bm{r}
= \bm{e}_i \hspace{.2ex} \scalebox{.9}{$ \displaystyle \frac{\raisemath{-0.2em}{\partial \bm{r}}}{\partial \hspace{-0.1ex} X_{\hspace{-0.1ex}i}} $} \hspace{-0.1ex}
= \hspace{.2ex} \scalebox{.9}{$ \displaystyle \frac{\raisemath{-0.2em}{\partial x_{\hspace{-0.2ex}j}}}{\partial \hspace{-0.1ex} X_{\hspace{-0.1ex}i}} $} \hspace{.25ex} \bm{e}_i \bm{e}_{\hspace{-0.1ex}j} \hspace{-0.2ex}
= \partial_i \hspace{.1ex} x_{\hspace{-0.2ex}j} \hspace{.1ex} \bm{e}_i \bm{e}_{\hspace{-0.1ex}j}
\end{array}
\end{equation*}

...

\begin{otherlanguage}{russian}

По~теореме о~полярном разложении~(\chapdotpararef{chapter:elementsoftensorcalculus}{para:polardecomposition}), градиент движения разлож\'{и}м на \en{rotation tensor}\ru{тензор поворота}~$\rotationtensor$ и~симметричные положительные \en{stretch tensors}\ru{тензоры искажений}~${\bm{U}\hspace{-0.25ex}}$ \en{and}\ru{и}~${\bm{V}\hspace{-0.1ex}}$:

\nopagebreak\vspace{-0.1em}\begin{equation*}
 \bm{F} \hspace{-0.1ex} = \rotationtensor \dotp \hspace{.25ex} \bm{U} \hspace{-0.2ex} = \bm{V} \hspace{-0.3ex} \dotp \hspace{.15ex} \rotationtensor
\end{equation*}

...

Когда нет поворота~(${\rotationtensor = \hspace{-0.1ex} \bm{E} \hspace{0.1ex}}$), тогда ${\bm{F} \hspace{-0.1ex} = \hspace{0.1ex} \bm{U} \hspace{-0.25ex} = \bm{V}\hspace{-0.3ex}}$.

...



\end{otherlanguage}

\en{\section{Measures (tensors) of deformation}}

\ru{\section{Меры (тензоры) деформации}}

\label{para:deformationtensors}

\en{Motion gradient}\ru{Градиент движения}~$\bm{F}$ \en{characterizes}\ru{характеризует} \en{both the~deformation of a~body and the~rotation of a~body as a~whole}\ru{и~деформацию тела, и~поворот тела как~целого}.
\inquotes{\en{Pure}\ru{Чистыми}} \en{deformation tensors}\ru{тензорами деформации} \en{are}\ru{являются} \en{stretch tensors}\ru{тензоры искажений}~${\bm{U}\hspace{-0.25ex}}$ \en{and}\ru{и}~${\bm{V}\hspace{-0.1ex}}$ \en{from polar decomposition}\ru{из полярного разложения} ${\bm{F} \hspace{-0.1ex} = \rotationtensor \dotp \hspace{.25ex} \bm{U} \hspace{-0.2ex} = \bm{V} \hspace{-0.3ex} \dotp \hspace{.15ex} \rotationtensor}$, \en{as well as}\ru{так~же как и}~\en{other tensors}\ru{другие тензоры}, \en{originating}\ru{происходящие} \en{from}\ru{от} ${\bm{U}\hspace{-0.25ex}}$ \ru{или\,(и)}\en{or\,(and)}~${\bm{V}\hspace{-0.3ex}}$.

\en{Widely used ones are}\ru{Широко используются} \inquotes{\en{squares}\ru{квадраты}} \en{of~}${\bm{U}\hspace{-0.25ex}}$ \en{and}\ru{и}~${\bm{V}\hspace{-0.1ex}}$

\nopagebreak\vspace{-0.1em}\begin{equation}\label{deformationtensors.nonlinear}
\begin{array}{c}
\hspace*{-2.33em} \bigl( \hspace{.1ex} \bm{U}^{\hspace{.1ex}2} \hspace{-0.3ex} = \hspace{.22ex} \bigr) \hspace{.8ex}
\bm{U} \hspace{-0.3ex} \dotp \hspace{.1ex} \bm{U} \hspace{-0.2ex}
= \bm{F}^{\hspace{.1ex}\T} \hspace{-0.4ex} \dotp \bm{F}
\equiv \bm{G}
\hspace{.1ex} ,
\\[.1em]
%
\hspace*{-2.33em} \bigl( \hspace{.1ex} \bm{V}^{\hspace{.04ex}2} \hspace{-0.3ex} = \hspace{.22ex} \bigr) \hspace{.8ex}
\bm{V} \hspace{-0.3ex} \dotp \bm{V} \hspace{-0.25ex}
= \bm{F} \dotp \bm{F}^{\hspace{.1ex}\T} \hspace{-0.36ex}
\equiv \mathboldPhi
\hspace{.1ex} .
\end{array}
\end{equation}

\vspace{-0.25em} \noindent \en{These are}\ru{Это} \ru{тензор деформации }Green’\en{s}\ru{а}\en{ deformation tensor}~(\en{or}\ru{или} \en{right}\ru{правый} \ru{тензор }Cauchy--Green\ru{’а}\en{ tensor})~$\bm{G}$ \en{and}\ru{и}~\ru{тензор деформации }Finger’\en{s}\ru{а}\en{ deformation tensor}~(\en{or}\ru{или} \en{left}\ru{левый} \ru{тензор }Cauchy--Green\ru{’а}\en{ tensor})~$\mathboldPhi$.
\en{The~convenient relation}\ru{Удобная связь} \en{to}\ru{с}~$\bm{F}$\:---  \en{without}\ru{без} \en{extracting roots}\ru{извлечения корней} \ru{как у}\en{as with}~${\bm{U}\hspace{-0.25ex}}$ \en{and}\ru{и}~${\bm{V}\hspace{-0.25ex}}$\:--- \en{explains}\ru{объясняет} \en{the~big popularity}\ru{больш\'{у}ю популярность} \en{of~tensors}\ru{тензоров}~$\bm{G}$ \en{and}\ru{и}~$\mathboldPhi$.
\en{Tensor}\ru{Тензор}~$\bm{G}$ \en{was first used}\ru{впервые использовал} \en{by }George Green\hspace{-0.1ex}%
\footnote{%
\href{https://en.wikipedia.org/wiki/George_Green_(mathematician)}{\bibauthor{Green, George}}. \href{https://hdl.handle.net/2027/mdp.39015027059651?urlappend=\%3Bseq=133}{(1839) On the~propagation of~light in crystallized media~//~Transactions of the~Cambridge Philosophical Society, 1842, vol.\:7, part~II, pages 121\hbox{--}140.}
}\hspace{-0.5ex}.

\en{Inversion}\ru{Обращение} \en{of~}$\mathboldPhi$ \en{and}\ru{и}~$\bm{G}$ \en{gives}\ru{даёт} \en{two more}\ru{ещё два} \en{deformation tensors}\ru{тензора деформации}

\nopagebreak\vspace{-0.2em}\begin{equation}\label{moredeformationtensors.nonlinear}
\begin{array}{c}
\bm{V}^{\expminustwo} \hspace{-0.25ex}
= \mathboldPhi^{\expminusone} \hspace{-0.2ex}
= \hspace{-0.1ex} \left( \bm{F} \dotp \bm{F}^{\hspace{.1ex}\T} \hspace{.1ex} \right)^{\hspace{-0.33ex}\expminusone} \hspace{-0.4ex}
= \bm{F}^{\expminusT} \hspace{-0.3ex} \dotp \bm{F}^{\expminusone} \hspace{-0.25ex}
\equiv {^2\hspace{-0.2ex}\bm{c}}
\hspace{.1ex} ,
\\
%
\bm{U}^{\expminustwo} \hspace{-0.25ex}
= \bm{G}^{\hspace{.12ex}\expminusone} \hspace{-0.2ex}
= \hspace{-0.1ex} \left( \bm{F}^{\hspace{.1ex}\T} \hspace{-0.4ex} \dotp \bm{F} \hspace{.2ex} \right)^{\hspace{-0.33ex}\expminusone} \hspace{-0.4ex}
= \bm{F}^{\expminusone} \hspace{-0.3ex} \dotp \bm{F}^{\expminusT} \hspace{-0.3ex}
\equiv {^2\hspace{-0.4ex}\bm{f}}
\hspace{-0.1ex} ,
\end{array}
\end{equation}

\vspace{-0.2em} \noindent \en{each of~which}\ru{каждый из~которых} \en{is sometimes called}\ru{иногда называется} \ru{тензором }\href{https://en.wikipedia.org/wiki/Gabrio_Piola}{Piola}\en{ tensor} \en{or}\ru{или} \ru{тензором }\href{https://en.wikipedia.org/wiki/Josef_Finger}{Finger}\ru{’а}\en{ tensor}. \en{The inverse}\ru{Обратный} \en{of left}\ru{к~левому} \ru{тензору }Cauchy--Green\ru{’а}\en{ tensor}~${\hspace{-0.1ex}\mathboldPhi\hspace{.1ex}}$\en{ is}\ru{\:---} \ru{тензор деформации }Cauchy\en{ deformation tensor}~${\hspace{-0.2ex}^2\hspace{-0.2ex}\bm{c}}$.
% \href{https://en.wikipedia.org/wiki/Augustin-Louis_Cauchy}{\bibauthor{Augustin Louis Cauchy}}. Exercices de math\'{e}matiques. Troisi\`{e}me ann\'{e}e: 1828~(vol.\:III). Paris, Chez de Bure fr\`{e}res.

\en{Components}\ru{Компоненты} \en{of the~presented}\ru{представленных} \en{deformation tensors}\ru{тензоров деформации}

\nopagebreak\vspace{-0.1em}\begin{equation}\label{componentsofdeformationtensors}
\begin{array}{r@{\hspace{.5em}}c}
\bm{G} = \bm{r}^i \hspace{-0.25ex} \bm{R}_i \hspace{-0.1ex} \dotp \hspace{-0.15ex} \bm{R}_{\hspace{-0.1ex}j} \bm{r}^j \hspace{-0.25ex}
= G_{\hspace{-0.15ex}i\hspace{-0.1ex}j} \hspace{.1ex} \bm{r}^i \bm{r}^j
\hspace{-0.3ex} , &
G_{\hspace{-0.15ex}i\hspace{-0.1ex}j} \hspace{-0.2ex} \equiv
\bm{R}_i \hspace{-0.1ex} \dotp \hspace{-0.15ex} \bm{R}_{\hspace{-0.1ex}j}
\hspace{.1ex} ,
\\[.25em]
%
{^2\hspace{-0.4ex}\bm{f}} =  \bm{r}_i \bm{R}^{\hspace{.1ex}i} \hspace{-0.2ex} \dotp \hspace{-0.15ex} \bm{R}^j \hspace{-0.1ex} \bm{r}_{\hspace{-0.2ex}j} \hspace{-0.2ex}
= G^{\hspace{.1ex}i\hspace{-0.1ex}j} \bm{r}_i \bm{r}_{\hspace{-0.2ex}j} \hspace{-0.2ex}
\hspace{.2ex} , &
G^{\hspace{.1ex}i\hspace{-0.1ex}j} \hspace{-0.2ex} \equiv
\bm{R}^{\hspace{.1ex}i} \hspace{-0.2ex} \dotp \hspace{-0.15ex} \bm{R}^j
\hspace{-0.3ex} ,
\\[.25em]
%
{^2\hspace{-0.2ex}\bm{c}} = \hspace{-0.1ex} \bm{R}^{\hspace{.1ex}i} \bm{r}_i \hspace{-0.1ex} \dotp \hspace{.1ex} \bm{r}_{\hspace{-0.2ex}j} \bm{R}^j \hspace{-0.25ex}
= \textsl{g}_{i\hspace{-0.1ex}j} \bm{R}^{\hspace{.1ex}i} \hspace{-0.2ex} \bm{R}^j
\hspace{-0.3ex} , &
\textsl{g}_{i\hspace{-0.1ex}j} \hspace{-0.15ex} \equiv
\bm{r}_i \hspace{-0.1ex} \dotp \hspace{.1ex} \bm{r}_{\hspace{-0.2ex}j}
\hspace{.1ex} ,
\\[.25em]
%
\mathboldPhi = \hspace{-0.1ex} \bm{R}_i \bm{r}^i \hspace{-0.15ex} \dotp \hspace{.1ex} \bm{r}^j \hspace{-0.25ex} \bm{R}_{\hspace{-0.1ex}j} \hspace{-0.2ex}
= \textsl{g}^{\hspace{.2ex}i\hspace{-0.1ex}j} \hspace{-0.2ex} \bm{R}_i \bm{R}_{\hspace{-0.1ex}j}
\hspace{.1ex} , &
\textsl{g}^{\hspace{.2ex}i\hspace{-0.1ex}j} \hspace{-0.15ex} \equiv
\bm{r}^i \hspace{-0.15ex} \dotp \hspace{.1ex} \bm{r}^j
\end{array}
\end{equation}

\vspace{-0.2em} \noindent \en{coincide}\ru{совпадают} \en{with components}\ru{с~компонентами} \en{of the~unit}\ru{единичного}~(\inquotes{\en{metric}\ru{метрического}}) \en{tensor}\ru{тензора}

\nopagebreak\vspace{-0.4em}\begin{multline*}
\shoveleft{ \hspace{3em} \bm{E}
= \hspace{-0.2ex} \bm{R}_i \hspace{-0.1ex} \bm{R}^{\hspace{.1ex}i} \hspace{-0.2ex}
= G_{\hspace{-0.15ex}i\hspace{-0.1ex}j} \bm{R}^{\hspace{.1ex}i} \hspace{-0.2ex} \bm{R}^j \hspace{-0.25ex}
= \hspace{-0.2ex} \bm{R}^{\hspace{.1ex}i} \hspace{-0.2ex} \bm{R}_i \hspace{-0.2ex}
= G^{\hspace{.1ex}i\hspace{-0.1ex}j} \hspace{-0.2ex} \bm{R}_i \bm{R}_{\hspace{-0.1ex}j}
= \hfill }
\\[-0.3em]
= \bm{r}^i \bm{r}_i \hspace{-0.2ex}
= \textsl{g}^{\hspace{.2ex}i\hspace{-0.1ex}j} \bm{r}_i \bm{r}_{\hspace{-0.2ex}j} \hspace{-0.2ex}
= \bm{r}_i \bm{r}^i \hspace{-0.25ex}
= \textsl{g}_{i\hspace{-0.1ex}j} \hspace{.1ex} \bm{r}^i \bm{r}^j
\hspace{-0.2ex} ,
\end{multline*}

\vspace{-0.3em} \noindent \en{but}\ru{но} ... lorem ipsum dolor sit amet where are the flowers gone wo sind sie geblieben

...


\subsection*{The right Cauchy\hbox{--}Green deformation tensor}

George Green discovered a deformation tensor known as the right Cauchy\hbox{--}Green deformation tensor or Green’s deformation tensor

\nopagebreak\begin{equation*}
\bm{G}
= \bm{F}^{\hspace{.1ex}\T} \hspace{-0.5ex} \dotp \bm{F}
= \bm{U}^{2}
\hspace{1em} \text{\en{or}\ru{или}} \hspace{1em}
G_{i\hspace{-0.1ex}j} \hspace{-0.2ex}
= F_{k' i} \hspace{.25ex} F_{k' \hspace{-0.15ex}j} \hspace{-0.15ex}
= \frac{\partial \hspace{-0.1ex} X_{\hspace{-0.1ex}k'}}{\partial x_{i}} \hspace{.15ex} \frac{\partial \hspace{-0.1ex} X_{\hspace{-0.1ex}k'}}{\partial x_{\hspace{-0.2ex}j}}
\hspace{.1ex} .
\end{equation*}

This tensor \textcolor{magenta}{gives the \inquotes{square} of local change in distances} due to deformation: ${\displaystyle d\bm{R} \dotp d\bm{R} = d\bm{r} \dotp \bm{G} \dotp d\bm{r}}$

Invariants of~${\bm{G}}$ are used in expressions for (density of) isotropic body’s potential energy of deformation. The most popular invariants are
\[
\begin{array}{r@{\hspace{.25em}}c@{\hspace{.33em}}l}
\mathrm{I}\hspace{.16ex}(\bm{G}) & \equiv &
\operatorname{tr} \hspace{.1ex} \bm{G}
= G_{ii} \hspace{-0.2ex} = \lambda_{1}^{2} + \lambda_{2}^{2} + \lambda_{3}^{2}
\\[.25em]
%
\mathrm{II}\hspace{.16ex}(\bm{G}) & \equiv &
\smalldisplaystyleonehalf \bigl( G_{\hspace{-0.2ex}j\hspace{-0.1ex}j}^{\hspace{.25ex}2} \hspace{-0.1ex} - G_{ik} G_{ki} \hspace{.1ex} \bigr) \hspace{-0.25ex}
= \lambda_{1}^{2}\lambda_{2}^{2} + \lambda_{2}^{2}\lambda_{3}^{2} + \lambda_{3}^{2}\lambda_{1}^{2}
\\[.4em]
%
\mathrm{III}\hspace{.16ex}(\bm{G}) & \equiv &
\operatorname{det} \hspace{.1ex} \bm{G}
= \lambda_{1}^{2}\lambda_{2}^{2}\lambda_{3}^{2}
\end{array}
\]
where ${\lambda_{i}\hspace{-0.2ex}}$ are stretch ratios for unit fibers that are initially oriented along directions of eigenvectors of the right stretch tensor~${\bm{U}\hspace{-0.2ex}}$.

\subsection*{The inverse of Green’s deformation tensor}

Sometimes called Finger tensor or Piola tensor, the inverse of the right Cauchy\hbox{--}Green deformation tensor

\nopagebreak\vspace{-0.25em}\begin{equation*}
{^2\hspace{-0.4ex}\bm{f}}
= \bm{G}^{\expminusone} \hspace{-0.25ex}
= \bm{F}^{\expminusone} \hspace{-0.4ex} \dotp \bm{F}^{\expminusT}
\hspace{1em} \text{\en{or}\ru{или}} \hspace{1em}
f_{i\hspace{-0.1ex}j} \hspace{-0.2ex} = \frac{\partial x_{i}}{\partial \hspace{-0.1ex} X_{\hspace{-0.1ex}k'}} \hspace{.15ex} \frac{\partial x_{\hspace{-0.2ex}j}}{\partial \hspace{-0.1ex} X_{\hspace{-0.1ex}k'}}
\end{equation*}

\subsection*{The left Cauchy\hbox{--}Green or Finger deformation tensor}

Reversing the order of multiplication in the formula for the right Green–Cauchy deformation tensor leads to the left Cauchy\hbox{--}Green deformation tensor, defined as
\[
\mathboldPhi
= \bm{F} \dotp \bm{F}^{\hspace{.1ex}\T} \hspace{-0.4ex}
= \bm{V}^{2}
\hspace{1em} \text{\en{or}\ru{или}} \hspace{1em}
\Phi_{i\hspace{-0.1ex}j} \hspace{-0.2ex}
= \frac{\partial \hspace{-0.1ex} X_{i}}{\partial x_{\mathcircabove{k}}} \hspace{.15ex} \frac{\partial \hspace{-0.1ex} X_{\hspace{-0.15ex}j}}{\partial x_{\mathcircabove{k}}}
\]

The left Cauchy\hbox{--}Green deformation tensor is often called the Finger’s deformation tensor, named after Josef Finger (1894).

Invariants of ${\mathboldPhi}$ are also used in expressions for strain energy density functions. The conventional invariants are defined as
\[
{\begin{aligned}
I_{1} & \equiv \Phi_{ii} = \lambda_{1}^{2}+\lambda_{2}^{2}+\lambda_{3}^{2}
\\
%
I_{2} & \equiv {\tfrac{1}{2}} \left( \Phi_{ii}^{2} - \Phi_{jk}\Phi_{kj} \right) = \lambda_{1}^{2}\lambda_{2}^{2} + \lambda_{2}^{2}\lambda_{3}^{2}+\lambda_{3}^{2}\lambda_{1}^{2}
\\
%
I_{3} & \equiv \det \mathboldPhi = J^{2} = \lambda_{1}^{2}\lambda_{2}^{2}\lambda_{3}^{2}
\end{aligned}}
\]
where ${J \equiv \det{\bm{F}}}$ is determinant of the motion gradient.

\subsection*{The Cauchy deformation tensor}

The Cauchy deformation tensor is defined as the inverse of left Cauchy\hbox{--}Green deformation tensor~${\mathboldPhi^{\expminusone}}$

\nopagebreak\vspace{-0.4em}\begin{equation*}
{^2\hspace{-0.2ex}\bm{c}} = \mathboldPhi^{\expminusone} \hspace{-0.25ex}
= \bm{F}^{\expminusT} \hspace{-0.4ex} \dotp \bm{F}^{\expminusone}
\hspace{1em} \text{\en{or}\ru{или}} \hspace{1em}
c_{i\hspace{-0.1ex}j} \hspace{-0.2ex}
= \frac{\partial x_{\mathcircabove{k}}}{\partial \hspace{-0.1ex} X_{i}} \hspace{.15ex} \frac{\partial x_{\mathcircabove{k}}}{\partial \hspace{-0.1ex} X_{\hspace{-0.15ex}j}}
\end{equation*}

${\displaystyle d\bm{r} \dotp d\bm{r} = d\bm{R} \dotp {^2\hspace{-0.2ex}\bm{c}} \dotp d\bm{R}}$

This tensor is also called Piola tensor and Finger tensor in rheology and fluid dynamics literature.

\subsection*{Finite strain tensors}

The concept of \emph{strain} is used to evaluate how much a given displacement differs locally from a~body displacement as a~whole (a~\inquotes{rigid body displacement}). One of such strains for large deformations is the \emph{Green strain tensor}, also called \emph{Green\hbox{--}Lagrangian strain tensor} or \emph{Green\hbox{--}Saint\hbox{-\hspace{-0.2ex}}Venant strain tensor}, defined as

\nopagebreak\begin{equation*}
\displaystyle \bm{C} = \smalldisplaystyleonehalf ( \bm{G} - \bm{E} \hspace{.1ex} )
\hspace{1em} \text{\en{or}\ru{или}} \hspace{1em}
C_{i\hspace{-0.1ex}j} \hspace{-0.2ex} = \onehalf \hspace{-0.25ex} \left( \frac{\partial \hspace{-0.1ex} X_{\hspace{-0.1ex}k'}}{\partial x_{i}} \hspace{.15ex} \frac{\partial \hspace{-0.1ex} X_{\hspace{-0.1ex}k'}}{\partial x_{\hspace{-0.2ex}j}} - \delta_{i\hspace{-0.1ex}j} \hspace{-0.2ex} \right)
\end{equation*}

\noindent or as function of the displacement gradient tensor

\nopagebreak\begin{equation*}
\displaystyle \bm{C} = \smalldisplaystyleonehalf \hspace{-0.2ex} \left( \hspace{-0.1ex}
\boldnablacircled\bm{u}
+ \hspace{-0.1ex} \boldnablacircled\bm{u}^{\hspace{-0.1ex}\T} \hspace{-0.3ex}
+ \hspace{-0.1ex} \boldnablacircled\bm{u} \dotp \hspace{-0.1ex} \boldnablacircled\bm{u}^{\hspace{-0.1ex}\T}
\right)
\end{equation*}

\noindent in cartesian coordinates

\nopagebreak\begin{equation*}
\displaystyle C_{i\hspace{-0.1ex}j} \hspace{-0.2ex} = \onehalf \hspace{-0.25ex} \left(
\frac{\partial u_{\hspace{-0.1ex}j}}{\partial x_{i}}
+ \frac{\partial u_{i}}{\partial x_{\hspace{-0.2ex}j}}
+ \frac{\partial u_{k}}{\partial x_{i}} \frac{\partial u_{k}}{\partial x_{\hspace{-0.2ex}j}}
\right)
\end{equation*}

The Green strain tensor measures how much $\bm{G}$ differs from~$\bm{E}$.

The \emph{Almansi\hbox{--}Hamel strain tensor}, referenced to the deformed configuration (\inquotes{Eulerian description}), is defined as

\nopagebreak\vspace{-0.5em}\begin{equation*}
{^2\hspace{-0.2ex}\bm{a}} = \smalldisplaystyleonehalf ( \bm{E} - \hspace{-0.15ex} {^2\hspace{-0.2ex}\bm{c}} \hspace{.1ex} ) = \smalldisplaystyleonehalf (\bm{E} - \mathboldPhi^{\expminusone})
\hspace{1em} \text{\en{or}\ru{или}} \hspace{1em}
a_{i\hspace{-0.1ex}j} \hspace{-0.2ex} = \onehalf \hspace{-0.25ex} \left( \hspace{-0.4ex} \delta _{i\hspace{-0.1ex}j} - \frac{\partial x_{\mathcircabove{k}}}{\partial \hspace{-0.1ex} X_{i}} \hspace{.15ex} \frac{\partial x_{\mathcircabove{k}}}{\partial \hspace{-0.1ex} X_{\hspace{-0.15ex}j}} \right)
\end{equation*}

\vspace{-0.4em} \noindent or as function of the displacement gradient

\nopagebreak\begin{equation*}
{^2\hspace{-0.2ex}\bm{a}} = \smalldisplaystyleonehalf \hspace{-0.2ex} \left( \hspace{-0.1ex}
\boldnabla\bm{u}^{\hspace{-0.1ex}\T} \hspace{-0.3ex}
+ \hspace{-0.1ex} \boldnabla\bm{u}
- \hspace{-0.1ex} \boldnabla\bm{u} \dotp \hspace{-0.1ex} \boldnabla\bm{u}^{\hspace{-0.1ex}\T}
\right)
\end{equation*}

\nopagebreak\vspace{-0.2em}\begin{equation*}
\displaystyle a_{i\hspace{-0.1ex}j} \hspace{-0.2ex} = \onehalf \hspace{-0.25ex} \left(
\frac{\partial u_{i}}{\partial \hspace{-0.1ex} X_{\hspace{-0.15ex}j}}
+ \frac{\partial u_{\hspace{-0.1ex}j}}{\partial \hspace{-0.1ex} X_{i}}
- \frac{\partial u_{k}}{\partial X_{i}} \frac{\partial u_{k}}{\partial \hspace{-0.1ex} X_{\hspace{-0.15ex}j}}
\right)
\end{equation*}

\subsection*{Seth\hbox{--}Hill family of abstract strain tensors}

B. R. Seth was the first to show that the Green and Almansi strain tensors are special cases of a more abstract strain measure. The idea was further expanded upon by Rodney Hill in~1968 \textcolor{red}{(publication??)}. The Seth\hbox{--}Hill family of strain measures (also called Doyle\hbox{--}Ericksen tensors) is expressed as

\nopagebreak\vspace{-0.1em}\begin{equation*}
\displaystyle \bm{D}_{(m)} \hspace{-0.2ex}
= \frac{\raisebox{-0.2em}{1}}{2m} \left( \hspace{.1ex} \bm{U}^{2m} \hspace{-0.4ex} - \bm{E} \hspace{.2ex} \right)
= \frac{\raisebox{-0.2em}{1}}{2m} \left( \bm{G}^{m} \hspace{-0.4ex} - \bm{E} \hspace{.1ex} \right) \end{equation*}

\vspace{.1em} \noindent \en{For various}\ru{Для разных}~$m$
\en{it gives}\ru{это даёт}

\nopagebreak\begin{equation*}
\begin{array}{r@{\hspace{0.1em}}ll}
\bm{D}_{(1)} & = \smalldisplaystyleonehalf \hspace{-0.25ex} \left( \bm{U}^{2} \hspace{-0.25ex} - \bm{E} \right) = \smalldisplaystyleonehalf (\bm{G} - \bm{E}) & \text{\scalebox{0.92}[0.92]{Green strain tensor}}
\\[.4em]
\bm{D}_{(\nicefrac{1}{2})} & = \bm{U} \hspace{-0.15ex} - \bm{E} = \bm{G}^{\hspace{.1ex}\nicefrac{1}{2}} \hspace{-0.25ex} - \bm{E} & \text{\scalebox{0.92}[0.92]{Biot strain tensor}}
\\[.4em]
\bm{D}_{(0)} & = \ln \bm{U} = \smalldisplaystyleonehalf \ln \bm{G} & \text{\scalebox{0.92}[0.92]{logarithmic strain, Hencky strain}}
\\[.4em]
\bm{D}_{(-\hspace{-0.1ex}1)} & = \smalldisplaystyleonehalf \hspace{-0.25ex} \left( \hspace{-0.1ex} \bm{E} - \bm{U}^{-2} \hspace{.1ex} \right) & \text{\scalebox{0.92}[0.92]{Almansi strain}}
\end{array}
\end{equation*}

The second\hbox{-}order approximation of these tensors is
\[ \bm{D}_{(m)} \hspace{-0.2ex} =
\mathboldepsilon
+ \smalldisplaystyleonehalf \hspace{.1ex} \boldnabla\bm{u} \dotp \hspace{-0.1ex} \boldnabla\bm{u}^{\hspace{-0.1ex}\T} \hspace{-0.3ex}
- (1 - m) \hspace{0.2ex} \mathboldepsilon \dotp \mathboldepsilon \]

\vspace{-0.25em} \noindent where ${\mathboldepsilon \equiv \hspace{-0.2ex} \boldnabla {\bm{u}}^{\hspace{.1ex}\mathsf{S}}}$ is the infinitesimal strain tensor.

Many other different definitions of measures~$\bm{D}$ are possible, provided that they satisfy these conditions:

\begin{itemize}
\item $\bm{D}$ vanishes for any movement of a~body as a~rigid whole
\item dependence of~$\bm{D}$ on displacement gradient tensor~${\nabla \bm{u}}$ is continuous, continuously differentiable and monotonic
\item it’s desired that $\bm{D}$ reduces to the infinitesimal strain tensor~${\mathboldepsilon}$ when ${\boldnabla \bm{u} \to 0}$
\end{itemize}

\noindent For example, tensors from the set
\[ \displaystyle \bm{D}^{(n)} \hspace{-0.32ex} = \left( {\bm{U}}^{n} \hspace{-0.4ex} - {\bm{U}}^{-n} \right) \hspace{-0.4ex} / \hspace{.25ex} 2n \]
aren’t from the Seth\hbox{--}Hill family, but for any~$n$ they have the same 2nd\hbox{-}order approximation as Seth\hbox{--}Hill measures with~${m=0}$.

\vspace{.4em} \noindent \hfill \textboldoblique{Wikipedia, the free encyclopedia}\:--- \href{https://en.wikipedia.org/wiki/Finite_strain_theory}{Finite strain theory}

\vspace{1cm}

\begin{otherlanguage}{russian}

...

Как отмечалось в~\chapdotpararef{chapter:elementsoftensorcalculus}{para:polardecomposition}, тензоры


...



\end{otherlanguage}

\en{\section{Velocity field}}

\ru{\section{Поле скоростей}}

\label{para:velocityfield}

\en{This topic}\ru{Эта тема} \en{is discussed}\ru{обсуждается} \en{in nearly any}\ru{в~почти любой} \en{book}\ru{книге} \en{about~continuum mechanics}\ru{о~механике сплошной среды}, \en{however}\ru{однако} \en{for}\ru{для} \en{solid elastic continua}\ru{твёрдых упругих сред} \en{it’s not very vital}\ru{она не~столь насущна}.
\en{Among various}\ru{Среди разных} \en{models of~material continuum}\ru{моделей материального контину\kern-0.11exума}, \en{an elastic solid body}\ru{упругое твёрдое тело} \en{is distinguished}\ru{выделяется} \en{by interesting possibility}\ru{интересной возможностью} \en{of~deriving}\ru{вывода} \en{the~complete set}\ru{полного набора}~(\en{system}\ru{системы}) \en{of~equations}\ru{уравнений} \en{for it}\ru{для него} \en{via single logically flawless procedure}\ru{единой логически безупречной процедурой}.
\en{But now}\ru{Но пока} \en{we follow the~way}\ru{мы идём путём}, \en{usual}\ru{обычным} \en{for}\ru{для} \en{fluid continuum mechanics}\ru{механики сплошной текучей среды}.

\en{So}\ru{Итак}, \en{there’s}\ru{есть} \en{velocity field}\ru{поле скоростей} \en{in spatial description}\ru{в~пространственном описании} ${\bm{v} \equiv \hspace{-0.1ex} \mathdotabove{\bm{R}} = \bm{v}(\bm{R}, t)}$.
\en{Decomposition}\ru{Разложение} \en{of~tensor}\ru{тензора} ${\hspace{-0.1ex}\boldnabla \bm{v} \hspace{-0.16ex} = \hspace{-0.2ex} \boldnabla \hspace{-0.16ex} \mathdotabove{\bm{R}} = \hspace{-0.16ex} \bm{R}^{\hspace{.1ex}i} \partial_i \mathdotabove{\bm{R}} = \hspace{-0.16ex} \bm{R}^{\hspace{.1ex}i} \hspace{-0.2ex} \mathdotabove{\bm{R}}_i}$%
\kern-0.15ex\footnote{For sufficiently smooth functions, partial derivatives always commute, space and time ones too. Thus

\nopagebreak\vspace{-0.6em}\begin{equation*}
\scalebox{0.92}{$\displaystyle
\frac{\raisemath{-0.2em}{\partial}}{\partial q^i} \frac{\raisemath{-0.2em}{\partial \hspace{-0.1ex} \bm{R}}}{\partial t}
= \frac{\raisemath{-0.2em}{\partial}}{\partial t} \frac{\raisemath{-0.2em}{\partial \hspace{-0.1ex} \bm{R}}}{\partial q^i}
$}
\hspace{.8em} \text{\en{or}\ru{или}} \hspace{.8em}
\partial_i \mathdotabove{\bm{R}} = \hspace{-0.1ex} \mathdotabove{\bm{R}}_i
\end{equation*}
}
\en{into symmetric and skew\hbox{-}symmetric parts}\ru{на симметричную и~кососимметричную части}~(\chapdotpararef{chapter:elementsoftensorcalculus}{para:tensors.symmetric+skewsymmetric})

\nopagebreak\vspace{-0.1em}\begin{equation*}
\boldnabla \hspace{-0.16ex} \mathdotabove{\bm{R}}
= \hspace{-0.2ex} \boldnabla \hspace{-0.16ex} \mathdotabove{\bm{R}}^{\hspace{.3ex}\mathsf{S}} \hspace{-0.25ex}
- \hspace{.1ex} \smalldisplaystyleonehalf \hspace{-0.33ex} \left( \hspace{-0.2ex} \boldnabla \hspace{-0.15ex}\times\hspace{-0.25ex} \mathdotabove{\bm{R}} \right) \hspace{-0.4ex} \times \hspace{-0.2ex} \bm{E}
\end{equation*}

\noindent \en{or}\ru{или}, \en{introducing}\ru{вводя} \en{strain rate tensor}\ru{тензор скорости деформации}~${\strainratetensor}$ \en{and}\ru{и} \en{vorticity tensor}\ru{тензор вихря}~${\vorticitytensor}$

\nopagebreak\begin{equation}
\begin{array}{c}
\boldnabla \bm{v} = \strainratetensor - \hspace{-0.1ex} \vorticitytensor
,
\\[.333em]
%
\strainratetensor \equiv \hspace{-0.12ex} \boldnabla {\bm{v}}^{\hspace{.2ex}\mathsf{S}} \hspace{-0.32ex}
=  \hspace{-0.2ex} \boldnabla \hspace{-0.16ex} \mathdotabove{\bm{R}}^{\hspace{.3ex}\mathsf{S}} \hspace{-0.32ex}
= \smalldisplaystyleonehalf \hspace{-0.33ex} \left( \hspace{-0.3ex} \bm{R}^{\hspace{.1ex}i} \hspace{-0.2ex} \mathdotabove{\bm{R}}_i \hspace{-0.1ex} + \hspace{-0.1ex} \mathdotabove{\bm{R}}_i \bm{R}^{\hspace{.1ex}i} \right)
\hspace{-0.44ex} ,
\\[.44em]
%
- \hspace{.1ex} \vorticitytensor \equiv \hspace{-0.12ex} \boldnabla {\bm{v}}^{\hspace{.2ex}\mathsf{A}} \hspace{-0.25ex}
= \hspace{-0.1ex} - \hspace{.24ex} \vorticityvector \hspace{-0.12ex} \times \hspace{-0.2ex} \bm{E}
\hspace{.1ex} , \hspace{.6em}
\vorticityvector \equiv \hspace{.1ex} \smalldisplaystyleonehalf \hspace{.15ex} \boldnabla \hspace{-0.2ex} \times \hspace{-0.1ex} \bm{v}
= \hspace{.1ex} \smalldisplaystyleonehalf \hspace{.15ex} \bm{R}^{\hspace{.1ex}i} \hspace{-0.4ex}\times\hspace{-0.35ex} \mathdotabove{\bm{R}}_i
\hspace{.1ex} ,
\end{array}
\end{equation}

\vspace{-0.1em} \noindent \en{where}\ru{где} \en{figures}\ru{фигурирует} \en{vorticity }(\en{pseudo}\ru{псевдо})\en{vector}\ru{вектор}\ru{ вихря}~${\vorticityvector}$, \en{the~companion of~vorticity tensor}\ru{сопутствующий тензору вихря}~${\vorticitytensor}$.

\ru{Компоненты}\en{Components} \en{of the~strain rate tensor}\ru{тензора скорости деформации} \en{in current configuration’s basis}\ru{в~базисе текущей конфигурации}

\nopagebreak\vspace{-0.66em}\begin{multline*}
\strainratetensor = \strainratetensorcomponents{i\hspace{-0.1ex}j} \bm{R}^{\hspace{.1ex}i} \hspace{-0.2ex} \bm{R}^j
\hspace{-0.3ex} , \hspace{.6em}
\strainratetensorcomponents{i\hspace{-0.1ex}j} \hspace{-0.2ex}
= \hspace{-0.1ex} \bm{R}_i \hspace{-0.1ex} \dotp \strainratetensor \hspace{-0.15ex} \dotp \hspace{-0.15ex} \bm{R}_{\hspace{-0.1ex}j} \hspace{-0.1ex}
= \hspace{.1ex} \smalldisplaystyleonehalf \hspace{.25ex} \bm{R}_i \hspace{-0.1ex} \dotp \hspace{-0.25ex} \left( \hspace{-0.25ex} \bm{R}^{\hspace{.05ex}k} \hspace{-0.2ex} \mathdotabove{\bm{R}}_k \hspace{-0.1ex} + \hspace{-0.1ex} \mathdotabove{\bm{R}}_k \bm{R}^{\hspace{.05ex}k} \hspace{-0.1ex} \right) \hspace{-0.4ex} \dotp \hspace{-0.15ex} \bm{R}_{\hspace{-0.1ex}j} \hspace{-0.2ex}
=
\\[-0.3em]
\shoveright{ \hfill \hspace{12em}
= \hspace{.1ex} \smalldisplaystyleonehalf \hspace{-0.33ex} \left( \hspace{-0.25ex} \mathdotabove{\bm{R}}_i \hspace{-0.1ex} \dotp \hspace{-0.15ex} \bm{R}_{\hspace{-0.1ex}j} \hspace{-0.1ex} + \hspace{-0.1ex} \bm{R}_i \hspace{-0.1ex} \dotp \hspace{-0.15ex} \mathdotabove{\bm{R}}_{\hspace{-0.1ex}j} \hspace{-0.1ex} \right) \hspace{-0.3ex}
= \hspace{.1ex} \smalldisplaystyleonehalf \hspace{-0.33ex} \left(^{\mathstrut} \hspace{-0.4ex}
\bm{R}_i \hspace{-0.1ex} \dotp \hspace{-0.15ex} \bm{R}_{\hspace{-0.1ex}j}
\hspace{-0.2ex} \right)^{\hspace{-0.3ex}\tikz[baseline=-0.5ex]\draw[black, fill=black] (0,0) circle (.266ex);}
}
\end{multline*}

...

\[
\mathdotabove{G}_{\hspace{-0.15ex}i\hspace{-0.1ex}j}
\]

\[
G_{\hspace{-0.15ex}i\hspace{-0.1ex}j} \hspace{-0.2ex} \equiv
\bm{R}_i \hspace{-0.1ex} \dotp \hspace{-0.15ex} \bm{R}_{\hspace{-0.1ex}j}
\]

\begin{otherlanguage}{russian}

...

Для упругих сред дискуссия о~поворотах не~нужна, истинное представление появляется в~ходе логически стройных выкладок \en{without additional hypotheses}\ru{без добавочных гипотез}.

\end{otherlanguage}

\newpage

\en{\section{Area vector. Surface change}} % Nanson’s formula

\ru{\section{Вектор пл\'{о}щади. Изменение площ\'{а}дки}} % формула Нансона

\en{Take an~infinitesimal surface. The area vector by length is equal to~the~surface’s area and is directed along the~normal to~this surface.}

\ru{Возьмём бесконечно м\'{а}лую площ\'{а}дку. Вектор пл\'{о}щади~(area vector) по~длине равен пл\'{о}щади площ\'{а}дки и направлен по~нормали к~этой площ\'{а}дке.}

\en{In~the~undeformed~(reference, original, initial, \inquotes{material}) configuration, the area vector can be represented as~${\bm{n} do}$. Surface’s area~$do$ is infinitely small, and~$\bm{n}$ is unit normal vector.}

\ru{В~недеформированной~(отсчётной, оригинальной, начальной, \inquotes{материальной}) конфигурации вектор пл\'{о}щади представ\'{и}м как~${\bm{n} do}$. Пл\'{о}щадь~$do$ бесконечно мал\'{а}, а~$\bm{n}$\:--- единичный вектор нормали.}

\en{In~the~present~(current, actual, deformed, \inquotes{spatial}) configuration, the~same surface has area vector~${\mathboldN dO}$.}

\ru{В~текущей~(актуальной, деформированной, \inquotes{пространственной}) конфигурации та~же площ\'{а}дка имеет вектор пл\'{о}щади~${\mathboldN dO}$.}

\en{With enough precision these infinitesimal surfaces are parallelograms}

\ru{С~достаточной точностью эти бесконечномалые площ\'{а}дки суть параллелограммы}

\nopagebreak\vspace{-0.1em}\en{\vspace{-0.32em}}
\begin{equation}\label{areavectorascrossproduct}
\begin{array}{c}
\bm{n} do = d \bm{r}^{'} \hspace{-0.5ex} \times \hspace{-0.1ex} d \bm{r}^{''} \hspace{-0.5ex}
= \frac{\partial \bm{r}}{\partial q^i} \hspace{.1ex} d q^i \hspace{-0.12ex} \times \frac{\partial \bm{r}}{\partial q^{j}} \hspace{.16ex} d q^{j}
= \bm{r}_i \hspace{-0.16ex} \times \hspace{-0.1ex} \bm{r}_{\hspace{-0.2ex}j} \hspace{0.2ex} dq^{i} dq^{j} \hspace{-0.2ex}, \\[.64em]
%
\mathboldN dO \hspace{-0.1ex} = d \bm{R}^{'} \hspace{-0.4ex} \times \hspace{-0.1ex} d \bm{R}^{''} \hspace{-0.5ex}
= \frac{\partial \bm{R}}{\partial q^i} \hspace{.1ex} d q^i \hspace{-0.12ex} \times \frac{\partial \bm{R}}{\partial q^{j}} \hspace{.16ex} d q^{j}
= \bm{R}_i \hspace{-0.1ex} \times \hspace{-0.2ex} \bm{R}_{\hspace{-0.1ex}j} \hspace{0.2ex} dq^{i} dq^{j} \hspace{-0.2ex}.
\end{array}
\end{equation}

\en{Applying transformation of~volume~\eqref{volumechange}, we have}

\ru{Применяя преобразование объёма~\eqref{volumechange}, имеем}

\nopagebreak\vspace{-0.25em}\begin{equation*}\begin{array}{c}
d\mathcal{V} \hspace{-0.2ex} = J d\mathcircabove{\mathcal{V}} \:\Rightarrow\,
\bm{R}_i \hspace{-0.1ex} \times \hspace{-0.2ex} \bm{R}_{\hspace{-0.1ex}j} \dotp \bm{R}_k = J \hspace{.12ex} \bm{r}_i \hspace{-0.16ex} \times \hspace{-0.1ex} \bm{r}_{\hspace{-0.2ex}j} \dotp \hspace{.16ex} \bm{r}_k \:\Rightarrow \\[.32em]
%
\Rightarrow\: \bm{R}_i \hspace{-0.1ex} \times \hspace{-0.2ex} \bm{R}_{\hspace{-0.1ex}j} \dotp \bm{R}_k \bm{R}^k = J \hspace{.12ex} \bm{r}_i \hspace{-0.16ex} \times \hspace{-0.1ex} \bm{r}_{\hspace{-0.2ex}j} \dotp \hspace{.16ex} \bm{r}_k \bm{R}^k
\:\Rightarrow\,
\bm{R}_i \hspace{-0.1ex} \times \hspace{-0.2ex} \bm{R}_{\hspace{-0.1ex}j} = J \hspace{.12ex} \bm{r}_i \hspace{-0.16ex} \times \hspace{-0.1ex} \bm{r}_{\hspace{-0.2ex}j} \dotp \bm{F}^{\hspace{.1ex}\expminusone} \hspace{-0.1ex}.
\end{array}\end{equation*}

\en{Hence through~\eqref{areavectorascrossproduct} we come to~the~relation}

\ru{Отсюда через~\eqref{areavectorascrossproduct} приходим к~соотношению}

\vspace{-0.16em}\begin{equation}\label{areachange:nansonformula}
\mathboldN dO = J \hspace{.1ex} \bm{n} do \dotp \bm{F}^{\hspace{.1ex}\expminusone} \hspace{-0.1ex},
\end{equation}

\nopagebreak \vspace{-0.25em}\en{\vspace{-0.15em}} \noindent \en{called Nanson’s formula}\ru{называемому формулой Nanson’а}.
%%~\cite{teodosiu-crystaldefects, chernyh-nonlinearelasticity}

\en{\section{Forces in continuum. Cauchy stress tensor}}

\ru{\section{Силы в сплошной среде. Тензор напряжения Cauchy}}

\label{para:stressviatetrahedron}

% forces are
% • linear force
% • angular torque (moment, moment of force, couple)

\en{Since}\ru{Поскольку} \en{particles of this model of~continuum}\ru{частицы этой модели контину\kern-0.11exума}\en{ are}\ru{\:---} \en{points with only translational degrees of~freedom}\ru{точки лишь с~трансляционными степенями свободы}%
\footnote{\en{Translational degrees of~freedom}\ru{Трансляционные степени свободы} \en{arise}\ru{возникают} \en{from}\ru{из} \en{a~particle’s ability}\ru{способности частицы} \en{to~move freely in~space}\ru{свободно двигаться в~пространстве}.}\hbox{\hspace{-0.5ex},}
\ru{то }\en{there’re no moments among generalized forces}\ru{среди обобщённых сил нет моментов}, \en{there’re no applied external couples}\ru{нет никаких приложенных внешних пар сил} \en{neither in~volume nor on~surface}\ru{ни в~объёме, ни по~поверхности}.

% forces are
% • body force, within a force field
% • contact force, via direct physical contact

\begin{otherlanguage}{russian}

На~бесконечно малый объём~$d\mathcal{V}$ действует сила~${\rho \bm{f} d\mathcal{V}}$; \hbox{если}~$\bm{f}$\:--- массовая сила (действующая на~единицу массы), то~${\rho \bm{f}}$\:--- объёмная. Такие силы происходят от силовых полей, например: гравитационные силы (\inquotes{силы тяжести}), силы инерции в~неинерциальных системах отсчёта, электромагнитные силы при~наличии в~среде зарядов и~токов.

На~бесконечно малую поверхность~${dO\hspace{-0.12ex}}$ действует поверхностная сила~${\bm{p} \hspace{0.2ex} dO\hspace{-0.12ex}}$. Это может~быть \en{contact pressure}\ru{контактное давление} \en{or/and}\ru{или/и}~\en{friction}\ru{трение}, электростатическая сила при~сосредоточенных на~поверхности зарядах.

\en{In the~material continuum}\ru{В~материальном контину\kern-0.11exуме}, \en{as in any mechanical system}\ru{как в~любой механической системе}, \en{external and~internal forces are distinguished}\ru{различаются силы внешние и~внутренние}.
Со~времён Euler’а и~Cauchy принимают, что внутренние силы в~среде\:--- это поверхностные силы близкодействия:
на~любой бесконечно малой площ\'{а}дке~${\mathboldN dO\hspace{-0.12ex}}$ внутри тела действует сила~${\tractionvector{N}\hspace{.16ex} dO\hspace{-0.12ex}}$.
Уточняя: действует с~той стороны, куда направлена единичная нормаль~$\mathboldN$.

\en{Vector}\ru{Вектор}~$\tractionvector{N}$ на~площ\'{а}дке с~нормалью~$\mathboldN$ называется вектором тракции~(traction vector) \en{or}\ru{или} \en{force\hbox{-}stress vector}\ru{вектором силового напряжения}.
\en{By the~action\hbox{--}reaction principle}\ru{По принципу действия и~противодействия}, $\tractionvector{N}$ меняет знак с~переменой направления~$\mathboldN$ на~противоположную сторону:
${\tractionvector{-N} = - \hspace{.2ex} \tractionvector{N}}$.
(\en{Sometimes}\ru{Иногда} \en{the~last}\ru{последний} \en{thesis}\ru{тезис} \en{is proved}\ru{доказывается} \en{via}\ru{через} \en{balance of~momentum}\ru{баланс импульса} \en{for an~infinitely short cylinder}\ru{для бесконечно короткого цилиндра} \en{with bases}\ru{с~основаниями} ${\mathboldN dO\hspace{-0.12ex}}$ \en{and}\ru{и}~${- \hspace{.1ex} \mathboldN dO\hspace{-0.12ex}}$.)

В~каждой точке среды имеем бесконечно много векторов~$\tractionvector{N}$, поскольку через точку проходят площ\'{а}дки любой ориентации.
\en{Thereby}\ru{Тем самым}, \en{stress at a~point of~continuum}\ru{напряжение в~точке среды}\en{ is not a~vector}\ru{\:--- не~вектор}, \en{but}\ru{но} совокупность~(множество) \en{of~all traction vectors}\ru{всех векторов тракции} для любых по\hbox{-}всякому ориентированных площ\'{а}док, содержащих в~себе эту точку.
\en{And it turns out}\ru{И~оказывается}, \en{the~infinite set}\ru{бесконечное множество} \en{of~all vectors}\ru{всех векторов}~$\tractionvector{N}$ \en{is completely determined}\ru{полностью определяется} \en{by the~only one second complexity tensor}\ru{одним-единственным тензором второй сложности}\:--- \ru{тензором напряжения }Cauchy\en{ stress tensor}~$\cauchystress$.
Рассмотрим содержащийся во~многих книгах вывод этого утверждения.

На~поверхности бесконечномалого материального тетраэдра ...


...


\end{otherlanguage}

\en{\section{Balance of momentum and rotational momentum}}

\ru{\section{Баланс импульса и момента импульса}}

\label{para:balance.elasticcontinuum}

\begin{otherlanguage}{russian}

Рассмотрим какой\hbox{-}либо кон\'{е}чный объём~$\mathcal{V}$ среды,
ограниченный поверхностью~${O(\boundary \mathcal{V})}$,
нагруженный поверхностными~${\bm{p}\hspace{.25ex}dO}$ и~массовыми/объёмными~${\bm{f}dm \hspace{-0.15ex} = \hspace{-0.1ex} \rho \bm{f} d\mathcal{V}}$ внешними силами.
Формулировка баланса импульса такова

\nopagebreak\vspace{-0.2em}\begin{equation}\label{balanceoftranslationalmomentum.integral}
\displaystyle\left( \integral\displaylimits_{\mathcal{V}} \hspace{-0.32ex} \rho \hspace{.2ex} \bm{v} \hspace{.1ex} d\mathcal{V} \hspace{-0.3ex} \right)^{\hspace{-0.32em}\tikz[baseline=-0.2ex]\draw[black, fill=black] (0,0) circle (.28ex);} \hspace{-0.1ex}
=
\integral\displaylimits_{\mathcal{V}} \hspace{-0.33ex} \rho \bm{f} \hspace{.1ex} d\mathcal{V}
\hspace{.25ex} + \hspace{.25ex}
\ointegral\displaylimits_{\mathclap{O(\boundary \mathcal{V})}} \hspace{-0.2ex} \bm{p} \hspace{.25ex} dO .
\end{equation}

\noindent
... ${\bm{p} = \hspace{-0.1ex} \tractionvector{N} \hspace{-0.1ex} = \mathboldN \dotp \cauchystress}$ ...

\noindent
Импульс слева найдём по~\eqref{volumeintegralinbothconfigurations}, а~поверхностный интеграл справа превратим в~объёмный по~теореме о~дивергенции. Получим

\nopagebreak\vspace{-0.1em}\begin{equation*}
\scalebox{0.96}[0.96]{$\displaystyle\integral\displaylimits_{\mathcal{V}} \hspace{-0.5ex} \left(^{\mathstrut} \hspace{-0.16ex} \boldnabla \dotp \cauchystress \hspace{0.15ex}
+ \rho \hspace{-0.1ex} \left( \bm{f} \hspace{-0.2ex} - \mathdotabove{\bm{v}} \right) \right) \hspace{-0.33ex} d\mathcal{V}$} \hspace{.1ex}
= \hspace{.1ex} \bm{0}
\hspace{.1ex} .
\end{equation*}

\vspace{-0.25em}\noindent Но объём~$\mathcal{V}$ произволен, поэтому равно нулю подынтегральное выражение. Приходим к~уравнению баланса импульса в~локальной~(дифференциальной) форме

\nopagebreak\vspace{-0.1em}\begin{equation}\label{balanceoftranslationalmomentum.local}
\boldnabla \dotp \cauchystress \hspace{.15ex}
+ \rho \hspace{-0.1ex} \left( \bm{f} \hspace{-0.2ex} - \mathdotabove{\bm{v}} \right)
= \hspace{.1ex} \bm{0}
\hspace{.1ex} .
\end{equation}

...

Теперь к~балансу момента импульса. Интегральная формулировка:

\nopagebreak\vspace{-0.32em}\begin{equation}\label{balanceofrotationalmomentum.integral}
\displaystyle\left( \integral\displaylimits_{\mathcal{V}} \hspace{-0.32ex} \bm{R} \times \hspace{-0.2ex} \rho \hspace{.2ex} \bm{v} \hspace{0.1ex} d\mathcal{V} \hspace{-0.25ex} \right)^{\hspace{-0.32em}\tikz[baseline=-0.2ex]\draw[black, fill=black] (0,0) circle (.28ex);} \hspace{-0.1ex}
= \integral\displaylimits_{\mathcal{V}} \hspace{-0.5ex} \bm{R} \times \hspace{-0.2ex} \rho \bm{f} \hspace{.1ex} d\mathcal{V}
\hspace{.25ex} + \hspace{.25ex}
\ointegral\displaylimits_{\mathclap{O(\boundary \mathcal{V})}} \hspace{-0.25ex} \bm{R} \times \hspace{-0.05ex} \bm{p} \hspace{.25ex} dO .
\end{equation}

\vspace{-0.12em}
Дифференцируя левую часть (${\bm{v} \equiv \hspace{-0.1ex} \mathdotabove{\bm{R}}\hspace{.2ex}}$)

\nopagebreak\vspace{-0.2em}\begin{equation*}
\displaystyle\left( \integral\displaylimits_{\mathcal{V}} \hspace{-0.4ex} \bm{R} \times \hspace{-0.2ex} \rho \hspace{.1ex} \mathdotabove{\bm{R}} \hspace{.4ex} d\mathcal{V} \hspace{-0.2ex} \right)^{\hspace{-0.32em}\tikz[baseline=-0.2ex]\draw[black, fill=black] (0,0) circle (.28ex);} \hspace{-0.1ex}
=
\integral\displaylimits_{\mathcal{V}} \hspace{-0.32ex} \bm{R} \times \hspace{-0.2ex} \rho \hspace{.1ex} \mathdotdotabove{\bm{R}} \hspace{.4ex} d\mathcal{V}
\hspace{.3ex} +
\integral\displaylimits_{\mathcal{V}} \hspace{-0.32ex} \tikzmark{RdotCrossRdotBegin} \mathdotabove{\bm{R}} \times \hspace{-0.2ex} \rho \hspace{.1ex} \mathdotabove{\bm{R}} \hspace{.1ex} \tikzmark{RdotCrossRdotEnd} \hspace{.3ex} d\mathcal{V} ,
\end{equation*}
\AddUnderBrace[line width=.75pt][0,-0.1ex][yshift=0.11em]%
{RdotCrossRdotBegin}{RdotCrossRdotEnd}{${\scalebox{0.8}{$\bm{0}$}}$}

\vspace{-0.6em}
\noindent применяя теорему о~дивергенции к~поверхностному интегралу
(... ${\bm{p} = \hspace{-0.1ex} \tractionvector{N} \hspace{-0.1ex} = \mathboldN \dotp \cauchystress}$ ...)

\nopagebreak\vspace{-0.33em}\begin{multline*}
\bm{R} \times \hspace{-0.2ex} \left( \mathboldN \dotp \cauchystress \hspace{.1ex} \right)
= \hspace{.1ex} - \hspace{-0.1ex} \left( \mathboldN \dotp \cauchystress \hspace{.1ex} \right) \hspace{-0.2ex} \times \hspace{-0.24ex} \bm{R}
%
= \hspace{.1ex} - \hspace{.2ex} \mathboldN \hspace{.1ex} \dotp \hspace{.1ex} \left( \cauchystress \times \hspace{-0.24ex} \bm{R} \hspace{.2ex} \right)
\: \Rightarrow \\
%
\Rightarrow \:
\ointegral\displaylimits_{\mathclap{O(\boundary \mathcal{V})}} \hspace{-0.25ex} \bm{R} \times \hspace{-0.2ex} \left( \mathboldN \dotp \cauchystress \hspace{.1ex} \right) \hspace{-0.1ex} dO
= - \hspace{-0.4ex} \integral\displaylimits_{\mathcal{V}} \hspace{-0.5ex} \boldnabla \dotp \left( \cauchystress \times \hspace{-0.24ex} \bm{R} \hspace{.2ex} \right) \hspace{-0.1ex} d\mathcal{V} ,
\end{multline*}

...

\nopagebreak\vspace{-0.2em}\begin{equation*}
\integral\displaylimits_{\mathcal{V}} \hspace{-0.5ex} \bm{R} \times \hspace{-0.2ex} \rho \hspace{.1ex} \mathdotdotabove{\bm{R}} \hspace{.4ex} d\mathcal{V} \hspace{.22ex}
= \hspace{-0.2ex} \integral\displaylimits_{\mathcal{V}} \hspace{-0.5ex} \bm{R} \times \hspace{-0.2ex} \rho \bm{f} \hspace{0.1ex} d\mathcal{V} \hspace{.2ex}
- \hspace{-0.2ex} \integral\displaylimits_{\mathcal{V}} \hspace{-0.5ex} \boldnabla \dotp \left( \cauchystress \times \hspace{-0.24ex} \bm{R} \hspace{.2ex} \right) \hspace{-0.1ex} d\mathcal{V} ,
\end{equation*}

\nopagebreak\vspace{-0.2em}\begin{equation*}
\integral\displaylimits_{\mathcal{V}} \hspace{-0.5ex} \bm{R} \times \hspace{-0.2ex} \rho \hspace{.15ex} \bigl( \bm{f} \hspace{-0.1ex} - \hspace{-0.1ex} \mathdotdotabove{\bm{R}} \hspace{.2ex} \bigr) \hspace{.1ex} d\mathcal{V} \hspace{.12ex}
- \hspace{-0.2ex} \integral\displaylimits_{\mathcal{V}} \hspace{-0.5ex} \boldnabla \dotp \left( \cauchystress \times \hspace{-0.24ex} \bm{R} \hspace{.2ex} \right) \hspace{-0.1ex} d\mathcal{V} \hspace{.1ex}
= \hspace{.15ex} \bm{0} \hspace{.1ex} ,
\end{equation*}

...

\nopagebreak\vspace{-0.2em}\begin{equation*}
\tikzmark{divergenceOfStressCrossLocationBegin} \hspace{-0.2ex} \boldnabla \dotp \hspace{.1ex} \left( \cauchystress \times \hspace{-0.24ex} \bm{R} \hspace{.24ex} \right) \hspace{-0.33ex} \tikzmark{divergenceOfStressCrossLocationEnd}
= \hspace{-0.33ex} \tikzmark{firstTermOfStressCrossLocationDivergenceBegin} \hspace{.1ex} \left( \hspace{.12ex} \boldnabla \dotp \cauchystress \hspace{.2ex} \right) \hspace{-0.25ex} \times \hspace{-0.2ex} \bm{R} \hspace{.2ex} \tikzmark{firstTermOfStressCrossLocationDivergenceEnd} \hspace{.15ex}
+ \hspace{.1ex} \bm{R}^{\hspace{.1ex}i} \hspace{-0.25ex} \dotp \hspace{-0.1ex} \left( \cauchystress \hspace{-0.1ex} \times \hspace{-0.12ex} \partial_i \bm{R} \hspace{.24ex} \right)
\end{equation*}
\AddUnderBrace[line width=.75pt][0.1ex,-0.1ex][yshift=0.11em]%
{divergenceOfStressCrossLocationBegin}{divergenceOfStressCrossLocationEnd}{${\scalebox{0.8}{$ \bm{R}^{\hspace{.1ex}i} \hspace{-0.25ex} \dotp \partial_i \hspace{-0.25ex} \left( \cauchystress \times \hspace{-0.24ex} \bm{R} \hspace{.24ex} \right) $}}$}%
\AddUnderBrace[line width=.75pt][0.2ex,-0.1ex][xshift=-0.1ex, yshift=0.11em]%
{firstTermOfStressCrossLocationDivergenceBegin}{firstTermOfStressCrossLocationDivergenceEnd}{${\scalebox{0.8}{$ \bm{R}^{\hspace{.1ex}i} \hspace{-0.25ex} \dotp \hspace{-0.1ex} \left( \hspace{.1ex} \partial_i \cauchystress \hspace{.2ex} \right) \hspace{-0.3ex} \times \hspace{-0.2ex} \bm{R} $}}$}

\noindent
${\cauchystress = \bm{e}_{i} \tractionvector{i}\hspace{.12ex}}$, ${\bm{e}_i \hspace{-0.16ex} = \boldconstant}$

\nopagebreak\vspace{-0.2em}\begin{multline*}
\bm{R}^{\hspace{.1ex}i} \hspace{-0.25ex} \dotp \hspace{-0.1ex} \left( \cauchystress \hspace{-0.2ex} \times \hspace{-0.2ex} \partial_i \bm{R} \hspace{.24ex} \right) \hspace{-0.1ex}
= \bm{R}^{\hspace{.1ex}i} \hspace{-0.2ex} \dotp \hspace{-0.1ex} \left( \hspace{.1ex} \bm{e}_{j} \tractionvector{j} \hspace{-0.2ex} \times \hspace{-0.2ex} \bm{R}_{\hspace{.1ex}i} \hspace{.12ex} \right) \hspace{-0.16ex}
= \bm{R}^{\hspace{.1ex}i} \hspace{-0.25ex} \dotp \bm{e}_{j} \tractionvector{j} \hspace{-0.2ex} \times \hspace{-0.2ex} \bm{R}_{\hspace{.1ex}i} \hspace{.2ex} =
\\[-0.15em]
%%= - \hspace{.2ex} \bm{R}^{\hspace{.1ex}i} \hspace{-0.25ex} \dotp \bm{e}_{j} \bm{R}_{\hspace{.1ex}i} \hspace{-0.25ex} \times \hspace{-0.25ex} \tractionvector{j}
= - \hspace{.25ex} \bm{e}_{j} \hspace{-0.15ex} \dotp \hspace{-0.1ex} \bm{R}^{\hspace{.1ex}i} \hspace{-0.2ex} \bm{R}_{\hspace{.1ex}i} \hspace{-0.25ex} \times \hspace{-0.25ex} \tractionvector{j}
= - \hspace{.25ex} \bm{e}_{j} \hspace{-0.15ex} \dotp \hspace{-0.1ex} \bm{E} \hspace{-0.2ex} \times \hspace{-0.25ex} \tractionvector{j}
= - \hspace{.25ex} \bm{e}_{j} \hspace{-0.25ex} \times \hspace{-0.2ex} \tractionvector{j}
= - \hspace{.2ex} \cauchystress_{\hspace{-0.1ex}\Xcompanion}
\end{multline*}


...


\end{otherlanguage}

\en{\section{Eigenvalues of Cauchy stress tensor}}

\ru{\section{Собственные числа тензора напряжения Коши}}

\begin{otherlanguage}{russian}

Как и~любой симметричный тензор, $\cauchystress$ имеет три вещественных собственных числ\'{а} $\mathsigma_i$, называемых главными напряжениями (principal stresses), а~также ортогональную тройку собственных векторов единичной длины $\bm{e}_i$.
\en{In~representation}\ru{В~представлении} ${\cauchystress = \hspace{-0.2ex} \sum \hspace{-0.16ex} \mathsigma_i \hspace{.16ex} \bm{e}_i \bm{e}_i}$ \en{most often}\ru{чаще всего} \en{indices are sorted as}\ru{индексы сортируются как} ${\mathsigma_1 \hspace{-0.1ex} \geq \mathsigma_2 \geq \mathsigma_3}$, а~тройка~${\bm{e}_i}$\:--- \inquotesx{правая}[.]

Известна теорема о~кругах Мора (Mohr’s circles)%
\footnote{Mohr’s circles, named after Christian Otto Mohr, is a~two-dimensional graphical representation of transformation for the Cauchy stress tensor.}

...



Чтобы замкнуть набор (систему) уравнений модели сплошной среды, нужно добавить определяющие отношения~(constitutive relations)\:--- уравнения, связывающие напряжение с~деформацией (и~другие необходимые связи). Однако, \en{for}\ru{для} \en{a~solid elastic continuum}\ru{твёрдого упругого контину\kern-0.11exума} такой длинный путь построения модели излишен, что читатель и~увидит ниже.

\end{otherlanguage}

\en{\section{Principle of virtual work (without Lagrange multipliers)}}

\ru{\section{Принцип виртуальной работы (без множителей Лагранжа)}}

\label{para:virtualworkprinciple.elastic}

\en{According to the~principle of~virtual work for some finite volume of~continuum}

\ru{Согласно принципу виртуальной работы для некоего конечного объёма сплошной среды}

\nopagebreak\ru{\vspace{-0.1em}}\begin{equation}\label{princlipleofvirtualwork.integral:nonlinearmomentlesscontinuum}
\integral\displaylimits_{\mathcal{V}} \hspace{-0.5ex} \left(^{\mathstrut} \hspace{-0.1ex} \rho \bm{f} \dotp \variation{\bm{R}} \hspace{.16ex} + \variation{\internalwork} \right) \hspace{-0.4ex} d\mathcal{V}
+ \ointegral\displaylimits_{\mathclap{O(\boundary \mathcal{V})}} \hspace{-0.2ex} \mathboldN \dotp \cauchystress \dotp \variation{\bm{R}} \hspace{.4ex} dO = \hspace{.1ex} 0
\hspace{.1ex} .
\end{equation}

\vspace{-0.1em} \noindent \en{Here}\ru{Здесь}
${\variation{\internalwork}}$\ru{\;---}\en{ is} \en{work of~internal forces per volume unit in the~current configuration}\ru{работа внутренних сил на~единицу объёма в~текущей конфигурации};
$\bm{f}$\ru{\:---}\en{ is} \en{mass force}\ru{массовая сила}, \en{with dynamics}\ru{с~динамикой} ${\bigl( \hspace{.1ex} \bm{f} \hspace{-0.1ex} - \hspace{-0.2ex} \mathdotdotabove{\bm{R}} \hspace{.33ex} \bigr)\hspace{-0.16ex}}$;
${\bm{p} = \hspace{-0.1ex} \tractionvector{N} \hspace{-0.1ex} = \mathboldN \dotp \cauchystress}$\ru{\:---}\en{ is} \en{surface force}\ru{поверхностная сила}.

\en{Applying the~divergence theorem to the~surface integral}\ru{Применяя к~поверхностному интегралу теорему о~дивергенции}, \en{using}\ru{используя}\footnote{%
${ \bm{R}^{\hspace{.1ex}i} \hspace{-0.25ex} \dotp \partial_i \hspace{-0.25ex} \left( \cauchystress \dotp \variation{\bm{R}} \hspace{.2ex} \right) \hspace{-0.15ex}
= \bm{R}^{\hspace{.1ex}i} \hspace{-0.25ex} \dotp \hspace{-0.1ex} \left( \partial_i \cauchystress \hspace{.12ex} \right) \hspace{-0.2ex} \dotp \variation{\bm{R}} \hspace{.15ex}
+ \hspace{-0.1ex} \bm{R}^{\hspace{.1ex}i} \hspace{-0.25ex} \dotp \hspace{-0.1ex} \cauchystress \dotp \partial_i \bigl( \variation{\bm{R}} \hspace{.2ex} \bigr)
\hspace{-0.1ex} , }$

 \hspace*{\fill}
${ \bm{R}^{\hspace{.1ex}i} \hspace{-0.25ex} \dotp \hspace{-0.1ex} \cauchystress \dotp \partial_i \bigl( \variation{\bm{R}} \hspace{.2ex} \bigr) \hspace{-0.2ex}
= \cauchystress \dotdotp \partial_i \bigl( \variation{\bm{R}} \hspace{.2ex} \bigr) \hspace{-0.1ex} \bm{R}^{\hspace{.1ex}i} \hspace{-0.1ex}
= \cauchystress \dotdotp \hspace{-0.12ex} \bigl( \bm{R}^{\hspace{.1ex}i} \partial_i \hspace{.12ex} \variation{\bm{R}} \hspace{.24ex} \bigr)^{\hspace{-0.33ex}\T} }$%
}

\nopagebreak\vspace{-0.1em}\begin{equation*}
\boldnabla \hspace{-0.1ex} \dotp \hspace{-0.1ex} \left( \cauchystress \dotp \variation{\bm{R}} \hspace{.2ex} \right) \hspace{-0.1ex}
= \boldnabla \hspace{-0.1ex} \dotp \cauchystress \dotp \variation{\bm{R}} \hspace{.15ex}
+ \cauchystress \dotdotp \hspace{-0.12ex} \boldnabla \hspace{.1ex} \variation{\bm{R}}^{\hspace{.1ex}\T}
\end{equation*}

\vspace{-0.1em} \noindent \en{and randomness of}\ru{и~случайность}~${\mathcal{V}\hspace{-0.2ex}}$, \en{we get the local differential edition of}\ru{получаем локальную дифференциальную формулировку}~\eqref{princlipleofvirtualwork.integral:nonlinearmomentlesscontinuum}

\nopagebreak\vspace{-0.2em}\begin{equation}\label{princlipleofvirtualwork.local:nonlinearmomentlesscontinuum}
\left(^{\mathstrut} \hspace{-0.2ex} \boldnabla \dotp \cauchystress \hspace{.1ex} + \hspace{-0.1ex} \rho \bm{f} \right) \hspace{-0.4ex} \dotp \variation{\bm{R}} \hspace{.2ex}
+ \cauchystress \dotdotp \hspace{-0.12ex} \boldnabla \hspace{.1ex} \variation{\bm{R}}^{\hspace{.1ex}\T} \hspace{-0.32ex}
+ \hspace{.1ex} \variation{\internalwork}
= \hspace{.1ex} 0 \hspace{.1ex}.
\end{equation}

\vspace{-0.25em}
\en{When}\ru{Когда} \en{a~body}\ru{тело} \en{virtually moves}\ru{виртуально движется} \en{as a~rigid whole}\ru{как жёсткое целое}, \en{the~work of~internal forces nullifies}\ru{работа внутренних сил обнуляется}

\nopagebreak\vspace{-0.25em}\begin{equation*}
%\label{princlipleofvirtualwork.worknullifies:nonlinearmomentlesscontinuum}
\begin{array}{c}
\variation{\bm{R}} = \constvarvector{\hspace{-0.1ex}\bm{\rho}} \hspace{.1ex} + \constvarvector{o} \hspace{-0.33ex} \times \hspace{-0.33ex} \bm{R}
%%\hspace{.1ex} , \:
%%\constvarvector{\hspace{-0.1ex}\bm{\rho}} = \boldconstant
%%\hspace{.1ex} , \:
%%\constvarvector{o} = \boldconstant
\hspace{.66ex}\Rightarrow\hspace{.4ex}
\variation{\internalwork} \hspace{-0.2ex} = 0
\hspace{.1ex} ,
\\[.2em]
%
\bigl( \hspace{.1ex} \boldnabla \dotp \cauchystress \hspace{.1ex} + \hspace{-0.1ex} \rho \bm{f} \hspace{.16ex} \bigr)
\hspace{-0.25ex} \dotp \hspace{-0.25ex}
\bigl(  \constvarvector{\hspace{-0.1ex}\bm{\rho}} \hspace{.1ex} + \constvarvector{o} \hspace{-0.33ex} \times \hspace{-0.33ex} \bm{R} \hspace{.2ex} \bigr) \hspace{-0.16ex}
+ \cauchystress^{\hspace{.16ex}\T} \hspace{-0.5ex}
\dotdotp
\hspace{-0.16ex} \boldnabla \hspace{.1ex} \bigl(
\constvarvector{\hspace{-0.1ex}\bm{\rho}} \hspace{.1ex} +
\constvarvector{o} \hspace{-0.33ex} \times \hspace{-0.33ex} \bm{R} \hspace{.2ex} \bigr) \hspace{-0.25ex}
= 0 \hspace{.1ex} ,
\\[.2em]
%
\constvarvector{\hspace{-0.1ex}\bm{\rho}} = \boldconstant
\hspace{.4ex} \Rightarrow \hspace{.2ex}
\boldnabla \hspace{.1ex} \constvarvector{\hspace{-0.1ex}\bm{\rho}} =  \hspace{-0.12ex} {^2\bm{0}} \hspace{.1ex} ,
\;\:
\constvarvector{o} = \boldconstant
\hspace{.4ex} \Rightarrow \hspace{.2ex}
\boldnabla \hspace{.1ex} \constvarvector{o} = \hspace{-0.12ex} {^2\bm{0}} \hspace{.1ex} ,
\\[.2em]
%
\boldnabla \hspace{.1ex} \bigl(
\constvarvector{\hspace{-0.1ex}\bm{\rho}} \hspace{.1ex} +
\constvarvector{o} \hspace{-0.33ex} \times \hspace{-0.33ex} \bm{R} \hspace{.2ex} \bigr) \hspace{-0.25ex}
= \hspace{-0.1ex} \boldnabla \hspace{.1ex} \bigl(
\constvarvector{o} \hspace{-0.33ex} \times \hspace{-0.33ex} \bm{R} \hspace{.2ex} \bigr) \hspace{-0.25ex}
= \boldnabla \hspace{.1ex} \constvarvector{o} \hspace{-0.33ex} \times \hspace{-0.33ex} \bm{R}
\hspace{.1ex} - \hspace{-0.2ex}
\boldnabla \hspace{-0.2ex} \bm{R} \times \hspace{-0.2ex} \constvarvector{o} =
\\[.1em] %
\hspace*{\fill}
= - \boldnabla \hspace{-0.2ex} \bm{R} \hspace{-0.1ex} \times \hspace{-0.2ex} \constvarvector{o}
= - \hspace{.1ex} \bm{E} \hspace{-0.16ex} \times \hspace{-0.2ex} \constvarvector{o}
%%= - \hspace{.2ex} \constvarvector{o} \hspace{-0.3ex} \times \hspace{-0.33ex} \bm{E}
\\[.2em]
%
\cdots
%%\hspace{.1ex} .
\end{array}
\end{equation*}

\vspace{-0.2em}
\en{Assuming}\ru{Полагая} ${\constvarvector{o} \hspace{-0.1ex} = \bm{0}}$ (\en{just translation}\ru{лишь трансляция}) ${\Rightarrow}$~${\hspace{-0.1ex} \boldnabla \hspace{.1ex} \variation{\bm{R}} = \hspace{-0.2ex} \boldnabla \hspace{.1ex} \constvarvector{\hspace{-0.1ex}\bm{\rho}} =  \hspace{-0.12ex} {^2\bm{0}}}$,
\en{it turns into the~balance of~forces~(of~momentum)}\ru{оно превращается в~баланс сил~(импульса)}

\nopagebreak\vspace{-0.2em}\begin{equation*}
\boldnabla \dotp \cauchystress \hspace{.1ex} + \hspace{-0.1ex} \rho \bm{f} \hspace{-0.1ex} = \bm{0}
\hspace{.1ex} .
\end{equation*}

\vspace{-0.1em}
\en{If}\ru{Если} ${\variation{\bm{R}} = \constvarvector{o} \hspace{-0.33ex} \times \hspace{-0.33ex} \bm{R}}$ (\en{just rotation}\ru{лишь поворот}) \en{with}\ru{с}~${\constvarvector{o} \hspace{-0.1ex} = \boldconstant}$, \en{then}\ru{то}

\nopagebreak\vspace{-0.1em}\begin{equation*}
\begin{array}{r@{\hspace{.8ex}}l}
\eqrefwithchapdotpara{gradientofcrossproductoftwovectors}{chapter:elementsoftensorcalculus}{para:differentiationoftensorfields}
\,\Rightarrow &
\boldnabla \hspace{.1ex} \variation{\bm{R}}
= \boldnabla \hspace{.1ex} \constvarvector{o} \hspace{-0.33ex} \times \hspace{-0.33ex} \bm{R}
\hspace{.1ex} - \hspace{-0.2ex}
\boldnabla \hspace{-0.2ex} \bm{R} \times \hspace{-0.2ex} \constvarvector{o}
= - \hspace{.1ex} \bm{E} \hspace{-0.16ex} \times \hspace{-0.2ex} \constvarvector{o}
%%= - \hspace{.2ex} \constvarvector{o} \hspace{-0.3ex} \times \hspace{-0.33ex} \bm{E}
\hspace{.1ex} ,
\\[.25em]
%
& \boldnabla \hspace{.1ex} \variation{\bm{R}}^{\hspace{.16ex}\T} \hspace{-0.32ex}
= \bm{E} \hspace{-0.16ex} \times \hspace{-0.2ex} \constvarvector{o}
%%= \constvarvector{o} \hspace{-0.3ex} \times \hspace{-0.33ex} \bm{E}
\end{array}
\end{equation*}

\noindent With

\nopagebreak\vspace{-0.8em}\begin{equation*}
\begin{array}{c}
\eqrefwithchapdotpara{pseudovectorinvariant}{chapter:elementsoftensorcalculus}{para:tensors.symmetric+skewsymmetric} \:\Rightarrow\,
\cauchystress_{\hspace{-0.1ex}\Xcompanion} \hspace{-0.1ex} = - \hspace{.1ex} \cauchystress \hspace{.1ex} \dotdotp \levicivitatensor
\hspace{.1ex} ,
\\[.2em]
%
\cauchystress \dotdotp \hspace{-0.32ex} \left( \hspace{.1ex} \bm{E} \hspace{-0.16ex} \times \hspace{-0.2ex} \constvarvector{o} \hspace{.1ex} \right) \hspace{-0.1ex}
= \cauchystress \dotdotp \hspace{-0.4ex} \left( \hspace{-0.1ex} - \hspace{.2ex} \levicivitatensor \dotp \constvarvector{o} \hspace{.1ex} \right) \hspace{-0.1ex}
= \left( \hspace{-0.1ex} - \hspace{.1ex} \cauchystress \dotdotp \levicivitatensor \hspace{.1ex} \right) \hspace{-0.3ex} \dotp \constvarvector{o} \hspace{.1ex}
= \cauchystress_{\hspace{-0.1ex}\Xcompanion} \dotp \hspace{.1ex} \constvarvector{o}
\end{array}
\end{equation*}

...

\en{In an~elastic continuum,}\ru{В~упругой среде} \en{internal forces are potential}\ru{внутренние силы потенциальны}

\nopagebreak\vspace{-0.2em}\begin{equation*}
\variation{\internalwork} = - \rho \hspace{.2ex} \variation{\widetilde{\Pi}}
%%\hspace{.1ex} .
\end{equation*}

...

\begin{equation}
\cauchystress \dotdotp \hspace{-0.12ex} \boldnabla \hspace{.1ex} \variation{\bm{R}}^{\hspace{.25ex}\mathsf{S}} \hspace{-0.25ex}
= \hspace{-0.2ex} - \hspace{.2ex} \variation{\internalwork}
= \rho \hspace{.2ex} \variation{\widetilde{\Pi}}
\end{equation}

...

\begin{otherlanguage}{russian}

Вид потенциала ${\widetilde{\Pi}}$ \en{per mass unit}\ru{на~единицу \hbox{массы}} пока неизвестен, но очевидно что ${\widetilde{\Pi}}$ определяется деформацией.

\en{With}\ru{С}~\en{the~balance of~mass}\ru{балансом массы} ${\rho \hspace{.2ex} J \hspace{-0.1ex} = \mathcircabove{\rho} \hspace{.5ex} \Leftrightarrow \hspace{.1ex} \rho = \hspace{-0.1ex} J^{\expminusone} \mathcircabove{\rho}}$
(${J \hspace{-0.1ex} \equiv \operatorname{det} \bm{F}\hspace{-0.12ex}}$\:--- якобиан, определитель градиента движения),
потенциал на~единицу объёма в~недеформированной конфигурации~$\smash{\mathcircabove{\Pi}}$ имеет вид

\nopagebreak\vspace{-0.2em}\begin{equation}
\begin{array}{c}
\mathcircabove{\Pi} \equiv \mathcircabove{\rho} \hspace{.4ex} \widetilde{\Pi}
\hspace{.4ex} \Rightarrow \hspace{.2ex}
\variation{\mathcircabove{\Pi}} = \mathcircabove{\rho} \hspace{.25ex} \variation{\widetilde{\Pi}}
\hspace{.1ex} ,
\\[.1em]
%
\rho \hspace{.2ex} \variation{\widetilde{\Pi}} = \hspace{-0.1ex} J^{\expminusone} \hspace{.1ex} \variation{\mathcircabove{\Pi}}
\hspace{.1ex} .
\end{array}
\end{equation}

\vspace{-0.25em}
Полным аналогом~(...) является равенство

...



\end{otherlanguage}

\en{\section{Constitutive relations of elasticity}}

\ru{\section{Определяющие отношения упругости}}

\en{Fundamental relation of~elasticity}\ru{Фундаментальное соотношение упругости}~\eqref{fundamentalrelationofelasticity}

...

{\small

\[ \Pi({\boldsymbol{\bm{C}}}) = \displaystyle \integral_{\raisemath{-0.25em}{\hspace{-0.1ex}\scalebox{0.85}{$0$}}}^{\raisemath{.15em}{\bm{C}}} \hspace{-0.25ex} \cauchystress \hspace{.1ex} \dotdotp d \hspace{.1ex} \boldsymbol{\bm{C}} \]

If the strain energy density is path independent, then it acts as a~potential for stress, that is
\[ \displaystyle \cauchystress = {\frac{\partial \Pi(\bm{C})}{\partial \bm{C}}} \]

For adiabatic processes, ${\Pi}$ is equal to the change in internal energy per unit volume.

For isothermal processes, ${\Pi}$ is equal to the Helmholtz free energy per unit volume.

The natural configuration of a~body is defined as the configuration in which the body is in stable thermal equilibrium with no external loads and zero stress and strain.

When we apply energy methods in elasticity, we implicitly assume that a~body returns to its natural configuration after loads are removed. This implies that the Gibbs’ condition is satisfied:
\[ \Pi({\boldsymbol{\bm{C}}}) \geq 0~~{\text{with}}~~\Pi({\boldsymbol{\bm{C}}}) = 0~~{\text{iff}}~~{\boldsymbol{\bm{C}}} = 0 \]

\par}

...

\begin{otherlanguage}{russian}

\noindent Начальная конфигурация считается естественной (natural configuration)\:--- недеформированной ненапряжённой: ${\bm{C} = \hspace{-0.1ex} {^2\bm{0}} \hspace{.4ex} \Leftrightarrow \hspace{.2ex} \cauchystress = \hspace{-0.2ex} {^2\bm{0}}}$, поэтому в~$\Pi$ нет линейных членов.

Тензор жёсткости~$\stiffnesstensor$

...

Rubber\hbox{-}like material (elastomer)

Материалу типа резины~(эластомеру) характерны больш\'{и}е деформации, и~функция~${\Pi\hspace{.12ex}(\mathrm{I}, \mathrm{II}, \mathrm{III})}$ для~него бывает весьма сложной\footnote{\bibauthor{Harold Alexander}. \href{https://kundoc.com/pdf-a-constitutive-relation-for-rubber-like-materials-.html}{A~constitutive relation for rubber-like ma\-te\-ri\-als~// International Journal of~Engineering Science, volume~6 (September 1968), pages 549\hbox{--}563.}}\hbox{\hspace{-0.5ex}.}
% https://www.researchgate.net/publication/232329906_A_constitutive_relation_for_rubber-like_materials

При~больш\'{и}х деформациях исчезают преимущества использования~$\bm{u}$ и~$\bm{C}$\:--- проще остаться с~радиусом\hbox{-}вектором $\bm{R}$ ...

...



\end{otherlanguage}

\newpage

\en{\section{Piola\hbox{--}Kirchhoff stress tensors and other measures of~stress}}

\ru{\section{Тензоры напряжения Piola\hbox{--}Kirchhoff’а и~другие меры напряжения}}

\label{para:piolakirchhoffstresstensor}

\begin{otherlanguage}{russian}

Соотношение Нансона ${\mathboldN dO = J \hspace{.1ex} \bm{n} do \dotp \bm{F}^{\hspace{.1ex}\expminusone} \hspace{-0.1ex}}$ между векторами бесконечно малой площ\'{а}дки в~отсчётной~(${\bm{n} do}$) и~в~актуальной~(${\mathboldN dO}$) конфигурациях%
\footnote{\en{Like before}\ru{По\hbox{-}прежнему},
${\bm{F} \hspace{-0.1ex}
= \scalebox{0.8}{$ \displaystyle \frac{\raisemath{-0.2em}{\partial \hspace{-0.15ex} \bm{R}}}{\partial \bm{r}} $}
= \hspace{-0.1ex} \bm{R}_i \bm{r}^{\hspace{-0.05ex}i} \hspace{-0.2ex}
= \hspace{-0.2ex} \boldnablacircled \hspace{-0.16ex} \bm{R}^{\hspace{.1ex}\T} \hspace{-0.2ex}}$\ru{\:---}\en{ is} \en{motion gradient}\ru{градиент движения}, ${J \hspace{-0.1ex} \equiv \operatorname{det} \bm{F}\hspace{-0.12ex}}$\en{ is}\ru{\:---} \en{Jacobian}\ru{якобиан} (\en{Jacobian determinant}\ru{определитель Якоби}).}

\nopagebreak\vspace{-0.12em}\begin{equation*}
\eqref{areachange:nansonformula}
\:\Rightarrow\,
\mathboldN dO \dotp \cauchystress
= J \hspace{.1ex} \bm{n} do \dotp \bm{F}^{\hspace{.1ex}\expminusone} \hspace{-0.2ex} \dotp \cauchystress
\:\Rightarrow\,
\mathboldN \dotp \cauchystress \hspace{.25ex} dO
= \hspace{.1ex} \bm{n} \dotp J \bm{F}^{\hspace{.1ex}\expminusone} \hspace{-0.2ex} \dotp \cauchystress \hspace{.2ex} do
\end{equation*}

\vspace{-0.2em} \noindent \en{gives the~dual expression of a~surface force}\ru{даёт двоякое выражение поверхностной силы}

\nopagebreak\vspace{-0.16em}\begin{equation}\label{dualexpressionofsurfaceforce}
\mathboldN \dotp \cauchystress \hspace{.25ex} dO
= \bm{n} \dotp \hspace{.12ex} \firstpiolakirchhoffstress \hspace{.1ex} do
\hspace{.1ex}, \:\:
\firstpiolakirchhoffstress \hspace{.1ex} \equiv J \bm{F}^{\hspace{.1ex}\expminusone} \hspace{-0.2ex} \dotp \cauchystress \hspace{.1ex}.
\end{equation}

\vspace{-0.2em} Тензор~${\hspace{.1ex}\firstpiolakirchhoffstress}$ называется первым~(несимметричным) тензором напряжения Piola--Kirchhoff\ru{’а}, иногда\:--- \inquotes{номинальным напряжением} (\inquotes{nominal stress}) или \inquotes{инженерным напряжением} (\inquotes{engineering stress}). Бывает и~когда какое\hbox{-}либо из этих (на)именований даётся транспонированному тензору

\nopagebreak\vspace{-0.1em}\begin{equation*}
\firstpiolakirchhoffstress^{\hspace{.1ex}\T} \hspace{-0.32ex}
= J \cauchystress^{\hspace{.16ex}\T} \hspace{-0.32ex} \dotp \bm{F}^{\hspace{.1ex}\expminusT} \hspace{-0.32ex}
= J \cauchystress \hspace{.16ex} \dotp \bm{F}^{\hspace{.1ex}\expminusT} \hspace{-0.25ex}.
\end{equation*}

Обращение~\eqref{dualexpressionofsurfaceforce}

\begin{equation*}
J^{\hspace{.12ex}\expminusone} \bm{F} \dotp \hspace{.16ex} \firstpiolakirchhoffstress = J^{\hspace{.12ex}\expminusone} \bm{F} \dotp J \bm{F}^{\hspace{.16ex}\expminusone} \hspace{-0.2ex} \dotp \cauchystress
\:\,\Rightarrow\:
\cauchystress = J^{\hspace{.12ex}\expminusone} \bm{F} \dotp \hspace{.16ex} \firstpiolakirchhoffstress
\end{equation*}

...



\nopagebreak\vspace{-0.2em}\begin{equation}
\variation{\Pi} = \hspace{.1ex} \firstpiolakirchhoffstress \hspace{-0.12ex} \dotdotp \hspace{.1ex} \variation{\hspace{.1ex} \boldnablacircled \hspace{-0.16ex} \bm{R}^{\hspace{0.12ex}\T}}
\hspace{.1ex} \Rightarrow \hspace{.32ex}
\Pi \hspace{-0.32ex}=\hspace{-0.25ex} \Pi(\boldnablacircled \hspace{-0.16ex} \bm{R}\hspace{.1ex})
\end{equation}

\vspace{-0.2em} \noindent --- этот немного неожиданный результат получился благодаря коммутативности $\variation$ и~$\smash{\hspace{-0.1ex}\boldnablacircled\hspace{.1ex}}$: ${\boldnablacircled \hspace{.1ex} \variation{\bm{R}}^{\hspace{.2ex}\T} \hspace{-0.4ex} = \variation{\hspace{.1ex} \boldnablacircled \hspace{-0.16ex} \bm{R}^{\hspace{.12ex}\T}}\hspace{-0.25ex}}$ ($\boldnabla$ \en{and}\ru{и}~$\variation$ \en{don’t commute}\ru{не~коммутируют}).

% энергетически сопряжённый с ... = energy conjugate to ...

Тензор~$\firstpiolakirchhoffstress$ оказался энергетически сопряжённым с~${\bm{F} \equiv \hspace{-0.1ex} \smash{\boldnablacircled} \hspace{-0.16ex} \bm{R}^{\hspace{.12ex}\T}}$

\nopagebreak\vspace{-0.12em}\begin{equation}
\firstpiolakirchhoffstress \hspace{-0.1ex}
= \scalebox{0.92}{$ \displaystyle \frac{\partial \hspace{.1ex} \Pi}{\raisemath{-0.4em}{\partial \hspace{.1ex} \smash{\boldnablacircled} \hspace{-0.16ex} \bm{R}^{\hspace{.12ex}\T}}} $}
= \scalebox{0.92}{$ \displaystyle \frac{\hspace{.2ex} \partial \hspace{.1ex} \Pi \hspace{.2ex}}{\raisemath{-0.25em}{\partial \bm{F}}} $} \hspace{.25ex}.
\vspace{.1em}\end{equation}

Второй~(симметричный) тензор напряжения Piola--Kirch\-hoff\ru{’а} $\secondpiolakirchhoffstress$ энергетически сопряжён с~${\bm{G} \equiv \hspace{-0.1ex} \bm{F}^{\hspace{.1ex}\T} \hspace{-0.36ex} \dotp \bm{F}}$ и~${\bm{C} \hspace{-0.1ex} \equiv \smalldisplaystyleonehalf \hspace{.1ex} (\bm{G} - \hspace{-0.12ex} \bm{E} \hspace{.1ex})}$

\nopagebreak\vspace{-0.4em}\begin{equation}
\begin{array}{c}
\variation{\Pi}(\bm{C}\hspace{.1ex}) \hspace{-0.2ex} = \secondpiolakirchhoffstress \dotdotp \variation{\hspace{.1ex} \bm{C}}
\hspace{.25ex} \Rightarrow \hspace{.32ex}
%%\Pi \hspace{-0.32ex}=\hspace{-0.25ex} \Pi(\bm{C}\hspace{.1ex}) \hspace{.1ex} ,
%%\:\:
\secondpiolakirchhoffstress = \scalebox{0.92}{$ \displaystyle \frac{\partial \hspace{.1ex} \Pi}{\raisemath{-0.1em}{\partial \hspace{.1ex} \bm{C}}} $} \hspace{.24ex} , \\[.5em]
%
d\bm{G} \hspace{-0.12ex} = 2 \hspace{.2ex} d \bm{C}
\hspace{.25ex} \Rightarrow \hspace{.32ex}
\variation{\Pi}(\bm{G}\hspace{.1ex}) \hspace{-0.2ex} = \hspace{.1ex} \smalldisplaystyleonehalf \hspace{.2ex} \secondpiolakirchhoffstress \dotdotp \variation{\hspace{.1ex} \bm{G}} ,
\:\,
\secondpiolakirchhoffstress \hspace{-0.1ex} = 2 \hspace{.25ex} \scalebox{0.92}{$ \displaystyle \frac{\partial \hspace{.1ex} \Pi}{\raisemath{-0.1em}{\partial \hspace{.1ex} \bm{G}}} $} \hspace{.25ex}.
\end{array}
\end{equation}

Связь между первым и~вторым тензорами

\nopagebreak\vspace{-0.12em}\begin{equation*}
\secondpiolakirchhoffstress \hspace{-0.1ex}
= \firstpiolakirchhoffstress \hspace{-0.1ex} \dotp \bm{F}^{\hspace{.1ex}\expminusT} \hspace{-0.4ex}
= \bm{F}^{\hspace{.1ex}\expminusone} \hspace{-0.3ex} \dotp \hspace{.1ex} \firstpiolakirchhoffstress^{\hspace{.1ex}\T}
%
\hspace{.25ex} \Leftrightarrow \hspace{.75ex}
%
\firstpiolakirchhoffstress = \secondpiolakirchhoffstress \dotp \bm{F}^{\hspace{.1ex}\T}
\hspace{-0.4ex} , \:\:
\firstpiolakirchhoffstress^{\hspace{.1ex}\T} \hspace{-0.4ex} = \bm{F} \dotp \secondpiolakirchhoffstress
\end{equation*}

\vspace{-0.2em} \noindent и между тензором~$\secondpiolakirchhoffstress$ и~тензором напряжения Cauchy~$\cauchystress$

\nopagebreak\vspace{-0.12em}\begin{equation*}
\secondpiolakirchhoffstress = J \bm{F}^{\hspace{.1ex}\expminusone} \hspace{-0.2ex} \dotp \cauchystress \hspace{.16ex} \dotp \bm{F}^{\hspace{.1ex}\expminusT}
\hspace{.2ex} \Leftrightarrow \hspace{.5ex}
J^{\hspace{.12ex}\expminusone} \bm{F} \hspace{-0.1ex} \dotp \secondpiolakirchhoffstress \dotp \bm{F}^{\hspace{.1ex}\T} \hspace{-0.32ex}
= \cauchystress \hspace{.1ex}.
\end{equation*}

...

\begin{equation*}
\firstpiolakirchhoffstress \hspace{-0.1ex}
= \scalebox{0.92}{$ \displaystyle \frac{\partial \hspace{.1ex} \Pi}{\raisemath{-0.1em}{\partial \hspace{.1ex} \bm{C}}} $} \dotp \hspace{-0.1ex} \bm{F}^{\hspace{.1ex}\T} \hspace{-0.5ex}
= 2 \hspace{.25ex} \scalebox{0.92}{$ \displaystyle \frac{\partial \hspace{.1ex} \Pi}{\raisemath{-0.1em}{\partial \hspace{.1ex} \bm{G}}} $} \dotp \hspace{-0.1ex} \bm{F}^{\hspace{.1ex}\T}
\end{equation*}

\begin{equation*}
\variation{\secondpiolakirchhoffstress}
= \scalebox{0.92}{$ \displaystyle \frac{\partial \hspace{.1ex} \bm{S}}{\raisemath{-0.1em}{\partial \hspace{.1ex} \bm{C}}} $} \dotdotp \variation{\hspace{.12ex}\bm{C}}
= \scalebox{0.92}{$ \displaystyle \frac{\partial^2 \hspace{0.1ex} \Pi}{\raisemath{-0.1em}{\partial \hspace{0.1ex} \bm{C} \hspace{0.1ex} \partial \hspace{0.1ex} \bm{C}}} $} \dotdotp \variation{\hspace{.12ex}\bm{C}}
\end{equation*}

\begin{equation*}
\variation{\hspace{.1ex}\firstpiolakirchhoffstress} \hspace{-0.2ex} =
\variation{\secondpiolakirchhoffstress} \dotp \bm{F}^{\hspace{.1ex}\T} \hspace{-0.4ex} + \hspace{.1ex}
\secondpiolakirchhoffstress \dotp \variation{\bm{F}}^{\hspace{.1ex}\T}
\end{equation*}

...

{\small
The quantity ${\bm{\kappa} = J \cauchystress}$ is called the \emph{Kirchhoff stress tensor} and is used widely in numerical algorithms in metal plasticity (where there’s no change in volume during plastic deformation). Another name for it is \emph{weighted Cauchy stress tensor}.
\par}

...

\end{otherlanguage}

\en{Here’s balance of~forces~(of~momentum) with tensor~${\hspace{.1ex}\firstpiolakirchhoffstress}$ for any undeformed volume~$\mathcircabove{\mathcal{V}}$}

\ru{Вот баланс сил~(импульса) с~тензором~${\hspace{.1ex}\firstpiolakirchhoffstress}$ для любого недеформированного объёма~$\mathcircabove{\mathcal{V}}$}

\nopagebreak\ru{\vspace{-0.12em}}\begin{equation*}
\scalebox{0.96}[0.94]{$ \displaystyle \integral\displaylimits_{\mathcal{V}} \hspace{-0.4ex} \rho \bm{f} \hspace{.1ex} d\mathcal{V} $} + \hspace{-0.2ex}
\scalebox{0.96}[0.94]{$ \displaystyle \integral\displaylimits_{\mathclap{O(\boundary \mathcal{V})}} \hspace{-0.5ex} \mathboldN \hspace{-0.12ex} \dotp \hspace{-0.1ex} \cauchystress \hspace{.2ex} dO $}
= \hspace{-0.2ex}
\scalebox{0.96}[0.94]{$ \displaystyle \integral\displaylimits_{\mathcircabove{\mathcal{V}}} \hspace{-0.4ex} \mathcircabove{\rho} \bm{f} \hspace{.12ex} d \mathcircabove{\mathcal{V}} $} + \hspace{-0.2ex}
\scalebox{0.96}[0.94]{$ \displaystyle \integral\displaylimits_{\mathclap{o \hspace{.1ex} (\boundary \smash{\mathcircabove{\mathcal{V}}})}} \hspace{-0.5ex} \bm{n} \hspace{-0.12ex} \dotp \firstpiolakirchhoffstress \hspace{.2ex} do $}
= \hspace{-0.2ex}
\scalebox{0.96}[0.94]{$ \displaystyle \integral\displaylimits_{\mathcircabove{\mathcal{V}}} \hspace{-0.5ex}
\left(^{\mathstrut} \hspace{-0.1ex} \mathcircabove{\rho} \bm{f} \hspace{-0.12ex} + \hspace{-0.4ex} \boldnablacircled \dotp \firstpiolakirchhoffstress \right) \hspace{-0.4ex} d \mathcircabove{\mathcal{V}} $} \hspace{-0.25ex}
= \bm{0}
\vspace{-0.25em}\end{equation*}

\noindent \en{and its local (differential) variant}\ru{и~его локальный (дифференциальный) вариант}

\nopagebreak\vspace{-0.25em}\begin{equation}\label{balanceoftranslationalmomentum.local.withfirstpiolakirchhoffstress}
\boldnablacircled \dotp \hspace{.12ex} \firstpiolakirchhoffstress + \hspace{.1ex} \mathcircabove{\rho} \bm{f} = \hspace{.1ex} \bm{0} \hspace{.12ex}.
\end{equation}

\en{Advantages of this equation in comparison with~\eqref{balanceoftranslationalmomentum.local} are: here figures the known mass density~${\hspace{-0.1ex}\mathcircabove{\rho}}$ of undeformed volume~${\hspace{-0.1ex}\mathcircabove{\mathcal{V}}\hspace{-0.25ex}}$, and the operator~${\hspace{-0.16ex}\boldnablacircled \equiv \bm{r}^i \partial_i}$ is defined through known vectors~${\bm{r}^i\hspace{-0.25ex}}$. Appearance of~${\hspace{.16ex}\firstpiolakirchhoffstress}$ reflects specific property of an~elastic solid body\:--- \inquotes{to~preserve} its reference configuration. In~fluid mechanics, for example, tensor~${\hspace{.16ex}\firstpiolakirchhoffstress}$ is unlikely useful.}

\ru{Преимущества этого уравнения в~сравнении с~\eqref{balanceoftranslationalmomentum.local}: здесь фигурирует известная плотность~${\hspace{-0.1ex}\mathcircabove{\rho}}$ массы недеформированного объёма~${\hspace{-0.1ex}\mathcircabove{\mathcal{V}}\hspace{-0.25ex}}$, и~оператор~${\hspace{-0.16ex}\boldnablacircled \equiv \bm{r}^i \partial_i}$ определяется через известные векторы~${\bm{r}^i\hspace{-0.25ex}}$. Появление~${\hspace{.16ex}\firstpiolakirchhoffstress}$ отражает специфическое свойство упругого твёрдого тела\:--- \inquotes{сохранять} свою отсчётную конфигурацию. В~механике жидкости, к~примеру, тензор~${\hspace{.16ex}\firstpiolakirchhoffstress}$ едва~ли полезен.}

\en{Principle of virtual work}\ru{Принцип виртуальной работы} \en{for an~arbitrary volume}\ru{для произвольного объёма}~$\mathcircabove{\mathcal{V}}$ \en{of~the~elastic}\ru{упругой} (${\variation{\internalwork} = - \hspace{.2ex} \variation{\Pi}}$) \en{continuum}\ru{среды}:

\nopagebreak\vspace{-0.16em}\ru{\vspace{-0.2em}}\begin{equation*}
\begin{array}{c}
\scalebox{0.96}[0.94]{$ \displaystyle \integral\displaylimits_{\mathcircabove{\mathcal{V}}} \hspace{-0.5ex}
\left(^{\mathstrut} \hspace{-0.1ex} \mathcircabove{\rho} \bm{f} \dotp \variation{\bm{R}} - \variation{\Pi} \right) \hspace{-0.4ex} d \mathcircabove{\mathcal{V}} $}
\hspace{-0.1ex} + \hspace{-0.25ex}
\scalebox{0.96}[0.94]{$ \displaystyle \integral\displaylimits_{\mathclap{o \hspace{.1ex} (\boundary \smash{\mathcircabove{\mathcal{V}}})}} \hspace{-0.4ex} \bm{n} \dotp \firstpiolakirchhoffstress \hspace{-0.16ex} \dotp \variation{\bm{R}} \hspace{.4ex} do $}
= 0 \hspace{.1ex} , \\[.1em]
%
\boldnablacircled \hspace{-0.1ex} \dotp \hspace{-0.1ex} \left( \hspace{.12ex} \firstpiolakirchhoffstress \hspace{-0.12ex} \dotp \variation{\bm{R}} \hspace{.12ex} \right)
= \boldnablacircled \hspace{-0.1ex} \dotp \firstpiolakirchhoffstress \hspace{-0.16ex} \dotp \variation{\bm{R}} \hspace{.1ex}
+ \hspace{.1ex} \firstpiolakirchhoffstress^{\hspace{.1ex}\T} \hspace{-0.5ex} \dotdotp \hspace{-0.2ex} \boldnablacircled \hspace{.1ex} \variation{\bm{R}} \hspace{.1ex},
\:\,
\firstpiolakirchhoffstress^{\hspace{.1ex}\T} \hspace{-0.5ex} \dotdotp \hspace{-0.2ex} \boldnablacircled \hspace{.1ex} \variation{\bm{R}} \hspace{.1ex}
= \firstpiolakirchhoffstress \hspace{-0.1ex} \dotdotp \hspace{-0.2ex} \boldnablacircled \hspace{.1ex} \variation{\bm{R}}^{\hspace{0.12ex}\T} \\[.25em]
%
\variation{\Pi}
= \hspace{-0.16ex} \left(^{\mathstrut} \hspace{-0.1ex} \mathcircabove{\rho} \bm{f} + \hspace{-0.25ex} \boldnablacircled \dotp \firstpiolakirchhoffstress \right) \hspace{-0.4ex} \dotp \variation{\bm{R}}
\hspace{.1ex}
+ \firstpiolakirchhoffstress \hspace{-0.1ex} \dotdotp \hspace{-0.2ex} \boldnablacircled \hspace{.1ex} \variation{\bm{R}}^{\hspace{0.12ex}\T}
\end{array}
\end{equation*}



....


First one is non-symmetric, it connects forces in deformed stressed configuration to underfomed geometry+mass (initially known volumes, areas, densities), and it is energetically conjugate to the motion gradient (commonly mistakenly called \inquotes{deformation gradient}, despite comprising of rigid rotations). First (sometimes its transpose) is also known as \inquotes{nominal stress} and \inquotes{engineering stress}.

Second one is symmetric, it connects loads in initial undeformed configuration to initial mass+geometry, and it’s conjugate to the right Cauchy\hbox{--}Green deformation tensor (and thus to the Cauchy\hbox{--}Green\hbox{--}Venant measure of deformation).

The first is simplier when you use just motion gradient and is more universal, but the second is simplier when you prefer right Cauchy\hbox{--}Green deformation and its offsprings.

There’s also popular Cauchy stress, which relates forces in deformed configuration to deformed geometry+mass.

\inquotes{energetically conjugate} means that their product is energy, here: elastic (potential) energy per unit of volume


...


{\small
In the case of finite deformations, the Piola\hbox{--}Kirchhoff stress tensors express the stress relative to the reference configuration. This is in contrast to the Cauchy stress tensor which expresses the stress relative to the present configuration. For infinitesimal deformations and rotations, the Cauchy and Piola\hbox{--}Kirchhoff tensors are identical.

Whereas the Cauchy stress tensor~${\cauchystress}$ relates stresses in the current configuration, the motion gradient and strain tensors are described by relating the motion to the reference configuration; thus not all tensors describing the material are in either the reference or current configuration. Describing the stress, strain and deformation either in the reference or current configuration would make it easier to define constitutive models. For example, the Cauchy stress tensor is variant to a pure rotation, while the deformation strain tensor is invariant; thus creating problems in defining a constitutive model that relates a varying tensor, in terms of an invariant one during pure rotation; as by definition constitutive models have to be invariant to pure rotations.

\subsection*{1st Piola\hbox{--}Kirchhoff stress tensor}

The \emph{1st~Piola\hbox{--}Kirchhoff stress tensor} is one possible solution to this problem. I\kern-0.12ext defines a family of tensors, which describe the configuration of the body in either the current or the reference configuration.

The 1st Piola\hbox{--}Kirchhoff stress tensor~$\bm{T}$ relates forces in the present~(\inquotes{spatial}) configuration with areas in the reference~(\inquotes{material}) configuration
\[
\bm{T} = J \, \cauchystress \dotp \bm{F}^{\expminusT}
\]
where~$\bm{F}$ is the motion gradient and~${J \equiv \operatorname{det} \bm{F}}$ is the Jacobian determinant.

In terms of components in an orthonormal basis, the first Piola\hbox{--}Kirchhoff stress is given by
\[
T_{iL} = J \, \tau_{ik}~F_{Lk}^{-1} = J \, \tau_{ik} \, \frac{\partial X_{L}}{\partial x_{k}}
\]

Because it relates different coordinate systems, the 1st~Piola\hbox{--}Kirchhoff stress is a two-point tensor. In common, it’s not symmetric. The 1st~Piola\hbox{--}Kirchhoff stress is the 3D generalization of the 1D concept of engineering stress.

If the material rotates without a change in stress (rigid rotation), the components of the 1st Piola\hbox{--}Kirchhoff stress tensor will vary with material orientation.

The 1st Piola\hbox{--}Kirchhoff stress is energy conjugate to the motion gradient.

\subsection*{2nd Piola\hbox{--}Kirchhoff stress tensor}

Whereas the 1st~Piola\hbox{--}Kirchhoff stress relates forces in the current configuration to areas in the reference configuration, the 2nd~Piola\hbox{--}Kirchhoff stress tensor~$\bm{S}$ relates forces in the reference configuration to areas in the reference configuration. The force in the reference configuration is obtained via a mapping that preserves the relative relationship between the force direction and the area normal in the reference configuration.
\[
\bm{S} = J \, \bm{F}^{\expminusone} \dotp \cauchystress \dotp \bm{F}^{\expminusT}
\]

In index notation using an orthonormal basis,
\[
S_{IL} = J \, F_{Ik}^{\expminusone} \, F_{Lm}^{\expminusT} \, \tau_{km} =
J \, \frac{\partial X_{I}}{\partial x_{k}} \, \frac{\partial X_{L}}{\partial x_{m}} \, \tau_{km}
\]

This tensor, a one\hbox{-}point tensor, is symmetric.

If the material rotates without a change in stress (rigid rotation), the components of the 2nd~Piola\hbox{--}Kirchhoff stress tensor remain constant, irrespective of material orientation.

The 2nd~Piola\hbox{--}Kirchhoff stress tensor is energy conjugate to the Green\hbox{--}Lagrange finite strain tensor.
\par}


...



\newpage

\en{\section{Variation of present configuration}}

\ru{\section{Варьирование текущей конфигурации}}

\label{para:variationofconfiguration}

%% ${\widetilde{\bm{R}} \equiv \hspace{-0.1ex} \variation{\bm{R}}}$
%% ${\widetilde{\bm{f}} \equiv \hspace{-0.1ex} \variation{\hspace{-0.2ex}\bm{f}}}$
%% ${\widetilde{\firstpiolakirchhoffstress} \equiv \hspace{-0.1ex} \variation{\hspace{.1ex}\firstpiolakirchhoffstress}}$
%% ${\widetilde{\bm{C}} \hspace{-0.1ex} \equiv \hspace{-0.1ex} \variation{\hspace{.12ex}\bm{C}}}$

\begin{otherlanguage}{russian}

Прежде упругая среда рассматривалась в~двух конфигурациях: отсчётной с~радиусами\hbox{-}векторами~$\bm{r}$ и~актуальной с~$\bm{R}$.
\hbox{Теперь} представим~себе малое изменение текущей конфигурации с~бесконечно\-малыми приращениями
радиуса\hbox{-}вектора~$\variation{\bm{R}}$,
вектора массовых сил~${\variation{\hspace{-0.2ex}\bm{f}}\hspace{-0.2ex}}$,
первого тензора напряжения Piola--Kirchhoff\ru{’а}~${\variation{\hspace{.1ex}\firstpiolakirchhoffstress}}$
и~тензора деформации~${\variation{\hspace{.12ex}\bm{C}}}$.
Варьируя
\eqref{balanceoftranslationalmomentum.local.withfirstpiolakirchhoffstress}, (...)\footnote{%
${\boldnabla \hspace{-0.08ex}
= \hspace{-0.2ex} \boldnabla \dotp \hspace{-0.16ex} \boldnablacircled \bm{r} \hspace{-0.08ex}
= \bm{R}^{\hspace{.1ex}i} \hspace{-0.1ex} \partial_i \hspace{-0.16ex} \dotp \bm{r}^j \hspace{-0.1ex} \partial_{\hspace{-0.08ex}j} \bm{r} \hspace{-0.08ex}
\stackrel{?}{=} \bm{R}^{\hspace{.1ex}i} \hspace{-0.1ex} \partial_i \bm{r} \hspace{-0.1ex} \dotp \bm{r}^j \hspace{-0.1ex} \partial_{\hspace{-0.08ex}j} \hspace{-0.32ex}
= \hspace{-0.2ex} \boldnabla \bm{r} \dotp \hspace{-0.16ex} \boldnablacircled \hspace{-0.1ex}
= \hspace{-0.1ex} \bm{F}^{\hspace{.1ex}\expminusT} \hspace{-0.3ex} \dotp \hspace{-0.12ex} \boldnablacircled}$ \\
%
${\boldnablacircled \hspace{-0.08ex}
= \hspace{-0.2ex} \boldnablacircled \dotp \hspace{-0.2ex} \boldnabla \bm{R}
= \bm{r}^i \partial_i \hspace{-0.16ex} \dotp \hspace{-0.12ex} \bm{R}^{\hspace{.1ex}j} \hspace{-0.1ex} \partial_{\hspace{-0.08ex}j} \hspace{-0.1ex} \bm{R}
\stackrel{?}{=} \bm{r}^i \partial_i \bm{R} \dotp \hspace{-0.24ex} \bm{R}^{\hspace{.1ex}j} \hspace{-0.1ex} \partial_{\hspace{-0.08ex}j} \hspace{-0.32ex}
= \hspace{-0.32ex} \boldnablacircled \hspace{-0.16ex} \bm{R} \hspace{.1ex} \dotp \hspace{-0.16ex} \boldnabla \hspace{-0.25ex}
= \hspace{-0.1ex} \bm{F}^{\hspace{.1ex}\T} \hspace{-0.3ex} \dotp \hspace{-0.12ex} \boldnabla}$} и~(...), получаем

\nopagebreak\vspace{-0.05em}\begin{equation}\label{variationsforcurrentconfiguration}
\begin{array}{c}
\mathcircabove{\rho} \hspace{.25ex} \variation{\hspace{-0.2ex}\bm{f}}
+ \hspace{-0.1ex} \boldnablacircled \hspace{-0.12ex} \dotp \variation{\hspace{.1ex}\firstpiolakirchhoffstress}
= \bm{0}
\hspace{.1ex} , \:\,
%
\variation{\hspace{.1ex}\firstpiolakirchhoffstress} \hspace{-0.1ex}
= \hspace{-0.2ex} \left( \hspace{.1ex} \scalebox{0.92}{$ \displaystyle \frac{\partial^2 \hspace{.1ex} \Pi}{\raisemath{-0.1em}{\partial \hspace{.1ex} \bm{C} \hspace{.1ex} \partial \hspace{.1ex} \bm{C}}} $} \dotdotp \variation{\hspace{.12ex}\bm{C}} \hspace{-0.16ex} \right) \hspace{-0.32ex} \dotp \bm{F}^{\hspace{.1ex}\T} \hspace{-0.24ex}
+ \hspace{.1ex}
\displaystyle \frac{\partial \hspace{.1ex} \Pi}{\raisemath{-0.1em}{\partial \hspace{.1ex} \bm{C}}} \dotp \variation{\bm{F}}^{\hspace{.1ex}\T}
\hspace{-0.4ex} ,
\\[1em]
%
\variation{\bm{F}}^{\hspace{.1ex}\T} \hspace{-0.5ex}
= \variation{\hspace{.1ex} \boldnablacircled \hspace{-0.16ex} \bm{R}}
= \hspace{-0.2ex} \boldnablacircled \hspace{.1ex} \variation{\bm{R}}
= \hspace{-0.12ex} \bm{F}^{\hspace{.1ex}\T} \hspace{-0.25ex} \dotp \boldnabla \hspace{.1ex} \variation{\bm{R}} \hspace{.16ex},
\:\,
\variation{\bm{F}} \hspace{-0.2ex} = \variation{\hspace{.1ex} \boldnablacircled \hspace{-0.16ex} \bm{R}^{\hspace{.12ex}\T}} \hspace{-0.4ex}
= \hspace{-0.2ex} \boldnabla \hspace{.1ex} \variation{\bm{R}}^{\hspace{.12ex}\T} \hspace{-0.25ex} \dotp \bm{F}
\hspace{-0.1ex} ,
\\[.5em]
%
\variation{\hspace{.12ex}\bm{C}} = \bm{F}^{\hspace{.1ex}\T} \hspace{-0.32ex} \dotp \hspace{.1ex} \varbivalent{\mathboldepsilon} \dotp \bm{F} ,
\:\,
\varbivalent{\mathboldepsilon} \equiv \boldnabla \hspace{.1ex} \variation{\bm{R}}^{\hspace{.25ex}\mathsf{S}}
\hspace{-0.15ex} .
\end{array}
\end{equation}

...

\begin{equation*}\begin{array}{c}
\eqref{areachange:nansonformula}
\:\Rightarrow\:
%
\bm{n} \hspace{.1ex} do = J^{\hspace{.12ex}\expminusone} \mathboldN \hspace{.1ex} dO \hspace{-0.1ex} \dotp \bm{F}
\:\Rightarrow\:
%
\bm{n} \dotp \variation{\firstpiolakirchhoffstress} \hspace{.1ex} do
= J^{\hspace{.12ex}\expminusone} \mathboldN \hspace{-0.1ex} \dotp \bm{F} \hspace{-0.1ex} \dotp \variation{\firstpiolakirchhoffstress} \hspace{.1ex} dO
\,\Rightarrow
\\[.2em]
%
\Rightarrow\:
\bm{n} \dotp \variation{\firstpiolakirchhoffstress} \hspace{.1ex} do
= \mathboldN \dotp \varbivalent{\hspace{-0.2ex}\cauchystress} \hspace{.25ex} dO ,
\:\:
\varbivalent{\hspace{-0.2ex}\cauchystress} \equiv J^{\hspace{.12ex}\expminusone} \bm{F} \hspace{-0.1ex} \dotp \variation{\firstpiolakirchhoffstress}
\end{array}\end{equation*}

\vspace{-0.2em} \noindent --- введённый так тензор~${\varbivalent{\hspace{-0.2ex}\cauchystress}}$ связан с~вариацией~$\variation{\hspace{.1ex}\firstpiolakirchhoffstress}$ как $\cauchystress$ связан с~$\firstpiolakirchhoffstress$ (${\cauchystress = \hspace{-0.1ex} J^{\hspace{.12ex}\expminusone} \bm{F} \dotp \hspace{.16ex} \firstpiolakirchhoffstress\hspace{.2ex}}$). Из~\eqref{variationsforcurrentconfiguration} и ...

...

... корректируя коэффициенты линейной функции~${\varbivalent{\hspace{-0.2ex}\cauchystress}\hspace{.15ex}(\boldnabla \hspace{.1ex} \variation{\bm{R}}\hspace{.1ex})}$.

\end{otherlanguage}

\en{\section{Internal constraints}}

\ru{\section{Внутренние связи}}

\label{para:internalconstraints}

\begin{otherlanguage}{russian}

До~сих~пор деформация считалась свободной, мера деформации~$\bm{C}$ могла быть любой. Однако, существуют материалы со~значительным сопротивлением некоторым видам деформации. Резина, например, изменению формы сопротивляется намного меньше, чем изменению объёма\:--- некоторые виды резины можно считать несжимаемым материалом.

Понятие геометрической связи, развитое в~общей механике ...

...

for incompressible materials ${\Pi \hspace{-0.4ex} = \hspace{-0.33ex} \Pi(\mathrm{I}, \mathrm{II})}$

Mooney\hbox{--}Rivlin model of incompressible material
\[
\Pi = c_1 \bigl( \mathrm{I} - 3\bigr) + c_2 \bigl( \mathrm{II} - 3\bigr)
\]

incompressible Treloar (neo-Hookean) material
\[
c_2 \hspace{-0.16ex} = 0
\;\;\Rightarrow\;\;
\Pi = c_1 \bigl( \mathrm{I} - 3\bigr)
\]

...


\end{otherlanguage}

\en{\section{Hollow sphere under pressure}}

\ru{\section{Полая сфера под действием давления}}

\label{para:hollowsphereunderpressure}

\begin{otherlanguage}{russian}

Решение этой относительно простой задачи описано во~многих книгах. В~отсчётной~(ненапряжённой) конфигурации имеем сферу с~внутренним радиусом~${r_0}$ и~наружным~${r_1}$. Давление равно $p_0$~внутри и~$p_1$~снаружи.

Введём удобную для этой задачи сферическую систему координат в~отсчётной конфигурации ${q^1 = \theta}$, ${q^2 = \phi}$, ${q^3 = r}$~(\figref{sphericalcoordinates}). Эти~же координаты будут и~материальными. Имеем

...



\end{otherlanguage}

\newpage

\en{\section{Stresses as Lagrange multipliers}}

\ru{\section{Напряжения как множители Лагранжа}}

\label{para:stressesAsLagrangeMultipliers}

\begin{otherlanguage}{russian}

Описанному ранее в~\pararef{para:virtualworkprinciple.elastic} использованию принципа виртуальной работы предшествовало введение тензора напряжения Cauchy через баланс сил для бесконечномалого тетраэдра~(\pararef{para:stressviatetrahedron}). Но~тут мы увидим, что сей принцип примен\'{и}м и без рассуждений с~тетра\-эдром.

Рассмотрим тело\:--- не~только~лишь упругое, с~любой виртуальной работой внутренних сил~${\variation{\internalwork}}$ \en{per mass unit}\ru{на~единицу массы}\:--- нагруженное массовыми~${\bm{f} dm}$ (для~краткости пишем~$\bm{f}$ вместо~${\bm{f} \hspace{-0.2ex} - \hspace{-0.2ex} \mathdotdotabove{\bm{R}}}$, так~что динамика присутствует) и~поверхностными~${\bm{p} \hspace{.25ex} dO}$ внешними силами.
Имеем вариационное уравнение

\nopagebreak\vspace{-0.1em}\begin{equation}\label{stressesAsLagrangeMultipliers:variations}
\integral\displaylimits_{\mathcal{V}} \hspace{-0.2ex} \rho \hspace{-0.1ex} \left(^{\mathstrut} \bm{f} \dotp \variation{\bm{R}} \hspace{.12ex} + \variation{\internalwork} \hspace{.15ex} \right) \hspace{-0.25ex} d\mathcal{V} + \integral\displaylimits_{\mathclap{O(\boundary \mathcal{V})}} \hspace{-0.2ex} \bm{p} \dotp \variation{\bm{R}} \hspace{0.4ex} dO = \hspace{0.2ex} 0 \hspace{0.1ex}.
\vspace{-0.25em}\end{equation}

Полагаем, что внутренние силы не~совершают работу при~виртуальном движении тела как целого\:--- когда от бесконечномалых виртуальных перемещений~${\variation{\bm{R}}}$ частиц тела нет деформации~$\varbivalent{\mathboldepsilon}$

\nopagebreak\vspace{-0.25em}\begin{equation}\label{stressesAsLagrangeMultipliers:zerovirtualmovements}
\varbivalent{\mathboldepsilon} \hspace{-0.1ex} = \hspace{-0.1ex} \boldnabla^{\displaystyle \mathstrut} \hspace{.1ex} \variation{\bm{R}}^{\hspace{.25ex}\mathsf{S}} \hspace{-0.2ex} = {^2\bm{0}} \hspace{.5ex} \Rightarrow \hspace{.33ex}
\variation{\internalwork} \hspace{-0.2ex} = 0
\hspace{.1ex} .
\end{equation}

\vspace{-0.1em} Отбросив~${\variation{\internalwork}}$ в~\eqref{stressesAsLagrangeMultipliers:variations} при~условии~\eqref{stressesAsLagrangeMultipliers:zerovirtualmovements}, получим вариационное уравнение со~связью. Приём с~множителями Лагранжа даёт возможность считать вариации~${\variation{\bm{R}}}$ независимыми. Поскольку в~каждой точке связь представлена симметричным тензором второй сложности, то таким~же тензором будут и~множители Лагранжа~${\hspace{-0.16ex} ^2\hspace{-0.2ex}\bm{\lambda}}$. Приходим к~уравнению

\nopagebreak\vspace{-0.1em}\begin{equation}\label{stressesAsLagrangeMultipliers:variationstoo}
\integral\displaylimits_{\mathcal{V}} \hspace{-0.32ex} \left( \hspace{.1ex} \rho \bm{f} \hspace{-0.15ex} \dotp \variation{\bm{R}} \hspace{.1ex} - \hspace{-0.1ex} {^2\hspace{-0.2ex}\bm{\lambda}} \dotdotp \hspace{-0.15ex} \boldnabla \hspace{.1ex} \variation{\bm{R}}^{\hspace{.25ex}\mathsf{S}} \right) \hspace{-0.32ex} d\mathcal{V} +
\integral\displaylimits_{\mathclap{O(\boundary \mathcal{V})}} \hspace{-0.2ex} \bm{p} \dotp \variation{\bm{R}} \hspace{.4ex} dO = \hspace{.2ex} 0
\hspace{.1ex} .
\vspace{-0.25em}\end{equation}

\vspace{-0.16em} Благодаря симметрии~${^2\hspace{-0.2ex}\bm{\lambda}}$ имеем\footnote{${%
\bm{\Lambda}^{\hspace{-0.16ex}\mathsf{S}} \hspace{-0.1ex} \dotdotp \hspace{-0.1ex} \bm{X} \hspace{-0.2ex} =
\hspace{.1ex} \bm{\Lambda}^{\hspace{-0.16ex}\mathsf{S}} \hspace{-0.1ex} \dotdotp \hspace{-0.1ex} \bm{X}^{\T} \hspace{-0.4ex} =
\hspace{.1ex} \bm{\Lambda}^{\hspace{-0.16ex}\mathsf{S}} \hspace{-0.1ex} \dotdotp \hspace{-0.1ex} \bm{X}^{\hspace{.08ex}\mathsf{S}}
\hspace{-0.25ex}}$,
\:\:
${%
\boldnabla \hspace{-0.1ex} \dotp \left( \hspace{.1ex} {\bm{B}} \hspace{-0.1ex} \dotp \bm{a} \hspace{.16ex} \right)
= \left( \hspace{.1ex} \boldnabla \dotp \hspace{-0.15ex} \bm{B} \hspace{.1ex} \right) \hspace{-0.1ex} \dotp \bm{a} \hspace{.1ex}
+ \bm{B}^{\T} \hspace{-0.3ex} \dotdotp \boldnabla \hspace{-0.15ex} \bm{a}
}$}

\nopagebreak\vspace{-0.2em}\begin{equation*}
{^2\hspace{-0.2ex}\bm{\lambda}} = \hspace{-0.2ex} {^2\hspace{-0.2ex}\bm{\lambda}}^{\hspace{-0.33ex}\T}
\: \Rightarrow \:
\hspace{.12ex}{^2\hspace{-0.2ex}\bm{\lambda}} \dotdotp \hspace{-0.15ex} \boldnabla \hspace{.1ex} \variation{\bm{R}}^{\hspace{.2ex}\mathsf{S}} \hspace{-0.2ex}
= {^2\hspace{-0.2ex}\bm{\lambda}} \dotdotp \hspace{-0.15ex} \boldnabla \hspace{.1ex} \variation{\bm{R}}^{\hspace{.15ex}\T}
\hspace{-0.33ex} ,
\end{equation*}

\nopagebreak\vspace{-0.2em}\begin{equation*}
{^2\hspace{-0.2ex}\bm{\lambda}} \dotdotp \hspace{-0.15ex} \boldnabla \hspace{.1ex} \variation{\bm{R}}^{\hspace{.2ex}\mathsf{S}} \hspace{-0.2ex} =
\boldnabla^{\mathstrut} \hspace{-0.1ex} \dotp \hspace{-0.1ex} \left( \hspace{.1ex} {^2\hspace{-0.2ex}\bm{\lambda}} \dotp \variation{\bm{R}} \hspace{.2ex} \right)
- \boldnabla \hspace{.1ex} \dotp \hspace{-0.1ex} {^2\hspace{-0.2ex}\bm{\lambda}} \dotp \variation{\bm{R}}
\hspace{.2ex} .
\end{equation*}

\vspace{-0.2em}\noindent Подставив это в~\eqref{stressesAsLagrangeMultipliers:variationstoo} и~применив теорему о~дивергенции, получаем

\nopagebreak\vspace{-0.25em}\begin{equation*}
\integral\displaylimits_{\mathcal{V}} \hspace{-0.32ex} \left( \rho \bm{f} \hspace{.12ex} +^{\mathstrut} \boldnabla \hspace{.1ex} \dotp \hspace{-0.1ex} {^2\hspace{-0.2ex}\bm{\lambda}} \right) \hspace{-0.36ex} \dotp \variation{\bm{R}} \hspace{.4ex} d\mathcal{V}
+ \integral\displaylimits_{\mathclap{O(\boundary \mathcal{V})}} \hspace{-0.32ex} \left( \bm{p} \hspace{.2ex} -^{\mathstrut} \mathboldN \hspace{.1ex} \dotp \hspace{-0.1ex} {^2\hspace{-0.2ex}\bm{\lambda}} \right) \hspace{-0.36ex} \dotp \variation{\bm{R}} \hspace{.4ex} dO
= \hspace{.2ex} 0
\hspace{.1ex} .
\vspace{-0.25em}\end{equation*}

\noindent
\en{But}\ru{Но}~$\variation{\bm{R}}$ \en{is random}\ru{случайна} \en{on a~surface}\ru{на~поверхности} \en{and}\ru{и}~\en{inside a~volume}\ru{в~объёме}, \en{thus}\ru{так что}

\nopagebreak\vspace{-0.2em}\begin{equation*}
\bm{p} \hspace{.1ex} = \mathboldN \hspace{.1ex} \dotp \hspace{-0.1ex} {^2\hspace{-0.2ex}\bm{\lambda}} \hspace{.2ex},
\:\:
\boldnabla \hspace{.1ex} \dotp \hspace{-0.1ex} {^2\hspace{-0.2ex}\bm{\lambda}} \hspace{.2ex} + \hspace{.2ex} \rho \bm{f} \hspace{-0.1ex} = \hspace{.1ex} \bm{0}
\end{equation*}

\en{\vspace{-0.25em} \noindent --- formally introduced symmetric multiplier~${\hspace{-0.16ex} ^2\hspace{-0.2ex}\bm{\lambda}}$ happened to be the Cauchy stress tensor.}

\ru{\vspace{-0.25em} \noindent --- формально введённый симметричный множитель~${\hspace{-0.16ex} ^2\hspace{-0.2ex}\bm{\lambda}}$ оказался тензором напряжения Cauchy.}

Подобное введение напряжений \en{was described}\ru{было описано} \en{in~book}\ru{в~книге}~\cite{rabotnov-mechanicsofdeformable}.
Новых результатов тут нет, но интересна сам\'{а} возможность одно\-времен\-ного вывода тех уравнений механики сплошной среды, которые раньше считались независимыми.
В~следующих главах эта техника используется для построения новых континуальных моделей.

\end{otherlanguage}

\section*{\small \wordforbibliography}

\begin{changemargin}{\parindent}{0pt}
\fontsize{10}{12}\selectfont

\begin{otherlanguage}{russian}

Глубина изложения нелинейной безмоментной упругости характерна для книг А.\,И.\:Лурье~\cite{lurie-nonlinearelasticity, lurie-theoryofelasticity}. Оригинальность как~основных идей, так~и~стиля присуща книге C.\:Truesdell’а~\cite{truesdell-firstcourse}. Много ценной информации можно найти у~К.\,Ф.\:Черн\'{ы}х~\cite{chernyh-nonlinearelasticity}. Ст\'{о}ит отметить и~книгу Л.\,М.\:Зубова~\cite{zubov}. Монография Ю.\,Н.\:Работнова~\cite{rabotnov-mechanicsofdeformable}, где напряжения представлены как множители Лагранжа, очень интересна и~своеобразна. О~применении нелинейной теории упругости в~смежных областях рассказано в~книге C.\:Teodosiu~\cite{teodosiu-crystaldefects}. Повышенным математическим уровнем отличается монография Ph.\:Ciarlet~\cite{ciarlet-mathematicalelasticity}.
\par

\end{otherlanguage}

\end{changemargin}
