\en{\section{Rotations in 3-dimensional space: rotation tensors}}

\ru{\section{Повороты в~трёхмерном пространстве: тензоры поворота}}

\label{section:rotationtensor}

\en{The relation}\ru{Соотношение}
\en{between}\ru{между}
\en{two}\ru{двумя}
\inquotes{\en{right}\ru{правыми}}
(\en{or}\ru{или} \en{two}\ru{двумя} \inquotes{\en{left}\ru{левыми}}
\en{orthonormal}\ru{ортонормальными}
\en{bases}\ru{базисами}
${\bm{e}_i}$ \en{and}\ru{и}~${\mathcircabove{\bm{e}}_i}$
\en{can be described}\ru{может быть описано}
\en{by }\en{a~two\hbox{-}index array}\ru{двухиндексным массивом}\ru{,}
\en{represented as a~matrix}\ru{представленным в~виде матрицы}~(\sectionref{section:vector}, \sectionref{section:matrices+permutations+determinants})

\nopagebreak\vspace{-0.2em}\begin{equation*}
\bm{e}_i \hspace{-0.2ex} = \bm{e}_i \dotp \hspace{.1ex} \tikzmark{beginEqualsE} \mathcircabove{\bm{e}}_j \mathcircabove{\bm{e}}_j \tikzmark{endEqualsE} \hspace{-0.1ex} = \hspace{.1ex} \cosinematrix{\hspace{-0.2ex}i\mathcircabove{j}} \, \mathcircabove{\bm{e}}_j
\hspace{.1ex} ,
\hspace{.4em}
%
\cosinematrix{\hspace{-0.2ex}i\mathcircabove{j}} \hspace{.1ex} \equiv \hspace{.1ex} \bm{e}_i \dotp \mathcircabove{\bm{e}}_j
%%\hspace{.1ex} .
\end{equation*}
\AddUnderBrace[line width=.75pt][0,-0.2ex]%
{beginEqualsE}{endEqualsE}{${\scriptstyle \UnitDyad}$}

\vspace{-0.3em}\noindent
(\inquotes{\en{a~matrix of~cosines}\ru{матрицы косинусов}}).

\en{Also}\ru{Также},
\en{a~rotation}\ru{поворот}
\en{of tensor}\ru{тензора}
\en{can be described}\ru{может быть описан}
\en{by another tensor}\ru{другим тензором},
\en{called}\ru{называемым}
\en{a~rotation tensor}\ru{тензором поворота}~$\rotationtensor$

\nopagebreak\vspace{-0.2em}\begin{equation}\label{introductionofrotationtensor}
\bm{e}_i = \bm{e}_j \hspace{.2ex} \tikzmark{beginEqualsKroneckerDelta} \mathcircabove{\bm{e}}_j \hspace{-0.2ex} \dotp \mathcircabove{\bm{e}}_i \tikzmark{endEqualsKroneckerDelta} = \rotationtensor \hspace{-0.2ex} \dotp \mathcircabove{\bm{e}}_i \hspace{.1ex}, \:\,
\rotationtensor \equiv \bm{e}_j \mathcircabove{\bm{e}}_j = \scalebox{.96}[1]{$ \bm{e}_1 \hspace{-0.1ex} \mathcircabove{\bm{e}}_1 + \bm{e}_2 \mathcircabove{\bm{e}}_2 + \bm{e}_3 \mathcircabove{\bm{e}}_3 $}
\hspace{.1ex} .
\end{equation}
\AddUnderBrace[line width=.75pt][.1ex,-0.2ex]%
{beginEqualsKroneckerDelta}{endEqualsKroneckerDelta}{${\scriptstyle \delta_{j\hspace{-0.06ex}i}}$}

\en{Components of}\ru{Компоненты}~$\rotationtensor$ \en{both }\ru{и~в~начальном}\en{in an~initial}~${\mathcircabove{\bm{e}}_i}$\ru{,} \en{and}\ru{и}~\en{in a~rotated}\ru{в~повёрнутом}~${\bm{e}_i}$ \en{bases}\ru{базисах} \en{are the same}\ru{одни и~те~же}

\nopagebreak\vspace{-0.2em}\begin{equation}\label{componentsofrotationtensor}
\begin{array}{c}
\bm{e}_{i} \dotp \rotationtensor \hspace{-0.1ex} \dotp \bm{e}_{j} =
\hspace{.2ex} \tikzmark{beginEqualsKroneckerDeltaPresent} \bm{e}_{i} \hspace{-0.1ex} \dotp \bm{e}_{k} \tikzmark{endEqualsKroneckerDeltaPresent} \hspace{.2ex} \mathcircabove{\bm{e}}_{k} \hspace{-0.1ex} \dotp \bm{e}_{j} =
\hspace{.2ex} \mathcircabove{\bm{e}}_{i} \hspace{-0.1ex} \dotp \bm{e}_{j}
\hspace{.1ex} ,
\\[1.4em]
%
\mathcircabove{\bm{e}}_{i} \dotp \rotationtensor \hspace{-0.1ex} \dotp \mathcircabove{\bm{e}}_{j} =
\hspace{.2ex} \mathcircabove{\bm{e}}_{i} \hspace{-0.1ex} \dotp \bm{e}_{k} \hspace{.2ex} \tikzmark{beginEqualsKroneckerDeltaPast} \mathcircabove{\bm{e}}_{k} \hspace{-0.1ex} \dotp \mathcircabove{\bm{e}}_{j} \tikzmark{endEqualsKroneckerDeltaPast} =
\hspace{.2ex} \mathcircabove{\bm{e}}_{i} \hspace{-0.1ex} \dotp \bm{e}_{j}
\hspace{.1ex} .
\end{array}
\end{equation}
\AddUnderBrace[line width=.75pt][.1ex, -0.2ex]%
{beginEqualsKroneckerDeltaPresent}{endEqualsKroneckerDeltaPresent}{${\scriptstyle \delta_{ik}}$}
\AddUnderBrace[line width=.75pt][0, -0.2ex]%
{beginEqualsKroneckerDeltaPast}{endEqualsKroneckerDeltaPast}{${\scriptstyle \delta_{kj}}$}

\vspace{.4em}\noindent
\en{In the~matrix notation}\ru{В~матричной записи}\en{,}
\en{these components}\ru{эти компоненты}
\en{present}\ru{представляют}
\en{the~matrix of~cosines}\ru{матрицу косинусов}~${\cosinematrix{\!j\mathcircabove{i}} = \hspace{.1ex} \mathcircabove{\bm{e}}_i \dotp \bm{e}_j}$\hspace{.1ex}:

\nopagebreak\vspace{-0.3em}\begin{equation*}
\rotationtensor = \cosinematrix{\!j\mathcircabove{i}} \hspace{.5ex} {\bm{e}}_{i} {\bm{e}}_{j} = \cosinematrix{\!j\mathcircabove{i}} \hspace{.5ex} \mathcircabove{\bm{e}}_{i} \mathcircabove{\bm{e}}_{j}
\hspace{.1ex} .
\end{equation*}

{\small
Spatial transformations in the 3-dimensional Euclidean space ${\mathbb{R}^{3}}$ are distinguished into active or alibi transformations, and passive or alias transformations.
An active transformation is a transformation which actually changes the physical position (alibi, elsewhere) of objects, which can be defined in the absence of a coordinate system; whereas a passive transformation is merely a change in the coordinate system in which the object is described (alias, other name) (change of coordinates, or change of basis). By transformation, math texts usually refer to active transformations.
\par}

\en{Tensor}\ru{Тензор}~$\rotationtensor$ \en{relates}\ru{соотносит} \en{the two vectors}\ru{два вектора}\:--- \inquotes{\en{before rotation}\ru{до~поворота}}~${\initiallocationvector = \hspace{-0.1ex} \rho_i \hspace{.1ex} \mathcircabove{\bm{e}}_i}$ \en{and}\ru{и}~\inquotes{\en{after rotation}\ru{после поворота}}~${\locationvector = \hspace{-0.1ex} \rho_i \hspace{.1ex} \bm{e}_i}$.
\en{Components}\ru{Компоненты}~${\rho_i \hspace{-0.2ex} = \hspace{-0.1ex} \constant}$ \en{of}\ru{у}~$\locationvector$ \en{in rotated basis}\ru{в~повёрнутом базисе}~${\bm{e}_i}$\en{ are}\ru{\:---} \en{the same}\ru{такие~же} \en{as}\ru{как} \en{of}\ru{у}~${\initiallocationvector}$ \en{in immobile basis}\ru{в~неподвижном базисе}~${\mathcircabove{\bm{e}}_i}$.
\en{So that}\ru{Так что} \en{the rotation tensor}\ru{тензор поворота} \en{describes}\ru{описывает} \en{the rotation of the vector together with the basis}\ru{вращение вектора вместе с~базисом}.
\en{And since}\ru{И~поскольку} ${\bm{e}_i = \bm{e}_j \mathcircabove{\bm{e}}_j \hspace{-0.15ex} \dotp \mathcircabove{\bm{e}}_i
\:\Leftrightarrow\,
\rho_i \hspace{.1ex} \bm{e}_i \hspace{-0.2ex} = \bm{e}_j \mathcircabove{\bm{e}}_j \hspace{-0.15ex} \dotp \rho_i \hspace{.1ex} \mathcircabove{\bm{e}_i}}$, \en{then}\ru{то}

\nopagebreak\vspace{-0.25em}\begin{equation}\label{rodriguesrotationformula}
\locationvector = \rotationtensor \dotp\hspace{.2ex} \initiallocationvector
\end{equation}

\vspace{-0.4em}\noindent
(\en{this is}\ru{это} \en{the }\href{https://fr.wikipedia.org/wiki/Rotation_vectorielle#Cas_g%C3%A9n%C3%A9ral}{\ru{формула поворота }Rodrigues\ru{’а}\en{ rotation formula}}).

\bookauthor{Olinde Rodrigues}.
\href{http://sites.mathdoc.fr/JMPA/PDF/JMPA_1840_1_5_A39_0.pdf}{Des lois géométriques qui régissent les déplacements d’un système solide dans l’espace, et de la variation des coordonnées provenant de ces déplacements considérés indépendants des causes qui peuvent les produire. \emph{Journal de mathématiques pures et appliquées}, tome~5~(1840), pages~380\hbox{--}440.}

\vspace{-0.1em}
\en{For}\ru{Для} \en{a~second complexity tensor}\ru{тензора второй сложности} ${\mathcircabove{\bm{C}} = C_{i\hspace{-0.1ex}j} \hspace{.1ex} \mathcircabove{\bm{e}}_i \mathcircabove{\bm{e}}_j}$\en{,} \en{a~rotation}\ru{поворот} \en{into the current position}\ru{в~текущую позицию} ${\bm{C} = C_{i\hspace{-0.1ex}j} \hspace{.1ex} \bm{e}_i \bm{e}_j}$ \en{looks like}\ru{выглядит как}

\nopagebreak\vspace{-0.25em}\begin{equation}
\bm{e}_i C_{i\hspace{-0.1ex}j} \bm{e}_j \hspace{-0.2ex}
= \bm{e}_i \mathcircabove{\bm{e}}_i \hspace{-0.1ex} \dotp \mathcircabove{\bm{e}}_p C_{pq} \mathcircabove{\bm{e}}_q \hspace{-0.1ex} \dotp \mathcircabove{\bm{e}}_j \bm{e}_j
\,\;\Leftrightarrow\;\:
\bm{C} = \rotationtensor \dotp\hspace{.1ex} \mathcircabove{\bm{C}} \dotp \rotationtensor^{\T}
\hspace{-0.3ex} .
\end{equation}

\en{The~essential property}\ru{Существенное свойство}
\en{of a~rotation tensor}\ru{тензора поворота}\:---
\en{the~orthogonality}\ru{ортогональность}\:---
\en{is expressed as}\ru{выражается как}

\nopagebreak\en{\vspace{-0.8em}}\ru{\vspace{-0.25em}}
\begin{equation}\label{orthogonalityofrotationtensor}
\tikzmark{beginJustARotation} \rotationtensor \tikzmark{endJustARotation}
\hspace{.2ex} \dotp \hspace{.2ex}
\tikzmark{beginATransposedRotation} \rotationtensor^{\T} \hspace{-0.5ex} \tikzmark{endATransposedRotation}
\hspace{.2ex} = \hspace{.2ex}
\tikzmark{beginAnotherTransposedRotation} \rotationtensor^{\T} \hspace{-0.5ex} \tikzmark{endAnotherTransposedRotation}
\hspace{.2ex} \dotp \hspace{.2ex}
\tikzmark{beginJustAnotherRotation} \rotationtensor \tikzmark{endJustAnotherRotation}
\hspace{.2ex} = \hspace{.2ex}
\tikzmark{justTheUnitDyadHere} \UnitDyad \tikzmark{justTheUnitDyadThere}
\hspace{.5ex} ,
\end{equation}
\AddUnderBrace[line width=.75pt][-0.1ex, -0.1ex]%
{beginJustARotation}{endJustARotation}{${\scriptstyle \bm{e}_i \mathcircabove{\bm{e}}_i}$}
\AddUnderBrace[line width=.75pt][-0.1ex, -0.1ex]%
{beginATransposedRotation}{endATransposedRotation}{${\scriptstyle \mathcircabove{\bm{e}}_j \bm{e}_j}$}
\AddUnderBrace[line width=.75pt][-0.1ex, -0.1ex]%
{beginAnotherTransposedRotation}{endAnotherTransposedRotation}{${\scriptstyle \mathcircabove{\bm{e}}_i \bm{e}_i}$}
\AddUnderBrace[line width=.75pt][-0.1ex, -0.1ex]%
{beginJustAnotherRotation}{endJustAnotherRotation}{${\scriptstyle \bm{e}_j \mathcircabove{\bm{e}}_j}$}
\AddUnderBrace[line width=.75pt][0ex, -0.1ex]%
{justTheUnitDyadHere}{justTheUnitDyadThere}{${\scriptstyle \mathclap{%
\begin{subarray}{l}
   \mathcircabove{\bm{e}}_i \mathcircabove{\bm{e}}_i \\
   \bm{e}_i \bm{e}_i
\end{subarray}%
}}$}

\vspace{-0.1em}\noindent
\en{that~is}\ru{то~есть} \en{the~transposed tensor}\ru{транспонированный тензор} \en{coincides}\ru{совпадает} \en{with}\ru{с}~\en{the~reciprocal tensor}\ru{обратным тензором}:
${\rotationtensor^{\T} \hspace{-0.32ex} = \rotationtensor^{\expminusone} \Leftrightarrow \rotationtensor = \rotationtensor^{\expminusT}\hspace{-0.25ex}}$.

\en{An~orthogonal tensor}\ru{Ортогональный тензор}
\en{retains}\ru{сохраняет}
\en{lengths and~angles}\ru{дл\'{и}ны и~углы}
(\en{the~metric}\ru{метрику})\ru{,}
\en{because}\ru{потому что}
\en{it}\ru{он}
\en{does not change}\ru{не~меняет}
\en{the }\hbox{\hspace{-0.2ex}\inquotes{${\dotp\hspace{.22ex}}$}\hspace{-0.2ex}}-\en{product}\ru{произведение}
\en{of vectors}\ru{векторов}

\nopagebreak\vspace{-0.1em}\begin{equation}
\label{orthogonaltensorkeepsmetrics}
\left( \rotationtensor \hspace{-0.2ex} \dotp \bm{a}\hspace{.2ex} \right) \dotp \left( \rotationtensor \hspace{-0.2ex} \dotp \bm{b}\hspace{.2ex} \right)
= \bm{a} \dotp \rotationtensor^{\T} \hspace{-0.4ex} \dotp \rotationtensor \hspace{-0.1ex} \dotp \hspace{.1ex} \bm{b}
= \bm{a} \dotp \UnitDyad \dotp \bm{b}
= \bm{a} \dotp \bm{b}
\hspace{.1ex} .
\end{equation}

\en{For all}\ru{Для всех}
\en{orthogonal tensors}\ru{ортогональных тензоров}
${\left(\operatorname{det} \orthogonaltensor\hspace{.1ex}\right)^2 \hspace{-0.1ex} = 1}$:

\nopagebreak\vspace{-0.1em}\begin{equation*}
1 = \operatorname{det} \UnitDyad = \operatorname{det} \left({\hspace{-0.1ex} \orthogonaltensor \dotp \orthogonaltensor^{\T} \hspace{.2ex}}\right) \hspace{-0.3ex}
= \left({\operatorname{det} \orthogonaltensor \hspace{.1ex}}\right) \left({\operatorname{det} \orthogonaltensor^{\T} \hspace{.2ex}}\right) \hspace{-0.3ex}
= \left(\operatorname{det} \orthogonaltensor \hspace{.1ex} \right)^{2}
\hspace{.1ex} .
\end{equation*}

\en{A~rotation tensor}\ru{Тензор поворота}
\en{is an~orthogonal tensor}\ru{это ортогональный тензор}
\en{with}\ru{с}~${\operatorname{det} \rotationtensor \hspace{-0.1ex} = +1}$.

\en{But}\ru{Но}
\en{not only}\ru{не~только лишь}
\en{rotation tensors}\ru{тензоры поворота}
\en{possess}\ru{обладают}
\en{the~property of~orthogonality}\ru{свойством ортогональности}.
\en{When}\ru{Когда}
\en{in}\ru{в}~\eqref{introductionofrotationtensor}
\en{the~first basis}\ru{первый базис}
\en{is }\inquotesx{\en{left}\ru{левый}}[,]
\en{and }\ru{а~}\en{the~second one}\ru{второй}
\en{is }\inquotesx{\en{right}\ru{правый}}[,]
\en{then}\ru{тогда}
\en{there’s}\ru{это}
\en{a~combination}\ru{комбинация}
\en{of rotating}\ru{вращения}
\en{and }\ru{и~}\en{mirroring}\ru{зеркалирования}
(\en{a~}\inquotes{roto\-reflexion})
${\rotationtensor = -\UnitDyad \dotp \rotationtensor}$
\en{with}\ru{с}~${\operatorname{det} \left( -\UnitDyad \dotp \rotationtensor \hspace{.15ex} \right) \hspace{-0.1ex} = -1}$.

\en{Any}\ru{У~любого}
\en{bivalent tensor}\ru{бивалентного тензора}
\en{in the three-dimensional}\ru{в~трёхмерном}~(3D)
\en{space}\ru{пространстве}\en{ has}
\en{at least}\ru{как минимум}
\en{one eigenvalue}\ru{одно собственное число}\:---
\en{the~root of}\ru{корень}~\eqref{chardetequation}\:---
\en{is }\en{non-complex}\ru{некомпл\'{е}ксное},
\en{or}\ru{или}
\ru{действительное}\ru{~(}real\ru{)}.
\en{For}\ru{Для}
\en{a~rotation tensor}\ru{тензора поворота}\en{,}
\en{it is equal to one}\ru{оно равно единице}

\nopagebreak\vspace{1em}\begin{equation*}
\begin{array}{c}
\rotationtensor \dotp \bm{a} = \eigenvalue \bm{a}
\:\Rightarrow\:
\tikzmark{BeginPaBrace} \bm{a} \hspace{.16ex} \dotp \tikzmark{BeginEBrace} \rotationtensor^{\T} \tikzmark{EndPaBrace} \hspace{-0.4ex} \dotp \rotationtensor \hspace{-0.32ex}\tikzmark{EndEBrace}\hspace{.32ex} \dotp \hspace{.16ex} \bm{a} = \eigenvalue \bm{a} \dotp \eigenvalue \bm{a}
\:\Rightarrow\: \eigenvalue^{2} \hspace{-0.2ex} = 1 \hspace{.1ex} .
\end{array}
\end{equation*}
\AddOverBrace[line width=.75pt][0.1ex,0.4ex]{BeginPaBrace}{EndPaBrace}{${\scriptstyle \rotationtensor \:\dotp\; \bm{a}}$}
\AddUnderBrace[line width=.75pt][-0.1ex,0.1ex]{BeginEBrace}{EndEBrace}{${\scriptstyle \UnitDyad}$}

\vspace{-0.5em}\noindent
\en{The~corresponding}\ru{Соответствующий}
\en{eigenvector}\ru{собственный вектор}
\en{is called}\ru{называется}
\en{the~axis of~rotation}\ru{осью поворота}.
\href{https://en.wikipedia.org/wiki/Euler%27s_rotation_theorem}{%
\en{The~}\ru{Теорема }Euler’\en{s}\ru{а}\en{ theorem}
\en{on finite rotation}\ru{о~конечном повороте}}
\en{is just about that}\ru{и~есть про~то, что}
\en{such an~axis exists}\ru{такая ось существует}
[http://eulerarchive.maa.org//docs/originals/E478.pdf].
\en{If}\ru{Если}
${\bm{k}}$\en{ is}\ru{\:---}
\en{the~unit vector of that axis}\ru{единичный вектор той оси},
\en{and}\ru{а}~${\vartheta}$\en{ is}\ru{\:---}
\en{the~finite angle of~rotation}\ru{кон\'{е}чный угол поворота},
\en{then}\ru{то}
\en{the~rotation tensor}\ru{тензор поворота}
\en{is representable as}\ru{представ\'{и}м как}

\nopagebreak\vspace{-0.1em}\begin{equation}\label{eulerfiniterotation}
\rotationtensor\hspace{.1ex}(\bm{k}, \vartheta) = \UnitDyad \operatorname{cos} \vartheta + \bm{k} \times\hspace{-0.2ex} \UnitDyad \operatorname{sin} \vartheta + \bm{k} \bm{k} \left( {1 - \operatorname{cos} \vartheta} \right)
\hspace{-0.2ex} .
\end{equation}

\begin{otherlanguage}{russian}

\vspace{-0.1em}
Доказывается эта формула так.
Направление~${\bm{k}}$
во~вр\'{е}мя поворота
не~меняется~(${\rotationtensor \hspace{-0.2ex}\dotp \bm{k} = \bm{k}\hspace{.12ex}}$), поэтому на~оси поворота ${\mathcircabove{\bm{e}}_3 \hspace{-0.2ex} = \bm{e}_3 \hspace{-0.2ex} = \bm{k}}$.
В~перпендикулярной плоскости~(\figureref{fig:eulerfiniterotation}) ${\mathcircabove{\bm{e}}_1 \hspace{-0.2ex} = \bm{e}_1 \operatorname{cos} \vartheta - \bm{e}_2 \operatorname{sin} \vartheta}$, ${\mathcircabove{\bm{e}}_2 \hspace{-0.2ex} = \bm{e}_1 \operatorname{sin} \vartheta + \bm{e}_2 \operatorname{cos} \vartheta}$, ${\rotationtensor = \bm{e}_i \mathcircabove{\bm{e}}_i \,\Rightarrow\hspace{.2ex}}$~\eqref{eulerfiniterotation}.

% ~ ~ ~ ~ ~
\begin{figure}[!htbp]

\vspace*{-0.5em}\[
\mathcircabove{\bm{e}}_i = \mathcircabove{\bm{e}}_i \dotp \bm{e}_j \bm{e}_j
\]

\vspace{-1.5em}\[
\left[ \begin{array}{c} \mathcircabove{\bm{e}}_1 \\ \mathcircabove{\bm{e}}_2 \\ \mathcircabove{\bm{e}}_3 \end{array} \right] =
\left[ \begin{array}{ccc}
\mathcircabove{\bm{e}}_1 \dotp \bm{e}_1 & \mathcircabove{\bm{e}}_1 \dotp \bm{e}_2 & \mathcircabove{\bm{e}}_1 \dotp \bm{e}_3 \\
\mathcircabove{\bm{e}}_2 \dotp \bm{e}_1 & \mathcircabove{\bm{e}}_2 \dotp \bm{e}_2 & \mathcircabove{\bm{e}}_2 \dotp \bm{e}_3 \\
\mathcircabove{\bm{e}}_3 \dotp \bm{e}_1 & \mathcircabove{\bm{e}}_3 \dotp \bm{e}_2 & \mathcircabove{\bm{e}}_3 \dotp \bm{e}_3
\end{array} \right] \hspace{-0.5ex}
\left[ \hspace{-0.12ex} \begin{array}{c} {\bm{e}}_1 \\ {\bm{e}}_2 \\ {\bm{e}_3} \end{array} \right]
\]

\vspace{-1.25em}

\begin{center}
\tdplotsetmaincoords{60}{120} % set orientation of axes
\pgfmathsetmacro{\angletheta}{42}
% three parameters for vector
\pgfmathsetmacro{\lengthofvector}{0.55}
\pgfmathsetmacro{\anglefromz}{40}
\pgfmathsetmacro{\anglefromx}{240}

\begin{tikzpicture}[scale=4, tdplot_main_coords] % tdplot_main_coords style to use 3dplot

	\coordinate (O) at (0,0,0);

	% draw initial axes
	\draw [line width=1.2pt, black, -{Stealth[round, length=4mm, width=2.4mm]}]
		(O) -- (1,0,0)
		node[pos=0.9, above, xshift=-0.8em] {$\mathcircabove{\bm{e}}_1$};

	\draw [line width=1.2pt, black, -{Stealth[round, length=4mm, width=2.4mm]}]
		(O) -- (0,1,0)
		node[pos=0.9, above, xshift=1em, yshift=-0.2em] {$\mathcircabove{\bm{e}}_2$};

	\draw [line width=1.2pt, red, -{Stealth[round,length=4mm,width=2.4mm]}]
		(O) -- (0,0,0.9)
		node[anchor=south] {$\mathcircabove{\bm{e}}_3 = \bm{e}_3 = \bm{k}$};

	% draw initial vector
	\tdplotsetcoord{point}{\lengthofvector}{\anglefromz}{\anglefromx} % {length}{angle from z}{angle from x}
		% it also defines (pointxy), (pointxz), and (pointyz) projections of point
	\draw [line width=1.2pt, black, -{Stealth[round, length=4mm, width=2.4mm]}]
		(O) -- (point)
		node[anchor=south] {$\mathcircabove{\bm{r}}$};
	% draw its projection on xy plane
	\draw [line width=0.4pt, dotted, color=black] (O) -- (pointxy);
	\draw [line width=0.4pt, dotted, color=black] (pointxy) -- (point);

	% draw the angle, and label it
	% syntax: \tdplotdrawarc[coordinate frame, draw options]{center point}{r}{angle}{end angle}{label options}{label}
	\tdplotdrawarc [line width=0.5pt, red, ->]
		{(O)}{0.4}{0}{\angletheta}{anchor=north}{$\vartheta$}
	\tdplotdrawarc [line width=0.5pt, red, ->]
		{(O)}{0.4}{90}{90+\angletheta}{anchor=west}{$\vartheta$}

	% rotate coordinates using Euler angles "z(\alpha)y(\beta)z(\gamma)"
	\tdplotsetrotatedcoords{\angletheta}{0}{0}

	% draw rotated axes
	\draw [line width=1.2pt, blue, tdplot_rotated_coords, -{Stealth[round, length=4mm, width=2.4mm]}]
		(O) -- (1,0,0)
		node[pos=0.9, left, xshift=-0.1em] {$\bm{e}_1$};

	\draw [line width=1.2pt, blue, tdplot_rotated_coords, -{Stealth[round, length=4mm, width=2.4mm]}]
		(O) -- (0,1,0)
		node[pos=0.9, above, xshift=0.2em, yshift=0.2em] {$\bm{e}_2$};

	%%\draw [line width=1.2pt, blue, tdplot_rotated_coords, -{Stealth[round, length=4mm, width=2.4mm]}]
		%%(O) -- (0,0,0.8) ;

	% draw rotated vector
	\tdplotsetcoord{rotatedpoint}%
		{\lengthofvector}{\anglefromz}{\anglefromx+\angletheta}
	\draw [line width=1.2pt, blue, tdplot_rotated_coords, -{Stealth[round, length=4mm, width=2.4mm]}]
		(O) -- (rotatedpoint)
		node[anchor=south] {$\bm{r}$};
	% draw its projection on xy plane
	\draw [line width=0.4pt, dotted, color=blue, tdplot_rotated_coords] (O) -- (rotatedpointxy);
	\draw [line width=0.4pt, dotted, color=blue, tdplot_rotated_coords] (rotatedpointxy) -- (rotatedpoint);

	\tdplotdrawarc [line width=0.5pt, red, ->]
		{(O)}{0.28}{\anglefromx}{\anglefromx+\angletheta}{anchor=south east, xshift=0.3em, yshift=-0.1em}{$\vartheta$}

\end{tikzpicture}
\end{center}

\vspace{-1em}\[
\scalebox{0.8}[0.85]{$\left[ \begin{array}{ccc}
\mathcircabove{\bm{e}}_1 \dotp \bm{e}_1 & \mathcircabove{\bm{e}}_1 \dotp \bm{e}_2 & \mathcircabove{\bm{e}}_1 \dotp \bm{e}_3 \\
\mathcircabove{\bm{e}}_2 \dotp \bm{e}_1 & \mathcircabove{\bm{e}}_2 \dotp \bm{e}_2 & \mathcircabove{\bm{e}}_2 \dotp \bm{e}_3 \\
\mathcircabove{\bm{e}}_3 \dotp \bm{e}_1 & \mathcircabove{\bm{e}}_3 \dotp \bm{e}_2 & \mathcircabove{\bm{e}}_3 \dotp \bm{e}_3
\end{array} \right]$} \hspace{-0.32ex} = \hspace{-0.2ex}
%
\scalebox{0.8}[0.85]{$\left[ \hspace{-0.2ex} \begin{array}{ccc}
\operatorname{cos} \vartheta & \hspace{-1ex} \operatorname{cos} \left( 90\degree \!+ \vartheta \right) & \operatorname{cos} 90\degree \\
\operatorname{cos} \left( 90\degree \!- \vartheta \right) & \operatorname{cos} \vartheta & \operatorname{cos} 90\degree \\
\operatorname{cos} 90\degree & \operatorname{cos} 90\degree & \operatorname{cos} 0\degree
\end{array} \right]$} \hspace{-0.32ex} = \hspace{-0.2ex}
%
\scalebox{0.8}[0.85]{$\left[ \hspace{-0.1ex} \begin{array}{ccc}
\operatorname{cos} \vartheta & - \operatorname{sin} \vartheta & 0 \\
\operatorname{sin} \vartheta & \operatorname{cos} \vartheta & 0 \\
0 & 0 & 1
\end{array} \right]$}
\]

\vspace{-0.8em}
\[\begin{array}{c}
\mathcircabove{\bm{e}}_1 \hspace{-0.16ex} = \bm{e}_1 \operatorname{cos} \vartheta \hspace{0.1ex} - \hspace{0.1ex} \bm{e}_2 \operatorname{sin} \vartheta \\[0.1em]
\mathcircabove{\bm{e}}_2 \hspace{-0.16ex} = \bm{e}_1 \operatorname{sin} \vartheta \hspace{0.1ex} + \hspace{0.1ex} \bm{e}_2 \operatorname{cos} \vartheta \\[0.1em]
\mathcircabove{\bm{e}}_3 \hspace{-0.16ex} = \bm{e}_3 = \bm{k}
\end{array}\]

\vspace{-1em}
\begin{multline*}
\shoveleft{ \bm{P} = \bm{e}_1 \hspace{-0.1ex} \mathcircabove{\bm{e}}_1 + \bm{e}_2 \mathcircabove{\bm{e}}_2 + \bm{e}_3 \mathcircabove{\bm{e}}_3 = \hfill }\\[1.5em]
%
= \hspace{0.2ex} \tikzmark{StartBraceE1E1} {\bm{e}_1 \bm{e}_1 \operatorname{cos} \vartheta - \bm{e}_1 \bm{e}_2 \operatorname{sin} \vartheta \hspace{0.2em}} \tikzmark{EndBraceE1E1} \hspace{-0.1ex} + \hspace{0.1ex} \tikzmark{StartBraceE2E2} {\bm{e}_2 \bm{e}_1 \operatorname{sin} \vartheta + \bm{e}_2 \bm{e}_2 \operatorname{cos} \vartheta \hspace{0.2em}} \tikzmark{EndBraceE2E2} \hspace{-0.1ex} + \tikzmark{StartBraceE3E3} {\hspace{0.25ex} \bm{k} \bm{k} \hspace{0.1ex}} \tikzmark{EndBraceE3E3} \hspace{0.1ex} =\\[0.32em]
%
= \hspace{0.1ex} \bm{E} \operatorname{cos} \vartheta - \hspace{-0.1ex} \tikzmark{StartBraceKk} {\hspace{0.1ex}\bm{e}_3 \bm{e}_3\hspace{0.1ex}} \tikzmark{EndBraceKk} \hspace{-0.25ex} \operatorname{cos} \vartheta \hspace{0.1ex} + \tikzmark{StartBraceLeviCivita} {\left( \bm{e}_2 \bm{e}_1 - \bm{e}_1 \bm{e}_2 \right)} \tikzmark{EndBraceLeviCivita} \operatorname{sin} \vartheta + \bm{k} \bm{k} \hspace{0.1ex} =\\[1.5em]
%
\shoveright{ \hfill = \bm{E} \operatorname{cos} \vartheta + \bm{k} \times\hspace{-0.2ex} \bm{E} \operatorname{sin} \vartheta + \bm{k} \bm{k} \left({1 - \operatorname{cos} \vartheta}\right) }
\end{multline*}

\AddOverBrace[line width=0.75pt]{StartBraceE1E1}{EndBraceE1E1}{${\scriptstyle \bm{e}_1 \mathcircabove{\bm{e}}_1}$}
\AddOverBrace[line width=0.75pt]{StartBraceE2E2}{EndBraceE2E2}{${\scriptstyle \bm{e}_2 \mathcircabove{\bm{e}}_2}$}
\AddOverBrace[line width=0.75pt]{StartBraceE3E3}{EndBraceE3E3}{${\scriptstyle \bm{e}_3 \mathcircabove{\bm{e}}_3}$}
\AddUnderBrace[line width=0.75pt][-0.1ex,-0.2ex]{StartBraceKk}{EndBraceKk}{${\scriptstyle \bm{k}\bm{k}}$}
\AddUnderBrace[line width=0.75pt][-0.1ex,-0.2ex][xshift=0.4ex]{StartBraceLeviCivita}{EndBraceLeviCivita}{${\scriptstyle \bm{e}_3 \times \bm{e}_i \bm{e}_i \:=\: \levicivita_{3ij} \bm{e}_j \bm{e}_i}$}

\vspace{-0.5em}
\caption{\inquotes{\en{Finite rotation}\ru{Конечный поворот}}}\label{fig:eulerfiniterotation}
\end{figure}

% ~ ~ ~ ~ ~

Из~\eqref{eulerfiniterotation} и~\eqref{rodriguesrotationformula} получаем формулу поворота Родрига в~параметрах~$\bm{k}$ и~$\vartheta$:

\nopagebreak\vspace{-0.3em}\begin{equation*}
\locationvector \hspace{.3ex}
= \hspace{.4ex} \initiallocationvector \operatorname{cos} \vartheta \hspace{.3ex} + \hspace{.3ex} \bm{k} \times \initiallocationvector \hspace{.2ex} \operatorname{sin} \vartheta \hspace{.3ex} + \hspace{.4ex} \bm{k} \bm{k} \dotp \hspace{.1ex} \initiallocationvector \left({1 - \operatorname{cos} \vartheta}\right)
\hspace{-0.25ex} .
\end{equation*}

\vspace{-0.2em}
В~параметрах конечного поворота транспонирование, оно~же обращение, тензора~$\rotationtensor$ эквивалентно перемене направления поворота\:--- знака угла~$\vartheta$
\[
\rotationtensor^{\T} \hspace{-0.1ex}=\hspace{.1ex} \rotationtensor \hspace{.1ex} \bigr|_{\vartheta \,=\hspace{.1ex} -\vartheta} \hspace{-0.1ex} = \UnitDyad \operatorname{cos} \vartheta - \bm{k} \times\hspace{-0.2ex} \UnitDyad \operatorname{sin} \vartheta + \bm{k} \bm{k} \left({1 - \operatorname{cos} \vartheta}\right)
\hspace{-0.3ex} .
\]

Пусть теперь тензор поворота меняется со~временем:
${\rotationtensor \narroweq \rotationtensor(t)}$.
Псевдовектор угловой скорости~${\bm{\omega}}$ вводится через тензор поворота~$\rotationtensor$ таким путём.
Дифференцируем тождество ортогональности~\eqref{orthogonalityofrotationtensor} по~времени%
\footnote{\en{Various notations are used}\ru{Используются различные записи} \en{to designate the~time derivative}\ru{для обозначения производной по времени}.
\en{In~addition}\ru{Вдобавок} \en{to}\ru{к~записи} \en{the }Leibniz’\en{s}\ru{а}\en{ notation} ${\scalebox{.9}{$ \raisemath{.3em}{dx} $} \hspace{-0.3ex} / \hspace{-0.4ex} \scalebox{.9}{$ \raisemath{-.3em}{dt} $}\hspace{.1ex}}$, \en{the~very popular one is}\ru{очень популярна запись} \en{the~}\inquotes{\en{dot above}\ru{точка сверху}} Newton’\en{s}\ru{а}\en{ notation}~${\mathdotabove{x}}$.}

\nopagebreak\vspace{-0.1em}\begin{equation*}
\mathdotabove{\rotationtensor} \dotp \rotationtensor^{\T} \hspace{-0.1ex} + \hspace{.25ex} \rotationtensor \dotp \mathdotabove{\rotationtensor}^{\T} \hspace{-0.1ex} = \hspace{.1ex} {^2\bm{0}}
\hspace{.1ex} .
\end{equation*}

По~\eqref{transposeofdotproductforbivalenttensors}
${\left({ \mathdotabove{\rotationtensor} \dotp \rotationtensor^{\T} }\right)^{\raisemath{-0.25em}{\!\T}} \hspace{-0.4ex} = \rotationtensor \dotp \mathdotabove{\rotationtensor}^{\T}}$,
поэтому
тензор~${\mathdotabove{\rotationtensor} \dotp \rotationtensor^{\T\!}}$
оказывается анти\-сим\-метрич\-ным.
Тогда
согласно~\eqref{companionvector}
он представ\'{и}м
через сопутствующий вектор
как
${\mathdotabove{\rotationtensor} \dotp \rotationtensor^{\T\!} = \bm{\omega} \times \UnitDyad = \bm{\omega} \times \rotationtensor \dotp \rotationtensor^{\T}}$\!.
То~есть

\nopagebreak\vspace{-0.1em}\begin{equation}\label{angularvelocityvector}
\mathdotabove{\rotationtensor} = \bm{\omega} \times \rotationtensor, \;\:\:
\bm{\omega} \equiv -\, \displaystyle \onehalf \left( \mathdotabove{\rotationtensor} \dotp \rotationtensor^{\T} \right)_{\hspace{-0.2em}\Xcompanion}
\vspace{-0.25em}\end{equation}

Помимо этого общего представления~вектора~${\bm{\omega}}$, для~него есть и~другие. Например, через параметры конечного поворота.

Производная~${\mathdotabove{\rotationtensor}}$ в~параметрах конечного поворота в~общем случае (оба параметра\:--- и~единичный вектор~$\bm{k}$, и~угол~$\vartheta$\:--- переменны во~времени):
\vspace{.3em}%
\[
\begin{array}{r@{\hspace{.33em}}c@{\hspace{.25em}}l}
\mathdotabove{\rotationtensor} \hspace{.2em} & = & \hspace{.1em} \left(\rotationtensor^{\mathsf{\hspace{.12ex}S}} \hspace{-0.2ex} +^{\mathstrut} \rotationtensor^{\mathsf{\hspace{.12ex}A}}\right)^{\hspace{-0.2ex}\tikz[baseline=-0.5ex]\draw[black, fill=black] (0,0) circle (.266ex);} =
\hspace{.1em} \left(\hspace{.2ex} \tikzmark{StartBracePs} {\UnitDyad \operatorname{cos} \vartheta + \bm{k} \bm{k} \left({1 \!-\! \operatorname{cos} \vartheta}\right)} \hspace{-0.2ex} \tikzmark{EndBracePs} \hspace{.32ex} +^{\mathstrut} \hspace{.2ex}
\tikzmark{StartBracePa} {\bm{k} \hspace{-0.24ex}\times\hspace{-0.4ex} \UnitDyad \operatorname{sin} \vartheta \hspace{.2ex}} \tikzmark{EndBracePa} {} \hspace{.16ex}\right)^{\hspace{-0.2ex}\tikz[baseline=-0.5ex]\draw[black, fill=black] (0,0) circle (.266ex);} \hspace{-0.05em} =
\\[0.4em]
%
& = & \hspace{.2em} \tikzmark{StartBraceDotPs} {\left( \hspace{.1ex} \bm{k} \bm{k} \hspace{-0.1ex} - \hspace{-0.2ex} \UnitDyad \hspace{.1ex} \right) \hspace{-0.1ex} \mathdotabove{\vartheta} \operatorname{sin} \vartheta + \hspace{-0.2ex} ( \bm{k} \mathdotabove{\bm{k}} + \mathdotabove{\bm{k}} \bm{k} ) \hspace{-0.2ex} \left({1 \!-\! \operatorname{cos} \vartheta}\right)} \tikzmark{EndBraceDotPs} \hspace{.64ex} + \\[0.64em]
& & \hspace{13.2em} + \hspace{.72ex} \tikzmark{StartBraceDotPa} {\hspace{.12ex} \bm{k} \hspace{-0.24ex}\times\hspace{-0.4ex} \UnitDyad \hspace{.4ex} \mathdotabove{\vartheta} \operatorname{cos} \vartheta + \mathdotabove{\bm{k}} \hspace{-0.24ex}\times\hspace{-0.4ex} \UnitDyad \operatorname{sin} \vartheta} \tikzmark{EndBraceDotPa}
\hspace{.1em} .
\end{array}\]%
\vspace{-1.2em}

\AddOverBrace[line width=0.75pt][0.12ex,0]{StartBracePs}{EndBracePs}{${\scriptstyle \rotationtensor^{\mathsf{\hspace{.12ex}S}}}$}
\AddOverBrace[line width=0.75pt][-0.12ex,0]{StartBracePa}{EndBracePa}{${\scriptstyle \rotationtensor^{\mathsf{\hspace{.12ex}A}}}$}
\AddUnderBrace[line width=0.75pt]{StartBraceDotPs}{EndBraceDotPs}{${\scriptstyle \mathdotabove{\rotationtensor}^{\mathsf{\hspace{.12ex}S}}}$}
\AddUnderBrace[line width=0.75pt][-0.1ex,0.1ex]{StartBraceDotPa}{EndBraceDotPa}{${\scriptstyle \mathdotabove{\rotationtensor}^{\mathsf{\hspace{.12ex}A}}}$}

\vspace{-1.32em} \noindent Находим
\vspace{.2em}\[\begin{array}{r@{\hspace{.25em}}c@{\hspace{.4em}}l}
\mathdotabove{\rotationtensor} \dotp \rotationtensor^{\T} & = & (\hspace{.1em} \mathdotabove{\rotationtensor}^{\mathsf{\hspace{.12ex}S}} \hspace{-0.16ex} + \mathdotabove{\rotationtensor}^{\mathsf{\hspace{.12ex}A}} \hspace{.05em}) \hspace{-0.1ex} \dotp \hspace{-0.1ex} (\hspace{.1em} \rotationtensor^{\mathsf{\hspace{.12ex}S}} \hspace{-0.16ex} - \rotationtensor^{\mathsf{\hspace{.12ex}A}} \hspace{.1ex} \hspace{.05em}) =
\\[0.25em]
& = & \mathdotabove{\rotationtensor}^{\mathsf{\hspace{.12ex}S}} \hspace{-0.2ex}\dotp \rotationtensor^{\mathsf{\hspace{.12ex}S}}
+ \hspace{.2ex} \mathdotabove{\rotationtensor}^{\mathsf{\hspace{.12ex}A}} \hspace{-0.2ex}\dotp \rotationtensor^{\mathsf{\hspace{.12ex}S}}
- \hspace{.2ex} \mathdotabove{\rotationtensor}^{\mathsf{\hspace{.12ex}S}} \hspace{-0.2ex}\dotp \rotationtensor^{\mathsf{\hspace{.12ex}A}}
- \hspace{.2ex} \mathdotabove{\rotationtensor}^{\mathsf{\hspace{.12ex}A}} \hspace{-0.2ex}\dotp \rotationtensor^{\mathsf{\hspace{.12ex}A}} ,
\end{array}\]

\vspace{-0.5em}\noindent
\en{using}\ru{используя}
\[\scalebox{0.95}[0.96]{$\begin{array}{c}
\bm{k} \dotp \bm{k} = 1 = \constant \,\Rightarrow\:
\bm{k} \dotp \mathdotabove{\bm{k}} + \mathdotabove{\bm{k}} \dotp \bm{k} = 0 \;\Leftrightarrow\; \mathdotabove{\bm{k}} \dotp \bm{k} = \bm{k} \dotp \mathdotabove{\bm{k}} = 0 \hspace{.1ex} ,
\\[.08em]
%
\bm{k} \bm{k} \hspace{-0.2ex}\dotp\hspace{-0.2ex} \bm{k} \bm{k} = \bm{k} \bm{k}, \:\:
\mathdotabove{\bm{k}} \bm{k} \hspace{-0.2ex}\dotp\hspace{-0.2ex} \bm{k} \bm{k} = \mathdotabove{\bm{k}} \bm{k} , \:\:
\bm{k} \mathdotabove{\bm{k}} \hspace{-0.2ex}\dotp\hspace{-0.2ex} \bm{k} \bm{k} = {\hspace{-0.2ex}^2\bm{0}} \hspace{.1ex},
\\[.16em]
%
\left( \hspace{.1ex} \bm{k} \bm{k} \hspace{-0.1ex} - \hspace{-0.2ex} \UnitDyad \hspace{.1ex} \right) \hspace{-0.2ex} \dotp \bm{k} = \bm{k} - \bm{k} = {\bm{0}} \hspace{.1ex}, \,\,
\left( \hspace{.1ex} \bm{k} \bm{k} \hspace{-0.1ex} - \hspace{-0.2ex} \UnitDyad \hspace{.1ex} \right) \hspace{-0.2ex} \dotp \bm{k} \bm{k} = \bm{k} \bm{k} - \bm{k} \bm{k} = {\hspace{-0.2ex}^2\bm{0}} \hspace{.1ex} ,
\\[.08em]
%
\bm{k} \dotp \hspace{-0.1ex} ( \bm{k} \hspace{-0.24ex}\times\hspace{-0.4ex} \UnitDyad ) \hspace{-0.1ex}
= \hspace{-0.1ex} ( \bm{k} \hspace{-0.24ex}\times\hspace{-0.4ex} \UnitDyad ) \hspace{-0.2ex} \dotp \bm{k}
= \bm{k} \hspace{-0.24ex}\times\hspace{-0.2ex} \bm{k} = \bm{0} \hspace{.1ex} , \,\,
\bm{k} \bm{k} \dotp \hspace{-0.1ex} ( \bm{k} \hspace{-0.24ex}\times\hspace{-0.4ex} \UnitDyad ) \hspace{-0.1ex}
= \hspace{-0.1ex} ( \bm{k} \hspace{-0.24ex}\times\hspace{-0.4ex} \UnitDyad ) \hspace{-0.2ex} \dotp \bm{k} \bm{k}
= {\hspace{-0.2ex}^2\bm{0}} \hspace{0.1ex},
\\[.08em]
%
\left( \hspace{0.1ex} \bm{k} \bm{k} \hspace{-0.1ex} - \hspace{-0.2ex} \UnitDyad \hspace{0.1ex} \right) \hspace{-0.2ex} \dotp \hspace{-0.1ex} ( \bm{k} \hspace{-0.24ex}\times\hspace{-0.4ex} \UnitDyad ) \hspace{-0.1ex} = \hspace{-0.1ex}
- \hspace{0.2ex} \bm{k} \hspace{-0.24ex}\times\hspace{-0.4ex} \UnitDyad \hspace{0.1ex},
\\
%
( \bm{a} \hspace{-0.24ex}\times\hspace{-0.4ex} \UnitDyad ) \hspace{-0.2ex} \dotp \bm{b} =
\bm{a} \hspace{-0.2ex}\times\hspace{-0.36ex} ( \UnitDyad \hspace{-0.1ex} \dotp \bm{b} ) \hspace{-0.16ex} =
\bm{a} \hspace{-0.16ex}\times\hspace{-0.12ex} \bm{b} \:\,\Rightarrow\,
( \mathdotabove{\bm{k}} \hspace{-0.24ex}\times\hspace{-0.4ex} \UnitDyad ) \hspace{-0.2ex} \dotp \bm{k} \bm{k} =
\mathdotabove{\bm{k}} \hspace{-0.16ex}\times\hspace{-0.2ex} \bm{k} \bm{k} \hspace{0.1ex},
\end{array}$}\]

\vspace{-0.4em} \noindent \eqref{vectorcrossidentitydotvectorcrossidentity} $\,\Rightarrow\,$
\scalebox{0.95}[0.96]{${( \bm{k} \hspace{-0.24ex}\times\hspace{-0.4ex} \UnitDyad ) \hspace{-0.2ex} \dotp \hspace{-0.2ex} ( \bm{k} \hspace{-0.24ex}\times\hspace{-0.4ex} \UnitDyad ) \hspace{-0.16ex} = \bm{k} \bm{k} \hspace{-0.1ex} - \hspace{-0.2ex} \UnitDyad}$},\hspace{0.4ex}
%
\scalebox{0.95}[0.96]{${\displaystyle ( \mathdotabove{\bm{k}} \hspace{-0.24ex}\times\hspace{-0.4ex} \UnitDyad ) \hspace{-0.2ex} \dotp \hspace{-0.2ex} ( \bm{k} \hspace{-0.24ex}\times\hspace{-0.4ex} \UnitDyad ) \hspace{-0.16ex} = \bm{k} \mathdotabove{\bm{k}} \hspace{-0.1ex} - \tikzbackcancel[black!25]{$\mathdotabove{\bm{k}} \hspace{-0.2ex}\dotp\hspace{-0.2ex} \bm{k} \hspace{.16ex} \UnitDyad$}}$\hspace{.16ex}},

\noindent \eqref{vectorcrossvectorcrossidentity} $\,\Rightarrow\,$
\scalebox{0.95}[0.96]{$\mathdotabove{\bm{k}} \bm{k} \hspace{-0.1ex} - \hspace{-0.1ex} \bm{k} \mathdotabove{\bm{k}} = \hspace{-0.16ex} ( \bm{k} \hspace{-0.2ex} \times \hspace{-0.24ex} \mathdotabove{\bm{k}} ) \hspace{-0.32ex} \times \hspace{-0.32ex} \UnitDyad$},\hspace{.4ex}
%
\scalebox{0.95}[0.96]{${\displaystyle ( \mathdotabove{\bm{k}} \hspace{-0.2ex}\times\hspace{-0.24ex} \bm{k} ) \hspace{.2ex} \bm{k} \hspace{-0.1ex} - \hspace{-0.1ex} \bm{k} \hspace{.16ex} ( \mathdotabove{\bm{k}} \hspace{-0.2ex}\times\hspace{-0.24ex} \bm{k} ) \hspace{-0.16ex} = \bm{k} \hspace{-0.2ex} \times \hspace{-0.32ex} ( \mathdotabove{\bm{k}} \hspace{-0.2ex}\times\hspace{-0.24ex} \bm{k} ) \hspace{-0.32ex} \times \hspace{-0.32ex} \UnitDyad}$\hspace{.16ex}}

\begin{fleqn}[0pt]
\begin{multline*}
\shoveleft{\scalebox{0.94}[0.96]{$\mathdotabove{\bm{P}}^{\mathsf{\hspace{0.12ex}S}} \hspace{-0.2ex}\dotp \bm{P}^{\mathsf{\hspace{.12ex}S}} \hspace{-0.25ex} = $} \hspace{2em} \hfill}
\\[-0.25em]
%
\shoveleft{\scalebox{0.8}[0.82]{$= \hspace{.2ex} \left( \hspace{.1ex} \bm{k} \bm{k} \hspace{-0.1ex} - \hspace{-0.2ex} \UnitDyad \hspace{.1ex} \right) \hspace{-0.1ex} \mathdotabove{\vartheta} \operatorname{sin} \vartheta \dotp \UnitDyad \operatorname{cos} \vartheta +
( \bm{k} \mathdotabove{\bm{k}} + \mathdotabove{\bm{k}} \bm{k} ) \hspace{-0.2ex} \left({1 \!-\! \operatorname{cos} \vartheta}\right) \dotp \UnitDyad \operatorname{cos} \vartheta \hspace{.32em} +$} \hfill}
\\[-0.2em]
\shoveright{\hfill \scalebox{0.8}[0.82]{$+\; \tikzbackcancel[black!25]{$\left( \hspace{.1ex} \bm{k} \bm{k} \hspace{-0.1ex} - \hspace{-0.2ex} \UnitDyad \hspace{.1ex} \right) \hspace{-0.1ex} \mathdotabove{\vartheta} \operatorname{sin} \vartheta \dotp \bm{k} \bm{k} \left({1 \!-\! \operatorname{cos} \vartheta}\right)$} \hspace{.2ex} +
( \bm{k} \mathdotabove{\bm{k}} + \mathdotabove{\bm{k}} \bm{k} ) \hspace{-0.2ex} \left({1 \!-\! \operatorname{cos} \vartheta}\right) \hspace{-0.2ex} \dotp \bm{k} \bm{k} \left({1 \!-\! \operatorname{cos} \vartheta}\right) =$}}
\\
%
\scalebox{0.8}[0.82]{$= \left( \hspace{.1ex} \bm{k} \bm{k} \hspace{-0.1ex} - \hspace{-0.2ex} \UnitDyad \hspace{.1ex} \right) \hspace{-0.1ex} \mathdotabove{\vartheta} \operatorname{sin} \vartheta \operatorname{cos} \vartheta
+ ( \bm{k} \mathdotabove{\bm{k}} + \mathdotabove{\bm{k}} \bm{k} ) \hspace{-0.1ex} \operatorname{cos} \vartheta \left({1 \!-\! \operatorname{cos} \vartheta}\right) + ( \tikzbackcancel[black!25]{$\bm{k} \mathdotabove{\bm{k}} \hspace{-0.1ex}\dotp\hspace{-0.1ex} \bm{k} \bm{k}$} + \mathdotabove{\bm{k}} \bm{k} \hspace{-0.1ex}\dotp\hspace{-0.1ex} \bm{k} \bm{k} ) \left({1 \!-\! \operatorname{cos} \vartheta}\right)^{\hspace{-0.12ex}2} \hspace{-0.25ex} =$}
\\
%
\shoveleft{\scalebox{0.8}[0.82]{$= \left( \hspace{.1ex} \bm{k} \bm{k} \hspace{-0.1ex} - \hspace{-0.2ex} \UnitDyad \hspace{.1ex} \right) \hspace{-0.1ex} \mathdotabove{\vartheta} \operatorname{sin} \vartheta \operatorname{cos} \vartheta + \bm{k} \mathdotabove{\bm{k}} \operatorname{cos} \vartheta \left({1 \!-\! \operatorname{cos} \vartheta}\right) +$} \hfill}
\\[-0.2em]
\shoveright{\hfill \scalebox{0.8}[0.82]{$+ \hspace{.24em} \mathdotabove{\bm{k}} \bm{k} \operatorname{cos} \vartheta - \mathdotabove{\bm{k}} \bm{k} \operatorname{cos}^{2\hspace{-0.4ex}} \vartheta + \mathdotabove{\bm{k}} \bm{k} - 2 \, \mathdotabove{\bm{k}} \bm{k} \operatorname{cos} \vartheta + \mathdotabove{\bm{k}} \bm{k} \operatorname{cos}^{2\hspace{-0.4ex}} \vartheta =$}}\\
%
%% \shoveright{\hfill \scalebox{0.8}[0.82]{$= \left( \hspace{.1ex} \bm{k} \bm{k} \hspace{-0.1ex} - \hspace{-0.2ex} \UnitDyad \hspace{.1ex} \right) \hspace{-0.1ex} \mathdotabove{\vartheta} \operatorname{sin} \vartheta \operatorname{cos} \vartheta \hspace{.1ex}
%% + \hspace{.1ex} \bm{k} \mathdotabove{\bm{k}} \operatorname{cos} \vartheta \left({1 \!-\! \operatorname{cos} \vartheta}\right)
%% + \mathdotabove{\bm{k}} \bm{k}
%% - \mathdotabove{\bm{k}} \bm{k} \operatorname{cos} \vartheta =$}}\\
%
\shoveright{\hfill \hspace{4.8em} \scalebox{0.94}[0.96]{$= \left( \hspace{.1ex} \bm{k} \bm{k} \hspace{-0.1ex} - \hspace{-0.2ex} \UnitDyad \hspace{.1ex} \right) \hspace{-0.1ex} \mathdotabove{\vartheta} \operatorname{sin} \vartheta \operatorname{cos} \vartheta \hspace{.1ex}
%% + \hspace{-0.1ex} ( \bm{k} \mathdotabove{\bm{k}} \operatorname{cos} \vartheta + \mathdotabove{\bm{k}} \bm{k} ) \hspace{-0.2ex} \left({1 \!-\! \operatorname{cos} \vartheta}\right)
+ \bm{k} \mathdotabove{\bm{k}} \operatorname{cos} \vartheta
- \bm{k} \mathdotabove{\bm{k}} \operatorname{cos}^{2\hspace{-0.4ex}} \vartheta
+ \mathdotabove{\bm{k}} \bm{k} \left({1 \!-\! \operatorname{cos} \vartheta}\right) \hspace{-0.16ex},$}}
\end{multline*}
\begin{multline*}
\shoveleft{\scalebox{0.94}[0.96]{$\mathdotabove{\bm{P}}^{\mathsf{\hspace{.12ex}A}} \hspace{-0.2ex}\dotp \bm{P}^{\mathsf{\hspace{.12ex}S}} \hspace{-0.25ex} = $} \hfill} \\[-0.25em]
%
\shoveleft{\scalebox{0.8}[0.82]{$= ( \hspace{.12ex} \bm{k} \hspace{-0.24ex}\times\hspace{-0.4ex} \UnitDyad \hspace{.1ex} ) \hspace{-0.2ex} \dotp \hspace{-0.2ex} \UnitDyad \hspace{.4ex} \mathdotabove{\vartheta} \operatorname{cos}^{2\hspace{-0.4ex}} \vartheta +
( \hspace{.12ex} \mathdotabove{\bm{k}} \hspace{-0.24ex}\times\hspace{-0.4ex} \UnitDyad \hspace{.1ex} ) \hspace{-0.2ex} \dotp \hspace{-0.2ex} \UnitDyad \hspace{.1ex} \operatorname{sin} \vartheta \operatorname{cos} \vartheta \:+$} \hfill} \\[-0.2em]
\shoveright{\hfill \scalebox{0.8}[0.82]{$+\; \tikzbackcancel[black!25]{$ ( \hspace{.12ex} \bm{k} \hspace{-0.24ex}\times\hspace{-0.4ex} \UnitDyad \hspace{.1ex} ) \hspace{-0.2ex} \dotp \hspace{-0.1ex} \bm{k} \bm{k} \hspace{.5ex} \mathdotabove{\vartheta} \operatorname{cos} \vartheta \left({1 \!-\! \operatorname{cos} \vartheta}\right) $} \hspace{.2ex} +
( \hspace{.12ex} \mathdotabove{\bm{k}} \hspace{-0.24ex}\times\hspace{-0.4ex} \UnitDyad \hspace{.1ex} ) \hspace{-0.25ex} \dotp \hspace{-0.1ex} \bm{k} \bm{k} \hspace{.2ex} \operatorname{sin} \vartheta \left({1 \!-\! \operatorname{cos} \vartheta}\right) =$}} \\
%
\hspace{3.85em} \scalebox{0.94}[0.96]{$= \hspace{.2ex} \bm{k} \hspace{-0.24ex}\times\hspace{-0.4ex} \UnitDyad \hspace{.4ex} \mathdotabove{\vartheta} \operatorname{cos}^{2\hspace{-0.4ex}} \vartheta +
\mathdotabove{\bm{k}} \hspace{-0.24ex}\times\hspace{-0.4ex} \UnitDyad \hspace{.1ex} \operatorname{sin} \vartheta \operatorname{cos} \vartheta +
%%\hspace{-0.12ex} ( \hspace{.12ex} \mathdotabove{\bm{k}} \hspace{-0.24ex}\times\hspace{-0.4ex} \UnitDyad \hspace{.1ex} ) \hspace{-0.25ex} \dotp \hspace{-0.1ex} \bm{k} \bm{k} \hspace{.2ex}
\mathdotabove{\bm{k}} \hspace{-0.24ex}\times\hspace{-0.32ex} \bm{k} \bm{k} \hspace{.2ex}
\operatorname{sin} \vartheta \left({1 \!-\! \operatorname{cos} \vartheta}\right) \hspace{-0.16ex},$}
\end{multline*}
\begin{multline*}
\shoveleft{\scalebox{0.94}[0.96]{$\mathdotabove{\bm{P}}^{\mathsf{\hspace{.12ex}S}} \hspace{-0.2ex}\dotp \bm{P}^{\mathsf{\hspace{.12ex}A}} \hspace{-0.25ex} = $} \hfill} \\[-0.25em]
%
\scalebox{0.8}[0.82]{$= ( \hspace{.1ex} \bm{k} \bm{k} \hspace{-0.1ex} - \hspace{-0.2ex} \UnitDyad \hspace{.1ex} ) \hspace{.25ex} \mathdotabove{\vartheta} \operatorname{sin} \vartheta \dotp ( \bm{k} \hspace{-0.24ex}\times\hspace{-0.4ex} \UnitDyad ) \operatorname{sin} \vartheta +
( \bm{k} \mathdotabove{\bm{k}} + \mathdotabove{\bm{k}} \bm{k} ) \hspace{-0.2ex} \left({1 \!-\! \operatorname{cos} \vartheta}\right) \hspace{-0.1ex} \dotp ( \bm{k} \hspace{-0.24ex}\times\hspace{-0.4ex} \UnitDyad ) \operatorname{sin} \vartheta =$} \\
%
\scalebox{0.78}[0.82]{$= \hspace{.2ex} \tikzbackcancel[black!25]{$\bm{k} \bm{k} \hspace{-0.12ex} \dotp \hspace{-0.12ex} ( \bm{k} \hspace{-0.24ex}\times\hspace{-0.4ex} \UnitDyad \hspace{.1ex} ) \hspace{.32ex} \mathdotabove{\vartheta} \operatorname{sin}^{\hspace{-0.1ex}2\hspace{-0.4ex}} \vartheta$} \hspace{.12ex}
- \hspace{-0.1ex} \UnitDyad \hspace{-0.16ex} \dotp \hspace{-0.12ex} ( \bm{k} \hspace{-0.24ex}\times\hspace{-0.4ex} \UnitDyad \hspace{.1ex} ) \hspace{.25ex} \mathdotabove{\vartheta} \operatorname{sin}^{\hspace{-0.1ex}2\hspace{-0.4ex}} \vartheta
+ \hspace{-0.2ex} \left( \hspace{-0.1ex} \bm{k} \mathdotabove{\bm{k}} \dotp \hspace{-0.1ex} ( \bm{k} \hspace{-0.32ex}\times\hspace{-0.4ex} \UnitDyad ) \hspace{-0.16ex} + \hspace{.1ex} \tikzbackcancel[black!25]{$\mathdotabove{\bm{k}} \bm{k} \dotp \hspace{-0.1ex} ( \bm{k} \hspace{-0.32ex}\times\hspace{-0.4ex} \UnitDyad$} ) \hspace{-0.12ex} \right) \hspace{-0.2ex} \operatorname{sin} \vartheta \left({1 \!-\! \operatorname{cos} \vartheta} \right) =$} \\[-0.25em]
%
\hspace{13em} \scalebox{0.94}[0.96]{$= \hspace{-0.16ex} - \hspace{.2ex} \bm{k} \hspace{-0.24ex}\times\hspace{-0.4ex} \UnitDyad \hspace{.4ex} \mathdotabove{\vartheta} \operatorname{sin}^{\hspace{-0.1ex}2\hspace{-0.4ex}} \vartheta
+ \bm{k} \mathdotabove{\bm{k}} \hspace{-0.24ex}\times\hspace{-0.32ex} \bm{k} \hspace{.2ex} \operatorname{sin} \vartheta \left({1 \!-\! \operatorname{cos} \vartheta}\right) \hspace{-0.16ex},$}
\end{multline*}
\begin{multline*}
\shoveleft{\scalebox{0.94}[0.96]{$\mathdotabove{\bm{P}}^{\mathsf{\hspace{.12ex}A}} \hspace{-0.2ex}\dotp \bm{P}^{\mathsf{\hspace{.12ex}A}} \hspace{-0.25ex} =
%
( \bm{k} \hspace{-0.24ex}\times\hspace{-0.4ex} \UnitDyad ) \hspace{.32ex} \mathdotabove{\vartheta} \operatorname{cos} \vartheta \dotp ( \bm{k} \hspace{-0.24ex}\times\hspace{-0.4ex} \UnitDyad ) \operatorname{sin} \vartheta + \hspace{-0.1ex}
( \mathdotabove{\bm{k}} \hspace{-0.24ex}\times\hspace{-0.4ex} \UnitDyad ) \hspace{-0.16ex} \dotp \hspace{-0.16ex} ( \bm{k} \hspace{-0.24ex}\times\hspace{-0.4ex} \UnitDyad ) \hspace{.1ex} \operatorname{sin}^{\hspace{-0.1ex}2\hspace{-0.4ex}} \vartheta = $} \hfill} \\
%
\shoveright{\hfill \hspace{16em} \scalebox{0.94}[0.96]{$= ( \bm{k} \bm{k} \hspace{-0.1ex} - \hspace{-0.2ex} \UnitDyad ) \hspace{.32ex} \mathdotabove{\vartheta} \operatorname{sin} \vartheta \operatorname{cos} \vartheta
+ \bm{k} \mathdotabove{\bm{k}} \operatorname{sin}^{\hspace{-0.1ex}2\hspace{-0.4ex}} \vartheta \hspace{.2ex};$}}
\end{multline*}
\end{fleqn}

\begin{multline*}
\scalebox{0.94}[0.96]{$\mathdotabove{\bm{P}} \dotp \bm{P}^{\T}$} \hspace{.25ex}
\scalebox{0.92}[0.96]{$= \hspace{.1ex}
\mathdotabove{\bm{P}}^{\mathsf{\hspace{.12ex}S}} \hspace{-0.2ex}\dotp \bm{P}^{\mathsf{\hspace{.12ex}S}}
+ \hspace{.2ex} \mathdotabove{\bm{P}}^{\mathsf{\hspace{.12ex}A}} \hspace{-0.25ex}\dotp \bm{P}^{\mathsf{\hspace{.12ex}S}}
- \hspace{.2ex} \mathdotabove{\bm{P}}^{\mathsf{\hspace{.12ex}S}} \hspace{-0.25ex}\dotp \bm{P}^{\mathsf{\hspace{.12ex}A}}
- \hspace{.2ex} \mathdotabove{\bm{P}}^{\mathsf{\hspace{.12ex}A}} \hspace{-0.25ex}\dotp \bm{P}^{\mathsf{\hspace{.12ex}A}} \hspace{-0.25ex} =$} \\[-0.25em]
%
\scalebox{0.8}[0.82]{$= {\color{black!50}{\left( \hspace{.1ex} \bm{k} \bm{k} \hspace{-0.1ex} - \hspace{-0.2ex} \UnitDyad \hspace{.1ex} \right) \hspace{-0.1ex} \mathdotabove{\vartheta} \operatorname{sin} \vartheta \operatorname{cos} \vartheta}}
+ \bm{k} \mathdotabove{\bm{k}} \operatorname{cos} \vartheta
- {\color{magenta!80!black}{\bm{k} \mathdotabove{\bm{k}}}} {\color{black!50}{\hspace{.4ex}\operatorname{cos}^{2\hspace{-0.4ex}} \vartheta}}
+ \mathdotabove{\bm{k}} \bm{k} \left({1 \!-\! \operatorname{cos} \vartheta}\right) +$} \\[-0.2em]
%
\scalebox{0.8}[0.82]{$+\; {\color{blue!80!black}{\bm{k} \hspace{-0.24ex}\times\hspace{-0.4ex} \UnitDyad \hspace{.4ex} \mathdotabove{\vartheta}}} {\color{black!50}{\hspace{.4ex}\operatorname{cos}^{2\hspace{-0.4ex}} \vartheta}}
+ \mathdotabove{\bm{k}} \hspace{-0.24ex}\times\hspace{-0.4ex} \UnitDyad \hspace{.1ex} \operatorname{sin} \vartheta \operatorname{cos} \vartheta
+ \mathdotabove{\bm{k}} \hspace{-0.24ex}\times\hspace{-0.32ex} \bm{k} \bm{k} \hspace{.2ex} {\color{green!50!black}{\hspace{.5ex}\operatorname{sin} \vartheta \left({1 \!-\! \operatorname{cos} \vartheta}\right)\hspace{.4ex}}} +$} \\[-0.25em]
%
\scalebox{0.8}[0.82]{$+\; {\color{blue!80!black}{\bm{k} \hspace{-0.24ex}\times\hspace{-0.4ex} \UnitDyad \hspace{.4ex} \mathdotabove{\vartheta}}} {\color{black!50}{\hspace{.4ex}\operatorname{sin}^{\hspace{-0.1ex}2\hspace{-0.4ex}} \vartheta}}
- \bm{k} \mathdotabove{\bm{k}} \hspace{-0.24ex}\times\hspace{-0.32ex} \bm{k} {\color{green!50!black}{\hspace{.5ex}\operatorname{sin} \vartheta \left({1 \!-\! \operatorname{cos} \vartheta}\right)\hspace{.4ex}}}
{\color{black!50}{-\hspace{.5ex} ( \bm{k} \bm{k} \hspace{-0.1ex} - \hspace{-0.2ex} \UnitDyad ) \hspace{.32ex} \mathdotabove{\vartheta} \operatorname{sin} \vartheta \operatorname{cos} \vartheta}}
- {\color{magenta!80!black}{\bm{k} \mathdotabove{\bm{k}}}} {\color{black!50}{\hspace{.4ex}\operatorname{sin}^{\hspace{-0.1ex}2\hspace{-0.4ex}} \vartheta}} =$} \\[-0.08em]
%
\scalebox{0.79}[0.82]{$= \bm{k} \hspace{-0.24ex}\times\hspace{-0.4ex} \UnitDyad \hspace{.4ex} \mathdotabove{\vartheta} \hspace{-0.1ex}
+ \hspace{-0.1ex} \hspace{-0.16ex} ( \hspace{.2ex} \mathdotabove{\bm{k}} \bm{k} \hspace{-0.2ex} - \hspace{-0.2ex} \bm{k} \mathdotabove{\bm{k}} \hspace{.2ex} ) \hspace{.1ex} ({1 \!-\! \operatorname{cos} \vartheta}) \hspace{-0.2ex}
+ \mathdotabove{\bm{k}} \hspace{-0.24ex}\times\hspace{-0.4ex} \UnitDyad \hspace{.1ex} \operatorname{sin} \vartheta \operatorname{cos} \vartheta \hspace{-0.1ex}
+ \hspace{-0.1ex} ( \mathdotabove{\bm{k}} \hspace{-0.28ex}\times\hspace{-0.32ex} \bm{k} \bm{k} \hspace{-0.12ex} - \hspace{-0.12ex} \bm{k} \mathdotabove{\bm{k}} \hspace{-0.28ex}\times\hspace{-0.32ex} \bm{k} ) \operatorname{sin} \vartheta \left({1 \!-\! \operatorname{cos} \vartheta}\right) = $} \\[-0.08em]
%
\scalebox{0.8}[0.82]{$= \bm{k} \hspace{-0.24ex}\times\hspace{-0.4ex} \UnitDyad \hspace{.4ex} \mathdotabove{\vartheta} \hspace{-0.1ex}
+ \hspace{-0.1ex} \bm{k} \hspace{-0.2ex}\times\hspace{-0.2ex}  \mathdotabove{\bm{k}} \hspace{-0.2ex}\times\hspace{-0.4ex} \UnitDyad \hspace{.25ex} ({1 \!-\! \operatorname{cos} \vartheta}) \hspace{-0.2ex}
+ \mathdotabove{\bm{k}} \hspace{-0.24ex}\times\hspace{-0.4ex} \UnitDyad \hspace{.1ex} \operatorname{sin} \vartheta \operatorname{cos} \vartheta \hspace{-0.1ex}
+ \bm{k} \hspace{-0.25ex} \times \hspace{-0.32ex} ( \mathdotabove{\bm{k}} \hspace{-0.2ex}\times\hspace{-0.2ex} \bm{k} ) \hspace{-0.4ex}\times\hspace{-0.4ex} \UnitDyad \hspace{.1ex} \operatorname{sin} \vartheta \left({1 \!-\! \operatorname{cos} \vartheta}\right) = $} \\[-0.08em]
%
\scalebox{0.78}[0.82]{$= \bm{k} \hspace{-0.24ex}\times\hspace{-0.4ex} \UnitDyad \hspace{.4ex} \mathdotabove{\vartheta} \hspace{-0.1ex}
+ \mathdotabove{\bm{k}} \hspace{-0.24ex}\times\hspace{-0.4ex} \UnitDyad \hspace{.1ex} \operatorname{sin} \vartheta \operatorname{cos} \vartheta \hspace{-0.1ex}
+ \hspace{-0.16ex} ( \mathdotabove{\bm{k}} \bm{k} \hspace{-0.1ex} \dotp \hspace{-0.16ex} \bm{k} \hspace{-0.1ex} - \tikzbackcancel[black!25]{$\bm{k} \mathdotabove{\bm{k}} \hspace{-0.16ex} \dotp \hspace{-0.1ex} \bm{k} \hspace{.1ex}$} ) \hspace{-0.4ex}\times\hspace{-0.4ex} \UnitDyad \hspace{.1ex} \operatorname{sin} \vartheta \left({1 \!-\! \operatorname{cos} \vartheta}\right) \hspace{-0.1ex}
+ \hspace{-0.1ex} \bm{k} \hspace{-0.32ex}\times\hspace{-0.25ex}  \mathdotabove{\bm{k}} \hspace{-0.25ex}\times\hspace{-0.42ex} \UnitDyad \hspace{.32ex} ({1 \!-\! \operatorname{cos} \vartheta}) = $} \\
%
\shoveright{\hfill \hspace{11.2em}\scalebox{.96}[.96]{$= \bm{k} \hspace{-0.24ex}\times\hspace{-0.4ex} \UnitDyad \hspace{.4ex} \mathdotabove{\vartheta}
+ \mathdotabove{\bm{k}} \hspace{-0.24ex}\times\hspace{-0.4ex} \UnitDyad \hspace{.1ex} \operatorname{sin} \vartheta
+ \hspace{-0.1ex} \bm{k} \hspace{-0.2ex}\times\hspace{-0.2ex}  \mathdotabove{\bm{k}} \hspace{-0.2ex}\times\hspace{-0.4ex} \UnitDyad \hspace{.32ex} ({1 \hspace{-0.2ex} - \hspace{-0.2ex} \operatorname{cos} \vartheta})
\hspace{.1ex} .
$}}
\end{multline*}

Этот результат, подставленный в~определение~\eqref{angularvelocityvector} псевдо\-вектора~$\bm{\omega}$, даёт

\nopagebreak\vspace{-0.5em}\begin{equation}
\bm{\omega} = \bm{k} \hspace{.2ex} \mathdotabove{\vartheta}
+ \mathdotabove{\bm{k}} \operatorname{sin} \vartheta
+ \bm{k} \hspace{-0.1ex}\times\hspace{-0.1ex} \mathdotabove{\bm{k}} \left( 1 - \operatorname{cos} \vartheta \right) \hspace{-0.4ex}.
\end{equation}

\vspace{-0.3em}\noindent
Вектор~$\bm{\omega}$ получился разложенным по~трём взаимно ортогональным направлениям\:--- $\bm{k}$, $\mathdotabove{\bm{k}}$ и~${\bm{k} \hspace{-0.1ex}\times\hspace{-0.1ex} \mathdotabove{\bm{k}}}$. При~неподвижной оси поворота ${\mathdotabove{\bm{k}} = \bm{0} \,\Rightarrow\, \bm{\omega} = \bm{k} \hspace{.1ex} \mathdotabove{\vartheta}}$.

Ещё одно представление~$\bm{\omega}$ связано с~компонентами тензора поворота~\eqref{componentsofrotationtensor}. Поскольку ${\bm{P} = \cosinematrix{\!j\mathcircabove{i}} \hspace{.4ex} \mathcircabove{\bm{e}}_i \mathcircabove{\bm{e}}_j}$, ${\bm{P}^{\T} \hspace{-0.32ex} = \cosinematrix{\hspace{-0.2ex}i\mathcircabove{j}} \hspace{.4ex} \mathcircabove{\bm{e}}_i \mathcircabove{\bm{e}}_j}$, а~векторы начального базиса~${\mathcircabove{\bm{e}}_i}$ неподвижны (со~временем не~меняются), то
\nopagebreak\vspace{.25em}\[ \mathdotabove{\bm{P}} = \cosinematrixdotted{\!j\mathcircabove{i}} \hspace{.4ex} \mathcircabove{\bm{e}}_i \mathcircabove{\bm{e}}_j
\hspace{.1ex} , \:\,
\mathdotabove{\bm{P}} \dotp \bm{P}^{\T} \hspace{-0.32ex} = \hspace{.1ex} \cosinematrixdotted{\hspace{-0.4ex}n\mathcircabove{i}} \hspace{.4ex} \cosinematrix{\hspace{-0.4ex}n\mathcircabove{j}} \hspace{.4ex} \mathcircabove{\bm{e}}_i \mathcircabove{\bm{e}}_j
\hspace{.1ex}, \]
\nopagebreak\vspace{-0.64em}\begin{equation}
\bm{\omega} = - \hspace{.1ex} \smalldisplaystyleonehalf \hspace{.4ex} \cosinematrixdotted{\hspace{-0.4ex}n\mathcircabove{i}} \hspace{.4ex} \cosinematrix{\hspace{-0.4ex}n\mathcircabove{j}} \hspace{.5ex} \mathcircabove{\bm{e}}_i \hspace{-0.3ex}\times\hspace{-0.3ex} \mathcircabove{\bm{e}}_j \hspace{-0.1ex} =
\smalldisplaystyleonehalf \hspace{.2ex} \permutationsparitysymbols{j\hspace{-0.06ex}ik} \hspace{.32ex} \cosinematrix{\hspace{-0.4ex}n\mathcircabove{j}} \hspace{.4ex} \cosinematrixdotted{\hspace{-0.4ex}n\mathcircabove{i}} \hspace{.4ex} \mathcircabove{\bm{e}}_k
\hspace{.2ex} .
\end{equation}

\vspace{-0.25em}
Отметим и~формулы
\nopagebreak\vspace{.16em}\begin{equation}\label{angularvelocityandbasisvectors}
\begin{array}{c}
\eqref{angularvelocityvector} \hspace{.32ex} \Rightarrow\:
\mathdotabove{\bm{e}}_i \mathcircabove{\bm{e}}_i \hspace{-0.1ex} = \bm{\omega} \times \hspace{-0.1ex} \bm{e}_i \mathcircabove{\bm{e}}_i \:\Rightarrow\:
\mathdotabove{\bm{e}}_i = \bm{\omega} \times \bm{e}_i \hspace{.12ex},
\\[.32em]
%
\eqref{angularvelocityvector} \hspace{.32ex} \Rightarrow\:
\bm{\omega} = \hspace{-0.1ex} - \hspace{.16ex} \smalldisplaystyleonehalf \hspace{-.2ex} \left( \mathdotabove{\bm{e}}_i \mathcircabove{\bm{e}}_i \hspace{-0.1ex} \dotp \mathcircabove{\bm{e}}_j \bm{e}_j \right)_{\hspace{-0.25ex}\Xcompanion} \hspace{-0.32ex}
= \hspace{-0.1ex} - \hspace{.16ex} \smalldisplaystyleonehalf \hspace{-.2ex} \left( \mathdotabove{\bm{e}}_i \bm{e}_i \hspace{.1ex} \right)_{\hspace{-0.1ex}\Xcompanion} \hspace{-0.25ex}
= \smalldisplaystyleonehalf \hspace{.4ex} \bm{e}_i \hspace{-0.16ex} \times \hspace{-0.1ex} \mathdotabove{\bm{e}}_i
\hspace{.12ex} .
\end{array}
\end{equation}


\textcolor{magenta}{Не всё то вектор, что имеет величину и направление.
Поворот тела вокруг оси представляет, казалось~бы, вектор:
его численное значение равно углу поворота, а~направление совпадает с~направлением оси вращения.}
Однако, повороты не~складываются как векторы%
\footnote{Когда углы поворота не~бесконечно-м\'{а}лые.}%
\hbox{\hspace{-0.5ex}.}

На~с\'{а}мом~же деле последовательные повороты не~складываются, а~умножаются.

Можно~ли складывать угловые скорости?\:--- Да, ведь угол поворота в~$\mathdotabove{\vartheta}$ бесконечномалый.\:--- Но только при вращении вокруг неподвижной оси?


...



Варьируя тождество~\eqref{orthogonalityofrotationtensor},
получим ${\variation{\rotationtensor} \hspace{-0.2ex} \dotp \rotationtensor^{\T} \hspace{-0.2ex} = - \hspace{.2ex} \rotationtensor \dotp \variation{\rotationtensor}^{\T}\!}$.
Этот тензор антисимметричен, и~потому выражается через свой сопутствующий вектор~${\varvector{o}}$ как~${\variation{\rotationtensor} \hspace{-0.1ex} \dotp \rotationtensor^{\T} \hspace{-0.3ex} = \varvector{o} \hspace{-0.2ex} \times \hspace{-0.2ex} \UnitDyad}$.
Приходим к~соотношениям

\nopagebreak\vspace{-0.5em}\begin{equation}
\variation{\rotationtensor} \hspace{-0.1ex} = \varvector{o} \hspace{-0.1ex} \times \hspace{-0.1ex} \rotationtensor , \:\:
\varvector{o} = - \hspace{.2ex} \scalebox{.93}{$ \displaystyle\onehalf $} \hspace{-0.1ex} \Bigl( \hspace{-0.1ex} \variation{\rotationtensor} \hspace{-0.1ex} \dotp \rotationtensor^{\T} \Bigr)_{\hspace{-0.25em}\Xcompanion}
\hspace{-0.1ex} ,
\end{equation}

\vspace{-0.5em}\noindent

аналогичным~\eqref{angularvelocityvector}.
Вектор бесконечно малого поворота~${\varvector{o}}$
это не~\inquotesx{вариация $\bm{\mathrm{o}}$}[,]
но единый символ
(в~отличие от~${\variation{\rotationtensor}}$).


Малый поворот определяется вектором~${\varvector{o}}$,
но конечный поворот тоже возможно представить как вектор.


...


\end{otherlanguage}
