\en{\section{Tensor functions}}

\ru{\section{Тензорные функции}}

\label{section:tensorfunctions}

\noindent
\en{In~the~concept of~function}\ru{В~представлении о~функции}~${y \narroweq \hspace{-0.15ex} f(x)}$
\en{as of mapping (morphism)}\ru{как отображении (морфизме)} ${\smash{f \hspace{-0.2ex}\colon x \mapsto \hspace{-0.16ex} y}}$,
\en{an~input~(argument)}\ru{прообраз~(аргумент)}~$x$ \en{and an~output~(result)}\ru{и~образ~(результат)}~$y$ \en{may be tensors of any complexities}\ru{могут быть тензорами любых сложностей}.

\en{Consider}\ru{Рассмотрим} \en{at~least}\ru{хотя~бы} \en{a~scalar function}\ru{скалярную функцию} \en{of~a~bivalent tensor}\ru{двухвалентного тензора}~${\varphi \narroweq \varphi(\bm{B}\hspace{.1ex})}$.
\en{Examples}\ru{Примеры}\ru{\:---}\en{ are} ${\bm{B} \hspace{-0.2ex} \dotdotp \hspace{-0.25ex} \scalebox{1.1}[1]{$\bm{\mathit{\Phi}}$}}$ (\en{or}\ru{или}~${\bm{p} \dotp \hspace{-0.2ex} \bm{B} \hspace{-0.16ex} \dotp \hspace{-0.1ex} \bm{q}}$) \en{and}\ru{и}~${\bm{B} \hspace{-0.2ex} \dotdotp \hspace{-0.2ex} \bm{B}}$.
\en{Then}\ru{Тогда} \en{in~each basis}\ru{в~каждом базисе}~${\bm{a}_i}$ \en{paired with}\ru{в~паре с}~\en{cobasis}\ru{кобазисом}~${\bm{a}^{\hspace{-0.1ex}i}}$ \en{we have}\ru{имеем}
\en{function}\ru{функцию}~${\varphi(\somebivalenttensorcomponents{i\hspace{-0.1ex}j})}$ \en{of~nine numeric arguments}\ru{девяти числовых аргументов}\:---
\en{components}\ru{компонент}~$\somebivalenttensorcomponents{i\hspace{-0.1ex}j}$ \en{of~tensor}\ru{тензора}~$\bm{B}$.
\en{For example}\ru{Для примера}

\nopagebreak\vspace{-0.2em}\begin{equation*}
\varphi(\bm{B}\hspace{.1ex}) \hspace{-0.2ex}
= \bm{B} \hspace{-0.2ex} \dotdotp \hspace{-0.28ex} \scalebox{1.1}[1]{$\bm{\mathit{\Phi}}$} \hspace{-0.1ex}
= \hspace{-0.1ex} \somebivalenttensorcomponents{i\hspace{-0.1ex}j} \hspace{.16ex} \bm{a}^{\hspace{-0.1ex}i} \hspace{-0.16ex} \bm{a}^j \hspace{-0.3ex} \dotdotp \bm{a}_m \bm{a}_n \hspace{-0.2ex} \scalebox{1.2}[1]{$\mathit{\Phi}$}^{mn} \hspace{-0.25ex}
= \hspace{-0.1ex} \somebivalenttensorcomponents{i\hspace{-0.1ex}j} \hspace{-0.15ex} \scalebox{1.2}[1]{$\mathit{\Phi}$}^{\hspace{.1ex}j\hspace{-0.06ex}i} \hspace{-0.25ex}
= \varphi(\somebivalenttensorcomponents{i\hspace{-0.1ex}j})
\hspace{.1ex} .
\end{equation*}

\vspace{-0.25em} \noindent
\en{With any transition}\ru{С~любым переходом} \en{to~a~new basis,}\ru{к~новому базису} \en{the~result}\ru{результат} \en{doesn’t change}\ru{не~меняется}:
${\varphi(\somebivalenttensorcomponents{i\hspace{-0.1ex}j}) \hspace{-0.2ex} = \varphi(B\hspace{.16ex}'_{\hspace{-0.32ex}i\hspace{-0.1ex}j}) \hspace{-0.2ex} = \varphi(\bm{B}\hspace{.1ex})}$.

\en{Differentiation of}\ru{Дифференцирование}~${\varphi(\bm{B}\hspace{.1ex})}$ \en{looks like}\ru{выглядит как}

\nopagebreak\vspace{-0.2em}\begin{equation}
d \hspace{.1ex} \varphi
= \displaystyle \frac{\partial \hspace{.1ex} \varphi}{\partial \hspace{-0.2ex} \somebivalenttensorcomponents{i\hspace{-0.1ex}j}} \hspace{.2ex} d \somebivalenttensorcomponents{i\hspace{-0.1ex}j} \hspace{-0.2ex}
= \displaystyle \frac{\partial \hspace{.1ex} \varphi}{\partial \hspace{-0.1ex} \bm{B}} \dotdotp d \bm{B}^{\T}
\hspace{-0.25ex} .
\end{equation}

\en{\vspace{-0.15em}}\ru{\vspace{-0.25em}}\noindent
\en{Tensor}\ru{Тензор}~${\scalebox{0.98}[1]{$\raisemath{.16em}{\scalebox{0.92}[0.92]{$\partial \hspace{.15ex} \varphi$}} \hspace{-0.1ex} / \hspace{-0.1ex} \raisemath{-0.32em}{\scalebox{0.92}[0.92]{$\partial \hspace{-0.1ex} \bm{B}$}}\hspace{.1ex}$}}$
\en{is called}\ru{называется} \en{the~derivative}\ru{производной} \en{of~function}\ru{функции}~$\varphi$ \en{by~argument}\ru{по~аргументу}~${\hspace{-0.15ex}\bm{B}\hspace{.1ex}}$;
${d \bm{B}}$\en{~is}\ru{\:---} \en{the~differential}\ru{дифференциал} \en{of~tensor}\ru{тензора}~$\bm{B}$,
${d \bm{B} \hspace{-0.1ex} = d \somebivalenttensorcomponents{i\hspace{-0.1ex}j} \hspace{.16ex} \bm{a}^{\hspace{-0.1ex}i} \hspace{-0.16ex} \bm{a}^j}$;
${\smash{\scalebox{0.98}[1]{$\raisemath{.16em}{\scalebox{0.9}{$\partial \hspace{.15ex} \varphi$}} \hspace{-0.1ex} / \hspace{-0.2ex} \raisemath{-0.32em}{\scalebox{0.9}{$\partial \hspace{-0.1ex} \somebivalenttensorcomponents{i\hspace{-0.1ex}j}$}}$}}}$\ru{\:---}\en{~are} \en{components}\ru{компоненты}~(\en{contra\-variant ones}\ru{контра\-вариант\-ные}) \en{of~\,}${\smash{\scalebox{0.98}[1]{$\raisemath{.16em}{\scalebox{0.92}[0.92]{$\partial \hspace{.15ex} \varphi$}} \hspace{-0.1ex} / \hspace{-0.2ex} \raisemath{-0.32em}{\scalebox{0.92}{$\partial \hspace{-0.1ex} \bm{B}$}}\hspace{.2ex}$}}}$

\nopagebreak\vspace{-0.1em}\begin{equation*}
\bm{a}^{\hspace{-0.1ex}i} \hspace{-0.15ex} \dotp \scalebox{0.92}{$\displaystyle\frac{\partial \hspace{.1ex} \varphi}{\partial \hspace{-0.1ex} \bm{B}}$} \dotp \bm{a}^j \hspace{-0.2ex}
=
\scalebox{0.92}{$\displaystyle\frac{\partial \hspace{.1ex} \varphi}{\partial \hspace{-0.1ex} \bm{B}}$} \dotdotp \bm{a}^j \hspace{-0.2ex} \bm{a}^{\hspace{-0.1ex}i} \hspace{-0.2ex}
=
\scalebox{0.92}{$\displaystyle\frac{\partial \hspace{.1ex} \varphi}{\partial \hspace{-0.15ex} \somebivalenttensorcomponents{i\hspace{-0.1ex}j}}$}
\;\Leftrightarrow\;
\scalebox{0.92}{$\displaystyle\frac{\partial \hspace{.1ex} \varphi}{\partial \hspace{-0.1ex} \bm{B}}$}
=
\scalebox{0.92}{$\displaystyle\frac{\partial \hspace{.1ex} \varphi}{\partial \hspace{-0.15ex} \somebivalenttensorcomponents{i\hspace{-0.1ex}j}}$} \hspace{.25ex} \bm{a}_i \bm{a}_{\hspace{-0.1ex}j}
\hspace{.1ex} .
\end{equation*}

...

\nopagebreak\begin{equation*}\begin{array}{c}
\varphi(\bm{B}\hspace{.1ex}) \hspace{-0.2ex}
= \bm{B} \hspace{-0.2ex} \dotdotp \hspace{-0.25ex} \scalebox{1.1}[1]{$\bm{\mathit{\Phi}}$}
\\[.2em]
%
d \hspace{.1ex} \varphi
= d \hspace{.2ex} ( \bm{B} \hspace{-0.2ex} \dotdotp \hspace{-0.25ex} \scalebox{1.1}[1]{$\bm{\mathit{\Phi}}$} \hspace{.1ex} ) \hspace{-0.2ex}
= d \bm{B} \hspace{-0.2ex} \dotdotp \hspace{-0.25ex} \scalebox{1.1}[1]{$\bm{\mathit{\Phi}}$}
= \hspace{-0.1ex} \scalebox{1.1}[1]{$\bm{\mathit{\Phi}}$} \hspace{-0.12ex} \dotdotp d \bm{B}
= \hspace{-0.1ex} \scalebox{1.1}[1]{$\bm{\mathit{\Phi}}$}^{\T} \hspace{-0.4ex} \dotdotp d \bm{B}^{\T}
\\[.2em]
%
d \hspace{.1ex} \varphi
= \scalebox{0.92}{$\displaystyle \frac{\partial \hspace{.1ex} \varphi}{\partial \hspace{-0.1ex} \bm{B}} \dotdotp d \bm{B}^{\T}$} \hspace{-0.3ex} ,
\:\:
\scalebox{0.92}{$\displaystyle \frac{\partial \hspace{-0.1ex} \left( \bm{B} \hspace{-0.2ex} \dotdotp \hspace{-0.25ex} \scalebox{1.1}[1]{$\bm{\mathit{\Phi}}$} \right)}{\partial \hspace{-0.1ex} \bm{B}}$}
= \hspace{-0.1ex} \scalebox{1.1}[1]{$\bm{\mathit{\Phi}}$}^{\T}
\end{array}\end{equation*}

${\bm{p} \dotp \hspace{-0.2ex} \bm{B} \hspace{-0.16ex} \dotp \hspace{-0.1ex} \bm{q} \hspace{.1ex} =
\hspace{-0.1ex} \bm{B} \hspace{-0.16ex} \dotdotp \hspace{-0.1ex} \bm{q} \bm{p}}$

\nopagebreak\begin{equation*}
\scalebox{0.92}[0.92]{$\displaystyle \frac{\partial \hspace{-0.1ex} \left( \hspace{.1ex} \bm{p} \dotp \hspace{-0.2ex} \bm{B} \hspace{-0.16ex} \dotp \hspace{-0.1ex} \bm{q} \right)}{\partial \hspace{-0.1ex} \bm{B}}$}
= \bm{p} \bm{q}
\end{equation*}

...

\nopagebreak\begin{equation*}\begin{array}{c}
\varphi(\bm{B}\hspace{.1ex}) \hspace{-0.2ex}
= \bm{B} \hspace{-0.2ex} \dotdotp \hspace{-0.2ex} \bm{B}
\\[.2em]
%
d \hspace{.1ex} \varphi
= d \hspace{.2ex} ( \bm{B} \hspace{-0.2ex} \dotdotp \hspace{-0.2ex} \bm{B} \hspace{.1ex} ) \hspace{-0.2ex}
= d ...
\end{array}\end{equation*}


...


\begin{otherlanguage}{russian}

Но согласно опять\hbox{-}таки~\eqref{cayley-hamilton equation}
${\hspace{-0.1ex} -\bm{B}^{2} \hspace{-0.2ex} + \mathrm{I}\hspace{.16ex} \bm{B} \hspace{-0.1ex} - \mathrm{II}\hspace{.16ex} \UnitDyad + \mathrm{III}\hspace{.16ex} \bm{B}^{\expminusone} \hspace{-0.25ex} = \zerobivalent}$,
поэтому


....


Скалярная функция~${\varphi(\bm{B})}$ называется изотропной,
если она не~чувствительна к~повороту аргумента:
\nopagebreak\vspace{.1em}\begin{equation*}
\varphi(\bm{B}) \hspace{-0.12ex} = \varphi ( \rotationtensor \narrowdotp \smash{\mathcircabove{\bm{B}}} \narrowdotp \hspace{.15ex} \rotationtensor^{\hspace{-0.1ex}\T} ) \hspace{-0.2ex} = \varphi(\smash{\mathcircabove{\bm{B}}}) \;\;\,
\forall \rotationtensor \hspace{-0.2ex} = \hspace{-0.1ex} \bm{a}_i \hspace{.1ex} \mathcircabove{\bm{a}}^i \hspace{-0.25ex} = \hspace{-0.1ex} \bm{a}^{\hspace{-0.2ex}i} \mathcircabove{\bm{a}}_i \hspace{-0.16ex} = \hspace{-0.1ex} \rotationtensor^{\hspace{-0.1ex}\expminusT}
\end{equation*}
\par\vspace{-0.25em}\noindent
для~любого ортогонального тензора~$\rotationtensor$ (тензора поворота, \sectionref{section:rotationtensor}).

Симметричный тензор~${\bm{B}^{\mathsf{\hspace{.1ex}S}}}$ полностью определяется тройкой инвариантов и~угловой ориентацией собственных осей (они~же взаимно ортогональны, \sectionref{section:eigenvectorseigenvalues}).
Ясно, что изотропная функция~${\varphi(\bm{B}^{\mathsf{\hspace{.1ex}S}})}$ симметричного аргумента является функцией, входы-аргументы которой\:--- только инварианты
${ \anyfirstinvariantof{\bm{B}^{\mathsf{\hspace{.1ex}S}}} }$,
${ \anysecondinvariantof{\bm{B}^{\mathsf{\hspace{.1ex}S}}} }$,
${ \anythirdinvariantof{\bm{B}^{\mathsf{\hspace{.1ex}S}}} }$.
Дифференцируется такая функция согласно~\eqref{fonvccbnmxghjsxmnxjsdjhga},
где транспонирование излишне.

\end{otherlanguage}
