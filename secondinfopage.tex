\begin{minipage}[b]{0.92\linewidth}%
{

\begin{tikzpicture}[overlay, remember picture]

	\def\widthofsample{2.5mm}
	\def\heightofsample{2.5mm}

	\fill [remember picture, overlay, fill=black]
		($ (current page.south east) + (-\widthofsample,0) $)
		rectangle ++(\widthofsample, \heightofsample);

	\fill [remember picture, overlay, fill=cyan]
		($ (current page.south east) + (-\widthofsample,\heightofsample) $)
		rectangle ++(\widthofsample, \heightofsample);
	\fill [remember picture, overlay, fill=magenta]
		($ (current page.south east) + (-\widthofsample,2*\heightofsample) $)
		rectangle ++(\widthofsample, \heightofsample);
	\fill [remember picture, overlay, fill=yellow]
		($ (current page.south east) + (-\widthofsample,3*\heightofsample) $)
		rectangle ++(\widthofsample, \heightofsample);

	\fill [remember picture, overlay, fill=red]
		($ (current page.south east) + (-\widthofsample,5*\heightofsample) $)
		rectangle ++(\widthofsample, \heightofsample);
	\fill [remember picture, overlay, fill=green]
		($ (current page.south east) + (-\widthofsample,6*\heightofsample) $)
		rectangle ++(\widthofsample, \heightofsample);
	\fill [remember picture, overlay, fill=blue]
		($ (current page.south east) + (-\widthofsample,7*\heightofsample) $)
		rectangle ++(\widthofsample, \heightofsample);

	\foreach \graysample in {1, 2, ..., 10}
	{
		\pgfmathsetmacro{\currentgray}{100 - (10 * (\graysample - 1))}
		\fill [remember picture, overlay, fill=black!\currentgray]
			($ (current page.north east) + (-\widthofsample,-\graysample*\heightofsample) $)
			rectangle ++(\widthofsample, \heightofsample);
	}

\end{tikzpicture}

\fontsize{10}{12}\selectfont

\begin{otherlanguage}{russian}

{\footnotesize%
\textcolor{black}{УДК\;539.3~\emph{Механика деформируемых тел.\:Упругость.\;Деформации}}\\
\textcolor{black!50}{УДК\;539~\emph{Строение материи}}\\
\textcolor{black!40}{УДК\;53~\emph{Физика}}\\
\textcolor{black!30}{УДК\;5~\emph{Математика.\:Естественные науки}}%
\par}

\end{otherlanguage}

\vspace{\baselineskip}

\begin{otherlanguage}{english}

{\footnotesize%
\textcolor{black}{UDC\;539.3~\emph{Elasticity.\:Deformation.\:Mechanics of elastic solids}}\\
\textcolor{black!50}{UDC\;539~\emph{Physical nature of matter}}\\
\textcolor{black!40}{UDC\;53~\emph{Physics}}\\
\textcolor{black!30}{UDC\;5~\emph{Mathematics.\:Natural sciences}}%
\par}

\end{otherlanguage}

\vspace{0.25\paperheight}

\begin{otherlanguage}{russian}

%%\hspace{12pt}
Представлены все модели упругих сред: нелинейные и~линейные, микрополярные и~классические безмоментные; трёх\-мерные, дву\-мерные~(оболочки и~пластины), одно\-мерные~(стержни, в~том~числе тонко\-стен\-ные). Для термо\-упругости и~магнито\-упругости дана сводка законов термо\-динамики и~электро\-динамики. Изложены основы динамики~--- колебания, волны и~устойчивость. Описаны теории дефектов, трещин, композитов и~периодических структур.

\end{otherlanguage}

\vspace{1.2\baselineskip}

\begin{otherlanguage}{english}

%%\hspace{12pt}
All models of elastic continua are presented: nonlinear and~\hbox{linear}, micropolar and classical momentless; three-di\-men\-sion\-al, two-di\-men\-sion\-al~(shells and~plates), one-di\-men\-sion\-al~(rods, including thin\hbox{-}walled ones). For thermo\-elasticity and~magneto\-elasticity, summary of~laws of~thermo\-dynamics and~electro\-dynamics is given. Fundamentals of dynamics~--- oscillations, waves and~stability~--- are explained. Theories of defects, fractures, composites and periodic structures are described.

%% micropolar elasticity, couple\hbox{-}stress elasticity, asymmetric (non\hbox{-}symmetric) elasticity, Cosserat elasticity

\end{otherlanguage}

}
\end{minipage}

\thispagestyle{empty}

\newpage
