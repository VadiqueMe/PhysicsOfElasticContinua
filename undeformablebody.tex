\begin{tikzpicture}[scale=.63]

\def\angleofrotation{44}

\def\Opointx{-1.65}
\def\Opointy{-1.05}
\def\Oinitialpointx{-6} %-5.8
\def\Oinitialpointy{-2.3} %-2.3

\def\bodypointx{-3.5} %-2
\def\bodypointy{2.5} %1.5

\newcommand\drawnotrotatedbasis{
	\draw [line width=1pt, black!50,
		style=double, double distance=0.5mm,
		rotate around={120:(\Opointx, \Opointy)},
		-{Triangle[open, angle=60:3.2mm]}]
		(\Opointx, \Opointy) -- ++(0, 1.6) ;
	\draw [line width=1pt, black!50,
	style=double, double distance=0.5mm, rotate around={-120:(\Opointx, \Opointy)},
	-{Triangle[open, angle=60:3.2mm]}]
		(\Opointx, \Opointy) -- ++(0, 1.6) ;
	\draw [line width=1pt, black!50,
		style=double, double distance=0.5mm, -{Triangle[open, angle=60:3.2mm]}]
		(\Opointx, \Opointy) -- ++(0, 1.6)
		node [pos=.93, above, inner sep=0pt, outer sep=3.5pt]
		{$ \widetilde{\bm{e}}_i $} ;
}

%%\newcommand\setundeformablebody{
%%	\coordinate (point0) at (-4.3, 2.5);
%%	\coordinate (point1) at (-3.1, 3.2);
%%	\coordinate (point2) at (-2, 2.4);
%%	\coordinate (point3) at (-0.4, 1.6);
%%	\coordinate (point4) at (0.5, 0);
%%	\coordinate (point5) at (0, -2);
%%	\coordinate (point6) at (-1.5, -3);
%%	\coordinate (point7) at (-3, -2.2);
%%	\coordinate (point8) at (-3.5, -0.5);
%%	\coordinate (point9) at (-4.5, 1);
%%}

\newcommand\drawnotrotatedundeformablebody{
	\begin{scope}[rotate around={-\angleofrotation:(\Opointx, \Opointy)}]
	\draw [line width=1pt, black!50, opacity=50]
		plot [smooth cycle, tension=0.8] coordinates {
			(-4.3, 2.5) (-3.1, 3.2) (-2, 2.4) (-0.4, 1.6) (0.5, 0)
			(0, -2) (-1.5, -3) (-3, -2.2) (-3.5, -0.5) (-4.5, 1)
		};
	\end{scope}
}

\newcommand\drawrotatedundeformablebody{
	\draw [line width=1.6pt, black]
		plot [smooth cycle, tension=0.8] coordinates {
			(-4.3, 2.5) (-3.1, 3.2) (-2, 2.4) (-0.4, 1.6) (0.5, 0)
			(0, -2) (-1.5, -3) (-3, -2.2) (-3.5, -0.5) (-4.5, 1)
			%% (point0) (point1) (point2) (point3) (point4)
			%% (point5) (point6) (point7) (point8) (point9)
		};
}

\newcommand\drawnotrotatedtorotated{
	\tkzDefPoint(\bodypointx, \bodypointy){bodypointnotrotated}
	\begin{scope}[rotate around={-\angleofrotation:(\Opointx, \Opointy)}]
	\tkzDefPoint(\Opointx, \Opointy){centerpoint}
	\tkzDefPoint(\bodypointx, \bodypointy){bodypoint}
	\tkzDrawArc[line width=1pt, color=black!50, opacity=50](centerpoint,bodypoint)(bodypointnotrotated) ;

	\path (\bodypointx, \bodypointy) circle (2mm) node [shape=circle, inner sep=.9mm, outer sep=0] (previousbodypoint) {};

	\draw [line width=1pt, black!50, opacity=50, -{Stealth[round, length=4.5mm, width=2.8mm]}]
		(\Opointx, \Opointy) -- (previousbodypoint)
		node [pos=0.57, color=black!50, opacity=99, right, inner sep=0pt, outer sep=3.5pt]
		{$ \widetilde{\bm{x}} $} ;

	\fill [white] (\bodypointx, \bodypointy) circle (2mm) ;
	\draw [line width=1pt, color=black!50, opacity=50] (\bodypointx, \bodypointy) circle (2mm) ;
	\end{scope}
}

\newcommand\drawfirstversionvectors{
	\draw [line width=1.6pt, black, fill=white] (\bodypointx, \bodypointy) circle (2mm)
		node [shape=circle, inner sep=0.9mm, outer sep=0] (pointcirc) {} ;

	\draw [line width=1.6pt, black, -{Stealth[round, length=5mm, width=3.6mm]}] (\Oinitialpointx, \Oinitialpointy) -- (pointcirc)
		node [pos=0.5, above left, inner sep=0pt, outer sep=1.5pt] {$ \locationvector $} ;

	\path (\Opointx, \Opointy) circle (1.6mm) node [shape=circle, inner sep=.64mm, outer sep=0] (Ocirc) {} ;

	\draw [line width=1.6pt, blue, -{Stealth[round, length=5mm, width=3.6mm]}] (\Oinitialpointx, \Oinitialpointy) -- (Ocirc)
		node [pos=0.48, below, inner sep=0pt, outer sep=5pt] {$ \positionofthepole $};

	\draw [line width=1.6pt, black, -{Stealth[round, length=5mm, width=3.6mm]}] (\Opointx, \Opointy) -- (pointcirc)
		node [pos=0.63, left, inner sep=2.5pt, outer sep=3.3pt] {$ \bm{x} $} ;
}

\newcommand\drawsecondversionvectors{
	\draw [line width=1.6pt, black, fill=white] (\bodypointx, \bodypointy) circle (2mm)
		node [shape=circle, inner sep=0.9mm, outer sep=0] (pointcirc) {} ;

	\draw [line width=1.6pt, black, -{Stealth[round, length=5mm, width=3.6mm]}] (\Oinitialpointx, \Oinitialpointy) -- (pointcirc)
		node [pos=0.5, above left, inner sep=0pt, outer sep=1.5pt] {$ \locationvector $} ;

	\path (\Oinitialpointx, \Oinitialpointy) circle (1.6mm) node [shape=circle, inner sep=.64mm, outer sep=0] (Oinitialcirc) {} ;

	\draw [line width=1.6pt, blue, -{Stealth[round, length=5mm, width=3.6mm]}] (\Opointx, \Opointy) -- (Oinitialcirc)
		node [pos=0.6, below, inner sep=0pt, outer sep=4.4pt] {$ - \hspace{.2ex} \positionofthepole $};

	\draw [line width=1.6pt, black, -{Stealth[round, length=5mm, width=3.6mm]}] (\Opointx, \Opointy) -- (pointcirc)
		node [pos=0.63, left, inner sep=2.5pt, outer sep=3.3pt] {$ \bm{x} $} ;
}

\newcommand\drawbodybasis{
	\draw [line width=1pt, blue, rotate around={{\angleofrotation + 120}:(\Opointx, \Opointy)},
		style=double, double distance=0.5mm, -{Triangle[open, angle=60:3.2mm]}]
		(\Opointx, \Opointy) -- ++(0, 1.6);
	\draw [line width=1pt, blue, rotate around={{\angleofrotation - 120}:(\Opointx, \Opointy)},
		style=double, double distance=0.5mm, -{Triangle[open, angle=60:3.2mm]}]
		(\Opointx, \Opointy) -- ++(0, 1.6);
 	\draw [line width=1pt, blue, rotate around={\angleofrotation:(\Opointx, \Opointy)},
		style=double, double distance=0.5mm, -{Triangle[open, angle=60:3.2mm]}]
		(\Opointx, \Opointy) -- ++(0, 1.6);

	\draw [line width=1pt, blue, fill=white] (\Opointx, \Opointy) circle (1.6mm)
		node [below right, inner sep=0pt, outer sep=3.5pt, xshift=-.7mm, yshift=-2.5mm] {$ \bm{e}_i $} ;
}

\newcommand\drawhatchlines{
	\def\hatchlength{.3}
	\def\loopfirst{.55}
	\def\looplast{1.15}
	\pgfmathsetmacro\loopstep{(\looplast - \loopfirst) / 2}
	\pgfmathsetmacro\loopsecond{\loopfirst + \loopstep}
	\foreach \econnection in {\loopfirst, \loopsecond, ..., \looplast} {
		\draw [line width=.5pt, color=blue]
			($ (\Oinitialpointx, \Oinitialpointy) + (0, \econnection) $) -- ++(-\hatchlength, -\hatchlength) ;
		\draw [line width=.5pt, color=blue, rotate around={120:(\Oinitialpointx, \Oinitialpointy)}]
			($ (\Oinitialpointx, \Oinitialpointy) + (0, \econnection) $) -- ++(-\hatchlength, -\hatchlength) ;
		\draw [line width=.5pt, color=blue, rotate around={-120:(\Oinitialpointx, \Oinitialpointy)}]
			($ (\Oinitialpointx, \Oinitialpointy) + (0, \econnection) $) -- ++(\hatchlength, -\hatchlength) ;
	}
}

\newcommand\drawabsolutebasis{
	\draw [line width=1pt, blue,
		style=double, double distance=0.5mm, rotate around={120:(\Oinitialpointx, \Oinitialpointy)}, -{Triangle[open, angle=60:3.2mm]}]
		(\Oinitialpointx, \Oinitialpointy) -- ++(0, 1.6);
	\draw [line width=1pt, blue,
		style=double, double distance=0.5mm, rotate around={-120:(\Oinitialpointx, \Oinitialpointy)}, -{Triangle[open, angle=60:3.2mm]}]
		(\Oinitialpointx, \Oinitialpointy) -- ++(0, 1.6);
 	\draw [line width=1pt, blue,
		style=double, double distance=0.5mm, -{Triangle[open, angle=60:3.2mm]}]
		(\Oinitialpointx, \Oinitialpointy) -- ++(0, 1.6);

	\draw [line width=1pt, blue, fill=white] (\Oinitialpointx, \Oinitialpointy) circle(1.6mm)
		node [anchor=north, inner sep=0pt, outer sep=8pt, yshift=-1.1mm, xshift=.33mm]
			{$ \mathcircabove{\bm{e}}_i $};
}

	%%draw undeformable body

	\drawnotrotatedbasis

	\drawnotrotatedundeformablebody

	\drawnotrotatedtorotated

	\drawrotatedundeformablebody

	\drawfirstversionvectors

	\drawbodybasis

	\drawhatchlines
	\drawabsolutebasis

\pgfmathsetmacro\textpositionx{.5 + \Oinitialpointx}
\pgfmathsetmacro\textpositiony{\Oinitialpointy + 4}

\node [anchor=east] at (\textpositionx, \textpositiony)
	{$ \locationvector = \positionofthepole + \bm{x} $} ;

%%\node [align=center] at (\textpositionx, \textpositiony)
%%	{$ \bm{x} = - \hspace{.2ex} \positionofthepole + \locationvector $} ;

\end{tikzpicture}

\begin{comment}
\makeatletter
\newcommand\xofcoordinate[2][center]{{%
	\pgfpointanchor{#2}{#1}%
	\pgfmathparse{\pgf@x/\pgf@xx}%
	\pgfmathprintnumber[precision=2]{\pgfmathresult}%
}}
\newcommand\yofcoordinate[2][center]{{%
	\pgfpointanchor{#2}{#1}%
	\pgfmathparse{\pgf@y/\pgf@yy}%
	\pgfmathprintnumber[precision=2]{\pgfmathresult}%
}}
\makeatother

\begin{tikzpicture}
	\coordinate (point0) at (-4.3, 2.5);
	\coordinate (point1) at (-3.1, 3.2);
	\coordinate (point2) at (-2, 2.4);
	\coordinate (point3) at (-0.4, 1.6);
	\coordinate (point4) at (0.5, 0);
	\coordinate (point5) at (0, -2);
	\coordinate (point6) at (-1.5, -3);
	\coordinate (point7) at (-3, -2.2);
	\coordinate (point8) at (-3.5, -0.5);
	\coordinate (point9) at (-4.5, 1);

	\draw [line width=1.2pt, red] plot [smooth cycle, tension=0.8] coordinates {
		(point0) (point1) (point2) (point3) (point4)
		(point5) (point6) (point7) (point8) (point9)
	};

	\newcommand\xyofcoordinate[1]{\xofcoordinate{#1},\,\yofcoordinate{#1}}

	\draw [black, fill=black] (point0) circle (1mm) node [anchor=south east] {\xyofcoordinate{point0}};
	\draw [black, fill=black] (point1) circle (1mm) node [anchor=south, outer sep=4pt] {\xyofcoordinate{point1}};
	\draw [black, fill=black] (point2) circle (1mm) node [anchor=south west] {\xyofcoordinate{point2}};
	\draw [black, fill=black] (point3) circle (1mm) node [anchor=south west] {\xyofcoordinate{point3}};
	\draw [black, fill=black] (point4) circle (1mm) node [anchor=south west] {\xyofcoordinate{point4}};

	\draw [black,fill=black] (point5) circle (1mm) node [anchor=north west, outer sep=1pt] {\xyofcoordinate{point5}};
	\draw [black,fill=black] (point6) circle (1mm) node [anchor=north, outer sep=4pt] {\xyofcoordinate{point6}};
	\draw [black,fill=black] (point7) circle (1mm) node [anchor=north east, outer sep=2pt] {\xyofcoordinate{point7}};
	\draw [black,fill=black] (point8) circle (1mm) node [anchor=east, outer sep=4pt] {\xyofcoordinate{point8}};
	\draw [black,fill=black] (point9) circle (1mm) node [anchor=east, outer sep=3pt] {\xyofcoordinate{point9}};
\end{tikzpicture}
\end{comment}

