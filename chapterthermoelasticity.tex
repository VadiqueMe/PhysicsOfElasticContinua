\en{\chapter{Thermoelasticity}}

\ru{\chapter{Термоупругость}}

\thispagestyle{empty}

\label{chapter:thermoelasticity}

\en{\section{First law of thermodynamics}}

\ru{\section{Первый закон термодинамики}}

\begin{otherlanguage}{russian}

{\small
The first law of thermodynamics is a~version of the law of~conservation of~energy, adapted for thermodynamic systems. The law of~conservation of~energy states that the total energy of an~isolated system is constant; energy can be transformed from one form to another, but can be neither created nor destroyed.
\par}

\lettrine[lines=2, findent=2pt, nindent=0pt]{Д}{о}~сих~пор мы ограничивались рамками механики и~не~рассматривали тепловые эффекты. Общеизвестно, однако, что изменение температуры вызывает деформацию тел. Температурные деформации и~обусловленные ими напряжения часто играют первостепенную роль и~могут приводить к~разрушению конструкций.

Принцип виртуальной работы, столь эффективный в~механике, не~имеет места в~термо\-механике~(хотя ниже мы увидим аналог этого принципа). Вводя в~рассмотрение температуру, можно опираться на~два закона термодинамики.

Первый закон, открытый ...

...

Тепло поступает в~объём двумя путями. Первый~--- теплопроводность, определяемая вектором потока тепла~$\bm{h}$. Предполагается, что через площ\'{а}дку

...



\end{otherlanguage}

\en{\section{Second law}}

\ru{\section{Второй закон}}

\begin{otherlanguage}{russian}

В~популярных курсах физики распространено следующее представление о~законах термодинамики: приращение энергии~$dE$ равно сумме работы внешних сил~$dA$ и~подведённого тепла~$dQ$. Величина~$dQ$ не~является полным дифференциалом, но~отношение~${\raisemath{.16em}{dQ}/\hspace{.1ex}\raisemath{-0.16em}{T}}$ становится таковым~--- дифференциалом энтропии~$dS$. Далее тепловые процессы делятся на ...

...



\end{otherlanguage}

\en{\section{Constitutive equations}}

\ru{\section{Определяющие уравнения}}

\begin{otherlanguage}{russian}

К~законам баланса импульса, момента импульса и~термодинамики необходимо добавить ...

...



\end{otherlanguage}

\en{\section{Heat equation}}

\ru{\section{Уравнение теплопроводности}}

\begin{otherlanguage}{russian}

{\small
The heat equation is a parabolic partial differential equation that describes the distribution of~heat (or variation in temperature) in a~given region over time.
\par}

В~курсах математической физики~\cite{tihonovsamarsky-mathphysicsequations} рассматривается уравнение теплопроводности

...



\end{otherlanguage}

\en{\section{Linear thermoelasticity}}

\ru{\section{Линейная термоупругость}}

\begin{otherlanguage}{russian}

Квадратичная аппроксимация свободной энергии наиболее естественна в~линейной теории

...



\section{Уравнения в перемещениях}

Полагая поле температуры известным

...



\section{Температурные напряжения}

Эти напряжения ст\'{о}ит рассмотреть детально, хотя они и~определяются очевидным образом полями перемещений и~температуры. При равновесии свободного тела без внешних нагрузок имеем

...



\section{Вариационные постановки}

Поскольку при данной\textcolor{red}{???(постоянной??)(конкретной??)} температуре уравнения термоупругости выглядят как в~механике

...




\vspace{8mm}
\hfill\begin{minipage}[b]{0.95\linewidth}
\fontsize{10}{12}\selectfont

\section*{\wordforbibliography}

В~формировании новых взглядов на~термодинамику сплошной среды велика роль C.\:Truesdell’а~\cite{truesdell-firstcourse}. Чёткое изложение основных законов ...

\end{minipage}

\end{otherlanguage}
