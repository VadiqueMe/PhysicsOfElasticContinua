\en{\section{Principle of virtual work (without Lagrange multipliers)}}

\ru{\section{Принцип виртуальной работы (без множителей Lagrange’а)}}

\label{section:virtualworkprinciple.elastic}

\en{According}\ru{Согласно}
\en{to the~principle of~virtual work}\ru{принципу виртуальной работы}
\en{for}\ru{для}
\en{some}\ru{некоего}
\en{finite}\ru{конечного}
\en{volume}\ru{объёма}
\en{of~a~continuous medium}\ru{сплошной среды} %%\en{of~a~continuum}\ru{\rucontinuum{}а}

\nopagebreak\ru{\vspace{-0.1em}}\begin{equation}\label{princlipleofvirtualwork.integral:nonlinearmomentlesscontinuum}
\integral\displaylimits_{\mathcal{V}} \hspace{-0.5ex} \Bigl( \massdensity \hspace{-0.1ex} \massloadvector \dotp \variation{\currentlocationvector} + \variation{\internalwork} \Bigr) \hspace{-0.1ex} d\mathcal{V}
+ \ointegral\displaylimits_{\mathclap{\mathcal{O}(\boundary \mathcal{V})}} \hspace{-0.2ex} \currentunitnormal \dotp \cauchystress \dotp \variation{\currentlocationvector} \hspace{.2ex} d\mathcal{O} = \hspace{.1ex} 0
\hspace{.1ex} .
\end{equation}

\vspace{-0.1em}\noindent
\en{Here}\ru{Здесь}
${\variation{\internalwork}}$\ru{\:---}\en{ is}
\en{the~work of~internal forces}\ru{работа внутренних сил}
\en{per volume unit}\ru{на~единицу объёма}
\en{in the~current configuration}\ru{в~текущей конфигурации},
%
$\massloadvector$\ru{\:---}\en{ is}
\en{the mass force}\ru{массовая сила}
(\en{including dynamics}\ru{включая динамику},
${\massloadvector \hspace{-0.2ex} \equiv \hspace{-0.2ex} \massloadvector_{\hspace{-0.25ex}*} \hspace{-0.3ex} - \mathdotdotabove{\currentlocationvector}\hspace{.25ex}}$),
%
${\bm{p} = \hspace{-0.1ex} \tractionvector{\currentunitnormal} \hspace{-0.1ex} = \currentunitnormal \dotp \cauchystress}$\ru{\:---}\en{ is}
\en{the surface force}\ru{поверхностная сила}.

\en{Applying the~divergence theorem to the~surface integral}\ru{Применяя к~поверхностному интегралу теорему о~дивергенции}, \en{using}\ru{используя}\footnote{%
${ \currentlocationvector^{i} \hspace{-0.25ex} \dotp \partial_i \hspace{-0.1ex} \bigl( \hspace{-0.1ex} \cauchystress \dotp \variation{\currentlocationvector} \hspace{-0.1ex} \bigr) \hspace{-0.25ex}
= \currentlocationvector^{i} \hspace{-0.25ex} \dotp \hspace{-0.2ex} \bigl( \partial_i \cauchystress \hspace{.1ex} \bigr) \hspace{-0.25ex} \dotp \variation{\currentlocationvector}
+ \currentlocationvector^{i} \hspace{-0.25ex} \dotp \hspace{-0.1ex} \cauchystress \dotp \partial_i \hspace{-0.1ex} \bigl( \variation{\currentlocationvector} \hspace{-0.1ex} \bigr)
\hspace{-0.1ex} , }$

\hspace*{\fill}
${ \currentlocationvector^{i} \hspace{-0.25ex} \dotp \hspace{-0.1ex} \cauchystress \dotp \partial_i \hspace{-0.1ex} \bigl( \variation{\currentlocationvector} \hspace{-0.1ex} \bigr) \hspace{-0.25ex}
= \cauchystress \dotdotp \partial_i \hspace{-0.1ex} \bigl( \variation{\currentlocationvector} \hspace{-0.1ex} \bigr) \currentlocationvector^{i} \hspace{-0.2ex}
= \cauchystress \dotdotp \hspace{-0.12ex} \bigl( \currentlocationvector^{i} \partial_i \hspace{.12ex} \variation{\currentlocationvector} \bigr)^{\raisemath{-0.1em}{\hspace{-0.4ex}\T}} }$%
}

\nopagebreak\vspace{-0.1em}\begin{equation*}
\boldnabla \hspace{-0.1ex} \dotp \hspace{-0.1ex} \left( \cauchystress \dotp \variation{\currentlocationvector} \right) \hspace{-0.1ex}
= \boldnabla \hspace{-0.1ex} \dotp \cauchystress \dotp \variation{\currentlocationvector}
+ \cauchystress \dotdotp \hspace{-0.12ex} \boldnabla \hspace{.1ex} \variation{\currentlocationvector}^{\T}
\end{equation*}

\vspace{-0.4em}\noindent
\en{and the~randomness of}\ru{и~случайность}~${\mathcal{V}\hspace{-0.2ex}}$,
\en{here comes}\ru{получается}
\en{the~local differential version of}\ru{локальная дифференциальная версия}~\eqref{princlipleofvirtualwork.integral:nonlinearmomentlesscontinuum}

\nopagebreak\vspace{-0.2em}
\begin{equation}\label{princlipleofvirtualwork.local:nonlinearmomentlesscontinuum}
\Bigl( \hspace{-0.2ex} \boldnabla \dotp \cauchystress \hspace{.1ex} + \hspace{-0.1ex} \massdensity \hspace{-0.1ex} \massloadvector \Bigr) \hspace{-0.4ex} \dotp \variation{\currentlocationvector}
+ \cauchystress \dotdotp \hspace{-0.12ex} \boldnabla \hspace{.1ex} \variation{\currentlocationvector}^{\T} \hspace{-0.33ex}
+ \hspace{.1ex} \variation{\internalwork}
= \hspace{.1ex} 0 \hspace{.1ex}.
\end{equation}

\vspace{-0.25em}
\en{When}\ru{Когда}
\en{a~body}\ru{тело}
\en{virtually moves}\ru{виртуально движется}
\en{as a~rigid whole}\ru{как жёсткое целое},
\en{the~work}\ru{работа}
\en{of~internal forces}\ru{внутренних сил}
\en{nullifies}\ru{обнуляется}

\nopagebreak\vspace{-0.25em}\begin{equation*}
%\label{princlipleofvirtualwork.worknullifies:nonlinearmomentlesscontinuum}
\begin{array}{c}
\variation{\currentlocationvector} = \constvarvector{\hspace{-0.1ex}\bm{\rho}} \hspace{.1ex} + \constvarvector{o} \hspace{-0.33ex} \times \hspace{-0.33ex} \currentlocationvector
%
\hspace{.66ex}\Rightarrow\hspace{.4ex}
%
\variation{\internalwork} \hspace{-0.2ex} = 0
\hspace{.1ex} ,
\\[.4em]
%
\bigl( \hspace{.1ex} \boldnabla \dotp \cauchystress \hspace{.1ex} + \hspace{-0.1ex} \massdensity \hspace{-0.1ex} \massloadvector \hspace{.16ex} \bigr)
\hspace{-0.25ex} \dotp \hspace{-0.25ex}
\bigl(  \constvarvector{\hspace{-0.1ex}\bm{\rho}} \hspace{.1ex} + \constvarvector{o} \hspace{-0.33ex} \times \hspace{-0.33ex} \currentlocationvector \hspace{.2ex} \bigr) \hspace{-0.16ex}
+ \cauchystress^{\hspace{.16ex}\T} \hspace{-0.5ex}
\dotdotp
\hspace{-0.16ex} \boldnabla \hspace{.1ex} \bigl(
\constvarvector{\hspace{-0.1ex}\bm{\rho}} \hspace{.1ex} +
\constvarvector{o} \hspace{-0.33ex} \times \hspace{-0.33ex} \currentlocationvector \hspace{.2ex} \bigr) \hspace{-0.25ex}
= 0 \hspace{.1ex} ,
\\[.4em]
%
\constvarvector{\hspace{-0.1ex}\bm{\rho}} = \boldconstant
\hspace{.4ex} \Rightarrow \hspace{.2ex}
\boldnabla \hspace{.1ex} \constvarvector{\hspace{-0.1ex}\bm{\rho}} =  \hspace{-0.12ex} \zerobivalent \hspace{.1ex} ,
\;\:
\constvarvector{o} = \boldconstant
\hspace{.4ex} \Rightarrow \hspace{.2ex}
\boldnabla \hspace{.1ex} \constvarvector{o} = \hspace{-0.12ex} \zerobivalent \hspace{.1ex} ,
\\[.4em]
%
\boldnabla \hspace{.1ex} \bigl(
\constvarvector{\hspace{-0.1ex}\bm{\rho}} \hspace{.1ex} +
\constvarvector{o} \hspace{-0.33ex} \times \hspace{-0.33ex} \currentlocationvector \hspace{.2ex} \bigr) \hspace{-0.25ex}
= \hspace{-0.1ex} \boldnabla \hspace{.1ex} \bigl(
\constvarvector{o} \hspace{-0.33ex} \times \hspace{-0.33ex} \currentlocationvector \hspace{.2ex} \bigr) \hspace{-0.25ex}
= \boldnabla \hspace{.1ex} \constvarvector{o} \hspace{-0.33ex} \times \hspace{-0.33ex} \currentlocationvector
\hspace{.1ex} - \hspace{-0.2ex}
\boldnabla \currentlocationvector \times \hspace{-0.2ex} \constvarvector{o} =
\\[.1em] %
\hspace*{\fill}
= - \boldnabla \currentlocationvector \hspace{-0.1ex} \times \hspace{-0.2ex} \constvarvector{o}
= - \hspace{.1ex} \UnitDyad \hspace{-0.16ex} \times \hspace{-0.2ex} \constvarvector{o}
%%= - \hspace{.1ex} \constvarvector{o} \hspace{-0.3ex} \times \hspace{-0.33ex} \UnitDyad
\\[.4em]
%
\cdots
%%\hspace{.1ex} .
\end{array}
\end{equation*}

\vspace{-0.2em}
\en{Assuming}\ru{Полагая}
${\constvarvector{o} \hspace{-0.1ex} = \zerovector}$
(\en{just a~translation}\ru{лишь трансляция})
${\Rightarrow}$~${\hspace{-0.1ex} \boldnabla \hspace{.1ex} \variation{\currentlocationvector}
= \hspace{-0.2ex} \boldnabla \hspace{.1ex} \constvarvector{\hspace{-0.1ex}\bm{\rho}}
= \hspace{-0.12ex} \zerobivalent}$,
\en{it turns into}\ru{оно превращается в}
\en{the~balance of~forces}\ru{баланс сил}~(\en{of~momentum}\ru{импульса})

\nopagebreak\vspace{-0.2em}
\begin{equation}\label{thebalanceofmomentum.nonlinear.withoutmultipliers}
\boldnabla \dotp \cauchystress \hspace{.1ex} + \hspace{-0.1ex} \massdensity \hspace{-0.1ex} \massloadvector \hspace{-0.1ex} = \zerovector
\hspace{.1ex} .
\end{equation}

\vspace{-0.1em}
\en{If}\ru{Если}
${\variation{\currentlocationvector} = \constvarvector{o} \hspace{-0.33ex} \times \hspace{-0.33ex} \currentlocationvector}$
(\en{just a~rotation}\ru{лишь поворот})
\en{with}\ru{с}~${\constvarvector{o} \hspace{-0.1ex} = \boldconstant}$,
\en{then}\ru{то}

\nopagebreak\vspace{-0.1em}\begin{equation*}
\begin{array}{r@{\hspace{.8ex}}l}
\eqrefwithchapterdotsection{gradientofcrossproductoftwovectors}{chapter:mathapparatus}{section:spatialdifferentiation}
\,\Rightarrow &
\boldnabla \hspace{.1ex} \variation{\currentlocationvector}
= \boldnabla \hspace{.1ex} \constvarvector{o} \hspace{-0.33ex} \times \hspace{-0.33ex} \currentlocationvector
\hspace{.1ex} - \hspace{-0.2ex}
\boldnabla \currentlocationvector \times \hspace{-0.2ex} \constvarvector{o}
= - \hspace{.1ex} \UnitDyad \hspace{-0.16ex} \times \hspace{-0.2ex} \constvarvector{o}
%%= - \hspace{.2ex} \constvarvector{o} \hspace{-0.3ex} \times \hspace{-0.33ex} \UnitDyad
\hspace{.1ex} ,
\\[.25em]
%
& \boldnabla \hspace{.1ex} \variation{\currentlocationvector}^{\T} \hspace{-0.32ex}
= \UnitDyad \hspace{-0.16ex} \times \hspace{-0.2ex} \constvarvector{o}
%%= \constvarvector{o} \hspace{-0.3ex} \times \hspace{-0.33ex} \UnitDyad
\end{array}
\end{equation*}

\noindent
\en{With}\ru{С}

\nopagebreak\vspace{-0.8em}\begin{equation*}
\begin{array}{c}
\eqrefwithchapterdotsection{pseudovectorinvariant}{chapter:mathapparatus}{section:tensors.symmetric+skewsymmetric} \:\Rightarrow\,
\cauchystress_{\hspace{-0.1ex}\Xcompanion} \hspace{-0.1ex} = - \hspace{.1ex} \cauchystress \hspace{.1ex} \dotdotp \permutationsparitytensor
\hspace{.1ex} ,
\\[.2em]
%
\cauchystress \dotdotp \hspace{-0.32ex} \left( \hspace{.1ex} \UnitDyad \hspace{-0.16ex} \times \hspace{-0.2ex} \constvarvector{o} \hspace{.1ex} \right) \hspace{-0.1ex}
= \cauchystress \dotdotp \hspace{-0.4ex} \left( \hspace{-0.1ex} - \hspace{.2ex} \permutationsparitytensor \dotp \constvarvector{o} \hspace{.1ex} \right) \hspace{-0.1ex}
= \left( \hspace{-0.1ex} - \hspace{.1ex} \cauchystress \dotdotp \permutationsparitytensor \hspace{.1ex} \right) \hspace{-0.3ex} \dotp \constvarvector{o} \hspace{.1ex}
= \cauchystress_{\hspace{-0.1ex}\Xcompanion} \dotp \hspace{.1ex} \constvarvector{o}
\end{array}
\end{equation*}

....

\en{In an~elastic continuum}\ru{В~упругой среде}\en{,}
\en{the~internal forces}\ru{внутренние силы}
\en{are potential}\ru{потенциальны}

\nopagebreak\vspace{-0.2em}
\begin{equation}\label{internalforcesarepotential.elastic:nonlinearmomentless}
\variation{\internalwork} = \hspace{.1ex} - \hspace{.2ex} \massdensity \hspace{.2ex} \variation{\potentialenergypermassunit}
%%\hspace{.1ex} .
\end{equation}

....


\en{Variational equation}\ru{Вариационное уравнение}~\eqref{princlipleofvirtualwork.local:nonlinearmomentlesscontinuum}
\en{with }\ru{с~}%
\en{the~balance}\ru{балансом}~%
\eqref{thebalanceofmomentum.nonlinear.withoutmultipliers}
\en{of~linear momentum}\ru{импульса}
(\en{of~forces}\ru{сил})
\en{and }\ru{и~}%
\en{the~balance}\ru{балансом}
\en{of~angular momentum}\ru{момента импульса}
${\cauchystress_{\hspace{-0.1ex}\Xcompanion} \hspace{-0.3ex} = \hspace{-0.1ex} 0}$
${\hspace{.3ex}\Leftrightarrow}$
${\cauchystress^{\hspace{.16ex}\T} \hspace{-0.5ex} = \hspace{-0.2ex} \cauchystress = \hspace{-0.2ex} \cauchystress^{\hspace{.22ex}\mathsf{S}}}$
\en{for}\ru{для}
\en{an~elastic}\ru{упругого}~\eqref{internalforcesarepotential.elastic:nonlinearmomentless}
\en{continuum}\ru{\rucontinuum{}а}

\nopagebreak\vspace{-0.2em}
\begin{equation}\label{variationalequation.local.forelastic:nonlinearmomentless}
\cauchystress \dotdotp \hspace{-0.2ex} \insideinfinitesimalstrainvariation
= \hspace{-0.1ex} - \hspace{.2ex} \variation{\internalwork} \hspace{-0.2ex}
= \massdensity \hspace{.2ex} \variation{\potentialenergypermassunit}
\hspace{.2ex} .
\end{equation}

\vspace{-0.4em}
\en{What}\ru{Как}
\ru{выглядит }\en{the~potential energy}\ru{потенциальная энергия}~${\potentialenergypermassunit}$
\en{per mass unit}\ru{на~единицу массы}\en{ looks like}\en{ is}\ru{\:---}
\en{yet unknown}\ru{пока неизвестно},
\en{but it’s obvious that}\ru{но очевидно, что}
${\potentialenergypermassunit}$
\en{is determined}\ru{определяется}
\en{by~deformation}\ru{деформацией}.
%
\en{The~potential energy}\ru{Потенциальная энергия}
\en{per unit volume}\ru{на~единицу объёма}~${\initialpotentialenergydensity}$
\en{in the~undeformed configuration}\ru{в~недеформированной конфигурации}
\en{can be}\ru{может быть}
\en{presented}\ru{представлена}
\en{as}\ru{как}

\nopagebreak\vspace{-0.2em}
\begin{equation}\label{elasticpotentialenergydensityofundeformed:nonlinearmomentless}
\initialpotentialenergydensity \equiv \initialmassdensity \hspace{.4ex} \potentialenergypermassunit
\hspace{.7ex} \Rightarrow \hspace{.5ex}
\variation{\initialpotentialenergydensity} = \initialmassdensity \hspace{.25ex} \variation{\potentialenergypermassunit}
\hspace{.1ex} .
\end{equation}

\vspace{-0.4em}\noindent
\en{With}\ru{С}~\en{the~balance of~mass}\ru{балансом массы}
${\massdensity \hspace{.2ex} \Jacobian \hspace{-0.1ex} = \initialmassdensity
\hspace{.5ex} \Leftrightarrow \hspace{.2ex}
\massdensity = \hspace{-0.1ex} \inverseJacobian \initialmassdensity}$
(${\Jacobian \hspace{-0.1ex} \equiv \determinant \bm{F}\hspace{-0.12ex}}$\en{ is}\ru{\:---}
\en{the }Jacobian,
\en{determinant}\ru{определитель}
\en{of the motion gradient}\ru{градиента движения})

\nopagebreak\vspace{-0.2em}\begin{equation*}
\massdensity \hspace{.2ex} \variation{\potentialenergypermassunit} = \hspace{-0.1ex} \inverseJacobian \hspace{.1ex} \variation{\initialpotentialenergydensity}
\hspace{.1ex} .
\end{equation*}


%----------
{\small
\setlength{\parindent}{0pt}

\begin{leftverticalbar}%%[oversize]

\inquotes{The elastic potential energy density per volume unit},
becomes when shorting
\inquotes{The elastic potential ....}

\begin{otherlanguage}{russian}
Плотность
упругой потенциальной энергии,
запасённой|накопленной
в~единице объёма тела (сред\'{ы}).
\end{otherlanguage}

\begin{otherlanguage}{russian}
Дословный перевод
c~english на~русский
фразы
\inquotesx{the elastic potential}
даёт
\inquotesx{упругий потенциал}[.]
\end{otherlanguage}

\end{leftverticalbar}
\par}
%----------


\begin{otherlanguage}{russian}

\vspace{-0.25em}
Полным аналогом~(...) является равенство

...

\end{otherlanguage}
