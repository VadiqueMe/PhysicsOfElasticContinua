\begin{changemargin}{\parindent}{\parindent}
\vspace{-2.5em}
{\noindent\small
\setlength{\parskip}{\spacebetweenparagraphs}

\en{When}\ru{Когда}
\en{relativistic mechanics}\ru{релятивистская механика}
(\en{for the~very fast}\ru{для очень быстрого})
\en{and}\ru{и}~\en{quantum mechanics}\ru{квантовая механика}
(\en{for the~very small}\ru{для очень маленького})
\en{emerged}\ru{возникли}
\en{at the~beginning}\ru{в~начале}
\en{of~the~}\hbox{XX$^{\textrm{\en{th}\ru{го}}}$\hspace{-0.2ex}}~\en{century}\ru{века},
\ru{существовавшие до~этого }\en{the~equations of~mechanics}\ru{уравнения механики}\en{ existed prior to that},
\en{still}\ru{всё ещё}
\en{perfectly suitable}\ru{совершенно подходящие}
\en{for}\ru{для}
\en{describing objects}\ru{описания объектов}
\en{of~everyday}\ru{повседневных}
\en{sizes and~speeds}\ru{размеров и~скоростей},
\en{needed}\ru{нуждались}
\en{a~new name}\ru{в~новом имени}.
\en{So}\ru{Так что}
\en{the~}\inquotes{\en{classical}\ru{классическая}}
\en{in mechanics}\ru{в~механике}
\en{doesn’t refer}\ru{не относится}
\en{to antiquity}\ru{к~античности}.
\en{This}\ru{Это}
\en{was just chosen}\ru{было просто выбрано}
\en{as the~name}\ru{как имя}
\en{for}\ru{для}
\en{description}\ru{описания}
\en{of~reality}\ru{реальности}
\en{without any}\ru{без каких-либо}
\en{quantum}\ru{квантовых}
\en{and}\ru{и}~%
\en{relativistic}\ru{релятивистских}
\en{effects}\ru{эффектов}\ru{,}
\en{influencing it}\ru{влияющих на~неё}.

\begin{comment}

The earliest formulation of classical mechanics is often referred to as Newtonian mechanics.

\hspace{\horizontalindentforchapterintro}%
\en{About}\ru{Около}
\en{half a~century}\ru{полвека}
\en{after}\ru{после того, как}
Isaac Newton
\en{formulated}\ru{сформулировал}
\en{his}\ru{свои}
\href{https://en.wikipedia.org/wiki/Newton%27s_laws_of_motion}{\en{laws of~motion for point particle}\ru{законы движения для частиц-точек}},
Leonhard Euler
\en{extended}\ru{распространил}
\en{the~}\ru{законы }Newton’\en{s}\ru{а}\en{ laws}
\href{https://en.wikipedia.org/wiki/Euler%27s_laws_of_motion}{\en{to~the~motion of~a~rigid body}\ru{на~движение твердого тела}}.

Despite that in the Newton’s times there was no concept of vector, and vector notation did not exist then
(Newton himself just wrote out each single component),
the mechanics covered by Newton’s laws and Euler’s laws is sometimes called vectorial mechanics.
This is because Newtonian mechanics considers actually vector quantities, such as accelerations, momenta, forces.

The first person to use the term \inquotes{vector} was William Rowan Hamilton,
a hundred years after Newton’s Philosophiæ Naturalis Principia Mathematica.

\hspace{\horizontalindentforchapterintro}%
\en{Analytical mechanics}\ru{Аналитическая механика},
\en{the~alternative formulations}\ru{альтернативные формулировки}
\en{of~classical mechanics}\ru{классической механики}\ru{,}
developed
\en{after}\ru{после}
Newton\en{ian}\ru{’овой}~(\en{vectorial}\ru{векторной})
\en{mechanics}\ru{механики}.

Analytical mechanics uses scalar properties of motion representing the system as a whole\:---
usually its total kinetic energy and potential energy\:---
not Newton’s vectorial forces of individual particles.
The equations of motion are derived from the scalar quantity by some underlying principle about the scalar's variation.

\end{comment}

\par}
\vspace{-1.2em}
\end{changemargin}
