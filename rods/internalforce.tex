\documentclass[tikz, border=10mm]{standalone}

\usepackage[utf8]{inputenc}

\usepackage{xcolor}
\usepackage{graphicx}

\renewcommand\thepage{\oldstylenums{\arabic{page}}}

\usepackage{tikz}
\usepackage{tikz-3dplot}
\usetikzlibrary{calc}
\usetikzlibrary{arrows, arrows.meta}

\usepackage[T2A, T1]{fontenc}
\usepackage[english, russian]{babel}

\def\horizontalindent{4ex}
\setlength{\parindent}{\horizontalindent} % offset of the first line
\usepackage{indentfirst}

\usepackage{setspace}
\setstretch{1.3} % spacing between lines

\usepackage{upgreek}

\usepackage{amssymb}
\usepackage{amsmath}
\usepackage{mathtools}
\usepackage{wasysym}

\usepackage{xargs}
\usepackage{xifthen}

\usepackage{verbatim}

\begin{document}

\newlength{\savedparindent}
\setlength{\savedparindent}{\parindent}

\def\referenceLength{5}

\def\beamcolor{black}
\def\epurecolor{blue}
\def\externalforcecolor{red}
\def\internalforcecolor{\epurecolor}

\def\beamlinewidth{2.8pt}
\tikzstyle{beam line} = [ line width=\beamlinewidth, line cap=round, color=\beamcolor ]

\tikzstyle{external force} =
	[ line width=(2/3)*\beamlinewidth, color=\externalforcecolor, line cap=round, -{Triangle[open, round, length=12pt, width=8pt]} ]

\tikzstyle{internal force} =
	[ line width=(1/2)*\beamlinewidth, color=\internalforcecolor, line cap=round, -{Triangle[round, length=10.5pt, width=7pt]} ]

\begin{tikzpicture}[ scale = 1 ]

\draw [ beam line ]
	( 0, 0 ) -- ++( \referenceLength, 0 ) ;
\draw [ beam line, fill=\beamcolor ] ( 0, 0 ) circle ( 1.2pt ) ;
\draw [ beam line, fill=\beamcolor ] ( \referenceLength, 0 ) circle ( 1.2pt ) ;

\def\externalforcelength{1}

\draw [ external force ]
	($ ( 0, 0 ) - ( \externalforcelength, 0 ) $) -- ++( \externalforcelength, 0 )  ;
\draw [ external force ]
	($ ( \referenceLength, 0 ) + ( \externalforcelength, 0 ) $) -- ++( -\externalforcelength, 0 )  ;

\def\numsteps{9}

\def\epureoffsetfirst{-1.5}
\def\epureoffsetnext{-3.5}
\def\epureoffsetthird{-5.75}
\def\epureoffsetfourth{-8.25}

\pgfmathsetmacro\zfrom{\referenceLength}
\pgfmathsetmacro\zto{0}
\pgfmathsetmacro\zstep{(\zto - \zfrom) / (\numsteps - 1)}
\pgfmathsetmacro\znext{\zfrom + \zstep}

\def\epureoffset{\epureoffsetfirst}

\foreach \z in { \zfrom, \znext, ..., \zto } {
\pgfmathparse{(\z == \zto) ? 0 : 1}\ifdim\pgfmathresult pt>0pt%
	\draw [ internal force, color=\epurecolor, rotate around={45:( \z, \epureoffset )} ] ($ ( \z, \epureoffset ) - ( \externalforcelength, 0 ) $) -- ++( \externalforcelength, 0 ) ;
\else
	\draw [ external force, color=\externalforcecolor, opacity=.66, rotate around={45:( \z, \epureoffset )} ] ($ ( \z, \epureoffset ) - ( \externalforcelength, 0 ) $) -- ++( \externalforcelength, 0 ) ;
\fi
\pgfmathparse{(\z == \zfrom) ? 0 : 1}\ifdim\pgfmathresult pt>0pt%
	\draw [ internal force, color=\epurecolor, rotate around={45:( \z, \epureoffset )} ] ($ ( \z, \epureoffset ) + ( \externalforcelength, 0 ) $) -- ++( -\externalforcelength, 0 ) ;
\else
	\draw [ external force, color=\externalforcecolor, opacity=.66, rotate around={45:( \z, \epureoffset )} ] ($ ( \z, \epureoffset ) + ( \externalforcelength, 0 ) $) -- ++( -\externalforcelength, 0 ) ;
\fi
}

\draw [ beam line, color=\beamcolor ]
	( 0, \epureoffset ) -- ++( \referenceLength, 0 ) ;

\def\epureoffset{\epureoffsetnext}

\foreach \z in { \zfrom, \znext, ..., \zto } {
\pgfmathparse{(\z == \zto) ? 0 : 1}\ifdim\pgfmathresult pt>0pt%
	\draw [ internal force, color=\epurecolor, rotate around={-45:( \z, \epureoffset )} ] ($ ( \z, \epureoffset ) - ( \externalforcelength, 0 ) $) -- ++( \externalforcelength, 0 ) ;
\else
	\draw [ external force, color=\externalforcecolor, opacity=.66, rotate around={-45:( \z, \epureoffset )} ] ($ ( \z, \epureoffset ) - ( \externalforcelength, 0 ) $) -- ++( \externalforcelength, 0 ) ;
\fi
\pgfmathparse{(\z == \zfrom) ? 0 : 1}\ifdim\pgfmathresult pt>0pt%
	\draw [ internal force, color=\epurecolor, rotate around={-45:( \z, \epureoffset )} ] ($ ( \z, \epureoffset ) + ( \externalforcelength, 0 ) $) -- ++( -\externalforcelength, 0 ) ;
\else
	\draw [ external force, color=\externalforcecolor, opacity=.66, rotate around={-45:( \z, \epureoffset )} ] ($ ( \z, \epureoffset ) + ( \externalforcelength, 0 ) $) -- ++( -\externalforcelength, 0 ) ;
\fi
}

\draw [ beam line, color=\beamcolor ]
	( 0, \epureoffset ) -- ++( \referenceLength, 0 ) ;

\def\epureoffset{\epureoffsetthird}

\foreach \z in { \zfrom, \znext, ..., \zto } {
\pgfmathparse{(\z == \zto) ? 0 : 1}\ifdim\pgfmathresult pt>0pt%
	\draw [ internal force, color=\epurecolor, rotate around={90:( \z, \epureoffset )} ] ($ ( \z, \epureoffset ) - ( \externalforcelength, 0 ) $) -- ++( \externalforcelength, 0 ) ;
\else
	\draw [ external force, color=\externalforcecolor, opacity=.66, rotate around={90:( \z, \epureoffset )} ] ($ ( \z, \epureoffset ) - ( \externalforcelength, 0 ) $) -- ++( \externalforcelength, 0 ) ;
\fi
\pgfmathparse{(\z == \zfrom) ? 0 : 1}\ifdim\pgfmathresult pt>0pt%
	\draw [ internal force, color=\epurecolor, rotate around={90:( \z, \epureoffset )} ] ($ ( \z, \epureoffset ) + ( \externalforcelength, 0 ) $) -- ++( -\externalforcelength, 0 ) ;
\else
	\draw [ external force, color=\externalforcecolor, opacity=.66, rotate around={90:( \z, \epureoffset )} ] ($ ( \z, \epureoffset ) + ( \externalforcelength, 0 ) $) -- ++( -\externalforcelength, 0 ) ;
\fi
}

\draw [ beam line, color=\beamcolor ]
	( 0, \epureoffset ) -- ++( \referenceLength, 0 ) ;

\def\epureoffset{\epureoffsetfourth}

\foreach \z in { \zfrom, \znext, ..., \zto } {
\pgfmathparse{(\z == \zto) ? 0 : 1}\ifdim\pgfmathresult pt>0pt%
	\draw [ internal force, color=\epurecolor, rotate around={-90:( \z, \epureoffset )} ] ($ ( \z, \epureoffset ) - ( \externalforcelength, 0 ) $) -- ++( \externalforcelength, 0 ) ;
\else
	\draw [ external force, color=\externalforcecolor, opacity=.66, rotate around={-90:( \z, \epureoffset )} ] ($ ( \z, \epureoffset ) - ( \externalforcelength, 0 ) $) -- ++( \externalforcelength, 0 ) ;
\fi
\pgfmathparse{(\z == \zfrom) ? 0 : 1}\ifdim\pgfmathresult pt>0pt%
	\draw [ internal force, color=\epurecolor, rotate around={-90:( \z, \epureoffset )} ] ($ ( \z, \epureoffset ) + ( \externalforcelength, 0 ) $) -- ++( -\externalforcelength, 0 ) ;
\else
	\draw [ external force, color=\externalforcecolor, opacity=.66, rotate around={-90:( \z, \epureoffset )} ] ($ ( \z, \epureoffset ) + ( \externalforcelength, 0 ) $) -- ++( -\externalforcelength, 0 ) ;
\fi
}

\draw [ beam line, color=\beamcolor ]
	( 0, \epureoffset ) -- ++( \referenceLength, 0 ) ;

\end{tikzpicture}

\begin{tikzpicture}[ scale = 1 ]

\draw [ beam line, color=\beamcolor ]
	( 0, 0 ) -- ++( \referenceLength, 0 ) ;
\draw [ beam line, color=\beamcolor, fill=\beamcolor ] ( 0, 0 ) circle ( 1.2pt ) ;
\draw [ beam line, color=\beamcolor, fill=\beamcolor ] ( \referenceLength, 0 ) circle ( 1.2pt ) ;

\def\externalforcelength{1}

\draw [ external force ]
	( 0, 0 ) -- ++( -\externalforcelength, 0 )  ;
\draw [ external force ]
	( \referenceLength, 0 ) -- ++( \externalforcelength, 0 )  ;

\def\numsteps{9}

\def\epureoffsetfirst{-1.5}
\def\epureoffsetnext{-3.5}
\def\epureoffsetthird{-5.75}
\def\epureoffsetfourth{-8.25}

\pgfmathsetmacro\zfrom{\referenceLength}
\pgfmathsetmacro\zto{0}
\pgfmathsetmacro\zstep{(\zto - \zfrom) / (\numsteps - 1)}
\pgfmathsetmacro\znext{\zfrom + \zstep}

\def\epureoffset{\epureoffsetfirst}

\foreach \z in { \zfrom, \znext, ..., \zto } {
\pgfmathparse{(\z == \zto) ? 0 : 1}\ifdim\pgfmathresult pt>0pt%
	\draw [ internal force, color=\epurecolor, rotate around={45:( \z, \epureoffset )} ] ( \z, \epureoffset ) -- ++( -\externalforcelength, 0 ) ;
\else
	\draw [ external force, color=\externalforcecolor, opacity=.66, rotate around={45:( \z, \epureoffset )} ] ( \z, \epureoffset ) -- ++( -\externalforcelength, 0 ) ;
\fi
\pgfmathparse{(\z == \zfrom) ? 0 : 1}\ifdim\pgfmathresult pt>0pt%
	\draw [ internal force, color=\epurecolor, rotate around={45:( \z, \epureoffset )} ] ( \z, \epureoffset ) -- ++( \externalforcelength, 0 ) ;
\else
	\draw [ external force, color=\externalforcecolor, opacity=.66, rotate around={45:( \z, \epureoffset )} ] ( \z, \epureoffset ) -- ++( \externalforcelength, 0 ) ;
\fi
}

\draw [ beam line, color=\beamcolor ]
	( 0, \epureoffset ) -- ++( \referenceLength, 0 ) ;

\def\epureoffset{\epureoffsetnext}

\foreach \z in { \zfrom, \znext, ..., \zto } {
\pgfmathparse{(\z == \zfrom) ? 0 : 1}\ifdim\pgfmathresult pt>0pt%
	\draw [ internal force, color=\epurecolor, rotate around={-45:( \z, \epureoffset )} ] ( \z, \epureoffset ) -- ++( \externalforcelength, 0 ) ;
\else
	\draw [ external force, color=\externalforcecolor, opacity=.66, rotate around={-45:( \z, \epureoffset )} ] ( \z, \epureoffset ) -- ++( \externalforcelength, 0 ) ;
\fi
\pgfmathparse{(\z == \zto) ? 0 : 1}\ifdim\pgfmathresult pt>0pt%
	\draw [ internal force, color=\epurecolor, rotate around={-45:( \z, \epureoffset )} ] ( \z, \epureoffset ) -- ++( -\externalforcelength, 0 ) ;
\else
	\draw [ external force, color=\externalforcecolor, opacity=.66, rotate around={-45:( \z, \epureoffset )} ] ( \z, \epureoffset ) -- ++( -\externalforcelength, 0 ) ;
\fi
}

\draw [ beam line, color=\beamcolor ]
	( 0, \epureoffset ) -- ++( \referenceLength, 0 ) ;

\def\epureoffset{\epureoffsetthird}

\foreach \z in { \zfrom, \znext, ..., \zto } {
\pgfmathparse{(\z == \zto) ? 0 : 1}\ifdim\pgfmathresult pt>0pt%
	\draw [ internal force, color=\epurecolor, rotate around={90:( \z, \epureoffset )} ] ( \z, \epureoffset ) -- ++( -\externalforcelength, 0 ) ;
\else
	\draw [ external force, color=\externalforcecolor, opacity=.66, rotate around={90:( \z, \epureoffset )} ] ( \z, \epureoffset ) -- ++( -\externalforcelength, 0 ) ;
\fi
\pgfmathparse{(\z == \zfrom) ? 0 : 1}\ifdim\pgfmathresult pt>0pt%
	\draw [ internal force, color=\epurecolor, rotate around={90:( \z, \epureoffset )} ] ( \z, \epureoffset ) -- ++( \externalforcelength, 0 ) ;
\else
	\draw [ external force, color=\externalforcecolor, opacity=.66, rotate around={90:( \z, \epureoffset )} ] ( \z, \epureoffset ) -- ++( \externalforcelength, 0 ) ;
\fi
}

\draw [ beam line, color=\beamcolor ]
	( 0, \epureoffset ) -- ++( \referenceLength, 0 ) ;

\def\epureoffset{\epureoffsetfourth}

\foreach \z in { \zfrom, \znext, ..., \zto } {
\pgfmathparse{(\z == \zfrom) ? 0 : 1}\ifdim\pgfmathresult pt>0pt%
	\draw [ internal force, color=\epurecolor, rotate around={-90:( \z, \epureoffset )} ] ( \z, \epureoffset ) -- ++( \externalforcelength, 0 ) ;
\else
	\draw [ external force, color=\externalforcecolor, opacity=.66, rotate around={-90:( \z, \epureoffset )} ] ( \z, \epureoffset ) -- ++( \externalforcelength, 0 ) ;
\fi
\pgfmathparse{(\z == \zto) ? 0 : 1}\ifdim\pgfmathresult pt>0pt%
	\draw [ internal force, color=\epurecolor, rotate around={-90:( \z, \epureoffset )} ] ( \z, \epureoffset ) -- ++( -\externalforcelength, 0 ) ;
\else
	\draw [ external force, color=\externalforcecolor, opacity=.66, rotate around={-90:( \z, \epureoffset )} ] ( \z, \epureoffset ) -- ++( -\externalforcelength, 0 ) ;
\fi
}

\draw [ beam line, color=\beamcolor ]
	( 0, \epureoffset ) -- ++( \referenceLength, 0 ) ;

\end{tikzpicture}

\end{document}
