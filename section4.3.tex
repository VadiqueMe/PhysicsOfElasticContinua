\en{\section{Hooke’s law}}

\ru{\section{Закон Hooke’а}}

\label{section:hookelaw}

\vspace{-2.4em}
%%\hspace*{-\parindent}
\begin{minipage}{\columnwidth}

\noindent
{ \small ${%
\linearstress
\hspace{.1ex} =
\displaystyle\frac{ \raisemath{-0.2em}{\partial \hspace{.1ex} \potentialenergydensity} }%
{ \partial \infinitesimaldeformation }
= \stiffnesstensor \dotdotp \hspace{-0.1ex} \infinitesimaldeformation
= \infinitesimaldeformation \hspace{-0.1ex} \dotdotp \stiffnesstensor
}$ }

\nopagebreak\vspace{1em}
\en{That}\ru{То}
\en{relation}\ru{соотношение}
\en{between}\ru{между}
\en{the stress}\ru{напряжением}
\en{and}\ru{и}
\en{the deformation~(strain)}\ru{деформацией},
\en{which}\ru{которое}
\en{in}\ru{в}~\en{the~}\hbox{XVII$^{\hspace{.15ex}\textrm{\en{th}\ru{ом}}}$\hspace{-0.12em}}~\en{century}\ru{веке}
Robert Hooke\ru{~(Роберт Гук)}
\en{could only phrase}\ru{мог высказать лишь}
\en{pretty vaguely}\ru{весьма расплывчато}%
\stepcounter{footnote}\setcounter{auxfootnotecounter}{\value{footnote}}\footnotemark[\value{auxfootnotecounter}]\hbox{\hspace{-0.5ex},}\hspace{.2em}
\en{is written}\ru{записано}
\en{as part}\ru{как часть}
\en{of~the~complete set}\ru{полного набора}
\en{of equations}\ru{уравнений}~\eqref{lineartheory:wholesetofequations}
\en{and is implemented}\ru{и~осуществляется}
\en{via the~stiffness tensor}\ru{тензором жёсткости}

\nopagebreak\vspace{-0.1em}
\begin{equation}\label{thestiffnesstensor.thelineartheory}
   \stiffnesstensor
   \hspace{.1ex} =
   \displaystyle\frac{
      \raisemath{-0.1em}{ \partial^2 \hspace{.1ex} \potentialenergydensity }
   }{
      \raisemath{-0.1em}{
         \partial \infinitesimaldeformation \hspace{.1ex}
         \partial \infinitesimaldeformation
      }
   }
   = \hspace{-0.1ex}
   A^{i\hspace{-0.1ex}jk\hspace{.06ex}l} \hspace{.16ex}
   \locationvector_\differentialindex{i}
   \locationvector_\differentialindex{\hspace{-0.1ex}j}
   \locationvector_\differentialindex{k}
   \locationvector_\differentialindex{l}
   \hspace{.1ex} ,
   \hspace{1em}
   %
   A^{i\hspace{-0.1ex}jk\hspace{.06ex}l} \hspace{-0.16ex}
   =
   \displaystyle\frac{
      \raisemath{-0.1em}{ \partial^2 \hspace{.1ex} \potentialenergydensity }
   }%
   {
      \raisemath{-0.1em}{\partial \varepsilon_{ i\hspace{-0.1ex}j } \hspace{.1ex}
      \partial \varepsilon_{ \hspace{-0.1ex}k\hspace{.06ex}l }}
   }
   \hspace{.16ex} .
\end{equation}



\vspace{\parskip}
\end{minipage}

\nicefootnotetext{auxfootnotecounter}{%
\hspace*{.2em}\inquotesx{\textit{ceiiinosssttuu, id est, Ut tensio sic vis}}[---]
\bookauthor{Robert Hooke}.
%%%https://play.google.com/books/reader?id=LAtPAAAAcAAJ&pg=GBS.PA1
\href{https://www.google.com/books/edition/Lectures_de_Potentia_Restitutiva_Or_of_S/LAtPAAAAcAAJ}%
{Lectures de Potentia Restitutiva, Or of Spring Explaining the Power of Springing Bodies. London, 1678. \howmanypages{56~pages.}}%
}

\vspace{-0.1em}
\en{The~stiffness tensor}\ru{Тензор жёсткости}
\en{is}\ru{есть}
\en{the~partial derivative}\ru{частная производная}
\en{of~the~scalar}\ru{скалярной}
\en{elastic potential energy density}\ru{плотности упругой потенциальной энергии}~$\potentialenergydensity$
\en{twice}\ru{дважды}
\en{by}\ru{по}
\en{the~same}\ru{тому~же}
\en{bivalent}\ru{бивалентному}
\en{infinitesimal deformation tensor}\ru{тензору бесконечномалой деформации}~$\infinitesimaldeformation$.
\en{It is symmetric}\ru{Он симметричен}
\en{in the pairs of~indices}\ru{по~парам индексов}\::
${
   \stiffnesstensor_{ \hspace{.12ex} \indexjuggling{12}{34} } \hspace{-0.25ex}
   = \hspace{-0.1ex}
   \stiffnesstensor
   \hspace{.5ex} \Leftrightarrow \hspace{.2ex}
   A^{i\hspace{-0.1ex}jk\hspace{.06ex}l} \hspace{-0.2ex}
   =
   A^{ k\hspace{.06ex}l\hspace{.06ex}i\hspace{-0.1ex}j }\hspace{-0.25ex}
}$.
\en{Therefrom}\ru{Оттог\'{о}}
36~\en{constants}\ru{констант}
\en{out of}\ru{из}
${ 3^4 \hspace{-0.25ex} = \hspace{-0.12ex} 81 }$
\inquotes{\en{have a~twin}\ru{имеют двойник\'{а}}}
\en{and}\ru{и}
\en{only}\ru{только}
45
\en{are independent}\ru{независимы}.
%
\en{Furthermore}\ru{К~тому~же},
\en{due}\ru{из\hbox{-}за}
\en{to the~symmetry}\ru{симметрии}
\en{of the infinitesimal deformation tensor}\ru{тензора бесконечном\'{а}лой деформации}~$\infinitesimaldeformation$,
\en{the stiffness tensor}\ru{тензор жёсткости}~$\stiffnesstensor$
\en{is symmetric}\ru{симметричен}
\en{inside}\ru{внутри}
\en{each pair of~indices}\ru{каждой пары индексов}\::
${
   A^{i\hspace{-0.1ex}j\hspace{-0.06ex}k\hspace{.06ex}l} \hspace{-0.2ex}
   = A^{j\hspace{-0.06ex}ik\hspace{.06ex}l} \hspace{-0.2ex}
   = A^{i\hspace{-0.1ex}jl\hspace{-0.06ex}k}
}$~(${
   {} = A^{j\hspace{-0.06ex}i\hspace{.06ex}l\hspace{-0.06ex}k}
}$).
\en{This}\ru{Это}
\en{reduces}\ru{снижает}
\en{the number}\ru{число}
\en{of the independent constants}\ru{независимых констант}
(\en{the }\inquotes{\en{elastic moduli}\ru{упругих модулей}})
\en{to}\ru{до}~21\::

\nopagebreak\vspace{-0.25em}
\begin{equation*}
\scalebox{.8}{$%
\begin{array}{l}
\scalebox{1.16}{$ A^{abcd^{\mathstrut}} = A^{cdab} = A^{bacd} = A^{abdc} $}
\\[.2em]
%
A^{1\hspace{-0.1ex}1\hspace{-0.1ex}1\hspace{-0.1ex}1}
\\
A^{1\hspace{-0.1ex}1\hspace{-0.1ex}12}
= A^{1\hspace{-0.1ex}121}
= A^{121\hspace{-0.1ex}1}
= A^{21\hspace{-0.1ex}1\hspace{-0.1ex}1}
\\
A^{1\hspace{-0.1ex}1\hspace{-0.1ex}13}
= A^{1\hspace{-0.1ex}131}
= A^{131\hspace{-0.1ex}1}
= A^{31\hspace{-0.1ex}1\hspace{-0.1ex}1}
\\
A^{1\hspace{-0.1ex}122}
= A^{221\hspace{-0.1ex}1}
\\
A^{1\hspace{-0.1ex}123}
= A^{1\hspace{-0.1ex}132}
= A^{231\hspace{-0.1ex}1}
= A^{321\hspace{-0.1ex}1}
\\
A^{1\hspace{-0.1ex}133}
= A^{331\hspace{-0.1ex}1}
\\
A^{1212} = A^{1221} = A^{21\hspace{-0.1ex}12} = A^{2121}
\\
A^{1213} = A^{1231} = A^{1312} = A^{1321} = A^{21\hspace{-0.1ex}13} = A^{2131} = A^{31\hspace{-0.1ex}12} = A^{3121}
\\
A^{1222} = A^{2122} = A^{2212} = A^{2221}
\\
A^{1223} = A^{1232} = A^{2123} = A^{2132} = A^{2312} = A^{2321} = A^{3212} = A^{3221}
\\
A^{1233} = A^{2133} = A^{3312} = A^{3321}
\\
A^{1313} = A^{1331} = A^{31\hspace{-0.1ex}13} = A^{3131}
\\
A^{1322} = A^{2213} = A^{2231} = A^{3122}
\\
A^{1323} = A^{1332} = A^{2313} = A^{2331} = A^{3123} = A^{3132} = A^{3213} = A^{3231}
\\
A^{1333} = A^{3133} = A^{3313} = A^{3331}
\\
A^{2222}
\\
A^{2223} = A^{2232} = A^{2322} = A^{3222}
\\
A^{2233} = A^{3322}
\\
A^{2323} = A^{2332} = A^{3223} = A^{3232}
\\
A^{2333} = A^{3233} = A^{3323} = A^{3332}
\\
A^{3333}
\\
\end{array}%
$}
\end{equation*}



\vspace{-0.2em}
\en{The moduli}\ru{Модули}
\en{of the tetravalent stiffness tensor}\ru{четырёхвалентного тензора жёсткости}
\en{are often written as}\ru{часто записываются}
\en{the~symmetric}\ru{симметричной}
\ru{матрицей }${6 \times 6}$\en{ matrix}

\nopagebreak\vspace{-.3ex}
\begin{gather}\label{stiffnessasmatrix}
\displaystyle
\matrixasletter{\mathcal{A}}{6}{6}
\hspace{.1ex} = \hspace{-0.2ex}
\scalebox{.82}{$ \left[ \hspace{.4em}
   \begin{matrix}
      a_{11} & a_{12} & a_{13} & a_{14} & a_{15} & a_{16} \\
      \mathgray{a_{12}} & a_{22} & a_{23} & a_{24} & a_{25} & a_{26} \\
      \mathgray{a_{13}} & \mathgray{a_{23}} & a_{33} & a_{34} & a_{35} & a_{36} \\
      \mathgray{a_{14}} & \mathgray{a_{24}} & \mathgray{a_{34}} & a_{44} & a_{45} & a_{46} \\
      \mathgray{a_{15}} & \mathgray{a_{25}} & \mathgray{a_{35}} & \mathgray{a_{45}} & a_{55} & a_{56} \\
      \mathgray{a_{16}} & \mathgray{a_{26}} & \mathgray{a_{36}} & \mathgray{a_{46}} & \mathgray{a_{56}} & a_{66}
   \end{matrix}
\hspace{.32em} \right] $}
%
\hspace{-0.2ex} \equiv \hspace{-0.2ex}
%
\scalebox{.82}{$ \left[ \hspace{.4em}
   \begin{matrix}
     A^{\oneone\hspace{-0.1ex}\oneone} &
     A^{\oneone 22} &
     A^{\oneone 33} &
     A^{\oneone\hspace{-0.1ex}12} &
     A^{\oneone\hspace{-0.1ex}13} &
     A^{\oneone 23}
     \\
     \mathgray{A^{22\oneone}} &
     A^{2222} &
     A^{2233} &
     A^{1222} &
     A^{1322} &
     A^{2223}
     \\
     \mathgray{A^{33\oneone}} &
     \mathgray{A^{3322}} &
     A^{3333} &
     A^{1233} &
     A^{1333} &
     A^{2333}
     \\
     \mathgray{A^{12\oneone}} &
     \mathgray{A^{2212}} &
     \mathgray{A^{3312}} &
     A^{1212} &
     A^{1213} &
     A^{1223}
     \\
     \mathgray{A^{13\oneone}} &
     \mathgray{A^{2213}} &
     \mathgray{A^{3313}} &
     \mathgray{A^{1312}} &
     A^{1313} &
     A^{1323}
     \\
     \mathgray{A^{23\oneone}} &
     \mathgray{A^{2322}} &
     \mathgray{A^{3323}} &
     \mathgray{A^{2312}} &
     \mathgray{A^{2313}} &
     A^{2323}
   \end{matrix}
\hspace{.4em} \right] $}
\end{gather}



\vspace{.1em}
\en{Even}\ru{Даже}
\en{in Cartesian coordinates}\ru{в~декартовых координатах}
$x$,\:$y$,\:$z$,
\en{the quadratic form}\ru{квадратичная форма}~\eqref{linear.potentialenergydensity.quadratic}
\en{looks}\ru{в\'{ы}глядит}
\en{pretty}\ru{весьма}
\en{huge}\ru{громоздко}\::

\nopagebreak\vspace{-0.3em}
\begin{equation}\label{elasticenergylooongcartesian}
\begin{multlined}
\hspace{-1em}
2 \hspace{.1ex} \potentialenergydensity
= a_{11} \infinitesimaldeformationcomponents{x}^2 \hspace{-0.1ex}
+ a_{22} \infinitesimaldeformationcomponents{y}^2 \hspace{-0.1ex}
+ a_{33} \infinitesimaldeformationcomponents{z}^2 \hspace{-0.1ex}
+ a_{44} \infinitesimaldeformationcomponents{xy}^2 \hspace{-0.2ex}
+ a_{55} \infinitesimaldeformationcomponents{xz}^2 \hspace{-0.2ex}
+ a_{66} \infinitesimaldeformationcomponents{yz}^2
\\[-0.1em]
%
\hspace{.5em}
{} + 2 \hspace{.2ex} \bigl[ \hspace{.1ex}
%
\infinitesimaldeformationcomponents{x} \hspace{.1ex}
\bigl( a_{12} \infinitesimaldeformationcomponents{y} \hspace{-0.2ex}
+ a_{13} \infinitesimaldeformationcomponents{z} \hspace{-0.2ex}
+ a_{14} \infinitesimaldeformationcomponents{xy} \hspace{-0.2ex}
+ a_{15} \infinitesimaldeformationcomponents{xz} \hspace{-0.2ex}
+ a_{16} \infinitesimaldeformationcomponents{yz} \bigr)
\\[-0.1em]
%
\hspace{.5em}
+ \infinitesimaldeformationcomponents{y} \hspace{.1ex} \bigl(
a_{23} \infinitesimaldeformationcomponents{z} \hspace{-0.2ex}
+ a_{24} \infinitesimaldeformationcomponents{xy} \hspace{-0.2ex}
+ a_{25} \infinitesimaldeformationcomponents{xz} \hspace{-0.2ex}
+ a_{26} \infinitesimaldeformationcomponents{yz} \bigr)
\\[-0.1em]
%
\hspace{.5em}
+ \infinitesimaldeformationcomponents{z} \bigl(
a_{34} \infinitesimaldeformationcomponents{xy} \hspace{-0.2ex}
+ a_{35} \infinitesimaldeformationcomponents{xz} \hspace{-0.2ex}
+ a_{36} \infinitesimaldeformationcomponents{yz} \bigr)
\\[-0.1em]
%
+ \infinitesimaldeformationcomponents{xy} \hspace{.1ex} \bigl(
a_{45} \infinitesimaldeformationcomponents{xz} \hspace{-0.2ex}
+ a_{46} \infinitesimaldeformationcomponents{yz} \bigr) \hspace{-0.11ex}
+ a_{56} \infinitesimaldeformationcomponents{xz} \infinitesimaldeformationcomponents{yz}
%
\hspace{.1ex} \bigr]
\hspace{.1ex} .
\end{multlined}
\end{equation}

\noindent
\en{When}\ru{Когда}
\ru{добавляется }\en{a~material symmetry}\ru{материальная симметрия}\en{ is added},
\en{then the number}\ru{тогда число}
\en{of the independent moduli}\ru{независимых модулей}
\en{of tensor}\ru{тензора}~${\stiffnesstensor}$
\en{decreases}\ru{уменьшается}.

\subsection*{\en{One plane of~material symmetry}\ru{Одна плоскость материальной симметрии}, \en{a~monoclinic material}\ru{моноклинный материал}}

\en{For}\ru{Для}
\en{a~material}\ru{материала}
\en{with}\ru{с}
\en{a~symmetry plane}\ru{плоскостью симметрии}
\en{of the elastic properties}\ru{упругих свойств},
\en{for example}\ru{например}
${z = \constant}$.

\en{The~change}\ru{Изменение}
\en{of~signs}\ru{знаков}
\en{of~}\ru{координат }$x$
\en{and}\ru{и}~$y$\en{ coordinates}
\en{does not change}\ru{не меняет}
\en{the~potential energy density}\ru{плотность потенциальной энергии}~$\potentialenergydensity$.
%
\en{And}\ru{А}
\en{this is possible}\ru{это возможно}
\en{only}\ru{только}
\en{when}\ru{когда}

\nopagebreak\vspace{-0.2em}
\begin{equation}\label{zeroconstants:oneplaneofsymmetry:monoclinic}
\potentialenergydensity \hspace{.1ex}
\raisemath{-0.2em} {
    \Bigr\vert
}_{\substack{
    \infinitesimaldeformationcomponents{xz} \hspace{.2ex} = \hspace{.2ex} -\infinitesimaldeformationcomponents{xz} \\
    \infinitesimaldeformationcomponents{yz} \hspace{.2ex} = \hspace{.2ex} -\infinitesimaldeformationcomponents{yz}
}
} \hspace{-0.2ex}
=
\potentialenergydensity
\hspace{2em} \Leftrightarrow \hspace{-3em}
\begin{array}{c}
   0 =
   a_{15} \hspace{-0.2ex}
   = a_{16} \hspace{-0.2ex}
   = a_{25} \hspace{-0.2ex}
   = a_{26}
\\ [-0.1em]
\hspace{8em}
   = a_{35} \hspace{-0.2ex}
   = a_{36} \hspace{-0.2ex}
   = a_{45} \hspace{-0.2ex}
   = a_{46}
\end{array}
\end{equation}

\vspace{-0.2em}\noindent
---
\en{the~number}\ru{число}
\en{of~independent coefficients}\ru{независимых коэффициентов}
\en{lowers}\ru{падает}
\en{to}\ru{до}~13.

\subsection*{\en{An orthotropic}\ru{Ортотропный} \en{material}\ru{материал}}

\en{Let there be then}\ru{Пусть далее будут тогда}
\en{the two}\ru{две}
\en{planes}\ru{плоскости}
\en{of symmetry}\ru{симметрии}\::
${z = \constant}$
\en{and}\ru{и}
${y = \constant}$.
\en{Because}\ru{Поскольку}
\en{energy}\ru{энергия}~$\potentialenergydensity$
\en{in such a~case}\ru{в~таком случае}
\en{is not sensitive}\ru{не~чувствительна}
\en{to the~signs}\ru{к~знакам}\en{ of}~$\infinitesimaldeformationcomponents{yx}$
\en{and}\ru{и}~$\infinitesimaldeformationcomponents{yz}$,
\en{in addition}\ru{вдобавок}
\en{to}\ru{к}~\eqref{zeroconstants:oneplaneofsymmetry:monoclinic}
\en{we have}\ru{мы имеем}

\nopagebreak\vspace{-0.3em}
\begin{equation}\label{zeroconstants:2orthogonalplanesofsymmetry:orthotropic}
a_{14} \hspace{-0.2ex}
= a_{24} \hspace{-0.2ex}
= a_{34} \hspace{-0.2ex}
= a_{56} \hspace{-0.2ex}
= 0
\end{equation}

\vspace{-0.33em}\noindent
---
\en{9~constants remained}\ru{осталось 9~констант}.

\en{A~material}\ru{Материал}
\en{with}\ru{с}~\en{the three}\ru{тремя}
\en{mutually orthogonal}\ru{взаимно ортогональными}
\en{planes}\ru{плоскостями}
\en{of~symmetry}\ru{симметрии}\:---
\en{let these be}\ru{пусть это будут}
\ru{плоскости }\en{the }$x$
\en{and}\ru{и}
$y$, $z$\en{ planes}\:---
\en{is called}\ru{называется}
\en{the orthotropic}\ru{ортотропным}
(\en{orthogonally anisotropic}\ru{ортогонально анизотропным}).
%
\en{It’s easy to see}\ru{Легко увидеть}\ru{,}
\en{that}\ru{что}~\eqref{zeroconstants:oneplaneofsymmetry:monoclinic}
\en{and}\ru{и}~\eqref{zeroconstants:2orthogonalplanesofsymmetry:orthotropic}
\en{is the~whole set}\ru{это весь набор}
\en{of~zero constants}\ru{нулевых констант}\en{,}
\en{in this case as well}\ru{и~в~этом случае тоже}.
%
\en{So}\ru{Итак},
\en{an orthotropic material}\ru{ортотропный материал}
\en{is characterized}\ru{характеризуется}
\en{by the~nine}\ru{девятью}
\en{elastic moduli}\ru{упругими модулями},
\en{and}\ru{и}
\en{for}\ru{для}
\en{orthotropy}\ru{ортотропности}
\ru{достаточны }\en{the two}\ru{две}
\en{mutually perpendicular}\ru{взаимно перпендикулярные}
\en{planes}\ru{плоскости}
\en{of~symmetry}\ru{симметрии}\en{ are enough}.
%
\en{The~expression}\ru{Выражение}
\en{for}\ru{для}
\en{the elastic energy density}\ru{плотности упругой энергии}
\en{here}\ru{тут}
\en{can be}\ru{может быть}
\en{simplified to}\ru{упрощено до}

\nopagebreak\vspace{-0.25em}\begin{multline*}
\potentialenergydensity =^{\mathstrut^{\mathstrut}}
\smalldisplaystyleonehalf a_{11} \infinitesimaldeformationcomponents{x}^2 \hspace{-0.1ex} +
\smalldisplaystyleonehalf a_{22} \infinitesimaldeformationcomponents{y}^2 \hspace{-0.1ex} +
\smalldisplaystyleonehalf a_{33} \infinitesimaldeformationcomponents{z}^2 \hspace{-0.1ex} +
\smalldisplaystyleonehalf a_{44} \infinitesimaldeformationcomponents{xy}^2 \hspace{-0.2ex} +
\smalldisplaystyleonehalf a_{55} \infinitesimaldeformationcomponents{xz}^2 \hspace{-0.2ex} +
\smalldisplaystyleonehalf a_{66} \infinitesimaldeformationcomponents{yz}^2
\\[-0.1em]
%
+ a_{12} \infinitesimaldeformationcomponents{x} \infinitesimaldeformationcomponents{y} \hspace{-0.2ex}
+ a_{13} \infinitesimaldeformationcomponents{x} \infinitesimaldeformationcomponents{z} \hspace{-0.2ex}
+ a_{23} \infinitesimaldeformationcomponents{y} \infinitesimaldeformationcomponents{z}
\hspace{.1ex}
.
\end{multline*}

\en{For an~orthotropic material}\ru{Для ортотропного материала}\en{,}
\en{the shear}\ru{сдвиговые}~(\en{angular}\ru{угловые})
\en{deformations}\ru{деформации}
$\infinitesimaldeformationcomponents{xy}$,
$\infinitesimaldeformationcomponents{xz}$,
$\infinitesimaldeformationcomponents{yz}$
\en{are not linked}\ru{не~связаны}
\en{to the normal stresses}\ru{с~нормальными напряжениями}
${
   \sigma_x \hspace{-0.25ex}
   = \raisemath{.16em} {
      \scalebox{.88}{$ \partial \hspace{.1ex} \potentialenergydensity $}
   }
   \hspace{-0.1ex} / \hspace{-0.2ex}
   \raisemath{-0.32em} {
      \scalebox{.88}{$ \partial \infinitesimaldeformationcomponents{x} $}
   }
}$,
${
   \sigma_y \hspace{-0.25ex}
   = \raisemath{.16em} {
      \scalebox{.88}{$ \partial \hspace{.1ex} \potentialenergydensity $}
}
\hspace{-0.1ex} / \hspace{-0.2ex}
\raisemath{-0.32em} {
   \scalebox{.88}{$ \partial \infinitesimaldeformationcomponents{y} $}
}
}$,
${\sigma_z \hspace{-0.25ex}
   =
   \raisemath{.16em} {
      \scalebox{.88}{$ \partial \hspace{.1ex} \potentialenergydensity $}
   }
   \hspace{-0.1ex} / \hspace{-0.2ex} \raisemath{-0.32em} {
      \scalebox{.88}{$ \partial \infinitesimaldeformationcomponents{z} $}
  }
}$ (\en{and vice versa}\ru{и~наоборот}).

\en{The~popular}\ru{Популярный}
\en{orthotropic}\ru{ортотропный}
\en{material}\ru{материал}\en{ is}\ru{\:---}
\en{wood}\ru{древесина}.
\en{Elastic}\ru{Упругие}
\en{properties}\ru{свойства}
\en{there}\ru{там}
\en{differ}\ru{разн\'{я}тся}
\en{along three}\ru{вдоль трёх}
\en{mutually perpendicular}\ru{взаимно перпендикулярных}
\en{lines}\ru{линий}\::
\en{by the radius}\ru{по радиусу},
\en{along the circumference}\ru{вдоль окружности}
\en{and}\ru{и}
\en{along the trunk height}\ru{по высоте ствола}.

\subsection*{\en{A\:transversely isotropic}\ru{Трансверсально изотропный} \en{material}\ru{материал}}

\en{One more}\ru{Ещё один}
\en{case of~anisotropy}\ru{случай анизотропии}
\en{is}\ru{это}
\en{a~transversely isotropic}\ru{трансверсально изотропный}
\en{material}\ru{материал}.
\en{It is characterized}\ru{Он характеризуется}
\en{by an~axis}\ru{осью}
\en{of~anisotropy}\ru{анизотропии}\:---
\en{let it be}\ru{пусть это}~$z$.
\en{Then}\ru{Тогда}
\en{any plane}\ru{любая плоскость}\ru{,}
\en{which}\ru{которая}
\en{is parallel}\ru{параллельна}\footnote{%
\en{If}\ru{Если}
\en{a~plane}\ru{плоскость}
\en{is parallel}\ru{параллельна}
\en{to a~line}\ru{прямой},
\en{this plane’s normal vector}\ru{вектор нормали этой плоскости}
\en{is perpendicular}\ru{перпендикулярен}
\en{to that line}\ru{той прямой}.%
}\en{ to}~$z$\ru{,}
\en{is a~plane}\ru{является плоскостью}
\en{of material symmetry}\ru{материальной симметрии}.
\ru{Ясно}\en{It is clear}\ru{,}
\en{that this material is orthotropic}\ru{что этот материал ортотропен}.
\en{But}\ru{Но}
\en{more than that}\ru{кроме того},
\en{any rotation}\ru{любой поворот}
\en{of the deformation tensor}\ru{тензора деформаций}~$\infinitesimaldeformation$
\en{around the}\ru{вокруг оси}~$z$\en{~axis}
\ru{не меняет}\en{doesn’t change}
\en{the elastic potential energy density}\ru{плотность упругой потенциальной энергии}~$\potentialenergydensity$.
\en{Thus}\ru{Поэтому}

\nopagebreak\vspace{-0.1em}
\begin{equation}
\label{transversely-isotropic-material:hooke's law}
\scalebox{.9}{$
   \displaystyle\frac{%
      \raisemath{-0.2em}{ \partial \hspace{.2ex} \potentialenergydensity }}%
      { \partial \infinitesimaldeformation }
$} %end of scalebox
\dotdotp \hspace{-.2ex}
\bigl( \hspace{.2ex}
   \bm{k}
   \hspace{-.2ex} \times \hspace{-.2ex}
   \infinitesimaldeformation
   \hspace{-.1ex} - \hspace{-.1ex}
   \infinitesimaldeformation
   \hspace{-.2ex} \times \hspace{-.2ex}
   \bm{k}
\hspace{.2ex} \bigr) \hspace{-.2ex}
= 0
\hspace{.1ex}
,
\end{equation}

\noindent
\en{because}\ru{поскольку}
\en{for any}\ru{для любого}
\en{small rotation}\ru{малого поворота}
\en{with vector}\ru{с~вектором}~${ \varvector{\bm{o}} }$,
\en{the variation}\ru{вариация}
\en{of the infinitesimal deformation tensor}\ru{тензора бесконечномалой деформации}~$\infinitesimaldeformation$
\en{is}\ru{есть}
${
   \varvector{\bm{o}} \times \infinitesimaldeformation
   - \infinitesimaldeformation \times \varvector{\bm{o}}
}$,
\en{and}\ru{и}~$\varvector{\bm{o}}$
\en{goes along}\ru{идёт вдоль}~$z$
\en{with the unit vector}\ru{с~единичным вектором}
${\bm{k} \equiv \bm{e}_z}$.
\en{The equation}\ru{Равенство}~\eqref{transversely-isotropic-material:hooke's law}
\en{is true}\ru{истинно}
\en{for}\ru{для}
\en{any}\ru{любой}
\en{infinitesimal deformation}\ru{бесконечномалой деформации}~$\infinitesimaldeformation$.
\en{In~components}\ru{В~компонентах}

\noindent
\begin{multline*}\label{equationforelasticmoduliforatransverselyisotropicmaterial}
\bigl(
    a_{11} \infinitesimaldeformationcomponents{x}
  + a_{12} \infinitesimaldeformationcomponents{y}
  + a_{13} \infinitesimaldeformationcomponents{z}
\bigr)
\bigl(
  - 2 \infinitesimaldeformationcomponents{xy}
\bigr)
+
\bigl(
    a_{12} \infinitesimaldeformationcomponents{x}
  + a_{22} \infinitesimaldeformationcomponents{y}
  + a_{23} \infinitesimaldeformationcomponents{z}
\bigr) 2 \infinitesimaldeformationcomponents{xy}
\\
%
{}+ 2 a_{44} \infinitesimaldeformationcomponents{xy}
  \bigl( \infinitesimaldeformationcomponents{x} - \infinitesimaldeformationcomponents{y} \bigr)
+ 2 a_{55} \infinitesimaldeformationcomponents{xz}
  \bigl( -\infinitesimaldeformationcomponents{yz} \bigr)
+ 2 a_{66} \infinitesimaldeformationcomponents{yz} \infinitesimaldeformationcomponents{xz}
= 0
%.
\end{multline*}

\nopagebreak\vspace{-1em}
\begin{equation*}
\Rightarrow \hspace{.5em}
a_{11} \hspace{-.2ex} = a_{12} + a_{44} \hspace{-.2ex} = a_{22}
\hspace{.1ex} ,
\hspace{.5em}
a_{13} = a_{23}
\hspace{.1ex} ,
\hspace{.5em}
a_{55} \hspace{-.2ex} = a_{66}
\hspace{.1ex} .
\end{equation*}

\en{Writing}\ru{Напис\'{а}в}
\en{the stress tensor}\ru{тензор напряжений}
\en{like}\ru{в~виде}

\nopagebreak
\begin{equation}\label{stress tensor for a transversely isotropic material}
\linearstress
= \withtheindexofperpendicularity{\linearstress} \hspace{-0.2ex}
+ \bm{s} \bm{k}
+ \bm{k} \bm{s}
+ \linearstresscomponents{zz} \hspace{.1ex} \bm{k} \bm{k}
\hspace{.1ex} ,
\end{equation}

\vspace{-0.8em}\noindent
\en{where}\ru{где}

\vspace{-1em}\nopagebreak
\begin{gather*}
\withtheindexofperpendicularity{\linearstress}\hspace{-0.2ex}
\equiv
\linearstresscomponents{\alpha \beta} \hspace{.2ex}
\bm{e}_{\alpha} \bm{e}_{\beta}
= \linearstresscomponents{x x} \hspace{.2ex} \bm{i} \bm{i}
+ \linearstresscomponents{x y} \bigl( \bm{i} \hspace{-0.1ex} \bm{j} + \bm{j} \bm{i} \hspace{.2ex} \bigr)
+ \linearstresscomponents{y y} \hspace{.2ex} \bm{j} \hspace{-0.2ex} \bm{j}
\hspace{.1ex} ,
\\
\bm{s}
\equiv \linearstresscomponents{\alpha z} \hspace{.2ex} \bm{e}_{\alpha}
= \linearstresscomponents{xz} \hspace{.2ex} \bm{i} + \linearstresscomponents{yz} \hspace{.2ex} \bm{j}
\\[.1em]
%
\bigl( \hspace{.5ex}
\alpha,\beta \hspace{1ex} \text{\en{are}\ru{это}} \hspace{.8ex} x \hspace{.8ex} \text{\en{or}\ru{или}} \hspace{.8ex} y
, \hspace{.66em}
\bm{e}_{x} \hspace{-.2ex} = \bm{i}
, \hspace{.5em}
\bm{e}_{y} \hspace{-.2ex} = \bm{j}
\hspace{.5ex} \bigr)
,
\end{gather*}

\noindent
\ru{закон }\en{the }Hooke’\en{s}\ru{а}\en{ law}
\en{for}\ru{для}
\en{a~transversely isotropic}\ru{трансверсально изотропного}
\en{material}\ru{материала}
\en{may be}\ru{может быть}
\en{presented as}\ru{представлен как}

\nopagebreak
\begin{equation}\label{Hooke's law for transversely isotropic material}
\withtheindexofperpendicularity{\linearstress} \hspace{-0.3ex}
=
a_{44} \withtheindexofperpendicularity{\infinitesimaldeformation}
+
\bigl( \hspace{.2ex}
   a_{12} \infinitesimaldeformationcomponents{\alpha \alpha}
   +
   a_{13} \infinitesimaldeformationcomponents{z}
\bigr)
\withtheindexofperpendicularity{\hspace{.1ex}\UnitDyad}
\hspace{.1ex} ,
\hspace{.6em}
\bm{s} = a_{55} \hspace{.2ex} \bm{\epsilon}
\hspace{.1ex} ,
\hspace{.6em}
\linearstresscomponents{zz} \hspace{-0.2ex} =
a_{33} \infinitesimaldeformationcomponents{z} +
a_{13} \infinitesimaldeformationcomponents{\alpha \alpha}
\end{equation}
%
\vspace{-3em}\begin{multline*}
\bigl( \hspace{.5ex}
\text{\en{here}\ru{здесь}} \hspace{1ex}
\withtheindexofperpendicularity{\infinitesimaldeformation} \hspace{-0.3ex}
\equiv
\infinitesimaldeformationcomponents{\alpha \beta} \hspace{.2ex}
\bm{e}_{\alpha} \bm{e}_{\beta}
\hspace{.1ex} , \hspace{.5em}
\infinitesimaldeformationcomponents{\alpha \alpha} \hspace{-0.2ex}
= \trace{\withtheindexofperpendicularity{\infinitesimaldeformation}} \hspace{-0.3ex}
= \infinitesimaldeformationcomponents{x} + \infinitesimaldeformationcomponents{y}
\hspace{.1ex} ,
\\[-0.2em]
%
\bm{\epsilon} \equiv \infinitesimaldeformationcomponents{\alpha z} \hspace{.2ex} \bm{e}_{\alpha}
\hspace{.1ex} , \hspace{.5em}
\withtheindexofperpendicularity{\hspace{.1ex}\UnitDyad} \hspace{-0.3ex}
\equiv \bm{e}_{\alpha} \bm{e}_{\alpha} \hspace{-0.3ex}
= \bm{i} \bm{i} + \bm{j} \hspace{-0.2ex} \bm{j}
\hspace{.5ex} \bigr)
\end{multline*}

\en{It comes that}\ru{Получается, что}
\en{a~transversely isotropic material}\ru{трансверсально изотропный материал}
\en{is characterized}\ru{характеризуется}
\en{by five}\ru{пятью}
\en{non-null}\ru{ненулевыми}
\en{mutually independent}\ru{взаимно независимыми}
\en{components}\ru{компонентами},
\en{the elastic moduli}\ru{упругими модулями}
${a_{12} \hspace{-0.2ex} = A^{\oneone 22}}$\hspace{-0.3ex},\hspace{.2ex}
${a_{13} \hspace{-0.2ex} = A^{\oneone 33}}$\hspace{-0.3ex},\hspace{.2ex}
${a_{33} \hspace{-0.2ex} = A^{3333}}$\hspace{-0.3ex},\hspace{.2ex}
${a_{44} \hspace{-0.2ex} = A^{1212}}$\hspace{-0.3ex},\hspace{.2ex}
${a_{55} \hspace{-0.2ex} = A^{1313}}$\hspace{-0.3ex}.

\subsection*{\en{A\:crystal symmetry}\ru{Симметрия кристаллов}}

\en{There are}\ru{Существует}
\en{only}\ru{всего}
\en{seven kinds}\ru{семь видов}
\en{of various primitive parallelepiped lattices}\ru{разных примитивных решёток-параллелепипедов}
(\href{https://en.wikipedia.org/wiki/Bravais_lattice}{\ru{решёток }Bravais\en{ lattices}})\:---
\en{the seven syngonies}\ru{семь сингоний}\footnote{%
\inquotes{\en{syngony}\ru{сингония}}
${\hspace{-0.5ex} = \hspace{-0.2ex}}$
\inquotes{\en{lattice system}\ru{система решёток}}
${\hspace{-0.5ex} = \hspace{-0.2ex}}$
\inquotes{\en{crystallographic system}\ru{кристаллографическая система}}
${\hspace{-0.5ex} = \hspace{-0.2ex}}$
\inquotes{\en{crystal symmetry}\ru{симметрия кристаллов}}
}\hbox{\hspace{-0.5ex},}
\en{namely}\ru{а~\'{и}менно}
\en{tri\-clinic}\ru{три\-клинная},
\en{mono\-clinic}\ru{моно\-клинная},
\en{ortho\-rhombic}\ru{орто\-ромбическая} (\en{or}\ru{или} \en{just}\ru{просто} \en{rhombic}\ru{ромбическая}),
\en{rhombo\-hedral}\ru{ромбо\-эдрическая} (\en{or}\ru{или} \en{tri\-gonal}\ru{три\-гональная}),
\en{tetra\-gonal}\ru{тетра\-гональная},
\en{hexa\-gonal}\ru{гекса\-гональная}
\en{and}\ru{и}~\en{cubic}\ru{кубическая}.

\en{Each}\ru{Каждый}
\en{case of crystal symmetry}\ru{случай симметрии кристаллов}
\en{is characterized}\ru{характеризуется}
\en{by the~set}\ru{набором}
\en{of ortho\-gonal}\ru{орто\-гональ\-ных}\footnote{%
\en{Orthogonal tensors}\ru{Ортогональные тензоры}
\en{are those that}\ru{это такие, которые}
\en{satisfy}\ru{удовлетворяют}
\en{the equality}\ru{равенству}
${ \orthogonaltensor \dotp \hspace{.1ex} \orthogonaltensor^\T \hspace{-0.5ex} = \hspace{-0.2ex} \UnitDyad }$
\eqrefwithchapterdotsection{orthogonalityofrotationtensor}{chapter:mathapparatus}{section:rotationtensors},
\en{describing}\ru{описывая}
\en{rotations}\ru{повороты}
\en{and}\ru{и}~\en{mirror flippings}\ru{зеркальные перевороты}.
}
\en{tensors}\ru{тензоров}~$\orthogonaltensor$,
\en{for which}\ru{для которых}
\en{the~following equation}\ru{следующее уравнение}

\nopagebreak
\begin{equation}\label{Hooke.for-symmetric-crystals}
   \stiffnesstensor \dotdotp
   \Bigl( \orthogonaltensor \dotp \infinitesimaldeformation \dotp \orthogonaltensor^\T \Bigr)
   = \orthogonaltensor \dotp
   \Bigl( \stiffnesstensor \dotdotp \infinitesimaldeformation \Bigr) \hspace{-0.2ex}
   \dotp \hspace{.2ex}
   \orthogonaltensor^\T
   \hspace{1.25em} \forall \hspace{.2ex} \infinitesimaldeformation
\end{equation}

\noindent
\en{is true}\ru{--- истина}
(\en{for}\ru{для}
\en{any}\ru{любой}
\en{infinitesimal deformation}\ru{бесконечномалой деформации}~$\infinitesimaldeformation$).

\subsection*{\en{Inverse relations}\ru{Обратные отношения}}

............

%
% the complementary energy
%

\nopagebreak\vspace{-0.2em}
\begin{equation}\label{legendretransformforlinearelasticenergy}
\infinitesimaldeformation ( \hspace{-0.1ex} \linearstress \hspace{.1ex} ) \hspace{-0.1ex}
= \displaystyle\frac{\raisemath{-0.2em}{\partial\hspace{.1ex} \complementaryenergydensity}}{\raisemath{.04em}{\partial \linearstress}}
= \hspace{-0.1ex} \pliabilitytensor \dotdotp \hspace{-0.1ex} \linearstress
\hspace{.1ex} ,
\hspace{.8em}
\complementaryenergydensity(\hspace{-0.1ex} \linearstress \hspace{.1ex}) \hspace{-0.1ex}
= \linearstress \dotdotp \infinitesimaldeformation - \potentialenergydensity ( \hspace{-0.1ex} \infinitesimaldeformation \hspace{.1ex} )
\end{equation}

\en{For}\ru{Для}
\en{the~linear model}\ru{линейной модели}
\begin{equation}\label{complementaryequalspotential.energy}
2 \potentialenergydensity
= \linearstress \dotdotp \infinitesimaldeformation
\hspace{.1ex} , \hspace{.8em}
\complementaryenergydensity = \potentialenergydensity = \smalldisplaystyleonehalf \hspace{.2ex} \linearstress \dotdotp \infinitesimaldeformation
\end{equation}
---
\en{the~complementary energy density}\ru{плотность добавочной~(дополнительной) энергии}
\en{is numerically equal to}\ru{численно равна}
\en{the~elastic potential energy density}\ru{плотности упругой потенциальной энергии}.

%
% material tensors
%

\subsection*{\en{Material tensors}\ru{Материальные тензоры}}

\en{Many physical phenomena}\ru{Многие физические явления}
\en{are described}\ru{описываются}
\en{by tensors}\ru{тензорами},
\en{including}\ru{включая}
\en{thermal}\ru{тепловые},
\en{mechanical}\ru{механические},
\en{electrical}\ru{электрические}
\en{and}\ru{и}~\en{magnetic}\ru{магнитные}
\en{properties}\ru{свойства}.

\en{The }\inquotes{\en{material tensors}\ru{Материальные тензоры}}
\en{define}\ru{определяют}
\en{the physical properties}\ru{физические свойства}
\en{of bodies}\ru{тел}
\en{and media}\ru{и~сред},
\en{kind of}\ru{как то}
\begin{itemize}
\item \en{elasticity}\ru{упругость},
\item \en{thermal expansion}\ru{тепловое расширение},
\item \en{thermal conductivity}\ru{теплопроводность},
\item \en{electrical conductivity}\ru{электропроводность},
\item \en{piezoelectric effect}\ru{пьезоэлектрический эффект}.
\end{itemize}

\vspace{.2em}{\small
Piezoelectricity (the piezoelectric effect) is the coupling (for example, linear)
between the mechanical strain and the electric charge in a material,
it is the transduction of electrical and mechanical energy.

Certain materials generate an electrical charge when mechanical stress is applied to them.
Piezoelectric materials directly transduce electrical and mechanical energy.
The most famous piezoelectric material is quartz crystal.
Certain ceramics are piezoelectric as well, piezoelectricity is often associated with ceramic materials.
Piezoelectric behaviour is also observed in many polymers.
And biomatter, such as bone and various proteins, too.
\par}
