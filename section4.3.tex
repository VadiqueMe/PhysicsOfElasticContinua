\en{\section{Hooke’s law}}

\ru{\section{Закон Hooke’а}}

\nopagebreak\vspace{-2.4em}
\noindent
\small ${%
\linearstress
\hspace{.1ex} =
\displaystyle\frac{ \raisemath{-0.2em}{\partial \hspace{.1ex} \potentialenergydensity} }%
{ \partial \infinitesimaldeformation }
= \stiffnesstensor \dotdotp \hspace{-0.1ex} \infinitesimaldeformation
= \infinitesimaldeformation \hspace{-0.1ex} \dotdotp \stiffnesstensor
}$
\vspace{.5em}

\label{section:hookelaw}

\nopagebreak
\en{That}\ru{То}
\en{relation}\ru{соотношение}
\en{between}\ru{между}
\en{the stress}\ru{напряжением}
\en{and}\ru{и}
\en{the deformation~(strain)}\ru{деформацией},
\en{which}\ru{которое}
\en{in}\ru{в}~\en{the~}\hbox{XVII$^{\hspace{.15ex}\textrm{\en{th}\ru{ом}}}$\hspace{-0.12em}}~\en{century}\ru{веке}
Robert Hooke\ru{~(Роберт Гук)}
\en{could only phrase}\ru{мог высказать лишь}
\en{pretty vaguely}\ru{весьма расплывчато}\footnote{%
\hspace*{.2em}\inquotesx{\textit{ceiiinosssttuu, id est, Ut tensio sic vis}}[---]
\bookauthor{Robert Hooke}.
%%%https://play.google.com/books/reader?id=LAtPAAAAcAAJ&pg=GBS.PA1
\href{https://www.google.com/books/edition/Lectures_de_Potentia_Restitutiva_Or_of_S/LAtPAAAAcAAJ}%
{Lectures de Potentia Restitutiva, Or of Spring Explaining the Power of Springing Bodies. London, 1678. \howmanypages{56~pages.}}%
}\hbox{\hspace{-0.32em},}\hspace{.2em}
\en{is written}\ru{записано}
\en{as part}\ru{как часть}
\en{of~the~complete set}\ru{полного набора}
\en{of equations}\ru{уравнений}~\eqref{lineartheory:wholesetofequations}
\en{and is implemented}\ru{и~осуществляется}
\en{via the~stiffness tensor}\ru{тензором жёсткости}

\nopagebreak\vspace{-0.1em}
\begin{equation}\label{thestiffnesstensor.thelineartheory}
   \stiffnesstensor
   \hspace{.1ex} =
   \displaystyle\frac{
      \raisemath{-0.1em}{ \partial^2 \hspace{.1ex} \potentialenergydensity }
   }{
      \raisemath{-0.1em}{
         \partial \infinitesimaldeformation \hspace{.1ex}
         \partial \infinitesimaldeformation
      }
   }
   = \hspace{-0.1ex}
   A^{i\hspace{-0.1ex}jk\hspace{.06ex}l} \hspace{.16ex}
   \locationvector_\differentialindex{i}
   \locationvector_\differentialindex{\hspace{-0.1ex}j}
   \locationvector_\differentialindex{k}
   \locationvector_\differentialindex{l}
   \hspace{.1ex} ,
   \hspace{1em}
   %
   A^{i\hspace{-0.1ex}jk\hspace{.06ex}l} \hspace{-0.16ex}
   =
   \displaystyle\frac{
      \raisemath{-0.1em}{ \partial^2 \hspace{.1ex} \potentialenergydensity }
   }%
   {
      \raisemath{-0.1em}{\partial \varepsilon_{ i\hspace{-0.1ex}j } \hspace{.1ex}
      \partial \varepsilon_{ \hspace{-0.1ex}k\hspace{.06ex}l }}
   }
   \hspace{.16ex} .
\end{equation}



\vspace{-0.1em}
\en{The~stiffness tensor}\ru{Тензор жёсткости}
\en{is}\ru{есть}
\en{the~partial derivative}\ru{частная производная}
\en{of~the~scalar}\ru{скалярной}
\en{elastic potential energy density}\ru{плотности упругой потенциальной энергии}~$\potentialenergydensity$
\en{twice}\ru{дважды}
\en{by}\ru{по}
\en{the same}\ru{тому~же}
\en{bivalent tensor}\ru{бивалентному тензору}
\en{of infinitesimal deformation}\ru{бесконечно малой деформации}
$\infinitesimaldeformation$.
\en{It is symmetric}\ru{Он симметричен}
\en{in the pairs of~indices}\ru{по~парам индексов}:
${
   \stiffnesstensor_{ \hspace{.12ex} \indexjuggling{12}{34} } \hspace{-0.25ex}
   = \hspace{-0.1ex}
   \stiffnesstensor
   \hspace{.5ex} \Leftrightarrow \hspace{.2ex}
   A^{i\hspace{-0.1ex}jk\hspace{.06ex}l} \hspace{-0.2ex}
   =
   A^{ k\hspace{.06ex}l\hspace{.06ex}i\hspace{-0.1ex}j }\hspace{-0.25ex}
}$.
\en{Therefrom}\ru{Оттог\'{о}} 
\en{we have}\ru{мы имеем}
36~\en{constants}\ru{констант}
\en{out of}\ru{из}
${ 3^4 \hspace{-0.25ex} = \hspace{-0.12ex} 81 }$
\inquotes{\en{have a~twin}\ru{имеют двойник\'{а}}}
\en{and}\ru{и}
\en{only}\ru{только}
45
\en{are independent}\ru{независимы}.
%
\en{Furthermore}\ru{К~тому~же},
\en{due}\ru{из\hbox{-}за}
\en{to the~symmetry}\ru{симметрии}
\en{of the infinitesimal deformation tensor}\ru{тензора бесконечном\'{а}лой деформации}~$\infinitesimaldeformation$,
\en{the stiffness tensor}\ru{тензор жёсткости}~$\stiffnesstensor$
\en{is symmetric}\ru{симметричен}
\en{inside}\ru{внутри}
\en{each pair of~indices}\ru{каждой пары индексов}:
${
   A^{i\hspace{-0.1ex}j\hspace{-0.06ex}k\hspace{.06ex}l} \hspace{-0.2ex}
   = A^{j\hspace{-0.06ex}ik\hspace{.06ex}l} \hspace{-0.2ex}
   = A^{i\hspace{-0.1ex}jl\hspace{-0.06ex}k}
}$~(${
   {} = A^{j\hspace{-0.06ex}i\hspace{.06ex}l\hspace{-0.06ex}k}
}$).
\en{This}\ru{Это}
\en{reduces}\ru{снижает}
\en{the number}\ru{число}
\en{of the independent constants}\ru{независимых констант}
(\en{the }\inquotes{\en{elastic moduli}\ru{упругих модулей}})
\en{to}\ru{до}~21:

\nopagebreak\vspace{-0.25em}
\begin{equation*}
\scalebox{.8}{$%
\begin{array}{l}
\scalebox{1.16}{$ A^{abcd^{\mathstrut}} = A^{cdab} = A^{bacd} = A^{abdc} $}
\\[.2em]
%
A^{1\hspace{-0.1ex}1\hspace{-0.1ex}1\hspace{-0.1ex}1}
\\
A^{1\hspace{-0.1ex}1\hspace{-0.1ex}12}
= A^{1\hspace{-0.1ex}121}
= A^{121\hspace{-0.1ex}1}
= A^{21\hspace{-0.1ex}1\hspace{-0.1ex}1}
\\
A^{1\hspace{-0.1ex}1\hspace{-0.1ex}13}
= A^{1\hspace{-0.1ex}131}
= A^{131\hspace{-0.1ex}1}
= A^{31\hspace{-0.1ex}1\hspace{-0.1ex}1}
\\
A^{1\hspace{-0.1ex}122}
= A^{221\hspace{-0.1ex}1}
\\
A^{1\hspace{-0.1ex}123}
= A^{1\hspace{-0.1ex}132}
= A^{231\hspace{-0.1ex}1}
= A^{321\hspace{-0.1ex}1}
\\
A^{1\hspace{-0.1ex}133}
= A^{331\hspace{-0.1ex}1}
\\
A^{1212} = A^{1221} = A^{21\hspace{-0.1ex}12} = A^{2121}
\\
A^{1213} = A^{1231} = A^{1312} = A^{1321} = A^{21\hspace{-0.1ex}13} = A^{2131} = A^{31\hspace{-0.1ex}12} = A^{3121}
\\
A^{1222} = A^{2122} = A^{2212} = A^{2221}
\\
A^{1223} = A^{1232} = A^{2123} = A^{2132} = A^{2312} = A^{2321} = A^{3212} = A^{3221}
\\
A^{1233} = A^{2133} = A^{3312} = A^{3321}
\\
A^{1313} = A^{1331} = A^{31\hspace{-0.1ex}13} = A^{3131}
\\
A^{1322} = A^{2213} = A^{2231} = A^{3122}
\\
A^{1323} = A^{1332} = A^{2313} = A^{2331} = A^{3123} = A^{3132} = A^{3213} = A^{3231}
\\
A^{1333} = A^{3133} = A^{3313} = A^{3331}
\\
A^{2222}
\\
A^{2223} = A^{2232} = A^{2322} = A^{3222}
\\
A^{2233} = A^{3322}
\\
A^{2323} = A^{2332} = A^{3223} = A^{3232}
\\
A^{2333} = A^{3233} = A^{3323} = A^{3332}
\\
A^{3333}
\\
\end{array}%
$}
\end{equation*}



\vspace{-0.2em}
\en{The moduli}\ru{Модули}
\en{of the tetravalent stiffness tensor}\ru{четырёхвалентного тензора жёсткости}
\en{are often written as}\ru{часто записываются}
\en{the~symmetric}\ru{симметричной}
\ru{матрицей }${6 \times 6}$\en{ matrix}

\nopagebreak\vspace{-.3ex}
\begin{gather}\label{stiffnessasmatrix}
\displaystyle
\matrixasletter{\mathcal{A}}{6}{6}
\hspace{.1ex} = \hspace{-0.2ex}
\scalebox{.82}{$ \left[ \hspace{.4em}
   \begin{matrix}
      a_{11} & a_{12} & a_{13} & a_{14} & a_{15} & a_{16} \\
      \mathgray{a_{12}} & a_{22} & a_{23} & a_{24} & a_{25} & a_{26} \\
      \mathgray{a_{13}} & \mathgray{a_{23}} & a_{33} & a_{34} & a_{35} & a_{36} \\
      \mathgray{a_{14}} & \mathgray{a_{24}} & \mathgray{a_{34}} & a_{44} & a_{45} & a_{46} \\
      \mathgray{a_{15}} & \mathgray{a_{25}} & \mathgray{a_{35}} & \mathgray{a_{45}} & a_{55} & a_{56} \\
      \mathgray{a_{16}} & \mathgray{a_{26}} & \mathgray{a_{36}} & \mathgray{a_{46}} & \mathgray{a_{56}} & a_{66}
   \end{matrix}
\hspace{.32em} \right] $}
%
\hspace{-0.2ex} \equiv \hspace{-0.2ex}
%
\scalebox{.82}{$ \left[ \hspace{.4em}
   \begin{matrix}
     A^{\oneone\hspace{-0.1ex}\oneone} &
     A^{\oneone 22} &
     A^{\oneone 33} &
     A^{\oneone\hspace{-0.1ex}12} &
     A^{\oneone\hspace{-0.1ex}13} &
     A^{\oneone 23}
     \\
     \mathgray{A^{22\oneone}} &
     A^{2222} &
     A^{2233} &
     A^{1222} &
     A^{1322} &
     A^{2223}
     \\
     \mathgray{A^{33\oneone}} &
     \mathgray{A^{3322}} &
     A^{3333} &
     A^{1233} &
     A^{1333} &
     A^{2333}
     \\
     \mathgray{A^{12\oneone}} &
     \mathgray{A^{2212}} &
     \mathgray{A^{3312}} &
     A^{1212} &
     A^{1213} &
     A^{1223}
     \\
     \mathgray{A^{13\oneone}} &
     \mathgray{A^{2213}} &
     \mathgray{A^{3313}} &
     \mathgray{A^{1312}} &
     A^{1313} &
     A^{1323}
     \\
     \mathgray{A^{23\oneone}} &
     \mathgray{A^{2322}} &
     \mathgray{A^{3323}} &
     \mathgray{A^{2312}} &
     \mathgray{A^{2313}} &
     A^{2323}
   \end{matrix}
\hspace{.4em} \right] $}
\end{gather}



\vspace{.1em}
\en{Even}\ru{Даже}
\en{in Cartesian coordinates}\ru{в~декартовых координатах}
$x$,\:$y$,\:$z$
\en{the quadratic form}\ru{квадратичная форма}
\en{of elastic energy density}\ru{плотности упругой энергии}
${\potentialenergydensity ( \hspace{-0.1ex}
   \infinitesimaldeformation
   \hspace{-0.1ex} )
   \hspace{-0.2ex} = \hspace{.1ex}
   \smallerdisplaystyleonehalf \hspace{.25ex}
   \infinitesimaldeformation
   \hspace{-0.1ex} \dotdotp \hspace{-0.1ex}
   \stiffnesstensor
   \dotdotp \hspace{-0.1ex}
   \infinitesimaldeformation
}$
\en{looks}\ru{в\'{ы}глядит}
\en{pretty}\ru{весьма}
\en{huge}\ru{громоздкой}:

\nopagebreak\vspace{-0.2em}
\begin{equation}\label{elasticenergylooongcartesian}
\begin{gathered}
\hspace{-5em}
2 \hspace{.1ex} \potential
= a_{11} \varepsilon^2_{x} \hspace{-0.1ex}
+ a_{22} \varepsilon^2_{y} \hspace{-0.1ex}
+ a_{33} \varepsilon^2_{z} \hspace{-0.1ex}
+ a_{44} \varepsilon^2_{xy} \hspace{-0.2ex}
+ a_{55} \varepsilon^2_{xz} \hspace{-0.2ex}
+ a_{66} \varepsilon^2_{yz}
\\[-0.1em]
%
\hspace{.2em}
+ 2 \hspace{.2ex} \bigl[ \hspace{.1ex}
%
\varepsilon_{x} \hspace{.1ex}
\bigl( a_{12} \varepsilon_{y} \hspace{-0.2ex}
+ a_{13} \varepsilon_{z} \hspace{-0.2ex}
+ a_{14} \varepsilon_{xy} \hspace{-0.2ex}
+ a_{15} \varepsilon_{xz} \hspace{-0.2ex}
+ a_{16} \varepsilon_{yz} \bigr)
\\[-0.1em]
%
\hspace{3em}
+ \varepsilon_{y} \hspace{.1ex} \bigl(
a_{23} \varepsilon_{z} \hspace{-0.2ex}
+ a_{24} \varepsilon_{xy} \hspace{-0.2ex}
+ a_{25} \varepsilon_{xz} \hspace{-0.2ex}
+ a_{26} \varepsilon_{yz} \bigr)
\\[-0.1em]
%
\hspace{5em}
+ \varepsilon_{z} \bigl(
a_{34} \varepsilon_{xy} \hspace{-0.2ex}
+ a_{35} \varepsilon_{xz} \hspace{-0.2ex}
+ a_{36} \varepsilon_{yz} \bigr)
\\[-0.1em]
%
\hspace{11em}
+ \varepsilon_{xy} \hspace{.1ex} \bigl(
a_{45} \varepsilon_{xz} \hspace{-0.2ex}
+ a_{46} \varepsilon_{yz} \bigr) \hspace{-0.11ex}
+ a_{56} \varepsilon_{xz} \varepsilon_{yz}
%
\hspace{.1ex} \bigr]
\hspace{.1ex}
.
\end{gathered}
\end{equation}

\en{When}\ru{Когда}
\ru{добавляется }\en{a~material symmetry}\ru{материальная симметрия}\en{ is added},
\en{then the number}\ru{тогда число}
\en{of the independent moduli}\ru{независимых модулей}
\en{of tensor}\ru{тензора}~${\stiffnesstensor}$
\en{decreases}\ru{уменьшается}.

\en{For}\ru{Для}
\en{a~material}\ru{материала}
\en{with}\ru{с}
\en{a~symmetry plane}\ru{плоскостью симметрии}
\en{of the elastic properties}\ru{упругих свойств},
\en{for example}\ru{например}
${z = \constant}$.

\en{The change}\ru{Изменение}
\en{of signs}\ru{знаков}
\en{of}\ru{у}~\en{the coordinates}\ru{координат}
$x$
\en{and}\ru{и}~$y$
\en{does not change}\ru{не меняет}
\en{the potential energy density}\ru{плотность потенциальной энергии}~${\potentialenergydensity}.
%
\en{And}\ru{А}
\en{it’s possible}\ru{это возможно}
\en{only}
\en{when}\ru{когда}

\noindent
\begin{equation}\label{zeroconstants:oneplaneofsymmetry}
\potentialenergydensity \hspace{.1ex}
\raisemath{-0.2em} {
    \Bigr\vert
}_{\substack{
    \varepsilon_{xz} \hspace{.2ex}=\hspace{.2ex} -\varepsilon_{xz} \\
    \varepsilon_{yz} \hspace{.2ex}=\hspace{.2ex} -\varepsilon_{yz}
}
} \hspace{-0.2ex}
=
\potential
\hspace{2em} \Leftrightarrow \hspace{-3em}
\begin{array}{c}
   0 =
   a_{15} \hspace{-0.2ex}
   = a_{16} \hspace{-0.2ex}
   = a_{25} \hspace{-0.2ex}
   = a_{26}
\\ [-0.1em]
\hspace{8em}
   = a_{35} \hspace{-0.2ex}
   = a_{36} \hspace{-0.2ex}
   = a_{45} \hspace{-0.2ex}
   = a_{46}
\end{array}
\vspace{.1em}\end{equation}

\noindent
---
\en{the number}\ru{число}
\en{of the independent coefficients}\ru{независимых коэффициентов}
\en{lowers}\ru{падает}
\en{to}\ru{до}~13.

\en{Let there be then}\ru{Пусть далее будут тогда}
\en{the two}\ru{две}
\en{planes}\ru{плоскости}
\en{of symmetry}\ru{симметрии}:
${z = \constant}$
\en{and}\ru{и}
${y = \constant}$.
\en{Because}\ru{Поскольку}~$\potentialenergydensity$
\en{in such a~case}\ru{в~таком случае}
\en{is not sensitive}\ru{не~чувствительна} %энергия, женского роду
{\en{to the signs}\ru{к~знакам}\en{of}~$\varepsilon_{yx}$ \en{and}\ru{и}~$\varepsilon_{yz}$,
\en{in addition}\ru{вдобавок}
\en{to}\ru{к}~\eqref{zeroconstants:oneplaneofsymmetry}
\en{we have}\ru{мы имеем}

\nopagebreak\vspace{-0.22em}
\begin{equation}\label{zeroconstants:2orthogonalplanesofsymmetry:orthotropic}
a_{14} \hspace{-0.2ex}
= a_{24} \hspace{-0.2ex}
= a_{34} \hspace{-0.2ex}
= a_{56} \hspace{-0.2ex}
= 0
\end{equation}

\vspace{-0.25em}\noindent
--- \en{9~constants are left}\ru{осталось 9~констант}.

\en{A~material}\ru{Материал}
\en{with}\ru{с}~\en{the three}\ru{тремя}
\en{mutually orthogonal}\ru{взаимно ортогональными}
\en{planes}\ru{плоскостями}
\en{of~symmetry}\ru{симметрии}\:---
\en{let these be}\ru{пусть это будут}
\ru{плоскости }\en{the }$x$
\en{and}\ru{и}
$y$, $z$\en{ planes}\:---
\en{is called}\ru{называется}
\en{the orthotropic}\ru{ортотропным}
(\en{orthogonally anisotropic}\ru{ортогонально анизотропным})
.
%
\en{It's easy to see}\ru{Легко увидеть}\ru{,}
\en{that}\ru{что}~\eqref{zeroconstants:oneplaneofsymmetry}
\en{and}\ru{и}~\eqref{zeroconstants:2orthogonalplanesofsymmetry:orthotropic}\:---
\en{this is the whole set}\ru{это весь набор}
\en{of null constants}\ru{нулевых констант}\en{,}
\en{in this case too}\ru{и~в~этом случае}.
\en{So}\ru{Итак},
\en{an orthotropic material}\ru{ортотропный материал}
\en{is characterized}\ru{характеризуется}
\en{by the nine}\ru{девятью}
\en{constants}\ru{константами},
\en{and}\ru{и}
\en{for}\ru{для}
\en{the orthotropy}\ru{ортотропности}
\ru{достаточны }\en{the two}\ru{две}
\en{mutually perpendicular}\ru{взаимно перпендикулярные}
\en{planes}\ru{плоскости}
\en{of symmetry}\ru{симметрии}
\en{ are enough} .
\en{The expression}\ru{Выражение}
\en{of the elastic energy density}\ru{плотности упругой энергии}
\en{can be simplified to}\ru{может быть упрощено до}

\nopagebreak\vspace{-0.25em}\begin{multline*}
\potentialenergydensity =^{\mathstrut^{\mathstrut}}
\smalldisplaystyleonehalf a_{11} \varepsilon^2_{x} \hspace{-0.1ex} +
\smalldisplaystyleonehalf a_{22} \varepsilon^2_{y} \hspace{-0.1ex} +
\smalldisplaystyleonehalf a_{33} \varepsilon^2_{z} \hspace{-0.1ex} +
\smalldisplaystyleonehalf a_{44} \varepsilon^2_{xy} \hspace{-0.2ex} +
\smalldisplaystyleonehalf a_{55} \varepsilon^2_{xz} \hspace{-0.2ex} +
\smalldisplaystyleonehalf a_{66} \varepsilon^2_{yz}
\\[-0.1em]
%
+ a_{12} \varepsilon_{x} \varepsilon_{y} \hspace{-0.2ex}
+ a_{13} \varepsilon_{x} \varepsilon_{z} \hspace{-0.2ex}
+ a_{23} \varepsilon_{y} \varepsilon_{z}
\hspace{.1ex}
.
\end{multline*}

В~ортотропном материале
сдвиговые~(угловые) деформации
$\varepsilon_{xy}$, $\varepsilon_{xz}$, $\varepsilon_{yz}$
никак не~влияют
на нормальные напряжения
${
   \sigma_x \hspace{-0.25ex}
   = \raisemath{.16em} {
      \scalebox{.88}{$ \partial \hspace{.1ex} \potentialenergydensity $}
   }
   \hspace{-0.1ex} / \hspace{-0.2ex}
   \raisemath{-0.32em} {
      \scalebox{.88}{$ \partial \varepsilon_x $}
   }
}$,
${
   \sigma_y \hspace{-0.25ex}
   = \raisemath{.16em} {
      \scalebox{.88}{$ \partial \hspace{.1ex} \potentialenergydensity $}
}
\hspace{-0.1ex} / \hspace{-0.2ex}
\raisemath{-0.32em} {
   \scalebox{.88}{$ \partial \varepsilon_y $}
}
}$,
${\sigma_z \hspace{-0.25ex}
   =
   \raisemath{.16em} {
      \scalebox{.88}{$ \partial \hspace{.1ex} \potentialenergydensity $}
   }
   \hspace{-0.1ex} / \hspace{-0.2ex} \raisemath{-0.32em} {
      \scalebox{.88}{$ \partial \varepsilon_z $}
  }
}$ (\en{and vice versa}\ru{и~наоборот}).and vice versa
Популярный
ортотропный материал\:---
древесина.
Её упругие свойства
различны
по~трём
взаимно перпендикулярным
направлениям:
по~радиусу,
вдоль~окружности
и~по~высоте ствола.

\subsection*{\en{A~transversely isotropic}\ru{Трансверсально изотропный} \en{material}\ru{материал}}

\en{The one more}\ru{Ещё один}
\en{case of~anisotropy}\ru{случай анизотропии}
\en{is}\ru{это}
\en{a~transversely isotropic}\ru{трансверсально изотропный} \en{material}\ru{материал}}.
\en{It is characterized by an~axis of~anisotropy}\ru{Он характеризуется осью анизотропии}\:--- \en{let it be}\ru{пусть это}~$z$.
\en{Any plane}\ru{Любая плоскость}\ru{,}
\en{parallel to}\ru{параллельная}~$z$,
\en{must be a~plane}\ru{должна быть плоскостью}
\en{of the~material symmetry}\ru{материальной симметрии}.
\ru{Ясно}\en{It is clear}\ru{,}
\en{that this material is orthotropic}\ru{что этот материал ортотропен}.
\en{But there is more}\ru{Но есть большее}:
\en{any rotation}\ru{любой поворот}
\en{of the deformation tensor}\ru{тензора деформаций}~$\infinitesimaldeformation$
\en{around the axis}\ru{вокруг оси}~$z$
\ru{не меняет}\en{doesn’t change}
\en{the elastic potential}\ru{упругий потенциал}~$\potentialenergydensity$.
\en{Thus}\ru{Поэтому}

\nopagebreak\vspace{-0.1em}
\begin{equation}
\label{transversely-isotropic-material:hooke's law}
\scalebox{.9}{$
   \displaystyle\frac{
      \raisemath{-0.2em}{ \partial \hspace{.2ex} \potentialenergydensity }
      { \partial \infinitesimaldeformation }
   } %end of frac
} %end of scalebox
\hspace{-.2ex} \dotdotp \hspace{-.2ex} %two dots (double dot)
\bigl( \hspace{.2ex}
   \bm{k}
   \hspace{-.2ex} \times \hspace{-.2ex}
   \infinitesimaldeformation
   \hspace{-.1ex} - \hspace{-.1ex}
   \infinitesimaldeformation
   \hspace{-.2ex} \times \hspace{-.2ex}
   \bm{k}
\hspace{.2ex} \bigr) \hspace{-.2ex}
= 0
\hspace{.1ex}
,
\end{equation}

\noindent
\en{because}\ru{поскольку}
\en{for any}\ru{для любого}
\en{small rotation}\ru{малого поворота}
\en{with vector}\ru{с~вектором}~${ \varvector{\bm{o}} }$
\en{the variation}\ru{вариация}
%%\en{of the infinitesimal}\ru{бесконечно малой}
%%\en{linear}\ru{линейной}
%%\en{deformation}\ru{деформации}
\en{of the infinitesimal linear deformation tensor}\ru{тензора бесконечномалой линейной деформации}
$\infinitesimaldeformation$
\en{is}\ru{есть}
${
   \varvector{\bm{o}} \times \infinitesimaldeformation
   - \infinitesimaldeformation \times \varvector{\bm{o}}
$}\ru{,}
\en{and}\ru{а}
$\varvector{\bm{o}}$
\en{is directed}\ru{направлен}
\en{along}\ru{вдоль}~$z$.
\en{Let’s write}\ru{Распишем}
\en{this equation}\ru{это равенство}
\en{in components}\ru{в~компонентах}
(\en{it must be true}\ru{оно должно быть истиной}
\en{for}\ru{для}
\en{any}\ru{любых}
\en{infinitesimal deformations}\ru{бесконечномалых деформаций}~$\infinitesimaldeformation$)

\noindentic
\begin{gather}\label{equationforelasticmoduliforatransverselyisotropicmaterial}
\bigl( a_{11} \mathepsilon_{x} + a_{12} \mathepsilon_{y} + a_{13} \mathepsilon_{z} \bigr)\bigl( -2 \mathepsilon_{xy} \bigr)
+
\bigl( a_{12} \mathepsilon_{x} + a_{22} \mathepsilon_{y} + a_{23} \mathepsilon{z} \bigr) 2 \mathepsilon_{xy}
\\
%
+ 2 a_{44} \mathepsilon_{xy} \bigl( -\mathepsilon_{y} + \mathepsilon_{x} \bigr)
+ 2 a_{55} \mathepsilon_{xz} \bigl( -\mathepsilon_{yz} \bigr)
+ 2 a_{66} \mathepsilon_{yz} \mathepsilon_{xz}
= 0
\end{gather}

\noindent
\begin{equation}\label{elasticmoduliforatransverselyisotropicmaterial}
\Rightarrow\hspace{.5em}
a_{11}
\end{equation}
.......................

\en{of the deformation tensor}\ru{тензора деформаций}
\en{is equal to}\ru{равна}
${
   \varvector{\bm{o}}
   \hspace{-.2ex} \times \hspace{-.2ex}
   \infinitesimaldeformation
}$
\hspace{-.1ex} - \hspace{-.1ex}
${
   \infinitesimaldeformation
   \hspace{-.2ex} \times \hspace{-.2ex}
   \varvector{\bm{o}}
}$,
\en{and}\ru{а}~{\varvector{\bm{o}}}
\en{directed}\ru{направлен}
\en{along }\ru{вдоль}
\en{the $z$ axis}\ru{оси $z$}.
\en{The equation}\ru{Равенство}~\eqref{transversely-isotropic-material:hooke's law}
\en{is true}\ru{истинно}
\en{for}\ru{для}
\en{any}\ru{любых}
\en{infinitesimaldeformation}\ru{бесконечно малых}
\en{deformations}\ru{деформаций}~$\infinitesimaldeformation$.
\en{In~components}\ru{В~компонентах}~\eqref{transversely-isotropic-material:hooke's law}
\en{will be}\ru{будет}
\en{like this}\ru{таким}:

\vspace{-0.25em}\noindent
\begin{gather*}
\bigl( \hspace{.2ex}
a_{11} \mathepsilon_{x} \hspace{-.2ex}
+ a_{12} \mathepsilon_{y} \hspace{-.2ex}
+ a_{13} \mathepsilon_{z} \hspace{.2ex} \bigr) \hspace{.2ex}
\bigl( \hspace{.2ex}
- 2 \mathepsilon_{xy}
\hspace{.2ex} \bigr)
+ \bigl( \hspace{.2ex}
a_{22} \mathepsilon_{y} \hspace{-.2ex}
+ a_{12} \mathepsilon_{x} \hspace{-.2ex}
+ a_{23} \mathepsilon_{z}
\hspace{.2ex} \bigr)
2 \mathepsilon_{xy} \hspace{-.2ex}
\\
%
+ 2 a_{44} \mathepsilon_{xy} \hspace{-.2ex}
\bigl( \hspace{.2ex} - \mathepsilon_y + \mathepsilon_x \hspace{.2ex} \bigr)
+ 2 a_{55} \mathepsilon_{xz}
\bigl( \hspace{.2ex} - \mathepsilon_{yz} \hspace{.2ex} \bigr)
+ 2 a_{66} \mathepsilon_{yz} \mathepsilon_{xz} \hspace{-.2ex}
= 0
.
\end{gather*}

\noindent
\en{As a~result}\ru{Как результат}

\vspace{-0.25em}\noindent
\begin{equation*}
a_{11} \hspace{-.2ex} = a_{12} + a_{44} \hspace{-.2ex} = a_{22}
\hspace{.1ex} ,
\hspace{.5em}
a_{13} = a_{23}
\hspace{.1ex} ,
\hspace{.5em}
a_{55} \hspace{-.2ex} = a_{66}
\hspace{.1ex} .
\end{equation*}

\en{Presenting}\ru{Представив}
\en{the stress tensor}\ru{тензор напряжений}
\en{as}\ru{как}

\begin{equation*}\label{Hooke's law for transversely isotropic material}
\linearstress
= \withtheindexofperpendicularity{\linearstress}
+ \bm{t} \bm{k}
+ \bm{k} \bm{t}
+ \cauchystresscomponents_{zz} \bm{k} \bm{k}
\hspace{.1ex} ,
\hspace{1em}
\withtheindexofperpendicularity{\linearstress}
\def
\cauchystresscomponents_{\alpha \beta}
\bm{e}_{\alpha} \bm{e}_{\beta}
\hspace{.1ex} ,
\hspace{1em}
\linearstress \def \cauchystresscomponents_{\alpha z} \bm{e}_{\alpha}
\end{equation*}
($
   \bm{e}_{1} = \bm{i}
   \hspace{.1ex} ,
   \bm{e}_{2} = \bm{j}
$),
\en{we can}\ru{можем}
\en{write}\ru{записать}
\ru{закон }\en{the }Hooke’\en{s}\ru{а}\en{ law}
\en{for}\ru{для}
\en{a~transversally isotropic}\ru{трансверсально изотропного}
\en{material}\ru{материала} 

\begin{equation*}
\withtheindexofperpendicularity{\linearstress}
=
a_4 \withtheindexofperpendicularity{\infinitesimaldeformation}
+
\Bigl(
   a_{12}
   \withtheindexofperpendicularity{\infinitesimaldeformation}
   +
   a_{13} \epsilon_z
\Bigr)
\withtheindexofperpendicularity{\UnitDyad}
\end{equation*}

\en{So}\ru{Итак},
\en{a~transversally isotropic material}\ru{трансверсально изотропный материал}
\en{is characterized}\ru{характеризуется}
\en{by the five}\ru{пятью}
\en{non-null}\ru{ненулевыми}
\en{mutually independent}\ru{взаимно независимыми}
\en{components}\ru{компонентами},
\en{the elastic moduli}\ru{упругими модулями}
$a_{33}$, $a_{44}$, $a_{55}$, $a_{12}$, $a_{13}$
\hspace{.1ex} .

%симметрия кристаллов

\en{There are}\ru{Есть}
\en{many different types}\ru{различные типы}
\en{of crystal symmetry}\ru{симметрии кристаллов}
( \en{triclinic}\ru{триклинная},
\en{monoclinic}\ru{моноклинная},
\en{rhombic}\ru{ромбическая},
\en{tetragonal}\ru{тетрагональная}
\textcolor{magenta}{\en{and others}\ru{и~другие}~\cite{chernyh-anisotropicelasticity}} ).
\en{The~each}\ru{Каждый}
\en{case of~symmetry}\ru{случай симметрии}
\en{is characterized}\ru{характеризуется}
\en{by the~set of orthogonal tensors}\ru{набором ортогональных тензоров}~$\orthogonaltensor$%
\footnote{%
\en{Orthogonal tensors}\ru{Ортогональные тензоры}
\en{are those that}\ru{это такие, которые}
\en{satisfy}\ru{удовлетворяют}
\en{the equality}\ru{равенству}~\eqrefwithchapterdotsection{orthogonalityofrotationtensor}{chapter:mathapparatus}{section:rotationtensor}
${ \orthogonaltensor \dotp \orthogonaltensor^\T = \UnitDyad }$,
\en{describing}\ru{описывая}
\en{rotations}\ru{повороты}
\en{and}\ru{и}~\en{reflections}\ru{отражения}.
}\hbox{\hspace{-0.5em}.}
...................
\en{and}\ru{и}
\en{satisfy}\ru{удовлетворяют}
\en{the following}\ru{следующему}

\noindent
\begin{equation}\label{Hooke.for-symmetric-crystals}
   \stiffnesstensor \dotdotp
   \Bigl( \orthogonaltensor \dotp \lineardeformation \dotp \orthogonaltensor^\T \Bigr)
   = \orthogonaltensor \dotp
   \Bigl( \stiffnesstensor \dotdotp \lineardeformation \Bigr)
   \dotp
   \orthogonaltensor^\T
   \hspace{1.5em} \forall \hspace{.3ex} \lineardeformation .
\end{equation}

\en{The theory}\ru{Теория},
\en{describing}\ru{описывающая}
\en{the linear properties}\ru{линейные свойства}
\en{of anisotropic materials}\ru{анизотропных материалов}\:---
\en{the~crystals}\ru{кристаллов},
\en{the~composites}\ru{композитов},
\en{the~wood}\ru{древесины}
\en{and others}\ru{и~других},
................

\en{The }\inquotes{\en{material tensors}\ru{Материальные тензоры}}
\en{define}\ru{определяют}
\en{the physical properties}\ru{физические свойства}
\en{of bodies}\ru{тел}
\en{and media}\ru{и~сред},
\en{kind of}\ru{как то}
\begin{itemize}
\item \en{the elasticity}\ru{упругость},
\item \en{the thermal expansion}\ru{тепловое расширение},
\item \en{the thermal conductivity}\ru{теплопроводность},
\item \en{the electrical conductivity}\ru{электропроводность},
\item \en{piezoelectric effect}\ru{пьезоэлектрический эффект}.
\end{itemize}

\en{The traction vector}\ru{Вектор тракции}~$\bm{t}$
\en{and its projections}\ru{и~его проекции},
$\withtheindexofperpendicularity{\bm{t}}$ \en{and}\ru{и}~$\withtheindexofparallelism{\bm{t}}$.

\begin{itemize}
%\item \en{the traction vector}\ru{вектор тяги~(тракции)}~$\bm{t}$,
\item \en{the projection}\ru{проекция}
\en{of the~traction vector}\ru{вектора тяги}%~$\bm{t}$
\en{on the unit normal vector}\ru{на вектор единичной нормали}%~$\bm{n}$
\begin{equation}\label{theprojection.ofthetractionvector.onthenormal}
\withtheindexofperpendicularity{\bm{t}} = \bm{t}_{\bm{n}} = \bm{t} \dotp \bm{n}
\end{equation}
(\en{a~perpendicular}\ru{перпендикуляр}
\en{to the cross-section area}\ru{к~площадке поперечного сечения}),
\item \en{the projection}\ru{проекция}
\en{of the~traction vector}\ru{вектора тяги}%~$\bm{t}$
\en{on the plane}\ru{на плоскость}
\begin{equation}\label{theprojection.ofthetractionvector.ontheplane}
\withtheindexofparallelism{\bm{t}} = \bm{t} - \withtheindexofperpendicularity{\bm{t}}
\end{equation}
\end{itemize}

