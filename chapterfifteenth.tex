\en{\chapter{Fractures}}

\ru{\chapter{Трещины}}

\thispagestyle{empty}

\label{chapter:fractures}

\begin{otherlanguage}{russian}

\section{Традиционные критерии прочности}

\lettrine[lines=2, findent=2pt, nindent=0pt]{К}{ак} судить о~прочности тела после определения напряжения в~нём? При одноосном растяжении напряжением~$\sigma$ есть, очевидно, некий предел прочности~${\sigma_{\!*}}$. Прочность считают достаточной при~${\sigma \leq \sigma_{\!*} / k}$, где~$k$~--- так называемый коэффициент запаса. Но такой подход не~является вполне удовлетворительным, поскольку определяемые из~опыта

...

\section{Антиплоская деформация среды с трещиной}

Любая регулярная функция компл\'{е}ксного переменного~${z = x + iy}$ содержит в~себе решение какой\hbox{-}либо антиплоской задачи статики без ...

...



\section{Трещина при плоской деформации}

Рассмотрим плоскую область произвольного очертания с трещиной внутри; нагрузка приложена и~\inquotesx{в~объёме}[,] и на~внешней границе. Как и~при антиплоской деформации, решение строится в~два этапа

...



\section{Трещинодвижущая сила}

Это едва~ли не~основное понятие механики трещин. Рассмотрим его, следуя

...



\section{Критерий роста трещины}

Связанная с~энергией~$\textit{Э}$ трещинодвижущая сила~$F$~--- не~единственное воздействие на~фронт трещины. Должна быть ещё некая сила сопротивления~$F_{*}$; рост трещины начинается при~условии

...



\section{Интеграл Райса}

Одно из самых известных

...



\section{Определение коэффициентов интенсивности}

Расчёт прочности тела с~трещиной сводится к~определению коэффициентов интенсивности. Методы расчёта таких коэффициентов~--- как~аналитические, так~и~численные~--- хорошо освещены в~литературе.

Рассмотрм ещё один подход к~задачам механики трещин, разработанный

...



\section{Модель Баренблатта}

Неограниченный рост напряжений на фронте трещины вызывает некоторые сомнения. Желательно \inquotes{дать поддержку} сингулярным решениям какими\hbox{-}либо дополнительными построениями или использованием иной модели. И~эту поддержку дала работа

...



\section{Деформационный критерий}

...



\section{Рост трещин}

...



\section{Упругое поле впереди движущейся трещины}

Рассмотрим этот вопрос

...



\section{Баланс энергии для движущейся трещины}

...



\vspace{8mm}
\hfill\begin{minipage}[b]{0.95\linewidth}
\fontsize{10}{12}\selectfont

\section*{\wordforbibliography}

Список книг по~механике трещин уж\'{е} вел\'{и}к. В~нём нельзя не~отметить ...

\end{minipage}

\end{otherlanguage}
